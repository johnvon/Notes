\lecture{21}{18 Oct.\ 9:00}{The General Argument for Rate of Convergence}
\subsection{A General Approach}
We're now going to show the general argument for bounding the \hyperref[def:rate-of-convergence]{rate of convergence}. In this section, we will assume that our \hyperref[prb:M-estimation]{\(M\)-estimation problem} is defined for minimum rather than maximum.\footnote{This can be done without loss of generality since we can simply add a negative sign for \(M\) and \(M_n\).}

\begin{note}
	Hence, \(M(\theta ) \geq M(\theta _0)\) for all \(\theta \in \Theta \) now.
\end{note}

\begin{definition}[Rate of convergence]\label{def:rate-of-convergence}
	The \emph{rate of convergence} for \(\hat{\theta} _n\) is defined as the sequence \(\{ \delta _n \} \) such that \(d(\hat{\theta} _n, \theta _0) = O_p(\delta _n)\).
\end{definition}

Recall that instead of using the old \hyperref[def:growth-condition]{growth condition} used when showing \hyperref[def:consistent]{consistency}, we need an alternative form. Consider the following.

\begin{definition}[Growth condition*]\label{def:growth-condition*}
	The \emph{growth condition} on \(M(\theta )\) is that for all \(\theta \in \Theta \),
	\[
		d(\theta , \theta _0)^2 \leq M(\theta )\ - M(\theta _0).
	\]
\end{definition}

\begin{note}
	For such \hyperref[def:growth-condition*]{growth condition}, the canonical choice of \(d\) is \(d(\theta , \theta _0) = \sqrt{M(\theta _0) - M(\theta )}\).
\end{note}

It suffices to show that given \(\epsilon > 0\), there exists \(M\) (not depending on \(n\)) such that for all \(n\),
\[
	\mathbb{P} (d(\hat{\theta} _n, \theta _0) > 2^M \delta _n) \leq \epsilon.
\]

Firstly, we apply the \emph{peeling step} to get
\begin{equation}\label{eq:peeling-step}
	\mathbb{P} (d(\hat{\theta} _n, \theta _0) > 2^M \delta _n)
	= \sum_{j > M} \mathbb{P} (2^{j-1} \delta _n < d(\hat{\theta} _n, \theta _0) \leq 2^j \delta _n).
\end{equation}

\begin{note}
	From the \hyperref[def:growth-condition*]{growth condition} and the basic inequality,
	\[
		\begin{split}
			d(\hat{\theta} _n, \theta _0)^2
			 & \leq M(\hat{\theta} _n ) - M(\theta _0)                                                                                                   \\
			 & = M(\hat{\theta} _n) - M_n(\hat{\theta} _n) + \underbrace{M_n(\hat{\theta} _n) - M_n(\theta _0)}_{\leq 0} + M_n(\theta _0) - M(\theta _0) \\
			 & \leq (M_n - M)(\theta _0) - (M_n - M)(\hat{\theta} _n).
		\end{split}
	\]
\end{note}

With this, we can further upper-bound \autoref{eq:peeling-step} by
\begin{align*}
	\mathbb{P} (d(\hat{\theta} _n, \theta _0) > 2^M \delta _n)
	 & = \sum_{j > M} \mathbb{P} (2^{j-1} \delta _n < d(\hat{\theta} _n, \theta _0) \leq 2^j \delta _n)\tag*{\hyperref[eq:peeling-step]{peeling step}}                                        \\
	 & \leq \sum_{j > M} \mathbb{P} ((M_n - M)(\theta _0) - (M_n - M)(\hat{\theta} _n) \geq 2^{2j-2} \delta _n^2 \cap d(\hat{\theta} _n, \theta _0) \leq 2^j \delta _n)                       \\
	 & \leq \sum_{j > M} \mathbb{P} \left( \sup _{\theta \colon d(\theta , \theta _0) \leq 2^j \delta _n} (M_n - M)(\theta _0) - (M_n - M)(\theta ) \geq 2^{2j - 2} \delta _n^2 \right)       \\
	 & \leq \sum_{j > M} \frac{\mathbb{E}_{}\left[\sup _{\theta \colon d(\theta , \theta _0) \leq 2^j \delta _n} (M_n - M)(\theta _0) - (M_n - M)(\theta ) \right] }{2^{2j - 2} \delta _n^2 }
\end{align*}
by \hyperref[lma:Markov-inequality]{Markov's  inequality}, where we need to assume that it's non-negative. Now, we define the following.

\begin{definition}[Localized empirical process]\label{def:localized-EP}
	The \emph{localized empirical process} for an \hyperref[prb:M-estimation]{\(M\)-estimaton problem} for \(t > 0\) is defined as
	\[
		\mathbb{E}_{}\left[\sup _{\theta \colon d(\theta , \theta _0) \leq t} (M_n - M)(\theta _0) - (M_n - M)(\theta ) \right].
	\]
\end{definition}

Note the following.

\begin{note}
	For nearly all \hyperref[prb:M-estimation]{\(M\)-estimaton problems}, the \hyperref[def:localized-EP]{localized empirical process} can be upper-bounded by a sequence of functions \(\phi _n\colon [0, \infty ] \to [0, \infty ]\) such that for all \(t > 0\),
	\[
		\mathbb{E}_{}\left[\sup _{\theta \colon d(\theta , \theta _0) \leq t} (M_n - M)(\theta _0) - (M_n - M)(\theta ) \right] \leq \phi _n(t).
	\]
\end{note}

Assuming \(\phi _n\)'s exist, we then proceed upper-bounding \autoref{eq:peeling-step} as
\[
	\mathbb{P} (d(\hat{\theta} _n, \theta _0) > 2^M \delta _n)
	\leq \sum_{j > M} \frac{\mathbb{E}_{}\left[\sup _{\theta \colon d(\theta , \theta _0) \leq 2^j \delta _n} (M_n - M)(\theta _0) - (M_n - M)(\theta ) \right] }{2^{2j - 2} \delta _n^2 }
	\leq \sum_{j > M} \frac{\phi _n(2^j \delta _n)}{2^{2j - 2} \delta _n^2}.
\]
To further bound the right-hand side, consider the following.

\begin{definition}[Sub-quadratic assumption]\label{def:sub-quadratic-assumption}
	The \emph{sub-quadratic assumption} assumes that there exists \(\alpha < 2\) such that for all \(n\), \(c > 1\), and \(x > 0\),
	\[
		\phi _n(cx) \leq c^\alpha \cdot \phi _n(x).
	\]
\end{definition}

With \hyperref[def:sub-quadratic-assumption]{sub-quadratic assumption}, we get
\[
	\mathbb{P} (d(\hat{\theta} _n, \theta _0) > 2^M \delta _n)
	\leq \sum_{j > M} \frac{\phi _n(2^j \delta _n)}{2^{2j - 2} \delta _n^2}
	\leq 4 \sum_{j > M} \frac{2^{\alpha j} \phi _n(\delta _n)}{2^{2j} \delta _n^2}.
\]

This is the final bound we will get by introducing \(\phi _n\). Now, to control them, consider the so-called \hyperref[def:rate-determining-equation]{rate-determining equation}.

\begin{definition}[Rate-determinig equation]\label{def:rate-determining-equation}
	Given a sequence \(\{ \delta _n \} \), the \emph{rate-determining equation} for a \hyperref[def:localized-EP]{localized empirical process}'s upper-bounds \(\phi _n\)'s is that for some \(c\) such that for all \(n\),
	\[
		\phi _n(\delta _n) \leq c \delta _n^2.
	\]
\end{definition}

\begin{remark}
	It's important to check that whether such \(c\) exists for all \(n\).
\end{remark}

\begin{intuition}
	We want to have \(\phi _n(\delta _n) \approx \delta _n^2\).
\end{intuition}

Finally, assuming the \hyperref[def:rate-determining-equation]{rate-determining equation} exists for some \(c\), we get
\[
	\mathbb{P} (d(\hat{\theta} _n, \theta _0) > 2^M \delta _n)
	\leq 4 \sum_{j > M} \frac{2^{\alpha j} \phi _n(\theta _n)}{2^{2j} \delta _n^2}
	\leq 4c\cdot \sum_{j > M} \frac{2^{\alpha j}}{2^{2j}} ,
\]
and from the \hyperref[def:sub-quadratic-assumption]{sub-quadratic assumption}, \(\alpha < 2\), so the above sum converges to \(0\) as \(M \to \infty \).

\begin{remark}
	We can choose \(M\) not depending on \(n\) such that the right-hand side is \(\leq \epsilon \).
\end{remark}

Putting the above together, we have the following.
\begin{theorem}[Non-asymptotic rate of convergence]\label{thm:non-asymptotic-rate-of-convergence}
	For an \hyperref[prb:M-estimation]{\(M\)-estimation problem}, assume the \hyperref[def:growth-condition*]{growth condition} on \(M\), and the \hyperref[def:sub-quadratic-assumption]{sub-quadratic assumption} (with parameter \(\alpha < 2\)) and the \hyperref[def:rate-determining-equation]{rate-determining equation} are valid for \(\phi _n\)'s arose from bounding the \hyperref[def:localized-EP]{localized empirical process},
	\[
		\mathbb{P} (d(\hat{\theta} _n, \theta _0) > 2^M \delta _n) \leq 4c \sum_{j > M} 2^{(\alpha -2) j}.
	\]
\end{theorem}

Back to the \hyperref[eg:mode-estimation]{mode estimation} example, we now want to formally show that the \hyperref[def:rate-of-convergence]{rate of convergence} for \(\vert \hat{\theta} _n - \theta _0 \vert \) is \(O_p(n^{-1 / 3})\).

\begin{proposition}\label{prop:mode-estimation-rate-of-convergence}
	For the \hyperref[eg:mode-estimation]{mode estimation problem}, the \hyperref[def:rate-of-convergence]{rate of convergence} for \(\hat{\theta} _n\) is \(O_p(n^{-1 / 3})\).
\end{proposition}
\begin{proof}
	We check the following.
	\begin{itemize}
		\item The \hyperref[def:growth-condition*]{growth condition} is checked in a ball around \(\theta _0\) and not for all \(\theta \in \Theta \) (since it's \hyperref[def:consistent]{consistent}). This is allowed as shown in \autoref{thm:non-asymptotic-rate-of-convergence-extend}.
		\item Consider the \hyperref[def:localized-EP]{localized empirical process} with notation \(m_\theta (x) = \mathbbm{1}_{[\theta -1, \theta +1]} \), we have
		      \[
			      \mathbb{E}_{}\left[\sup _{\theta \colon \vert \theta - \theta _0 \vert \leq t} \vert \mathbb{P} _n(m_\theta - m_{\theta _0}) - \mathbb{P} (m_\theta - m_{\theta _0}) \vert \right].
		      \]
		      Let \(f_\theta (x) = m_\theta (x) - m_{\theta _0}(x)\), and \(\mathscr{F} = \{ f_\theta \colon \vert \theta -\theta _0 \vert \leq t \} \). One can check that
		      \[
			      f(x) = \mathbbm{1}_{\theta _0 - 1 - t \leq x \leq \theta _0 - 1 + t} + \mathbbm{1}_{\theta _0 + 1 - t \leq x \leq \theta _0 + 1 + t}
		      \]
		      is an \hyperref[def:envelope]{envelope} for \(\mathscr{F} \). Then, we have
		      \[
			      \mathbb{P} F^2
			      \leq \int_{\theta _0 - 1 - t}^{\theta _0 - 1 + t} p_{\theta _0}(x) \,\mathrm{d}x + \int_{\theta _0 + 1 - t}^{\theta _0 + 1 + t} p_{\theta _0}(x)  \,\mathrm{d}x
			      \leq p_{\theta _0}(\theta _0) \cdot 4t
			      \leq C_{p_{\theta _0}} t
			      < \infty
		      \]
		      for some constant \(C_{p_{\theta _0}}\) depending on \(p_{\theta _0}\). With the \hyperref[thm:bracketing-bound]{bracketing bound}, for some constant \(C > 0\),
		      \[
			      \mathbb{E}_{}\left[\sup _{f\in \mathscr{F} } \sqrt{n} \vert \mathbb{P} _n f - \mathbb{P} f \vert \right]
			      \leq C\cdot \lVert F \rVert _{L_2(\mathbb{P} )} \int_{0}^{1} \sqrt{\log N_{[\ ]}(\mathscr{F} , L_2(\mathbb{P} ), \epsilon \lVert F \rVert _{L_2(\mathbb{P} )})} \,\mathrm{d}\epsilon .
		      \]
		      \begin{claim}
			      For all \(\epsilon\), there exists some constant \(C^{\prime} > 0\) such that
			      \[
				      N_{[\ ]}(\mathscr{F} , L_2(\mathbb{P} ), \epsilon )
				      \leq \left( \frac{1}{\epsilon } \right) ^{C^{\prime} } < \infty .
			      \]
		      \end{claim}
		      With the above claim and \(\lVert F \rVert _{L_2(\mathbb{P} )} \leq \sqrt{C_{p_{\theta _0}} t} \), the integral can be further bounded as
		      \[
			      \int_{0}^{1} \sqrt{\log N_{[\ ]}(\mathscr{F} , L_2(\mathbb{P} ), \epsilon \lVert F \rVert _{L_2(\mathbb{P} )})} \,\mathrm{d}\epsilon
			      \leq \int_{0}^{1} \sqrt{C^{\prime} \log \frac{1}{\epsilon \lVert F \rVert _{L_2(\mathbb{P} )}} } \,\mathrm{d}\epsilon
			      < \infty,
		      \]
		      hence, there exists some constant \(C> 0\) such that
		      \[
			      \mathbb{E}_{}\left[\sup _{f\in \mathscr{F} } \vert \mathbb{P} _n f - \mathbb{P} f \vert  \right]
			      \leq C \sqrt{\frac{t}{n}}.
		      \]
		      This motivates us to define \(\phi _n (t)\) as \(C \sqrt{t / n} \).
		\item To check the \hyperref[def:sub-quadratic-assumption]{sub-quadratic assumption}, for all \(n\) and \(c > 1\), we have
		      \[
			      \phi _n (ct)
			      = C \sqrt{\frac{ct}{n}}
			      = \sqrt{c} \cdot \left( C \sqrt{\frac{t}{n}} \right)
			      = \sqrt{c} \cdot \phi _n
		      \]
		      hence the \hyperref[def:sub-quadratic-assumption]{sub-quadratic assumption} is satisfied with \(\alpha = 1 / 2\).
		\item Consider the \hyperref[def:rate-determining-equation]{rate-determining equation} \(\phi _n(t) \leq t^2\), for \(t = \delta _n\),
		      \[
			      \sqrt{t / n} \leq t^2
			      \iff \sqrt{\delta _n / n} \leq \delta _n^2
			      \implies 1 / \sqrt{n} \leq \delta _n^{3 / 2}
			      \implies \delta _n \approx 1 / n^{1 / 3}.
		      \]
	\end{itemize}
	In all, we have \(\vert \hat{\theta} _n - \theta _0 \vert = O_p(n^{-1 / 3})\).
\end{proof}

\autoref{prop:mode-estimation-rate-of-convergence} is not fully proven yet, since we only have the \hyperref[def:growth-condition*]{growth condition} satisfied in a ball around \(\theta _0\), not for all \(\theta \in \Theta \).