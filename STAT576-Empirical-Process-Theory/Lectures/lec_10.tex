\lecture{10}{15 Sep.\ 9:00}{Discretization of a Space}
\section{Metric Entropy Methods}
We have been focusing on boolean function class with finite \hyperref[def:VC-dimension]{VC dimension}, and our goal now is to generalize beyond the boolean case. This can be done by discretizing of a space.

\begin{intuition}[Informal principle]\label{int:informal-principle}
	We want to bound \(\mathbb{E}_{}\left[\sup _{t\in T } X_t \right]\). If \(\{ X_t \} _{t\in T}\) is ``sufficiently continuous'', then \(\mathbb{E}_{}\left[\sup _{t\in T} X_t\right] \) is governed by metric properties of \(T\) (\hyperref[def:metric-entropy]{metric entropy}!).
\end{intuition}

\begin{definition}[Pseudo-metric]\label{def:pseudo-metric}
	Given a space \(T\), a function \(d\colon T \times T \to \mathbb{R} ^+\) is a \emph{pseudo-metric} if
	\begin{enumerate}[(a)]
		\item \(d(x, x) = 0\) for all \(x \in T\);\footnote{If \(d\) further satisfies that \(d(x, y) > 0\) for all \(x \neq y\), then it becomes a \emph{metric}.}
		\item \(d(x, y) = d(y, x)\) for all \(x, y\in T\);
		\item \(d(x, y) \leq d(x, z) + d(y, z)\) for all \(x, y, z\in T\).
	\end{enumerate}
\end{definition}

\begin{note}
	The motivation of looking at \hyperref[def:pseudo-metric]{pseudo-metric} instead of the usual metric is because, consider observed data \(x_1, \dots , x_n\) at hands, the most natural distance might be
	\[
		(f, g) \mapsto \sqrt{\frac{1}{n} \sum_{i=1}^{n} \big(f(x_i) - g(x_i)\big)^2},
	\]
	which is a \hyperref[def:pseudo-metric]{pseudo-metric} since \(f\) and \(g\) can agree only on \(x_i\)'s and vary elsewhere.
\end{note}

\subsection{Covering Number and Packing Number}
Now, let \((T, d)\) denote a \hyperref[def:pseudo-metric]{pseudo-metric} space in the remaining of this section, unless specified.

\begin{definition}[\(\epsilon \)-net]\label{def:eps-net}
	A set \(N\) is an \emph{\(\epsilon \)-net} of \((T, d)\) if for all \(t\in T\), there exists \(\pi (t) \in N\) such that \(d(t, \pi (t)) \leq \epsilon \).
\end{definition}

\begin{definition}[Covering number]\label{def:covering-number}
	The \emph{\(\epsilon \)-covering number} \(N(T, d, \epsilon )\) of \((T, d)\) is defined as
	\[
		N(T, d, \epsilon ) \coloneqq \inf \{ \vert N \vert \colon N \text{ is an \hyperref[def:eps-net]{\(\epsilon\)-net} for \((T, d)\)} \}.
	\]
\end{definition}

\begin{remark}
	\(N\) is not necessary a subset of \(T\) for convenience. Furthermore, if \(N \nsubseteq T\), one can construct another \hyperref[def:eps-net]{net} \(N^{\prime} \) such that \(N^{\prime} \subseteq T\) and \(N^{\prime} \) is a \hyperref[def:eps-net]{\(2 \epsilon\)-net}.
\end{remark}

\begin{definition}[Totally bounded]\label{def:totally-bounded}
	\((T, d)\) is \emph{totally bounded} if for all \(\epsilon > 0\), \(N(T, d, \epsilon ) < \infty \).
\end{definition}

\begin{definition}[\(\epsilon \)-packing]\label{def:eps-packing}
	A set \(N \subseteq T\) is an \emph{\(\epsilon \)-packing} of \((T, d)\) if for all \(t \neq t^{\prime} \) in \(N\), \(d(t, t^{\prime} ) > \epsilon \).
\end{definition}

\begin{definition}[Packing number]\label{def:packing-number}
	The \emph{\(\epsilon \)-pacing number} \(M(T, d, \epsilon )\) of \((T, d)\) is defined as
	\[
		M(T, d, \epsilon ) = \sup \{ \vert N \vert \colon N \text{ is an \hyperref[def:eps-packing]{\(\epsilon\)-packing} of \((T, d)\)}\}.
	\]
\end{definition}

As the title suggests, we define the following \hyperref[def:pseudo-metric]{metric} properties, which is an essential notion helps us bound the expected \hyperref[def:EP]{empirical process} supremum.

\begin{definition}[Metric entropy]\label{def:metric-entropy}
	The \emph{metric entropy} of \((T, d)\) is defined as \(\log M(T, d, \epsilon )\).
\end{definition}

The fact that we're using \hyperref[def:packing-number]{packing number} \(M(T, d, \epsilon )\) when defining \hyperref[def:metric-entropy]{metric entropy} is not relevant here due to the following.

\begin{lemma}\label{lma:covering-packing}
	For any \(\epsilon > 0\),
	\[
		M(T, d, 2\epsilon ) \leq N(T, d, \epsilon ) \leq M(T, d, \epsilon ).
	\]
\end{lemma}
\begin{proof}
	We show them one by one.

	\begin{claim}
		\(M(T, d, 2\epsilon ) \leq N(T, d, \epsilon )\).
	\end{claim}
	\begin{explanation}
		Take \(\mathcal{M} \) to be a \hyperref[def:eps-packing]{\(2\epsilon\)-packing} and \(\mathcal{N} \) to be an \hyperref[def:eps-net]{\(\epsilon\)-net}. Then for any \(t\in \mathcal{N} \), consider \(B(t, \epsilon )\). We see that there is at most one \(x\in \mathcal{M} \) such that \(d(t, x) \leq \epsilon \) since otherwise, if \(x, x^{\prime} \in \mathcal{M} \) such that \(x \neq x^{\prime} \) and \(d(t, x), d(t, x^{\prime} ) \leq \epsilon \), then \(d(x, x^{\prime} ) \leq 2\epsilon \), a contradiction to \(\mathcal{M} \).
	\end{explanation}

	\begin{claim}
		\(N(T, d, \epsilon ) \leq M(T, d, \epsilon )\).
	\end{claim}
	\begin{explanation}
		Take \(\mathcal{M} \) to be a maximum \hyperref[def:eps-packing]{\(\epsilon\)-packing}, it suffices to show that \(\mathcal{M} \) is also an \hyperref[def:eps-net]{\(\epsilon\)-net}, i.e., for all \(t\in T\), there exists \(x\in \mathcal{M} \) such that \(d(x, t) \leq \epsilon \). Suppose not, then \(d(t, x) > \epsilon \) for all \(x\in \mathcal{M} \), i.e., we can add \(x\) to \(\mathcal{M} \), contradiction.
	\end{explanation}
\end{proof}

For simplicity, we will use the following notations.

\begin{notation}
	If \((T, d)\) and \(\epsilon \) are clear from the context, we write \(N \coloneqq N(T, d, \epsilon )\) and \(M \coloneqq M(T, d, \epsilon )\).
\end{notation}

Turns out that there's a characterization of the \hyperref[def:packing-number]{packing number} of the unit ball in euclidean space.

\begin{proposition}\label{prop:packing-covering}
	Consider \((\mathbb{R} ^d, \lVert \cdot \rVert )\) where \(\lVert \cdot \rVert \) is any norm. Denote \(B = \{ x \colon \lVert x \rVert \leq 1 \} \), then for all \(\epsilon > 0\),
	\[
		(1 / \epsilon) ^d \leq M(B, \lVert \cdot \rVert , \epsilon ) \leq (1 + 2 / \epsilon )^d.
	\]
\end{proposition}
\begin{proof}
	For the lower-bound, we see that
	\[
		N \mathop{\mathrm{Vol}}(\epsilon B) \geq \mathop{\mathrm{Vol}}(B)
		\implies N \epsilon ^d \geq 1.
	\]
	With \(N \leq M\) from \autoref{lma:covering-packing}, we get the lower-bound.

	For the upper-bound, since \(\epsilon / 2\) balls around points in \(M\) are disjoint, union of these \(\epsilon / 2\) balls will lie in \((1 + \epsilon / 2)B\). This implies
	\[
		M \times \left( \frac{\epsilon}{2} \right) ^d \times \mathop{\mathrm{Vol}}(B)
		\leq \left( 1 + \frac{\epsilon}{2} \right) ^d \times \mathop{\mathrm{Vol}}(B)
		\implies M \leq \left( 1 + \frac{2}{\epsilon } \right) ^d.
	\]
\end{proof}

\begin{note}
	From \autoref{prop:packing-covering}, \(\log M(\mathbb{R} ^d, \lVert \cdot \rVert , \epsilon ) \approx d \log 1 / \epsilon \).
\end{note}

\subsection{Hölder Smooth Functions}
We are interested in looking at function spaces, and the following are the canonical smooth function classes studied in \emph{nonparametric regression}.

\begin{definition}[Hölder smooth function class]\label{def:Holder-smooth-function-class}
	Let \(\alpha > 0\) and \(\beta = \lfloor \alpha \rfloor\). Then the \emph{Hölder smooth function class} \(\mathcal{S} _\alpha \) is defined to be the class of functions on \([0, 1]\) such that
	\begin{enumerate}[(a)]
		\item \(f\) continuous on \([0, 1]\);
		\item \(f\) is \(\beta \)-times differentiable;
		\item \(\vert f^{(k)} \vert \leq 1\) for all \(k = 0, \dots , \beta \);
		\item \(\vert f^{(\beta )}(x) - f^{(\beta )}(y) \vert \leq \vert x - y \vert ^{\alpha - \beta }\) for all \(x, y \in [0, 1]\).
	\end{enumerate}
\end{definition}

\begin{note}
	When \(\alpha = 1\), \(\mathcal{S} _\alpha \) is a class of \(1\)-Lipschitz functions.
\end{note}

\begin{remark}
	The \hyperref[def:Holder-smooth-function-class]{Hölder smooth function classes} are nested, so it's not surprising that the \hyperref[def:metric-entropy]{metric entropies} decrease as \(\alpha \) increases.
\end{remark}

Now, let \(d(f, g) = \sup _{x\in [0, 1]} \vert f(x) - g(x) \vert \), then \((\mathcal{S} _\alpha , d)\) is a \hyperref[def:pseudo-metric]{pseudo-metric} space.

\begin{theorem}\label{thm:metric-entropy}
	There exists \(c_1, c_2\) such that for all \(\epsilon > 0\),
	\[
		\exp \left( c_2 \epsilon ^{-1 / \alpha } \right)
		\leq M(\mathcal{S} _\alpha , d, \epsilon )
		\leq \exp \left( c_1 \epsilon ^{-1 / \alpha } \right) .
	\]
\end{theorem}
\begin{proof}[Proof sketch]\todo{Add proof}
	Here we illustrate the basic idea when \(\alpha = 1\), i.e., the set of \([0,1]\) valued \(1\)-Lipschitz functions on \([0,1]\). We only sketch the proof of the upper-bound, since the lower-bound is similar.

	Firstly, we partition both the domain and the range of \(f\) with small intervals with width \(\epsilon\), resulting in \(1/\epsilon\) small intervals on both the \(x\)-axis and the \(y\)-axis.

	Take any function \(f \in \mathscr{F} \). We construct a piece-wise constant function \(\widetilde{f}\) which approximates \(f\). On each small interval in the \(x\)-axis, we can define \(\widetilde{f}\) to be constant, taking value equal to the midpoint of the interval in the \(y\)-axis where the value of \(f\) at the left endpoint of this interval (in the \(x\)-axis) lies. Then, we have the following.
	\begin{claim}
		\(\sup_{x \in [0,1]} |f(x) - \widetilde{f}(x)| \leq C \epsilon\).
	\end{claim}
	\begin{explanation}
		Since \(f\) cannot vary by more than \(\epsilon\) in any interval of length \(\epsilon\).
	\end{explanation}

	Now, as we vary \(f \in \mathscr{F} \), consider the following.

	\begin{problem*}
		What is the number of distinct \(\widetilde{f}\) we can get?
	\end{problem*}
	A trivial bound is that, in each small interval on the \(x\)-axis, it takes one of the midpoints of the intervals on the \(y\)-axis and hence, the number of such functions is bounded by \((\frac{1}{\epsilon})^{\frac{1}{\epsilon}}\).

	We can do slightly better. Note that, for the first interval, the number of possible values of \(\widetilde{f}\) is \(\frac{1}{\epsilon}\). However, after that, in the next interval, the value of \(\widetilde{f}\) can only go up one interval, down one interval, or stay the same (due to \(1\)-Lipschitzness of \(f\)), i.e., there are only \(3\) choices afterward for every interval, going from left to right, resulting an upper bound on the number of distinct \(\widetilde{f}\) as
	\[
		\frac{1}{\epsilon} 3^{\frac{1}{\epsilon} - 1} \leq \exp(\frac{C}{\epsilon}).
	\]
	For a complete (and long) proof, see \autoref{pf-thm:metric-entropy}.
\end{proof}

\begin{remark}
	Comparing \autoref{prop:packing-covering} and \autoref{thm:metric-entropy}, we see that the \hyperref[def:metric-entropy]{metric entropy} is logarithmic in \(1 / \epsilon \) versus some exponent of \(1 / \epsilon \). This is typically the hallmark of a parametric versus a nonparametric function class.
\end{remark}
