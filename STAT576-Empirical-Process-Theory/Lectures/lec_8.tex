\lecture{8}{11 Sep.\ 9:00}{Symmetrization on 1-D Threshold Classification}
Analogous to the \hyperref[def:Rademacher-complexity]{Rademacher complexity} defined for a function class w.r.t.\ \(\mathbb{P} \), we can define it on a set.

\begin{definition}[Rademacher width]\label{def:Rademacher-width}
	Given \(A \subseteq \mathbb{R} ^n\), the \emph{Rademacher width}\footnote{Also called \emph{Rademacher average}.} of \(A\) is defined as
	\[
		R_n(A) = \mathbb{E}_{\epsilon _i}\left[ \sup _{a\in A} \frac{1}{n} \sum_{i=1}^{n} \epsilon _i a_i \right].
	\]
\end{definition}

\begin{notation}
	People sometimes just say ``Rademacher complexity'' for \hyperref[def:Rademacher-width]{Rademacher width}.
\end{notation}

Now, applying the \hyperref[lma:symmetrization]{symmetrization lemma} to \(\mathscr{F} = \{ \mathbbm{1}_{X \leq \theta} \colon \theta \in \mathbb{R} \} \), we have the following result that is comparable to \autoref{prop:bracketing}.

\begin{proposition}\label{prop:symmetrization}
	Let \(x_1, \dots , x_n \overset{\text{i.i.d.} }{\sim } \mathbb{P} \), and \(\mathscr{F} = \{ \mathbbm{1}_{x \leq \theta} \colon \theta \in \mathbb{R} \} \). Then
	\[
		\mathbb{E}_{X}\left[ \sup _{f\in \mathscr{F} } \left( \mathbb{P} (X \leq \theta ) - \frac{1}{n}\sum_{i=1}^{n} \mathbbm{1}_{x_i \leq \theta } \right) \right] = O\left( \sqrt{\frac{\log n}{n}}  \right).
	\]
\end{proposition}
\begin{proof}
	From the \hyperref[lma:symmetrization]{symmetrization lemma},
	\begin{align*}
		\mathbb{E}_{X, x_i}\left[\sup _{\theta \in \mathbb{R} } \left( \mathbb{P} (X \leq \theta ) - \frac{1}{n} \sum_{i=1}^{n} \mathbbm{1}_{x_i \leq \theta } \right) \right]
		 & \leq 2 \mathbb{E}_{\epsilon _i, x_i}\left[\sup _{\theta \in \mathbb{R} } \frac{1}{n}\sum_{i=1}^{n} \epsilon _i \mathbbm{1}_{x_i \leq \theta } \right]\tag*{condition on \(x_1, \dots , x_n\)}                   \\
		 & = 2 \mathbb{E}_{x_i}\left[ \mathbb{E}_{\epsilon _i \mid x_i}\left[ \sup _{f\in \mathscr{F} } \frac{1}{n} \sum_{i=1}^{n} \epsilon _i \mathbbm{1}_{x_i \leq \theta } \middle| x_1, \dots , x_n \right]  \right] .
	\end{align*}
	Let
	\[
		V_\theta \coloneqq \frac{1}{n}\sum_{i} \epsilon _i \mathbbm{1}_{x_i \leq \theta } ,
	\]
	we see that there are only \(n+1\) distinct \(V_\theta \)'s, and it's constant in the intervals \(\theta \in [X_{(k)}, X_{(k+1)})\) for \(k = 0, \dots , n-1\) where \(X_{(k)}\) are the order statistics with \(X_{(0)} \coloneqq -\infty \). Now, define \(\theta _k \coloneqq X_{(k)}\), we can then write
	\[
		\sup _{\theta \in \mathbb{R} }\frac{1}{n} \sum_{i=1}^{n} \epsilon _i \mathbbm{1}_{x_i \leq \theta }
		= \max _{k=0, \dots , n} \frac{1}{n}\sum_{i=1}^{n} \epsilon _i \mathbbm{1}_{x_i \leq \theta _k},
	\]
	hence,
	\begin{align*}
		2 \mathbb{E}_{x_i}\left[ \mathbb{E}_{\epsilon _i \mid x_i}\left[ \sup _{f\in \mathscr{F} } \frac{1}{n} \sum_{i=1}^{n} \epsilon _i \mathbbm{1}_{x_i \leq \theta } \middle| x_1, \dots , x_n \right]  \right]
		 & = 2 \mathbb{E}_{x_i}\left[ \mathbb{E}_{\epsilon _i \mid x_i}\left[ \max _{k=0, \dots , n} V_{\theta _k} \middle| x_1, \dots , x_n \right]  \right] \\
		\shortintertext{with \(V_{\theta _k} \sim \Subg(1 / n) \) and \autoref{lma:lec7},}
		 & \leq 2 \mathbb{E}_{x_i}\left[ \sqrt{\frac{2}{n} \log (n+1)} \right]                                                                                \\
		 & = O\left( \sqrt{\frac{\log n	}{n}} \right).
	\end{align*}
\end{proof}

\begin{remark}
	Looking back to the \hyperref[eg:1D-classification-thresholds]{example of 1-D thresholds classification}, we see that the \hyperref[not:excess-risk]{excess risk} of \hyperref[prb:ERM]{ERM} is \(O(\sqrt{\log n / n} )\).
\end{remark}

\section{Glivenko-Cantelli Class and Vapnik-Chervonenkis Dimension}
\subsection{Glivenko-Cantelli Class}
\begin{definition}[Glivenko-Cantelli]\label{def:Glivenko-Cantelli}
	A function class \(\mathscr{F} = \{ f\colon \chi \to \mathbb{R} \} \) is called \emph{Glivenko-Cantelli} w.r.t.\ \(\mathbb{P} \) if
	\[
		\sup _{f\in \mathscr{F} } \left\vert \mathbb{P} f - \mathbb{P} _n f \right\vert \to 0
	\]
	as \(n \to \infty \).
\end{definition}

From \hyperref[prop:bracketing]{bracketing} and \hyperref[prop:symmetrization]{symmetrization}, we know that \(\mathscr{F} = \{ \mathbbm{1}_{X \leq \theta } \colon \theta \in \mathbb{R} \} \) is \hyperref[def:Glivenko-Cantelli]{Glivenko-Cantelli}. Let's see some counterexamples.

\begin{eg}
	Let \(\chi = \mathbb{R} \), \(\mathscr{F} = \{ \mathbbm{1}_{A} \colon A \subseteq \chi, \vert A \vert < \infty \} \), and \(\mathbb{P} \) be any continuous measure on \(\chi \). Then \(\mathscr{F} \) is not \hyperref[def:Glivenko-Cantelli]{Glivenko-Cantelli} w.r.t.\ \(\mathbb{P} \).
\end{eg}
\begin{explanation}
	For \(f = \mathbbm{1}_{A} \), \(\mathbb{P} f = \mathbb{P} (X\in A) = 0\) since \(\vert A \vert < \infty \). On the other hand, let \(A_0 = \{ X_1, \dots , X_n \} \) be the observed empirical data, \(\mathbb{P} _n f = 1\), i.e., \(\sup _{f\in \mathscr{F} } \vert \mathbb{P} f - \mathbb{P} _n f \vert = 1\) for all \(n\in \mathbb{N} \).
\end{explanation}

\begin{eg}
	Let \(\chi = \mathbb{R} \), \(\mathscr{F} = \{ f\colon \chi \to \mathbb{R} \text{ bounded and continuous}\} \), and \(\mathbb{P} = \mathcal{U} [0, 1]\). Then \(\mathscr{F} \) is not \hyperref[def:Glivenko-Cantelli]{Glivenko-Cantelli}.
\end{eg}
\begin{explanation}
	Consider
	\begin{itemize}
		\item \(f(X_i) = 1\) for \(i = 1, \dots , n\) and \(f = 0\) elsewhere (in a continuous manner).\footnote{E.g., sharp peak at \(X_i\)'s.}
		\item Then we can make \(\int_{0}^{1} f(t) \,\mathrm{d}t < \delta \) for some \(\delta \in (0, 1)\).
	\end{itemize}
	This implies \(\sup _{f\in \mathscr{F} } \vert \mathbb{P} f - \mathbb{P} _n f \vert \geq 1 - \delta  \) for all \(n \in \mathbb{N} \).
\end{explanation}

Now, let's introduce some notation.

\begin{notation}
	Let \(\mathscr{F} (x_1, \dots , x_n) \coloneqq \{ \big(f(x_1), \dots , f(x_n)\big) \} _{f\in \mathscr{F} } \subseteq \mathbb{R} ^n\).
\end{notation}

Then we have
\[
	\mathbb{E}_{X_i}\left[R_n( \mathscr{F} (X_1, \dots , X_n)) \right]
	= \sup _{f\in \mathscr{F} }\frac{1}{n}\sum_{i=1}^{n} \epsilon _i f(X_i),
\]
i.e., we get back the \hyperref[def:Rademacher-complexity]{Rademacher complexity}. Moreover, we see that if \(\mathscr{F} \) be the set of indicator functions as before, then \(\mathscr{F} (X_1, \dots , X_n)\) is finite, we then have
\[
	\mathbb{E}_{X_i}\left[R_n(\mathscr{F} (X_1, \dots , X_n)) \right]
	\leq 2 \sqrt{\frac{2 \log \vert \mathscr{F} (X_1, \dots , X_n) \vert }{n}} .
\]

\begin{remark}
	The same rate holds for all \(\vert \mathscr{F} (X_1, \dots , X_n) \vert \leq n^d\).
\end{remark}

\subsection{Vapnik-Chervonenkis Dimension}


\begin{definition}[Boolean function class]\label{def:boolean-function-class}
	A function class \(\mathscr{F} \) is a \emph{boolean function class} on \(\chi \) if it has a polynomial discrimination for all \(x_1, \dots , x_n\in \chi \), i.e.,
	\[
		\left\vert \mathscr{F} (x_1, \dots , x_n) \right\vert \leq \poly(n).
	\]
\end{definition}

\begin{definition}[Shatter]\label{def:shatter}
	A finite set \(\{ x_1, \dots , x_D \} \subseteq \chi \) is \emph{shattered} by a \hyperref[def:boolean-function-class]{boolean function class} \(\mathscr{F} \) if \(\mathscr{F} (x_1, \dots , x_d) = \{ 0, 1 \} ^D\).
\end{definition}

\begin{definition}[VC dimension]\label{def:VC-dimension}
	The \emph{VC dimension} of a \hyperref[def:boolean-function-class]{boolean function class} \(\mathscr{F} \) on \(\chi \) is the maximum integer \(D\) such that there exists a size \(D\) finite set \(A \subseteq \chi \) \hyperref[def:shatter]{shattered} by \(\mathscr{F} \).
\end{definition}

\begin{remark}
	We take the convention that \(\varnothing \) is always \hyperref[def:shatter]{shattered}.
\end{remark}

Consider \(\chi = \mathbb{R} \).

\begin{eg}
	The \hyperref[def:VC-dimension]{VC dimension} of \(\mathscr{F} = \{ \mathbbm{1}_{X \leq \theta } \colon \theta \in \mathbb{R} \} \) is \(1\).
\end{eg}

\begin{eg}
	The \hyperref[def:VC-dimension]{VC dimension} of \(\mathscr{F} = \{ \mathbbm{1}_{[a, b]} \colon a, b \in \mathbb{R}  \} \) is \(2\).
\end{eg}

Let's look at one example with \(\chi = \mathbb{R} ^2\).
\begin{eg}
	The \hyperref[def:VC-dimension]{VC dimension} of \(\mathscr{F} = \{ \mathbbm{1}_{[a, b]\times [c, d]} \colon a, b , c, d\in \mathbb{R} \} \) is \(4\).
\end{eg}