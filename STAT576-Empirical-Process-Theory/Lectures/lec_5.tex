\lecture{5}{1 Sep.\ 9:00}{Proof of McDiarmid's Inequality}
We should note that the usual proof of \hyperref[thm:McDiarmid-inequality]{McDiarmid inequality} involves \hyperref[def:martingale-decomposition]{martingale decomposition} and \hyperref[thm:Azuma-Hoeffding-inequality]{Azuma-Hoeffding inequality}, a generalization of \hyperref[thm:Hoeffding-inequality]{Hoffding's inequality} for \hyperref[def:martingale-difference-sequence]{martingale difference sequence}. However, we will not go with this route; instead, we prove something weaker but tricker.\footnote{In fact, what we're going to prove is not even a weaker version: we prove something weaker while we really need the original (stronger) statement to hold.}

\begin{note}
	The condition \(\sup _{x_1, \dots , x_n, x_i^{\prime} }\vert f(x_1, \dots , x_n) - f(x_1, \dots , x_i^{\prime} , \dots , x_n) \vert \leq c_i\) is equivalent to
	\[
		\vert f(x_1, \dots , x_n) - f(z_1, \dots , z_n) \vert \leq \sum_{i=1}^{n} c_i \mathbbm{1}_{x_i \neq z_i}.
	\]
\end{note}

Now, we need one last lemma to prove \hyperref[thm:McDiarmid-inequality]{McDiarmid inequality}.

\begin{lemma}\label{lma:lec5}
	For all \(x \neq y \in \mathbb{R} \),
	\[
		\frac{e^x - e^y}{x-y} \leq \frac{e^x + e^y}{2} \implies \vert e^x - e^y \vert \leq \vert x - y \vert \left( \frac{e^x + e^y}{2} \right).
	\]
\end{lemma}
\begin{proof}
	Since
	\[
		\frac{e^x - e^y}{x-y}
		= \int_{0}^{1} e^{s x + (1 - s)y} \,\mathrm{d}s
		= \frac{1}{x - y} \int_{x}^{y} e^t \,\mathrm{d}t
	\]
	where we let \(t = sx + (1 - s)y\). On the other hand, due to convexity, we also have
	\[
		\frac{e^x - e^y}{x-y}
		= \int_{0}^{1} e^{s x + (1 - s)y} \,\mathrm{d}s
		\leq \int_{0}^{1} s\cdot e^x + (1 - s)e^y \,\mathrm{d}s
		= \frac{e^x + e^y}{2}.
	\]
\end{proof}

We're now ready.

\begin{proof}[Proof of \autoref{thm:McDiarmid-inequality}]
	Firstly, we note that it's equivalent to show that \(f(X_1, \dots , X_n) - \mathbb{E}_{}\left[f \right] \in \Subg(\sum_{i} c_i^2 / 4) \). Without loss of generality, let \(\mathbb{E}_{}\left[f \right] = 0\), and we want to show that
	\[
		\mathbb{E}_{}\left[e^{\lambda (f(X) - \mathbb{E}_{}\left[f \right] )} \right] \leq e^{\frac{\lambda ^2 \sum_{i} c_i}{8}}
		\iff  M(\lambda ) = \mathbb{E}_{}\left[e^{\lambda f(X)} \right] \leq \exp \left( \frac{\lambda ^2 \left( \sum_{i} c_i^2 \right) }{8} \right)
		\iff  \log M(\lambda ) \leq \lambda ^2 \frac{\sum_{i} c_i^2}{8}.
	\]
	Observe that since both sides of the inequality is \(0\) at \(\lambda = 0\), it's enough to show
	\[
		\frac{\mathrm{d}\log M(\lambda )}{\mathrm{d}\lambda } = \frac{M^{\prime} (\lambda )}{M(\lambda )} \leq \lambda \cdot \frac{\sum_{i} c_i^2}{4}
	\]
	Let \(X = (X_1, \dots , X_n)\), and \(X^{\prime} \overset{\text{i.i.d.} }{\sim } X\) be the i.i.d.\ copy of \(X\). Then define the following.

	\begin{notation}\label{not:lec5}
		\(X^{(i)} \coloneqq (X_1^{\prime} , \dots , X_i^{\prime} , X_{i+1}, \dots , X_n)\) and \(X^{[i]} \coloneqq (X_1, \dots , X_{i-1}, X_i^{\prime} , X_{i+1}, \dots , X_n)\).
	\end{notation}

	Note that this implies \(X^{(0)} = X\) and \(X^{(n)} = X^{\prime} \). Then, we can show that
	\begin{align*}
		M^{\prime} (\lambda )
		 & = \mathbb{E}_{}\left[f(X) e^{\lambda f(X)} \right] \tag*{As \(\mathbb{E}_{}\left[f \right] = 0\) and \(X, X^{\prime} \) are independent}                                                                                                                                                     \\
		 & = \mathbb{E}_{}\left[\big(f(X) - f(X^{\prime} )\big) e^{\lambda f(X)}\right]                                                                                                                                                                                                                          \\
		 & = \mathbb{E}_{}\left[ \sum_{i=1}^{n} (f(X^{(i-1)}) - f(X^{(i)})) \cdot e^{\lambda f(X)}\right]                                                                                                                                                                                                        \\
		\intertext{if \(i^{\text{th} }\) position of \(X\) and \(X^{\prime} \) are swapped, then for the new data \(X^{(i-1)}\) and \(X^{(i)}\) will also be swapped,}
		 & = \mathbb{E}_{}\left[\frac{1}{2} \sum_{i=1}^{n} \left( f(X^{(i-1)}) - f(X^{(i)}) \right) \cdot \left( e^{\lambda f(X)} - e^{\lambda f(X^{[i]})} \right)  \right]                                                                                                                             \\
		 & \leq \mathbb{E}_{}\left[\frac{\lambda }{2} \sum_{i=1}^{n} \left\vert f(X^{(i-1)}) - f(X^{(i)}) \right\vert \cdot \left\vert f(X) - f(X^{[i]}) \right\vert \cdot \left( \frac{e^{\lambda f(X)} + e^{\lambda f(X^{[i]})}}{2} \right)  \right] \tag*{from \autoref{lma:lec5}} \\
		 & \leq \frac{\lambda}{2}\left( \sum_{i=1}^{n} c_i^2 \right) \cdot M(\lambda ).
	\end{align*}
\end{proof}

We note the following.

\begin{note}
	The above proof doesn't even show a weaker version of \hyperref[thm:McDiarmid-inequality]{McDiarmid's inequality}.
\end{note}
\begin{explanation}
	While in the proof, we need to show
	\[
		\frac{\mathrm{d}\log M(\lambda )}{\mathrm{d}\lambda } = \frac{M^{\prime} (\lambda )}{M(\lambda )} \leq \lambda \cdot \frac{\sum_{i} c_i^2}{4},
	\]
	we only show
	\[
		\frac{\mathrm{d}\log M(\lambda )}{\mathrm{d}\lambda } = \frac{M^{\prime} (\lambda )}{M(\lambda )} \leq \lambda \cdot \frac{\sum_{i} c_i^2}{2}.
	\]
	For a complete proof, see \autoref{pf-thm:McDiarmid-inequality}.
\end{explanation}

\subsection{Applications of McDiarmid's Inequality}
\subsubsection{\(U\)-Statistics}

Let \(g\colon \mathbb{R} ^2 \to \mathbb{R} \) be a symmetric function, and let \(X_1, \dots , X_n \overset{\text{i.i.d.} }{\sim } \mathbb{P} \). Consider
\[
	U(X) = \frac{1}{\binom{n}{2}} \sum_{j < k} g(X_j, X_k).
\]

Here're some examples of \(g\).
\begin{eg}
	\(g(x, y) = (x - y)^2\).
\end{eg}

\begin{eg}
	\(g(x, y) = \vert x - y \vert \).
\end{eg}

\begin{eg}[Wilcoxm's ranksom test]
	\(g(x, y) = \mathbbm{1}_{x_1 + x_2 > 0} \).
\end{eg}

We're interested to know about \(\mathbb{E}_{}\left[g(X_1, X_2) \right] \). Assume \(g\) is bounded by \(B\), then
\[
	U(X) - U(X^{[k]}) \leq \frac{1}{\binom{n}{2}} (n-1) 2B \leq \frac{4B}{n},
\]
implying
\[
	\mathbb{P} (U - \mathbb{E}_{}\left[U \right] \geq t) \leq e^{-\frac{nt^2}{8b^2}}
\]
from \hyperref[thm:McDiarmid-inequality]{McDiarmid's inequality} with \(c_i \coloneqq 2B\).

\subsubsection{Beyond McDiarmid's Inequality}
Let's see some more advanced inequalities. In many cases, we want variance to be small. While
\[
	\Var_{}\left[X_1 + \dots + X_n \right] \leq \sum_{i=1}^{n} \Var_{}\left[X_i \right],
\]
to have an inequality for a non-linear function, we have the following.

\begin{theorem}[Efron-Stein inequality]\label{thm:Efron-Stein-inequality}
	Let \(X_1, \dots , X_n\) be independent random variables, and \(X_1^{\prime} , \dots , X_n^{\prime} \) be i.i.d.\ copies of \(X_i\)'s. Then
	\[
		\Var_{}\left[f(X) \right] \leq \frac{1}{2} \sum_{i=1}^{n} \mathbb{E}_{}\left[\big(f(X) - f(X^{[i]}) \big)^2 \right].
	\]
\end{theorem}
\begin{proof}
	For a proof, see \autoref{pf-thm:Efron-Stein-inequality}.
\end{proof}

\begin{note}
	We see that since \(\Var_{}\left[X \right] = \frac{1}{2}\mathbb{E}_{}\left[(X-X^{\prime} )^2 \right]\), by letting \(f(X_1, \dots , X_n) = \sum_{i} X_i\), if \(f\) satisfies bounded condition, then \(\Var_{}\left[f \right] \leq \frac{1}{2}\sum_{i} c_i^2\).
\end{note}

Now, recall that by using \hyperref[thm:McDiarmid-inequality]{McDiarmid's inequality}, we can show that for \(\mathscr{F} \ni f\) being \(B\)-bounded,
\[
	S_n \leq \mathbb{E}_{}\left[S_n \right] + B \sqrt{\frac{2}{n}\log \frac{1}{\delta }}
\]
with probability at least \(1 - \delta \). However, what if the variance \(\Var_{}\left[f(X) \right] \) is small, but the maximum spread (\(B\)) is very large? In this case, we would want to replace \(B\) in the inequality by \(\Var_{}\left[f(X) \right] \).

\begin{notation}[Empirical process notation]
	Let \(\mathbb{P} f = \mathbb{E}_{}\left[f \right] \) and \(\mathbb{P} _n f = \sum_{i} f(X_i) / n\).
\end{notation}

This is achieved by the following, although it's much harder to prove~\cite[\defaultS 12]{boucheron2013concentration}.

\begin{theorem}[Talagrand's concentration inequality]\label{thm:Talagrand-concentration-inequality}
	Let \(\mathscr{F} \) is \(B\)-bounded, and \(S_n = \sup _{f\in \mathscr{F}} \vert \mathbb{P} _n f - \mathbb{P} f \vert \). Then
	\[
		S_n \leq c\cdot \mathbb{E}_{}\left[S_n \right] + c \sqrt{\frac{\sup_{f\in \mathscr{F}} \Var_{}\left[f(X_1) \right] }{n} \log \frac{1}{\alpha }} + c\cdot \frac{B}{n} \log \frac{1}{\alpha }
	\]
	with probability at least \(1 - \alpha \).
\end{theorem}

\begin{remark}
	We might encounter an explicit situation where \hyperref[thm:Talagrand-concentration-inequality]{Talagrand's concentration} is more profitable to use than \hyperref[thm:McDiarmid-inequality]{bounded differences inequality} later in the course.
\end{remark}