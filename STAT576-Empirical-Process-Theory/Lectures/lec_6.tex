\chapter{Statistical Learning and Empirical Risk Minimization}
\lecture{6}{6 Sep.\ 9:00}{A Glance at Statistical Learning Theory}
Let's first see another motivation on studying uniform law of large numbers, i.e., the \emph{goodness of fit testing}. Given \(X_1, \dots , X_n \overset{\text{i.i.d.}}{\sim } \mathbb{P} \), we want to distinguish between
\begin{itemize}
	\item \(H_0 \colon P = P_0\); and
	\item \(H_1\colon P \neq P_0\).
\end{itemize}
Many tests are possible. One approach could be the \href{https://en.wikipedia.org/wiki/Kolmogorov-Smirnov_test}{Kolmogorov-Smirnov test}: assume \(F\) is the CDF of \(\mathbb{P} _0\), then consider the \hyperref[def:Kolmogorov-Smirnov-statistics]{Kolmogorov-Smirnov statistics}:

\begin{definition}[Kolmogorov-Smirnov statistics]\label{def:Kolmogorov-Smirnov-statistics}
	The \emph{Kolmogorov-Smirnov statistics} for a distribution \(\mathbb{P} \) is defined as
	\[
		D_n = \sup _{t\in \mathbb{R} } \vert F_n(t) - F(t) \vert
	\]
	where \(F_n(t)\) and \(F\) is the \hyperref[def:empirical-CDF]{empirical CDF} and the CDF of \(\mathbb{P} \), respectively.
\end{definition}

From \href{https://en.wikipedia.org/wiki/Glivenko-Cantelli_theorem}{Glivenko-Cantelli theorem}, \(D_n \to 0\) under \(H_0\), and \(D_n\) should not converge to \(0\), under some alternative.

Assuming continuity of \(F\), Kolmogorov showed that
\begin{enumerate}[(a)]
	\item the distribution \(D_n\) does not depend on \(F\);
	\item \(D_n = O_p(1 / \sqrt{n} )\);
	\item \(\sqrt{n} D_n \to \sup _{t\in [0, 1]} \vert B(t) \vert \) where \(B(t)\) is the \href{https://en.wikipedia.org/wiki/Brownian_bridge}{Broweian bridge} on \([0, 1]\).
	\item \(\mathbb{P} (\sqrt{n}D_n  \leq 2.4) \approx 0.999973\).
\end{enumerate}
We'll take a non-asymptotic approach to this problem. And the second main application that we have in mind for this course if the \(M\)-estimation.

\section{Statistical Learning and Empirical Risk Minimization}
Consider the following problem.

\begin{problem}[Empirical risk minimization]\label{prb:ERM}
Let \(S = \{ (x_1, y_1) , \dots , (x_n, y_n)\} \) be \(n\) i.i.d.\ copies of \((X, Y) \in \chi \times \mathcal{Y} \subseteq \mathbb{R} ^d\times \mathbb{R}\) with distribution \(\mathbb{P} = \mathbb{P} _X \times \mathbb{P} _{Y \mid X}\). Given a loss function \(\ell \colon \mathcal{Y} \times \mathcal{Y} \to \mathbb{R}\) and a class of functions \(\mathscr{F} = \{ f\colon \chi \to \mathcal{Y}  \} \), the \emph{empirical risk minimization} problem is defined as the problem of finding
\[
	\hat{f} \in \argmin_{f\in \mathscr{F} } \frac{1}{n}\sum_{i=1}^{n} \ell (f(x_i), y_i).
\]
\end{problem}

\begin{eg}
	Some examples of \(\mathscr{F} \) can be the set of neural networks, decision trees, linear functions.
\end{eg}

\begin{eg}[Linear regression]
	Consider \(\chi = \mathbb{R} ^d\) and \(\mathcal{Y} = \mathbb{R} \), with \(\mathscr{F} = \{ x \to w^{\top} x \colon w\in \mathbb{R} ^d \} \) and \(\ell (a, b) = (a - b)^2\).
\end{eg}

\begin{eg}[Linear classification]
	Consider \(\chi = \mathbb{R} ^d\) and \(\mathcal{Y} = \{ 0, 1 \} \), with
	\[
		\mathscr{F} = \{ x \to (\sgn (w^{\top} x ) + 1)/ 2 \colon w\in B_2^d \}
	\]
	where \(B_2^d\) is the unit ball in \(d\)-dimension, and \(\ell (a, b) = \mathbbm{1}_{a \neq b} \).
\end{eg}

We also define the following.

\begin{definition*}
	Consider the set-up of \hyperref[prb:ERM]{ERM}.
	\begin{definition}[Expected loss]\label{def:expected-loss}
		The \emph{expected loss}\footnote{Also called \emph{population loss} and \emph{test error}.} of \(f\in \mathscr{F} \) is defined as
		\[
			L(f) = \mathbb{E}_{(X, Y) \sim \mathbb{P} }\left[\ell (f(X), Y) \right].
		\]
	\end{definition}

	\begin{definition}[Empirical loss]\label{def:empirical-loss}
		The \emph{empirical loss} is defined as
		\[
			\hat{L} (f) = \frac{1}{n}\sum_{i=1}^{n} \ell (f(x_i), y_i).
		\]
	\end{definition}
\end{definition*}

The main question in statistical learning is that, what is an upper-bound on the \hyperref[def:expected-loss]{expected loss} of \hyperref[prb:ERM]{ERM}? If we plug in \(\hat{f} \) instead of \(f\), this is asking the \hyperref[def:expected-loss]{test error} of \(\hat{f} \).

To be specific, \(\hat{f} \) is basically a function of training data \(S\), but when we look at
\[
	L(\hat{f} ) = \mathbb{E}_{(X, Y)}\left[\ell (\hat{f} (x), Y) \right],
\]
it is the expectation of future data points, i.e., it becomes a random variable, which is a function of \(S\).

\begin{lemma}\label{lma:ERM}
	For any \(\mathscr{F} \), the \hyperref[prb:ERM]{ERM} \(\hat{f} \) satisfies
	\[
		\mathbb{E}_{}[L(\hat{f} ) ] - \inf _{f\in \mathscr{F} } L(f)
		\leq \mathbb{E}_{}\left[\sup _{f\in \mathscr{F} } \big(L(f) - \hat{L} (f) \big) \right] .
	\]
\end{lemma}

Before proving \autoref{lma:ERM}, let us first decode what is it claiming.
\begin{itemize}
	\item Since \(L(f)\) is the \hyperref[def:expected-loss]{population error} of \(f\) and \(\hat{L} (f)\) is the \hyperref[def:empirical-loss]{empirical loss} of \(f\),
	      \[
		      \sup _{f\in \mathscr{F} } \big( L(f) - \hat{L} (f) \big)
	      \]
	      is the supremum of an \hyperref[def:EP]{empirical process}.
	\item For the left-hand side, it represents the \hyperref[def:expected-loss]{expected loss} of \(\hat{f} \) and the best possible out-of-sample error.\footnote{Or the best possible prediction error of \(\mathscr{F} \).} This is often called the \hyperref[not:excess-risk]{excess risk}.
	      \begin{notation}[Excess risk]\label{not:excess-risk}
		      \(\mathbb{E}_{}[L(\hat{f} ) ] - \inf _{f\in \mathscr{F} } L(f)\) is often called the \emph{excess risk} of an \hyperref[prb:ERM]{ERM}.
	      \end{notation}
\end{itemize}

\begin{proof}[Proof of \autoref{lma:ERM}]
	Let \(f^{\ast} = \inf _{f\in \mathscr{F} } L(f)\). Then
	\[
		L(\hat{f} ) - L(f^{\ast} )
		= [L(\hat{f} ) - \hat{L} (\hat{f} )] + [\hat{L} (\hat{f} ) - \hat{L} (f^{\ast} )] + [\hat{L} (f^{\ast} ) - L(f^{\ast} )].
	\]
	We see that
	\begin{itemize}
		\item \(\hat{L} (\hat{f} ) - \hat{L} (f^{\ast} ) \leq 0\) by \hyperref[prb:ERM]{definition};
		\item \(\hat{L} (f^{\ast} ) - L(f^{\ast} ) = 0\) in expectation since \(f^{\ast} \) is fixed,
		\item We can't say \(\mathbb{E}_{}[L(\hat{f} ) - \hat{L} (\hat{f} ) ] = 0\) since \(\hat{f} \) is also random.
	\end{itemize}
	Combine all these, we have
	\[
		\mathbb{E}_{}[L(\hat{f} ) ] - \inf _{f\in \mathscr{F} } L(f)
		= \mathbb{E}_{}[ L(\hat{f} ) - L(f^{\ast} ) ]
		\leq \mathbb{E}_{}[ L(\hat{f} ) - \hat{L} (\hat{f} ) ]
		\leq \mathbb{E}_{}\left[ \sup _{f\in \mathscr{F} } \big(L(f) - \hat{L} (f) \big) \right].
	\]
\end{proof}

\begin{remark}
	For ``curved'' loss function like square loss, supremum can be further ``localized''.
\end{remark}

\begin{remark}
	The bound in \autoref{lma:ERM} can be vacuumed for now, e.g., for linear regression.
\end{remark}

\begin{eg}[1-D classification with thresholds]
	Let \(\ell (a, b) = \mathbbm{1}_{a \neq b} = a + (1 - 2a)b\) for \(a, b\in \{ 0, 1 \} \). Then consider \(a = y\) and \(b = f(x)\),
	\[
		\mathbb{E}_{}\left[\sup _{f\in \mathscr{F} } \big( L(f) - \hat{L} (f) \big) \right]
		= \mathbb{E}_{}\left[\sup _{f\in \mathscr{F} } \left( \mathbb{E}_{}\left[Y + (1 - 2Y)f(X) \right] - \frac{1}{n} \sum_{i=1}^{n} \big( y_i + (1 - 2y_i)f(x_i) \big) \right)  \right],
	\]
	which can be viewed essentially as\footnote{For now, since \(Y - \sum_{i} y_i / n\) is independent of \(f\), so let's drop it; and \(1 - 2Y\) is the sign which can also be dropped essentially.} the \hyperref[def:EP]{empirical process} on the function \(f\) instead of \(\ell \),
	\[
		\mathbb{E}_{}\left[\sup _{f\in \mathscr{F} } \left( \mathbb{E}_{}\left[f(X) \right] - \frac{1}{n}\sum_{i=1}^{n} f(x_i) \right)  \right].
	\]
	For 1-D case, assume that
	\[
		\mathscr{F} = \{ x \to \mathbbm{1}_{x \leq \theta } \colon \theta \in \mathbb{R}  \},
	\]
	then
	\[
		\mathbb{E}_{}\left[\sup _{\theta \in \mathbb{R} } \left( \mathbb{P} (X \leq \theta ) - \frac{1}{n} \sum_{i=1}^{n} \mathbbm{1}_{x_i \leq \theta } \right) \right]
		= \mathbb{E}_{}\left[\sup _{\theta \in \mathbb{R} } (F(\theta ) - F_n(\theta )) \right] ,
	\]
	i.e., \(P(X \leq \theta )\) is the CDF of the marginal distribution of \(X\), \(F(\theta )\), and \(\frac{1}{n}\sum_{i=1}^{n} \mathbbm{1}_{x_i \leq \theta } \) is the \hyperref[def:empirical-CDF]{empirical CDF} \(F_n(\theta )\). Therefore, we go back to the same problem we introduced in the beginning of the chapter, i.e., the \hyperref[def:Kolmogorov-Smirnov-statistics]{Kolmogorov-Smirnov statistics}.

	Let the term \(\mathbb{P} (X \leq \theta ) - \frac{1}{n}\sum_{i=1}^{n} \mathbbm{1}_{x_i \leq \theta } \) to be a random variable \(U_\theta \). One problem here is, we have infinitely many random variables, and they are also correlated with each other quite a lot. So how does this supremum behave?

	Since each \(U_\theta \) is at most \(1\), for any \(\theta \), i.e., \(\sup U_\theta \leq 1\). So the worst case here is \(1\), and probably the best case is \(O(1 / \sqrt{n} )\).
\end{eg}