\lecture{20}{16 Oct.\ 9:00}{Rate of Convergence}
To show \(\hat{\theta} \overset{p}{\to} \theta _0\), we need ``curvature'' or ``growth'' condition on \(M\) at \(\theta _0\).

\begin{center}
	\incfig{mode-estimation-growth}
\end{center}

Consider the following.

\begin{definition}[Growth condition]\label{def:growth-condition}
	The \emph{growth condition} on \(M(\theta )\) is that for all \(\epsilon > 0\),
	\[
		\sup _{\theta \colon d(\theta, \theta _0) \vert > \epsilon } M(\theta ) < M(\theta _0).
	\]
\end{definition}

Equivalently, the \hyperref[def:growth-condition]{growth condition} implies that for all \(\epsilon > 0\), there exists \(\delta > 0\) such that \(\delta < \epsilon \) and
\[
	\inf _{\theta \colon d(\theta, \theta _0 ) > \epsilon } M(\theta _0) - M(\theta ) > \delta .
\]
Hence, as \(\epsilon \to 0\), \(\mathbb{P} (d(\hat{\theta} , \theta _0 ) > \epsilon ) \leq \mathbb{P} (M(\theta _0) - M(\hat{\theta} ) > \delta ) \to 0\). In our \hyperref[prb:mode-estimation]{mode estimation} problem, since
\[
	M(\theta )
	= \mathbb{P} (\theta -1 \leq X \leq \theta +1)
	= \int_{\theta -1}^{\theta + 1} p(x) \,\mathrm{d}x,
\]
if we assume that \(p\) is increasing until \(\theta _0\), and then decreasing after \(\theta _0\),\footnote{Since \(\theta _0\) is the unique maximum.} one can check that
\[
	\sup _{\theta \colon d(\theta, \theta _0) > \epsilon } M(\theta )
	= \max (M(\theta _0 + \epsilon ), M(\theta _0 - \epsilon ))
	< M(\theta _0)
\]
from \(p^{\prime\prime} < 0\) at \(\theta _0\). More generally, one can refer to the following~\cite[Theorem 5.7]{vaartAsymptoticStatistics1998}.

\begin{theorem}\label{thm:lec20}
	Let \(\hat{\theta} \) be any minimizer of \(M_n(\theta )\), and \(\theta _0\) be the unique minimizer of \(M(\theta )\). If
	\begin{enumerate}[(a)]
		\item\label{thm:lec20-a} \(\sup _{\theta \in \Theta } \vert M_n(\theta ) - M(\theta ) \vert \overset{p}{\to } 0\);
		\item for all \(\epsilon >0\), \(\inf _{\theta \colon d(\theta , \theta _0) > \epsilon } M(\theta ) > M(\theta _0)\) for some metric \(d\) on \(\Theta \),
	\end{enumerate}
	then we have \(d(\theta , \theta _0) \overset{p}{\to } 0\).
\end{theorem}

\begin{remark}
	Condition \autoref{thm:lec20-a} may be too strong. One may need a preliminary ``localization''.
\end{remark}

\section{Rate of Convergence}
Let's first see some heuristics before introducing the general theory. In the \hyperref[prb:mode-estimation]{mode estimation} problem,
\[
	M(\theta _0) - M(\hat{\theta} )
	\leq (\mathbb{P} _n - \mathbb{P} ) (m_{\hat{\theta} } - m_\theta )
	\leq 2 \sup _{\theta \in \Theta } \vert \mathbb{P} _n m_\theta - \mathbb{P} m_\theta  \vert,
\]
in our \hyperref[prb:mode-estimation]{mode estimation} problem, in expectation, we can further upper-bound it by \(c / \sqrt{n} \), hence
\[
	M(\theta _0) - M(\hat{\theta} ) = O_p(1 / \sqrt{n} ).
\]

\begin{note}
	We cannot in general, get a rate better than \(1 / \sqrt{n} \) for this suprema.
\end{note}

\begin{remark}
	If \(M\) is twice differentiable at \(\theta _0\) with \(M^{\prime\prime} (\theta _0) < 0\), then there exists a neighborhood \(I\) of \(\theta _0\) such that for all \(\theta \in I\),
	\[
		M(\theta _0) - M(\theta ) \geq c \cdot d(\theta, \theta _0)^2.
	\]
\end{remark}
\begin{explanation}
	From Taylor expansion, there exists \(\chi\) between \(\theta \) and \(\theta _0\) such that
	\[
		M(\theta ) - M(\theta _0)
		= d(\theta, \theta _0) M^{\prime} (\theta _0) + \frac{d(\theta, \theta _0)^2}{2} M^{\prime\prime} (\xi ),
	\]
	with \(M^{\prime} (\theta _0) = 0\), \(M(\theta ) - M(\theta _0) > c \cdot d(\theta, \theta _0)^2\) for some \(c\).
\end{explanation}

We know that \(\hat{\theta} \) is consistent, we can assume \(\hat{\theta} \in I\).

\begin{remark}
	\hyperref[def:growth-condition]{Growth condition} is satisfied not for all \(\theta \in \mathbb{R} \), but for \(\theta \) in a fixed neighborhood \(I\) of \(\theta _0\).
\end{remark}

\begin{intuition}
	Since \(\hat{\theta} \) is consistent, the \hyperref[def:growth-condition]{growth condition} on a ball around \(\theta _0\) should suffice.
\end{intuition}

In this case, we have
\[
	d(\hat{\theta}, \theta _0)^2 \lesssim M(\theta ) - M(\theta _0) = O_p(1 / \sqrt{n} ),
\]
hence \(d(\hat{\theta}, \theta _0) = O_p(n^{-1 / 4})\). In our example,
\[
	M^{\prime\prime} (\theta _0)
	= p^{\prime} (\theta _0 + 1) - p^{\prime} (\theta _0 - 1)
	< 0
\]
since \(p^{\prime} (\theta _0+1) < 0\) and \(p^{\prime} (\theta _0 - 1) > 0\).

We cannot get better than \(O_p(n^{-1 / 4})\) with this argument. But there exist problems where the rate is better! In our example, the right rate is \(O_p(n^{-1 / 3})\), for \hyperref[def:parametric]{``parametric''} problems like mean estimation and quantile estimation, the right rate is \(O_p(1 / \sqrt{n} )\). We need to do ``localization''! Specifically, we now have
\[
	d(\hat{\theta}, \theta _0)^2
	\lesssim M(\theta _0) - M(\hat{\theta} )
	\leq (\mathbb{P} _n - \mathbb{P} ) (m_{\hat{\theta} } - m_{\theta _0})
	\leq 2 \sup _{\theta \in \Theta } \vert \mathbb{P} _n m_\theta - \mathbb{P} m_\theta \vert
	\leq \frac{c}{\sqrt{n} }
	= O_p \left( \frac{1}{\sqrt{n} } \right)
\]
in expectation.

Let's first compute the bound for \(\mathbb{E}_{}\left[(\mathbb{P} _n - \mathbb{P} ) (m_\theta - m_{\theta _0})\right] \). For a fixed \(\theta \) close to \(\theta _0\),
\[
	\begin{split}
		\mathbb{E}_{}\left[\vert (\mathbb{P} _n - \mathbb{P} ) (m_\theta - m_{\theta _0}) \vert \right]
		&\leq \sqrt{\Var_{}\left[\mathbb{P} _n(m_\theta - m_{\theta _0}) \right] } \\
		&= \frac{1}{\sqrt{n} } \sqrt{\Var_{}\left[m_\theta (X_1) - m_{\theta _0}(X_1) \right] } \\
		&\leq \frac{1}{\sqrt{n} } \sqrt{\mathbb{E}_{}\left[\left( m_\theta (X_1) - m_{\theta _0}(X_1) \right) ^2 \right] } .
	\end{split}
\]
In our \hyperref[prb:mode-estimation]{mode estimation} problem, if \(\theta \) is close to \(\theta _0\) (with \(\theta < \theta _0\)),
\[
	m_\theta (x) - m_{\theta _0}(x)
	= \mathbbm{1}_{\theta -1 \leq x \leq \theta _0 - 1} + \mathbbm{1}_{\theta +1 \leq x \leq \theta _0 + 1},
\]
hence
\[
	\begin{split}
		\mathbb{E}_{}\left[\left( m_\theta (X_1) - m_{\theta _0}(X_1) \right) ^2 \right]
		&\leq \mathbb{P} (\theta -1 \leq X_1 \leq \theta _0 - 1) + \mathbb{P} (\theta +1 \leq X_1 \leq \theta _0 + 1) \\
		&\leq 2 p(\theta _0) \cdot d(\theta , \theta _0) \\
		&\lesssim d(\theta , \theta _0)
	\end{split}
\]
since \(p\) is bounded. In all, we have
\[
	(\mathbb{P} _n - \mathbb{P} )(m_\theta - m_{\theta _0})
	= O_p \left( \sqrt{\frac{d(\theta , \theta _0)}{n}} \right) .
\]
Heuristically,
\[
	(\mathbb{P} _n - \mathbb{P} ) (m_{\hat{\theta} } - m_{\theta _0})
	= O_p \left( \sqrt{\frac{d(\hat{\theta} - \theta _0)}{n}} \right),
\]
which is a better bound than before. This implies
\[
	d(\hat{\theta} , \theta _0)^2
	\leq O_p \left( \sqrt{\frac{d(\hat{\theta} - \theta _0)}{n}} \right) .
\]
Canceling \(d(\hat{\theta} - \theta _0)\) from both sides,
\[
	d(\hat{\theta} , \theta _0)^{3 / 2} = O_p \left( \frac{1}{\sqrt{n} } \right)
	\implies d(\hat{\theta} , \theta _0) = O_p(n^{-1 / 3}),
\]
which is the correct rate for \(\hat{\theta} - \theta _0\).

\begin{remark}
	In fact, the limiting distribution is also known where we have
	\[
		n^{1 / 3} (\hat{\theta} - \theta _0) \overset{d}{\to } \argmax_{h\in \mathbb{R} } aB_h - b h^2,
	\]
	where \(a, b\) are constants depend on \(p\).
\end{remark}