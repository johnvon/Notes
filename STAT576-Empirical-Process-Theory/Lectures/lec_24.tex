\lecture{24}{25 Oct.\ 9:00}{Contraction Principle for Rademacher Complexity}

\begin{prev}
	Given \(\mathscr{F} = \{ f\colon \chi \to \mathscr{Y} \} \) and a loss \(\ell (f(x), y)\), the \hyperref[not:excess-risk]{excess risk} of \(\hat{f} \in \mathscr{F} \) is
	\[
		\mathbb{E}_{}[L(\hat{f} ) ] - \inf _{f\in \mathscr{F} } L(f)
		= \mathbb{E}_{}[\ell (\hat{f} (X), Y) ] - \inf _{f\in \mathscr{F} } \mathbb{E}_{}\left[\ell (f(X), Y) \right]
	\]
	where \(\hat{f} \) is the \hyperref[prb:ERM]{ERM} \(\hat{f} = \argmin_{f\in \mathscr{F} } \frac{1}{n} \sum_{i=1}^{n} \ell (f(x_i), y_i)\).
\end{prev}

We know that from \autoref{lma:ERM} and \hyperref[lma:symmetrization]{symmetrization}, the \hyperref[not:excess-risk]{excess risk} is bounded by
\[
	\mathbb{E}_{}\left[\sup _{f\in \mathscr{F} } \left( \mathbb{E}_{}\left[\ell (f(X), Y) \right] - \frac{1}{n} \sum_{i=1}^{n} \ell (f(x_i), y_i) \right) \right]
	\leq 2 \cdot \mathbb{E}_{}\left[\sup _{f\in \mathscr{F} } \frac{1}{n} \sum_{i=1}^{n} \epsilon _i \ell (f(x_i), y_i) \right] ,
\]
i.e., two times of the \hyperref[def:Rademacher-complexity]{Rademacher complexity} (actually the \hyperref[def:Rademacher-width]{Rademacher width}) of the set
\[
	\ell \circ \mathscr{F} \big( (x_1, y_1), \dots , (x_n, y_n) \big)
	\coloneqq \left\{ \ell (f(x_1), y_1), \dots , \ell (f(x_n), y_n) \right\} _{f\in \mathscr{F} }.
\]
We would like to further upper-bound this \hyperref[def:Rademacher-complexity]{Rademacher complexity} by the \hyperref[def:Rademacher-complexity]{Rademacher complexity} of \(\mathscr{F} \) only. This can be done if \(\ell \) is Lipschitz in the first argument for every fixed second argument.

\begin{theorem}[Contraction principle]\label{thm:contraction-principle}
	Suppose we have \(\Theta \subseteq \mathbb{R} ^n\) with the \hyperref[def:Rademacher-width]{Rademacher width} \(R(\Theta ) = \mathbb{E}_{}\left[\sup _{\theta \in \Theta } \langle \epsilon , \theta  \rangle \right] \). Let \(\phi _i \colon \mathbb{R} \to \mathbb{R} \) be \(1\)-Lipschitz for \(i=1, \dots , n\), and let
	\[
		\phi \circ \Theta \coloneqq \{ \left(\phi _1(\theta _1), \phi _2(\theta _2), \dots , \phi _n(\theta _n) \right) \colon \theta \in \Theta \}.
	\]
	Then, \(R(\phi \circ \Theta ) \leq R(\Theta )\).
\end{theorem}
\begin{proof}
	Let \(\phi \circ \theta = (\phi _1(\theta _1), \phi _2(\theta _2), \dots , \phi _n(\theta _n))\), then
	\[
		R(\phi \circ \Theta )
		= \mathbb{E}_{\epsilon _1, \dots , \epsilon _{n-1}}\left[ \mathbb{E}_{\epsilon _n}\left[ \sup _{\theta \in \Theta } \langle \phi \circ \theta , \epsilon  \rangle  \right] \right] .
	\]
	Condition on \(\epsilon _1, \dots , \epsilon _{n-1}\), we have
	\begin{align*}
		 & \mathbb{E}_{\epsilon _n}\left[ \sup _{\theta \in \Theta } \langle \phi \circ \theta , \epsilon  \rangle  \right]                                                                                                                                                                                                                                \\
		 & = \frac{1}{2} \left[ \sup _{\theta \in \Theta } \big( \langle (\phi \circ \theta) _{1 \colon n-1}, \epsilon _{1 \colon n-1} \rangle + \phi _n(\theta _n) \big) + \sup _{\theta ^{\prime} \in \Theta } \big( \langle (\phi \circ \theta ^{\prime}) _{1\colon n-1} , \epsilon _{1\colon n-1} \rangle - \phi _n (\theta ^{\prime} _n) \big)\right] \\
		 & = \frac{1}{2} \left[ \sup _{\theta , \theta ^{\prime} \in \Theta } \big( \langle (\phi \circ \theta )_{1 \colon n-1}, \epsilon _{1 \colon n-1} \rangle + \langle (\phi \circ \theta ^{\prime} )_{1\colon n-1} , \epsilon _{1 \colon n-1}\rangle + \phi _n(\theta _n) - \phi _n(\theta ^{\prime} _n) \big) \right]                               \\
		\shortintertext{sine \(\phi _n(\theta _n) - \phi _n(\theta _n^{\prime} ) \leq \vert \theta _n - \theta _n^{\prime} \vert \),}
		 & \leq \frac{1}{2} \left[ \sup _{\theta , \theta ^{\prime} \in \Theta } \big( \langle (\phi \circ \theta )_{1 \colon n-1}, \epsilon _{1 \colon n-1} \rangle + \langle (\phi \circ \theta ^{\prime} )_{1\colon n-1} , \epsilon _{1 \colon n-1} \rangle + \vert \theta _n - \theta _n^{\prime} \vert \big) \right]                                  \\
		 & = \frac{1}{2} \left[ \sup _{\theta , \theta ^{\prime} \in \Theta } \big(\langle (\phi \circ \theta )_{1 \colon n-1}, \epsilon _{1 \colon n-1} \rangle+ \langle (\phi \circ \theta ^{\prime} )_{1\colon n-1} , \epsilon _{1 \colon n-1} \rangle + \theta _n - \theta _n^{\prime} \big) \right] \tag*{symmetry of \(\theta , \theta ^{\prime} \)} \\
		 & = \frac{1}{2} \left[ \sup _{\theta \in \Theta } \big( \langle (\phi \circ \theta )_{1 \colon n-1}, \epsilon _{1 \colon n-1} \rangle + \theta _n \big) + \sup _{\theta ^{\prime} \in \Theta } \big( \langle (\phi \circ \theta ^{\prime} )_{1\colon n-1} , \epsilon _{1 \colon n-1} \rangle - \theta _n^{\prime} \big) \right]                   \\
		 & = \frac{1}{2} \left[ \sup _{\theta \in \Theta } \big( \langle (\phi \circ \theta )_{1 \colon n-1}, \epsilon _{1 \colon n-1} \rangle + \theta _n \big) + \sup _{\theta \in \Theta } \big( \langle (\phi \circ \theta )_{1\colon n-1} , \epsilon _{1 \colon n-1} \rangle - \theta _n^{\prime} \big) \right]                                       \\
		 & = \mathbb{E}_{\epsilon _n}\left[ \sup _{\theta \in \Theta } \langle (\phi \circ \theta )_{1 \colon n-1} , \epsilon _{1 \colon n-1} \rangle + \theta _n \epsilon _n \right].
	\end{align*}
	We see that we have got rid of \(\phi _n\)! Now, by considering iterate this method for all \(i\), and taking the expectation over all \(\epsilon _i\)'s, we can then get rid of \(\phi \) entirely, and hence get \(R(\Theta )\) finally.
\end{proof}

\begin{corollary}\label{col:contraction-principle}
	The same conclusion of \hyperref[thm:contraction-principle]{contraction principle} holds for \(\phi _i\)'s being \(L\)-Lipschitz, specifically, \(R(\phi \circ \Theta ) \leq L\cdot R(\Theta )\).
\end{corollary}

Now, going back to bounding \hyperref[not:excess-risk]{excess risk}, we let \(\phi _i (x) \coloneqq \ell (x, y_i)\).\footnote{The second argument is fixed at \(y_i\) for each \(\phi _i\) as we noted before.} If these \(\phi _i\)'s are \(L\)-Lipschitz, the \hyperref[col:contraction-principle]{contraction principle} with \(\Theta = \{ (f(x_1), \dots , f(x_n)) \} _{f\in \mathscr{F} } \subseteq \mathbb{R} ^n \) gives
\[
	\mathbb{E}_{}\left[\sup _{f\in \mathscr{F} } \frac{1}{n} \sum_{i=1}^{n} \epsilon _i \ell (f(x_i), y_i) \right]
	\leq L\cdot \mathbb{E}_{}\left[\sup _{f\in \mathscr{F} } \frac{1}{n} \sum_{i=1}^{n} \epsilon _i f(x_i) \right]
\]
where \(\ell (f(x_i), y_i) = \phi _i(f(x_i))\).

\begin{eg}
	Square loss is Lipschitz if \(\mathscr{F} \) is uniformly bounded and \(\mathscr{Y} \) is also uniformly bounded.
\end{eg}

\subsection{Constrained Least Square}
Now, we consider the generalization of \hyperref[prb:smooth-LS]{smooth constrained least square} with well specified case.

\begin{problem}[Constrained least square]\label{prb:constrained-LS}
Given a function class \(\mathscr{F} \) on \(\chi \), let \(x_1, \dots , x_n \in \chi \) and \(y_i = f^{\ast} (x_i) + \epsilon _i\) where \(\epsilon _1, \dots , \epsilon _n \overset{\text{i.i.d.} }{\sim } \mathcal{N} (0, \sigma ^2) \)\footnote{This is known as \emph{Gaussian sequence model}.} and assume that \(f^{\ast} \in \mathscr{F} \). Then, the \emph{constrained least square} aims to estimate \(f^{\ast} \) via
\[
	\hat{f} = \argmin_{f\in\mathscr{F} } \frac{1}{n} \sum_{i=1}^{n} (y_i - f(x_i))^2.
\]
\end{problem}

\begin{note}
	The \hyperref[prb:constrained-LS]{constrained least square} generalizes \hyperref[prb:smooth-LS]{smooth constrained least square} by a general given \(\mathscr{F} \) with data given not on the fixed grid, but just some fixed points in \(\chi \).
\end{note}

Next, we will see that ``local''\footnote{Recall the localization we did before, i.e., the \hyperref[def:localized-EP]{localized empirical process}. We will do the similar things here.} \hyperref[def:Gaussian-width]{Gaussian complexity} of \(\mathscr{F} \) around \(f^{\ast} \) determines the \hyperref[def:rate-of-convergence]{rate of convergence} of \(\hat{f} \). Specifically, the goal is to give non-asymptotic bound on
\[
	\lVert \hat{f} - f^{\ast} \rVert ^2 _{L_2(\mathbb{P} _n)}
	= \frac{1}{n} \sum_{i=1}^{n} \left( \hat{f} (x_i) - f^{\ast} (x_i) \right) ^2
\]
for \(\mathbb{P} _n = \{ x_1, \dots , x_n \} \).

\begin{note}
	In contrast, in random design, we assume \(x_1, \dots , x_n \overset{\text{i.i.d.} }{\sim } \mathbb{P} \), and the goal is to give bound on
	\[
		\lVert \hat{f} - f^{\ast} \rVert ^2_{L_2(\mathbb{P} )}
		= \mathbb{E}_{X}\left[(\hat{f} (X) - f^{\ast} (X))^2 \right] .
	\]
\end{note}

\begin{eg}[Image denoising]
	In the case of fixed design, a typical example is image denoising where \(x_i = (i / n, j / n)\) are the pixels.
\end{eg}