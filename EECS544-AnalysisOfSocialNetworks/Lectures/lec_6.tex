\lecture{6}{20 Sep. 12:30}{Homophily}
\subsection{Algorithm similar to BFS to find the distance between nodes}
Something like
\[
	d_G(i, j) = 1 + \min_{k\in N_G(j)} d_g(i, k).
\]

\subsubsection{Dynamic programming}
There are many different approach to calculate the distance between nodes, one way is \emph{Djikstra's algorithm}.

\subsection{Homophily}
Links in a (social) network are formed between people that are \emph{similar}. This is an important feature that leads to
the formation of communities.

\subsubsection{Define \emph{like} and \emph{opposite}}
Consider a network with two types of individuals:
\begin{eg}
	Example: Middle school. Consider a school of boys and girls. Suppose the fraction of boy is $p$ and the fraction of girls is $q = 1 - p$. We have
	\[
		\frac{\binom{np}{2}}{\binom{n}{2}} \cong p^2
	\]
	and also
	\[
		1 - p^2 - q^2 = 2pq.
	\]

	If the fraction of cross connection $\ll 2pq$, homophily is considered likely. Otherwise, if the fraction $\gg 2pq$, inverse-homophily is likely to occur.
\end{eg}

We see that some social implications arise from this
\begin{enumerate}
	\item schelling model
	\item parameters
	      \begin{enumerate}
		      \item satisfaction threshold
		      \item population of each partition
		      \item vacancy
	      \end{enumerate}
	\item a grid of X and O (denoting partition of agents).
\end{enumerate}

\begin{itemize}
	\item Satisfied Agent: A satisfied agent is one that is surrounded by at least \(t\) percent of agents that are like itself.
	\item Dynamics: When an agent is not satisfied, it can be moved to any vacant location in the grid.
\end{itemize}

