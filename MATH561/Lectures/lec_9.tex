\lecture{9}{29 Sep. 08:00}{Practical Simplex Algorithm}
%────────────────────────────────────────────────────────────────────────────────────────────────────────────────────────────────────────────────────
\subsection{\(A_{\beta}^{-1}\) in Reality}

\begin{note}
	In reality, we don't really calculate \(A^{-1}_{\beta}\), since in order to calculate
	\[
		A_{\beta}x_{\beta} = b,
	\]
	we do not use
	\[
		\overline{x}_{\beta} = A^{-1}_{\beta}b.
	\]
	Instead, we use \emph{LU-Factorization}. And since after applying pivot change,
	there is only a column change in \(A^{-1}_{\beta}\), we can use the previous result
	to calculate the new \(\overline{x}_{\beta}\) much faster.
\end{note}

%────────────────────────────────────────────────────────────────────────────────────────────────────────────────────────────────────────────────────
\subsection{Why \emph{Simplex}?}
For a standard form problem
\begin{align*}
	\min~ & c^Tx    \\
	      & Ax = b  \\
	      & x\geq 0
\end{align*}

But instead, we consider
\begin{align*}
	\min~ & z                                  \\
	      & z - c^{T} x = 0 \iff (c^{T} x = z) \\
	      & Ax = b                             \\
	      & x\geq 0
\end{align*}

\begin{prev}
	Our picture is in \(\mathbb{\MakeUppercase{R}}^{n-m}\), but we consider \emph{Dantzig picture}, which is in \(\mathbb{\MakeUppercase{R}}^{m+1}\)
\end{prev}

%────────────────────────────────────────────────────────────────────────────────────────────────────────────────────────────────────────────────────
\subsubsection{Column geometry}
Plot columns:
\[
	\underbrace{
		\begin{pmatrix}
			c_1 \\
			A_1 \\
		\end{pmatrix}
		\begin{pmatrix}
			c_2 \\
			A_2 \\
		\end{pmatrix}
		\cdots
		\begin{pmatrix}
			c_n \\
			A_n \\
		\end{pmatrix}}_{n \text{ points in }\mathbb{\MakeUppercase{R}}^{m+1}}
\]

The requirement line is
\[
	\begin{pmatrix}
		z \\
		b \\
	\end{pmatrix}.
\]

\begin{figure}[H]
	\centering
	\incfig{column-geometry}
	\caption{column-geometry}
	\label{fig:column-geometry}
\end{figure}

%────────────────────────────────────────────────────────────────────────────────────────────────────────────────────────────────────────────────────
\subsubsection{Simplices(plural of simplex)}
\begin{figure}[H]
	\centering
	\incfig{simplex}
	\caption{Simplex shapes}
	\label{fig:simplex}
\end{figure}

\begin{eg}
	Example of a simplex:
	\[
		\{x\in \mathbb{\MakeUppercase{R}}^n : \sum\limits_{i=1}^{n} x_i = 1, x_i \geq 0\}.
	\]
	Which is \(n-1\) dimensional simplex in \(\mathbb{\MakeUppercase{R}}^n\) with \(n\)
	standard unit vectors are the corners.
\end{eg}

\begin{note}
	\(m+1\) points of a simplex of dimension.
\end{note}

A simplicial cone is rather simple, the graph below is informative enough.
\begin{figure}[H]
	\centering
	\incfig{simplicial-cones}
	\caption{Simplicial Cones}
	\label{fig:simplicial-cones}
\end{figure}

%────────────────────────────────────────────────────────────────────────────────────────────────────────────────────────────────────────────────────
\subsection{Complete Simplex Algorithm}
Now, we have the standard form problem \((P)\):
\begin{align*}
	\min~ & c^Tx             \\
	      & Ax = b           \\
	      & x\geq 0 &  & (P)
\end{align*}
and the \emph{Phase one problem}\((\Phi)\)(Getting started):
\(A_{n+1} = -A_{\widetilde{\beta}}\vec{1}\):
\begin{itemize}
	\item First pivot is special
	\item Last pivot is special
\end{itemize}:
\begin{align*}
	\min~ & x_{n+1}                            \\
	      & Ax +A_{n+1} x_{n+1}= b             \\
	      & x\geq 0, x_{n+1}\geq 0 &  & (\Phi)
\end{align*}
with the perturbed problem \((P_{\epsilon})\)(making sure we stop):
\begin{align*}
	\min~ & c^Tx                                          \\
	      & Ax = b + B \vec{\epsilon}                     \\
	      & x\geq 0                   &  & (P_{\epsilon})
\end{align*}

%────────────────────────────────────────────────────────────────────────────────────────────────────────────────────────────────────────────────────
\section{Duality}
Consider the standard problem and its duality:
\[
	\begin{alignedat}{5}
		\min~&c^{T}x\qquad\qquad &&\max ~ &&y^{T}b\\
		&Ax = b && &&y^{T}A\leq c^{T}\\
		(P)\quad&x\geq  0 &&(D)\quad&&
	\end{alignedat}.
\]

\begin{prev}
	Weak duality theorem: If \(\hat{x}\) is feasible for \(P\), and \(\hat{y}\)
	is feasible for \(D\), then
	\[
		c^{T} \hat{x} \geq  \hat{y}^{T} b.
	\]
	Moreover, the equality holds if and only if \(\hat{x}\) and \(\hat{y}\) are optimal.
\end{prev}

\begin{theorem}
	Weak Optimal Basis Theorem: If we have a basic partition \(\beta, \eta\), and we also have
	\(\overline{x}_{\beta}\geq  \vec{0}\)(\(\overline{x}\) is feasible for \(P\)) and \(\overline{c}_{\eta} \geq  \vec{0}\)(\(\overline{y}\) is feasible for \(D\))
	\[
		\implies \overline{x}\ \&\ \overline{y} \text{ are optimal}.
	\]
\end{theorem}

Now, we have so-called \emph{strong optimal basis theorem}.
\begin{theorem}
	Strong Optimal Basis Theorem: If \((P)\) has a feasible solution, and if \((P)\) is not unbounded, then there exist a basic partition \(\beta, \eta\) such that
	\(\overline{x}\) and \(\overline{y}\) are optimal, and
	\[
		c^{T} \overline{x} = \overline{y}^{T} b.
	\]
\end{theorem}
\begin{proof}
	Since if \(P\) has a feasible solution and is not unbounded, we can just run the Simplex Algorithm, which will terminate with a basis \(\beta\) such
	that the associated basic solution \(\overline{x}\) and the associated dual solution \(\overline{y}\) are optimal.
\end{proof}

We see that this leads to another similar result.
\begin{theorem}
	\label{strong duality theorem}
	Strong Duality Theorem: If \(P\) has a feasible solution and \(P\) is not unbounded, then there exist optimal solutions
	\(\hat{x}\) and \(\hat{y}\) with
	\[
		c^{T} \hat{x} = \hat{y}^{T} b.
	\]
\end{theorem}

\begin{note}
	The proof of these two theorems are by directly using the \emph{mathematical complete} version of Simplex Algorithm, hence the completeness of
	Simplex Algorithm(namely the Phase I problem and the perturbation) is important.
\end{note}

\begin{table}[H]
	\centering
	\begin{tabular}{c|c|c|c|c}
		\toprule
		\emph{Simplex Algorithm}                             & P\(\backslash\) D          & \begin{tabular}{@{}c@{}}optimal\\solution\end{tabular} & infeasible  & unbounded   \\
		\midrule
		\(\overline{c}_{\eta}\geq \vec{0}\implies\) Stop     & \begin{tabular}{@{}c@{}}optimal\\solution\end{tabular} & \(\surd \)                 & \(\times \) & \(\times \) \\
		\midrule
		\begin{tabular}{@{}c@{}}optimal \(x_{n+1}\) in \(\Phi\) \\is positive\end{tabular}                           & infeasible                 & \(\times \)                & \(\surd\)   & \(\surd \)  \\
		\midrule
		\(\overline{A}_{\eta_{j}}\leq \vec{0}\implies\) Stop & unbounded                  & \(\times \)                & \(\surd \)  & \(\times \) \\
		\bottomrule
	\end{tabular}
	\caption{Comparison between \(P\) and \(D\) }
	\label{tab:label}
\end{table}