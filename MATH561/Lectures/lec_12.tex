\lecture{12}{11 Oct. 08:00}{Farkas' Lemma}
Before we prove Farkas' Lemma, we first see something similar. There is a lemma called \emph{Gauss' Lemma}, which is highly related to Farkas' Lemma.
\begin{lemma}
	Gauss' Lemma\(\colon\) Exactly one of the following has a solution\(\colon\)
	\[
		\begin{alignedat}{3}
			& (I) \qquad&& Ax = b       \\\\
			& (II) \qquad&& y^{T}A\geq 0 \\
			&      && y^{T}b\neq  0
		\end{alignedat}
	\]
\end{lemma}

This just follows from the Gauss elimination. By doing the elimination, there are two cases\(\colon\)
\begin{enumerate}
	\item The system has no solution.
	\item There is a(some) solution(s).
\end{enumerate}

For second case, it's just \(Ax = b\) is solvable.  For the fist case, we see that after the elimination, we will have something like
\[
	\begin{pmatrix}
		  &   &   &        &   &   &   \\
		  &   &   &        &   &   &   \\
		0 & 0 & 0 & \ldots & 0 & 0 & 0 \\
		  &   &   &        &   &   &   \\
		  &   &   &        &   &   &   \\
	\end{pmatrix}
	\begin{pmatrix}
		\\
		\\
		a \\
		\\
		\\
	\end{pmatrix}
\]
where \(a\neq 0\), which just indicates this system is unsolvable.

Now we start to proof Farkas' Lemma.
\begin{proof}
	As what we have outlined, we divide the proof into two cases.
	\begin{enumerate}
		\item I \& II can't both have solutions. Suppose \(\hat{x}\) solves I and \(\hat{y}\) solves II. Then we have
		      \[
			      \hat{y}^{T}(\hat{Ax} = b) \implies \underbrace{(\hat{y}^{T} A)}_{\geq  \vec{0}}\underbrace{\vphantom{\hat{y}^{T}}\hat{x}}_{\geq  \vec{0}} = \hat{y} b > 0\contd
		      \]
		\item At least one of I or II has a solution \(\cong\) If I has no solution, then II has a solution. Assume that I has no solution, which
		      means that \(P\) is infeasible with \(P\) being
		      \begin{align*}
			      \max~    & \vec{0}^{T} x \\
			               & Ax = b        \\
			      (P)\quad & x\geq 0.
		      \end{align*}

		      The dual of this \(P\) is
		      \begin{align*}
			      \max~    & y^{T}b                    \\
			      (D)\quad & y^{T} A \leq \vec{0}^{T}.
		      \end{align*}

		      But this means that \(D\) is infeasible or unbounded.	But we see that \(D\) can't be infeasible, because \(y = \vec{0}\) is a
		      feasible solution, then we know
		      \[
			      \begin{split}
				      &\implies D \text{ is unbounded} \\
				      &\implies \text{ there exist a feasible solution \(\widetilde{y}\) to \(D\) with positive objective}
			      \end{split}
		      \]
	\end{enumerate}
\end{proof}

\begin{remark}
	Now, consider \(\lambda \widetilde{y}\) (feasible for \(D\)). Drive to \(+\infty \) by increasing \(\lambda\). We now see what Farkas' Lemma really tells us.
	\begin{align*}
		\max~    & c^{T} x                                            \\
		         & Ax = b                                             \\
		(P)\quad & x\geq 0            &  & \text{feasibility}         \\
		         &                    &  & \Updownarrow               \\
		\max~    & y^{T}b             &  & \text{unbounded direction} \\
		(D)\quad & y^{T} A \leq c^{T}                                 \\
	\end{align*}
	Suppose \(\widetilde{y}\) is feasible to \(D\) and suppose \(\hat{y}\) satisfies II, then
	\[
		(\widetilde{y}+\lambda \hat{y})^{T} A = \underbrace{\widetilde{y}^{T}A}_{\leq c^{T}} + \underbrace{\vphantom{y}\lambda}_{>0} \underbrace{\hat{y}^{T} A}_{\leq \vec{0}} \leq c^{T}.
	\]
	Furthermore, we have
	\[
		(\widetilde{y}+\lambda \hat{y})^{T} b = \widetilde{y}^{T} b+\lambda \hat{y}^{T} b \implies \infty \text{ as } \lambda \uparrow  .
	\]
\end{remark}

\begin{eg}
	\begin{align*}
		(I)\quad  & Ax\leq b \\
		(II)\quad & ?
	\end{align*}
	Find out what II is.

	We simply set up the \(P\) and then find its dual.
	\[
		\begin{alignedat}{5}
			\max~&\vec{0}^{T} x\qquad\qquad	&&\max ~&&y^{T}b\\
			&Ax \leq b 			&&			&&y^{T} A =\vec{0}\\
			(P)\quad&			&&(D)\quad	&&y\leq \vec{0}
		\end{alignedat}.
	\]

	Then we have
	\begin{align*}
		(I)\quad  & Ax\leq b         \\
		(II)\quad & y^{T}A = \vec{0} \\
		          & y\leq \vec{0}    \\
		          & y^{T} b>0
	\end{align*}
	Check\(\colon\)
	\[
		0 = \underbrace{\hat{y}^{T} A}_{=\vec{0}} \hat{x} \underset{\hat{y} \leq \vec{0}}{\geq}  \hat{y}^{T} b>0 \contd
	\]
	or,
	\[
		\begin{split}
			&Ax \overset{y\leq \vec{0}}{\leq} b\qquad (y^{T}b>0)\\
			&0 \overset{?}{\geq} \underbrace{y^{T}A}_{ = \vec{0}} x\geq y^{T}b>0\contd
		\end{split}
	\]
\end{eg}

\begin{eg}
	\begin{align*}
		(\min~   & \vec{0}^{T} x + \vec{0}^{T} w)   \\
		         & Ax + Bw = b                      \\
		         & -Fw \geq f                       \\
		(I)\quad & x\geq  0,\ w\text{ unrestricted} \\
	\end{align*}
	with the dual variables \(y, w\), we have
	\begin{align*}
		               & (\text{Suppose I has no solution.}) \\
		\cancel{\max}~ & y^{T} b + v^{T} b (>0)              \\
		               & y^{T} A \leq \vec{0}                \\
		(II)\quad      & y^{T} B - v^{T} F = \vec{0}
	\end{align*}
	with \(y\) unrestricted, \(v\geq  \vec{0}\).
\end{eg}

\hr

Now, we should have a general picture about what Farkas' Lemma really means. For conditions I and II, we have
\[
	\begin{alignedat}{3}
		& (I) \qquad&& Ax = b       \\
		&      && x\geq 0    && \iff b \text{ is in the cone }K \\
		& (II) \qquad&& y^{T}b> 0 &&\iff y \text{ makes an acute angle with }b.\\
		&      && y^{T}A\leq 0^{T}&&\quad y\text{ makes a non-acute angle with all columns of }A
	\end{alignedat}
\]

Suppose \(\hat{z}\) in \(K\), then
\[
	\hat{z} = A \hat{x} \text{ for some }\hat{x} \geq  \vec{0}.
\]
Then we have
\[
	y^{T} \hat{z} = \underbrace{y^{T} A}_{\leq\vec{0}^{T}} \underbrace{\vphantom{y}\vec{x}}_{\geq  \vec{0}}  \leq 0.
\]

We see that \(y\) makes a non-acute angle with everything in \(K\). Now, suppose \(\hat{y}\) solves II. Consider
\[
	\underbrace{\hat{y}^{T}}_{\text{numbers}} \underbrace{\vphantom{y}z}_{\text{variables}} = 0.
\]

Now, we have the hyperplane\(\colon\)\(\{z\colon \hat{y} ^{T} z = 0\}\) separates \(b\) and \(K\).

\begin{figure}[H]
	\centering
	\incfig{Farkas-lemma-extended}
	\caption{Case II of the Farkas' Lemma with \(m = 2\)}
	\label{fig:Farkas-lemma-extended}
\end{figure}

%────────────────────────────────────────────────────────────────────────────────────────────────────────────────────────────────────────────────────
\subsection{The Big Picture of Cones}
Consider the linear programming problem
\begin{align*}
	\max~ & y^{T} b            \\
	      & y^{T} A \leq c^{T} \\
\end{align*}
with the partition \(\beta, \eta\), we see that
\[
	y^{T} A \leq c^{T} \implies \begin{dcases}
		y^{T} A_{\beta} & \leq c^{T}_{\beta} \\
		y^{T} A_{\eta}  & \leq c^{T}_{\eta}
	\end{dcases}.
\]
By solving only for \(\beta\), then we have \(\overline{y}^{T} = c_{\beta}^{T} A^{-1}_{\beta}\). And then, by considering the cones, we have

\begin{figure}[H]
	\centering
	\incfig{opt-cones}
	\caption{Optimality of Cones. (* This corresponds to the case that we run into the overlapping issue in \autoref{fig:cones-join})}
	\label{fig:opt-cones}
\end{figure}
with
\begin{figure}[H]
	\centering
	\incfig{cones-join}
	\caption{Cones join together}
	\label{fig:cones-join}
\end{figure}

\begin{note}
	Consider \(b = \vec{0}\)(\(\hat{y}\)). It's in every cone \(\implies\) every point is optimal.
\end{note}

\begin{remark}
	We see that each corner(extreme point) corresponds to a solution for \(\beta\), while the blue vector \(\vec{b}\) corresponds to the dual constraints
	\(y^{T} A_{\eta}<c_{\eta}^{T}\). Only when the blue vector are in the region of orange sectors span by two \emph{normal vectors} of
	\(y^{T}A_{\cdot \beta_i}\leq c_{\beta_i}\), the constraints are satisfied.
\end{remark}

\begin{eg}
	Exercise 5.5. \cancel{Over} Strictly Complementarity.
	Consider
	\[
		\begin{alignedat}{5}
			\min~&c^{T}x\qquad\qquad&&\max ~&&y^{T}b\\
			&Ax = b 				&&		&&y^{T}A\leq \vec{0}.\\
			(P)\quad	&x\geq  0 	&&(D)\quad&&
		\end{alignedat}
	\]
	\begin{prev}
		Complementarity of \(\hat{x}\) and \(\hat{y}\)\(\colon\)
		\[
			\begin{split}
				(c_{j} - \hat{y}^{T} A_{\cdot j}) \hat{x}_j = 0&, \text{ for }j = 1\ldots n\\
				y^{T}_i (A_{i\cdot}\hat{x} - b_{i}) = 0&, \text{ for } i = 1\ldots m
			\end{split}
		\]
	\end{prev}

	\begin{definition}
		For feasible solutions \(\hat{x}\) and \(\hat{y}\) are strictly complementary if they are complementary and exactly one of
		\[
			c_{j} - \hat{y}^{T}A_{\cdot j}\text{ and }\hat{x}_j \text{ is } 0.
		\]
	\end{definition}

	\begin{theorem}
		Strictly Complementarity\(\colon\) If \(P\) and \(D\) are both feasible, then for \(P\) and \(D\) \underline{there exist} \underline{strictly}
		complementary(feasible) optimal solutions.
	\end{theorem}

	\begin{intuition}
		Let \(v\) be the optimal value of \(P\)\(\colon\)
		\begin{align*}
			v = \min~ & c^Tx    \\
			          & Ax = b  \\
			(P)\quad  & x\geq 0
		\end{align*}
		Now, we try to find an optimal solution with
		\[
			x_{j}>0, \quad \text{fix }j
		\]
		by formulating the following linear programming
		\begin{align*}
			\max~      & x_{j}        \\
			           & c^{T}x\leq v \\
			           & Ax = b       \\
			(P_j)\quad & x\geq  0
		\end{align*}
		where \(P_{j}\) seeks an optimal solution of \(P\) that has \(x_{j}\) being positive. If failed, then construct an optimal solution \(\hat{y}\) to \(D\) with
		\[
			c_{j} - \hat{y}^{T} A_{\cdot j}>0.
		\]

		We then see for any \textbf{fixed} \(j\), the desired property holds. The only thing we need to do is combine these \(n\) pairs of \(\hat{x}\) and \(\hat{y}\)
		appropriately to construct optimal \(\hat{x}\) and \(\hat{y}\) that are overly complementary.
	\end{intuition}
\end{eg}