\lecture{17}{3 Nov. 08:00}{Large-Scale Linear Optimization}
\begin{prev}
	We now focus on one particular problems: What's the conditions for a variable to enter the basis?
	\begin{enumerate}
		\item What's the reduced coast of \(s_{i}\)?
		      \[
			      0 - \begin{pmatrix}
				      \overline{y}^{T} & \overline{\sigma} \\
			      \end{pmatrix}\begin{pmatrix}
				      -e_{i} \\
				      0      \\
			      \end{pmatrix} = \overline{y}_i.
		      \]
		      If \(\overline{y}_i<0\), then \(s_{i}\) can enter the basis.
		\item What's the reduced cost of \(\lambda_{j}\)?
		      \[
			      (c^{T}\hat{x}^j) - \begin{pmatrix}
				      \overline{y}^{T} & \overline{\sigma} \\
			      \end{pmatrix}\begin{pmatrix}
				      E\hat{x}^j \\
				      1          \\
			      \end{pmatrix} = c^{T}\hat{x}^j - \hat{y}^{T}E\hat{x}^j - \overline{\sigma} = (c^{T} - \overline{y}^{T}E)\hat{x}^j - \overline{\sigma}.
		      \]
		      We consider a sub problem
		      \begin{align*}
			      -\sigma + \min~     & (c^{T} - \overline{y}^{T}E)x \\
			                          & Ax = b                       \\
			      (\mathrm{SUB})\quad & x\geq 0.
		      \end{align*}

		      If the optimal values\(<0\), then the optimal basic solution \(\hat{x}^j\) has an associated \(\lambda_{j}\) with negative reduced cost, so \(\lambda_{j}\) can
		      enter the basis  of \(M\). Else if the optimal value \(\geq 0\), then no \(\lambda_{j}\) can enter the basis.

		      \begin{note}
			      We need to include \(-\sigma\) for evaluating the optimal values.
		      \end{note}

		      \begin{problem}
		      What if the optimal value is unbounded?
		      \end{problem}
		\item What's the reduced cost of \(\mu^k\)?
		      \[
			      (c^{T}\hat{z}^k) - \begin{pmatrix}
				      \overline{y}^{T} & \overline{\sigma} \\
			      \end{pmatrix}\begin{pmatrix}
				      E \hat{z}^k \\
				      0           \\
			      \end{pmatrix} =  (c^{T} - \overline{y}^{T}E)\hat{z}^k.
		      \]
		      Again, consider a sub problem
		      \begin{align*}
			      \min~ & (c^{T} - \overline{y}^{T}E)z \\
			            & Az = \vec{0}                 \\
			            & z\geq 0
		      \end{align*}

		      \begin{remark}
			      Compare this problem to the previous sub problem \(\mathrm{SUB}\).
			      \begin{enumerate}
				      \item Notice that the objective value of this problem will always be \(0\) or unbounded. Since \(0\) is always a feasible solution, or
				            if once it's negative, we can multiply it by a positive number and make the optimal values smaller.
				      \item When solving \(\mathrm{SUB}\), the optimal values of \(\mathrm{SUB}\) is
				            \begin{enumerate}
					            \item negative \(\implies \lambda_{j}\) to enter the basis.
					            \item \underline{non-negative} \(\implies\) no \(\lambda_{j}\) can enter the basis.
					            \item unbounded \(\implies\) we get a \(\overline{z}\) that is a basic ray with \(c^{T}\overline{z} <0\), which implies for some \(\hat{z}^k\), \(\mu^k\) with negative reduced cost.
				            \end{enumerate}
			      \end{enumerate}
			      \begin{note}
				      We stop when \(\mathrm{SUB}\) has the optimal values being \(0\).\todo{WHy the optimal values of \(\mathrm{SUB}\) will always be non-positive?}
			      \end{note}
		      \end{remark}
	\end{enumerate}
\end{prev}

Now, we know what variable can enter the basis, but we have not yet consider what variable can leave. Recall that the basic matrix \(B\) for \(\overline{M}\) will consists
the following columns
\[
	\qquad s_{i} = \begin{pmatrix}
		0      \\
		1      \\
		\vdots \\
		0      \\
		\bm{0} \\
	\end{pmatrix}, \qquad \lambda_{j} = \begin{pmatrix}
		\\
		E \hat{x}^j \\
		\\
		\bm{1}      \\
	\end{pmatrix}, \qquad \mu^k =\begin{pmatrix}
		\\
		E \hat{z}^k \\
		\\
		\bm{0}      \\
	\end{pmatrix},
\]
where we see that the last entries of \(\lambda_{j}\) will always be \(1\), and at least one of \(\lambda_{j}\) will be in the basis due to the fact that \(B\) is invertible.
For simplicity, we just consider
\[
	B = \underset{s_1\ s_2\quad \ldots\ s_k\quad \lambda_1}{
		\begin{pmatrix}
			  &        &   &             \\
			  & -I     &   & E \hat{x}^1 \\
			  &        &   &             \\
			0 & \ldots & 0 & 1           \\
		\end{pmatrix}}
\]
where we get \(\hat{x}^1\) by solving
\begin{align*}
	\min~ & e^{T}\hat{x} \\
	      & Ax = b       \\
	      & x\geq 0
\end{align*}

If \(E \hat{x}^1\geq h\), then \(\overline{s}\geq \vec{0}\implies\) directly go to Phase II. Then,
\[
	\begin{pmatrix}
		\overline{y}^{T} & \overline{\sigma} \\
	\end{pmatrix} = \begin{pmatrix}
		(\overline{c}\hat{x}^j) & (c^{T}\hat{z}^k) & 0 \\
	\end{pmatrix}B^{-1},
\]
where \(\overline{c}\hat{x}^j\) initially is
\[
	\begin{pmatrix}
		0 & \ldots & 0 & c^{T}\hat{x}^1 \\
	\end{pmatrix}.
\]

Recall the ratio test for determining what entry should enter the basis and what should leave. Namely,
\[
	\overline{y}^{T} = c_{\beta}^{T} A_{\beta}^{-1},\quad \overline{x}_{\beta} = A_{\beta}^{-1} b = \begin{pmatrix}
		\overline{x}_{\beta_1} \\
		\vdots                 \\
		\overline{x}_{\beta_m} \\
	\end{pmatrix}, \quad \overline{A}_{\eta_{j}} = A_{\beta}^{-1} A_{\eta_{j}} = \begin{pmatrix}
		\overline{a}_{1, \eta_{j}} \\
		\vdots                     \\
		\overline{a}_{m, \eta_{j}} \\
	\end{pmatrix}
\]
with the ratio being
\[
	\min_{i\colon \overline{a}_{i, \eta_{j}}>0}\left\{\frac{\overline{x}_{\beta_{i}}}{\overline{a}_{i, \eta_{j}}}\right\}.
\]

Now, in our situation, we carry out the ratio test by noting that the basic variable values is just
\[
	B^{-1}\begin{pmatrix}
		h \\
		1 \\
	\end{pmatrix},
\]
and the updated entering column is
\[
	B^{-1} 	\begin{pmatrix}
		-e_{i} \\
		0      \\
	\end{pmatrix} \text{ or } B^{-1}\begin{pmatrix}
		E \hat{x}^j \\
		1           \\
	\end{pmatrix} \text{ or } B^{-1}\begin{pmatrix}
		E \hat{z}^k \\
		0           \\
	\end{pmatrix},
\]
which corresponds to \(\lambda_{j}\), \(\mu_k\), \(s_{i}\) is entering the basis, respectively.

Then we just do the ratio test. If \(B^{-1}\begin{pmatrix}
	h \\
	1 \\
\end{pmatrix}\geq \vec{0}\implies\) go to Phase II. If not we create an artificial column
\[
	\begin{pmatrix}
		E\hat{x}^1 \\
		1          \\
	\end{pmatrix}.
\]