\lecture{22}{29 Nov. 08:00}{Optimization of Integer Variables}
\section{Integer-Linear Optimization}
Let first see some common pitfalls of integer programming.

\begin{itemize}
	\item If \(A \) has big entries and small entries, then these two constraints is like parallel to each other,
	      which will lead the intersection be very far away. Then, if we simply round down the variable, the optimal value will drop significantly.
	      \begin{figure}[H]
		      \centering
		      \incfig{integer-programming-pitfall}
		      \caption{Pitfall of Integer Programming}
		      \label{fig:integer-programming-pitfall}
	      \end{figure}

	      But as one can see, we can often avoid this situation by carefully design our model and the problem is solved.
	\item Another possibility is the following. Consider an integer with the following constraints.
	      \[
		      \begin{alignedat}{3}
			      & \underset{1 < i < j < n}{\forall}\      &  & x_{i}+x_{j} &  & \leq 1                  \\
			      & \underset{i = 1, \ldots , n}{\forall}\  &  & x_{i}       &  & \geq 0 \text{ integer}.
		      \end{alignedat}
	      \]

	      Then, there are two feasible solutions one can observe immediately, namely
	      \[
		      x_1 = x_2 = \ldots = x_{n} = \frac{1}{2};
	      \]
	      and
	      \[
		      x_1 = 1, x_2 = \ldots = x_{n} = 0.
	      \]
	      It's then really hard to tell which is better. But again, if the right-hand side is \(2\) for the first constraint, then the problem
	      is gone.
\end{itemize}

\begin{note}
	We see that this is totally opposite to the linear programming. The modern integer programming solver can easily solve a programming with
	like one hundred of variables, but the in practice, we're often facing more than thousands of variables. The one need to carefully design
	his model in terms of number of variables.
\end{note}

\subsection{Modeling Techniques}
\begin{eg}
	Consider the following constraints.
	\[
		-12\leq x\leq 2\ \lor\ 5\leq x\leq 20.
	\]
	\begin{figure}[H]
		\centering
		\incfig{integer-programming-eg1}
		\label{fig:integer-programming-eg1}
	\end{figure}
	Then we need to find the smallest convex set which contains all feasible points. It's just
	\[
		-12\leq x\leq 20.
	\]
	But that empty space between \(2\) and \(5\) causes the problem. To solve this, we simply introduce a new \emph{indicator variable}, denotes
	by \(y\). \(y\) will be \(0\) if we are in \([-12, 2]\), and \(1\) if we are in \([5, 20]\). Then the smallest convex feasible region becomes
	the following quadrilateral.
	\begin{figure}[H]
		\centering
		\incfig{integer-programming-eg1.2}
		\label{fig:integer-programming-eg1.2}
	\end{figure}
	Put it in the constraints, we have
	\[
		\begin{split}
			-12&\leq x\leq 20\\
			0&\leq y\leq 1, \text{ integer}\\
			x&\leq 2+M_1 y\\
			x&+M_2(1 - y)\geq 5,
		\end{split}
	\]
	where we let \(M_1\) be big enough to let the constraints always be satisfied when \(x\) is in \([5, 20]\). For example, we can let \(M_1\coloneqq 18\),
	then
	\[
		\begin{dcases}
			y = 0,\ x\leq 2 \\
			y = 1,\ x\leq 20.
		\end{dcases}
	\]
	Analogously, we use \(M_2\) to help us to model the case that when \(y = 1\), \(x\geq 5\) and when \(y = 0\), \(x\geq -12\). For example, we can let
	\(M_2 \coloneqq 17\).

	The last three constraints exactly corresponds to the line segment in the graph:
	\begin{figure}[H]
		\centering
		\incfig{integer-programming-eg1.3}
		\label{fig:integer-programming-eg1.3}
	\end{figure}

	We further see that if we make the constant \(M_{i}\) too large, we will have
	\begin{figure}[H]
		\centering
		\incfig{integer-programming-eg1.4}
		\label{fig:integer-programming-eg1.4}
	\end{figure}
	In terms of integer programming, this doesn't affect the integer feasible region. We call this \textbf{Big-M Method}. Although
	this is fine mathematically, but this is unfriendly to the solver.
\end{eg}

\subsection{Uncapacitated Facility-location Problem}
Assume that there are \(m\) facilities with the fixed costs \(f_{i}\), \(i = 1, \ldots , m\). And assume there are \(n\) customers,
denote by \(j = 1, \ldots , n\). Now, let \(c_{ij}\) be the cost of satisfying all demand of customer \(j\) from facility \(i\).
The goal is to minimize the total cost of satisfying all customer's demand. We then define our variables as \(x_{ij}\) such that
\[
	x_{ij} \coloneqq  \text{proportion of customer \(j\) demand satisfied facility \(i\)},
\]
where \(i = 1, \ldots , m\), \(j = 1, \ldots , n\). Furthermore, we need indicator variables \(y_{i}\) such that
\[
	y_{i}\coloneqq \begin{dcases}
		1, & \text{ if facility \(i\) operates} \\
		0, & \text{ if not}
	\end{dcases}
\]
for all \(i = 1, \ldots , m\).

The optimization problem can now be modeled as
\begin{align*}
	\min~ & \sum\limits_{i=1}^{m} f_{i}y_{i} + \sum\limits_{i=1}^{m}\sum\limits_{j=1}^{n} c_{ij}x_{ij}                                                         \\
	      & \sum\limits_{i=1}^{m} x_{ij} = 1,                                                          &  & \text{ for }j = 1, \ldots , n                      \\
	      & -y_{i} + x_{ij}\leq 0,                                                                     &  & \text{ for }i = 1, \ldots , m,\ j = 1, \ldots , n  \\
	      & 0\leq y_{i}\leq 1 \text{ integers},                                                        &  & \text{ for }i = 1, \ldots , m                      \\
	      & x_{ij}\geq 0,                                                                              &  & \text{ for }i = 1, \ldots , m,\ j = 1, \ldots , n.
\end{align*}

\begin{note}
	The second constraint
	\[
		\sum\limits_{i=1}^{m} x_{ij} = 1,\text{ for }j = 1, \ldots , n
	\]
	is for the following reason. For any \(i\), if \(x_{ij}\) is positive for any \(j\), then we need \(y_{i} = 1\) to \emph{force} us to pay the fixed cost
	to operate the facility if anything is shipped out of facility \(i\). To get this constraint, we first see that we want
	\[
		x_{ij}>0 \implies y_{i} = 1
	\]
	for any \(j\). It is equivalent to say
	\[
		\sum\limits_{j=1}^{n} x_{ij}>0 \implies y_{i} = 1.
	\]
	We see that from the first expression, the constraint immediately follows. As for the second constraint, we start from considering
	\[
		\sum\limits_{j=1}^{n} x_{ij}\leq y_{i}
	\]
	for every \(i = 1, \ldots , m\). But this causes some problem. If the facility is really cheap, then the sum may exceed \(1\).
	To solve this problem, we simply make \(y\) become \(n\cdot y\), namely
	\[
		\sum\limits_{j=1}^{n} x_{ij}\leq n\cdot y_{i}
	\]
	for \(i = 1, \ldots, m\), where \(n\) is just the \textbf{Big-M} in the big-M Method.
\end{note}

We now have two equivalent constraints, namely
\[
	\underset{i, j}{\forall }\ -y_{i} + x_{ij}\leq 0\quad  \text{ and }\quad \underset{i}{\forall }\ \sum\limits_{j=1}^{n} x_{ij}\leq n\cdot y_{i}.
\]
Now the problem is which to use? The answer is the first one. We call the first model as the \emph{strong model}, while the second model as
the \emph{weak model}.

\begin{intuition}
	The second model has the big-M constant. As we just discuss, we prefer \(M\) to be as small as possible. But in the first model, we don't have that
	big-M coefficient. And since
	\[
		\sum\limits_{j=1}^{n} \left(x_{ij}\leq y_i\right) \iff \sum\limits_{j=1}^{n} x_{ij}\leq n y_{i},
	\]
	we see that the weak constraint is just the sum over all strong constraint. In other words, we have
	\[
		x_{ij}\leq y_i \implies \sum\limits_{j=1}^{n} x_{ij}\leq n y_{i}.
	\]
	\begin{figure}[H]
		\centering
		\incfig{strong-weak-constraint}
		\caption{Venn diagram of Strong/Weak Feasible Region}
		\label{fig:strong-weak-constraint}
	\end{figure}

	\begin{eg}
		For \(m = 2\), \(n = 3\), find \(x, y\) where \emph{weak} constraints are satisfied while \emph{strong} constraints are not.
		\begin{figure}[H]
			\centering
			\incfig{integer-programming-eg2}
			\label{fig:integer-programming-eg2}
		\end{figure}
		It's easy to check that \(x_{11}\nleq y_1\), but
		\[
			x_{11}+x_{12} \leq 3y_1
		\]
		and
		\[
			x_{22}+x_{23}\leq 3y_2.
		\]
	\end{eg}
\end{intuition}
\begin{remark}
	It's important to see that although we said we should keep the number of variables down when setting up the integer programming, but in this case,
	few is not always better!
\end{remark}

It's worth noting that the process of un-summing from the weak constraint to the strong constraint is called \emph{disaggregation}.

\hr

We now see some potential algorithm to solve the integer-programming problem.
\begin{itemize}
	\item \textbf{Cutting-Plane} algorithm. If we have the following feasible region, then the cutting-plane algorithm suggests
	      that we should use a plane at a corner(corresponds to an optimal solution to the linear version of this programming) and
	      \emph{reduce} the feasible region by a little until we touch an integer point.
	      \begin{figure}[H]
		      \centering
		      \incfig{integer-programming-cutting-plane}
		      \label{fig:integer-programming-cutting-plane}
	      \end{figure}
	\item \label{eg:branch-and-bound}\textbf{Branch-and-Bound} algorithm. We first consider the following feasible region and try to use cutting-plane algorithm.
	      \begin{figure}[H]
		      \centering
		      \incfig{integer-programming-branch-and-bound}
		      \label{fig:integer-programming-branch-and-bound}
	      \end{figure}
	      We see that if we simply start from cutting-plane algorithm, it takes forever to get to the answer. More generally, when the
	      integer solution is far from the linear solution, the cutting-plane algorithm performs poorly.

	      \par Instead, we consider Branch-and-Bound algorithm. It essentially just goes from \(n\) variables to two \(n-1\) variables
	      programming problem, and until we get to the bottom(\(1\) variable). We see that we are doing exactly the opposite with what we have
	      introduced, namely we are not modeling the \textbf{or}, but bring it into the algorithm. In this example, right after we branch,
	      we solve the problem instantly since there in both branches, we only have one point to consider.
\end{itemize}
\begin{note}
	Every modern solver which solves the integer programming exactly, will first go for branch-and-bound method, and then on top of that, solve the
	remaining problem by cutting-plane algorithm.
\end{note}