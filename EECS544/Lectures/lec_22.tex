\lecture{22}{29 Nov. 12:30}{Auctions}
\begin{eg}
	There are more types of auction.
	\begin{enumerate}
		\item[3.] Second-price auction(Sealed bid):
			\begin{enumerate}
				\item Sort the bids from the highest
				\item Find the highest bidder
				\item Give the good to this bidder but charge him/her the second-highest bid.
			\end{enumerate}
			\par Ties breaking:
			\begin{itemize}
				\item Random order.
				\item Choose some fixed order(specified earlier).
			\end{itemize}
			\begin{eg}
				For example, we have
				\begin{table}[H]
					\centering
					\begin{tabular}{c|c|c|c|c}
						\toprule
						          & A  & B & C & D \\
						\midrule
						\(b_{i}\) & 10 & 2 & 3 & 7 \\
						\bottomrule
					\end{tabular}
				\end{table}
				After sorted:
				\begin{table}[H]
					\centering
					\begin{tabular}{c|c|c|c|c}
						\toprule
						          & A  & D & C & B \\
						\midrule
						\(b_{i}\) & 10 & 7 & 3 & 2 \\
						\bottomrule
					\end{tabular}
				\end{table}
				We then sell the good to \(A\) with price \(7\).
			\end{eg}
		\item[4.] English auction(Ascending price auction):
			\begin{enumerate}
				\item Start from price \(0\)
				\item Keeping increasing price until only one left
				\item Sell the good to the person remaining and at the price at which everyone else drops out.
			\end{enumerate}
			\begin{eg}
				For example, we start with \(0\), and everyone wants the good. Then we slightly increase the price, until:
				\begin{enumerate}
					\item Price is \(2\) or \((2+\epsilon)\):
					      A, D and C want the good, B drops out.
					\item Price is \(3\) or \((3+\epsilon)\):
					      A and D want the good, C drops out too.
					\item Price is \(7\) or \((7+\epsilon)\):
					      A want the good, D drops out too.
				\end{enumerate}
				Then A gets the good with price being \(7\).
			\end{eg}
	\end{enumerate}
\end{eg}

\subsection{Order Statistics}
For all kinds of auctions we mentioned, we see that to analyze auctions, it'll involve some kind of order of bidding. In the content of auctions, we often refer
this as \emph{Order Statistics}.

Let \(x_1, x_2, \ldots , x_n\) be i.i.d. and non-negative, continuous valued random variables. Let
\(F_{X}(\cdot)\) be the CDF, and \(f_{x}(\cdot)\) be the probability density function.
\begin{eg}
	Consider
	\begin{enumerate}
		\item \(X = \mathrm{uniform}([0, 1])\):
		      \[
			      F_X(x) = \begin{dcases}
				      x,         & \text{ if } x\in[0, 1] \\
				      \min(x, 1) & \text{ if }x>1         \\
			      \end{dcases}
		      \]
		      with
		      \[
			      f_X(x) = \begin{dcases}
				      1, & \text{ if } x\in[0, 1] \\
				      0, & \text{ if }x>1.
			      \end{dcases}
		      \]
		\item \(\mathrm{Exp}(\lambda)\):
		      \[
			      F_X(x) = 1 - e^{-\lambda x}, \quad x\geq 0
		      \]
		      with
		      \[
			      f_X(x) = \lambda e^{-\lambda x},\quad x\geq 0.
		      \]
	\end{enumerate}
\end{eg}

\hr

The idea is simple, we first sort \(x_1, x_2, \ldots , x_n\) in decreasing order and denotes
\[
	y_1, y_2, \ldots , y_{n}
\]
such that
\[
	\begin{split}
		y_1 &= \max\{x_1, \ldots , x_n\}\\
		y_2 &= \max\{x_1, \ldots , x_n\}\setminus \{y_1\}\\
		y_3 &= \max\{x_1, \ldots , x_n\}\setminus \{y_2, y_3\}\\
		&\vdots
	\end{split}
\]

Then we have
\[
	\max\{x_1, \ldots , x_n\} = y_1\geq y_2\geq \ldots \geq y_n = \min\{x_1, \ldots , x_{n}\}.
\]
Now, we want to find out \(\expectation{}{y_1} \), which is just
\[
	\expectation{}{y_1} = \int_0^{\infty } y f_{y_1}(y)\,\mathrm{d}y
\]
where \(f_{y_1}\) is the probability density function of \(y_1\). Since \(f_{y_1}(y) = \frac{\mathrm{d}}{\mathrm{d}y} \left(F_{y_1}(y)\right)\),
where
\[
	\begin{split}
		F_{y_1}(y) &= \probability{}{y_1\leq y}\\
		&= \probability{}{\max\{x_1, \ldots , x_{n}\}\leq y}\\
		&= \probability{}{x_1\leq y, x_2\leq y, \ldots , x_n\leq y}\\
		&= \prod\limits_{i=1}^{n} \probability{}{x_{i}\leq y}\\
		&= \prod\limits_{i=1}^{n} F_{x_{i}}(y)\\
		&= \left(F_x(y)\right)^n,
	\end{split}
\]
where we have used the properties like independence and identically distributed(i.i.d.) assumptions.

Further, since we now know
\[
	f_{y_1}(y) = \frac{\mathrm{d}}{\mathrm{d}y}\left(\left(F_x(y)\right)^n\right) = n \left(F_x(y)\right)^{n-1} \frac{\mathrm{d}}{\mathrm{d}y} F_x(y) = n\left(F_{x}(y)\right)^{n-1}f_x(y),
\]
we have
\[
	\expectation{y_1} = \int_0^{\infty }n y \left(F_{x}(y)\right)^{n-1} f_x(y)\,\mathrm{d}y.
\]

\begin{eg}
	We now revisit the examples we just show.
	\begin{enumerate}
		\item Let \(F_{x}(\cdot)\sim \mathrm{uniform}([0,1]) \). Then
		      \[
			      \expectation{y_1} = \int_0^1 n y y^{n-1}\,\mathrm{d}y = n\int_0^1 y^n \,\mathrm{d}y = \at{\frac{ny^{n+1}}{n+1}}{0}{1} = \frac{n}{n+1} .
		      \]
		      Hence, \(\expectation{x_1}= \frac{1}{2} \).
		\item Exercise - Try exponential.
	\end{enumerate}
\end{eg}
\begin{remark}
	For the uniform case, the expected value of the \(k^{th}\) element is exactly
	\[
		\frac{k}{n+1}
	\]
	as one can check.
\end{remark}


\subsection{Game-Form for Auction}
We relate auctions with game as we discussed before. Given the types for player \(i\in \mathcal{I} \) being i.i.d. such that
\[
	t_1, t_2, \ldots , t_I,
\]
and also for the valuations such that
\[
	v_1, v_2, \ldots , v_I.
\]

\begin{note}
	We see that
	\begin{itemize}
		\item For simplicity, without specifying, we assume they are distributed in
		      \[
			      \mathrm{uniform} ([0, 1]).
		      \]
		\item The valuation \(v_{i}\) is taken as the satisfaction in dollar amount from buying the good.
	\end{itemize}
\end{note}

Then, the utility function is determined based on specific forms of auction. We now see some explicit examples.

\subsection{Second-price Auction}
We first analyze second-price auction, since it's easier to analyze. Recall the Interim Setting, namely each agent know its valuation but
only the distribution of the valuation of others. Now, let
\[
	\sigma_{i}\colon [0, 1]\to [0, 1]\qquad v_{i}\mapsto  \sigma_{i}(v_{i})
\]
where \(\sigma_{i}\) is just the strategy of bidding. Then the utility is
\[
	\expectation{-i}{u_{i}(\underbrace{v_{i}, \sigma_{i}(v_{i})}_{\text{known for \(i\)}}, \underbrace{v_{-i}, \sigma_{-i}(v_{-i})}_{\text{random}})}.
\]

The goal is to choose \(\sigma_{i}(\cdot)\) such that expected utility is maximized for each \(v_{i}\)
assuming \(\sigma_{-i}\) is fixed.

Then,
\[
	\begin{split}
		&u_{i}(v_{i}, \sigma_{i}(v_{i}), v_{-i}, \sigma_{-i}(v_{-i}))\\
		=&\left(v_{i} - \max_{j\in I\setminus\{i\}}\sigma_{j}(v_{j})\right)\mathbbm{1}_{\{\text{\(i\) wins the auction}\}}\\
		=&\left(v_{i} - \max_{j\in -i}\sigma_{j}(v_{j})\right)\mathbbm{1}_{\{ \sigma_{i}(v_{i})\geq \sigma_{j}(v_{j}) \forall j\in -i \}}.
	\end{split}
\]

We now aim to maximize
\[
	\expectation{-i}{u_{i}(v_{i}, \sigma_{i}(v_{i}), v_{-i}, \sigma_{-i}(v_{-i}))}.
\]

\subsubsection{Reason of Response}
We need to reason how agents will bid strategically.
\begin{itemize}
	\item Case 1: If \(b_{i} = \sigma_{i}(v_{i})>v_{i}\).
	      \begin{enumerate}
		      \item[(a)] \(b_{i}< \widetilde{P}_i = \max\limits_{j\in -i}\sigma_{j}(v_{j}) = b_{j}\), then we see that
			      \[
				      \max(b_{i}, \widetilde{P}_{i}) = \widetilde{P}_{i},
			      \]
			      so \(i\) never wins the auction.
			      \par Continue to do so even if he reduces bid to \(v_{i}\).
		      \item[(b)] \(b_{i} = \widetilde{P}_{i}\), then \(i\) can win the auction.
			      \par Since \(v_{i} - \widetilde{P}_{i}<0\),  and the second-highest bid is \(\widetilde{P}_{i}\). So He better to bid \(v_{i}\) and
			      lost the auction.
		      \item[(c)] \(b_{i} > \widetilde{P}_{i}\), then \(i\) definitely wins the auction.
			      \begin{itemize}
				      \item \(v_{i}<\widetilde{P}_{i}\): utility is \(v_{i} - \widetilde{P}_{i}<0\), not a viable option.
				      \item \(v_{i}\geq \widetilde{P}_{i}\): utility is \(v_{i} - \widetilde{P}_{i}\geq 0\), but
				            same result would also hold if \(b_{i}\) is exactly \(v_{i}\).
			      \end{itemize}
	      \end{enumerate}
	\item Case 2: If \(b_{i} = \sigma_{i}(v_{i})<v_{i}\).
	      \begin{enumerate}
		      \item[(a)] \(\widetilde{P}_{i}\leq b_{i}<v_{i}\). Agent \(i\) wins with utility being \(v_{i} - \widetilde{P}_{i}\geq 0\) and
			      no lose in increasing bid to \(v_{i}\).
		      \item[(b)] \(b_{i}< v_{i}\leq \widetilde{P}_{i}\). Don't win the auction. No lose in increasing bid to \(v_{i}\).
		      \item[(c)] \(b_{i}\leq \widetilde{P}_{i}\leq v_{i}\). Lose the auction. However, bidding \(v_{i}\) the agent could have won the
			      auction and gotten non-negative utility.
	      \end{enumerate}
	      \begin{remark}
		      Bidding \(v_i\) is a (weakly) dominant strategy for agent \(i\). This further implies that for second-price auction, agents
		      bid their true value. This is essentially \emph{Incentive Compatibility}, or \emph{Truth-telling desirable}.
	      \end{remark}
\end{itemize}

\begin{remark}
	We see that \(2^{nd}\) price auction is truth-telling in dominant strategies. Hence, irrespective of how the other players bid, truthfully bidding is optimal, namely
	\[
		\sigma_{i}(v_{i}) = v_{i}.
	\]
\end{remark}
\subsubsection{Revenue of the Auctioneer}
Given \(v_1, v_2, \ldots , v_I\), defined \(\widetilde{v}_i\) as before, namely
\[
	\max\{v_1, \ldots , v_I\} = \widetilde{v}_{I}\geq \widetilde{v}_{I-1}\geq \ldots \geq \widetilde{v}_1 = \min\{v_1, \ldots , v_I\}.
\]

And since it's a \(2^{nd}\) auction, so the revenue is \(\widetilde{v}_{I - 1}\).

\begin{note}
	The expected revenue of \(\widetilde{v}_{I - 1}\) is
	\[
		\expectation{\widetilde{v}_{I-1}} = \frac{I-1}{I+1}
	\]
	if we assume uniform valuation.
\end{note}

Hence, we see that the winner gets \(\widetilde{v}_{I} - \widetilde{v}_{I-1}\), and the expected utility is
\[
	\expectation{\widetilde{v}_{I}} - \expectation{\widetilde{v}_{I - 1}}.
\]
In the uniform case, the above further equals to
\[
	\frac{I}{I+1} - \frac{I - 1}{I + 1} = \frac{1}{I+1}.
\]

With winner and auctioneer together, their utility is
\[
	\widetilde{v}_{I} - \widetilde{v}_{I - 1}+\widetilde{v}_{I - 1} = \widetilde{v}_{I}.
\]

\begin{remark}
	We see that the good always goes to the highest valuation.
\end{remark}

\hr

Sums of utilities of all players in auctioneer is Social welfare - max value possible of \(\widetilde{v}_{I}\).

\begin{remark}
	\(2^{nd}\) price auction is social-welfare maximizing.
\end{remark}

We see that if agent \(i\) bids \(v_{i}\) and wins, then he gets
\[
	v_{i} = \max_{j\in -i} v_{j}\geq 0,
\]
and if lose then get \(0\). Hence, the expected utility is always non-negative if you attend the auction, while staying out will get you \(0\), hence
we see that it's better to participate in the auction. This phenomenon is called \emph{voluntary participation}(Individual rationality, IR).

And since the probability of winning is \(\frac{1}{I}\), hence the expected utility is
\[
	\frac{1}{I}\times \frac{1}{I+1} = \frac{1}{I(I + 1)}
\]
for every agent.

\begin{note}
	Even if we are using Ex Post setting, it's still optimal to bid truthfully in a \(2^{nd}\) auction.
\end{note}

\subsection{First-price Auction}
We'll see that the first price auction is not as "simple". There are no dominant strategy equilibrium, only so-called
Bayes-Nash Equilibrium in interim.

\hr

Consider \(I = 2\), and \(v_{i}\) choose \(b_{i} = \sigma_{i}(v_{i})\). The utility is
\[
	u_{i}(v_{i}, b_{i}, v_{-i}, b_{-i}) = (v_{i} - \max_{j\in \mathcal{I} }b_{i})\mathbbm{1}_{\{ i \text{ is the winner} \}} = (v_{i} - b_{i})\mathbbm{1}_{\{ b_{i}\geq b_{-i} \}}
\]

If \(I>2\), then the utility is now
\[
	(v_{i} - b_{i})\mathbbm{1}_{\{ \sigma_{i}(v_{i})\geq \sigma_{j}(v_{j})\forall j\in -i \}} = (v_{i} - b_{i})\prod\limits_{j\in -i} \mathbbm{1}_{\{ \sigma_{i}(v_{i})\geq \sigma_{j}(v_{j}) \}}
\]
Then, since we're in the interim set-up, the expected value of the utility is
\[
	\begin{split}
		&\expectation{-i}{u_{i}(v_{i}, \sigma_{i}(v_{i}), v_{-i}, \sigma_{-i}(v_{-i}))} \\
		= &\expectation{-i}{(v_{i} - \sigma_{i}(v_{i})) \prod\limits_{j\in -i}\mathbbm{1}_{\{ \sigma_{i}(v_{i})\geq \sigma_{j}(v_{j}) \}}}\\
		= &(v_{i} - \sigma_{i}(v_{i}))\prod\limits_{j\in -i}\probability{}{\sigma_{j}(v_{j})\leq \sigma_{i}(v_{i})},
	\end{split}
\]
where the last equality comes from independence.

\begin{note}
	There are two facts to note.
	\begin{itemize}
		\item Fact 1: Identical valuation implies \(\sigma_{i} \equiv \sigma\) for all \(i\in \mathcal{I} \). Same bidding function so it's symmetric.
		\item Fact 2: \(\sigma(\cdot)\) is monotonically increasing. Then
		      \[
			      \probability{}{\sigma_{j}(v_{j})\leq \sigma_{i}(v_{i})} = \probability{}{\sigma(v_{i})\leq \sigma(v_{i})} = \probability{}{v_{j}\leq v_{i}} = F(v_{i}) = v_{i},
		      \]
		      where \(F\) is the CDF, and the last equality comes from the fact that we assume the valuation is \(\mathrm{uniform}([0, 1])\).

		      This is a fairly reasonable assumption, since it'll rule out the situations like if two buyers' true value are different, but they somehow decide to bid
		      the same value.
	\end{itemize}
\end{note}

With the second fact of the note above, the expected utility is
\[
	(v_{i} - \sigma(v_{i})) F^{I-1}(v_{i}).
\]
If we assume that this is uniform valuation, then above further equals to
\[
	(v_{i} - \sigma(v_{i}))v^{I-1}_{i}.
\]