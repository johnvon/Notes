\lecture{24}{06 Dec. 12:30}{Matching Market}
\subsection{Bayesian Nash Equilibrium}
\begin{prev}
	For a valuation \(\vec{v}\), the bid \(\sigma^{*}(\vec{v})\) is in the same space as \(\vec{v}\) by \hyperref[Revelation-Principle]{revelation principle}.
	With allocation being specified as
	\[
		\vec{x}_{i}(v_{i})
	\]
	which is monotonically increasing, and the payment being specified as
	\[
		\overline{p}_{i}(v_{i}) = v_{i}\overline{x}_{i}(v_{i}) - \int_0^{v_{i}}\overline{x}_{i}(z)\,\mathrm{d}z + \overline{p}_{i}(0),
	\]
	we now try to analyze a general auction.
\end{prev}

\subsubsection{Revenue Equivalence}
We start with the expected revenue. The expected revenue is
\[
	\expectation{v}{\sum\limits_{i}\overline{p}_{i}(v_{i}) } = \expectation{v}{F(\vec{v})} + \sum\limits_{i=1}^{I} \overline{p}_{i}(v).
\]
\begin{enumerate}
	\item If two mechanisms have the same allocation rule in equilibrium, then
	      \begin{itemize}
		      \item \(\overline{x}_{i}(v_{i})\) will be the same
		      \item \(\overline{p}_{i}(v_{i})\) will be the same except for the \(\overline{p}_{i}(0)\) term.
	      \end{itemize}
	\item If agent with valuation zero, pay zero, then
	      \[
		      \overline{p}_{i}(0) = 0
	      \]
	      in both mechanisms. This implies that the revenue will be the same.
\end{enumerate}

\begin{note}
	Similar characterization holds for dominant strategy equilibrium but those ex-post.
	\begin{itemize}
		\item Allocation rule property
		\item Payment rule property
	\end{itemize}
\end{note}

Assume the independent and identical valuation setting, together with the valuation being \(\mathrm{uniform}([0,1])\). Then the \(2^{nd}\) price auction
and \(1^{st}\) price auction are \emph{revenue equivalent}.

By symmetry, the payment rules will be identical in each auction, hence the revenue is
\[
	I\times \text{payment rule for each agent}.
\]

\begin{itemize}
	\item For the second price auction(SP),
	      \[
		      \overline{p}_1^{\mathrm{SP}}(v_1) = \expectation{}{\left(\max_{j\in -1}v_{j}\right)\mathbbm{1}_{\{ \max\limits_{j\in -1} v_{j}\leq v_1\}} }.
	      \]
	      Denote \(\widetilde{p}_1 = \max\limits_{j\in -1} v_{j}\), then
	      \[
		      F_{\widetilde{p}^{\mathrm{SP}}_i}(x) = \probability{}{\widetilde{p}_1\leq x} = \probability{}{v_2\leq x, v_3\leq x, \ldots , v_I\leq x} = x^{I-1},
	      \]
	      where \(x\in [0, 1]\). Then,
	      \[
		      f_{\widetilde{p}^{\mathrm{SP}}}(x) = (I - 1)x^{I-2}.
	      \]
	      We see that
	      \[
		      \begin{split}
			      \widetilde{p}_1^{\mathrm{SP}} &= \expectation{}{\widetilde{p}_{1}\mathbbm{1}_{\{ \widetilde{p}_1\leq v_1 \}} }\\
			      &=\int_0^{v_1} xf_{\widetilde{p}_1}(x)\,\mathrm{d}x \\
			      &= (I - 1)\int_0^{v_1}x\cdot x^{I-2}\,\mathrm{d}x \\
			      &= (I - 1)\int_0^{v_1}x^{I-1}\,\mathrm{d}x\\
			      & = \frac{I-1}{I}v_1^I.
		      \end{split}
	      \]
	\item For the first price auction(FP),
	      \[
		      \overline{p}_1^{\mathrm{FP}}(v_1) = \sigma^{\mathrm{FP}}(v_1)\probability{}{\text{Agent 1 has the highest value \(v_1\)}} = \sigma^{\mathrm{FP}}(v_1)v_1^{I-1},
	      \]
	      where every other agent's value is less or equal to \(v_1\).

	      From Revenue equivalence,
	      \[
		      \widetilde{p}^{\mathrm{SP}}(v_1) = \widetilde{p}^{\mathrm{FP}}(v_1),
	      \]
	      which implies
	      \[
		      \frac{I-1}{I}v_{1}^I = \sigma^{\mathrm{FP}}(v_1)v_1^{I-1} \implies \sigma^{\mathrm{FP}} = \frac{I-1}{I}v_1,
	      \]
	      as we discussed before.
\end{itemize}

\section{Matching Market}
Beyond the current setup of auctions, we can see a more general setup. Specifically, we introduce so-called \textbf{Matching Market}. The comparison is as follows.
\begin{itemize}
	\item One good to many buyers: Auctions.
	\item Many goods to many buyers: Matching Market.
\end{itemize}

\begin{eg}
	We first see an example of matching market in terms of a general form of auction, namely \textbf{Ad-Auctions}. Specifically, consider a particular keyword searching result,
	the ads related to which may show up in the following order:
	\begin{itemize}
		\item AD1 with \(r_1\)
		\item AD2 with \(r_2\)
		\item AD3 with \(r_3\)
		\item AD4 with \(r_4\)
		\item \(\vdots\)
	\end{itemize}
	where \(r_i\) is the \emph{click-through rate} for ad location \(i\). We see that the location of ads matter.

	\begin{note}
		We see that
		\begin{itemize}
			\item Click-through rate is location dependent. Person needs to click on ad and then buy the product the ad is about.
			\item Often, \(r_i < r_j\) for \(i > j\).
		\end{itemize}
	\end{note}
\end{eg}

\hr

We see that there is a \emph{bipartite structure}. On the one side of the graph contains buyers, and the other side contains sellers(Ad positions).
Now, assuming for each buyer's good, the cost is \(b_i\). Then the expected revenue to advertiser \(i\) is
\[
	b_i r_j
\] if one purchase the location \(j\). Now, let \(v_{ij} \coloneqq b_{i}r_{i}\), where \(i\) denotes the buyers, and \(j\) denotes the sellers.
Then we have the so-called \emph{Valuation matrix} \(V\) such that
\[
	V \coloneqq  \vec{r}^{T}\vec{b},
\]
which is also called the \textbf{rank one valuation matrix}.

If \(V_{ij}\geq 0\) and beyond rank-one imply good allocation here.

We allow the sellers to set prices to goods.
\begin{remark}
	There are different types of prices.
	\begin{itemize}
		\item Posted. Item comes with price.
		\item Anonymous. The price is the same for any buyers.
	\end{itemize}
\end{remark}

\begin{definition}
	The market is called \emph{balance} if the number of sellers and buyers are the same.
\end{definition}

Without loss of generality, we assume that the market is \textbf{balanced} by letting
\[
	V\in \mathbb{\MakeUppercase{R}}^{N\times N},
\]
which implies there are a same number(\(N\)) of buyers and sellers. This can be achieved by adding dummy buyers or sellers.
\begin{itemize}
	\item Dummy Buyers: Value every good as \(0\).
	\item Dummy Sellers: Valued as \(0\) for every buyer.
\end{itemize}

Consider each seller/item coming with a posted and anonymous price, namely item \(j\) has price \(p_j\geq 0\).
\begin{problem}
How does a buyer \(i\) react to this? Assuming \(i\) knows her/his valuation, and each buyer can only purchase one good.
\end{problem}
\begin{answer}
	Suppose the valuation for buyer \(i\) is
	\[
		V_{i1}, V_{i2}, \ldots , V_{iN},
	\]
	and the prices for each good is
	\[
		p_{1}, p_2, \ldots , p_{N}.
	\]
	Then it's clear that buyer \(i\) will
	\begin{itemize}
		\item only consider item whose valuation exceeds the price
		\item further, focus only on goods whose payoff is maximum(\(\text{valuation}-\text{price}\))
	\end{itemize}
	Mathematically,
	\[
		\underset{j \text{ s.t. }V_{ij}\geq p_{j}}{\arg\max} V_{ij} - p_{j}.
	\]

	For convenience, we define a function \(\left[\ \cdot\ \right]_+\) such that
	\[
		\left[ x \right]_+ \coloneqq \max(x, 0) = \begin{dcases}
			x, & \text{ if } x\geq 0 \\
			0, & \text{ if } x\leq 0 \\
		\end{dcases}
	\]

	The utility of buyer \(i\) at price vector \(\vec{p}\) is
	\[
		u_{i} = \left[\max_{j\in\{1, \ldots , N\}} V_{ij} - p_{j}\right]_+ = \max_{j\in \{1, \ldots , N\}}\left[V_{ij}-p_{j}\right]_+.
	\]

	We then defined so-called \emph{preferred sellers list} \(s(i)\) for buyer \(i\), which contains goods that buyer \(i\) is interested in.
	\[
		s(i) = \begin{dcases}
			\varnothing ,                           & \text{ if }  \forall j\ V_{ij}<p_{j} \\
			\arg\max_j \left(V_{ij} - p_{j}\right), & \text{ otherwise}.
		\end{dcases}
	\]

	\begin{remark}
		If \(s(i)\neq \varnothing \), then there is at last one seller that buyer \(i\) is interested in.
	\end{remark}

	We then use the preferred sellers lists to construct an undirected bipartite graph between buyers and sellers.
\end{answer}

\begin{eg}
	We see some examples.
	\begin{enumerate}
		\item Let \(N = 3\), and
		      \[
			      V\coloneqq \begin{pmatrix}
				      5 & 3 & 4 \\
				      3 & 8 & 3 \\
				      5 & 4 & 7 \\
			      \end{pmatrix}, \quad p \coloneqq \begin{pmatrix}
				      4 \\
				      3 \\
				      7 \\
			      \end{pmatrix}.
		      \]
		      Then, we have
		      \begin{enumerate}
			      \item Buyer 1. \(V_{11}-p_1 = 1\), \(V_{12} - p_2 = 0\), \(V_{13} - p_3 = -3\). Hence,
			            \[
				            s(1) = \left\{1\right\}.
			            \]
			      \item Buyer 2. \(V_{21}-p_1 = -1\), \(V_{22} - p_2 = 5\), \(V_{23} - p_3 = -4\). Hence,
			            \[
				            s(2) = \left\{2\right\}.
			            \]
			      \item Buyer 3. \(V_{31}-p_1 = 1\), \(V_{32} - p_2 = 1\), \(V_{33} - p_3 = 0\). Hence,
			            \[
				            s(3) = \left\{1, 2\right\}.
			            \]
		      \end{enumerate}
		      We see that \(c\) is not desired by any buyers at current price.
		      \begin{figure}[H]
			      \centering
			      \incfig{satisfaction-eg-1}
			      \label{fig:satisfaction-eg-1}
		      \end{figure}
		\item Consider changing \(p\) into
		      \[
			      p = \begin{pmatrix}
				      0 \\
				      0 \\
				      0 \\
			      \end{pmatrix}.
		      \]
		      Then we see that satisfaction is possible.
		      \begin{figure}[H]
			      \centering
			      \incfig{satisfaction-eg-2}
			      \label{fig:satisfaction-eg-2}
		      \end{figure}
		\item Consider changing \(p\) into
		      \[
			      p = \begin{pmatrix}
				      5 \\
				      8 \\
				      7 \\
			      \end{pmatrix}.
		      \]
		      Then we see that satisfaction is possible as well.
		      \begin{figure}[H]
			      \centering
			      \incfig{satisfaction-eg-3}
			      \label{fig:satisfaction-eg-3}
		      \end{figure}
	\end{enumerate}
\end{eg}

\begin{remark}
	We see that the satisfaction match is not always possible.
\end{remark}

Now, the task is that for any given price, find the preferred sellers bipartite graph and perform matching. Furthermore, we try to
maximize the matching in terms of size. In other words, we are trying to satisfy as many buyers as possible.

\begin{note}
	We call a matching is \emph{perfect} if the size of the maximum matching is \(N\).
\end{note}

\subsection{Perfect Matching}
To discuss perfect matching, we first defined some notations. Let
\[
	S\subseteq \{1, \ldots , N\}
\] be the subset of sellers, and
\[
	B\subseteq\{1, \ldots , N\}
\]
be the subset of buyers. We further denote \(N(S)\) be the union of all buyers that like goods in \(S\), and \(N(B)\) be the union
of preferred sellers of all the buyers in \(B\).

If \(\left\vert S \right\vert > \left\vert N(S) \right\vert \), then we call \(S\) is a \emph{constricted set}. Similarly, if
\(\left\vert B \right\vert > \left\vert N(B) \right\vert \) then \(B\) is constricted.

\begin{theorem}
	Kőnig-Hall Maximize theorem. A bipartite graph has a perfect matching if and only if there are no constricted sets.
\end{theorem}
\begin{note}
	This is a necessary and sufficient conditions for a perfect matching to exist given a bipartite graph.
\end{note}
\begin{remark}
	We see that
	\begin{itemize}
		\item With naive approach, we need to check every non-empty subset of sellers. Hence, there are \(2^N - 1\) checks.
		\item Polynomial time checks exists.
		\item We can also find a perfect matching in polynomial time.
		\item Prices for which a perfect matching exists are called \emph{market clearing prices}.
	\end{itemize}
\end{remark}