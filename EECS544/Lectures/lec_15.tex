\lecture{15}{27 Oct. 12:30}{Random Graph}
\begin{prev}
	Main idea is that to construct a Markov Chain on nodes of the graph. At none \(i\), the random walker toss a biased coin and
	determine where to go.
\end{prev}

\subsubsection{\(4^{th}\) Estimator for \(\pi_V(s)\)}
Although the \(3^{rd}\) estimator for \(\pi_V(s)\) is better than the other two, but it still has its flaws. The main thing is, it throws away so much
data, which means it will converge slower. We introduce a final estimator, which make the full use of the data given.

We first introduce Renewal Theorem.\footnote{\url{https://en.wikipedia.org/wiki/Renewal_theory}}
\begin{theorem}
	Renewal Theorem.
	\[
		\frac{\expectation{\# \text{ of visits to node \(k\) in this reset}} }{\underbrace{\expectation{\text{Duration of reset}}}_{= \frac{1}{1-s}}} = \pi_k(s).
	\]

	Noting that each reset duration is independent of the other \(T\) rest duration starting at uniformly chosen node. Then we have
	\[
		\hat{\pi}_k(s) = \frac{\frac{1}{T}\sum\limits_{i=1}^{T} N_t(k)}{1/1-s} = \frac{1-s}{T}\sum\limits_{t=1}^{T} N_t(k)
	\]
	where \(N_t(k)\) is the number of time we visited \(k\) in reset duration.
\end{theorem}

\begin{remark}
	We see that
	\begin{itemize}
		\item This is also a non-biased estimator.
		\item \(N_t(k)\) is i.i.d.
		\item This estimator does not throw out those useful data, hence the estimator computed by this algorithm converges faster.
	\end{itemize}
\end{remark}

\begin{figure}[H]
	\centering
	\incfig{Monte-Carlo-Estimator-4}
	\caption{Forth Monte Carlo Estimator}
	\label{fig:Monte-Carlo-Estimator-4}
\end{figure}

\section{Random Graph}
We are interested in some functions depend on the number of nodes \(n\) in the graph such that
\[
	\lim_{n\to \infty }f(n),
\]
namely we are interested in so-called \emph{large graph property}.
\subsection{Order Relationship between Functions}
We first introduce some common notations for asymptomatic comparison.
\begin{definition}
	\(f(n)\) is said to be \(O(g(n))\) if there is a positive \(k\) such that
	\[
		f(n)\leq k\cdot g(n)
	\]
	for all large enough \(n\).
\end{definition}

\begin{definition}
	\(f(n)\) is said to be \(\Omega(g(n))\) if there is a positive \(k\) such that
	\[
		f(n) \geq k\cdot g(n)
	\]
	for all large enough \(n\).
\end{definition}

\begin{definition}
	\(f(n)\) is said to be \(\Theta(g(n))\) if there are positive \(k_1\leq k_2\) such that
	\[
		k_1\cdot g(n)\leq f(n) \leq k_2\cdot g(n)
	\]
	for all large enough \(n\).
\end{definition}

\begin{definition}
	\(f(n)\) is said to be \(o(g(n))\) if for every positive \(\epsilon\), there exists a \(N\) large enough such that
	\[
		f(n) \geq \epsilon\cdot g(n)\ \forall n\geq N.
	\]
\end{definition}

\begin{definition}
	\(f(n)\) is said to be \(\omega(g(n))\) if for every positive \(\epsilon\), there exists a \(N\) large enough such that
	\[
		f(n) \leq \epsilon\cdot g(n)\ \forall n\geq N.
	\]
\end{definition}

\begin{definition}
	\(f(n)\) is said to be of the same order as \(g(n)\) asymptotically if
	\[
		\lim_{n\to \infty } \frac{f(n)}{g(n)} = 1.
	\]
	We can write \(f(n)\sim g(n)\) for short in this case.
\end{definition}

\begin{remark}
	If \(f(n)\) or \(g(n)\) are not non-negative, we simply use \(\left\vert f(n) \right\vert \) and \(\left\vert g(n) \right\vert \)
	and apply the definition.
\end{remark}

We consider the size of the largest connected component, and see how does this relation behave as \(n\to \infty\).
\begin{eg}
	We see some example for a function \(f(n)\) which represent the size of a graph, where \(n\) is the number of nodes in the graph.
	\begin{itemize}
		\item \(f(n) = \log n\): Islands with \(\frac{n}{\log(n)}\) of them. When \(n\to \infty \), \(\frac{n}{\log(n)}\to 0\).
		      \begin{figure}[H]
			      \centering
			      \incfig{random-graph-logn}
			      \label{fig:random-graph-logn}
		      \end{figure}
		\item \(f(n) = n^c\) with \(c<1\): We see that \(\frac{n^c}{n}\to 0\) for \(c<1\) as \(n\to \infty \). Better than island but not that great.
		      \begin{figure}[H]
			      \centering
			      \incfig{random-graph-n^c}
			      \label{fig:random-graph-n^c}
		      \end{figure}
		\item \(f(n) = cn\): There is a giant component roughly \(c\) percent of nodes in the graph. We see that as \(n\to \infty \), \(\frac{cn}{n}\to c\).
		      \begin{figure}[H]
			      \centering
			      \incfig{random-graph-cn}
			      \label{fig:random-graph-cn}
		      \end{figure}
	\end{itemize}
\end{eg}
