\lecture{22}{25 Feb.\ 10:00}{Calculation of Homology}
\subsection{Calculation of Homology}
We start from some calculation about \hyperref[def:homology-group]{homology group} of some spaces.
\begin{eg}[Homology group of \(\mathbb{R} P^{2}\)]
	Calculate the \hyperref[def:homology-group]{homology group} by \autoref{def:homology-group} with the \hyperref[def:chain-complex]{chain complex}
	being the \hyperref[def:simplicial-complex]{simplicial chain complex}.
\end{eg}
\begin{explanation}
	Let \(X = \mathbb{R} P^{2} \).
	\begin{figure}[H]
		\centering
		\incfig{eg:homology-RP2}
		\label{fig:eg:homology-RP2}
	\end{figure}
	We see that we have
	\begin{itemize}
		\item \(C_0 = \mathbb{Z} \left< v, w \right> \)
		\item \(C_1 = \mathbb{Z} \left< a, b, c \right> \)
		\item \(C_2 = \mathbb{Z} \left< A, B \right> = \mathbb{Z} A \oplus \mathbb{Z} B\)
	\end{itemize}
	The \hyperref[def:chain-complex]{chain complex} is then
	\[
		\begin{tikzcd}
			0 & {C_2} & {C_1} & {C_0} & 0
			\arrow["{\partial_3}", from=1-1, to=1-2]
			\arrow["{\partial_2}", from=1-2, to=1-3]
			\arrow["{\partial_1}", from=1-3, to=1-4]
			\arrow["{\partial_0}", from=1-4, to=1-5]
		\end{tikzcd}
	\]
	Where we let \(A = [v_0, v_1, v_2]\) and \(B = [u_0, u_1, u_2]\), then
	\[
		\partial_2\colon \begin{dcases}
			A & \mapsto b - c + a \\
			B & \mapsto -a-c-b    \\
		\end{dcases},\qquad \partial_1\colon \begin{dcases}
			a & \mapsto w - v     \\
			b & \mapsto v - w     \\
			c & \mapsto v - v = 0 \\
		\end{dcases}
	\]
	We can also calculate the image and the kernel at \(C_{i} \), i.e.,
	\[
		\begin{alignedat}{3}
			C_2&\colon \im \partial _3 = 0, &\ker \partial _2 &= 0, \\
			C_1&\colon \im \partial _2 = \left< 2c, b - c + a \right> , \qquad&\ker \partial _1 &= \left< b +a, c \right>,\\
			C_0&\colon \im \partial _1 = \left< v - w\right> , &\ker \partial _0 &= \left< v, w\right>.
		\end{alignedat}
	\]
	Hence,
	\[
		\begin{split}
			H_0 &\cong \quotient{\mathbb{Z} \left< v, w \right> }{\mathbb{Z} \left< v - w \right> } \cong \mathbb{Z}\\
			H_1 &\cong \quotient{\mathbb{Z} \left< b + a, c \right> }{\mathbb{Z} \left< 2c, b+a-c \right> } \cong \quotient{\mathbb{Z} \left< b + a - c, c \right> }{\mathbb{Z} \left< 2c, b+a-c \right> } \cong \quotient{\mathbb{Z} }{2\mathbb{Z}}\\
			H_2 &= 0\\
		\end{split}
	\]
\end{explanation}

\begin{remark}
	Given a basis for a \hyperref[def:free-Abelian-group]{free Abelian group} \(\left<  b_1, \dots, b_n  \right>\) we can replace \(b_{i} \) with
	\[
		b_i \pm m_1b_1 \pm \cdots \pm \widehat{m_ib_i} \pm \cdots \pm m_n b_n.
	\]
\end{remark}

\begin{remark}
	Care is needed when doing \emph{change of bases} over \(\mathbb{Z} \). For example,
	if \(b_1, b_2\) is a basis for \(A \subseteq \mathbb{Z} ^n\), then \(b_1 - b_2, b_1 + b_2\) is \textbf{not} a basis, it is an index-2 subgroup.
	The key to this is that \(\begin{bmatrix} 1 & 1 \\ 1 & -1 \end{bmatrix}\) has determinant \(-2\) (\textbf{not} unit in \(\mathbb{Z}\)).
\end{remark}

We can transform a basis for a \hyperref[def:free-group]{free group} into a different basis by applying a matrix of determinant \(\pm 1\).
If we apply a matrix of determinant \(D\) we will obtain generators for a subgroup of index \(\left\vert D \right\vert \).
\[
	\begin{bmatrix}
		1      & 0      & \cdots   & 0       & \pm m_1       & 0      & \cdots & 0      \\
		0      & 1      & \cdots 0 & \pm m_2 & 0             & \cdots & 0               \\
		\vdots & \vdots & \ddots   & \vdots  & \vdots        & \vdots & \ddots & \vdots \\
		0      & 0      & \cdots   & 1       & \pm m_{i - 1} & 0      & \cdots & 0      \\
		0      & 0      & \cdots   & 0       & 1             & 0      & \cdots & 0      \\
		0      & 0      & \cdots   & 0       & \pm m_{i + 1} & 1      & \cdots & 0      \\
		\vdots & \vdots & \ddots   & \vdots  & \vdots        & \vdots & \ddots & \vdots \\
		0      & 0      & \cdots   & 0       & \pm m_n       & 0      & \cdots & 1
	\end{bmatrix}
\]

As a summary, we have the following procedures to compute \(H_{n} (X)\).
\begin{enumerate}[(1)]
	\item Choose \hyperref[def:delta-complex]{\(\Delta\)-complex} structure on \(X\). (We will prove \(H_\ast(X)\) is independent of the choice of \hyperref[def:delta-complex]{\(\Delta\)-complex} structure)
	\item Choose orientations on each \hyperref[def:standard-simplex]{simplex} (Any choice is okay, but you must commit to a choice, or you will make a sign error!)
	\item For each \hyperref[def:standard-simplex]{\(n\)-simplex} \(\sigma\) compute \(\partial_n(\sigma)\) (careful with signs!)
	\item \(\im \partial_n = \langle \partial_n(\sigma) \mid \sigma \text{ an \hyperref[def:standard-simplex]{\(n\)-simplex}} \rangle\). Use linear algebra to compute \(\ker(\partial_n)\).
	\item For each \(n\) compute \(H_n(X)= \quotient{\ker \partial_n}{\im \partial_{n + 1}}\). Be careful that any change-of-variables map you apply is invertible over \(\mathbb{Z}\).
\end{enumerate}