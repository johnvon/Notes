\lecture{20}{21 Feb.\ 10:00}{Simplicial Complex}
We now give some definitions about \hyperref[def:standard-simplex]{standard simplex}.

\begin{definition*}
	With \autoref{def:standard-simplex}, we have the followings.
	\begin{definition}[Subsimplex]\label{def:subsimplex}
		Combinatorially, a \emph{subsimplex} of a \hyperref[def:standard-simplex]{simplex} \(\Delta^n\) is a subset of the vertices; while topologically, it's the convex hull of the subset of vertices.
		\begin{figure}[H]
			\centering
			\incfig{def:subsimplex}
			\label{fig:def:subsimplex}
		\end{figure}
	\end{definition}

	\begin{definition}[Face]\label{def:face}
		A \emph{face} of a \hyperref[def:standard-simplex]{simplex} \(\Delta ^n\) is a \hyperref[def:subsimplex]{subsimplex} of \(1\) dimensional lower
		than \(\Delta ^n\) (\underline{codimension \(1\)}).
	\end{definition}

	\begin{definition}[Boundary]\label{def:boundary}
		The \emph{boundary} \(\partial \sigma \) of a \hyperref[def:standard-simplex]{simplex} \(\sigma \) is the union of its \hyperref[def:face]{faces}.
	\end{definition}

	\begin{definition}[Open simplex]\label{def:open-simplex}
		The \emph{open simplex} of \(\Delta \) is defined as
		\[
			\mathring{\Delta}^n \coloneqq \Delta^n - \partial \Delta^n.
		\]
	\end{definition}
\end{definition*}

\begin{definition}[\(\Delta \)-Complex]\label{def:delta-complex}
	A \emph{\(\Delta \)-complex} structure on \(X\) is a collection of maps
	\[
		\sigma _\alpha \colon \Delta ^n\to X
	\]
	such that
	\begin{enumerate}[(a)]
		\item \(\at{\sigma _\alpha }{\mathring{\Delta}^n}{}\) injective, each point of \(X\) is in the image of exactly one such map.
		\item Each restriction of \(\sigma _\alpha \) to a \hyperref[def:face]{face} coincides with a map
		      \[
			      \sigma _\beta \colon \Delta^{n-1} \to X.
		      \]
		\item A set \(A\subseteq X\) is open if and only if \(\sigma ^{-1} _\alpha (A)\) is open in \(\mathring{\Delta }^n\) for all \(\sigma _\alpha \), i.e., \(X\) is a \hyperref[CW-complex-quotient]{quotient}
		      \[
			      \begin{tikzcd}
				      {\coprod\limits_{n, \alpha }\Delta ^n_\alpha} & {X.}
				      \arrow["{\coprod \sigma _\alpha}", from=1-1, to=1-2]
			      \end{tikzcd}
		      \]
	\end{enumerate}
\end{definition}

\begin{exercise}
	A \hyperref[def:delta-complex]{\(\Delta\)-complex } \(X\) is a \hyperref[def:CW-Complex]{CW complex} \(W\) with characteristic maps \(\sigma _\alpha \) with extra constraints on the \hyperref[def:attaching-map]{attaching maps}.
\end{exercise}

\begin{note}
	We see that the second condition of \autoref{def:delta-complex} implies that \hyperref[def:attaching-map]{attaching maps} injective on interior of \hyperref[def:face]{faces}.
\end{note}

\begin{definition}[Simplicial complex]\label{def:simplicial-complex}
	A \emph{simplicial complex} is a \hyperref[def:delta-complex]{\(\Delta \)-complex} such that
	\begin{itemize}
		\item \(\sigma _\alpha \) must map every \hyperref[def:face]{face} to a \underline{different} \hyperref[def:standard-simplex]{\((n-1)\)-simplex}.
		\item Every \hyperref[def:standard-simplex]{simplex} is uniquely determined by its vertex set.
		\item Any \((n+1)\) vertices in \(X^0\) is the vertex set of at most \(1\) \hyperref[def:standard-simplex]{\(n\)-simplex}.
	\end{itemize}
\end{definition}

\begin{eg}[Difference between simplicial and \(\Delta \)-complex structure of \(S^1\)]
	With \autoref{def:delta-complex} and \autoref{def:simplicial-complex}, we see the followings.
	\begin{figure}[H]
		\centering
		\incfig{rmk:simplicial-complex}
		\label{fig:rmk:simplicial-complex}
	\end{figure}
\end{eg}

\begin{eg}[Difference between simplicial and \(\Delta \)-complex structure of a torus]
	The torus with the following edges, \(a, b, c\) and the gluing in triangles \(A\) and \(B\) can be seen as follows.
	\begin{figure}[H]
		\centering
		\incfig{eg:constructing-torus-simplicial}
		\label{fig:eg:constructing-torus-simplicial}
	\end{figure}
	This structure is only valid as a \hyperref[def:delta-complex]{\(\Delta\)-complex}.
\end{eg}
\begin{explanation}
	For this \hyperref[def:delta-complex]{\(\Delta \)-complex}, notice that we've glued down a triangle whose vertices are all identified. This is not allowed in a \hyperref[def:simplicial-complex]{simplicial complex}/triangulation.

	\begin{remark}
		The minimum number of triangles in a \hyperref[def:simplicial-complex]{simplicial complex} structure is \(\bm{14}\).
		\begin{exercise}
			Try to come up with the corresponding \hyperref[def:simplicial-complex]{simplicial complex}.
		\end{exercise}
	\end{remark}
\end{explanation}