\lecture{12}{2 Feb. 10:00}{Presentations for Fundamental Group of CW Complexes}
Let's first see an exercise.
\begin{exercise}
	Consider \(G_1 = \left< S_1 \mid R_1 \right> \) and \(G_2 = \left< S_2 \mid R_2 \right> \). Then
	\begin{itemize}
		\item \(G_1\ast G_2 = \left< S_{1}\cup S_2  \mid R_1 \cup R_2 \right> \)
		\item \(G_1\oplus G_2 = \left< S_1 \cup S_2  \mid R_1 \cup R_2\cup \left\{[g_1, g_2] \mid g_1\in G_1, g_2\in G_2\right\} \right> \)
		\item \(G_1 \ast_H G_2 \) where \(f_1\colon H\to G_1\) and \(f_2\colon H\to G_2\). Then we have
		      \[
			      G_1\ast_H G_2 = \left< S_1 \cup S_2  \mid R_1 \cup  R_2\cup \left\{f_1(h)f_2(h)^{-1}  \mid h\in H\right\} \right>.
		      \]
	\end{itemize}
\end{exercise}

\subsection{Presentations for Fundamental Group of CW Complexes}
For \(X\) a \hyperref[def:CW-Complex]{CW complex}, we have
\begin{enumerate}[(1)]
	\item A \(1\)-dimensional \hyperref[def:CW-Complex]{CW complex} has \hyperref[def:free-group]{free} \(\pi _1\) (call its generators as \(a_1, \ldots , a_n \)).
	\item Gluing a \(2\)-disk by its boundary along a \hyperref[def:word]{word} \(w\) in the generators \emph{kills} \(w\) in \(\pi _1\). We then get a
	      \hyperref[def:group-presentation]{presentation} for \(\pi _1(X^2)\) given by
	      \[
		      \left< a_1, \ldots , a_n \mid w \text{ for each \hyperref[def:cell]{\(2\)-cell} in \(X_2\)}\right>.
	      \]
	\item Gluing in any higher dimensional \hyperref[def:cell]{cells} along their boundary will not change \(\pi _1\). That is, in a \hyperref[def:CW-Complex]{CW complex},
	      we have \(\pi _1(X) = \pi _1(X^2)\).
\end{enumerate}

\begin{remark}
	We can write the above more precise.
	\begin{enumerate}[(1)]
		\item Find \hyperref[def:free-group]{free} generators \(\{a _i\}_{i\in I}\) for \(\pi _1(X^1)\).
		\item For each \(2\)-disk \(D^2_\alpha \), write \hyperref[def:attaching-map]{attaching map} as \hyperref[def:word]{word} \(w_\alpha \) in \(a_{i}\). i.e.,
		      \(\pi _1(X^2) = \left< a_{i} \mid w_\alpha  \right>\).
		\item \(\pi_1(X) = \pi _1(X^2)\).
	\end{enumerate}
\end{remark}

\begin{remark}
	Every group is \(\pi _1\) of some space. Specifically, given a group \(G\), we work with its \hyperref[def:group-presentation]{presentation} \(\left< S\mid R \right> \).
\end{remark}
\begin{explanation}
	We first see a simple example to grab some intuition.
	\begin{eg}[Fundamental group as \(\quotient{\mathbb{Z}}{n \mathbb{Z} }\)]
		Given \(G = \quotient{\mathbb{Z} }{n\mathbb{Z} } = \left< a\mid a^n \right>\),
		find a space with its \hyperref[def:fundamental-group]{fundamental group} being \(G\).
	\end{eg}
	\begin{explanation}
		We see that we can simply take a loop and then wind a \(2\)-disk around the loop \(a\) for \(n\) times.
		\begin{figure}[H]
			\centering
			\incfig{lec12-eg}
			\caption{For \(G = \quotient{\mathbb{Z}}{n\mathbb{Z} } = \left< a \mid a^n \right> \), we wind the boundary around \(a\) for \(n\) times.}
			\label{fig:lec12-eg}
		\end{figure}
	\end{explanation}
	We then see that given a group \(G\) with \hyperref[def:group-presentation]{presentation} \(\left< S \mid R \right> \), one can construct a \(2\)-dimensional \hyperref[def:CW-Complex]{CW complex}
	with \(\pi _1 = G\) by
	\begin{itemize}
		\item Set \(X^1 = \bigvee_{s\in S} S^1\)
		\item For each relation \(r\in R\), glue in a \(2\)-disk along loops specified by the \hyperref[def:word]{word} \(r\).
	\end{itemize}
\end{explanation}

\begin{theorem}
	If \(X\) is a \hyperref[def:CW-Complex]{CW complex} and \(\iota _1\colon X^1\hookrightarrow X\) and \(\iota_2 \colon X^2\hookrightarrow X\),
	then \((\iota _1)_{\ast}\) surjects onto \(\pi _1\) and \((\iota _2)_{\ast}\) is an isomorphism on \(\pi _1\).
\end{theorem}
\begin{proof}
	\todo{HW}
\end{proof}