\lecture{29}{21 Mar.\ 10:00}{Proof of \autoref{thm:singular-homology-agrees-with-simplicial-homology}}
We now give a proof sketch for \autoref{thm:singular-homology-agrees-with-simplicial-homology}.
\begin{proof}[Proof Sketch of \autoref{thm:singular-homology-agrees-with-simplicial-homology}]
	The idea is as follows.
	\begin{itemize}
		\item We can use the \hyperref[thm:long-exact-sequence-of-a-pair]{long exact sequence of a pair} and the \autoref{lma:the-five-lemma} to reduce to proving the result for \hyperref[def:homology-group]{absolute homology groups} (and we will recover the general result).
		\item Because the image \(\Delta^n \to X\) is \emph{compact}, it is contained in some finite \hyperref[def:skeleton]{skeleton} \(X^k\). Use this to reduce the proof to the finite \hyperref[def:skeleton]{skeleton} \(X^k\) of \(X\), namely we can use induction.
	\end{itemize}
	From the \hyperref[thm:long-exact-sequence-of-a-pair]{long exact sequence of a pair} we get
	\[
		\begin{tikzcd}
			{H^\Delta_{n+1}(X^k, X^{k-1})} & {H^\Delta_{n}(X^{k-1})} & {H^\Delta_{n}(X^{k})} & {H^\Delta_{n}(X^k, X^{k-1})} & {H^\Delta_{n-1}(X^{k-1})} \\
			{H_{n+1}(X^k, X^{k-1})} & {H_{n}(X^{k-1})} & {H_{n}(X^{k})} & {H_{n}(X^k, X^{k-1})} & {H_{n-1}(X^{k-1})}
			\arrow["\alpha", color={rgb,255:red,214;green,92;blue,92}, from=1-1, to=2-1]
			\arrow["\beta", color={rgb,255:red,92;green,92;blue,214}, from=1-2, to=2-2]
			\arrow["\gamma", color={rgb,255:red,124;green,194;blue,112}, from=1-3, to=2-3]
			\arrow["\delta", color={rgb,255:red,214;green,92;blue,92}, from=1-4, to=2-4]
			\arrow["\epsilon", color={rgb,255:red,92;green,92;blue,214}, from=1-5, to=2-5]
			\arrow[from=1-1, to=1-2]
			\arrow[from=1-2, to=1-3]
			\arrow[from=1-3, to=1-4]
			\arrow[from=1-4, to=1-5]
			\arrow[from=2-1, to=2-2]
			\arrow[from=2-2, to=2-3]
			\arrow[from=2-3, to=2-4]
			\arrow[from=2-4, to=2-5]
		\end{tikzcd}
	\]
	The Goal is to prove \(\gamma\) is an isomorphism using the \autoref{lma:the-five-lemma}.

	We assume that \textcolor{blue}{\(\beta, \epsilon \)} are isomorphisms by induction, checking the case manually for \(X^0\) (which will be a discrete set of points).

	\begin{exercise}
		Check the base case, namely when \(X^0\).
	\end{exercise}

	It remains to show that \textcolor{red}{\(\alpha, \delta\)} are isomorphisms. We know then that
	\[
		\Delta_n(X^k, X^{k - 1})  = \begin{dcases}
			\mathbb{Z} [\text{\hyperref[def:singular-simplex]{\(k\)-simplices}}], & \text{ if } k=n ; \\
			0,                                                                    & \text{ otherwise}
		\end{dcases}
		\cong H_n^\Delta(X^k, X^{k - 1}).
	\]

	We claim that \(H_n(X^k, X^{k - 1})\) are also \hyperref[def:free-Abelian-group]{free Abelian} on the
	\hyperref[def:singular-simplex]{singular \(k\)-simplices} defined by the characteristic maps \(\Delta^k \to X^k\) when \(n = k\), and \(0\) otherwise. Consider the map
	\[
		\Phi \colon \coprod_\alpha (\Delta^k_\alpha, \partial \Delta^k_\alpha) \to (X^k, X^{k - 1})
	\]
	defined by the characteristic map. This induces an isomorphism on \hyperref[def:homology-group]{homology} since
	\[
		\quotient{\coprod_\alpha \Delta_\alpha^k}{\coprod \partial \Delta_\alpha^k} \overset{\cong}{\longrightarrow}  \quotient{X^k}{X^{k - 1}}.
	\]

	This reduces to check that
	\[
		H_n(\Delta^k, \partial \Delta^k) = \begin{dcases}
			0,           & \text{ if } n \neq k ; \\
			\mathbb{Z} , & \text{ if } n=k
		\end{dcases}
	\]
	generated by the identity map \(\Delta^k \to \Delta^k\).
\end{proof}

\begin{corollary}
	If \(X\) has a \hyperref[def:delta-complex]{\(\Delta \)-complex} structure (or is \hyperref[def:homotopy-equivalence]{homotopy equivalent} to one), then we have the followings.
	\begin{enumerate}[(a)]
		\item If the dimension is \(\leq d\), then \(H_n(X) = 0\) for all \(n>d\).
		\item If \(\overline{X} \) has no \hyperref[def:cell]{cells} of dimension \(p\), then \(H_p(X) = 0\).
		\item If \(\overline{X} \) has no \hyperref[def:cell]{cells} of dimension \(p\), then \(H_{p-1}(X)\) is \hyperref[def:free-Abelian-group]{free Abelian}.
	\end{enumerate}
\end{corollary}

\begin{corollary}
	Given a \hyperref[def:singular-homology-group]{singular} \hyperref[def:homology-class]{homology class} on \(X\), without loss of generality we can choose a \hyperref[def:delta-complex]{\(\Delta \)-complex} structure on \(X\), and we then we can assume the \hyperref[def:homology-class]{class} is represented by a \hyperref[def:simplicial-complex]{simplicial} \hyperref[def:cycle]{\(n\)-cycle}.
\end{corollary}

\begin{remark}
	Recall the definition of \hyperref[def:homology-class]{homology class}, as we noted before, this means we can view \hyperref[def:singular-chain-complex]{singular chain complex} as some kind of geometric subjects. The construction can be found in Hatcher~\cite{hatcher2002algebraic}.
\end{remark}

\section{Degree}
\begin{definition}[Degree]\label{def:degree}
	Let \(f \colon S^n \to S^n\), then
	\[
		f_\ast \colon \mathbb{Z} \cong H_n(S^n) \to H_n(S^n) \cong \mathbb{Z}
	\]
	is a multiplication by some integer\footnote{This just follows from group theory.} \(d \in \mathbb{Z}\), which we call it as the \emph{degree}, denotes as \(\deg(f)\) of \(f\).
\end{definition}

\begin{remark}[Properties of Degree]\label{rmk:property-of-degree}
	We first see some properties of \hyperref[def:degree]{degree}.
	\begin{enumerate}[(a)]
		\item \(\deg(\id_{S^n}) = 1\) since \((\id_{S_n})_\ast = \id_{\mathbb{Z}}\).
		\item If \(f \colon S^n \to S^n\), \(n\geq 0\) is not surjective, then \(\deg(f) = 0\). To see this, we know that \(f_\ast\) factors as
		      \[
			      \begin{tikzcd}
				      {H_n(S^n)} & {H_n(S^n-\{\ast\})=0} & {H_n(S^n)}
				      \arrow[from=1-1, to=1-2]
				      \arrow[from=1-2, to=1-3]
				      \arrow["f_\ast", curve={height=18pt}, from=1-1, to=1-3]
			      \end{tikzcd}
		      \]
		      And since the middle group is zero because \(S^{n} \setminus \{\ast\} \) is \hyperref[def:contractible]{contractible}, \(f_\ast = 0\), so is its \hyperref[def:degree]{degree}.
		\item If \(f \simeq g\), then \(f_\ast = g_\ast\), so \(\deg(f) = \deg(g)\).
		      \begin{note}
			      The converse is true! We'll see this later.
		      \end{note}
		\item \((f \circ g)_\ast = f_\ast \circ g_\ast\), and so \(\deg(f \circ g) = \deg(f)\deg(g)\). Consequently, if \(f\) is a \hyperref[def:homotopy-equivalence]{homotopy equivalence} then \(f\circ g\simeq \id_{} \), hence
		      \[
			      \deg f \deg g = \deg \id_{} = 1,
		      \]
		      so \(\deg f = \pm 1\).
		\item If \(f\) is a reflection fixing the equator, and swapping the \hyperref[def:cell]{\(2\)-cells}, then \(\deg f = -1\).
		      \begin{exercise}
			      It is possible to put a \hyperref[def:delta-complex]{\(\Delta\)-complex} structure with \(2\) \hyperref[def:cell]{\(n\)-cells}, \(\Delta_1\) and \(\Delta_2\) glued together along their \hyperref[def:boundary]{boundary} \((\cong S^{n-1})\), and
			      \[
				      H_n(S^n) = \langle \Delta_1 - \Delta_2 \rangle .
			      \]
			      \begin{figure}[H]
				      \centering
				      \incfig{reflection-about-equator}
				      \label{fig:reflection-about-equator}
			      \end{figure}
		      \end{exercise}
		      \begin{answer}
			      By doing so, we see that the reflection \(f\) interchanges \(\Delta _1\) and \(\Delta _2\), hence the generator is now its negative.
		      \end{answer}
		\item We now have the following linear algebra exercise.
		      \begin{exercise}
			      The antipodal map \(-\id_{}\colon S^{n + 1} \to S^{n + 1}\) given by \(x \mapsto -x\) is the composite of
			      \((n + 1)\) reflections.
		      \end{exercise}
		      So the antipodal map \(-\id_{}\colon S^n \to S^n\) given by \(x \mapsto -x\) has \hyperref[def:degree]{degree} which is the product of \(n + 1\) copies of \((-1)\), and so it has \hyperref[def:degree]{degree} \((-1)^{n + 1}\). (i.e., since the \((n+1)\times (n+1)\) scalar matrix \((-1)\) is composition of \((n+1)\) reflections.)
		\item We see the following.
		      \begin{exercise}
			      If \(f\colon S^n \to S^n\) has no fixed points, then we can homotope \(f\) to the antipodal map via
			      \[
				      f_t(x) = \frac{(1 - t)f(x) - tx}{\left\lVert (1 - t)f(x) - tx\right\rVert}.
			      \]
		      \end{exercise}
		      Therefore, \(\deg f = (-1)^{n + 1}\).
	\end{enumerate}
\end{remark}