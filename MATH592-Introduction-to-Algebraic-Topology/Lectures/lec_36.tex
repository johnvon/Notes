\lecture{36}{6 Apr.\ 10:00}{Lefschetz Fixed Point Theorem}
\section{Lefschetz Fixed Point Theorem}
\begin{definition}[Lefschetz number]\label{def:Lefschetz-number}
	Let \(X\) be a space with the assumption that \(\bigoplus_k H_k(X)\) is finitely generated.\footnote{That is, each \hyperref[def:homology-group]{homology group} is finitely	generated, and there are finitely many nonzero \hyperref[def:homology-group]{homology groups}. For example \(X\) could be a finite \hyperref[def:CW-Complex]{CW complex}.}
	Then the \emph{Lefschetz number} \(\tau(f)\) of a map \(f \colon X \to X\) is
	\[
		\tau(f) \coloneqq \sum_k (-1)^k \trace(f_\ast \colon H_k(X) \to H_k(X)).
	\]
\end{definition}

\begin{remark}
	In particular, we can also write \(\trace(f_{\ast} \circlearrowright H_{k} (X))\).
\end{remark}

\begin{eg}[Euler characteristic via homology]\label{eg:lec36}
	When \(f \simeq \id _X\), we have \(f_\ast = \id_{H_k(X)}\) for all \(k\). Then
	\[
		\trace(f_\ast \colon H_k(X) \to H_k(X)) = \rank(H_k(X)).
	\]
	Hence, we further have
	\[
		\tau(f) = \sum_k (-1)^k\rank(H_k(X)) \eqqcolon \chi(X),
	\]
	where \(\chi(X)\) is the \emph{Euler characteristic}.
\end{eg}

This is not what we would use typically. Traditionally, we have the following.

\begin{definition}[Euler characteristic]\label{def:Euler-characteristic}
	Let \(X\) be a finite \hyperref[def:CW-Complex]{CW complex}, then the \emph{Euler characteristic} \(\chi (X)\) of \(x\) is defined by the alternating sum
	\[
		\chi (X) = \sum_{n}^{} (-1)^{n} c_{n}
	\]
	where \(c_{n} \) is the number of \hyperref[def:cell]{\(n\)-cells} of \(X\).
\end{definition}

\begin{proposition}
	\autoref{def:Euler-characteristic} and the definition given \hyperref[eg:lec36]{in this example} coincides
\end{proposition}
\begin{proof}
	This is immediately by the fact that the \hyperref[def:homology-group]{homology group} can be calculated by \hyperref[def:cellular-homology-group]{cellular homology}, hence \(c_{n}\) is just \(\rank(H_{n} (X))\) for all \(n\), so the result follows.
\end{proof}

\begin{theorem}[Lefschetz Fixed Point Theorem]\label{thm:Lefschetz-fixed-point}
	Suppose \(X\) admits a finite triangulation,\footnote{i.e. a finite \hyperref[def:simplicial-complex]{simplicial complex} structure.} or more generally, \(X\) is a \hyperref[def:retraction]{retract} of a finite \hyperref[def:simplicial-complex]{simplicial complex}. If \(f \colon X \to X\) is a map with \(\tau(f) \neq 0\), then \(f\) has a fixed point.
\end{theorem}
\begin{note}
	Note that the converse does not hold. And in particular, we have
	\[
		\tau (f) = \sum\limits_{k} \trace(f_\# \circlearrowright C^{\CW}_k(X) ).
	\]
\end{note}

\begin{theorem}\label{thm:retract-simplicial-complex}
	If \(X\) is a compact, locally \hyperref[def:contractible]{contractible} space that can be embedded in \(\mathbb{R}^n\) for some \(n\), then \(X\) is a \hyperref[def:retraction]{retract} of a finite \hyperref[def:simplicial-complex]{simplicial complex}.
\end{theorem}

\begin{remark}
	This includes compact manifolds and finite \hyperref[def:CW-Complex]{CW complexes}. Hence, we see that \autoref{thm:Lefschetz-fixed-point} applies on compact manifolds and finite \hyperref[def:CW-Complex]{CW complexes} in particular.
\end{remark}

\begin{definition}\label{def:Lefschetz-number-better}
	Let \(\mathbb{F}\) be a field, and let \(H_k(X; \mathbb{F})\) be the \(k\)-th homology of \(X\) with coefficients in \(\mathbb{F}\). Then \(H_k(X; \mathbb{F})\) is always a vector space over \(\mathbb{F}\). Define \(\tau^{\mathbb{F}}(X)\) be
	\[
		\sum_k (-1)^k \trace(f_\ast \colon H_k(X; \mathbb{F}) \to H_k(X; \mathbb{F})).
	\]
\end{definition}

\begin{remark}
	The \hyperref[thm:Lefschetz-fixed-point]{Lefschetz fixed point theorem} still holds if we replace \(\tau(X) \neq 0\) with \(\tau^{\mathbb{F}}(X) \neq 0\).
\end{remark}