\lecture{26}{14 Mar.\ 10:00}{Continue on Relative Homology}
We start from a definition.

\begin{definition}[Relative chain group]\label{def:relative-chain-group}
	Let \(X\) be a space and let \(A \subseteq X\) be a subspace. Then we define the \emph{relative chain group} as
	\[
		C_n(X, A) = \quotient{C_n(X)}{C_n(A)},
	\]
	which is a quotient of \hyperref[def:Abelian-group]{Abelian groups} of the \hyperref[def:singular-chain-group]{singular chain groups} between \(X\) and \(A\).
\end{definition}

\begin{definition}[Relative chain complex]\label{def:relative-chain-complex}
	The \emph{relative chain complex} \((C_\ast, \partial_\ast )\) consists of \hyperref[def:relative-chain-group]{relative chain group} and the usual \hyperref[def:differential]{differential} associated with the \hyperref[def:singular-chain-group]{singular chain groups} which induces our \hyperref[def:relative-chain-group]{relative chain group}.
\end{definition}

\begin{remark}
	We can indeed adapt \autoref{def:relative-chain-complex} by either \hyperref[def:singular-chain-complex]{singular chain complex} structure or \hyperref[def:simplicial-complex]{simplicial chain complex} structure.
\end{remark}

It's not entirely clear that whether \autoref{def:relative-chain-complex} is well-defined, hence we have the following exercise.

\begin{exercise}
	Since \(\partial^\ast_n(C_{n} (A))\subseteq C_{n-1}(A) \), hence there exists a well-defined map
	\[
		\partial_n\colon \quotient{C_{n} (X)}{C_{n} (A)} \to \quotient{C_{n-1} (X)}{C_{n-1}(A)}.
	\]
	We can verify that \(\partial^{2} =0\). Then, since \(\partial^2 = 0\) we can conclude that these groups will in fact form a \hyperref[def:chain-complex]{chain complex} \((C_\ast(X, A), \partial)\).
\end{exercise}

\begin{definition}[Relative homology group]\label{def:relative-homology-group}
	The \hyperref[def:homology-group]{homology groups} of the \hyperref[def:relative-chain-complex]{relative chain complex} \((C_\ast(X, A), \partial)\) are denoted by \(H_n(X, A)\), and they are called \emph{relative homology groups}.
\end{definition}

We see that there are something interesting going on in \hyperref[def:relative-chain-group]{relative chain group}. Indeed, we can further classify the \hyperref[def:cycle]{cycles} in which as follows.

\begin{definition*}
	Let \(C_\ast(X)\) be the \hyperref[def:relative-chain-complex]{relative chain complex}.
	\begin{definition}[Relative cycle]\label{def:relative-cycle}
		Elements in \(\ker \partial_n\) are called \emph{relative \(n\)-cycles}. These are elements \(\alpha \in C_n(X)\) such that \(\partial_n\alpha \in C_{n - 1}(A)\).
		\begin{figure}[H]
			\centering
			\incfig{def:relative-homology-1}
			\label{fig:def:relative-homology-1}
		\end{figure}
	\end{definition}
	\begin{definition}[Relative boundary]\label{def:relative-boundary}
		Elements \(\alpha\) in \(\im \partial_{n + 1}\) are called \emph{relative \(n\)-boundaries}. This means that \(\alpha = \partial \beta + \gamma\) where \(\beta \in C_n(X)\) and \(\gamma \in C_{n - 1}(A)\).
		\begin{figure}[H]
			\centering
			\incfig{def:relative-homology-2}
			\caption{We see that we have \(\alpha +\gamma =\partial \beta \), where \(\alpha \) is a \hyperref[def:relative-boundary]{relative boundary},
				and \(\gamma \in C_{n-1}(A)\).}
			\label{fig:def:relative-homology-2}
		\end{figure}
	\end{definition}
\end{definition*}

\begin{theorem}[Long exact sequence of a pair]\label{thm:long-exact-sequence-of-a-pair}
	Let \(A \subseteq X\) be spaces, then there exists a long \hyperref[def:exact-sequence]{exact sequence}
	\[
		\begin{tikzcd}
			{\dots} & {\widetilde{H}_n(A)} & {\widetilde{H}_n(X)} & {\widetilde{H}_n(X, A)} \\
			& {\widetilde{H}_{n-1}(A)} & \dots & {\widetilde{H}_{0}(X, A)} & 0
			\arrow[from=1-1, to=1-2]
			\arrow["{i_\ast}", from=1-2, to=1-3]
			\arrow["q", from=1-3, to=1-4]
			\arrow[from=2-4, to=2-5]
			\arrow["{q}", from=2-3, to=2-4]
			\arrow["{i_\ast}", from=2-2, to=2-3]
			\arrow["{\partial}"', from=1-4, to=2-2]
		\end{tikzcd}
	\]
	where \(i_\ast\) is induced by \(A\hookrightarrow X\), and \(q\) is induced by \(C_{n} (X)\twoheadrightarrow \quotient{C_{n} (X)}{C_{n} (A)}\).
\end{theorem}

We will prove that when \((X, A)\) is a \hyperref[def:good-pair]{good pair}, then \(H_n(X, A) \cong \widetilde{H}_n(\quotient{X}{A})\). Then \autoref{thm:les-of-a-good-pair} is a special case of \autoref{thm:long-exact-sequence-of-a-pair}. The key to the proof of \autoref{thm:long-exact-sequence-of-a-pair} above is the following remark.

\begin{remark}
	A \hyperref[def:short-exact-sequence]{short exact sequence} of \hyperref[def:chain-complex]{chain complexes} gives rise to a long \hyperref[def:exact-sequence]{exact sequence} of \hyperref[def:homology-group]{homology groups}. Namely, given a \hyperref[def:short-exact-sequence]{short exact sequence} of \hyperref[def:chain-complex]{chain complexes} \((A_\ast, \partial^A), (B_\ast, \partial^B), (C_\ast, \partial^C)\) such that
	\[
		\begin{tikzcd}
			0 & {A_\ast} & {B_\ast} & {C_\ast} & 0
			\arrow[from=1-1, to=1-2]
			\arrow["\iota", from=1-2, to=1-3]
			\arrow["q", from=1-3, to=1-4]
			\arrow[from=1-4, to=1-5]
		\end{tikzcd}
	\]
	where \(\iota , q\) are \hyperref[def:chain-map]{chain maps} such that
	\[
		\begin{tikzcd}
			0 & {A_n} & {B_n} & {C_n} & 0
			\arrow[from=1-1, to=1-2]
			\arrow["\iota_{n}", from=1-2, to=1-3]
			\arrow["q_{n}", from=1-3, to=1-4]
			\arrow[from=1-4, to=1-5]
		\end{tikzcd}
	\]
	is \hyperref[def:exact]{exact} for all \(n\). Then \autoref{thm:les-of-a-good-pair} will follow from a \hyperref[def:short-exact-sequence]{short exact sequence}
	\[
		\begin{tikzcd}
			0 & {\widetilde{C}_\ast(A)} & {\widetilde{C}_\ast(X)} & {\widetilde{C}_\ast(X, A)} & 0
			\arrow[from=1-1, to=1-2]
			\arrow[from=1-2, to=1-3]
			\arrow[from=1-3, to=1-4]
			\arrow[from=1-4, to=1-5]
		\end{tikzcd}
	\]
	where \(\widetilde{C}_\ast\) denotes the \emph{augmented chain complex} (the one with \(\mathbb{Z}\) after it, as in \autoref{def:reduced-homology-group}).
\end{remark}

\begin{exercise}
	If \(A\) is a single point in \(X\), then \(H_n(X, A) = \widetilde{H}_n(\quotient{X}{A}) = \widetilde{H}_n(X)\).
\end{exercise}