\lecture{34}{1 Apr.\ 10:00}{Proof of \autoref{thm:cellular-homology-agrees-with-singular-homology}}
We're now going to work towards proving that \hyperref[thm:cellular-homology-agrees-with-singular-homology]{cellular homology agrees with singular homology}.
First we need some nontrivial preliminaries.
\begin{lemma}\label{lma:lec34}
	We have that
	\begin{enumerate}[(1)]
		\item \(H_k(X^n, X^{n - 1}) = \begin{dcases}
			      0,                                                  & \text{ if } k\neq n ; \\
			      \left< \hyperref[def:cell]{n\text{-cells}}\right> , & \text{ if } k=n .
		      \end{dcases}\)
		\item \(H_k(X^n) = 0\) for all \(k > n\). If \(X\) is finite dimensional, then \(H_k(X^n) = 0\) for all \(k > \dim X\).
		\item The inclusion \(X^n \hookrightarrow X\) induces \(H_k(X^n) \to H_k(X)\). Then this map is
		      \begin{itemize}
			      \item an isomorphism for \(k < n\)
			      \item surjective for \(k = n\)
			      \item zero for \(k > n\).
		      \end{itemize}
	\end{enumerate}
\end{lemma}
\begin{proof}
	For (1), we see that
	\[
		\quotient{X^n}{X^{n-1}} \cong \text{\hyperref[CW-complex-wedge-sum]{wedge} of one \(n\)-sphere for each \hyperref[def:cell]{\(n\)-cell}}.
	\]
	The result then follows from \autoref{thm:good-pairs-relative-homology} and its immediately corollary, namely
	\[
		\bigoplus_\alpha \widetilde{H} _n(X_\alpha )\cong \widetilde{H} _n\left(\bigvee_\alpha X_\alpha \right)
	\]
	provided that the \hyperref[CW-complex-wedge-sum]{wedge sum} is formed at basepoints \(x_\alpha \in X_\alpha \) such that \((X_\alpha , x_\alpha )\)
	are \hyperref[def:good-pair]{good}, and then we simply consider \((X, A) = (\coprod_{\alpha } X_\alpha , \coprod_{\alpha } \{x_\alpha\})\).

	Now we prove (2) and (3), We consider the \hyperref[thm:long-exact-sequence-of-a-pair]{long exact sequence of a pair} for fixed \(n\),
	\[
		\begin{tikzcd}
			\dots & {H_{k+1}(X^n, X^{n-1})} & {H_k(X^{n-1})} \\
			& {H_k(X^n)} & {H_k(X^n, X^{n-1})} & \dots
			\arrow[from=1-1, to=1-2]
			\arrow[from=1-2, to=1-3]
			\arrow["\cong"', from=1-3, to=2-2]
			\arrow[from=2-2, to=2-3]
			\arrow[from=2-3, to=2-4]
		\end{tikzcd}
	\]

	When \(k + 1 < n\) or \(k > n\) then \(H_{k + 1}(X^n, X^{n - 1}) = 0\) and \(H_k(X^n, X^{n - 1}) = 0\), so the above map \(H_k(X^{n - 1}) \to H_k(X^n)\)
	is an isomorphism. We also get sequences telling us the injective and surjective maps when \(k = n\) or \(k = n - 1\),
	\[
		\begin{tikzcd}
			\dots & {0=H_{n+1}(X^n, X^{n-1})} & {H_n(X^{n-1})} & {H_n(X^n)} \\
			& {H_{n}(X^n, X^{n-1})} & {H_{n-1}(X^{n-1})} & {H_{n-1}(X^n)} \\
			& {H_{n-1}(X^n, X^{n-1})=0} & \dots
			\arrow[from=1-1, to=1-2]
			\arrow[from=1-2, to=1-3]
			\arrow[from=1-3, to=1-4]
			\arrow[from=1-4, to=2-2]
			\arrow[from=2-2, to=2-3]
			\arrow[from=2-3, to=2-4]
			\arrow[from=2-4, to=3-2]
			\arrow[from=3-2, to=3-3]
		\end{tikzcd}
	\]
	So the maps \(H_n(X^{n - 1}) \to H_n(X^n)\) is injective, and the map \(H_{n - 1}(X^{n - 1}) \to H_{n - 1}(X^n)\) is surjective.

	Fix \(k\), then we get a pile of maps induced by the inclusions \(X^n \hookrightarrow X^{n + 1}\)
	\[
		\begin{tikzcd}
			{H_k(X^0)} & {H_k(X^1)} & {H_k(X^2)} & \dots \\
			& {H_k(X^{k-1})} & {H_k(X^{k})} & {H_k(X^{k+1})} \\
			& {H_k(X^{k+2})} & {H_k(X^{k+3})} & \dots
			\arrow["\cong"{description}, color={rgb,255:red,214;green,92;blue,92}, from=1-4, to=2-2]
			\arrow["{\text{inj.}}"{description}, from=2-2, to=2-3]
			\arrow["{\text{surj.}}"{description}, from=2-3, to=2-4]
			\arrow["\cong"{description}, color={rgb,255:red,214;green,92;blue,92}, from=2-4, to=3-2]
			\arrow["\cong"{description}, color={rgb,255:red,214;green,92;blue,92}, from=3-2, to=3-3]
			\arrow["\cong"{description}, color={rgb,255:red,214;green,92;blue,92}, from=3-3, to=3-4]
			\arrow["\cong"{description}, color={rgb,255:red,214;green,92;blue,92}, from=1-1, to=1-2]
			\arrow["\cong"{description}, color={rgb,255:red,214;green,92;blue,92}, from=1-2, to=1-3]
			\arrow["\cong"{description}, color={rgb,255:red,214;green,92;blue,92}, from=1-3, to=1-4]
		\end{tikzcd}
	\]
	\begin{note}
		This sequence is not \hyperref[def:exact-sequence]{exact}. Descriptions of maps (in \textcolor{red}{red}) follow
		from our analysis of the \hyperref[thm:long-exact-sequence-of-a-pair]{long exact sequence of a pair} above.
	\end{note}

	To prove (2),
	\begin{itemize}
		\item \(k = 0\), we do this by hand.
		\item \(k \geq 1\), then \(H_k(X^0) = 0\), so we have that \(H_k(X^0), \dots, H_k(X^{k - 1})\) are all zero from the
		      isomorphisms above. That is the \(k\)-th \hyperref[def:singular-homology-group]{homology} \(H_k(X^n) = H_k(X^n)\)
		      is zero for every \hyperref[def:skeleton]{\(n\)-skeleton} where \(n < k\), just as desired.
	\end{itemize}
	We also have the following collection of maps for fixed \(k\)
	\[
		\begin{tikzcd}
			{H_k(X^k)} & {H_k(X^{k+1})} & {H_k(X^{k+2})} & \dots
			\arrow["{\text{surj.}}", color={rgb,255:red,214;green,92;blue,92}, from=1-1, to=1-2]
			\arrow["\cong", color={rgb,255:red,214;green,92;blue,92}, from=1-2, to=1-3]
			\arrow["\cong", color={rgb,255:red,214;green,92;blue,92}, from=1-3, to=1-4]
		\end{tikzcd}
	\]
	This implies (3) when \(X\) is finite dimensional. For general \(X\), we use the fact that every \hyperref[def:standard-simplex]{simplex}
	has image contained in some finite \hyperref[def:skeleton]{skeleton} (since image is compact).
\end{proof}
\begin{exercise}
	Check (2) and (3) in \autoref{lma:lec34} directly in the case that the \hyperref[def:CW-Complex]{CW complex} structure is a \hyperref[def:delta-complex]{\(\Delta\)-complex}
	structure using \hyperref[def:simplicial-chain-group]{simplicial chains}.
\end{exercise}

We now prove \autoref{thm:cellular-homology-agrees-with-singular-homology}.
\begin{proof}[Proof of \autoref{thm:cellular-homology-agrees-with-singular-homology}]\label{pf:thm:cellular-homology-agrees-with-singular-homology}
	We get some \hyperref[def:exact-sequence]{exact sequences} from our \hyperref[lma:lec34]{preliminaries},
	\[
		\begin{tikzcd}[column sep=small,row sep=tiny]
			{0=H_{n+1}(X^n)} & {H_n(X^{n})} & {H_n(X^{n}, X^{n+1})} & {H_{n-1}(X^{n,-1})} \\
			{H_{n+1}(X^{n+1}, X^n)} & {H_n(X^n)} & {H_n(X^{n+1})} & {H_n(X^{n+1}, X^n)=0}
			\arrow[from=1-2, to=1-3]
			\arrow[from=1-3, to=1-4]
			\arrow[from=1-1, to=1-2]
			\arrow[from=2-1, to=2-2]
			\arrow[from=2-2, to=2-3]
			\arrow[from=2-3, to=2-4]
		\end{tikzcd}
	\]
	These come from the \hyperref[thm:long-exact-sequence-of-a-pair]{long exact sequences of a pair} combined with the things we've
	deduced in the \hyperref[lma:lec34]{preliminaries}. We can paste these together into a diagram, we have
	\par
	\adjustbox{scale=0.85,center}{%
		\begin{tikzcd}[column sep=tiny]
			&&&& {\color{red}0} \\
			& {\color{red}0} && {\color{red}H_n(X^{n+1})\cong H_n(X)} \\
			&& {\color{red}H_n(X^n)} \\
			\dots & {\color{red}H_{n+1}(X^{n+1}, X^n)} && {\color{red}H_n(X^n, X^{n-1})} && {H_{n-1}(X^{n-1}, X^{n-2})} & \dots \\
			&&&& {\color{red}H_{n-1}(X^{n-1})} \\
			&&& 0
			\arrow[color={rgb,255:red,214;green,92;blue,92}, from=2-2, to=3-3]
			\arrow["{\partial_{n+1}}", color={rgb,255:red,214;green,92;blue,92}, from=4-2, to=3-3]
			\arrow[color={rgb,255:red,214;green,92;blue,92}, from=2-4, to=1-5]
			\arrow[color={rgb,255:red,214;green,92;blue,92}, from=3-3, to=2-4]
			\arrow[from=4-6, to=4-7]
			\arrow["{d_n}"', from=4-4, to=4-6]
			\arrow["{\partial_n}"', color={rgb,255:red,214;green,92;blue,92}, from=4-4, to=5-5]
			\arrow["{j_{n-1}}"', from=5-5, to=4-6]
			\arrow[from=6-4, to=5-5]
			\arrow["{d_{n+1}}", from=4-2, to=4-4]
			\arrow["{j_n}", color={rgb,255:red,214;green,92;blue,92}, from=3-3, to=4-4]
			\arrow[from=4-1, to=4-2]
		\end{tikzcd}
	}
	\par Hatcher\cite{hatcher2002algebraic} tells us this diagram commutes, and what we've done here tells us that the two red diagonal pieces
	crossing at \(H_n(X^n)\) are \hyperref[def:exact]{exact}. We also have \hyperref[def:exact]{exactness} of the bottom right diagonal
	by just going down a degree.

	Then the horizontal row has to at least be a \hyperref[def:chain-complex]{chain complex} since the diagram commutes, and we have
	\[
		d_{n} \circ d_{n+1}  = (j_{n - 1} \circ \underbrace{\partial_n) \circ (j_n}_{0} \circ \partial_{n + 1}) = 0,
	\]
	hence we see that \(d^{2} = 0\).\footnote{This is the missing part of the proof of \autoref{thm:lec32}.}

	By \hyperref[def:exact]{exactness}, we know that if \(\iota_\ast \colon H_n(X^n) \to H_n(X^{n + 1})\), then using the first isomorphism theorem,
	\[
		H_n(X) \cong H_n(X^{n + 1}) = \im \iota_\ast \cong  \quotient{H_n(X^n)}{\ker \iota_\ast} = \quotient{H_n(X^n)}{\im \partial_{n + 1}}.
	\]

	Since \(j_n\) injects by \hyperref[def:exact]{exactness},
	\begin{align*}
		j_n : H_n(X^n)       & \xrightarrow{\cong} j_n(H_n(X^n))                                     \\
		\im \partial_{n + 1} & \xrightarrow{\cong} \im (j_n \circ \partial_{n + 1}) = \im d_{n + 1},
	\end{align*}
	so \(j_{n - 1}\) must also inject by \hyperref[def:exact]{exactness}, and by applying \hyperref[def:exact]{exactness}, we have
	\[
		\ker d_n = \ker \partial_n = \im j_n.
	\]

	Then we just do some group theory, the \(n\)-th \hyperref[def:cellular-homology-group]{cellular homology group} is
	\[
		\quotient{\ker d_n}{\im d_{n + 1}} \cong \quotient{\im j_n}{\im (j_n \circ \partial_{n + 1})} \cong \quotient{H_n(X^n)}{\im \partial_{n + 1}} \cong H_n(X).
	\]

	There is one thing left to show, namely commutativity of this map. We claim that the differentials \(d_n = j_n \circ \partial_{n + 1}\)
	satisfy the formula (in terms of degree) that we stated. This is done by direct analysis of definitions of maps; details in Hatcher\cite{hatcher2002algebraic}.
\end{proof}
