\lecture{35}{4 Apr.\ 10:00}{Eilenberg-Steenrod Axioms}
\section{Eilenberg-Steenrod Axioms}
We can approach the homology theory in an \textbf{axiomatic} way. Specifically, we're interested in the Eilenberg-Steenrod axioms.
To start with, we first see some definitions.

\begin{definition}[Natural transformation]\label{def:natural-transformation}
	Given two \hyperref[def:functor]{functors}
	\[
		F, G \colon \mathscr{C} \to \mathscr{D},
	\]
	a \emph{natural transformation} \(\eta \colon F \to G\) is a collection of maps \(\eta_X \colon F(X) \to G(X)\) lying
	in \(\mathscr{D}\) for every \(X \in \mathscr{C}\) so that for any map \(f \colon X \to Y\), we have a commutative diagram
	\[
		\begin{tikzcd}
			{F(X)} & {G(X)} \\
			{F(Y)} & {G(Y)}
			\arrow["{\eta X}", from=1-1, to=1-2]
			\arrow["{G(f)}", from=1-2, to=2-2]
			\arrow["{\eta Y}"', from=2-1, to=2-2]
			\arrow["{F(f)}"', from=1-1, to=2-1]
		\end{tikzcd}
	\]
\end{definition}

\begin{definition}[Homology theory]\label{def:homology-theory}
	A \emph{homology theory} is a sequence of \hyperref[def:functor]{functors}
	\[
		H_n \colon \text{\hyperref[def:good-pair]{pairs} \((X, A)\) of spaces} \to \text{\hyperref[def:Abelian-group]{Abelian groups}}
	\]
	equipped with \hyperref[def:natural-transformation]{natural transformations} \(\partial \colon H_n(X, A) \to H_{n - 1}(A)\),
	where \(H_{n - 1}(A) \coloneqq H_{n - 1}(A, \varnothing)\), is called the \underline{boundary map}.

	Naturality here means that for any map \(f \colon (X, A) \to (Y, B)\) we have a commutative diagram
	\[
		\begin{tikzcd}
			{H_{n}(X, A)} & {H_{n-1}(A)} \\
			{H_n(Y, B)} & {H_{n-1}(B)}
			\arrow["{\partial }", from=1-1, to=1-2]
			\arrow["{f_\ast}", from=1-2, to=2-2]
			\arrow["{\partial }"', from=2-1, to=2-2]
			\arrow["{f_\ast}"', from=1-1, to=2-1]
		\end{tikzcd}
	\]
	These must satisfy the following \(5\) axioms.
	\begin{enumerate}[(1)]
		\item (Homotopy) If \(f, g \colon (X, A) \to (Y, B)\) and \(f \simeq g\), then \(f_\ast = g_\ast\).
		\item (Excision) If \(U \subseteq A \subseteq X\) such that \(\overline{U} \subseteq \Int  (A)\), then
		      \[
			      \iota \colon (X \setminus U, A \setminus U) \hookrightarrow (X, A)
		      \]
		      induces isomorphisms on \(H_n\).
		\item (Dimension) \(H_n(\ast) = 0\) for all \(n \neq 0\).
		\item (Additivity) \(H_n\left(\coprod_\alpha X_\alpha\right) = \bigoplus_\alpha H_n(X_\alpha)\).
		\item (Exactness) If we have an inclusion \(\iota \colon A \hookrightarrow X\)\footnote{Note that we use
			      \(X\coloneqq (X, \varnothing )\) for every space \(X\).} and \(j \colon X \to (X, A)\) induces a
		      long \hyperref[def:exact-sequence]{exact sequence}
		      \[
			      \begin{tikzcd}[column sep=small]
				      \dots & {H_n(A)} & {H_n(X)} & {H_{n}(X, A)} & {H_{n-1}(A)} & \dots
				      \arrow[from=1-1, to=1-2]
				      \arrow["{\iota_\ast}", from=1-2, to=1-3]
				      \arrow["{j_\ast}", from=1-3, to=1-4]
				      \arrow["\partial", from=1-4, to=1-5]
				      \arrow[from=1-5, to=1-6]
			      \end{tikzcd}
		      \]
	\end{enumerate}
\end{definition}

\begin{definition}[Extraordinary homology theory]\label{def:extraordinary-homology-theory}
	If \(H_\ast\) satisfies all \hyperref[def:homology-theory]{axioms} but dimension, it is called an \emph{extraordinary homology theory}.
\end{definition}
\begin{eg}
	Topological \(K\)-theory, bordism, and cobordism.\footnote{\url{https://en.wikipedia.org/wiki/Cobordism}}
\end{eg}

\begin{theorem}\label{thm:characterization-of-singular-homology}
	If \(H_n \colon \text{\hyperref[def:CW-Complex]{CW} \hyperref[def:good-pair]{pairs}} \to \underline{\mathrm{Ab}}\) is a
	\hyperref[def:homology-theory]{homology theory} and \(H_0(\ast) = \mathbb{Z}\), then \(H_n\) are exactly
	the \hyperref[def:singular-homology-group]{singular homology} \hyperref[def:functor]{functors} up to a natural isomorphism of \hyperref[def:functor]{functors}.

	More generally, if \(H_0(\ast) = G\), then \(H_n\) are exactly the
	\hyperref[def:singular-homology-group]{singular homology} \hyperref[def:functor]{functors} with coefficients in the
	\hyperref[def:Abelian-group]{Abelian group} \(G\).
\end{theorem}
\begin{proof}
	Given \(H_\ast\), reconstruct the \hyperref[def:cellular-chain-complex]{cellular chain groups} \(H_n(X^n, X^{n-1})\) using the
	\hyperref[def:homology-theory]{axioms}.
	\begin{itemize}
		\item Show the \hyperref[def:homology-theory]{homology} of this \hyperref[def:chain-complex]{chain complex} are
		      the \hyperref[def:cellular-homology-group]{cellular homology groups} of \(X\).
		\item Show these agree with \(H_{n} (X^n, X^{n-1})\). The exact same argument in \autoref{thm:cellular-homology-agrees-with-singular-homology} applies.
	\end{itemize}

	We then check that the \hyperref[def:cellular-homology-group]{cellular homology groups} we just constructed satisfies the
	\hyperref[thm:calculation-with-local-degree]{degree formula} as in our last step. This is a bit more difficult, but we won't get into it.
\end{proof}

\chapter{Lefschetz Fixed Point Theorem}
After developing \hyperref[def:homology-theory]{homology theory} rigorously, we now see an incredibly useful and interesting application and some of its
implication as well.

\section{Trace}
\begin{definition}[Trace]\label{def:trace}
	Let \(\varphi \colon \mathbb{Z}^n \to \mathbb{Z}^n\) be a group homomorphism, we may represent this with a matrix
	\(A = \left[a_{ij}\right]_{i, j}\) with \emph{trace} being
	\[
		\trace(A) \coloneqq a_{11} + \dots + a_{nn}.
	\]

	For a group homomorphism \(\varphi \colon M \to M\) where \(M\) is a \hyperref[apx:ssc:finitely-generated-Abelian-group]{finitely generated Abelian group},
	we define the \emph{trace} of \(\varphi\) to be the \emph{trace} of the induced map \(\overline{\varphi} \colon \quotient{M}{M_T} \to \quotient{M}{M_T}\),
	where \(M_T\) is the \hyperref[def:torsion-subgroup]{torsion subgroup} of \(M\).
\end{definition}

\begin{exercise}
	We have
	\begin{enumerate}[(1)]
		\item \(\trace(AB) = \trace(BA)\).
		\item \(\trace(A) = \trace(BAB^{-1})\).
	\end{enumerate}

	Thus, \hyperref[def:trace]{trace} is independent of change of basis of \(\mathbb{Z} ^n\).
\end{exercise}