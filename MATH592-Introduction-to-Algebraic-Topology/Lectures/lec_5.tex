\lecture{5}{14 Jan. 10:00}{Operation on Spaces}
\section{Operations on CW Complexes}
\subsection{Products}\label{CW-complex-product}
We can consider the product of two \hyperref[def:CW-Complex]{CW complex} given by a \hyperref[def:CW-Complex]{CW complex} structure. Namely, given \(X\) and \(Y\)
two \hyperref[def:CW-Complex]{CW complexes}, we can take two \hyperref[def:cell]{cells} \(e^n_{\alpha }\) from \(X\) and \(e^m_{\beta }\) from \(Y\) and
form the product space \(e^n_{\alpha }\times e^m_{\beta }\), which is homeomorphic to an \hyperref[def:cell]{\((n+m)\)-cell}. We then
take these products as the \hyperref[def:cell]{cells} for \(X\times Y\).

Specifically, given \(X\), \(Y\) are \hyperref[def:CW-Complex]{CW complexes}, then \(X\times Y\) has a \hyperref[def:cell]{cells} structure
\[
	\left\{e_{\alpha}^m \times e_{\alpha}^n\colon e^m_{\alpha}\text{ is an \hyperref[def:cell]{\(m\)-cell} on \(X\)}, e^n_{\alpha}\text{ is an \hyperref[def:cell]{\(n\)-cell} on \(Y\)}\right\}.
\]
\begin{remark}
	The product topology may not agree with the \hyperref[def:weak-topology]{weak topology} on the \(X\times Y\). However, they do agree if
	\(X\) or \(Y\) is locally compact \underline{or} if \(X\) and \(Y\) both have at most countably many \hyperref[def:cell]{cells}.
\end{remark}

\subsection{Wedge Sum}\label{CW-complex-wedge-sum}
Given \(X\), \(Y\) are \hyperref[def:CW-Complex]{CW complexes}, and \(x_0\in X^0\), \(y_0\in Y^0\) (only points). Then we define \(X\vee Y = X\coprod Y\)
with quotient topology.
\begin{remark}
	\(X\lor Y\) is a \hyperref[def:CW-Complex]{CW complex}.
\end{remark}

\subsection{Quotients}\label{CW-complex-quotient}
Let \(X\) be a \hyperref[def:CW-Complex]{CW complex}, and \(A\subseteq X\) \hyperref[def:CW-subcomplex]{subcomplex} (closed union of \hyperref[def:cell]{cells}), then
\(\quotient{X}{A}\) is a quotient space collapse \(A\) to one point and inherits a \hyperref[def:CW-Complex]{CW complex} structure.
\begin{remark}
	\(\quotient{X}{A} \) is a \hyperref[def:CW-Complex]{CW complex}.
\end{remark}
\begin{explanation}
	With the \hyperref[def:skeleton]{\(0\)-skeleton} being
	\[
		(X^0 - A^0)\coprod \ast
	\]
	where \(\ast\) is a point for \(A\), then each \hyperref[def:cell]{cell} of \(X-A\) is attached to \(\left(\quotient{X}{A} \right)^n\)
	by \hyperref[def:attaching-map]{attaching map}
	\[
		\begin{tikzcd}
			S^n \ar[r,"\phi_{\alpha}"] & X^n \ar[r, "\text{quotient}"] & \quotient{X^n}{A^n}
		\end{tikzcd}
	\]
\end{explanation}

\begin{eg}
	We can take the sphere and squish the equator down to form a \hyperref[CW-complex-wedge-sum]{wedge} of two spheres.
	\begin{figure}[H]
		\centering
		\incfig{eg:quotient-cw-complex-sphere}
		\label{fig:eg:quotient-cw-complex-sphere}
	\end{figure}
\end{eg}

\begin{eg}
	We can take the torus and squish down a ring around the hole.
	\begin{figure}[H]
		\centering
		\incfig{eg:quotient-cw-complex-torus}
		\label{fig:eg:quotient-cw-complex-torus}
	\end{figure}
	We see that \(\quotient{X}{A}\) is \hyperref[def:homotopy-equivalence]{homotopy equivalent}
	to a \(2\)-sphere \hyperref[CW-complex-wedge-sum]{wedged} with a \(1\)-sphere via extending the orange point into a line, and then
	sliding the left point to the line along the \(2\)-sphere towards the other points, forming a circle.
\end{eg}
