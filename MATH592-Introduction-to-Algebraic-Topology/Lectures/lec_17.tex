\lecture{17}{14 Feb.\ 10:00}{Deck Transformation}
\begin{eg}
	\hyperref[def:deck-transformation]{Deck transformations} \(G(\widetilde{X} )\) are a subgroup of the group of homeomorphisms of \(\widetilde{X} \).
\end{eg}

\begin{eg}
	Given the \hyperref[def:covering-map]{cover} \(p\colon \mathbb{R} \to S^1\).
	\begin{itemize}
		\item \hyperref[def:deck-transformation]{Deck maps}: translation by \(n\in \mathbb{Z} \) units.
		\item \(G(\mathbb{R} )\cong \mathbb{Z} \)
	\end{itemize}
\end{eg}
\begin{eg}
	Given the \hyperref[def:covering-map]{cover} \(p_{n}\colon S^{1}\to S^1\) be an \(n\)-sheeted \hyperref[def:covering-map]{cover}.
	\begin{itemize}
		\item \hyperref[def:deck-transformation]{Deck maps}: rotation by \(2\pi / n\).
		\item \(G(S^1, p_{n})\cong \quotient{\mathbb{Z} }{n \mathbb{Z} } \)
	\end{itemize}
	\begin{figure}[H]
		\centering
		\incfig{eg:lec17:N-sheeted-cover}
		\caption{\(p_{n} \colon S^1 \to S^1\) be an \(n\)-sheeted \hyperref[def:covering-map]{cover}, here \(n = 3\).}
		\label{fig:eg:lec17:N-sheeted-cover}
	\end{figure}
\end{eg}

\begin{exercise}[Deck Transformation is determined by the image of one point]\label{ex:lec17}
	Given \(X, \widetilde{X} \) are \hyperref[def:path]{path}-connected, locally \hyperref[def:path]{path}-connected,
	\hyperref[def:deck-transformation]{deck map} is determined by the image of any one point.
\end{exercise}
\begin{answer}
	\[
		\begin{tikzcd}
			& {\widetilde{X}} \\
			{\widetilde{X}} & X
			\arrow["f", from=2-1, to=1-2]
			\arrow["p"', from=2-1, to=2-2]
			\arrow["p", from=1-2, to=2-2]
		\end{tikzcd}
	\]
\end{answer}

\begin{corollary}\label{col:lec17}
	If a \hyperref[def:deck-transformation]{deck transformation} has a fixed point, it is the identity transformation.
\end{corollary}

\begin{exercise}
	Let \(X\) be connected. Given a \hyperref[def:deck-transformation]{deck transformation} \(\tau \colon \widetilde{X} \to \widetilde{X} \), \(\tau \)
	defines a permutation of \(p^{-1} (\{x_0\})\). If this permutation has a fixed point, then it is the identity.
\end{exercise}

\begin{definition}[Regular (normal) cover]\label{def:regular-cover}\label{def:normal-cover}
	A \hyperref[def:covering-space]{covering space} \(p\colon \widetilde{X} \to X\) is \emph{regular} or \emph{normal}
	if \(\forall x_0\in X\), \(\forall \widetilde{x}_0, \widetilde{x} _1 \in p^{-1} (\{x_0\})\), there exists
	a \hyperref[def:deck-transformation]{deck transformation} such that
	\[
		\widetilde{x} _0 \mapsto \widetilde{x} _1.
	\]
\end{definition}
\begin{eg}[Regular and non-regular cover of \(S^1\vee S^1\)]
	Given the following \hyperref[def:covering-map]{covers} of \(S^1\vee S^1\), determine which cover is \hyperref[def:regular-cover]{regular}.
	\begin{figure}[H]
		\centering
		\incfig{eg:lec17:regular}
		\label{fig:eg:lec17:regular}
	\end{figure}
\end{eg}
\begin{explanation}
	The left one is \hyperref[def:regular-cover]{regular}, while the right one is not since there is no automorphism from \(\widetilde{x} _0\) to
	\(\widetilde{x} _1\) or \(\widetilde{x} _2\).
\end{explanation}

\begin{remark}
	A \hyperref[def:regular-cover]{regular} \hyperref[def:covering-map]{cover} is \emph{as symmetric as possible}.
\end{remark}

\begin{exercise}
	\hyperref[def:regular-cover]{Regular} means that the group \(G(\widetilde{X} )\) acts transitively on \(p^{-1} (\{x_0\})\). Explain why we cannot ask for
	more than this:
	\begin{center}
		\(G(\widetilde{X})\) cannot induce the full symmetric group on \(p^{-1} (\{x_0\})\) provided that \(\left\vert p^{-1} (\{x_0\}) \right\vert > 2\).
	\end{center}
\end{exercise}
\begin{answer}
	The key is uniqueness.
\end{answer}

Since we're talking about symmetric, it's natural to introduce the following concept for groups.
\begin{definition}[Normal subgroup]\label{def:normal-subgroup}
	A subgroup \(N\) of \(G\) is called a \emph{normal subgroup} if it's invariant under conjugation, and we denote this relation as \(N \triangleleft G\).
\end{definition}

\begin{definition}[Normalizer]\label{def:normalizer}
	Given \(G\) as a group, \(H\subseteq G\) is a subgroup of \(G\). Then the \emph{normalizer} of \(H\), denoted by \(N(H)\), is defined as
	\[
		N(H) \coloneqq \left\{g\in G \mid gH = H g\right\}.
	\]
\end{definition}

\begin{exercise}
	We can prove the followings.
	\begin{enumerate}[(1)]
		\item \(N(H)\) is a subgroup.
		\item \(H\leq N(H)\).
		\item \(H\) is \hyperref[def:normal-subgroup]{normal} in \(N(H)\).
		\item If \(H\leq G\) is \hyperref[def:normal-subgroup]{normal}, \(N(H) = G\).
		\item \(N(H)\) is the largest subgroup (under containment) of \(G\) containing \(H\) as \hyperref[def:normal-subgroup]{normal subgroup}.
	\end{enumerate}
\end{exercise}

\begin{proposition}\label{prop:lec17}
	Given \(p\colon (\widetilde{X} , \widetilde{x} _0)\to (X, x_0)\) be a \hyperref[def:covering-map]{cover}, and \(\widetilde{X} , X\)
	are \hyperref[def:path]{path}-connected, locally \hyperref[def:path]{path}-connected. Let
	\[
		H = p_\ast (\pi _1(\widetilde{X} , \widetilde{x} _0))\subseteq \pi _1(X, x_0).
	\]
	Then
	\begin{enumerate}[(1)]
		\item \(p\) is \hyperref[def:normal-cover]{normal} if and only if \(H\subseteq \pi _1(X, x_0)\) is \hyperref[def:normal-subgroup]{normal}.
		\item We have
		      \[
			      G(\widetilde{X} )\cong \quotient{N(H)}{H},
		      \]
		      where \(G(\widetilde{X} )\) are \hyperref[def:deck-transformation]{deck maps}, and \(N(H)\) is the \hyperref[def:normalizer]{normalizer}
		      of \(H\) in \(\pi _1(X, x_0)\).
	\end{enumerate}
\end{proposition}

\begin{remark}
	A fact is worth noting is the following. Let \(\widetilde{\gamma} \) be a \hyperref[def:path]{path} \(\widetilde{x} _1\) to \(\widetilde{x} _0\).
	Then
	\[
		p_\ast (\pi _1(\widetilde{X} , \widetilde{x} _0))= [\gamma ] p_\ast (\pi _1(\widetilde{X} , \widetilde{x} _1)) [\gamma ^{-1}].
	\]
	\begin{figure}[H]
		\centering
		\incfig{rmk:lec17:1}
		\label{fig:rmk:lec17:1}
	\end{figure}
\end{remark}