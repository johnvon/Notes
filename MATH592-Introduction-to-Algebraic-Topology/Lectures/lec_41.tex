\lecture{41}{18 Apr.\ 10:00}{Prepare for Final}
We now go through more problems to get prepared for the final!
\begin{exercise}[QR Jan. 2021]
	Let \(G\) be a connected topological space and \(\overline{w} \) a topological group structure, i.e., continuous multiplication
	\(M\colon G\times G \to G\) and inverse \(i\colon G\to G\) that define a group structure. Assume \(G\) has finite \hyperref[def:CW-Complex]{CW complex}
	structure. Show \(\chi (G) = 0\) unless \(G = \{1\}\).
\end{exercise}
\begin{answer}
	Let \(g\in G\) such that \(g \neq \id_{} \). We can then choose a \hyperref[def:path]{path} \(\gamma \) from \(g\) to \(\id_{} \).
	Then we have
	\[
		\begin{split}
			\tau _g\colon G &\to G\\
			h &\mapsto gh.
		\end{split}
	\]
	Then \(\gamma \) gives \hyperref[def:homotopy]{homotopy} from \(\tau _g\) to \(\id_{} \) such that \(h \mapsto \gamma (t) h\),
	so
	\[
		\tau (\tau _{g} )= \tau (\id_{} )= \chi (G) = 0,
	\]
	hence by \autoref{thm:Lefschetz-fixed-point}, \(\tau _g\) has no fixed points.
\end{answer}

\begin{exercise}[QR May 2019]
	For what \(g \geq 0\) is it true that for all \(h \geq g\), a compact oriented genus-\(g\) surface \(X\) (no boundary)
	has covering \(f\colon Y\to X\) where \(Y\) is a compact, oriented surface of genus-\(h\)?
\end{exercise}
\begin{answer}
	From \hyperref[def:Euler-characteristic]{Euler characteristic}, we have
	\[
		(2 - 2g) d = 2 - 2h
	\]
	for a degree \(d\) \hyperref[def:covering-space]{covering}. This implies \(h \equiv 1\pmod {g-1}\). We see that the only solutions
	for all \(h\) if \(g = 2\). Then for a particular \(d\), we have the following \hyperref[def:covering-space]{cover}.
	\begin{figure}[H]
		\centering
		\incfig{lec-41-2}
		\label{fig:lec-41-2}
	\end{figure}
\end{answer}

\begin{exercise}[QR Jan. 2019]
	Let \(S_1, S_2\) be two disjoint  copies of \(n\)-sphere, \(n>1\). Choose distinct points \(A_{i} , B_{i} \in S_{i} \), and let
	\(Z\) is obtained by gluing \(A_1\sim A_2\), \(B_1\sim B_2\). What's the lowest number of \hyperref[def:cell]{cells} in
	\hyperref[def:CW-Complex]{CW complex} structure on \(Z\).
\end{exercise}
\begin{answer}
	In \(n=2\), we have the following figure.
	\begin{figure}[H]
		\centering
		\incfig{lec-41-3}
		\label{fig:lec-41-3}
	\end{figure}
\end{answer}

\begin{exercise}[QR Jan. 2016]
	Let \(U, V\subseteq S^n\), \(n \geq 2\) be non-empty connected open sets such that \(S^n = U \cup V\). Show \(U \cap V\) is
	connected.
\end{exercise}
\begin{answer}
	We simply calculate \(\widetilde{H} _0\) via Mayer-Vietoris.
\end{answer}

\begin{exercise}[QR Jan. 2016]
	Let \(p\) be a prime, and \(X\) be a finite \hyperref[def:CW-Complex]{CW complex}, \(\quotient{\mathbb{Z} }{p \mathbb{Z} }\circlearrowright X \).

	\begin{enumerate}[(a)]
		\item If \(\chi (X)\) is not divisible by \(p\), show that the action has a (global) fixed point.
		\item Give an example of such an action that is fixed point free when \(\chi (X) = 0\).
	\end{enumerate}
\end{exercise}
\begin{answer}
	We have the following.
	\begin{enumerate}[(a)]
		\item We use the \hyperref[def:Euler-characteristic]{Euler characteristic} on the \hyperref[def:covering-space]{covering space} action.\footnote{Since we assume
			      has no fixed point, with the fact that \(X\) admits a \hyperref[def:CW-Complex]{CW complex}
			      structure hence Hausdorff, so \(\quotient{\mathbb{Z} }{p\mathbb{Z} } \) is indeed a \hyperref[def:covering-space]{covering space} action.}
		\item Rotation \(2\pi / p\) on \(S^1\).
	\end{enumerate}
\end{answer}