\lecture{13}{4 Feb.\ 10:00}{Graph Structure of CW Complex}
We can now put some kind of \hyperref[def:graph]{graph} structure on a space \(X\) and relate it to the calculation of \(\pi _1\).

\begin{definition*}
	We import some topological definitions of graph theoretic concepts.
	\begin{definition}[Graph]\label{def:graph}
		A \emph{graph} is a \(1\)-dimensional \hyperref[def:CW-Complex]{CW complex}.
	\end{definition}
	\begin{definition}[Subgraph]\label{def:subgraph}
		A \emph{subgraph} is a \hyperref[def:CW-subcomplex]{subcomplex}.
	\end{definition}
	\begin{definition}[Tree]\label{def:tree}
		A \emph{tree} is a contractible \hyperref[def:graph]{graph}.
	\end{definition}
	\begin{definition}[Maximal tree]\label{def:maximal-tree}
		A \hyperref[def:tree]{tree} in a \hyperref[def:graph]{graph} \(X\) (necessarily a \hyperref[def:subgraph]{subgraph}) is
		\emph{maximal} or \emph{spanning} if it contains all the vertices.
	\end{definition}
\end{definition*}

\begin{theorem}
	Every connected \hyperref[def:graph]{graph} has a \hyperref[def:maximal-tree]{maximal tree}.
	Every \hyperref[def:tree]{tree} is contained in a \hyperref[def:maximal-tree]{maximal tree}.
\end{theorem}
\begin{proof}
	\todo{DIY}
\end{proof}

\begin{corollary}
	Suppose \(X\) is a connected \hyperref[def:graph]{graph} with basepoint \(x_0\). Then \(\pi _1(X, x_0)\) is a
	\hyperref[def:free-group]{free group}.

	\par Furthermore, we can give a \hyperref[def:group-presentation]{presentation} for \(\pi _1(X, x_0)\) by finding a
	\hyperref[def:maximal-tree]{spanning tree} \(T\) in \(X\). The generators of
	\(\pi _1\) will be indexed by \hyperref[def:cell]{cells} \(e_\alpha \in X-T\), and \(e_\alpha \) will correspond to a loop that passes through \(T\),
	traverses \(e_\alpha \) once, then returns to the basepoint \(x_0\) through \(T\).
\end{corollary}
\begin{proof}
	The idea is simple. \(X\) is \hyperref[def:homotopy-equivalence]{homotopy equivalent} to \(\quotient{X}{T}\) via previous work on the homework,
	\(T\) contains all the vertices, so the \hyperref[CW-complex-quotient]{quotient} has a single vertex. Thus, it is a
	\hyperref[CW-complex-wedge-sum]{wedge} of circles, and each \(e_\alpha \) projects to a loop in \(\quotient{X}{T} \).
	\begin{figure}[H]
		\centering
		\incfig{pf:lec12-1}
		\label{fig:pf:lec12-1}
	\end{figure}

	To be formal, we calculate the \hyperref[def:fundamental-group]{fundamental group} of \(X\) by considering its \hyperref[def:CW-Complex]{CW complex} structures.
	For now, we need to see that the \hyperref[def:fundamental-group]{fundamental group} of a \hyperref[def:skeleton]{\(1\)-skeleton} (a graph) can be found by taking
	a \hyperref[def:maximal-tree]{maximal tree}, and then \hyperref[CW-complex-quotient]{quotienting} out the space by the \hyperref[def:tree]{tree}
	to get a \hyperref[CW-complex-wedge-sum]{wedge} of circles.
	\begin{figure}[H]
		\centering
		\incfig{pf:lec12-2}
		\label{fig:pf:lec12-2}
	\end{figure}

	\par We now prove that the \hyperref[def:maximal-tree]{maximal trees} exist. Recall that \(X\) is a \hyperref[CW-complex-quotient]{quotient} of
	\(X^0\coprod_\alpha I_\alpha\). Since each subset \(U\) is open if and only if it intersects each edge \(\overline{e} _\alpha \) in an open subset.
	A map \(X\to Y\) if and only if its restriction to each edge \(\overline{e} _\alpha \) is continuous. Now, take \(X_0\) to be a \hyperref[def:subgraph]{subgraph}.
	Our goal is to construct a \hyperref[def:subgraph]{subgraph} \(Y\) with
	\begin{itemize}
		\item \(X_0 \subset Y\subset X\)
		\item \(Y\) \hyperref[def:deformation-retraction]{deformation retracts} to \(X_0\)
		\item \(Y\) contains all vertices of \(X\).
	\end{itemize}

	So if we take \(X_0\) to be a vertex, then \(Y\) is our \hyperref[def:tree]{tree}, and the result follows.

	So, we now build a sequence \(X_0\subset X_1\subset \dots  \) and correspondingly, \(Y_0\subset Y_1\subset \dots\). We start with
	\(X_0\) and inductively define
	\[
		X_i \coloneqq X_{i-1}\bigcup \text{ all edges \(\overline{e} _\alpha \) with one or both vertices in \(X_{i-1}\) }.
	\]
	We then see that \(X = \bigcup_{i} X_{i} \).\todo{Check.}\footnote{Hatcher\cite{hatcher2002algebraic} do this by arguing the union on the right is both open and closed.}
	Now, let \(Y_0 = X_0\). By induction, we'll assume that \(Y_{i}\) is a \hyperref[def:subgraph]{subgraph} of \(X_{i}\) such that
	\begin{itemize}
		\item \(Y_{i}\) contains all vertices of \(X_{i}\).
		\item \(Y_{i}\) \hyperref[def:deformation-retraction]{deformation retracts} to \(Y_{i-1}\).
	\end{itemize}
	We can then construct \(Y_{i+1}\) by taking \(Y_{i}\) and adding to it one edge to adjoin every vertex of \(X_{i+1}\), namely
	\[
		Y_{i+1} \coloneqq  Y_{i}\bigcup \text{ one edge to adjoint every vertex of \(X_{i}\)},
	\]
	which is possible if we assume Axiom of Choice.

	We then see that \(Y_{i+1}\) \hyperref[def:deformation-retraction]{deformation retracts} to \(Y_{i}\) by just smashing down each edge. Now, we can show that
	\(Y\) \hyperref[def:deformation-retraction]{deformation retracts} to \(Y_0 = X_0\) by performing the \hyperref[def:deformation-retraction]{deformation retraction}
	from \(Y_{i}\) to \(Y_{i-1}\) during the time interval \([1/2^i, 1/2^{i-1}]\).
\end{proof}

\begin{eg}[Fundamental group of \(S^n\)]
	Let
	\begin{itemize}
		\item \(S^n\): decompose into \(2\) open disks
		\item \(A_1\): neighborhood of top hemisphere
		\item \(A_2\): neighborhood of lower hemisphere
	\end{itemize}
	We can then apply \autoref{thm:Seifert-Van-Kampen-Theorem} and conclude that \(\pi _1(S^n) = 0\).
\end{eg}
\begin{explanation}
	We see that \(A_1 \cap A_2\simeq S^{n-1}\), where we need \(n\geq 2\) to let \(S^{n-1}\) be connected. We then have
	\[
		\pi _1(S^n)\cong 0\underset{\pi _1(A_1 \cap A_2)}{\ast}0 = 0.
	\]

	On the other hand, if \(n\geq 3\), then we see that
	\[
		S^n = \quotient{D^{n} \cup \ast}{\sim} .
	\]
	Since \hyperref[def:skeleton]{\(2\)-skeleton} is a point, thus \(\pi _1(S^n) = 0\).
\end{explanation}
