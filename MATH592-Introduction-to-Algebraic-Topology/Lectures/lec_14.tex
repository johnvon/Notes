\chapter{Covering Spaces}
\lecture{14}{7 Feb. 10:00}{Covering Spaces Theory}
\section{Lifting Properties}\todo{Lack of content... Things are in Hw.}
As always, we start with a definition.
\begin{definition}[Covering space]\label{def:covering-space}
	A \emph{covering space} \(\widetilde{X} \) of \(X\) is a space \(\widetilde{X} \) and a map \(p\colon \widetilde{X} \to X\)
	such that \(\forall x\in X\ \exists \text{ neighborhood } u_{x}\) with \(p^{-1} (u_{x})\) the disjoint union of open sets
	\(\coprod_\alpha u_\alpha\) such that
	\[
		\at{p}{u_\alpha }{} \colon u_{x}\to u_{x}
	\]
	is a homeomorphism for every \(\alpha \).
	\begin{figure}[H]
		\centering
		\incfig{def:covering-space}
		\label{fig:def:covering-space}
	\end{figure}

	\begin{definition}[Covering map]\label{def:covering-map}
		We sometimes call \(p\) as the \emph{covering map}.
	\end{definition}
\end{definition}

Although we already investigate into \hyperref[def:covering-space]{covering spaces} quite a lot in homework, but some terminologies are still
worth mentioning.
\begin{definition}[Evenly covered]\label{def:evenly-covered}
	Let \(p\colon \widetilde{X} \to X\) be a continuous map of spaces. Then an open subset \(U\subseteq X\) is called \emph{evenly covered by \(p\) }
	if
	\[
		\at{p}{V_{i}}{} \colon V_{i}\to U
	\]
	is a homeomorphism.
\end{definition}
\begin{definition}[Slice]\label{def:slice}
	Given a \hyperref[def:covering-space]{covering space} \(\widetilde{X} \) and the relating map \(p\), we call the parts \(V_{i}\) of the
	partition \(\coprod_i V_{i}\) of \(p^{-1} (U)\) \emph{slices}.
\end{definition}
\begin{remark}
	We see that \(p\) is a \hyperref[def:covering-map]{covering map} if and only if every point \(x\in X\) has a neighborhood which is
	\hyperref[def:evenly-covered]{evenly covered}.
\end{remark}

We immediately have the following proposition.
\begin{proposition}[Homotopy lifting property]\label{prop:homotopy-lifting-property}
	The \hyperref[def:covering-space]{covering spaces} satisfy the \emph{\hyperref[def:homotopy]{homotopy} lifting property} such that
	the following diagram commutes.
	\[
		\begin{tikzcd}
			{X\times \{0\}} && {\widetilde{Y} } \\
			\\
			{X\times I} && Y
			\arrow["{F_t}"', from=3-1, to=3-3]
			\arrow[hook, from=1-1, to=3-1]
			\arrow["{\widetilde{F}_0 }", from=1-1, to=1-3]
			\arrow["p", from=1-3, to=3-3]
			\arrow["{\exists ! \widetilde{F}_t }", from=3-1, to=1-3]
		\end{tikzcd}
	\]
\end{proposition}
\begin{proof}
	We already proved this in homework!
\end{proof}

\begin{definition}[Lift]\label{def:lift}
	We call \(\widetilde{F} _t\) the \emph{lift} of \(F_t\) in \autoref{prop:homotopy-lifting-property}.
\end{definition}

\begin{corollary}[Path lifting property]\label{col:path-lifting-property}
	For each \hyperref[def:path]{path} \(\gamma \colon I\to X\) in \(X\), \(\widetilde{x} _0\in p^{-1} (\gamma (0)) \) such that there exists a
	unique \hyperref[def:lift]{lift} \(\widetilde{\gamma} \) starting at \(\widetilde{x} _0\).

	\par And for each \hyperref[def:homotopy-path]{path homotopy} \(I\times I\to X\), there exists a unique \hyperref[def:homotopy-path]{path homotopy}
	\(\widetilde{\gamma} \colon I\times I\to \widetilde{X} \) starting at \(\widetilde{x}_0\).
	\begin{figure}[H]
		\centering
		\incfig{col:lec14:1}
		\label{fig:col:lec14:1}
	\end{figure}
\end{corollary}
\begin{proof}
	We prove them separately.
	\begin{note}
		Though we can directly use \autoref{prop:homotopy-lifting-property} to prove this, but we can see some insight by directly proving this.
	\end{note}

	\paragraph{\hyperref[def:lift]{Lifting} a \hyperref[def:path]{path}.} Assume that we have the following \hyperref[def:lift]{lift}.
	\begin{figure}[H]
		\centering
		\incfig{pf:col:lec14}
		\label{fig:pf:col:lec14}
	\end{figure}
	We first prove that a \hyperref[def:path]{path} will be \hyperref[prop:homotopy-lifting-property]{lifted} uniquely to a \hyperref[def:path]{path} \(\widetilde{\gamma} \)
	from \(\widetilde{x} _0\). For every \(x\in X\), there exists an open neighborhood \(U_x\) such that
	\[
		p^{-1} (U_{x} ) = \coprod\limits_{\alpha }U_{x_\alpha },
	\]
	where for every \(\alpha \),
	\[
		\at{p}{U_{x_\alpha }}{} \colon U_{x_{\alpha }}\to U_{x}
	\]
	is a homeomorphism. We see that \(\left\{U_{x} \mid x\in X\right\}\) is an open cover of \(X\), hence
	\[
		\left\{\gamma ^{-1} (U_{x} )\mid x\in X\right\}
	\]
	is an open cover of \([0, 1]\). Note that since \([0, 1]\) is a compact metric space, from Lebesgue Lemma\footnote{\url{https://en.wikipedia.org/wiki/Lebesgue\%27s_number_lemma}},
	there exists a partition of \([0, 1]\) such that
	\[
		0 = t_0 < t_1 < \ldots  < t_{k} = 1
	\]
	such that for every \(i\), \([t_{i} , t_{i+1}]\subset \gamma ^{-1} (U_{x} ) \) for some \(x\). Without loss of generality, we assume that \([t_{i} , t_{i+1}]\subset \gamma ^{-1} (U_{x_{i}})\),
	i.e.,
	\[
		\gamma ([t_{i} , t_{i+1}])\subset U_{x_{i} }.
	\]
	\begin{figure}[H]
		\centering
		\incfig{pf:col:lec14-2}
		\label{fig:pf:col:lec14-2}
	\end{figure}
	Now, since \(p(\widetilde{x} _0) = \gamma (0)\) for \(\gamma _0\in U_{x_1}\) and \(\widetilde{x} _0\in p^{-1} (U_{x_1})\), we may assume \(\widetilde{x} _0\in U_{x_1 \alpha _1}\).
	Consider \hyperref[prop:homotopy-lifting-property]{lifting} the first segment, namely \(\gamma ([0, t_1])\).
	\begin{figure}[H]
		\centering
		\incfig{pf:col:lec14-3}
		\label{fig:pf:col:lec14-3}
	\end{figure}
	Specifically, let \(\widetilde{\gamma}_1 (t) = \left(\at{p}{U_{x_1 \alpha _1}}{}\right)^{-1} \circ \gamma (t) \) for \(0\leq t\leq t_1\), we see that
	\[
		\widetilde{\gamma} _1\colon [0, t_1]\to \widetilde{X}
	\]
	is a \hyperref[def:lift]{lift} of \(\at{\gamma }{[0, t_1]}{} \) from \(\widetilde{x} _0\). We claim that this \hyperref[def:lift]{lift}
	is unique. Consider there exists another \hyperref[def:lift]{lift} from \(\widetilde{x} _0\) \(\widetilde{\widetilde{\gamma}}_1\colon [0, t_1]\to \widetilde{X}\), then since
	\begin{itemize}
		\item \(\widetilde{\widetilde{\gamma}}_1(0) = \widetilde{x} _0\)
		\item \(\widetilde{\widetilde{\gamma}}_1\) is continuous
		\item \(\widetilde{x} _0\in U_{x_1 \alpha _1}\),
	\end{itemize}
	we see that \(\widetilde{\widetilde{\gamma}}_1\left(0, t_1\right)\subset U_{x_1 \alpha _1}\), which implies
	\[
		\begin{tikzcd}
			{[0, t_1]} & {U_{x_1\alpha_1}} \\
			& {U_{x_1}}
			\arrow["{\at{p}{U_{x_1\alpha_1}}{}}", from=1-2, to=2-2]
			\arrow["{\at{\gamma}{[0, t_1]}{}}"', from=1-1, to=2-2]
			\arrow["{\widetilde{\widetilde{\gamma}}_1}", from=1-1, to=1-2]
		\end{tikzcd}
		\implies \widetilde{\widetilde{\gamma}}_1 = \left(\at{p}{U_{x_1 \alpha _1}}{} \right)^{-1} \circ \at{\gamma }{[0, t_1]}{} = \widetilde{\gamma} _1,
	\]
	hence this \hyperref[def:lift]{lift} is unique. Now, we see that we can simply repeat this argument, namely replacing \(t_{i} \) by \(t_{i+1} \),
	\(\widetilde{\gamma}_i (t_{i} )\) by \(\widetilde{\gamma}_{i+1}  (t_{i+1} )\) and so on. Since this partition is finite, hence in finitely many steps, we obtain a unique
	\hyperref[def:homotopy-path]{path homotopy} \(\widetilde{\gamma} \) by concatenating all \(\widetilde{\gamma} _{i} \) starting at \(\widetilde{x} _0\).

	\paragraph{\hyperref[def:lift]{Lifting} a \hyperref[def:homotopy-path]{path homotopy}.} We now consider lifting a \hyperref[def:homotopy-path]{path homotopy}. Consider
	\[
		\gamma _1\underset{F}{\simeq } \gamma _2 \ \mathrm{rel} \{0, 1\}
	\]
	we'll show that \(\widetilde{\gamma}_1\underset{\widetilde{F} }{\simeq} \widetilde{\gamma}_2 \ \mathrm{rel} \{0, 1\}\) where \(p\circ \widetilde{F} = F\). Firstly,
	we denote \(x_0\coloneqq\gamma _1(0) = \gamma _2(0)\), such that
	\begin{figure}[H]
		\centering
		\incfig{pf:col:lec14-4}
		\label{fig:pf:col:lec14-4}
	\end{figure}
	We claim that it's sufficient to show that there exists a continuous \(\widetilde{F} \colon I\times I\to X\) such that \(p\circ \widetilde{F} = F\), and \(\widetilde{F} (\{0\}\times I)= x_0\).
	It's because
	\[
		p\circ \widetilde{F} _0 = F_0 = \gamma _1,\quad p\circ \widetilde{F} _1= F_1= \gamma _2
	\]
	where \(\widetilde{F} _0, \widetilde{F} _1\) is \(\gamma _1, \gamma _2\)'s \hyperref[prop:homotopy-lifting-property]{lifting} starting at \(\widetilde{x} _0\), respectively. And since
	\(p\circ\widetilde{F} = F\), we have
	\[
		p\left(\widetilde{F} (\{1\}\times I)\right) = x_1 \implies \widetilde{F} (\{1\}\times I)\subset p^{-1} (\{x_1\}),
	\]
	which implies \(\exists \widetilde{x} _1 \in p^{-1} (\{x_1\})\) such that \(\widetilde{F} (\{1\}\times I) = \widetilde{x} _1\) since we know that \(p^{-1} (\{x_1\})\) is a discrete
	points-set and \(\widetilde{F} \) is assumed to be continuous, and \(\{1\}\times I\) is connected. We now show \(\widetilde{F} \) exists.

	We define
	\[
		\begin{split}
			\widetilde{F} \colon I\times I &\to X\\
			(s, t)&\mapsto \widetilde{F} _{t} (s),
		\end{split}
	\]
	where \(\widetilde{F} _{t} \colon [0, 1]\to \widetilde{X}\) is a \hyperref[def:lift]{lift} starting at \(\widetilde{x} _0\) of \(F_{t} \colon [0, 1]\to X, s\mapsto F(s, t)\).
	Obviously, \(p\circ \widetilde{F} = F\) from the uniqueness of the \hyperref[def:lift]{lift} of a path, and also, \(\widetilde{F} (\{0\}\times I) = \widetilde{x} _0\)
	holds trivially, hence we only need to show \(\widetilde{F} \) is continuous.
	\begin{enumerate}[(1)]
		\item We show that \(\exists \epsilon _0>0\) such that \(\at{\widetilde{F} }{[0, \epsilon _0]\times I}{} \) is continuous.
		      \begin{figure}[H]
			      \centering
			      \incfig{pf:col:lec14-5}
			      \label{fig:pf:col:lec14-5}
		      \end{figure}
		      Since \(F\) is continuous, we see that there exists an open neighborhood \(U_{x_0}\) of \(x_0\) such that \(p^{-1} (U_{x_0})= \coprod_\alpha U_{x_0 \alpha } \), where
		      \[
			      \at{p}{U_{x_0 \alpha }}{} \colon U_{x_0 \alpha }\overset{\cong }{\to } U_{x_0}.
		      \]
		      Since \(F^{-1} (U_{x_0})\) is an open set contain \(\{0\}\times I\), there exists a \(\epsilon _0>0\) such that \([0, \epsilon _0]\times I\subset F^{-1} (U_{x_0})\),\footnote{Notice that we're working on product topology here.}
		      which implies
		      \[
			      F\left([0, \epsilon _0]\times I\right)\subset U_{x_0}.
		      \]
		      Note that \(x_0\in U_{x_0}\) and \(p(\widetilde{x} _0) = x_0\), we may assume \(\widetilde{x} _0\in U_{x_0 \alpha _1}\). Consider \(\left(\at{p}{U_{x_0 \alpha _1}}{} \right)^{-1} \circ \at{F}{[0, \epsilon _0]\times I}{} \),
		      which is a \hyperref[def:lift]{lift} of \(\at{F}{[0, \epsilon _0]\times I}{} \). We claim that
		      \[
			      \left(\at{p}{U_{x_0 \alpha _1}}{} \right)^{-1} \circ \at{F}{[0, \epsilon _0]\times I}{} = \at{\widetilde{F} }{[0, \epsilon _0]\times I}{}.
		      \]
		      This is because for every \(t\in I\),
		      \[
			      s\mapsto \left(\at{p}{U_{x_0 \alpha _1}}{} \right)^{-1} \circ \at{F}{[0, \epsilon _0]\times I}{}(s, t)
		      \]
		      is a \hyperref[def:lift]{lift} starting at \(\widetilde{x} _0\); also, for every \(t\in I\),
		      \[
			      s\mapsto \at{\widetilde{F}}{[0, \epsilon _0]\times I}{}(s, t)
		      \]
		      is a \hyperref[def:lift]{lift} of \(F_{t} \) starting at \(\widetilde{x} _0\). From the uniqueness of the
		      \hyperref[def:lift]{lift} of \hyperref[def:path]{paths}, we see that they're equal. Note that this implies
		      \(\widetilde{F} \) is now continuous at \([0, \epsilon _0]\times I\), since \(F\) is continuous and \(\at{p}{U_{x_0 \alpha _1}}{}\) is a homeomorphism, hence continuous, then from
		      \[
			      \at{\widetilde{F} }{[0, \epsilon _0]\times I}{}
			      = \underbrace{\left(\at{p}{U_{x_0 \alpha _1}}{} \right)^{-1}}_{\text{continuous} }
			      \circ \underbrace{\vphantom{\left(\at{p}{U_{x_0 \alpha _1}}{} \right)^{-1}}\at{F}{[0, \epsilon _0]\times I}{}}_{\text{continuous}},
		      \]
		      we see that \(\widetilde{F} \) is indeed continuous at \([0, \epsilon _0]\times I\).
		\item We now prove that \(\widetilde{F} \colon I\times I\to \widetilde{X} \) is continuous. Assume there exists \((s_0, t_0)\in I\times I\) such that \(\widetilde{F} \) is
		      discontinuous at \((s_0, t_0)\). Then consider
		      \[
			      0 < \epsilon _0\leq \inf \underbrace{\left\{s\mid \widetilde{F} \text{ is discontinuous at \(s, t_0\) } \right\}}_{\ni s_0 \implies \neq \varnothing} \eqqcolon s_1,
		      \]
		      where the first inequality is from the first step.
		      \begin{figure}[H]
			      \centering
			      \incfig{pf:col:lec14-6}
			      \label{fig:pf:col:lec14-6}
		      \end{figure}
		      Let \(x_1\coloneqq F(s_1, t_0)\), \(\widetilde{x} _1\coloneqq \widetilde{F} (s_1, t_0)\), then there exists an open neighborhood \(U_{x_1}\) in \(X\) such that
		      \(x_1\in U_{x_1} = \coprod_\alpha U_{x_1 \alpha }\), where
		      \[
			      \at{p}{U_{x_1 \alpha }}{}\colon U_{x_1 \alpha }\overset{\cong }{\to }U_{x_1}.
		      \]
		      Since \(F\) is continuous, there exists an \(\epsilon _1> 0\), \(\delta _1> 0\) such that
		      \[
			      F\left((s_1 - \epsilon _1, s_1 + \epsilon _1) \times (t_{0}-\delta _1, t_0 + \delta _1 )\right) \subset U_{x_1}.
		      \]
		      Notice that here we're considering \textbf{open} box.
		      \begin{figure}[H]
			      \centering
			      \incfig{pf:col:lec14-7}
			      \label{fig:pf:col:lec14-7}
		      \end{figure}
		      We may assume \(\widetilde{x} _1\in U_{x_1 \alpha _1}\). Then, we see that \(\widetilde{F} _{t_0}\) is a \hyperref[def:lift]{lift} of
		      \(F_{t_0}\), which means \(\widetilde{F} _{t_0}\) is continuous, hence there exists an \(s_{2} \) such that \(s_{1}-\epsilon _1<s_{2}< s_1  \) such that
		      \[
			      \widetilde{F} (s_2, t_0)\in U_{x_1 \alpha _1}.
		      \]
		      \begin{figure}[H]
			      \centering
			      \incfig{pf:col:lec14-8}
			      \label{fig:pf:col:lec14-8}
		      \end{figure}
		      We see that \(\widetilde{F} \) is continuous at \((s_2, t_0)\), hence there exists a \(\delta _2>0\) such that
		      \[
			      \widetilde{F} \left(\{s_2\}\times (t_0 - \delta _2, t_0 + \delta_2)\right)\subset U_{x_1\alpha _1}.
		      \]
		      Note that here we can also consider a closed interval, which matches what we're going to do. Namely, we're going to construct a
		      \textbf{closed} box \(B\). But this is just a technical detail.
		      \begin{figure}[H]
			      \centering
			      \incfig{pf:col:lec14-9}
			      \label{fig:pf:col:lec14-9}
		      \end{figure}
		      Now, observe that \(\widetilde{F} (B)\subset U_{x_1 \alpha _1}\). To see this, consider a fixed \(t\in (t_0 + \delta _2, t_0 - \delta _2)\), then the map \(\widetilde{F} \) is
		      \[
			      [s_{1}-\epsilon _1, s_{1}+\epsilon _1]\to \widetilde{X} ,\quad s\mapsto \widetilde{F} (s, t) = \widetilde{F} _{t} (s).
		      \]
		      Specifically,
		      \[
			      \widetilde{F} _{t} ([s_{1}-\epsilon _1, s_{1}+\epsilon _1]) \subset p^{-1} (U_{x_1}) = \coprod\limits_{\alpha}U_{x_1 \alpha },
		      \]
		      with the fact that \(\widetilde{F} _{t} ([s_{1}-\epsilon _1, s_{1}+\epsilon _1])\) is connected, and \(\widetilde{F} _{t} (s_2)\in U_{x_1 \alpha _1}\) with
		      \(\widetilde{F} _t\) is a \hyperref[def:lift]{lift} of \(F_{t} \), hence continuous, so
		      \[
			      \widetilde{F} _{t} ([s_{1}-\epsilon _1, s_{1}+\epsilon _1])\subset U_{x_1 \alpha _1}.
		      \]
		      This is true for every \(t\in [t_0-\delta _2, t_0 + \delta _2]\), hence \(\at{\widetilde{F} }{B}{} \subset U_{x_1 \alpha _1}\). Now, since
		      \[
			      \at{p}{U_{x_1 \alpha _1}}{}\circ \at{\widetilde{F} }{B}{} = \at{F}{B}{},
		      \]
		      and
		      \[
			      \left(\at{p}{U_{x_1 \alpha _1}}{} \right)^{-1} \circ \at{F}{B}{} \colon B\to U_{x_1 \alpha _1},
		      \]
		      so
		      \[
			      \at{p}{U_{x_1 \alpha _1}}{} \circ \left(\left(\at{p}{U_{x_1 \alpha _1}}{} \right)^{-1} \circ \at{F}{B}{}\right) = \at{F}{B}{}
		      \]
		      obviously. Since \(\at{p}{U_{x_1 \alpha _1}}{}\) is a homeomorphism, we have
		      \[
			      \at{\widetilde{F} }{B}{} = \underbrace{\left(\at{p}{U_{x_1 \alpha _1}}{} \right)^{-1}}_{\text{continuous} } \circ \underbrace{\vphantom{\left(\at{p}{U_{x_1 \alpha _1}}{} \right)^{-1}}\at{F}{B}{}}_{\text{continuous} },
		      \]
		      hence we have \(\at{\widetilde{F} }{B}{} \) is continuous, which leads to a contradiction since
		      \[
			      s_1 = \inf \left\{s\mid \widetilde{F} \text{ is discontinuous at \(s, t_0\) } \right\},
		      \]
		      while \(\widetilde{F} \) is continuous for all \(B\), hence we see that \(\widetilde{F} :I\times I\to \widetilde{X} \) is continuous.\footnote{There is a tricky situation, namely while \(s_1 = 1\). But this can be considered also.}
	\end{enumerate}
\end{proof}

\begin{eg}[Covers of \(S^1\vee S^1\)]
	We have the following \hyperref[def:covering-map]{covers} of \(S^1 \vee S^1\).
	\begin{figure}[H]
		\centering
		\incfig{eg:lec14:1}
		\label{fig:eg:lec14:1}
	\end{figure}
	Note that in each cover (those three on the top), the black dot is the preimage of \(\{x_0\}\), namely \(p_{i}^{-1} (\{x_0\})\).
	\begin{remark}
		We see that for each \(p_{i}^{-1} (\{x_0\})\), there are exactly
		\begin{itemize}
			\item one \(a\) edge goes out
			\item one \(b\) edge goes out
			\item one \(a\) edge goes in
			\item one \(b\) edge goes in
		\end{itemize}

		It turns out that there are much more \hyperref[def:covering-map]{covers} of \(S^1\vee S^1\), as long as this main property is satisfied.
	\end{remark}
\end{eg}

\begin{proposition}
	Let \(p\colon (\widetilde{X} , \widetilde{x} _0)\to (X, x_0)\) be a \hyperref[def:covering-map]{covering map}. Then
	\begin{enumerate}[(1)]
		\item \(p_*\colon \pi _1(\widetilde{X} , \widetilde{x} _0)\to \pi _1(X, x_0)\) is injective.
		\item \(p_*(\pi _1(\widetilde{X} , \widetilde{x} _0))\subseteq \pi _1(X, x_0)\), which picks out the subset
		      \[
			      \left\{[\gamma ] \mid \text{\hyperref[def:lift]{lift} \(\widetilde{\gamma} \) starting at \(\widetilde{x} _0\) is a loop.} \right\}.
		      \]
	\end{enumerate}
\end{proposition}
\begin{proof}
	We prove this one by one.
	\begin{enumerate}[(1)]
		\item Suppose \(\widetilde{\gamma} \in \pi _1(\widetilde{X} , \widetilde{x} _0)\) is in \(\ker (p_*) \). Then
		      \[
			      [\gamma] =p_*([\widetilde{\gamma} ]) = \left[p \circ \widetilde{\gamma} \right].
		      \]
		      Let \(\gamma _{t}\) be a \hyperref[def:nullhomotopic]{nullhomotopy} from \(\gamma \) to the \hyperref[not:constant-loop]{constant loop}
		      \(c_{x_0}\ \mathrm{rel} \{0, 1\}\).
		      We can then \hyperref[def:lift]{lift} \(\gamma _{t}\) to \(\widetilde{\gamma} _{t}\) where \(\widetilde{\gamma} _0 = \widetilde{\gamma} \).
		      Now, we claim that
		      \begin{itemize}
			      \item \(\widetilde{\gamma} \) is a \hyperref[def:homotopy-relative]{homotopy \(\mathrm{rel} \{0, 1\}\)}.
			      \item \(\widetilde{\gamma} _1\) is the \hyperref[not:constant-loop]{constant loop} \(c_{\widetilde{x} _0}\).
		      \end{itemize}
		      \[
			      \begin{tikzcd}
				      & {\widetilde{X}} \\
				      I & X
				      \arrow["{\widetilde{\gamma}}", from=2-1, to=1-2]
				      \arrow["p", from=1-2, to=2-2]
				      \arrow["\gamma"', from=2-1, to=2-2]
			      \end{tikzcd}\qquad
			      \begin{tikzcd}
				      & {\widetilde{X}} \\
				      {I\times I} & X
				      \arrow["{\widetilde{\gamma}_t}", from=2-1, to=1-2]
				      \arrow["p", from=1-2, to=2-2]
				      \arrow["{\gamma_t}"', from=2-1, to=2-2]
			      \end{tikzcd}
		      \]
		      We see that the above diagrams prove the first claim, since we know that the left and right edge of \(I\times I\) maps to \(x_0\) under
		      \(\gamma _{t}\), and \(c_{\widetilde{x} _0}\) \hyperref[prop:homotopy-lifting-property]{lifts} this, so by uniqueness \(t\mapsto \widetilde{\gamma} _t(0)\) and
		      \(t\mapsto \widetilde{\gamma} _{t}(1)\) must be constant \hyperref[def:path]{paths} at \(\widetilde{x} _0\) as desired.

		      \par Then the \hyperref[def:lift]{lift} \(\widetilde{\gamma} _{t}\) is a \hyperref[def:homotopy-path]{homotopy of paths}
		      to the constant loop, so \([\widetilde{\gamma} ] = 1\).
		\item Let see an example to show the idea of the proof.
		      \begin{eg}
			      Given
			      \begin{figure}[H]
				      \centering
				      \incfig{eg:lec14:2}
				      \label{fig:eg:lec14:2}
			      \end{figure}
			      Then
			      \[
				      p_*\pi _1 = \left< b, a^{2} , ab \overline{a}  \right> \subseteq \pi _1(X) = \left< a, b \mid \ \right>.
			      \]
		      \end{eg}
	\end{enumerate}
\end{proof}
\begin{proposition}[Lifting criterion]\label{prop:lifting-criterion}
	Let \(p\colon (\widetilde{Y} , \widetilde{y} _0)\to (Y, y_0 )\) be a \hyperref[def:covering-map]{covering map}. Given
	\begin{itemize}
		\item \(f\colon (X, x_0) \to (Y, y_0)\);
		\item \(X\) is \hyperref[def:path]{path}-connected, locally \hyperref[def:path]{path}-connected,
	\end{itemize}
	then a \hyperref[def:lift]{lift}
	\[
		\widetilde{f} \colon (X, x_0)\to (\widetilde{Y} , y_0)
	\]
	exists if and only if
	\[
		f_*\left(\pi _1(X, x_0)\right)\subseteq p_*(\pi _1(\widetilde{Y} , \widetilde{y} _0 )).
	\]
	In diagram, we have
	\[
		\begin{tikzcd}
			& {(\widetilde{Y}, \widetilde{y}_0)} \\
			{(X, x_0)} & {(Y, y_0)}
			\arrow["p", from=1-2, to=2-2]
			\arrow["f"', from=2-1, to=2-2]
			\arrow["{\exists \widetilde{f}}", dashed, from=2-1, to=1-2]
		\end{tikzcd}\qquad
		\begin{tikzcd}
			& {\pi_1(\widetilde{Y}, \widetilde{y}_0)} \\
			{\pi_1(X, x_0)} & {\pi_1(Y, y_0)}
			\arrow["{f_\ast}"', from=2-1, to=2-2]
			\arrow["{p_\ast}", from=1-2, to=2-2]
			\arrow["{\widetilde{f}_\ast}", from=2-1, to=1-2]
		\end{tikzcd}
	\]
\end{proposition}