\lecture{40}{15 Apr. 10:00}{Prepare for Final}
\section{Review Problems}
We now go through some problems to get prepared for the final!
\begin{exercise}[QR May 2019]
	\(T = S^1 \times S^1\), \(T^\prime = \quotient{T}{S \times \{1\}} \). Find \(H_\ast\), \(\pi _1\) of \(T \# T^\prime \) (connected sum).
\end{exercise}
\begin{answer}
	We see that
	\begin{figure}[H]
		\centering
		\incfig{ans:ex-1:lec-40}
		\label{fig:ans:ex-1:lec-40}
	\end{figure}
	We then see that \(T\# T^\prime \) is a \hyperref[CW-complex-wedge-sum]{wedge} of \(T\) and \(S^1\), hence the fundamental group is just
	\(\mathbb{\MakeUppercase{z}} ^{2} \ast \mathbb{\MakeUppercase{z}} \), where \(\mathbb{\MakeUppercase{z}} ^{2} \) is from \(T\) and \(\mathbb{\MakeUppercase{z}} \)
	is from \(S^{1} \).
\end{answer}

\begin{exercise}[QR Aug. 2019]
	Let \(X\) be a \hyperref[def:CW-Complex]{CW complex} obtain from a \(k\)-sphere, \(k\geq 1\), by attaching two \hyperref[def:cell]{\((k+1)\)-cells}
	along \hyperref[def:attaching-map]{attaching maps} of \(\deg m, n\). Calculate \(H_\ast (X)\).
\end{exercise}
\begin{answer}
	We use \hyperref[def:cellular-homology-group]{cellular homology}. Specifically, we have the following.
	% file:///Users/pbb/Developer/quiver/src/index.html?q=WzAsNixbMCwwLCJcXG1hdGhiYntafV4yPUNfe2srMX0iXSxbMiwxLCJtIl0sWzIsMiwibiJdLFswLDEsIigxLCAwKSJdLFswLDIsIigwLCAxKSJdLFsyLDAsIlxcbWF0aGJie1p9PUNfayJdLFszLDEsIiIsMCx7InN0eWxlIjp7InRhaWwiOnsibmFtZSI6Im1hcHMgdG8ifX19XSxbNCwyLCIiLDAseyJzdHlsZSI6eyJ0YWlsIjp7Im5hbWUiOiJtYXBzIHRvIn19fV0sWzAsNSwiXFxwYXJ0aWFsX3trKzF9Il1d
	\[\begin{tikzcd}[row sep=tiny]
			{\mathbb{Z}^2=C_{k+1}} && {\mathbb{Z}=C_k} \\
			{(1, 0)} && m \\
			{(0, 1)} && n
			\arrow[maps to, from=2-1, to=2-3]
			\arrow[maps to, from=3-1, to=3-3]
			\arrow["{\partial_{k+1}}", from=1-1, to=1-3]
		\end{tikzcd}\]
	Hence, we see that \(\ker(\partial _{k+1}) \cong \mathbb{\MakeUppercase{z}} \), and \(\im(\partial _{k+1}) = \gcd(m, n)\mathbb{\MakeUppercase{z}} \).
\end{answer}

\begin{exercise}[QR May 2017]
	Let \(S^1\) be the complex numbers of absolute value \(1\) with induced topology. Let \(K\) be the quotient of \(S^1 \times [0, 1]\) by identifying
	\((z, 0)\) with \((z^{-2}, 1 )\). Compute \(H_\ast (K)\).
\end{exercise}
\begin{answer}
	Again, we use \hyperref[def:cellular-homology-group]{cellular homology}. We use the following gluing instruction.
	\begin{figure}[H]
		\centering
		\incfig{ans:ex-3:lec-40}
		\label{fig:ans:ex-3:lec-40}
	\end{figure}
\end{answer}

\begin{exercise}[QR Sep. 2016]
	Let \(Z= \{(x, y)\in \mathbb{\MakeUppercase{c}} ^{2} \mid x = 0 \text{ or } y = 0\}\). Find \(H_\ast (\mathbb{\MakeUppercase{c}} ^{2} \setminus Z)\).
\end{exercise}
\begin{answer}
	We see that
	\[
		(\mathbb{\MakeUppercase{c}} \setminus \{0\})\times (\mathbb{\MakeUppercase{c}} \setminus \{0\}) \simeq S^1 \times S^1,
	\]
	which is a torus.
\end{answer}

\begin{exercise}[QR May 2017]
	Let \(X\) be the connected \hyperref[def:CW-Complex]{CW complex}, \(H_i(X) = 0\) for all \(i > 0\). Prove that
	\[
		H_n(X \times S^k) = \begin{dcases}
			\mathbb{\MakeUppercase{z}}, & \text{ if } n = 0; \\
			0,                          & \text{ otherwise}.
		\end{dcases}
	\]
\end{exercise}
\begin{answer}
	We can induct on \(k\) and use Mayer-Vietoris long sequence. Let \(U = X \times \text{ upper hemisphere}\), \(V = X \times \text{lower hemisphere}\), then we have
	\[
		U\simeq X,\quad V\simeq X,
	\]
	and hence \(U \cap V \simeq X \times S^{k-1}\).
\end{answer}