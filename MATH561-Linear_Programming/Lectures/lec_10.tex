\lecture{10}{4 Oct. 08:00}{Duality}
%────────────────────────────────────────────────────────────────────────────────────────────────────────────────────────────────────────────────────
\subsection{Complementary}
Solutions \(\hat{x}\) to \((P)\) and \(\hat{y}\) to \((D)\) are \emph{complementary} if
\[
	m+n\text{ equations}\begin{cases}
		(\underbrace{c_{j} - \hat{y}^{T} A_{\cdot j}}_{ = 0\text{ for }j\in \beta}) \underbrace{\hat{x}_j}_{ \substack{= 0 \\\text{ for }\\j\in \eta}} = 0, & j = 1\cdots n; \\\\
		\hat{y}_i(\underbrace{A_{i\cdot} \hat{x} - b_{i}}_{ = 0 \text{ for }\overline{x}}) = 0, & i = 1\cdots m.
	\end{cases}
\]

Now, suppose we have a basic partition \(\beta, \eta\) such that
\[
	\begin{split}
		\overline{x}&: \overline{x}_{\beta} = A^{-1}_{\beta}b,\ \overline{x}_{\eta} = \vec{0}\\
		\overline{y}&: \overline{y}^{T} = c^{T}_{\beta}A^{-1}_{\beta}.
	\end{split}
\]

\begin{note}
	Specifically, we see that \(c_{j} - \hat{y}^{T}A_{\cdot j} = 0\) for \(j\in\beta\) is because \(\overline{y}^{T} = c^{T}_{\beta}A^{-1}_{\beta}\), and then
	\[
		c_{j} - \hat{y}^{T}A_{\cdot j} = c_{j} - c^{T}_{\beta}\underbrace{A^{-1}_{\beta}A_{\cdot j}}_{e_{j}} = c_{j} - c_{j} = 0.
	\]
\end{note}

Then just from above, we see that the following theorems hold.
\begin{theorem}
	If \(\overline{x}\) and \(\overline{y}\) are basic solutions for \(\beta, \eta\), then \(\overline{x}\) and \(\overline{y}\) are
	complementary.
\end{theorem}

\begin{theorem}
	\label{complementary with equal objective value}
	If \(\hat{x}\) and \(\hat{y}\) are complementary, then
	\[
		c^{T}\hat{x} = \hat{y}^{T} b.
	\]
\end{theorem}
\begin{note}
	\[
		c^{T}_{\beta} A^{-1}_{\beta}b = \overline{y}^{T} b,\qquad c^{T}(A^{-1}_{\beta} b) = c^{T}_{\beta}\overline{x}_{\beta} = c^{T}\overline{x}.
	\]
\end{note}
\begin{proof}
	We show that
	\[
		c^{T} \hat{x} - \hat{y}^{T} b = 0.
	\]
	We have
	\[
		\begin{split}
			c^{T} \hat{x} - \hat{y}^{T} b &= (c^{T} - \underbrace{\hat{y}^{T}A)\hat{x} + \hat{y}^{T}(A \hat{x}}_{\text{added terms}} - b)\\
			&=\sum\limits_{j=1}^{n} \underbrace{(c_{j}-\hat{y}^{T}A_{\cdot j})x_{j}}_{ = 0\text{ for }i = 1\ldots n}
			+ \sum\limits_{i=1}^{m} \underbrace{\hat{y}_i(A_{i\cdot}\hat{x} - b_{i})}_{ = 0\text{ for }i = 1\ldots m}\\
			&= 0.
		\end{split}
	\]
\end{proof}

\begin{theorem}
	\label{Weak Complementary Slackness}
	Weak Complementary Slackness Theorem: If \(\hat{x}\) and \(\hat{y}\) are feasible and complementary, then they are optimal.
\end{theorem}
\begin{proof}
	Follows from weak duality theorem(\autoref{Weak Duality Theorem}) and complementary solutions having equal objective value(\autoref{complementary with equal objective value}).
\end{proof}

\begin{theorem}
	\label{Strong Complementary Slackness}
	Strong Complementary Slackness Theorem: If \(\hat{x}\) and \(\hat{y}\) are optimal, then \(\hat{x}\) and \(\hat{y}\) are complementary.
\end{theorem}
\begin{proof}
	Recall that
	\[
		\overbrace{\sum\limits_{j=1}^{n} \underbrace{(c_{j}-\hat{y}^{T}A_{\cdot j})}_{ \geq 0\text{ for each }j}\underbrace{\hat{x}_{j}}_{\geq 0 \text{ for each }j}}^{\cancel{\geq 0}\implies = 0\text{ for each }j}
		+ \overbrace{\sum\limits_{i=1}^{m} \underbrace{\hat{y}_i(A_{i\cdot}\hat{x} - b_{i})}_{ = 0\text{ for each }i}}^{=0}
		\underbrace{= 0 = c^{T}\hat{x} - \hat{y}^{T}b}_{\substack{\text{if }\hat{x}\text{ and }\hat{y}\text{ are optimal:}\\ \text{same object value}}}
	\]
	Hence, the equality can only hold if
	\[
		(c_{j} - \hat{y}^{T}A_{\cdot j})\hat{x}_j = 0, \text{ for }j = 1, 2, \ldots , n;
	\]
	with the obvious fact that
	\[
		\hat{y}_i(A_{i\cdot}\hat{x} - b_{i}) = 0, \text{ for }i = 1, 2, \ldots , m,
	\]
	so they are complementary.
\end{proof}

%────────────────────────────────────────────────────────────────────────────────────────────────────────────────────────────────────────────────────
\subsection{Duality for General Linear-Optimization Problems}
So far, we only discuss the dual of the standard form problem. But we will see that \emph{every} linear-optimization problem has a natural dual.

Now consider a general linear programming problem
\begin{align*}
	\min~              & c^{T}_P x_P + c^{T}_N x_N + c^{T}_U x_U          \\
	                   & A_{GP}x_P +A_{GN}x_N + A_{GU}x_U\geq b_G         \\
	                   & A_{LP}x_P +A_{LN}x_N + A_{LU}x_U \leq b_L        \\
	                   & A_{EP}x_P +A_{EN}x_N + A_{EU}x_U = b_E           \\
	(\mathcal{G})\quad & x_P\geq 0, x_N \leq 0, x_U \text{ unrestricted}.
\end{align*}

We first turn this into a standard form problem:
\begin{enumerate}
	\item \(\widetilde{x}_N \coloneqq -x_N\):
	      \begin{align*}
		      \min~ & c^{T}_P x_P + c^{T}_N x_N + c^{T}_u x_U         \\
		            & A_{GP}x_P -A_{GN}x_N + A_{GU}x_U\geq b_G        \\
		            & A_{LP}x_P -A_{LN}x_N + A_{LU}x_U \leq b_L       \\
		            & A_{EP}x_P -A_{EN}x_N + A_{EU}x_U = b_E          \\
		            & x_P\geq 0, x_N \leq 0, x_U \text{ unrestricted}
	      \end{align*}
	\item \(x_U = \widetilde{x}_U - \widetilde{\widetilde{x}}_U\), where \(\widetilde{x}_U, \widetilde{\widetilde{x}}_U \geq 0\):
	      \begin{align*}
		      \min~ & c^{T}_P x_P + c^{T}_N x_N + c^{T}_U \widetilde{x}_U - c_U \widetilde{\widetilde{x}}_U    \\
		            & A_{GP}x_P -A_{GN}x_N + A_{GU}\widetilde{x}_U -A_{GU}\widetilde{\widetilde{x}}_U \geq b_G \\
		            & A_{LP}x_P -A_{LN}x_N + A_{LU}\widetilde{x}_U -A_{LU}\widetilde{\widetilde{x}}_U\leq b_L  \\
		            & A_{EP}x_P -A_{EN}x_N + A_{EU}\widetilde{x}_U -A_{EU}\widetilde{\widetilde{x}}_U= b_E     \\
		            & x_P\geq 0, x_N \leq 0, \widetilde{x}_U \geq 0, \widetilde{\widetilde{x}}_U\geq 0
	      \end{align*}
	\item Adding slack variables:
	      \[
		      \begin{alignedat}{3}
			      \min~ & c^{T}_P x_P + c^{T}_N x_N + c^{T}_U \widetilde{x}_U - c_U \widetilde{\widetilde{x}}_U    \\
			      & A_{GP}x_P -A_{GN}x_N + A_{Gu}\widetilde{x}_U -A_{GU}\widetilde{\widetilde{x}}_U - s_G &       && = b_G \\
			      & A_{LP}x_P -A_{LN}x_N + A_{Lu}\widetilde{x}_U -A_{LU}\widetilde{\widetilde{x}}_U       & + t_L && = b_L \\
			      & A_{EP}x_P -A_{EN}x_N + A_{Eu}\widetilde{x}_U -A_{EU}\widetilde{\widetilde{x}}_U       &       && = b_E \\
			      & x_P\geq 0, x_N \leq 0, \widetilde{x}_U \geq 0, \widetilde{\widetilde{x}}_U\geq 0, s_G\geq 0, t_L \geq 0
		      \end{alignedat}
	      \]
\end{enumerate}

With \emph{Dual variables} \(y_G, y_L, y_E\), we have
\[
	\begin{alignedat}{4}
		\max~ & y^{T}_G b_G &&+ y^{T}_L b_L &&+ y^{T}_E b_E                     \\
		& y^{T}_G A_{GP} &&+ y^{T}_L A_{LP} &&+ y^{T}_E A_{EP} &&\leq c^{T}_P   \\
		-& y^{T}_G A_{GN} &&- y^{T}_L A_{LN} &&- y^{T}_E A_{EN} &&\leq -c^{T}_N \\
		& y^{T}_G A_{GU} &&+ y^{T}_L A_{LU} &&+ y^{T}_E A_{EU} &&\leq c^{T}_U   \\
		-& y^{T}_G A_{GU} &&- y^{T}_L A_{LU} &&- y^{T}_E A_{EU} &&\leq -c^{T}_U \\
		&y^{T}_G \geq 0&&, y^{T}_L\leq 0.
	\end{alignedat}.
\]
We time \(-1\) to the both side of the second constraint, and we see that last two structure constraints can be reduced to a single equality, results in
\[
	\begin{alignedat}{4}
		\max~ & y^{T}_G b_G &&+ y^{T}_L b_L &&+ y^{T}_E b_E                     \\
		& y^{T}_G A_{GP} &&+ y^{T}_L A_{LP} &&+ y^{T}_E A_{EP} &&\leq c^{T}_P   \\
		& y^{T}_G A_{GN} &&+ y^{T}_L A_{LN} &&+ y^{T}_E A_{EN} &&\geq c^{T}_N \\
		& y^{T}_G A_{GU} &&+ y^{T}_L A_{LU} &&+ y^{T}_E A_{EU} && = c^{T}_U     \\
		(\mathcal{H})\quad &y^{T}_G \geq 0&&, y^{T}_L\leq 0.
	\end{alignedat}
\]

Finally, we remark that this gives us a simple result as we have already seen before.
\begin{theorem}
	We rephrase the weak and strong duality theorem in a more general term.
	\begin{itemize}
		\item Weak Duality Theorem: If \((\hat{x}_P, \hat{x}_N, \hat{x}_U)\) is feasible in \(\mathcal{G}\) and the dual variables \((\hat{y}_G, \hat{y}_L, \hat{y}_E)\) is
		      feasible in \(\mathcal{H}\), then
		      \[
			      c^{T}_P \hat{x}_P + c^{T}_N \hat{x}_N + c^{T}_U \hat{x}_U \geq \hat{y}^{T}_G b_G + \hat{y}^{T}_L b_L + \hat{y}^{T}_E b_E.
		      \]
		\item Strong Duality Theorem: If \(\mathcal{G}\) has a feasible solution, and \(\mathcal{G}\) is not unbounded, then there exist feasible
		      solutions \((\hat{x}_P, \hat{x}_N, \hat{x}_U)\) for \(\mathcal{G}\) and \((\hat{y}_G, \hat{y}_L, \hat{y}_E)\) for \(\mathcal{H}\) that are optimal.
		      Moreover,
		      \[
			      c^{T}_P \hat{x}_P + c^{T}_N \hat{x}_N + c^{T}_U \hat{x}_U = \hat{y}^{T}_G b_G + \hat{y}^{T}_L b_L + \hat{y}^{T}_E b_E.
		      \]
	\end{itemize}
\end{theorem}

\begin{remark}
	We can also rephrase the Weak Complementary Slackness Theorem (\autoref{Weak Complementary Slackness}) and also the Strong Complementary Slackness Theorem (\autoref{Strong Complementary Slackness})
	in this setup. The proof follows the same idea, but with some more works.
\end{remark}