\lecture{7}{22 Sep. 08:00}{Simplex Algorithm}
\section{Simplex Algorithm}
\subsection{Simplex Algorithm}
We start with considering the standard form problem
\[
	\begin{alignedat}{5}
		\min~&c^{T}x\qquad\qquad &&\max ~ &&y^{T}b\\
		&Ax = b && &&y^{T}A\leq c^{T}\\
		(P)\quad&x\geq  0 &&(D)\quad&&
	\end{alignedat}.
\]

\begin{definition}
	Define \emph{dual basic solution} \(\overline{y}\in \mathbb{\MakeUppercase{R}}^m\) as
	\[
		\overline{y}^{T} = c_{\beta}^{T} A^{-1}_{\beta}.
	\]
\end{definition}

\begin{lemma}
	If \(\beta, \eta\) is a basic partition, and \(\overline{x}, \overline{y}\) are the associated primal and dual basic solution,
	then
	\[
		c^{T}\overline{x} = \overline{y}^{T} b.
	\]
\end{lemma}
\begin{proof}
	\[
		c^{T}\overline{x} = \begin{pmatrix}
			c^{T}_{\beta} & c^{T}_{\eta} \\
		\end{pmatrix}\begin{pmatrix}
			\overline{x}_{\beta} \\
			\overline{x}_{\eta}  \\
		\end{pmatrix}= c_{\beta}^{T} \overline{x}_{\beta}+ c_{\eta}^{T} \overline{c}_{\eta} = c_{\beta} A^{-1}_{\beta}b = \overline{y}b.
	\]
\end{proof}

\begin{prev}
	\begin{align*}
		\min~ & c^{T}_{\beta}x_{\beta} + c^{T}_{\eta}x_{\eta} \\
		      & A_{\beta}x_{\beta} + A_{\eta}x_{\eta} = b     \\
		      & x_{\beta}\geq 0, x_{\eta}\geq 0
	\end{align*}, and hence
	\begin{align*}
		c^{T}_{\beta}A^{-1}_{\beta}b + \min~ & (c_{\eta}^{T} - c_{\beta}^{T}A^{-1}_{\beta}A_{\eta})x_{\eta} \\
		                                     & A^{-1}_{\beta}A_{\eta}x_{\eta} \leq A^{-1}_{\beta}b          \\
		                                     & x_{\beta}\geq 0, x_{\eta}\geq 0.
	\end{align*}
\end{prev}

We now formalize the concept of \emph{reduced cost}.
\begin{definition}
	\(\overline{c}_{\eta}\) is called \emph{reduced cost} for non-basic variables, where \(\overline{c}_{\eta}\) is
	\[
		\overline{c}_{\eta}^{T} \coloneqq c_{\eta}^{T} - c_{\beta}^{T} A^{-1}_{\beta}A_{\eta} = c_{\eta}^{T} - \overline{y}^{T}A_{\eta}.
	\]
\end{definition}

\subsubsection{Dual Feasibility}
\begin{lemma}
	\(\overline{y}\) is feasible for \(D\) if and only if \(\overline{c}_{\eta}\geq 0\).
\end{lemma}
\begin{proof}
	\[
		y^{T}A\leq c^{T} \iff y^{T}\begin{bmatrix}
			A_{\beta} & A_{\eta} \\
		\end{bmatrix}\leq \begin{pmatrix}
			c_{\beta}^{T} & c_{\eta}^{T} \\
		\end{pmatrix}
	\]
	since
	\[
		\begin{alignedat}{3}
			&y^{T} A_{\beta}\leq c_{\beta}^{T}\\
			&y^{T} A_{\eta}\leq c_{\eta}^{T} \implies c_{\eta}^{T} - y^{T}A_{\eta}\geq 0.
		\end{alignedat}
	\]
\end{proof}

\begin{corollary}
	If \(\hat{x}\) is feasible for \(P\) and \(\hat{y}\) is feasible for \(D\), and if \(c^{T}\hat{x} = \hat{y}^{T} b\), then \(\hat{x}\) and \(\hat{y}\) are optimal.
\end{corollary}

\begin{theorem}[Weak optimal basis theorem]
	Let \(\overline{x}\) and \(\overline{y}\) are basic primal and dual solutions for \(P\) and \(D\). Then if \(\beta\) is a feasible basis and \(\overline{c}_{\eta}\geq 0\), \(\overline{x}\) and
	\(\overline{y}\) are optimal.
\end{theorem}
\begin{proof}
	Obvious from the standard problem in the form of
	\begin{align*}
		c^{T}_{\beta}A^{-1}_{\beta}b + \min~ & (c_{\eta}^{T} - c_{\beta}^{T}A^{-1}_{\beta}A_{\eta})x_{\eta} \\
		                                     & A^{-1}_{\beta}A_{\eta}x_{\eta} \leq A^{-1}_{\beta}b          \\
		                                     & x_{\beta}\geq 0, x_{\eta}\geq 0.
	\end{align*}
\end{proof}

\begin{note}
	The order of the arguments in text book for weak optimal basis theorem is slightly different.
\end{note}

\subsubsection{Naive Algorithmic Approach}
\begin{enumerate}
	\item[0] Start with a basis partition \(\beta, \eta\) with \(\overline{x}_{\beta}\geq 0\).
	\item[1] If \(\overline{c}_{\eta}\geq 0\), then \(\overline{x}\) and \(\overline{y}\) are optimal and \emph{STOP}.
	\item[2] Otherwise, choose \(\eta_j	\) with \(\overline{c}_{\eta_j}<0\). Consider the associated basis direction \(\overline{z}\).(Idea: \(\overline{x}\to \overline{x}+\lambda \overline{z}\) with \(\lambda >0\)) Then
		\[
			c^{T}(\overline{x} + \lambda \overline{z}) = c^{T} \overline{x} + \lambda c^{T} \overline{z} = c^{T} \overline{x} + \lambda \overline{c}_{\eta_j},
		\]
		where \begin{itemize}
			\item \(c^{T} \overline{x}\) is the current objective value
			\item \(c^{T}\overline{z}\) is
			      \[
				      \begin{split}
					      c^{T}\overline{z} &= c_{\eta}^{T} \overline{z}_{\eta}+c_{\beta}^{T} \overline{z}_{\beta}\\
					      &=c_{\eta}^{T}e_{j} - c_{\beta}^{T}(A^{-1}_{\beta}A_{\eta_j})\\
					      &=c_{\eta_{j}} - c_{\beta}^{T} A^{-1}_{\beta}A_{\eta_j}\\
					      &= \overline{c}_{\eta_{j}}
				      \end{split}
			      \]
			\item \(\lambda \overline{c}_{\eta_{j}}\) is the \emph{rate} of change of objective value as we move in direction \(\overline{z}\).
		\end{itemize}
	\item[3] Move from \(\overline{x}\) to \(\overline{x}+\overline{\lambda} \overline{z}\), where we let \(\overline{\lambda}\) as large as possible. Operationally, since we need
		\[
			\overline{x}_{\beta} + \lambda \overline{z}_{\beta} \geq 0,
		\]
		where \(\overline{z}_{\eta} = e_{j},\ \overline{z}_{\beta} = -A^{-1}_{\beta}A_{\eta_{j}}\). We then have
		\[
			\begin{alignedat}{3}
				&\overline{x}_{\beta_{i}} - \lambda \overline{a}_{i, \eta_{j}}\geq 0, &&\text{ for }i = 1, \ldots , m\\
				& \lambda \leq \frac{\overline{x}_{\beta_{i}}}{\overline{a}_{i, \eta_{j}}}, &&\text{ for }i \text{ such that }\overline{a}_{i, \eta_j}>0.
			\end{alignedat}
		\]
		Hence,
		\[
			\overline{\lambda} \coloneqq \min_{i: \overline{a}_{i, \eta_{j}}>0} \left\{ \frac{\overline{x}_{\beta_{i}}}{\overline{a}_{i, \eta_{j}}} \right\} \geq 0.
		\]
		\begin{remark}
			If \(\overline{a}_{i, \eta_{j}}\leq 0\) for all \(i = 1, \ldots , m\), namely
			\[
				\overline{A}_{i, \eta_{j}}\leq 0 \iff -A^{-1}_{\beta}A_{\eta_{j}}\geq 0 \iff \overline{z}\geq 0,
			\]
			then \(\overline{z}\) is a ray. This means \(P\) is unbounded below, hence we \emph{STOP}.
		\end{remark}
\end{enumerate}

\subsubsection{Worry-Free Simplex Algorithm}
Now, we give the very first generation about our simplex algorithm. Consider the standard form problem
\[
	\begin{alignedat}{5}
		\min~&c^{T}x\\
		&Ax = b \\
		(P)\quad&x\geq  0.
	\end{alignedat}
\]
The simplex algorithm is described as follows.
\begin{enumerate}
	\item[0.] Start with a basic feasible partition \(\beta, \eta\) and assume that \(x_{\beta}\geq 0\)(\(x\) is a basic feasible solution)
	\item[1.] \begin{enumerate}
			\item Compute \(\overline{x}_{\beta} \coloneqq A^{-1}_{\beta} b \geq 0\)
			\item Compute \(\overline{c}_{\eta}^{T} \coloneqq c_{\eta}^{T} - c_{\beta}^{T} A^{-1}_{\beta} A_{\eta}\)
		\end{enumerate}
		If \(\overline{c}_{\eta} \geq 0\), then \emph{STOP}. \(\overline{x}\) is optimal for \(P\). \\(Recall that the dual solution \(\overline{y}\coloneqq c_{\beta}^{T}A^{-1}_{\beta}\) is optimal for \(D\) )
	\item[2.] Otherwise, choose \(j\) such that \(1\leq j\leq n-m\) for \(\eta_j\) such that \(\overline{c}_{\eta_j} < 0\). (Basic direction \(\overline{z}\), then \(c^{T}\overline{z} = \overline{c}_{\eta_{j}}<0\))
	\item[3.] Replace \(\overline{x}\) with \(\overline{x} + \lambda \overline{z}\) where
		\[
			\lambda \coloneqq \min_{i:\overline{a}_{i, \eta_{j}}>0}\left\{ \frac{\overline{x}_{\beta_{i}}}{\overline{a}_{i, \eta_{j}}} \right\}.
		\]
		(Largest choice so that \(\overline{x} + \lambda \overline{z} \geq 0\))\\
		If we can't compute this, namely
		\[
			\overline{A}_{\eta_{j}} \leq 0\implies P \text{ is unbounded }\implies \text{ \emph{STOP}}
		\]
		Otherwise, \emph{GOTO} 1.
\end{enumerate}

\begin{problem}
The problem is that is \(\overline{x}+\lambda \overline{z}\) still a basic solution? And if it is, what is the basic partition that goes with it?
\end{problem}

\begin{answer}
	We see that after one iteration, one of the basic index \(i^{*}\) will become non-basic, namely
	\[
		(\overline{x} + \lambda \overline{z})_{\beta_{i^{*}}} = 0;
	\]
	while one of the non-basic index will need to become basic, since
	\[
		(\overline{x} + \lambda \overline{z})_{\beta_{i^{*}}} = \lambda \overline{e}_j.
	\]
	Namely
	\begin{table}[H]
		\centering
		\begin{tabular}{cc|c|c|cc}
			\toprule
			                       &           & \(\overline{x}\)            & \(\overline{z}\)                & \(\overline{x} + \lambda \overline{z}\) &                                            \\
			\midrule
			\(\beta_{i^{*}} \to \) & \(\beta\) & \(\overline{x}_{\beta}\)    & \(\overline{z}_{\beta}\)        & \(\to 0\)                               & \emph{\(\beta_{i^{*}}\) becomes non-basic} \\\hline
			                       & \(\eta\)  & \(\overline{x}_{\eta} = 0\) & \(\overline{x}_{\eta} = e_{j}\) & \(\lambda \overline{e}_j\)              & \emph{\(\eta_{j}\) becomes basic}          \\
			\bottomrule
		\end{tabular}
	\end{table}
\end{answer}

Now, suppose \(i^{*}\) is that chosen index, which means \(\overline{a}_{i^{*}, \eta_{j}}>0\) and \(\frac{\overline{x}_{\beta_{i^{*}}}}{\overline{a}_{i^{*}, \eta_{j}}} = \overline{\lambda}\). Then we have
\(\beta_{i^{*}}\) such that
\[
	\overline{x} + \lambda \overline{z} \implies \overline{x}_{\beta_{i^{*}}} + \overline{\lambda} \overline{z}_{\beta_{i^{*}}} = \overline{x}_{\beta_{i^{*}}}+\frac{\overline{x}_{\beta_{i^{*}}}}{\overline{a}_{i^{*}, \eta_{j}}}\left( -\overline{a}_{i^{*}, \eta_{j}} \right) = 0.
\]

So we reasonably suspect that there is a new basic partition such that
\[
	\begin{alignedat}{5}
		\widetilde{\beta} &\coloneqq ( \beta_1, \beta_2, \ldots , \beta_{i^{*}-1}, &&\eta_{j}, \beta_{i^{*}+1}, \ldots , \beta_m ) \\
		& &&\updownarrow\\
		\widetilde{\eta} &\coloneqq ( \eta_1, \eta_2, \ldots , \eta_{j - 1}, &&\beta_{i^{*}}, \eta_{j+1}, \ldots , \eta_{n-m} ).
	\end{alignedat}
\]

The remaining question is that, is \(A_{\widetilde{\beta}}\) still invertible? Namely, is \(\det(A_{\widetilde{\beta}}) \neq 0\)?
\begin{lemma}
	\(A_{\widetilde{\beta}}\) is invertible.
\end{lemma}
\begin{proof}
	We see that \(A_{\widetilde{\beta}}\) is invertible if and only if \(A^{-1}_{\beta}A_{\widetilde{\beta}}\) is invertible. And since
	\[
		A^{-1}_{\beta}A_{\widetilde{\beta}} = \begin{bmatrix}
			e_1 & e_2 & \ldots & e_{i^{*}-1} & \overline{A}_{\eta_{j}} & e_{i^{*}+1} & \ldots & e_m \\
		\end{bmatrix},
	\]
	and since \(\det(A^{-1}_\beta A_{\widetilde{\beta}}) = \overline{a}_{i^{*}, \eta_{j}}\), if \(\overline{a}_{i^{*}, \eta_{j}}\neq 0\), then we see that this is indeed invertible. But this is
	a obvious fact by our choice of \(i^{*}\).
\end{proof}

\begin{figure}[H]
	\centering
	\incfig{pivot-swap}
	\caption{Pivot Swap in terms of feasible region.}
	\label{fig:pivot-swap}
\end{figure}

Finally, we check that the unique \emph{basic} solution for this basic partition \(\widetilde{\beta},\ \widetilde{\eta}\) are exactly \(\overline{x} + \overline{\lambda}\overline{z}\).
\begin{lemma}
	The unique solution of \(Ax = b\) having \(x_{\widetilde{\eta}} = 0\) is \(\overline{x} + \overline{\lambda}\overline{z}\).
\end{lemma}
\begin{proof}
	Firstly, \((\overline{x} + \overline{\lambda}\overline{z})_j = 0\) for \(j\in \widetilde{\eta}\). Moreover, \(\overline{x} + \overline{\lambda}\overline{z}\) is
	the unique solution to \(Ax = b\) having \(x_{\widetilde{\eta}} = 0\) because \(A_{\widetilde{\beta}}\) is invertible, namely
	\[
		Ax = b \implies \underbrace{\cancel{A_{\widetilde{\eta}}x_{\widetilde{\eta}}}}_{=0} + A_{\widetilde{\beta}}x_{\widetilde{\beta}} = b \implies x_{\widetilde{\beta}} = A^{-1}_{\widetilde{\beta}}b.
	\]
\end{proof}