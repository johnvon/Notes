\lecture{18}{8 Nov. 08:00}{Lagrangian Relaxation}
\begin{prev}
	The Simplex Algorithm.
	\begin{enumerate}
		\item[1.] Initialization(Phase I). Find an initial basic feasible partition \(\beta, \eta\)
		\item[2.] Is there a nonbasic variable with negative reduced cost?
			\[
				\overline{c}_j \coloneqq c_{j} - \overline{y}^{T}A_{\eta_{j}}<0.
			\]
			If not, then we have an optimal solution.
		\item[3.] Find the leaving variable.
			\[
				i^{*} \coloneqq \arg\max_{\overline{a}_{i, \eta_{j}}>0} {\left\{\frac{\overline{x}_{\beta_{i}}}{\overline{a}_{i, \eta_{j}}}\right\}}.
			\]
			If \(i^{*}\) is undefined, then problem is unbounded.
		\item[4.] Swap \(\beta_{i}\) and \(\eta_{j}\) and \textbf{GOTO 2}.
	\end{enumerate}
\end{prev}

\begin{prev}
	The Decomposition Algorithm. We change the step 0 and 2 of the original Simplex Algorithm into the following.
	\begin{enumerate}
		\item[0.] Reformulate \(Q\) as \(M\) and apply The Simplex Algorithm to \(M\), where
			\begin{align*}
				\min~    & c^Tx      \\
				         & Ex \geq h \\
				         & Ax = b    \\
				(Q)\quad & x\geq 0
			\end{align*}
			and
			\begin{align*}
				\min~    & \sum\limits_{j\in\mathcal{J}}\left(c^T\hat{x}^j\right)\lambda_{j} + \sum\limits_{k\in\mathcal{K}}\left(c^{T} \hat{z}^k  \right)\mu_k \\
				         & \sum\limits_{j\in\mathcal{J}}\left(E\hat{x}^j\right)\lambda_{j} + \sum\limits_{k\in\mathcal{K}}\left(E \hat{z}^k \right)\mu_k \geq h \\
				         & \sum\limits_{j\in\mathcal{J}}\lambda_{j} = 1                                                                                         \\
				(M)\quad & \lambda_{j}\geq 0 \text{ for }j\in\mathcal{J},\ \mu_k\geq 0 \text{ for }k\in\mathcal{K}.
			\end{align*}
		\item[2.] Solve the sub-problem
			\begin{align*}
				-\overline{\sigma}+\min~ & \overline{c} - \overline{y}^{T}E \\
				                         & Ax = b                           \\
				                         & x\geq 0
			\end{align*}
			\begin{itemize}
				\item Optimal \& Obj.val \(<0 \implies\) a \(\lambda\) variable can enter the basis.
				\item Optimal \& Obj.val \(>0 \implies\) have an optimal for \(M\).
				\item Unbounded \(\implies\) a \(\mu_k\) variable can enter the basis.
			\end{itemize}
	\end{enumerate}

	\begin{note}
		Compare 2. here and 2. in the Simplex Algorithm.
	\end{note}
	\begin{remark}
		For the real implementation in step 2., we
		\begin{enumerate}
			\item Keep all generated columns.
			\item First check reduced costs of columns already generated. \textbf{Repeat}. Only solve for sub-problem when needed.
		\end{enumerate}

		\begin{note}
			We see that we are solving \(M\) over the known columns. So instead, we can pass \(M\) to a solver (\texttt{Gurobi}). And since
			it will give us the dual variable \(\overline{y}\) and \(\overline{\sigma}\), we can continue to solve the sub-problem without problems.
			Furthermore, we solve the sub-problem and append new column to known ones and go solve the sub-problem again. In short, let the solver
			keep track of the basis.
		\end{note}
	\end{remark}
\end{prev}

\subsection{Lagrangian Relaxation}
The motivation is to get a good lower bound of optimal objective value for
\begin{align*}
	z\coloneqq \min~ & c^{T}x    \\
	                 & Ex \geq h \\
	                 & Ax = b    \\
	(Q)\quad         & x\geq 0
\end{align*}

Since the problem is large, hence we want to exit the algorithm whenever we get a \emph{good enough} solution such that it's not far away from the objective
value. But the problem is, when should we stop? Do we stop at plateaus? What if there is a second drop in terms of objective value?

\begin{figure}[H]
	\centering
	\incfig{plateau}
	\caption{Early Arrival, can we?}
	\label{fig:plateau}
\end{figure}

\subsubsection{Laguargian Bounds}
We choose \(\hat{y}\geq \vec{0}\), such that
\begin{align*}
	v(\hat{y})\coloneqq \hat{y}^{T}h + \min~ & (c^{T} - \hat{y}^{T}E)x \\
	                                         & Ax = b                  \\
	(L_{\hat{y}})\quad                       & x\geq 0
\end{align*}
where \(L\) stands for Lagrange.
\begin{intuition}
	We see that we are trying to \emph{bring} the complex constraint \(Ex\geq h\) into the objective function.
\end{intuition}

To characterize how good will this approximation be, we first see a simple result.
\begin{lemma}
	For any \(\hat{y}\geq \vec{0}\), \(v(\hat{y})\leq z\).
\end{lemma}
\begin{proof}
	Let \(x^{*}\) be an optimal solution for \(Q\). Then \(x^{*}\) is feasible for \(L_{\hat{y}}\). Then we see
	\[
		v(\hat{y}) \leq \hat{y}^{T}h + (c^{T} - \hat{y}^{T} E )x^{*} = \underbrace{\vphantom{\hat{y}^{T}}c^{T}x^{*}}_{z} + \underbrace{\hat{y}^{T}}_{\geq \vec{0}}(\underbrace{\vphantom{\hat{y}^{T}}h - Ex^{*}}_{\leq \vec{0}}) \leq z
	\]
\end{proof}

Denote the dual variables of \(Q\) as \(y\) and \(\pi\). The dual of \(Q\) is
\begin{align*}
	\max~ & y^{T}h + \pi^{T} b          \\
	      & y^{T}E + \pi^{T}A\leq c^{T} \\
	      & y\geq 0
\end{align*}
and the dual of \(L_{\hat{y}}\) is
\begin{align*}
	\hat{y}^{T}h + \max~ & \pi^{T}b                          \\
	                     & \pi^{T}A\leq c^{T} - \hat{y}^{T}E \\
\end{align*}
\begin{note}
	\(\hat{y}\) is not the variable.
\end{note}

\begin{theorem}
	\label{lagrangian dual}
	Suppose \(x^{*}\) is optimal for \(Q\). Further, suppose \(\hat{y}\) and \(\hat{\pi}\) are optimal for the dual of \(Q\).
	Then
	\begin{itemize}
		\item \(x^{*}\) is optimal for \(L_{\hat{y}}\)
		\item \(\hat{\pi}\) is optimal for the dual of \(L_{\hat{y}}\)
		\item \(\hat{y}\) is a maximizer of \(v(y)\) over \(y>\vec{0}\)
		\item The maximum value of \(v(y)\) over \(y\geq \vec{0}\) is \(z\).
	\end{itemize}
\end{theorem}
\begin{note}
	In above, we want to find
	\begin{align*}
		z = \max~ & v(y)     \\
		          & y\geq 0.
	\end{align*}
\end{note}

\begin{proof}
	\(x^{*}\) is feasible for \(L_{\hat{y}}\) and \(\hat{y}\) and \(\hat{\pi}\) is feasible for the dual of \(Q\). Then
	\[
		\hat{y}^{T}E + \hat{\pi}^{T}A \leq c^{T}
	\]
	with \(\hat{y}^{T}\geq \vec{0}\). But we see that this is equivalent to
	\[
		\hat{\pi}^{T}A\leq c^{T} - \hat{y}^{T}E,
	\]
	which implies \(\hat{\pi}\) is feasible for the dual of \(L_{\hat{y}}\).

	From Strong Duality Theorem(\autoref{strong duality theorem}) for \(Q\),
	\[
		c^{T}x^{*} = \hat{y}^{T}h + \hat{\pi}^{T}b.
	\]
	Then, by using \(E \hat{x}^{*} \geq h\), we see that
	\[
		(c^{T} - \hat{y}^{T}E)x^{*} \leq \hat{\pi}^{T}b.
	\]
	Moreover, recall \(\hat{\pi}\) is feasible for the dual of \(L_{\hat{y}}\) with \(\hat{\pi}^{T} A \leq c^{T} - \hat{y}^{T}E\), then since \(x^{*}>0\), we have
	\[
		\hat{\pi}^{T} \underbrace{Ax^{*}}_{b} \leq (c^{T} - \hat{y}^{T}E)x^{*} \iff \hat{\pi}^{T} b \leq (c^{T} - \hat{y}^{T}E)x^{*}.
	\]
	We conclude
	\[
		\hat{\pi}^{T} b = (c^{T} - \hat{y}^{T}E)x^{*}.
	\]

	Now, we claim that \(x^{*}\) is optimal for \(L_{\hat{y}}\) and \(\hat{\pi}\) is optimal for the dual of \(L_{\hat{y}}\).
	Indeed, recall that the objective function of \(L_{\hat{y}}\) is
	\[
		\hat{y}^{T}h + \min~  (c^{T} - \hat{y}^{T}E)x,
	\]
	with \((c^{T} - \hat{y}^{T}E)x^{*} \geq \hat{\pi}^{T}b\), we see that the objective value of \(x^{*}\) in \(L_{\hat{y}}\)
	is equal to \(\hat{y}h+\hat{\pi}^{T}b \), which implies that \(x^{*}\) is optimal in \(L_{\hat{y}}\) from Weak Duality Theorem(\autoref{Weak Duality Theorem}).

	Lastly, since \(x^{*}\) is optimal for \(L_{\hat{y}}\),
	\[
		z\geq v(\hat{y})= \hat{y}^{T}h + (c^{T} - \hat{y}^{T}E)x^{*}= \hat{y}^{T}h + \hat{\pi}^{T}b = z
	\]
	by optimally for \(\hat{y}\) and \(\hat{\pi}\), hence we see that \(\hat{y}\) solves
	\begin{align*}
		\max~ & v(y)    \\
		      & y\geq 0
	\end{align*}
	and \(v(\hat{y}) = z\).
\end{proof}

\begin{theorem}
	\label{converse-lagrangian dual}
	Suppose that \(\hat{y}\) is a maximizer of \(v(y)\) over \(y\geq \vec{0}\). Suppose \(\hat{\pi}\) solves the dual of \(L_{\hat{y}}\).
	Then \(\hat{\pi}\) and \(\hat{y}\) solve the dual of \(Q\) and the optimal value of \(Q\) is \(v(\hat{y})\).
\end{theorem}
\begin{intuition}
	We see that compare to the last theorem, we now try to say something \emph{backwards}. But we immediately see that it suffers from
	the situations depicts as follows.
\end{intuition}
\begin{figure}[H]
	\centering
	\incfig{Lagrangian-maximizer}
	\caption{There may exist several \(\hat{y} \)!}
	\label{fig:lagrangian-maximizer}
\end{figure}
