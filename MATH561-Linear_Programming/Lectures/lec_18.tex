\lecture{18}{8 Nov. 08:00}{Lagrangian Relaxation}
\begin{prev}
	The \hyperref[algo:simplex-algorithm]{Simplex Algorithm}.
	\begin{enumerate}
		\item[1.] Initialization (Phase I). Find an initial \hyperref[def:basic-partition]{basic feasible partition} \(\beta, \eta\)
		\item[2.] Is there a \hyperref[def:non-basic]{non-basic} variable with negative \hyperref[def:reduced-cost]{reduced cost}?
			\[
				\overline{c}_j \coloneqq c_{j} - \overline{y}^{\top}A_{\eta_{j}}<0.
			\]
			If not, then we have an \hyperref[def:optimal-solution]{optimal solution}.
		\item[3.] Find the leaving variable.
			\[
				i^{\ast} \coloneqq \arg\max_{\overline{a}_{i, \eta_{j}}>0} {\left\{\frac{\overline{x}_{\beta_{i}}}{\overline{a}_{i, \eta_{j}}}\right\}}.
			\]
			If \(i^{\ast}\) is undefined, then problem is unbounded.
		\item[4.] Swap \(\beta_{i}\) and \(\eta_{j}\) and \textbf{GOTO 2}.
	\end{enumerate}
\end{prev}
Then the decomposition algorithm can be written as follows. We change the step 0. and 2. of the above \hyperref[algo:simplex-algorithm]{simplex algorithm} into the following.
\begin{enumerate}
	\item[0.] Reformulate \(Q\) as \(M\) and apply \hyperref[algo:simplex-algorithm]{simplex algorithm} to \(M\), where
		\begin{align*}
			\min~    & c^{\top}x \\
			         & Ex \geq h \\
			         & Ax = b    \\
			(Q)\quad & x\geq 0
		\end{align*}
		and
		\begin{align*}
			\min~    & \sum\limits_{j\in\mathcal{J}}\left(c^{\top}\hat{x}^j\right)\lambda_{j} + \sum\limits_{k\in\mathcal{K}}\left(c^{\top} \hat{z}^k  \right)\mu_k \\
			         & \sum\limits_{j\in\mathcal{J}}\left(E\hat{x}^j\right)\lambda_{j} + \sum\limits_{k\in\mathcal{K}}\left(E \hat{z}^k \right)\mu_k \geq h         \\
			         & \sum\limits_{j\in\mathcal{J}}\lambda_{j} = 1                                                                                                 \\
			(M)\quad & \lambda_{j}\geq 0 \text{ for }j\in\mathcal{J},\ \mu_k\geq 0 \text{ for }k\in\mathcal{K}.
		\end{align*}
	\item[2.] Solve the sub-problem
		\begin{align*}
			-\overline{\sigma}+\min~ & \overline{c} - \overline{y}^{\top}E \\
			                         & Ax = b                              \\
			                         & x\geq 0
		\end{align*}
		\begin{itemize}
			\item \hyperref[def:optimal-solution]{optimal} \& Objective value \(<0 \implies\) a \(\lambda\) variable can enter the \hyperref[def:basic]{basis}.
			\item \hyperref[def:optimal-solution]{optimal} \& Objective value \(>0 \implies\) have an optimal for \((M)\).
			\item Unbounded \(\implies\) a \(\mu_k\) variable can enter the \hyperref[def:basic]{basis}.
		\end{itemize}
\end{enumerate}

\begin{note}
	Compare 2. here and 2. in the \hyperref[algo:simplex-algorithm]{simplex algorithm}.
\end{note}
\begin{remark}
	For the real implementation in step 2., we
	\begin{enumerate}
		\item Keep all generated columns.
		\item First check \hyperref[def:reduced-cost]{reduced costs} of columns already generated. \textbf{Repeat}. Only solve for sub-problem when needed.
	\end{enumerate}

	\begin{note}
		We see that we are solving \((M)\) over the known columns. So instead, we can pass \((M)\) to a solver (\texttt{Gurobi}). And since
		it will give us the \hyperref[def:dual]{dual} variable \(\overline{y}\) and \(\overline{\sigma}\), we can continue to solve the sub-problem without problems.
		Furthermore, we solve the sub-problem and append new column to known ones and go solve the sub-problem again. In short, let the solver
		keep track of the \hyperref[def:basic]{basis}.
	\end{note}
\end{remark}

\section{Lagrangian Relaxation}
The motivation is to get a good lower bound of optimal objective value for
\begin{align*}
	z\coloneqq \min~ & c^{\top}x \\
	                 & Ex \geq h \\
	                 & Ax = b    \\
	(Q)\quad         & x\geq 0
\end{align*}

Since the problem is large, hence we want to exit the algorithm whenever we get a \emph{good enough} solution such that it's not far away from the objective
value. But the problem is, when should we stop? Do we stop at plateaus? What if there is a second drop in terms of objective value?

\begin{figure}[H]
	\centering
	\incfig{plateau}
	\caption{Early Arrival, can we?}
	\label{fig:plateau}
\end{figure}

\subsection{Lagrangian Bounds}
We first start with a specific problem which is important in our analysis.
\begin{definition}[Lagrangian subproblem]\label{def:Lagrangian-subproblem}
	We choose \(\hat{y}\geq \vec{0}\), and the corresponding \emph{Lagrangian subproblem} \(L_{\hat{y}}\) is defined as
	\begin{align*}
		v(\hat{y})\coloneqq \hat{y}^{\top}h + \min~ & (c^{\top} - \hat{y}^{\top}E)x \\
		                                            & Ax = b                        \\
		(L_{\hat{y}})\quad                          & x\geq 0
	\end{align*}
	where \(L\) stands for Lagrange\footnote{\url{https://en.wikipedia.org/wiki/Joseph-Louis_Lagrange}}.
\end{definition}

\begin{intuition}
	We are trying to \emph{bring} the complex constraint \(Ex\geq h\) into the objective function.
\end{intuition}

To characterize how good will this approximation be, we first see a simple result.
\begin{lemma}\label{lma:lec18-1}
	For any \(\hat{y}\geq \vec{0}\), \(v(\hat{y})\leq z\).
\end{lemma}
\begin{proof}
	Let \(x^{\ast}\) be an \hyperref[def:optimal-solution]{optimal solution} for \(Q\). Then \(x^{\ast}\) is \hyperref[def:feasible-solution]{feasible} for \(L_{\hat{y}}\).
	Then we see
	\[
		v(\hat{y}) \leq \hat{y}^{\top}h + (c^{\top} - \hat{y}^{\top} E )x^{\ast} = \underbrace{\vphantom{\hat{y}^{\top}}c^{\top}x^{\ast}}_{z} + \underbrace{\hat{y}^{\top}}_{\geq \vec{0}}(\underbrace{\vphantom{\hat{y}^{\top}}h - Ex^{\ast}}_{\leq \vec{0}}) \leq z
	\]
\end{proof}

Denote the \hyperref[def:dual]{dual} variables of \(Q\) as \(y\) and \(\pi\). The \hyperref[def:dual]{dual} of \(Q\) is
\begin{align*}
	\max~ & y^{\top}h + \pi^{\top} b             \\
	      & y^{\top}E + \pi^{\top}A\leq c^{\top} \\
	      & y\geq 0
\end{align*}
and the \hyperref[def:dual]{dual} of \(L_{\hat{y}}\) is
\begin{align*}
	\hat{y}^{\top}h + \max~ & \pi^{\top}b                                \\
	                        & \pi^{\top}A\leq c^{\top} - \hat{y}^{\top}E \\
\end{align*}
\begin{note}
	\(\hat{y}\) is not the variable.
\end{note}

\begin{theorem}[Laguargian dual theorem]\label{thm:Lagrangian-dual-theorem}
	Suppose \(x^{\ast}\) is \hyperref[def:optimal-solution]{optimal} for \(Q\). Further, suppose \(\hat{y}\) and \(\hat{\pi}\) are \hyperref[def:optimal-solution]{optimal}
	for the \hyperref[def:dual]{dual} of \(Q\). Then
	\begin{itemize}
		\item \(x^{\ast}\) is \hyperref[def:optimal-solution]{optimal} for \(L_{\hat{y}}\)
		\item \(\hat{\pi}\) is \hyperref[def:optimal-solution]{optimal} for the \hyperref[def:dual]{dual} of \(L_{\hat{y}}\)
		\item \(\hat{y}\) is a maximizer of \(v(y)\) over \(y>\vec{0}\)
		\item The maximum value of \(v(y)\) over \(y\geq \vec{0}\) is \(z\).
	\end{itemize}
\end{theorem}
\begin{proof}
	We first make a note.
	\begin{note}
		In above, we want to find
		\begin{align*}
			z = \max~ & v(y)     \\
			          & y\geq 0.
		\end{align*}
	\end{note}

	\(x^{\ast}\) is \hyperref[def:feasible-solution]{feasible} for \(L_{\hat{y}}\) and \(\hat{y}\) and \(\hat{\pi}\) is \hyperref[def:feasible-solution]{feasible} for the
	\hyperref[def:dual]{dual} of \(Q\). Then
	\[
		\hat{y}^{\top}E + \hat{\pi}^{\top}A \leq c^{\top}
	\]
	with \(\hat{y}^{\top}\geq \vec{0}\). But we see that this is equivalent to
	\[
		\hat{\pi}^{\top}A\leq c^{\top} - \hat{y}^{\top}E,
	\]
	which implies \(\hat{\pi}\) is \hyperref[def:feasible-solution]{feasible} for the \hyperref[def:dual]{dual} of \(L_{\hat{y}}\).

	From \hyperref[thm:strong-duality-theorem]{strong duality theorem} for \(Q\),
	\[
		c^{\top}x^{\ast} = \hat{y}^{\top}h + \hat{\pi}^{\top}b.
	\]
	Then, by using \(E \hat{x}^{\ast} \geq h\), we see that
	\[
		(c^{\top} - \hat{y}^{\top}E)x^{\ast} \leq \hat{\pi}^{\top}b.
	\]
	Moreover, recall \(\hat{\pi}\) is \hyperref[def:feasible-solution]{feasible} for the \hyperref[def:dual]{dual} of \(L_{\hat{y}}\) with \(\hat{\pi}^{\top} A \leq c^{\top} - \hat{y}^{\top}E\), then since \(x^{\ast}>0\), we have
	\[
		\hat{\pi}^{\top} \underbrace{Ax^{\ast}}_{b} \leq (c^{\top} - \hat{y}^{\top}E)x^{\ast} \iff \hat{\pi}^{\top} b \leq (c^{\top} - \hat{y}^{\top}E)x^{\ast}.
	\]
	We conclude
	\[
		\hat{\pi}^{\top} b = (c^{\top} - \hat{y}^{\top}E)x^{\ast}.
	\]

	Now, we claim that \(x^{\ast}\) is \hyperref[def:optimal-solution]{optimal} for \(L_{\hat{y}}\) and \(\hat{\pi}\) is \hyperref[def:optimal-solution]{optimal}
	for the \hyperref[def:dual]{dual} of \(L_{\hat{y}}\). Indeed, recall that the \hyperref[def:objective-function]{objective function} of \(L_{\hat{y}}\) is
	\[
		\hat{y}^{\top}h + \min~  (c^{\top} - \hat{y}^{\top}E)x,
	\]
	with \((c^{\top} - \hat{y}^{\top}E)x^{\ast} \geq \hat{\pi}^{\top}b\), we see that the objective value of \(x^{\ast}\) in \(L_{\hat{y}}\)
	is equal to \(\hat{y}h+\hat{\pi}^{\top}b \), which implies that \(x^{\ast}\) is \hyperref[def:optimal-solution]{optimal} in \(L_{\hat{y}}\) from \hyperref[thm:weak-duality-theorem]{weak duality theorem}.

	Lastly, since \(x^{\ast}\) is \hyperref[def:optimal-solution]{optimal} for \(L_{\hat{y}}\),
	\[
		z\geq v(\hat{y})= \hat{y}^{\top}h + (c^{\top} - \hat{y}^{\top}E)x^{\ast}= \hat{y}^{\top}h + \hat{\pi}^{\top}b = z
	\]
	by optimally for \(\hat{y}\) and \(\hat{\pi}\), hence we see that \(\hat{y}\) solves
	\begin{align*}
		\max~ & v(y)    \\
		      & y\geq 0
	\end{align*}
	and \(v(\hat{y}) = z\).
\end{proof}

\begin{theorem}[Converse Lagrangian dual theorem]\label{thm:converse-Lagrangian-dual-theorem}
	Suppose that \(\hat{y}\) is a maximizer of \(v(y)\) over \(y\geq \vec{0}\). Suppose \(\hat{\pi}\) solves the \hyperref[def:dual]{dual} of \(L_{\hat{y}}\).
	Then \(\hat{\pi}\) and \(\hat{y}\) solve the \hyperref[def:dual]{dual} of \(Q\) and the \hyperref[def:optimal-solution]{optimal} value of \(Q\) is \(v(\hat{y})\).
\end{theorem}
\begin{intuition}
	We see that compare to \autoref{thm:Lagurgian-dual-theorem}, we now try to say something \emph{backwards}. But we immediately see that it suffers from
	the situations depicts as follows.
	\begin{figure}[H]
		\centering
		\incfig{Lagrangian-maximizer}
		\caption{There may exist several \(\hat{y} \)!}
		\label{fig:lagrangian-maximizer}
	\end{figure}
\end{intuition}