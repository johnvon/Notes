\lecture{5}{15 Sep. 08:00}{Convex Set}
\subsection{Convex Set}
\begin{definition}
	A set \(S\subseteq \mathbb{\MakeUppercase{R}}^n\) is a \emph{convex set} if
	\[
		x^1, x^2\in S, \text{ and }0<\lambda<1 \implies \lambda x^1 + (1-\lambda)x^2 \in S.
	\]
\end{definition}

\begin{figure}[H]
	\centering
	\incfig{convex-set}
	\caption{Convex Sets}
	\label{fig:convex-set}
\end{figure}

\begin{intuition}
	A convex set is a set that contains every line segment between two points in which.
\end{intuition}

\begin{remark}
	The feasible region of any linear program is a convex set.
\end{remark}
\begin{proof}
	Suppose there are two points \(x^1\) and \(x^2\in S\) which means they are feasible. Consider a standard form problem, then we know
	\[
		\begin{dcases}
			Ax^1 = b, & x^1\geq 0 \\
			Ax^2 = b, & x^2\geq 0 \\
		\end{dcases}.
	\]
	Then we simply have
	\[
		A\underbrace{(\lambda x^1 + (1-\lambda)x^2)}_{\geq 0} = \lambda A x^1 + (1-\lambda)A x^2 = (\lambda + (1-\lambda))b = b
	\]
	for every \(\lambda \in (0,1)\). With the fact that \(\lambda x^1 + (1-\lambda)x^2)\) is non-negative, hence it's feasible.
\end{proof}

\subsection{Extreme Point}
\begin{definition}
	Suppose \(S\) is a convex set. Consider \(\hat{x} \in S\). \(\hat{x}\) is an \emph{extreme-point} of \(S\) if we \textbf{cannot} write
	\[
		\hat{x} = \lambda x^1 + (1-\lambda)x^2 \text{ with }x^1 \neq x^2, x^1, x^2\in S, 0<\lambda<1.
	\]
\end{definition}

Then we have an important theorem.
\begin{theorem}
	Every basic feasible solution of standard form problem \(P\) is an extreme-point of the feasible region of \(P\).
\end{theorem}
\begin{proof}
	Consider a basic feasible solution \(\overline{x}: \overline{x}_{\eta} = \vec{0}, \overline{x}_{\beta} = A^{-1}_{\beta}b\geq \vec{0}\).
	If it is not an extreme-point, then we have
	\[
		\exists x^1\neq x^2 \text{ which is feasible}, \text{ for }0<\lambda<1\text{ with }\overline{x} = \lambda x^1 + (1-\lambda)x^2,
	\]
	we will have
	\[
		\overline{x}_{\eta} = \underbrace{\vphantom{x^1_{\eta}}\lambda}_{>0}\underbrace{x^1_{\eta}}_{>0} + \underbrace{\vphantom{x^1_{\eta}}(1-\lambda)}_{>0}\underbrace{x^2_{\eta}}_{\geq 0} \implies x^1_{\eta} = x^2_{\eta} = 0 \implies x^1_{\beta} = x^2_{\beta} = A^{-1}_{\beta}b.
	\]
	Hence, we see that \(\overline{x} = x^1 = x^2\)\contd
\end{proof}

The converse is also true, but it's harder to show.\ldots
\begin{theorem}
	If \(\hat{x}\) is an extreme-point of the feasible region of \(P\), then \(\hat{x}\) is basic.
\end{theorem}
\begin{proof}
	Skip\ldots we leave it here.
\end{proof}

