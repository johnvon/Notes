\lecture{6}{20 Sep. 08:00}{Feasible Direction and Ray}
\subsection{Feasible Directions}
We'll talk about an important concept, but before this, we first play around with the standard form problem a little. Consider
\begin{align*}
	\min~ & c^Tx     \\
	      & Ax = b   \\
	      & x\geq 0.
\end{align*}
It's obvious that it's equivalent to
\begin{align*}
	\min~ & c^{T}_{\beta}x_{\beta} + c^{T}_{\eta}x_{\eta} \\
	      & A_{\beta}x_{\beta} + A_{\eta}x_{\eta} = b     \\
	      & x_{\beta}\geq 0, x_{\eta}\geq 0
\end{align*}
Further, we have
\begin{align*}
	\min~ & c^{T}_{\beta}(A^{-1}_{\beta}b - A^{-1}_{\beta}A_{\eta}x_{\eta} ) + c_{\eta}^{T}x_{\eta} \\
	      & x_{\beta} + A^{-1}_{\beta}A_{\eta}x_{\eta} = A^{-1}_{\beta}b                            \\
	      & x_{\beta}\geq 0, x_{\eta}\geq 0
\end{align*}
since from the constraint, we have \(x_{\beta} = A^{-1}_{\beta}b - A^{-1}_{\beta}A_{\eta}x_{\eta}\). Finally, we see that the objective function now
only depends on \(x_{\eta}\), hance
\begin{align*}
	c^{T}_{\beta}A^{-1}_{\beta}b + \min~ & (c_{\eta}^{T} - c_{\beta}^{T}A^{-1}_{\beta}A_{\eta})x_{\eta} \\
	                                     & A^{-1}_{\beta}A_{\eta}x_{\eta} \leq A^{-1}_{\beta}b          \\
	                                     & x_{\beta}\geq 0, x_{\eta}\geq 0.
\end{align*}

\begin{note}
	\(c_{\eta}^{T} - c_{\beta}^{T}A^{-1}_{\beta}A_{\eta}\) is what we called \emph{reduced costs}. We'll see that we want this to be zero.
\end{note}

Now, with this intuition, we have the following definition.
\begin{definition}
	Suppose \(\hat{x}\in \mathcal{S}\), where \(\mathcal{S}\) is a convex set. \(\hat{z}\) is a \emph{feasible direction} relative to \(\hat{x}\) if
	\[
		\exists \epsilon>0,\quad \hat{x}+\epsilon \hat{z} \in \mathcal{S}.
	\]
\end{definition}

\begin{figure}[H]
	\centering
	\incfig{feasible-direction}
	\caption{Feaisble Direction}
	\label{fig:feasible-direction}
\end{figure}

We see that in order to let \(\hat{z}\) to be a feasible direction, we need to have
\[
	A(\hat{x} + \epsilon \hat{z}) = \underbrace{A \hat{x}}_{=b} + \epsilon A \hat{z} = b \iff A \hat{z} = 0
\]

\begin{remark}
	For \(P\), we must have \(A \hat{z} = 0\) if \(\hat{z}\) is a feasible direction.
\end{remark}

\hr
Let the basic partition \(\beta, \eta\) being
\[
	\beta = (\beta_1, \ldots , \beta_m), \qquad \eta = (\eta_1, \underset{\substack{\uparrow \\ \eta_j}}{\ldots} , \eta_{n-m}),
\]
where we choose \(j\) from \(1\leq j \leq n-m\), which means we choose an \(\eta_j\) from \(\eta\). Then, we see that there is a \emph{basic direction}
\(\overline{z}\) associated with this particular basic and this \(j\) such that
\[
	\begin{split}
		\overline{z}_{\eta_j} = 1 \implies &\overline{z}_{\eta} \coloneqq e_j = \begin{pmatrix}
			0      \\
			\vdots \\
			1      \\
			\vdots \\
			0      \\
		\end{pmatrix}\leftarrow j\\
		&\overline{z}_{\beta} \coloneqq -A^{-1}_{\beta}A_{\eta_j}.
	\end{split}
\]

\begin{note}
	This needs
	\[
		A \overline{z} = 0 \implies A_{\beta}\overline{z}_{\beta} + A_{\eta}\overline{z}_{\eta} = 0 \implies A_{\beta}\overline{z}_{\beta}+A_{\eta}e_{j} = A_{\beta}\overline{z}_{\beta}+A_{\eta_j} = 0.
	\]
\end{note}

We check for feasibility:
\begin{enumerate}
	\item \(A \overline{z} = 0\): \(A(\overline{x} + \epsilon \overline{z}) = b\surd\)
	\item \(\overline{x} + \epsilon \overline{z}\):
	      \[
		      \begin{split}
			      &\overline{x}_{\eta} + \epsilon \overline{z}_{\eta} = 0 + \epsilon e_j \geq  0\\
			      &\overline{x}_{\beta}+\epsilon \overline{x}_{\beta} = \underbrace{A^{-1}_{\beta}b}_{\geq 0} - \underbrace{\vphantom{A^{-1}_{\beta}}\epsilon}_{>0} A^{-1}_{\beta}A_{\eta_j}\overset{?}{\geq}  0
		      \end{split}
	      \]
\end{enumerate}

Denote \(\overline{b} \coloneqq A^{-1}_{\beta} b,\ \overline{A}_{\eta_j} \coloneqq A^{-1}_{\beta}A_{\eta_j}\), then
\[
	\begin{split}
		\overline{b} - \epsilon \overline{A}_{\eta_{j}}\geq 0 &\iff \overline{b}_i - \epsilon \overline{a}_{i \eta_{j}}\geq 0, \text{ for }i = 1, \ldots , m\\
		&\iff \underbrace{\overline{b}_i}_{\geq 0} \geq \epsilon \overline{a}_{i \eta_{j}}, \text{ for }i - 1, \ldots , m
	\end{split}
\]
We finally have
\[
	\epsilon \leq \frac{\overline{b}_i}{\overline{a}_{i \eta_j}},\qquad \underset{1\leq i\leq m}{\forall}\  \overline{a}_{i \eta_{j}}>0.
\]
Notice that if \(\overline{a}_{i \eta_{j}}\leq 0\), there is no restriction on \(\epsilon\) being \(\geq 0\). Hence, we have a main result:
\(\overline{z}\) is a feasible direction from \(\overline{x}\) if
\[
	0< \min\{\frac{\overline{b}_i}{\overline{a}_{i, \eta_{j}}}\geq 0 \text{ for }i \text{ such that }\overline{a}_{i \eta_{j}}>0\}.
\]

\begin{note}
	Notice that we can denote \(A\) by
	\[
		A = \begin{bmatrix}
			A_{\eta} & A_{\beta} \\
		\end{bmatrix}.
	\]
	Then since \(A_{\beta}\) is invertible, so
	\[
		A^{-1}_{\beta}\begin{bmatrix}
			A_{\eta} & A_{\beta} \\
		\end{bmatrix} = \begin{bmatrix}
			A^{-1}_{\beta}A_{\eta} & I \\
		\end{bmatrix}_{m\times n}.
	\]

	We now consider
	\[
		\begin{bmatrix}
			I                       \\
			-A^{-1}_{\beta}A_{\eta} \\
		\end{bmatrix}.
	\]
	We then have
	\[
		\underbrace{
			\begin{bmatrix}
				I                       \\
				-A^{-1}_{\beta}A_{\eta} \\
			\end{bmatrix}}_{\dim(CS) = n-m}\underbrace{\vphantom{\begin{bmatrix}
					\\ \\
				\end{bmatrix}}\begin{bmatrix}
				A^{-1}_{\beta}A_{\eta} & I \\
			\end{bmatrix}}_{\dim(RS)=m} = 0.
	\]
	And since the dimension for the first matrix is \(n\times (m-n)\), we see that the columns of the first matrix form a \emph{basis} for the null space of \(\begin{bmatrix}
		A_{\eta} & A_{\beta} \\
	\end{bmatrix}\), namely \(A\). Furthermore, one can see that \(\overline{z}\) is the \(j^{th}\) columns of \(\begin{bmatrix}
		I                       \\
		-A^{-1}_{\beta}A_{\eta} \\
	\end{bmatrix}\) for a choice of \(j\).
\end{note}

\subsection{Feasible Rays}
\begin{definition}
	\(\hat{z}\) is called a \emph{ray} of a convex set \(\mathcal{C}\) of \(\hat{x}\in \mathcal{C}\) if
	\[
		\forall \lambda>0\quad \hat{x} + \lambda \hat{z} \in \mathcal{C}.
	\]
\end{definition}

\begin{figure}[H]
	\centering
	\incfig{ray}
	\caption{Ray}
	\label{fig:ray}
\end{figure}

Suppose \(\hat{x}\in\mathcal{C}\), where \(\mathcal{C}\) is the feasible region of
\begin{align*}
	 & Ax \geq  b \\
	 & x\geq 0
\end{align*},
then we see that in order to let \(\lambda\) arbitrarily large, we need
\[
	A(\hat{x} + \lambda \hat{z}) = \underbrace{A \hat{x}}_{=b} + \lambda \underbrace{A \hat{z}}_{=0} = b\implies \hat{z} \in n.s.(A).
\]

\begin{problem}
\[
	\underbrace{\hat{x}}_{\geq 0} + \underbrace{\lambda}_{>0} \hat{z} \overset{?}{\geq} 0 \implies \hat{z} \geq 0.
\]
\end{problem}

This means that starts from the idea of basic direction, \(\hat{z}\) is a \emph{ray} if
\[
	\hat{z} \geq 0 \iff A^{-1}_{\beta}A_{\eta_j} \leq 0.
\]

We have another concept about ray.
\begin{definition}
	\(\hat{z}\) is an \emph{extreme ray} of a convex set \(\mathcal{S}\) if we \textbf{cannot} write
	\[
		\hat{z} = z^1 + z^2 \text{ with }z^1 \neq \mu z^2,
	\]
	where \(z^1, z^2\) being rays of \(\mathcal{S}\) and \(\mu\neq 0\).
\end{definition}
\begin{figure}[H]
	\centering
	\incfig{extreme-ray}
	\caption{Extreme Rays}
	\label{fig:extreme-ray}
\end{figure}

\begin{summary}
	We can compare the \emph{basic direction that are non-negative} with \emph{extreme ray}.
	\[
		\begin{alignedat}{4}
			\text{Basic solution }\overline{x}=\begin{cases}
				 & \overline{x}_{\beta}\coloneqq A^{-1}_{\beta}b\geq 0 \\
				 & \overline{x}_{\eta}\coloneqq 0
			\end{cases} &&\quad\iff\qquad&\text{Extreme points of the feasible region}\\
			\text{\textbf{basic feasible direction}} && \quad\text{v.s.}\qquad&\text{\textbf{Geometry}}\\
			\text{(Basic direction that are non-negative)}&& &\text{(Extreme Ray)}
		\end{alignedat}
	\]
\end{summary}