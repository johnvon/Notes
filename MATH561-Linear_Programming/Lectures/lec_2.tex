\lecture{2}{1 Sep. 08:00}{Duality}
\subsection{First Glance of Duality}
We can associate the standard form problem with another linear programming problem, called the \emph{dual} of the original problem.
We sometimes called the original problem the \emph{primal}. The dual of the primal is
\[
	\begin{alignedat}{5}
		\min~&c^{T}x\qquad\qquad &&\max ~ &&y^{T}b\\
		&Ax = b && &&y^{T}A\leq c^{T}\\
		(P)\quad&x\geq  0 &&(D)\quad&&
	\end{alignedat}.
\]

\begin{note}
	We see that the dual is equivalent to
	\begin{align*}
		\max~ & b^{T}y         \\
		      & A^{T}y \leq c.
	\end{align*}
\end{note}

Then we have a direct, but important theorem.
\begin{theorem}\label{Weak Duality Theorem}
	Weak Duality Theorem: If \(\hat{x}\) is feasible for \(P\), and \(\hat{y}\) is feasible for \(D\), then we have
	\[
		c^{T}\hat{x} \geq  \hat{y}^{T} b.
	\]
\end{theorem}
\begin{proof}
	Since
	\[
		\hat{y}^{T}A\leq c^{T} \underset{\hat{x}\geq 0}{\implies} \hat{y}^{T}A \hat{x} \leq \hat{c}^{T} \hat{x} \underset{A \hat{x} = b}{\implies} \hat{y}^{T}b \leq c^{T} \hat{x}.
	\]
\end{proof}

\begin{eg}
	Consider
	\begin{align*}
		\min~ & c^Tx       \\
		      & Ax \geq b,
	\end{align*}
	turn this into the standard form problem and find the dual.

	We see that \(x\) is unrestricted. We first minus a surplus variable \(S\), we have
	\begin{align*}
		\min~ & c^Tx      \\
		      & Ax - S= b \\
		      & S \geq 0.
	\end{align*}
	Now, we turn \(x\) into \(x^+ - x^-\), namely
	\[
		x = \begin{pmatrix}
			x_1    \\
			\vdots \\
			x_n    \\
		\end{pmatrix},\qquad
		x^+ \coloneqq \begin{pmatrix}
			x^+_1  \\
			\vdots \\
			x^+_n  \\
		\end{pmatrix},\qquad
		x^- \coloneqq \begin{pmatrix}
			x^-_1  \\
			\vdots \\
			x^-_n  \\
		\end{pmatrix}, x^\pm \geq \vec{0}.
	\]
	Then we see the original problem becomes
	\begin{align*}
		\min~ & c^T(x^+ - x^-)       \\
		      & A(x^+ - x^-) - S = b \\
		      & x^+, x^-, S \geq 0
	\end{align*}
	\hr
	We can further have
	\begin{align*}
		\min~ & \begin{pmatrix}
			c^{T} & -c^{T} & 0 \\
		\end{pmatrix}\begin{pmatrix}
			x^+ \\
			x^- \\
			S   \\
		\end{pmatrix}     \\
		      & \begin{pmatrix}
			A & -A & -I \\
		\end{pmatrix}\begin{pmatrix}
			x^+ \\
			x^- \\
			S   \\
		\end{pmatrix} = b \\
		      & \begin{pmatrix}
			x^+ \\
			x^- \\
			S   \\
		\end{pmatrix}\geq 0.
	\end{align*}
	Set the dual variable being \(y\), we further have
	\begin{align*}
		\max~ & y^{T}b                                                           \\
		      & y^{T} \begin{pmatrix}
			A & -A & -I \\
		\end{pmatrix} \leq \begin{pmatrix}
			c^{T} & -c^{T} & 0^{T} \\
		\end{pmatrix}
	\end{align*}


\end{eg}

\begin{note}
	The dual of the dual is the primal.
\end{note}
