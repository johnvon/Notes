\lecture{8}{27 Sep. 08:00}{Simplex Algorithm}
\subsection{Remaining Problem}
\begin{prev}
	Simplex Algorithm:
	\begin{enumerate}
		\item Start with a basic feasible partition
		      \begin{enumerate}
			      \item \begin{enumerate}
				            \item Compute \(\overline{x}_{\beta} \coloneqq A^{-1}_{\beta} b \geq 0\)
				            \item Compute \(\overline{c}_{\eta}^{T} \coloneqq c_{\eta}^{T} - c_{\beta}^{T} A^{-1}_{\beta} A_{\eta}\)
			            \end{enumerate}
			      \item If \(\overline{c}_{\eta} \geq 0\), then \emph{STOP}. \(\overline{x}\) is optimal.
			      \item Otherwise, choose \(\eta_j\) such that \(\overline{c}_{\eta_j} < 0\).
			      \item Define \(i^{*}\coloneqq \underset{i:\overline{a}_{i, \eta_j}>0}{\arg\min} \{\frac{\overline{x}_{\rho_i}}{\overline{a}_{i, \eta_j}}\} \)
			      \item If \(i^{*}\) is undefined, then \emph{STOP}. (P) is unbounded.
		      \end{enumerate}
		\item Swap \(\beta_i^{*}\) out of \(\beta \) and \(\eta_j\) out of \(\eta \). \emph{GOTO 1}.
	\end{enumerate}
\end{prev}

\begin{problem}
How do we start with a basic feasible partition?
\end{problem}

\begin{answer}
	We consider the so-called \emph{Phase One Problem}.
\end{answer}
\subsubsection{Phase one problem}
\begin{align*}
	\min~ & c^T x                                \\
	      & Ax = b                               \\
	      & x\geq 0                  &  & (P)    \\\\
	\min~ & x_{n+1}                              \\
	      & Ax + A_{n+1}x_{n+1} = b              \\
	      & x\geq 0, x_{n+1} \geq  0 &  & (\Phi)
\end{align*}

\begin{enumerate}
	\item If min value of \(x_{n+1}\) in \(\Phi \) is 0, then we get a feasible solution of (P).
	\item If min value of \(x_{n+1}\) in \(\Phi \) is \(>0\), then there is no feasible solution of (P).
	      \begin{itemize}
		      \item How do we get an initial basic feasible solution for \(\Phi \)
		      \item Need a basic feasible solution.
	      \end{itemize}
\end{enumerate}

Solution:
\begin{enumerate}
	\item Start with a basic solution of (P), \(\tilde{\beta}, \tilde{\eta}\) is the basic partition.
	\item If its feasible(\(\overline{x}_{\tilde{\beta}}\)) then we just use \(\tilde{\beta}\) and \(\tilde{\eta}\) for \(\beta\) and \(\eta\)
	\item Otherwise, set \(A_{n+1} = -A^{-1}_{\beta}\vec{1}\). If \(\eta_j = n+1\)
	      \[
		      \overline{z} : \overline{z}_{\tilde{\eta}} = \begin{pmatrix}
			      0      \\
			      0      \\
			      \vdots \\
			      1      \\
		      \end{pmatrix},\qquad \overline{z}_{\beta} \coloneqq -A^{-1}_{\tilde{\beta}}(A_{n+1}) = \vec{1}
	      \]
	      and\[
		      \vec{x} \to \vec{x} + \lambda \vec{z} \geq \vec{0}.
	      \] \begin{eg}
		      \[
			      \vec{x}_{\tilde{\beta}} + \lambda \vec{z}_{\tilde{\beta}} = \begin{pmatrix}
				      7  \\
				      0  \\
				      3  \\
				      -5 \\
				      6  \\
				      -8 \\
			      \end{pmatrix} + \lambda \begin{pmatrix}
				      1 \\
				      1 \\
				      1 \\
				      1 \\
				      1 \\
				      1 \\
			      \end{pmatrix}
		      \], then \[
			      i^{*} = \underset{i:\vec{x}_{\tilde{\beta}} < 0}{\arg\min}\{-\vec{x}_{\tilde{\beta}}\}.
		      \]
	      \end{eg}
\end{enumerate}

\begin{problem}
What if \(x_{n+1} = 0\)?
\end{problem}
\begin{intuition}
	Just stop right before \(x_{n+1} = 0\), let other variable do that.
\end{intuition}

\subsubsection{Non degeneracy hypothesis}
\[
	\begin{split}
		&\vec{x}_{\beta_i} > 0 \text{ for all \(i\) at every iteration}\\
		\implies &\overline{\lambda} \neq 0\\
		\implies &\text{objective value decrease at each iteration.}\\
		\implies &\text{algorithm must (because a finite \# of basis)}
	\end{split}
\]

\begin{align*}
	\min~ & c^T x                                \\
	      & Ax = b + B\begin{pmatrix}
		\epsilon   \\
		\epsilon^2 \\
		\epsilon^3 \\
		\vdots     \\
		\epsilon^m \\
	\end{pmatrix} \\
	      & x\geq 0
\end{align*}
where \(\epsilon \) is an arbitrarily small \emph{indeterminate}.
\begin{remark}
	\[
		\epsilon \neq 0.
	\]
\end{remark}

\begin{observe}
	polynomial in \(\epsilon \):
	\[
		\begin{split}
			&p(\epsilon) = p_0 + p_1\epsilon + p_2 \epsilon^2 + \cdots + p_{m}\epsilon^m.	\\
			&\vec{x}_{\beta} = A^{-1}_{\beta}\left(b + B\begin{pmatrix}
					\epsilon   \\
					\epsilon^2 \\
					\vdots     \\
					\epsilon^m \\
				\end{pmatrix}\right) = A^{-1}_{\beta}b + A^{-1}_{\beta}B\begin{pmatrix}
				\epsilon   \\
				\epsilon^2 \\
				\vdots     \\
				\epsilon^m \\
			\end{pmatrix}
		\end{split}
	\]
\end{observe}

\begin{definition}
	Let \(K\) be the minimal index with \(p_K \neq 0\).
	\begin{itemize}
		\item If \(p_K < 0\), then \(p(\epsilon) < 0\)
		\item If \(p_K > 0\), then \(p(\epsilon) > 0\)
		\item If \(p_K = 0\), namely \(p_0 = p_1 = \cdots = p_m = 0\), then \(p(\epsilon) = 0\)
	\end{itemize}
\end{definition}

\begin{note}
	\[
		\begin{split}
			p(\epsilon) = p_0 + p_1\epsilon + p_2 \epsilon^2 + \cdots + p_{m}\epsilon^m.	\\
			q(\epsilon) = q_0 + q_1\epsilon + q_2 \epsilon^2 + \cdots + q_{m}\epsilon^m.	\\
		\end{split}.
	\]
	with \(K_p\) and \(K_q\). Then \(K_{p+q}\) depends on \(K_p\) and \(K_q\).
	\[
		p(\epsilon) - q(\epsilon) \geq 0?\qquad \text{ Then }p(\epsilon) \geq  q(\epsilon).
	\]
\end{note}

\begin{problem}
Where does this \(\epsilon \) thing links with the Simplex algorithm? (d)
\end{problem}

\subsubsection{Perturbed Problem}
Suppose
\[
	\overbrace{p(\epsilon)}^{\substack{\text{value of}\\ \text{some basic}\\ \text{variable}}} = p_0 + p_1 \epsilon + p_2 \epsilon^2 + \cdots + p_m \epsilon^m.
\]

Feasible for perturbed problem means \(p(\epsilon) \geq \vec{0}\) \(\implies p(0) = p_0 \geq 0\).

\begin{align*}
	\min~ & c^Tx                              \\
	      & Ax = b + \cancel{B\vec{\epsilon}} \\
	      & x\geq 0
\end{align*}
Find an initial feasible basis \(\beta, \eta \) for unperturbed problem, \(B\coloneqq A_{\beta}\),
\[
	\vec{x}_{\beta}
	= A_{\beta}^{-1}(b + A_{\beta} \vec{\epsilon})
	= \underbrace{A^{-1}_{\beta}b}_{\geq \vec{0}} + \vec{\epsilon}
	= \vec{x}_{\beta} + \begin{pmatrix}
		\epsilon   \\
		\epsilon^2 \\
		\epsilon^3 \\
		\vdots     \\
		\epsilon^4 \\
	\end{pmatrix} = \begin{pmatrix}
		\vec{x}_{\beta_1} + \epsilon   \\
		\vec{x}_{\beta_2} + \epsilon^2 \\
		\vdots                         \\
		\vec{x}_{\beta_m} + \epsilon^m \\
	\end{pmatrix} \geq \vec{0}.
\]

Claim: Perturbed problem is non-degenerate. \(\implies\) some later basis \(\tilde{\beta}\)
\[
	\vec{x}_{\tilde{\beta}} \coloneqq A^{-1}_{\tilde{\beta}}(b + A_{\beta}\vec{\epsilon}) = A^{-1}_{\tilde{\beta}}b + A^{-1}_{\tilde{\beta}}A_{\beta}\vec{\epsilon}
\]
(\(\vec{x}_{\tilde{\beta_i}}\overset{?}{=}0\))
\[
	i^{\text{th}} \text{ element of }A^{-1}_{\tilde{\beta}}A_{\beta}\begin{pmatrix}
		\epsilon   \\
		\epsilon^2 \\
		\epsilon^3 \\
		\vdots     \\
		\epsilon^m \\
	\end{pmatrix} \implies \underbrace{i^{\text{th}} \text{ row of } A^{-1}_{\tilde{\beta}}A_{\beta}}_{=\vec{0}} \text{ dot }\vec{\epsilon}\ \contd
\]
because \(A^{-1}_{\tilde{\beta}}A_{\beta}\) is invertible(\(A^{-1}_{\beta}A_{\tilde{\beta}}\) is the inverse)