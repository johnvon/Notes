\chapter{Introduction to Linear Programming}
\lecture{1}{30 Aug. 08:00}{Introduction}
\section{General Linear Programming Problem}
A general linear programming problem is to either minimize or maximize an \emph{objective function} in the form of
\[
	c_1 x_1 + c_2 x_2 + \ldots +c_n x_n,
\]
where \(x_{i}\) are our variables, \(i = 1, \ldots, n\), and with the constraints
\[
	\begin{alignedat}{4}
		&a_{11} x_1 + &&\ldots + &&a_{1n}x_n &&\gtreqqless b_1\\
		&a_{21} x_1 + &&\ldots + &&a_{2n}x_n &&\gtreqqless b_2\\
		&\vdots &&\ddots &&\vdots && \\
		&a_{n1} x_1 + &&\ldots + &&a_{nn}x_n &&\gtreqqless b_n\\
	\end{alignedat},
\]
which is sometimes called \emph{structured constraints}, and finally with the constraints
\[
	x_1 \gtreqqless 0, x_2\gtreqqless 0, \ldots ,x_n\gtreqqless 0,
\]
which is called the \emph{signed constraints}.

We called an assignment of values to variable \(x\) as a \emph{solution}, and if this solution satisfies the linear constraints, we say that
this solution is feasible. And a solution is \emph{optimal} if there is no feasible solution with better objective value. Finally, the set of feasible
solutions is called \emph{feasible region.}

\begin{remark}
	A feasible region is a polyhedron.
\end{remark}


\begin{notation}
	We often referred \(\gtreqqless \) to either \(\geq , \leq\) or \(=\).
\end{notation}

We will denote
\[
	c= \begin{pmatrix}
		c_1    \\
		\vdots \\
		c_n    \\
	\end{pmatrix},\quad
	x = \begin{pmatrix}
		x_1    \\
		\vdots \\
		x_n    \\
	\end{pmatrix}, \quad
	A = \begin{pmatrix}
		a_{11} & \ldots & a_{1n} \\
		\vdots & \ddots & \vdots \\
		a_{n1} & \ldots & a_{nn} \\
	\end{pmatrix},\quad
	b = \begin{pmatrix}
		b_1    \\
		\vdots \\
		b_n    \\
	\end{pmatrix}.
\]

It's convenient to only consider so-called \emph{Standard form problem}, which has the form of
\begin{align*}
	\min~ & c^Tx    \\
	      & Ax = b  \\
	      & x\geq 0
\end{align*}
with the condition that rows of \(A\) are linear independent, which means that no redundant equations  and the system is consistent.

\begin{remark}
	Notice that we only consider finitely many of constrains, since the property that the objective function can attain its extremum only on a compact set, which
	requires finite dimensional vector space.
\end{remark}

Surprisingly, every linear programming problem can be converted to standard form, we now see how is this done.
\begin{enumerate}
	\item Sign:\begin{itemize}
		      \item If \(x_{j}\leq 0 \implies x_j \to -x_j^-\), where \(x_{j}^- \geq 0\).
		      \item If \(x_{j}\) is unrestricted \(\implies x_j \to x_j^+ - x_{j}^-\), where \(x_j^{\pm} \geq 0\).
	      \end{itemize}
	\item Constraints:\begin{itemize}
		      \item \(\sum\limits_{j=1}^{n} a_{ij} x_{j} \leq b \implies \sum\limits_{j=1}^{n} a_{ij} x_{j} + s_i = b_{i}\), where \(s_i \geq 0\). This \(s_i\) sometimes called \emph{slack variable}.
		      \item \(\sum\limits_{j=1}^{n} a_{ij} x_{j} \geq b \implies \sum\limits_{j=1}^{n} a_{ij} x_{j} - s_i = b_{i}\), where \(s_i \geq 0\). This \(s_i\) sometimes called \emph{surplus variable}.
	      \end{itemize}
	\item Maximize: \(\max \sum c_{j}x_{j} \implies -\min -\sum c_{j}x_{j}\).
\end{enumerate}




