\lecture{9}{28 Sep. 10:30}{Euclidean \(k\)-Median}
Consider the following problem.

\begin{problem}[Euclidean \(k\)-median]\label{prb:Euclidean-k-median}
Given a \hyperref[def:metric]{metric space} \((X, d) = (\mathbb{\MakeUppercase{r}} ^\ell, \left\lVert \cdot \right\rVert _2 )\) and \(P, Q\subseteq X\) with \(k\in \mathbb{\MakeUppercase{n}} \), find \(Q^\prime \subseteq Q\) with \(\left\vert Q^\prime  \right\vert = k\) which minimizes \(\sum_{i\in P} \min _{j\in Q^\prime } d(i, j)\).
\end{problem}

We see that \autoref{prb:Euclidean-k-median} is like \autoref{prb:k-median}, but with a specific \hyperref[def:metric]{metric space}, i.e., \((\mathbb{\MakeUppercase{r}} ^\ell , \left\lVert \cdot \right\rVert _2)\).

\begin{note}
	We assume that \(\ell \) is large. If it's not the case, then actually for all \(\epsilon > 0\), there exists a \((1+\epsilon )\)-approximation algorithm with running time \(2^{2^{O(\ell )}}\cdot \mathop{\mathrm{poly}}(n)\). Hence, if \(\ell \) is small (or constant), we can use this algorithm. But if it's not, what we're going to see is better.
\end{note}

It's natural to ask that whether we can solve \autoref{prb:Euclidean-k-median} by the similar algorithm which solves \autoref{prb:facility-location} and \autoref{prb:k-median}. In particular, we're going to use \autoref{algo:facility-location-PD}.

\begin{prev}[Dual LP for \autoref{prb:facility-location}]
	Recall the dual of \autoref{prb:facility-location}
	\begin{align*}
		\max~    & \sum_{i} \alpha _i                                  \\
		         & \alpha _i - \beta _{ij} \leq d(i, j) & \forall i, j \\
		         & \sum_{i} \beta _{ij} \leq f          & \forall j    \\
		(D)\quad & \alpha , \beta \geq 0
	\end{align*}
	and the \autoref{algo:facility-location-PD}.
\end{prev}

Again, we think of \(\alpha _i\) as the time that \(i\) is connected, and \(t_j\) be the time that \(j\) is open in \autoref{algo:facility-location-PD}, and the only thing we change is the phase two, i.e., how we trim down the solution. We now see the algorithm, which essentially achieves \(\rho \coloneqq (1 + \delta)\)-\hyperref[def:LMP]{LMP} approximation where \(\delta \coloneqq \sqrt{8 / 3} \approx 1.633\ldots\).\footnote{We'll see why we have this artificial number soon.}

\begin{algorithm}[H]\label{algo:Euclidean-k-median-PD}
	\DontPrintSemicolon
	\caption{\hyperref[prb:Euclidean-k-median]{Euclidean \(k\)-Median} -- Primal-Dual}
	\KwData{A set of clients \(P\subseteq X\), a set of (possible) facilities \(Q\subseteq X\), facility cost \(f\)}
	\KwResult{A set of opened facilities \(Q^\prime \subseteq Q\)}
	\SetKw{Break}{break}
	\BlankLine
	\(S\gets \varnothing \), \(Q^\prime \gets \varnothing \), \(\alpha \gets 0\) \Comment*[r]{\(S\):connected clients, \(O\):open facilities}
	\;
	\While(){\(S \neq P\)}{
	\While(\label{algo:Euclidean-k-median-PD-while-2}){\textsf{True}}{
	increase all \(\left\{ \alpha _i \right\}_{i\in P\setminus S} \) by a \emph{unit}

	\uIf(\label{algo:Euclidean-k-median-PD-if-1}\Comment*[f]{\(j\) gets tight (open)}){some \(j\in Q\setminus Q^\prime \) s.t. \(\sum_{i\in P} \beta _{ij} = f\)}{
		\Break
	}
	\ElseIf(\label{algo:Euclidean-k-median-PD-if-2}\Comment*[f]{\(i\) can connect to \(j\in Q^\prime \)}){some \(i\in P\setminus S\) s.t. \(\alpha_i \geq d(i, j)\)}{
		\Break
	}
	}
	\(Q^\prime \gets\left\{ \text{tight facilities}\right\} \)\label{algo:Euclidean-k-median-PD-lin10}\Comment*[r]{Update \(Q^\prime \)}
	\(S\gets \left\{ \text{clients connected to \(Q^\prime\)} \right\} \)\label{algo:Euclidean-k-median-PD-lin11}\Comment*[r]{Update \(S\)}
	}
	\;
	\label{algo:Euclidean-k-median-PD-comment}\Comment{Trim down \(Q^\prime \)}
	\(G = (Q^\prime , E\coloneqq \left\{ (j, j^\prime )\colon \exists i\in P \text{ such that } \mathcolor{red}{d(j, j^\prime ) \leq \delta \cdot \min (t_j , t_{j^\prime })}, j, j^\prime \in Q^\prime \right\})\)\;
	Compute \(Q^{\prime\prime}\) s.t. \(\forall j\in Q^\prime \), either \(j\in Q^{\prime\prime}\) or \(\exists j^\prime\in Q^{\prime\prime} \) s.t. \((j, j^\prime )\in E\)\Comment*[r]{\hyperref[prb:max-strongly-independent-set]{max independent set}}
	\Return{\(Q^{\prime\prime} \)}\;
\end{algorithm}

\begin{intuition}
	The problematic part is when \(i\) contributing to too many facilities at once, but we'll see that this can't happen if we're considering Euclidean \hyperref[def:metric]{metric}.
\end{intuition}

\begin{remark}
	As before, let \(w(i)\in Q^\prime\) for all \(i\) such that \(\alpha \geq t_{w(i)}\), i.e., \(w(i)\) is the connected witness of \(i\). Then we have the following.
	\begin{enumerate}[(a)]
		\item \(\alpha \) is dual-feasible.
		\item If \(\beta _{ij} > 0\), then \(\alpha _i \leq t_j\).
		\item For all \(i\), \(\exists w(i)\in Q^\prime \) such that \(\alpha_i \geq t_{w(i)}\).
	\end{enumerate}
\end{remark}

We now do the analysis similarly. Fix a client \(i\in P\), then observe that given \(S = Q^{\prime\prime} \cap \left\{ j\colon \beta _{ij} > 0 \right\}\), if \(\delta = 2\), then \(\left\vert S \right\vert \leq 1\).\footnote{Since if both \(\beta _j\) and \(\beta _{j^\prime }\) is greater than \(0\), then \(d(j, j^\prime ) \leq 2\alpha _i \leq 2 \min (t_j , t_{j^\prime })\). This means \(j\) and \(j^\prime \) will have an edge but from the property of \hyperref[prb:max-strongly-independent-set]{max independent set}, one of them will not be included.} We see that
\begin{enumerate}[(a)]
	\item If \(\left\vert S \right\vert = 1\), \(S = \left\{ j \right\} \). We see that \(\mathop{\mathrm{conn}}(i) \leq d(i, j)\) and \(\mathop{\mathrm{open}}(i) = \alpha _i - d(i, j)\), so
	      \[
		      \mathop{\mathrm{conn}}(i) + \mathop{\mathrm{open}}(i) \leq d(i, j) + (\alpha _i - d(i, j)) \leq \alpha _i
	      \]
	\item If \(\left\vert S \right\vert = 0\), then \(\mathop{\mathrm{open}}(i) = 0\) and either \(w(i)\in Q^{\prime\prime}\), or \(j^\prime \in Q^{\prime\prime}\) such that \((w(i), j^\prime )\in E\). In any case, \(\mathop{\mathrm{conn}}(i) \leq d(i, j^\prime ) \leq d(i, w(i)) + d(w(i), j^\prime )\), hence
	      \[
		      \mathop{\mathrm{conn}}(i) + \mathop{\mathrm{open}}(i) \leq d(i, j^\prime )\leq \alpha _i + \delta t_{w(i)} \leq (1 + \delta )\alpha _i.
	      \]
\end{enumerate}

Generally, our goal is to prove that for all \(i\),
\begin{equation}
	\frac{\mathop{\mathrm{conn}}(i)}{\rho} + \mathop{\mathrm{open}}(i) \leq \alpha _i,
\end{equation}
which implies
\[
	\frac{\mathop{\mathrm{conn}}}{\rho } + \left\vert Q^{\prime\prime}\right\vert f \leq \sum_{i} \alpha _i,
\]
i.e., we get a \(\rho \)-\hyperref[def:LMP]{LMP} approximation algorithm.

\begin{remark}
	We see that we get the exactly \(3\)-\hyperref[def:LMP]{LMP} approximation for \(\delta = 2\) case! Notice that in this case, since \(\left\vert S \right\vert \leq 1\) as we noted, \autoref{algo:Euclidean-k-median-PD} and \autoref{algo:facility-location-PD} are equivalent.
\end{remark}

We first see two \emph{magics}. Let \(i^\prime = \frac{1}{s}\sum_{j\in S} j\). Then
\[
	\begin{split}
		\sum_{j\in S} \left\lVert j - i\right\rVert ^{2}
		&= \sum_{j\in S} \left\langle j - i + i^\prime - i^\prime , j - i + i^\prime - i^\prime  \right\rangle \\
		&= \sum_{j} \left( \left\lVert j - i\right\rVert ^{2} + \left\lVert i^\prime - i\right\rVert ^{2} + 2\left\langle j c i^\prime , i^\prime - i \right\rangle  \right) \\
		&= \sum_{j} \left\lVert j - i^\prime \right\rVert ^{2} + s \left\lVert i^\prime - i\right\rVert ^{2} .
	\end{split}
\]

Also,
\[
	\begin{split}
		\sum_{j, j^\prime \in S} \left\lVert j - j^\prime \right\rVert ^{2}
		&= \sum_{j, j^\prime }\left\langle j- j^\prime + i^\prime - i^\prime , j- j^\prime + i^\prime - i^\prime  \right\rangle  \\
		&= \sum_{j, j^\prime } \left( \left\lVert j - i^\prime \right\rVert^{2} + \left\lVert j^\prime - i^\prime \right\rVert ^{2} + 2 \left\langle j -i^\prime , i^\prime - j^\prime  \right\rangle   \right) \\
		&= 2s\cdot \sum_{j\in S} \left\lVert j - i^\prime \right\rVert ^{2} .
	\end{split}
\]

\begin{claim}
	\(\left\vert S \right\vert \leq 3\).
\end{claim}
\begin{explanation}
	From the above tricks, we have
	\[
		3
	\]
\end{explanation}