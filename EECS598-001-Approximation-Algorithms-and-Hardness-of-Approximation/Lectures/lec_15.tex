\lecture{15}{24 Oct. 10:30}{Lasserre Hierarchy and Max Cut}
\section{Lasserre Hierarchy}
The Lasserre hierarchy is a systematic procedure to strengthen a relaxation for an optimization problem by adding additional variables and \hyperref[def:SDP]{SDP} constraints. In the last years this hierarchy moved into the focus of researchers in approximation algorithms as they obtain relaxations have provably nice properties.

\subsection{Local Distributions}
Firstly, recall \hyperref[prb:max-cut]{max cut} and its IP formulation and the \hyperref[def:SDP]{SDP} relaxation. We're interested in whether there's something \emph{between} the IP and this \hyperref[def:SDP]{SDP} relaxation?
\[
	\begin{aligned}
		\max~            & \frac{1}{4} \sum_{(i, j)\in \mathcal{\MakeUppercase{e}} } \left\lVert u_i - u_j\right\rVert _2^2                     \\
		(\text{IP})\quad & \left\lVert u_i\right\rVert _2^2 = 1                                                             & \forall i\in [n],
	\end{aligned} \to \ldots \to \begin{aligned}
		\max~             & \frac{1}{4} \sum_{(i, j)\in \mathcal{\MakeUppercase{e}} }  (x_i - x_j)^{2}                     \\
		(\text{SDP})\quad & x_i \in \left\{ \pm 1 \right\}                                             & \forall i\in [n].
	\end{aligned}
\]

The upshot is that the \hyperref[def:SDP]{SDP} solutions are kind of telling us the second moment information. In order to see this, instead of optimizing over \(\pm 1\), we now optimize over \(\left\{ 0, 1 \right\} \).

\[
	\begin{aligned}
		\max~            & \sum_{(i, j)\in \mathcal{\MakeUppercase{e}} } \left\lVert u_i - u_j\right\rVert _2^2                    \\
		(\text{IP})\quad & \left\lVert u_{\varnothing}\right\rVert _2^2 = 1                                     & \forall i\in [n] \\
		                 & \left\langle u_i, u_{\varnothing } \right\rangle = \left\lVert u_i\right\rVert _2^2  & \forall i\in [n]
	\end{aligned} \to \ldots \to \begin{aligned}
		\max~             & \sum_{(i, j)\in \mathcal{\MakeUppercase{e}} }  (x_i - x_j)^{2}                     \\
		(\text{SDP})\quad & x_i \in \left\{ 0, 1 \right\}                                  & \forall i\in [n].
	\end{aligned}
\]

Now, we ask the following question.

\begin{problem*}
	If \(u_\varnothing \), \(\left\{ u_{i}  \right\} _{i\in [n]}\) are feasible to \(\left\{ 0, 1 \right\} \)-SDP, what does it tell us about the integral solutions?
\end{problem*}
\begin{answer}
	If \(\left\{ u_i \right\} _{i\in [n]}\), \(u_\varnothing \) is \(1\)-dimensional, then this solution \(\left\{ u_i \right\} _{i\in [n]}, u_\varnothing\) encodes a cut \(S \subseteq \mathcal{\MakeUppercase{v}} \) such that
	\[
		S = \left\{ i\in [n] \mid u_i = 1 \right\},
	\]
	in which case we can get
	\[
		\left\langle u_i, u_j \right\rangle = \mathbbm{1}(i, j\in S)
	\]
	for all \(i, j\in [n]\). Now, if \(\left\{ u_i \right\} _{i\in [n]}\), \(u_\varnothing \) is \textbf{not} \(1\)-dimensional, then we hope that we can view the solution \(\left\{ u_i \right\} _{i\in [n]}, u_\varnothing \) is encoding a \emph{distribution} \(\mathcal{\MakeUppercase{d}} \) over cuts \(S \subseteq \mathcal{\MakeUppercase{v}} \), in which case, we can think of
	\[
		\left\langle u_i, u_j \right\rangle = \mathbb{E}_{S\sim \mathcal{\MakeUppercase{D}} }\left[ \mathbbm{1}(i, j\in S)\right]
	\]
	for all \(i, j\in [n]\), i.e., a covariance matrix. But sadly, this is not true in this exact form since a \hyperref[def:PSD]{PSD matrix} doesn't always stand for a covariance matrix of some distribution over \(\left\{ 0, 1 \right\} \)-valued assignments.
\end{answer}

To get at least some versions of what we want, we first introduce a special kind of distribution called \hyperref[def:2-local-distribution]{\(2\)-local distribution}.

\begin{definition}[\(2\)-local distribution]\label{def:2-local-distribution}
	A \emph{\(2\)-local distribution} is a set of distributions consisting of
	\begin{itemize}
		\item \(\widetilde{P} _i\): distribution over \(\left\{ 0, 1 \right\} \)-assignments for \(X_i= \mathbbm{1}(i\in S) \) for all \(i\in [n]\);
		\item \(\widetilde{P} _{ij}\): distribution over \(\left\{ 0, 1 \right\} \)-assignments for \((X_i, X_j)\) for all \((i, j)\in [n]\times [n]\),
	\end{itemize}
	such that For all \(i, j\in [n]\) and for all \(\theta \in \left\{ 0, 1 \right\} \), \emph{local consistency} means
	\[
		\widetilde{P} _i(X_i = \theta ) = \sum_{\theta ^\prime \in \left\{ 0, 1 \right\} } \widetilde{P} _{ij}(X_i=\theta , X_j = \theta ^\prime )= \widetilde{P} _{ij}(X_i = \theta ).
	\]
\end{definition}

\begin{eg}
	If \(\widetilde{P} _i (X_i = \theta ) = 1\) and \(\widetilde{P} _{ij}(X_i = \theta ) = 0\), then this set is not a \hyperref[def:2-local-distribution]{\(2\)-local distribution}.
\end{eg}

Now, consider the \(\left\{ 0, 1 \right\} \)-SDP with local probabilities. The \hyperref[def:2-local-distribution]{\(2\)-local} variables are \(\{ \widetilde{P} _i \} _{i\in [n]} \cup  \{ \widetilde{P} _{ij} \}_{i, j\in [n]} \) with \(\left\{ v_i \right\} _{i\in [n]}\) and \(v_\varnothing \). Then the SDP for \hyperref[prb:max-cut]{max cut} is defined as
\[
	\begin{aligned}
		\max~ & \sum_{(i, j)\in \mathcal{\MakeUppercase{e}} }\left\lVert u_i - u_j\right\rVert _2^2                       \\
		      & \left\langle u_i, u_\varnothing  \right\rangle = \widetilde{P} _i(X_i = 1)          & \forall i\in [n]    \\
		      & \left\langle u_i, u_j  \right\rangle = \widetilde{P} _{ij}(X_i = X_j = 1)           & \forall i, j\in [n] \\
	\end{aligned}
\]

\begin{remark}
	Technically, we should also introduce another distribution \(\widetilde{P} _{\varnothing }(X_\varnothing) = 1\), and \(\widetilde{P} _{ij}\) is defined for all \((i, j)\in ([n]\cup \left\{ \varnothing  \right\} )\times ([n]\cup \left\{ \varnothing  \right\} )\). In this case, the \hyperref[def:SDP]{SDP} constraint reduces to
	\[
		\left\langle  u_i, u_j \right\rangle = \widetilde{P} _{ij}(X_{i} = X_{j} =1)
	\]
	for all \((i, j)\in ([n]\cup \left\{ \varnothing  \right\} )\times ([n]\cup \left\{ \varnothing \right\} )\).
\end{remark}

We first investigate the objective. We see that
\[
	\begin{split}
		\left\lVert u_i - u_j\right\rVert _2^2
		&= \left\lVert u_i\right\rVert _2^2 + \left\lVert u_j\right\rVert _2^2 - 2\left\langle u_{i} , u_{j}  \right\rangle \\
		&= \widetilde{P} _i(X_i = 1)+ \widetilde{P} _i(X_j = 1) - 2 \widetilde{P} _{ij} ( X_i = X_j = 1)\\
		&= \widetilde{P} _{ij}(X_i = 1)+ \widetilde{P} _{ij}(X_j = 1) - 2 \widetilde{P} _{ij} ( X_i = X_j = 1)
		= \widetilde{P} _{ij} ( X_i \neq X_j).
	\end{split}
\]

Also, observe that
\[
	\left\langle u_i, u_j \right\rangle = \mathbb{E}_{\widetilde{P} _{ij}}\left[ x_i x_j\right],
\]
hence we create a matrix \(M\in \mathbb{\MakeUppercase{r}} ^{([n] \cup \left\{ \varnothing  \right\} )\times ([n] \times \left\{ \varnothing  \right\} )}\) such that \(M_{ij} = \mathbb{E}_{_{\widetilde{P} _{ij}}}\left[ x_i x_j\right] \), i.e., \(M = U U ^{\top} \). In this case, the original constraint implies that \(M\) is \hyperref[def:PSD]{PSD}, hence overall, the \hyperref[def:SDP]{SDP} becomes
\[
	\begin{aligned}
		\max~                               & \sum_{(i, j)\in \mathcal{\MakeUppercase{e}} } \widetilde{P} _{ij} (X_i \neq X_j)                                                                              \\
		                                    & \{\widetilde{P} _i\}\cup \{\widetilde{P} _{ij}\}\text{ is a \hyperref[def:2-local-distribution]{\(2\)-local distribution}}                                    \\
		(\mathcal{\MakeUppercase{p}} )\quad & M = \left( \mathbb{E}_{\widetilde{P} _{ij}}\left[X_i X_j \right] \right) _{i, j\in [n]\cup \left\{ \varnothing  \right\} }\text{ is \hyperref[def:PSD]{PSD}},
	\end{aligned}
\]
where we call this \hyperref[def:SDP]{SDP} \(\mathcal{\MakeUppercase{p}} \).

We see that the notion of \hyperref[def:2-local-distribution]{\(2\)-local distribution} can be generalized to arbitrary number \(R\), i.e., we can now look at the so-called \(R\)-\hyperref[def:local-distribution]{local distribution}.

\begin{definition}[Local distribution]\label{def:local-distribution}
	The set of distributions \(\{ \widetilde{P} _A \}_{A \subseteq [n] \cup \left\{ \varnothing  \right\}, \left\vert A \right\vert \leq R} \) is an \emph{\(R\)-local distribution} if for all \(A, B \subseteq [n]\cup \left\{ \varnothing  \right\} \) with \(\left\vert A \cup B \right\vert \leq R\), for all \(C \subseteq A \cup B\) and \(\theta _i \in \left\{ 0, 1 \right\} \),
	\[
		\widetilde{P} _C (X_i = \theta _i\ \forall i\in C)
		= \widetilde{P} _A (X_i = \theta _i\ \forall i\in C)
		= \widetilde{P} _B (X_i = \theta _i\ \forall i\in C).
	\]
\end{definition}

Now, the \(R\)-\hyperref[def:local-distribution]{local version} of \(\mathcal{\MakeUppercase{p}} \) becomes
\[
	\begin{aligned}
		\max~                                                                & \sum_{(i, j)\in \mathcal{\MakeUppercase{e}} } \widetilde{P} _{ij} (X_i \neq X_j)                                                                              \\
		                                                                     & \{\widetilde{P} _A\}_{\left\vert A \right\vert \leq R} \text{ is an \hyperref[def:2-local-distribution]{\(R\)-local distribution}}                            \\
		\mathop{\mathrm{Lass}}\nolimits_R(\mathcal{\MakeUppercase{p}} )\quad & M = \left( \mathbb{\MakeUppercase{e}}_{\widetilde{P} _{A \cup B}} \left[ \prod_{i\in A \cup B} X_i \right] \right) _{A, B}\text{ is \hyperref[def:PSD]{PSD}},
	\end{aligned}
\]
where we call this \hyperref[def:SDP]{SDP} \(\mathop{\mathrm{Lass}}_R(\mathcal{\MakeUppercase{p}} )\), which is how we define Lasserre hierarchy.

\begin{note}
	Notice that \(M\in \mathbb{\MakeUppercase{r}} ^{\binom{[n]}{\leq R / 2} \times \binom{[n]}{\leq R / 2}}\)
\end{note}