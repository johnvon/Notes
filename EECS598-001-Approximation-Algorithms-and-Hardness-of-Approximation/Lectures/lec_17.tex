\lecture{17}{31 Oct. 10:30}{Graph Coloring}
Let's first prove \autoref{thm:PTAS-for-max-cut}.

\begin{proof}[Proof of \autoref{thm:PTAS-for-max-cut}]
	The running time for \autoref{algo:max-cut-PTAS} is clear. Denote \(p_i = \widetilde{P} ^\prime (X_i = 1)\), then we see that the expected fraction of edges cuts is
	\[
		\begin{split}
			&\mathbb{E}_{(i, j)\in \mathcal{E} }\left[ \Pr\nolimits_{\mathrm{ALG}}(X_i \neq X_j) \right]\\
			=& \mathbb{E}_{X_S}\mathbb{E}_{(i, j)\in \mathcal{E} }\left[p_i + p_j - 2 p_i p_j \right]\\
			\geq& \mathbb{E}_{X_S}\mathbb{E}_{(i, j)\in \mathcal{E} }\left[ p_i + p_j - 2 \mathbb{E}_{\widetilde{P} ^\prime }\left[X_i X_j \right] - 2 \left\vert p_i p_j - \mathbb{E}_{\widetilde{P} ^\prime }\left[X_i X_j \right]  \right\vert  \right]\\
			=& \underbrace{\mathbb{E}_{(i, j)\in \mathcal{E} }\left[ \mathbb{E}_{\widetilde{P} }\left[X_i \right] + \mathbb{E}_{\widetilde{P} }\left[X_j \right] - 2 \mathbb{E}_{\widetilde{P} }\left[ X_i X_j\right]  \right]}_{\mathrm{Lass}} - \underbrace{2 \mathbb{E}_{X_S} \mathbb{E}_{(i, j)\in \mathcal{E} }\left[ p_i p_j - \mathbb{E}_{\widetilde{P} ^\prime }\left[ X_i X_j\right] \right]}_{\mathrm{Err}}.
		\end{split}
	\]
	Recall that previously, we only have control on \(\mathop{\mathrm{Cov}}\nolimits_{\widetilde{P} ^\prime }^{2} = \mathop{\mathrm{Cov}}^2(\widetilde{P} \mid X_S\gets \alpha _S)\), which is over the whole \(i, j\sim[n]\). But now the error term (the second term) is only over \((i, j)\sim \mathcal{E} \), hence we define
	\[
		\mathop{\mathrm{Cov}}\nolimits_{\mathcal{E} }^2\left[ \widetilde{P} \mid X_S \gets \alpha _S\right]
		= \mathbb{E}_{X_S} \mathbb{E}_{(i, j)\sim \mathcal{E} }\left[ \mathop{\mathrm{Cov}}\nolimits_{}\left[X_i , X_j \mid S_S \right]^{2}\right].
	\]
	\begin{claim}
		\(\mathop{\mathrm{Cov}}\nolimits_{\mathcal{E} }^2 [ \widetilde{P} \mid S ] \leq \mathop{\mathrm{Cov}}\nolimits_{}[\widetilde{P} \mid X_S \gets \alpha _S ] / \epsilon \leq \epsilon ^3 \).
	\end{claim}
	\begin{explanation}
		We see that
		\[
			\begin{split}
				n^2\cdot \mathop{\mathrm{Cov}}\nolimits_{}^2\left[\widetilde{P} \mid X_S \gets \alpha _S \right]
				&= \mathop{\mathrm{Cov}}\nolimits_{\Sigma }^2\left[\widetilde{P} \mid X_S\gets \alpha _S \right] \\
				&\geq \mathop{\mathrm{Cov}}\nolimits_{\mathcal{E} , \Sigma }^2\left[\widetilde{P} \mid X_S\gets \alpha _S \right]
				= m \cdot \mathop{\mathrm{Cov}}\nolimits_{\mathcal{E} }^2\left[\widetilde{P} \mid X_S \gets \alpha _S \right],
			\end{split}
		\]
		where subscript \(\Sigma \) is when we replace the expectation by summation in the covariance. With the fact that \(m\gets \epsilon n^2\), we're done.
	\end{explanation}
	Now, since we know that \(\mathrm{Lass} \geq \OPT \geq 1 / 2\), we have
	\[
		\mathbb{E}_{(i, j)\in \mathcal{E} }\left[ \Pr\nolimits_{\mathrm{ALG}}(X_i \neq X_j) \right]
		\geq \mathrm{Lass} - 2 \epsilon \geq \mathrm{Lass}(1 - 4 \epsilon ),
	\]
	finishing the proof.
\end{proof}

\section{Graph Coloring}
Return to the \hyperref[def:SDP]{SDP}, first, we introduce a new definition.

\begin{definition}[Coloring]\label{def:coloring}
	Given a graph \(\mathcal{G} =(\mathcal{V} , \mathcal{E} )\), a \emph{(valid) coloring} \(\chi \colon \mathcal{V} \to [c]\) is a function \(\chi \) such that for all \((i, j)\in \mathcal{E} \), \(\chi (i) \neq \chi (j)\).
\end{definition}

Now, consider the following problem.

\begin{problem}[Graph coloring]\label{prb:graph-coloring}
Given a graph \(\mathcal{G} =(\mathcal{V} , \mathcal{E} )\), find a \hyperref[def:coloring]{coloring} \(\chi \colon \mathcal{V} \to [c]\) while minimizing \(c\).
\end{problem}

Before trying to solve the \hyperref[prb:graph-coloring]{graph coloring}, we note that it's trivial to get \(n\)-\hyperref[def:coloring]{coloring} (by using different color for every node). But in fact, this is the best we can do: \hyperref[prb:graph-coloring]{graph coloring} is extremely hard!

\begin{theorem}
	For all \(\epsilon > 0\), it's \(\NP\) to get \(n^{1 - \epsilon }\)-approximation.
\end{theorem}

People start to consider some promise version of \hyperref[prb:graph-coloring]{graph coloring}, i.e., if we directly assume that \(\mathcal{G} \) admits a \(c\)-\hyperref[def:coloring]{coloring}, what can we say?
\begin{itemize}
	\item \(c = 1\): \(\mathcal{G} \) has no edges, hence trivial.
	\item \(c = 2\): \(\mathcal{G} \) is bipartite, so we can color the graph alternatively, so this can be solved exactly.
	\item \(c = 3\): We can do \(\widetilde{O} (n^{1 / 4})\)-approximation (quite shameful...)
\end{itemize}
\begin{remark}[SOTA for \(c=3\)]
	For \(c = 3\), someone showed that we can do \(\widetilde{O} (n^{0.199\ldots  })\), i.e., around \(\widetilde{O} ^{1 / 5}\). Also, \(\omega (1)\)-approximation is \(\NP\)!
\end{remark}

Analogous to \hyperref[prb:max-cut]{max cut}, we design the following \hyperref[eq:graph-coloring]{SDP} relaxation of \hyperref[prb:graph-coloring]{graph coloring} with variables being vectors \(v_i\) for \(i\in \mathcal{V} \).

\begin{equation}\label{eq:graph-coloring}
	\begin{aligned}
		\min~ & 0                                                                             \\
		      & \left\langle v_i, v_i \right\rangle = 1      & \forall i\in \mathcal{V}       \\
		      & \left\langle v_i, v_j \right\rangle = -1 / 2 & \forall (i, j)\in \mathcal{E}.
	\end{aligned}
\end{equation}

\begin{note}
	We don't have an actual objective function!
\end{note}

\begin{claim}
	If \(\mathcal{G} \) is \(3\)-\hyperref[def:coloring]{colorable}, there exists \(\left\{ v_i \right\}_{i\in \mathcal{V} } \) that are feasible for \autoref{eq:graph-coloring}.
\end{claim}
\begin{explanation}
	Consider \(3\) colors \(C_1\), \(C_2\), and \(C_3\), then if \(i\) has \(C_j\), we let \(i\) gets vector \(v_j\) with all vectors \(v_j\), \(v_{j^\prime }\) are \(120^{\circ } \) away.
	\begin{center}
		\incfig{3-coloring}
	\end{center}
\end{explanation}

\subsection{Independent Sets}
Turns out that the \hyperref[def:independent-set]{independent set} is highly related to solving just \autoref{eq:graph-coloring}, as we'll soon see.

\begin{definition}[Independent set]\label{def:independent-set}
	Given a graph \(\mathcal{G} =(\mathcal{V} , \mathcal{E} )\), a set \(S \subseteq \mathcal{V} \) is \emph{independent} if for all \((i, j)\in \mathcal{E} \), either \(i \notin S\) or \(j \notin S\).
\end{definition}

The notion of \hyperref[def:independent-set]{independent set} is useful since we can transform the \hyperref[prb:graph-coloring]{graph coloring} problem into finding \hyperref[def:independent-set]{independent sets} as suggests by \autoref{lma:lec17}.

\begin{lemma}\label{lma:lec17}
	For a graph \(\mathcal{G} = (\mathcal{V} , \mathcal{E} )\), then \(\mathcal{G} \) is \(c\)-\hyperref[def:coloring]{colorable} if and only if \(\mathcal{V} \) can be partitioned to \(V_1, \ldots, V_c\) such that \(V_i\) is \hyperref[def:independent-set]{independent}.
\end{lemma}

With \autoref{lma:lec17}, if we can find large \hyperref[def:independent-set]{independent sets} and partition the graph into not too many of those, we're done. Note that the size of \hyperref[def:independent-set]{independent sets} is related to the maximum degree.

\begin{notation}
	We denote the maximum degree of a graph \(\mathcal{G}=(\mathcal{V} , \mathcal{E} ) \) by \(\Delta \coloneqq \Delta (\mathcal{G} )\coloneqq \max_{v\in \mathcal{V} } \deg(v)\).
\end{notation}

\begin{remark}[Usefullness of \(\Delta \)]
	Given \(\Delta \), we will have a trivial \(\Delta \)-\hyperref[def:coloring]{coloring}; also, we know that we can find \hyperref[def:independent-set]{independent set} with size \(n / (\Delta +1) = \Omega (n \Delta ^{-1} )\).
\end{remark}
\begin{explanation}
	The coloring part is clear. As for finding \hyperref[def:independent-set]{independent set}, we see that by randomly include one vertex to our \hyperref[def:independent-set]{independent set}, we at most \(\Delta \) vertices will be ruled out: they can't be in the \hyperref[def:independent-set]{independent set} now.
\end{explanation}

Now, after solving \autoref{eq:graph-coloring}, notice that we only have feasibility, with the fact that the solution are not guaranteed to be perfectly aligned in exactly three vectors, hence it's a bit confusing what to do next. However, recall that it's also good enough to find a large \hyperref[def:independent-set]{independent set}, and recall the \hyperref[prb:max-cut]{max cut} problem, where we want to maximize the number of edges crossing a cut set, which is similar to what we're trying to do here. So, inspired by which, we can round the solution, but observe the following.

\begin{figure}[H]
	\centering
	\incfig{3-coloring-threshold}
	\caption{If we do the rounding as in \hyperref[prb:max-cut]{max cut}, we may end-up including more than we want, so we set up some threshold.}
	\label{fig:3-coloring-threshold}
\end{figure}

This suggests that we round it with threshold, i.e., consider the following algorithm.

\begin{algorithm}[H]\label{algo:graph-coloring-independent-set}
	\DontPrintSemicolon
	\caption{\hyperref[prb:graph-coloring]{Graph Coloring} -- \hyperref[def:independent-set]{Independent Set} Rounding of \(3\)-\hyperref[def:coloring]{Colorable} Graph}
	\KwData{A \(3\)-\hyperref[def:coloring]{colorable} graph \(\mathcal{G} =(\mathcal{V} , \mathcal{E} )\)}
	\KwResult{An \hyperref[def:independent-set]{independent set} \(S\)}
	\SetKwFunction{Sol}{Solve}
	\BlankLine
	\(\left\{ v_i \in \mathbb{R} ^d\right\} _{i=1}^n \gets\)\Sol{\hyperref[eq:graph-coloring]{SDP}}\;
	\(r \gets \mathcal{N} (0, I_d)\)\label{algo:graph-coloring-independent-set-r}\Comment*[r]{\(r_i \sim \mathcal{N} (0, 1)\)}
	\(S(\epsilon )\gets\left\{ i\in \mathcal{V} \colon \left\langle r, v_i \right\rangle \geq \epsilon   \right\} \)\Comment*[r]{\(\epsilon = \sqrt{2 / 3\cdot \ln \Delta } \) }
	\(S^\prime (\epsilon )= \left\{ i\in S(\epsilon )\colon \nexists j\in S(\epsilon ) \text{ s.t. } (i, j)\in \mathcal{E} \right\} \)\Comment*[r]{Make \(S\) \hyperref[def:independent-set]{independent}}
	\Return{\(S^\prime (\epsilon )\)}\;
\end{algorithm}

\begin{remark}
	\autoref{algo:graph-coloring-independent-set} is the rounding algorithm of \autoref{eq:graph-coloring} in the sense of feasibility,\footnote{Recall that there's no objective in \autoref{eq:graph-coloring}} and notice that it gives us an \hyperref[def:independent-set]{independent set}, rather than a \hyperref[def:coloring]{coloring}.
\end{remark}