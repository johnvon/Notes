\lecture{3}{7 Sep.\ 10:30}{Covering-Packing Duality and Primal-Dual Method}
\section{Covering-Packing Duality}
We first define some useful notions.

\begin{definition}[Strongly independent]\label{def:strongly-independent}
	Given a set system \((U, \mathcal{S} )\), we say \(C\subseteq U\) is \emph{strongly independent} if there does not exist \(S\in \mathcal{S} \) such that \(\left\vert C \cap S \right\vert \geq 2\).
\end{definition}

\begin{remark}
	Then for any \hyperref[def:strongly-independent]{strongly independent set} \(C\subseteq U\), we know that \(\OPT_{\mathrm{SC}} \geq \left\vert C \right\vert\).\footnote{\(\mathrm{SC}\) denotes \hyperref[prb:set-cover]{set cover}.}
\end{remark}

Now, we're trying to find the \textbf{strongest witness} of \hyperref[def:strongly-independent]{strongly independent set}, which suggests we define the following problem.

\begin{problem}[Maximum strongly independent set]\label{prb:max-strongly-independent-set}
Given a set system \((U, \mathcal{S} )\), we want to find the largest \hyperref[def:strongly-independent]{strongly independent set}.
\end{problem}

\begin{remark}
	For any set system, we have \(\OPT_{\mathrm{SIS}}\leq \OPT_{\mathrm{SC}}\).\footnote{\(\mathrm{SIS}\) denotes \hyperref[prb:max-strongly-independent-set]{maximum strongly independent set}.}
\end{remark}

\begin{prev}[LP dual]
	Recall how we get the dual of a given LP:
	\[
		\begin{alignedat}{5}
			\min~    & c^{\top}x\qquad\qquad &  & \max ~   &  & y^{\top} b              \\
			         & Ax \geq b             &  &          &  & y^{\top} A\leq c^{\top} \\
			(P)\quad & x\geq  0              &  & (D)\quad &  & y\geq 0.
		\end{alignedat}
	\]
	Also, recall the weak duality (\(\OPT_{P} \geq \OPT_D\)) and strong duality (\(\OPT_P = \OPT_D\)).
\end{prev}

\begin{definition}[Covering LP]\label{def:covering-LP}
	A primal LP with \(A, b, c \geq 0\) is called a \emph{covering LP}.
\end{definition}

\begin{definition}[Packing LP]\label{def:packing-LP}
	A dual LP with \(A, b, c \geq 0\) is called a \emph{packing LP}.
\end{definition}

We now give another LP formulation for the unweighted \hyperref[prb:set-cover]{set cover}. Given \(\mathcal{S} = \left\{ S_1, \dots , S_m  \right\} \), \(U = \left\{ e_1, \dots , e_n  \right\} \) and define \(A\in \mathbb{R} ^{n\times m}\) such that
\[
	A_{ij} = \begin{dcases}
		1, & \text{ if } e_i\in S_j ; \\
		0, & \text{ otherwise}.
	\end{dcases}
\]

Then our LP is defined as
\[
	\begin{alignedat}{5}
		\min~    & \sum_{j=1}^{m} x_j\qquad\qquad &  & \max ~   &  & \sum_{i =1}^{n} y_i  \\
		         & Ax \geq \bm{1}                 &  &          &  & y^{\top}A\leq \bm{1} \\
		(P)\quad & x\geq  0                       &  & (D)\quad &  & y \geq 0.
	\end{alignedat}
\]

We see that if we restrict \(y_i\in \left\{ 0, 1 \right\} \), we see that the dual \((D)\) is just \autoref{prb:max-strongly-independent-set}. This can be seen via writing the constraint explicitly:
\[
	\sum_{i=1}^{n} A_{ij} y_{i} \leq 1 \iff \sum_{i\colon e_i\in S_j} y_{i} \leq 1 \text{ for } j\in[m].
\]
And indeed, if we look at the weighted version, we have \(\sum_{i\colon e_i\in s_j} y_{i} \leq w(S_j)\).

Now, recall the \hyperref[clm:set-cover-cost]{claim} in \autoref{thm:set-cover-k-approx}, i.e., \(y_{e_i} \leq \frac{w(S_{j})}{k - i + 1}\). We see that the \(y_{e_i} \) are just the dual variables in our setup. Additionally, with the observation that we can do this for any set \(S = \left\{ e_1, \dots , e_k\right\} \in \mathcal{S}\), we have the following lemma.
\begin{lemma}\label{lma:set-cover-greedy}
	The variable \(y^\prime \coloneqq y/H_k\) is dual-feasible, i.e., it's feasible for \((D)\).
\end{lemma}
\begin{proof}
	We see that \(y_{e_i} \geq 0\) (and hence \(y_i\)) trivially, so we only need to show that
	\[
		\sum_{i=1}^{n} A_{ij}y^\prime = \sum_{i=1}^{n} A_{ij}\frac{y_{e_i}}{H_k} \leq w(S_{j})
	\]
	for \(j\in [m]\). But this is trivial by plugging in \(y_{e_i} \leq \frac{w(S_j)}{k - i + 1}\) as shown in \autoref{thm:set-cover-k-approx}, hence
	\[
		\sum_{i=1}^{n} A_{ij} \frac{y_{e_i} }{H_k} \leq \frac{1}{H_k}\sum_{i=1}^{n} A_{ij} \frac{w(S_j) }{k - i + 1} \leq \frac{1}{H_k}\sum_{i=1}^{k} \frac{w(S_j)}{k - i + 1} = w(S_j),
	\]
	and we're done.\footnote{Note that in the above derivation, \(i\) is kind of overloading, i.e., \(e_i\) corresponding to only some \(i\) (confusing, but it's how it is...).}
\end{proof}

With \autoref{lma:set-cover-greedy}, we simply run \autoref{algo:set-cover-greedy} while maintaining \(y_e\) for every \(e\), and we're done.

\begin{theorem}\label{thm:set-cover-greedy}
	\autoref{algo:set-cover-greedy} is an \(H_k\)-approximation algorithm in the view of its dual.
\end{theorem}
\begin{proof}
	Same as \autoref{thm:set-cover-k-approx}, but now we have different interpretation. Specifically, if \(y^\prime = y / H_k\) is dual-feasible, we know that the corresponding objective value of \(y^\prime \) is at most \(\OPT_{\LP_D} = \OPT_{\LP_P}\), which is at most \(\OPT_{\mathrm{SC}}\) further. Now, since we're dealing with LP, everything is linear includes the objective value, i.e., \(y\) is at most \(H_k\cdot \OPT_{\mathrm{SC}}\).
\end{proof}

\begin{remark}[Dual fitting]\label{rmk:dual-fitting}
	The above method is called \emph{dual fitting}, which is universal as one can easily see. The way to do this is the following.
	\begin{enumerate}
		\item Given an algorithm, distribute the algorithm to \(\left\{ y_i \right\}\).
		\item Prove that \(y / \alpha \) is dual-feasible.
		\item This shows the algorithm is \(\alpha\)-approximation algorithm.
	\end{enumerate}
\end{remark}

\section{Primal-Dual Method}
We first see the general description of the so-called \emph{primal-dual method}.

\begin{enumerate}
	\item Maintain \(x\) (primal solution) and \(y\) (dual solution) where \(x\) is integral and infeasible, while \(y\) is fractional and feasible. Start from \(x = y = 0\).
	\item \textbf{Somehow} increase \(y\) until some dual constraints get tight.
	\item \textbf{Choose} primal variables correspond to tight dual constraints, and update input accordingly.
\end{enumerate}

\begin{remark}
	We're using dual variables to get a certificate of the lower bound of the optimal problem we're solving.
\end{remark}

In terms of \hyperref[prb:set-cover]{set cover}, we have the following.

\begin{algorithm}[H]\label{algo:set-cover-PD}
	\DontPrintSemicolon
	\caption{\hyperref[prb:set-cover]{Set cover} -- Primal-Dual}
	\KwData{A \hyperref[def:set-system]{set system} \((U, \mathcal{S})\)}
	\KwResult{A \hyperref[def:covering]{covering} \(\mathcal{S} ^\prime \)}
	\BlankLine

	\(\mathcal{S} ^\prime \gets \varnothing\), \(y \gets 0\)\;
	\While(){\(U \neq \varnothing \)}{
		Choose any \(e\in U\)\;
		Raise \(y_e\) until some constraints get tight\;
		\(\mathcal{S} ^\prime \gets \mathcal{S} ^\prime \cup \left\{ \text{sets corresponding to tight dual constraints} \right\} \)\;
		Update \(U\)\Comment*[r]{Remove newly covered element in \(U\)}
	}
	\Return{\(\mathcal{S} ^\prime \)}\;
\end{algorithm}

\begin{remark}
	\autoref{algo:set-cover-PD} is correct and can be implemented efficiently.
\end{remark}

\begin{theorem}\label{thm:set-cover-PD}
	\autoref{algo:set-cover-PD} is a \(d\)-approximation algorithm.
\end{theorem}
\begin{proof}
	Firstly, \(y\) is feasible. And we see that
	\[
		w(\mathcal{S} ^\prime ) = \sum_{S\in \mathcal{S} ^\prime } w(S) = \sum_{S\in \mathcal{S} ^\prime }\sum_{e\in S}y_e \leq d\cdot \sum_{e\in U} y_e \leq d\cdot \OPT_{\LP_{D}} = d\cdot \OPT_{\LP_P}\leq d\cdot \OPT_{\mathrm{SC} }.
	\]
\end{proof}