\lecture{16}{26 Oct. 10:30}{Lasserre Hierarchy Continued}
Up to this time, we see that
\begin{enumerate}[(a)]
	\item \(\mathop{\mathrm{Lass}}_R(\mathcal{\MakeUppercase{p}})\) is a convex program with \(2^R n^{O(R)}\) variables and constraints.
	\item We can solve this in \(n^{O(R)}\) time.
	\item \(\mathop{\mathrm{Lass}}_2(\mathcal{\MakeUppercase{p}})\) is equivalent to a basic \hyperref[def:SDP]{SDP}, and \(\mathop{\mathrm{Lass}}_R(\mathcal{\MakeUppercase{p}})\) is equivalent to an IP.
\end{enumerate}

\subsection{Probabilistic Consequences}
Consider
\begin{itemize}
	\item \(\mathop{\mathrm{Var}}_{\widetilde{P} _i}\left[ x_i \right] = \mathbb{E}_{\widetilde{P} _i}\left[ x_i ^{2} \right] - \mathbb{E}_{\widetilde{P} _i}\left[ x_i \right]^{2}  \)
	\item \(\mathop{\mathrm{Cov}}_{\widetilde{P} _{ij}}\left[ x_i, x_j\right] = \mathbb{E}_{\widetilde{P} _{ij}}\left[ x_i x_j\right] - \mathbb{E}_{\widetilde{P} _{i}}\left[ x_i \right] \mathbb{E}_{\widetilde{P} _{j}}\left[ x_j\right]  \).
\end{itemize}
We see that we can do a conditioning: let \(\widetilde{P} \coloneqq \{ \widetilde{P} _{A} \} _{\left\vert A \right\vert \leq R}\) be an \(R\)-\hyperref[def:local-distribution]{local distribution}. Now, fix \(S \subseteq [n]\), \(\left\vert S \right\vert = t\), let \(\alpha _S = \left\{ 0, 1 \right\} ^{\left\vert S \right\vert }\), condition on \(\widetilde{P} \), we get
\[
	\widetilde{P} ^\prime = \widetilde{P} \mid x_S \gets \alpha _S,
\]
where \(x_S\gets \alpha _S\) means \(x_i \gets \alpha _i\) for all \(i\in S\).
\begin{remark}
	\(\widetilde{P} ^\prime \) is a \((R-t)\)-\hyperref[def:local-distribution]{local distribution}.
\end{remark}
\begin{explanation}
	For all \(A \subseteq [n]\), \(\left\vert A \right\vert \leq R-t\), we have
	\[
		\widetilde{P} ^\prime _A (x_A = \theta _A) = \frac{\widetilde{P} (x_A = \theta _A, x_S = \alpha _S)}{\widetilde{P} (x_S = \alpha _S)}.
	\]
\end{explanation}

Apart from this, we also see that
\begin{enumerate}[(a)]
	\item \(\widetilde{P} ^\prime \) is \((R-t)\)-wise locally consistent.
	\item If \(\widetilde{P} \) was \(\mathop{\mathrm{Lass}}_R(\mathcal{\MakeUppercase{p}})\), \(\widetilde{P} ^\prime \) is feasible for \(\mathop{\mathrm{Lass}}_{R-t} (\mathcal{\MakeUppercase{p}})\).
\end{enumerate}

\begin{lemma}[Conditioning reduces variance]\label{lma:conditioning-reduces-variance}
	For all \(i, j\in [n]\),
	\[
		\mathop{\mathrm{Var}}\nolimits_{\widetilde{P} _i}\left[x_i \right] - \mathbb{E}_{x_j\sim \widetilde{P} _j}\left[\mathop{\mathrm{Var}}\nolimits_{\widetilde{P} _{ij}}\left[x_i \mid x_j \right]  \right] \geq 4\mathop{\mathrm{Cov}}\nolimits_{\widetilde{P} _{ij}}\left[ x_i, x_j\right] ^{2} .
	\]
\end{lemma}
\begin{proof}
	From of law of total variance, we have
	\[
		\mathop{\mathrm{Var}}\nolimits_{\widetilde{P} _i}\left[x_i \right]  - \mathbb{E}_{x_j}\left[\mathop{\mathrm{Var}}\nolimits_{\widetilde{P} }\left[x_i \mid x_j \right]  \right] = \mathop{\mathrm{Var}}\nolimits_{\widetilde{P} _{j}}\left[\mathbb{E}_{\widetilde{P} _j}\left[x_i \mid x_j \right]  \right] .
	\]
	Now, let \(P_i = \widetilde{P} _i(x_i = 1)\), \(P_j= \widetilde{P} _j(x_j = 1)\), \(P_{ij}= \widetilde{P} _{ij}(x_i=1, x_j=1) \), we have
	\[
		\begin{split}
			\mathop{\mathrm{Var}}\nolimits_{x_j}\left[\mathbb{E}_{\widetilde{P} }\left[x_i \mid x_j \right] \right]
			&= \mathbb{E}_{x_j}\left[\mathbb{E}_{\widetilde{P} }\left[x_i \mid x_j \right] ^{2} \right] - \left( \mathbb{E}_{x_j}\left[ \mathbb{E}_{}\left[x_i\mid x_j \right] \right]  \right)^{2} \\
			&=\widetilde{P} _j(x_j = 1)\cdot \frac{\mathbb{E}_{\widetilde{P} }\left[x_i x_j \right]^{2}}{\widetilde{P} _j(x_j = 1)^{2}} + \widetilde{P} _j(x_j = 0)\cdot \frac{\mathbb{E}_{\widetilde{P} }\left[x_i (1-x_j) \right]^{2}}{\widetilde{P} _j(x_j = 0)^{2}} - \mathbb{E}_{\widetilde{P} }\left[ x_i\right] ^{2} \\
			&=\frac{P_{ij} ^{2} }{P_j} + \frac{(P_i - P_{ij} )^{2} }{1-P_j}- P_i ^{2} \\
			&=\frac{1}{P_j(1-P_j)} \left( P_{ij}^{2} (1-P_j) + (P_i - P_{ij} )^{2} \cdot P_j - P_i ^{2} P_j(1-P_j)  \right)\\
			&= \frac{(P_{ij} - P_i P_j)^{2} }{P_j(1-P_j)}\\
			&= \frac{\left( \mathbb{E}_{\widetilde{P} }\left[ x_i x_j\right] - \mathbb{E}_{\widetilde{P} _i}\left[ x_i\right] \mathbb{E}_{\widetilde{P} _j}\left[x_j \right] \right) ^{2} }{\mathbb{E}_{}\left[x_j^2 \right] - \mathbb{E}_{}\left[x_j \right] ^{2} }\\
			&= \frac{\mathop{\mathrm{Cov}}\nolimits_{\widetilde{P} }\left[x_i, x_j \right]^{2}  }{\mathop{\mathrm{Var}}\nolimits_{\widetilde{P} _i}\left[x_j \right] }.
		\end{split}
	\]
	Since \(x_i\) are \(0\)-\(1\) variable, the variance in the denominator is less than \(1 / 4\), hence we finally have
	\[
		\mathop{\mathrm{Var}}\nolimits_{\widetilde{P} _i}\left[x_i \right]  - \mathbb{E}_{x_j}\left[\mathop{\mathrm{Var}}\nolimits_{\widetilde{P} }\left[x_i \mid x_j \right]  \right] = \mathop{\mathrm{Var}}\nolimits_{x_j}\left[ \mathbb{E}_{\widetilde{P} }\left[x_i \mid x_j \right] \right] \geq 4\cdot \mathop{\mathrm{Cov}}\nolimits_{\widetilde{P} }\left[x_i, x_j \right]^{2} .
	\]
\end{proof}

\begin{corollary}\label{col:lec16}
	Suppose \(\widetilde{P} =\{ \widetilde{P} _A \}_{\left\vert A \right\vert \leq R} \) is an \(R\)-\hyperref[def:local-distribution]{local distribution} which is \(\mathop{\mathrm{Lass}}_R(\mathcal{\MakeUppercase{p}})\) feasible, then
	\[
		\mathbb{E}_{j\sim [n]} \mathbb{E}_{x_j\sim \widetilde{P} _j}\left[ \mathbb{E}_{i\sim [n]}\left[\mathop{\mathrm{Var}}\nolimits_{\widetilde{P} _i}\left[x_i \right]\right] - \mathbb{E}_{i\sim [n]}\left[\mathop{\mathrm{Var}}\nolimits_{\widetilde{P} _{ij}}\left[x_i \mid x_j \right]\right] \right] \geq 4 \mathbb{E}_{i, j\sim [n]}\left[\mathop{\mathrm{Cov}}\nolimits_{\widetilde{P} _{ij}}\left[ x_i, x_j\right]^{2} \right].
	\]
	Furthermore, given \(a\in \mathbb{\MakeUppercase{r}} ^+\), either one of the following will happen.
	\begin{enumerate}[(a)]
		\item \(\mathbb{E}_{i, j\sim [n]}\left[\mathop{\mathrm{Cov}}\nolimits_{\widetilde{P} _{ij}}\left[ x_i, x_j\right]^{2} \right] \leq a\).
		\item \(\exists j\in [n]\), \(\theta _j\in \left\{ 0, 1 \right\} \), \(\widetilde{P} ^\prime \coloneqq \widetilde{P} \mid x_j \gets \theta _j\) satisfies \(\mathbb{E}_{i\sim [n]}\left[\mathop{\mathrm{Var}}\nolimits_{\widetilde{P} _i}\left[x_i \right]\right] - \mathbb{E}_{i\sim [n]}\left[\mathop{\mathrm{Var}}\nolimits_{\widetilde{P}^\prime }\left[x_i \right]\right] \geq 4a\).
	\end{enumerate}
\end{corollary}
\begin{proof}
	We first prove the first statement. \autoref{lma:conditioning-reduces-variance} gives a point-wise inequality, taking the expectation on both sides with the \href{https://en.wikipedia.org/wiki/Dominated_convergence_theorem}{dominanted convergence theorem}, we have
	\[
		\mathbb{E}_{i, j\sim [n]}\left[\mathop{\mathrm{Var}}\nolimits_{}\left[x_i \right] - \mathbb{E}_{x_j}\left[\mathop{\mathrm{Var}}\nolimits_{}\left[x_i \mid x_j \right] \right] \right] \geq 4 \mathbb{E}_{i, j\sim [n]}\left[\mathop{\mathrm{Cov}}\nolimits_{}\left[x_i, x_j \right] ^{2} \right],
	\]
	with the fact that
	\[
		\mathbb{E}_{i, j\sim [n]}\left[\mathop{\mathrm{Var}}\nolimits_{}\left[x_i \right] - \mathbb{E}_{x_j}\left[\mathop{\mathrm{Var}}\nolimits_{}\left[x_i \mid x_j \right] \right] \right] = \mathbb{E}_{j\sim [n]}\mathbb{E}_{x_j}\left[\mathbb{E}_{i\sim [n]}\left[\mathop{\mathrm{Var}}\nolimits_{}\left[x_i \right] \right] - \mathbb{E}_{i\sim [n]}\left[\mathop{\mathrm{Var}}\nolimits_{}\left[x_i\mid x_j \right] \right] \right],
	\]
	hence conclude the first part. A probabilistic argument proves the \emph{either-or} statement.
\end{proof}

\begin{remark}
	\autoref{col:lec16} says that either we have a small covariance, or we can reduce it by a lot.
\end{remark}

\begin{theorem}\label{thm:lec16}
	Suppose \(\widetilde{P} =\{ \widetilde{P} _A \}_{\left\vert A \right\vert \leq R} \) is an \(R\)-\hyperref[def:local-distribution]{local distribution} which is \(\mathop{\mathrm{Lass}}_R(\mathcal{\MakeUppercase{p}})\) feasible and \(R \geq 1 / \epsilon ^4 + 2\), then there exists \(S \subseteq [n]\) such that \(\left\vert S \right\vert \leq 1 / \epsilon ^4\), \(\alpha _S \in \left\{ 0, 1 \right\} ^{\left\vert S \right\vert }\), and \(\widetilde{P} ^\prime \coloneqq \widetilde{P} \mid x_S\gets \alpha _S\), we have
	\[
		\mathbb{E}_{ij\sim [n]}\left[\mathop{\mathrm{Cov}}\nolimits_{\widetilde{P} ^\prime }\left[x_i, x_j \right]^{2} \right] \leq \frac{\epsilon ^4}{4}.
	\]
	Moreover, \(S\) and \(\alpha _S\) can be found in \(\poly(n, 1 / \epsilon )\).
\end{theorem}
\begin{proof}
	We actually have a constructive proof, i.e., we directly give an algorithm which runs in \(\poly(n, 1 / \epsilon )\) and find the desired \(i, j\).

	\begin{algorithm}[H]\label{algo:thm-lec16}
		\DontPrintSemicolon
		\caption{\autoref{thm:lec16} -- Construction}
		\KwData{\(\widetilde{P} \), \(\epsilon > 0\)}
		\KwResult{\(\widetilde{P} ^\prime\) with expected covariance smaller than \(\epsilon ^4 / 4\), \(S\)}
		\BlankLine
		\(\ell \gets 0\), \(\widetilde{P} ^{(\ell )}\gets \widetilde{P} \), \(S\gets \varnothing \)\;
		\For(){\(\ell = 0, 1, \ldots  , 1 / \epsilon ^4\) }{
			\uIf(\label{algo:thm-lec16-if}){\(\mathbb{E}_{i, j}\left[\mathop{\mathrm{Cov}}\nolimits_{\widetilde{P} ^{(\ell )}}\left[x_i, x_j \right] ^{2} \right] \leq \epsilon ^4 / 4\)}{
				\Return{\(\widetilde{P} ^{(\ell )}\)}\Comment*[r]{\(\widetilde{P} ^{(\ell )}= \widetilde{P} \mid x_S\gets \alpha _S\), \(S\)}
			}\Else(\label{algo:thm-lec16-else}){
				Find \(j_{\ell +1}\in [n]\setminus S\), \(\theta _{\ell +1}\in \left\{ 0, 1 \right\} \) \Comment*[r]{Guaranteed in \autoref{col:lec16}}
				\(\widetilde{P} ^{(\ell +1)}\gets \widetilde{P} ^{(\ell )}\mid x_{j_{\ell +1}}\gets \theta _{\ell +1}\)\;
				\(S\gets S \cup \left\{ j _{\ell +1}\right\} \)\;
			}
		}
	\end{algorithm}
	To analyze \autoref{algo:thm-lec16}, observe that if \autoref{algo:thm-lec16} returns, then we have a desired property, so we only need to ensure it'll meet the condition in \autoref{algo:thm-lec16-if} in \(1 / \epsilon ^4\) iterations. Now, for a \hyperref[def:local-distribution]{local distribution} \(Q\), let \(\mathop{\mathrm{Var}}\nolimits_{}\left[Q \right] \coloneqq \mathbb{E}_{i\sim [n]}\left[\mathop{\mathrm{Var}}\nolimits_{Q}\left[x_i \right]  \right] \) and \(\mathop{\mathrm{Cov}}\nolimits_{}\left[Q \right] \coloneqq \mathbb{E}_{i, j\sim [n]}\left[\mathop{\mathrm{Cov}}\nolimits_{Q}\left[x_i, x_j \right]  \right] \). We see that we only fail if in every iteration, we reach \autoref{algo:thm-lec16-else}, i.e., \(\mathop{\mathrm{Cov}}\nolimits_{}[ \widetilde{P} ^{(\ell )}] \leq \epsilon ^4 / 4\) for all \(\ell \). But from \autoref{col:lec16}, we know that the \(\widetilde{P} ^{(\ell +1)}\) we find will have the property that
	\[
		\mathop{\mathrm{Var}}\nolimits_{}[\widetilde{P} ^{(\ell -1 )} ] - \mathop{\mathrm{Var}}\nolimits_{}[\widetilde{P} ^{(\ell)}] \geq 4\cdot \epsilon ^4 / 4 = \epsilon ^4
		\implies \mathop{\mathrm{Var}}\nolimits_{}[ \widetilde{P} ^{(\ell )} ] \leq \mathop{\mathrm{Var}}\nolimits_{}[ \widetilde{P} ^{(\ell -1)}] - \epsilon ^4.
	\]
	By telescoping, we have
	\[
		\mathop{\mathrm{Var}}\nolimits_{}[ \widetilde{P} ^{(1 / \epsilon ^4)} ]
		\leq \mathop{\mathrm{Var}}\nolimits_{}[ \widetilde{P}^{(0)} ] - \frac{1}{\epsilon ^4}\cdot \epsilon ^4
		= \mathop{\mathrm{Var}}\nolimits_{}[ \widetilde{P} ] - 1
		\leq \frac{1}{4} - 1 < 0,
	\]
	a contradiction, and hence we must terminate, finishing the proof.
\end{proof}

\begin{remark}
	\autoref{thm:lec16} says that suppose we have a \hyperref[def:local-distribution]{local distribution} over \(n\) variables with sufficient large locality. Then turns out that there's a small subset of variables, if we fix them, they'll almost determine all other variables.
\end{remark}

\begin{theorem}[\href{https://en.wikipedia.org/wiki/Polynomial-time_approximation_scheme}{PTAS} for max cut]\label{thm:PTAS-for-max-cut}
	For any \(\epsilon >0\), given a graph \(\mathcal{\MakeUppercase{g}} =(\mathcal{\MakeUppercase{v}} , \mathcal{\MakeUppercase{e}} )\) such that \(\left\vert \mathcal{\MakeUppercase{e}}  \right\vert \geq \epsilon n^{2} \), there exists a \((1 - 6\epsilon )\)-approximation algorithm runs in \(n^{O(1 / \epsilon ^4)}\)-time.
\end{theorem}
\begin{proof}
	Again, we give a constructive proof here.

	\begin{algorithm}[H]\label{algo:max-cut-PTAS}
		\DontPrintSemicolon
		\caption{\hyperref[prb:max-cut]{Max Cut} -- \href{https://en.wikipedia.org/wiki/Polynomial-time_approximation_scheme}{PTAS}}
		\KwData{A dense graph \(\mathcal{\MakeUppercase{g}} =(\mathcal{\MakeUppercase{v}} , \mathcal{\MakeUppercase{e}} )\) with \(\left\vert \mathcal{\MakeUppercase{e}}  \right\vert \geq \epsilon n^{2} \), \(\epsilon > 0\)}
		\KwResult{A cut \(S\)}
		\SetKwFunction{rand}{rand}
		\SetKwFunction{Ber}{Ber}
		\SetKwFunction{Sol}{Solve}
		\SetKwFunction{Red}{\hyperref[algo:thm-lec16]{Reduce-Variance}}
		\BlankLine
		\(R\gets 1 / \epsilon ^4 + 2\)\;
		\(\widetilde{P} \coloneqq \{ \widetilde{P} _A \}_{\left\vert A \right\vert \leq R} \gets\)\Sol{\(\mathop{\mathrm{Lass}}_R(\mathcal{\MakeUppercase{p}})\)}\;
		\(\widetilde{P} ^\prime \gets\)\Red{\(\widetilde{P}\)}\Comment*[r]{\(\mathbb{E}_{ij\sim [n]}\left[\mathop{\mathrm{Cov}}\nolimits_{\widetilde{P} ^\prime }\left[x_i, x_j \right]^{2} \right] \leq \epsilon ^4 / 4\)}
		\;
		\label{algo:PTAS-max-cut-comment}\Comment{Rounding}
		\For(){\(i\in \mathcal{\MakeUppercase{v}} \)}{
			\(\lambda _i\gets\)\Ber{\(\widetilde{P} ^\prime (x_i = 1)\)}\;
		}
		\(S\gets \left\{ i\in \mathcal{\MakeUppercase{v}} \colon \lambda _i = 1 \right\} \)\;
		\Return{\(S\)}\;
	\end{algorithm}
	We skip the formal proof here, but the intuition is that since the covariance is small, so independent rounding is almost as good as rounding from the joint distribution, and with some analysis we can ensure what we want.
\end{proof}


We see that as long as the graph is dense enough, we can spend more and more time to get a better approximation, which is the whole point of Lasserre hierarchy.

\begin{remark}
	The rounding method in \autoref{algo:max-cut-PTAS} (i.e., \autoref{algo:PTAS-max-cut-comment}) is ridiculously simple compare to \autoref{algo:max-cut-randomized-rounding}! \(\widetilde{P} ^\prime \) basically tells you everything.
\end{remark}