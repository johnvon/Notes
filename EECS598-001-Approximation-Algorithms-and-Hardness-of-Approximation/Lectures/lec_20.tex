\lecture{20}{9 Nov. 10:30}{FGLSS Graph}
\subsection{An Equivalent PCP Theorem}
With \autoref{def:PCP}, \hyperref[thm:PCP]{PCP theorem} is equivalent as saying that there exists \(\epsilon > 0\) such that
\[
	\NP = \PCP_{1, 1-\epsilon }(O(\log n), O(1)).
\]
It's easy to see that \(\NP \supseteq \PCP_{1, 1-\epsilon }(O(\log n), O(1))\) just by considering iterating through all the possible \(R\). Another direction worth a whole class, so we're not going to dive into that.

Nevertheless, if we accept that \(\NP = \PCP_{1, 1-\epsilon }(O(\log n), O(1))\), we can actually show the equivalence between this and the \hyperref[thm:PCP]{PCP theorem}, i.e., why \hyperref[thm:PCP]{PCP theorem} implies hardness of approximation, specifically, the \hyperref[def:c-s-Gap]{\((1, 1-\epsilon )\)-Gap} \hyperref[prb:max-3SAT]{3SAT} problem.

Firstly, from \hyperref[thm:Cook-Levin]{Cook-Levin theorem}, \hyperref[prb:max-3SAT]{3SAT} is \(\NP=\PCP_{1, 1-\epsilon }(O(\log n), O(1))\) from assumption. Explicitly, we assume \(q = 2\), \(\psi = \left\{ 01, 10 \right\} \), and \(r = O(\log n)\). Then there exists \(V\) such that given a \hyperref[def:k-CNF]{\(3\)-CNF formula} \(\phi \), it runs in \(\poly(\vert \phi  \vert )\) and only flips \(r\) random coins \(R\in \left\{ 0, 1 \right\} ^r\), which decides \(Q_1^R, Q_2^R\) such that
\begin{itemize}
	\item if \(\phi \) is satisfiable, \(\exists y\) such that \(\Pr_{R}(\psi (y_{Q_1^R}, y_{Q_2^R}) = 1) \geq c\);
	\item if \(\phi \) is not satisfiable, \(\forall y\) such that \(\Pr_{R}(\psi (y_{Q_1^R}, y_{Q_2^R}) = 1) \leq s\).
\end{itemize}
Notice that the above event \(\psi (y_{Q_1^R}, y_{Q_2^R})\) is exactly \(y_{Q_1^R} \neq y_{Q_2^R}\). We see that there are at most \(2^r \leq n^{O(1)}\) possible \(R\)'s, and for each \(R\), we access exactly \(2\) positions, so \(V\) will access at most \(N\coloneqq 2\cdot 2^r\) positions. Now, without loss of generality, we may assume that \(\max _R (Q_1^R, Q_2^R) \leq N\).\footnote{Since \(V\) is going to access at most \(N\) positions anyway, we can just rearrange it.}

Consider the optimization problem that finds \(y\in \left\{ 0, 1 \right\} ^N\) to maximize the probability of accepting. In this viewpoint, this is just like \hyperref[prb:max-cut]{max cut} on \(\mathcal{\MakeUppercase{g}} =([N], \left\{ (Q_1^R, Q_2^R)\colon R\in \left\{ 0, 1 \right\} ^r \right\} )\). Namely, we find a \hyperref[def:reduction]{reduction} from \hyperref[prb:max-3SAT]{3SAT} to \hyperref[def:c-s-Gap]{\((c, s)\)-Gap} \hyperref[prb:max-cut]{max cut}.


\section{FGLSS Graph}
To see how we utilize \hyperref[thm:PCP]{PCP theorem}, we first see one example.
\begin{problem}[Vertex cover]\label{prb:vertex-cover}
Given a graph \(\mathcal{\MakeUppercase{g}} =(\mathcal{\MakeUppercase{v}} , \mathcal{\MakeUppercase{e}} )\), find the smallest \(C \subseteq \mathcal{\MakeUppercase{v}} \) that covers all \(\mathcal{\MakeUppercase{e}} \).
\end{problem}

\begin{problem}[Independent set]\label{prb:independent-set}
Given a graph \(\mathcal{\MakeUppercase{g}} =(\mathcal{\MakeUppercase{v}} , \mathcal{\MakeUppercase{e}} )\), find the largest \(I \subseteq \mathcal{\MakeUppercase{v}} \) that contains no edge.
\end{problem}

\autoref{prb:vertex-cover} and \autoref{prb:independent-set} are often considered together due to the following relation.

\begin{claim}
	For all \(\mathcal{\MakeUppercase{g}} \), \(\OPT_{\hyperref[prb:vertex-cover]{\text{VC}}}(\mathcal{\MakeUppercase{g}} ) = \vert \mathcal{\MakeUppercase{v}} \vert - \OPT_{\hyperref[prb:independent-set]{\text{IS}}}(\mathcal{\MakeUppercase{g}} )\).
\end{claim}
\begin{explanation}
	Observe that for all \(C \subseteq \mathcal{\MakeUppercase{v}} \), \(C\) is a \hyperref[prb:vertex-cover]{vertex cover} if and only if \(\mathcal{\MakeUppercase{v}} \setminus C\) is an \hyperref[prb:independent-set]{independent set}.
\end{explanation}

We now see the hardness of \hyperref[prb:vertex-cover]{vertex cover} and \hyperref[prb:independent-set]{independent set} by using the so-called FGLSS graph.~\cite{10.1145/226643.226652} This allows us to do \hyperref[def:reduction]{reduction} from From \hyperref[def:c-s-Gap]{\((1, s)\)-Gap} \hyperref[prb:max-3SAT]{3SAT} with \(s < 1\), given the input
\begin{itemize}
	\item \hyperref[def:k-CNF]{\(3\)CNF formula} \(\phi \);
	\item \(n\) variables \(\left\{ x_1, \ldots , x_n\right\} \eqqcolon X\);
	\item \(2n\) literals \(\left\{ x_1, \overline{x}_1, x_2, \overline{x}_2, \ldots \right\} \eqqcolon L\);
	\item \(m\) clauses with three literals.
\end{itemize}

The, the output is a graph \(\mathcal{\MakeUppercase{g}} =(\mathcal{\MakeUppercase{v}} , \mathcal{\MakeUppercase{e}} )\) such that
\begin{itemize}
	\item \(\mathcal{\MakeUppercase{v}} = [m]\times \left( \left\{ T, F \right\} ^3 \setminus (F, F, F) \right)\) with \(\vert \mathcal{\MakeUppercase{v}}  \vert = 7m\)
	\item \(\big((i, \ell _{i_1}, \ell _{i_2}, \ell _{i_3}), (j, \ell _{j_1}, \ell _{j_2}, \ell _{j_3})\big)\in \mathcal{\MakeUppercase{e}} \) if they \emph{contradict}.
\end{itemize}

\begin{note}[FGLSS graph]
	Such a graph is called a FGLSS graph.
\end{note}

For \hyperref[prb:independent-set]{independent set}, \(c = m\); and for \hyperref[prb:vertex-cover]{vertex cover}, \(c = 6m\).

\paragraph*{Completeness.}
Suppose \(\phi \) is satisfiable, then there exists \(\sigma \colon X \to \left\{ T, F \right\} \) that satisfies every \(C_i\). Then from each \(V_i\), choose a vertex \emph{consistent} with \(\sigma \), i.e., \(\OPT_{\hyperref[prb:independent-set]{\text{IS}} }(\mathcal{\MakeUppercase{g}} ) \geq m\), and \(\OPT_{\hyperref[prb:vertex-cover]{\text{VC}} }(\mathcal{\MakeUppercase{g}} ) \leq 6m\).

\paragraph*{Soundness.}
We want to show that \(\OPT_{\hyperref[prb:independent-set]{\text{IS}} }(\mathcal{\MakeUppercase{g}} ) \geq s\cdot m\) and \(\OPT_{\hyperref[prb:vertex-cover]{\text{VC}} }(\mathcal{\MakeUppercase{g}} ) \leq (7+s)m\), which implies \(\OPT_{\hyperref[prb:independent-set]{\text{IS}} }(\phi) \geq s\). Let \(I \subseteq \mathcal{\MakeUppercase{v}} \) be an \hyperref[def:independent-set]{independent set} such that \(\vert I \vert \geq s\cdot m\), and let \(\sigma \colon X \to \left\{ T, F \right\} \) such that for all \(C_i\) with \(\vert I \cap V_i \vert = 1\), then assign variables in \(C_i\) according to \(I \cap V_i\).\footnote{This is well-defined, i.e., there are no contradictions since \(I\) is an \hyperref[def:independent-set]{independent set}.} We then extend it arbitrarily for unassigned variables, then for all \(C_i\), such that \(\vert I \cap  Vvi \vert = 1\), i.e, \(C_i\) is satisfied by \(\sigma \). We see that \(\beta = s < 1\), \(\alpha = (7-s) / 6 > 1\).

\subsection{Label Cover}
Consider the following problem.

\begin{problem}[Label cover]\label{prb:label-cover}
Given a bipartite graph \(\mathcal{\MakeUppercase{g}} =(\mathcal{\MakeUppercase{u}} \cup \mathcal{\MakeUppercase{v}} , \mathcal{\MakeUppercase{e}} )\) with label sets \(L\) (for \(\mathcal{\MakeUppercase{u}} \)) and \(R\) (for \(\mathcal{\MakeUppercase{v}} \)) with \(\vert R \vert \geq \vert L \vert \) such that for all \(e=(u, v)\in \mathcal{\MakeUppercase{e}} \), we have \(\Pi _e \colon [R]\to [L]\). Find an assignment \(\sigma \) such that
\[
	\sigma = \begin{dcases}
		\mathcal{\MakeUppercase{u}} \to L; \\
		\mathcal{\MakeUppercase{v}} \to R
	\end{dcases}
\]
that satisfies the maximum number of edge, where \(e=(u, v)\) is satisfied by \(\sigma \) if \(\Pi _e(\sigma (v)) = \sigma (u)\).
\end{problem}

For \hyperref[prb:label-cover]{label cover}, the parameters are \(\vert \mathcal{\MakeUppercase{u}}  \vert, \vert \mathcal{\MakeUppercase{v}}  \vert , \vert \mathcal{\MakeUppercase{e}}  \vert , L, R\).

\begin{remark}[Baseline]
	There is a trivial \(1/L\)-approximation algorithm.
\end{remark}
\begin{explanation}
	Consider a random assignment \(\sigma \) such that
	\begin{itemize}
		\item for all \(v\in \mathcal{\MakeUppercase{v}} \), \(\sigma (v)\) randomly from \([R]\);
		\item for all \(u\in \mathcal{\MakeUppercase{u}} \), \(\sigma (u)\) randomly from \([L]\).
	\end{itemize}
	Fix \(e=(u, v)\), we see that \(\Pr_{}(\Pi _e(\sigma (v)) = \sigma (u)) = 1 / L \).
\end{explanation}

\begin{theorem}[PCP theorem with parallel repetitoin theorem]
	For all \(\epsilon > 0\), there exists \(L, R\) such that the \hyperref[def:c-s-Gap]{\((1, \epsilon )\)-Gap} \hyperref[prb:label-cover]{label cover} for \(L, R\) is \(\NP\)-hard.
\end{theorem}

\subsection{Hardness of Max \(k\)-Coverage}


