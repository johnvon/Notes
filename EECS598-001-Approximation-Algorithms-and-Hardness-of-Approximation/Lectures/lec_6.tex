\lecture{6}{19 Sep. 10:30}{Facility Location with LMP Approximation}
\subsection{Hardness}
For \autoref{prb:facility-location}, we have the following.
\begin{enumerate}[(a)]
	\item \(1.488\)-approximation~\cite{LI201345}
	\item \(1.463\)-approximation is \(\NP\)-hard~\cite{GUHA1999228}
\end{enumerate}

Turns out that specifically for \autoref{prb:facility-location}, we can have a more refine notion of approximation ratio defined below.

\begin{definition}[LMP approximation]\label{def:LMP}
	An algorithm \(\mathrm{ALG}\) which solves \hyperref[prb:facility-location]{facility location} is called \emph{\(\gamma \)-Lagrangian multiplier preserving approximation} (LMP-approximation) if
	\[
		\frac{\mathop{\mathrm{conn}}(\mathrm{ALG})}{\gamma }+ \mathop{\mathrm{open}}(\mathrm{ALG}) \leq \sum_{i} \alpha _i
	\]
	for some \(\gamma > 0\).\footnote{The opening cost is just \(k^\prime f\) if \(\mathrm{ALG}\) opens \(k^\prime \) centers.}
\end{definition}
\begin{remark}
	The notion of \hyperref[def:LMP]{LMP approximation} is due to Lagrangian multiplier in the field of optimization, where the dual variables  are treated as a Lagrangian multipliers. And \autoref{def:LMP} says that we're not approximating \(k^\prime f\) at all, hence it's \emph{preserving}.
\end{remark}

And indeed, we now have a more refined characterization about \autoref{algo:facility-location-PD}.

\begin{corollary}
	\autoref{algo:facility-location-PD} is a \(3\)-\hyperref[def:LMP]{LMP approximation} algorithm.
\end{corollary}

\begin{remark}[SOTA]
	If we look at the SOTA result in terms of \hyperref[def:LMP]{LMP}, we have the following.
	\begin{enumerate}[(a)]
		\item \(3\)-\hyperref[def:LMP]{LMP approximation}~\cite{10.1145/375827.375845}
		\item \(2\)-\hyperref[def:LMP]{LMP approximation}~\cite{10.1145/509907.510012}
		\item \(1.99\ldots 9 \)-\hyperref[def:LMP]{LMP approximation}~\cite{https://doi.org/10.48550/arxiv.2207.05150}
		\item \(1.73\)-\hyperref[def:LMP]{LMP approximation}\footnote{The number comes from \(1 + 2 / e\).} is \(\NP\)-hard~\cite{10.1145/509907.510012}
	\end{enumerate}
\end{remark}

\subsection{Greedy Method}
Let's take another look at \autoref{prb:facility-location} and see it as an instance of \autoref{prb:set-cover} where the universe is all the clients \(P\), while the collection of sets are pairs of facility and its connected clients, i.e., \hyperref[def:cluster]{clusters}. Then, it's natural to consider using a similar algorithm as \autoref{algo:set-cover-greedy} to solve this.

\begin{algorithm}[H]\label{algo:facility-location-greedy}
	\DontPrintSemicolon
	\caption{\hyperref[prb:facility-location]{Facility location} -- Greedy}
	\KwData{A set of clients \(P\subseteq X\), a set of (possible) facilities \(Q\subseteq X\), facility cost \(f\)\footnote{We didn't use it explicitly in the algorithm since we hide it in the cost function \(c(\cdot)\).}}
	\KwResult{A set of opened facilities \(Q^\prime \subseteq Q\)}
	\SetKw{Break}{break}
	\BlankLine

	\(S \gets \varnothing \), \(Q^\prime \gets \varnothing\)\;
	\While(){\(S \neq P \)}{
		choose \((j, T)\in Q \times \mathcal{P} (P\setminus X)\) with minimum \(c((j, T)) / \left\vert T \right\vert \)\;
		\(Q^\prime \gets Q^\prime  \cup \left\{ j \right\}\)\;
		\(S \gets S \cup T\)\;
	}
	\Return{\(Q^\prime \)}\;
\end{algorithm}

This is just \autoref{algo:set-cover-greedy}, hence we have \(H_n\)-approximation. But as we have seen in \autoref{thm:lec5}, we have achieved a constant approximation ratio for \autoref{prb:facility-location}. Hence, we should be able to do better based on \autoref{algo:facility-location-greedy}.

\begin{remark}
	If we modify \autoref{algo:facility-location-greedy} such that for all \((j, T)\), if \(j\) is open, then we define the cost of this \hyperref[def:cluster]{cluster} as
	\[
		c((j, T))\coloneqq \frac{\sum_{i\in T} d(i, j)}{\left\vert T \right\vert }.
	\]
	We'll achieve \(1.861\)-approximation, but the analysis is complex.
\end{remark}

Instead, we're going to see other variations based on \autoref{algo:facility-location-greedy}.

\subsubsection{First Modification}
We see observe that \(c((j, T)) / \left\vert T \right\vert \) is increasing in \autoref{algo:facility-location-greedy}.\todo{proof!} Also, if \(\alpha \coloneqq c((j, T)) / \left\vert T \right\vert\), then for all \(i\in T\), \(d(i, j) \leq \alpha \) where we interpret this as \(i\) \emph{pays} \(\alpha _i\) to cover the connection cost \(d(i, j)\) and the opening cost \(\alpha _i - d(i, j)\) of \(j\). Following this intuition, if we change \autoref{algo:facility-location-PD-if-1} in \autoref{algo:facility-location-PD} (with only first phase) such that the summation is over \(P\setminus S\), it becomes exactly \autoref{algo:facility-location-greedy}.

\begin{algorithm}[H]\label{algo:facility-location-greedy-I}
	\DontPrintSemicolon
	\caption{\hyperref[prb:facility-location]{Facility location} -- Greedy Modification I}
	\KwData{A set of clients \(P\subseteq X\), a set of (possible) facilities \(Q\subseteq X\), facility cost \(f\)}
	\KwResult{A set of opened facilities \(Q^\prime \subseteq Q\)}
	\SetKw{Break}{break}
	\BlankLine
	\(S\gets \varnothing \), \(Q^\prime \gets \varnothing \), \(\alpha \gets 0\) \Comment*[r]{\(S\):connected clients, \(Q^\prime \):open facilities}
	\;
	\While(){\(S \neq P\)}{
	\While(){\textsf{True}}{
	increase all \(\left\{ \alpha _i \right\}_{i\in P\setminus S} \) by a \emph{unit}\;
	\uIf(\Comment*[f]{\(j\) gets tight (open)}){some \(j\in Q\setminus Q^\prime \) s.t. \(\sum_{i\in P\mathcolor{red}{\setminus S}} \beta _{ij} = f\)}{
		\Break
	}
	\ElseIf(\Comment*[f]{\(i\) can connect to \(j\in Q^\prime \)}){some \(i\in P\setminus S\) s.t. \(\alpha_i \geq d(i, j)\)}{
		\Break
	}
	}
	\(Q^\prime \gets \mathcolor{red}{Q^\prime  \cup \left\{ j \right\}}\)\Comment*[r]{Update \(Q^\prime \)}
	\(S\gets \mathcolor{red}{S \cup \left\{ i\in P\setminus S\colon \alpha _i \geq d(i, j)\right\} }\)\Comment*[r]{Update \(S\)}
	}
	\Return{\(Q^\prime \)}\;
\end{algorithm}

\begin{remark}
	Since \autoref{algo:facility-location-PD-if-1} and \autoref{algo:facility-location-PD-if-2} can happen simultaneously, while what we just said assumes the opposite, so we need to further modify \autoref{algo:facility-location-PD} in \autoref{algo:facility-location-PD-lin10} and \autoref{algo:facility-location-PD-lin11}.
\end{remark}

\subsubsection{Second Modification}
Another potential modification gives us a \(1.61\)-approximation. We essentially allow \(i\in S\) to \emph{switch} in \autoref{algo:facility-location-greedy-I}, i.e., after \(i\) connects to \(j\), if \(j^\prime \) is closer to \(i\) later, \(i\) can \emph{offer} with \(d(i, j) - d(i, j^\prime )\) to other facilities.

\begin{algorithm}[H]\label{algo:facility-location-greedy-II}
	\DontPrintSemicolon
	\caption{\hyperref[prb:facility-location]{Facility location} -- Greedy Modification II}
	\KwData{A set of clients \(P\subseteq X\), a set of (possible) facilities \(Q\subseteq X\), facility cost \(f\)}
	\KwResult{A set of opened facilities \(Q^\prime \subseteq Q\)}
	\SetKw{Break}{break}
	\BlankLine
	\(S\gets \varnothing \), \(Q^\prime \gets \varnothing \), \(\alpha \gets 0\) \Comment*[r]{\(S\):connected clients, \(Q^\prime \):open facilities}
	\;
	\While(){\(S \neq P\)}{
	\While(){\textsf{True}}{
	increase all \(\left\{ \alpha _i \right\}_{i\in P\setminus S} \) by a \emph{unit}\;
	\If(){some \(j\in Q\setminus Q^\prime \) s.t. \(\mathcolor{red}{\sum_{i\in S} (d(i, w(i)) - d(i, j))^+ } + \sum_{i\in P\setminus S} \beta _{ij} = f\)\footnote{We define \(a^+ \coloneqq \max (a, 0)\) and also, \(w(i)\) is now the \emph{current} facility \(i\) is connected to.}}{
			\Break
		}
		\If(\Comment*[f]{\(i\) can connect to \(j\in Q^\prime \)}){some \(i\in P\setminus S\) s.t. \(\alpha_i \geq d(i, j)\)}{
			\Break
		}
	}
	\(Q^\prime \gets Q^\prime  \cup \left\{ j \right\} \)\Comment*[r]{Update \(Q^\prime \)}
	\(S\gets S \cup \left\{ i\in P\setminus S\colon \alpha _i \geq d(i, j)\right\} \)\Comment*[r]{Update \(S\)}
	}
	\Return{\(Q^\prime \)}\;
\end{algorithm}

\subsubsection{Third Modification}
If we run \autoref{algo:facility-location-greedy-II} with facility cost being \(\hat{f} \coloneqq 2f\), we can have a \(2\)-\hyperref[def:LMP]{LMP approximation} algorithm as follows.

\begin{algorithm}[H]\label{algo:facility-location-greedy-III}
	\DontPrintSemicolon
	\caption{\hyperref[prb:facility-location]{Facility location} -- Greedy Modification III}
	\KwData{A set of clients \(P\subseteq X\), a set of (possible) facilities \(Q\subseteq X\), facility cost \(f\)}
	\KwResult{A set of opened facilities \(Q^\prime \subseteq Q\)}
	\SetKw{Break}{break}
	\BlankLine
	\(S\gets \varnothing \), \(Q^\prime \gets \varnothing \), \(\alpha \gets 0\) \Comment*[r]{\(S\):connected clients, \(Q^\prime \):open facilities}
	\;
	\While(){\(S \neq P\)}{
	\While(){\textsf{True}}{
	increase all \(\left\{ \alpha _i \right\}_{i\in P\setminus S} \) by a \emph{unit}\;
	\If(){some \(j\in Q\setminus Q^\prime \) s.t. \(\sum_{i\in S} (d(i, w(i)) - d(i, j))^+ + \sum_{i\in P\setminus S} \beta _{ij} = \mathcolor{red}{\hat{f}}\)\footnote{We define \(a^+ \coloneqq \max (a, 0)\) and also, \(w(i)\) is now the \emph{current} facility \(i\) is connected to.}}{
			\Break
		}
		\If(\Comment*[f]{\(i\) can connect to \(j\in Q^\prime \)}){some \(i\in P\setminus S\) s.t. \(\alpha_i \geq d(i, j)\)}{
			\Break
		}
	}
	\(Q^\prime \gets Q^\prime  \cup \left\{ j \right\} \)\Comment*[r]{Update \(Q^\prime \)}
	\(S\gets S \cup \left\{ i\in P\setminus S\colon \alpha _i \geq d(i, j)\right\} \)\Comment*[r]{Update \(S\)}
	}
	\Return{\(Q^\prime \)}\;
\end{algorithm}

It's clear that in \autoref{algo:facility-location-greedy-III}, the connection cost plus \(2\) times the opening cost is \(\sum_{i\in P} \alpha _i\) from how we design the algorithm by changing the facility cost from \(f\) to \(\hat{f} \coloneqq 2f\). Now, a crucial lemma is the following.

\begin{lemma}\label{lma:lec6}
	\((\alpha ^\prime , \beta ^\prime )\) is dual feasible, where \(\alpha ^\prime \coloneqq \alpha  / 2\), \(\beta ^\prime _{ij}\coloneqq (\alpha ^\prime _{i} - d(i, j))^+\).
\end{lemma}
\begin{proof}
	It's sufficient to consider \(j\in Q\) and prove that \(\sum_{i\in P^\prime } \beta ^\prime _{ij} \leq f \) where \(P^\prime \coloneqq \left\{ i\colon \beta ^\prime _{ij} > 0 \right\} = [n]\) where we're overloading \(n\) here. Let's order \(\alpha _i\) such that \(\alpha _1 \leq \ldots \leq \alpha _n \) where \(\alpha _i\) is the time \(i\) when \(i\) is \emph{first connected}.

	\begin{claim}
		For all \(i, k\in P^\prime \) such that \(\alpha _k \leq \alpha _i\), at time (right before) \(\alpha _i\), offer from \(k\) to \(j\)\footnote{We assume \(k\) currently (or is going to) connects to \(j^\prime \).} is at most \(\alpha _i - d(i, j) - 2d(k, j)\) for any \(j\in Q\).
	\end{claim}
	\begin{explanation}
		We see that if \(\alpha _i = \alpha _k\), the offer is just \((\alpha _k - d(i, j))^+\). Otherwise, we have \(\alpha _k < \alpha _i\). If \(\alpha _i > d(k, j^\prime ) + d(k, j) + d(i, j)\), we immediately get a contradiction since from triangle inequality, \(\alpha _i > d(i, j^\prime )\), i.e., \(i\) already connect to \(j^\prime \). Hence,
		\[
			\alpha _i \leq d(k, j^\prime )+ d(k, j) + d(i, j).
		\]
		Then, the offer from \(k\) to \(j\) is \((d(k, j^\prime ) - d(k, j))^+ \geq \alpha _i - d(i, j) - 2d(k, j)\).
	\end{explanation}

	Observe that for all \(i\in [n]\), we have
	\begin{equation}\label{eq:lec6}
		\sum_{k=1} ^{i-1} (\alpha _i - d(i, j) - 2d(k, j)) + \sum_{k=i} ^n (\alpha _i - d(k, j)) \leq \hat{f}
	\end{equation}
	by considering the total offer from \(k\) to \(j\) at time (right before) \(\alpha _i\). Now, we add \autoref{eq:lec6} for all \(i\in [n]\), we have
	\[
		n \sum_{i=1} ^n \alpha _i - \sum_{i=1} ^n (i-1)d(i, j) - 2 \sum_{k=1} ^n (n - k)d(k, j) - \sum_{i=1} ^n k\cdot d(k, j) \leq n \hat{f} = 2n f.
	\]
	Since the summation over \(k\) is just indexes, we can change it to \(i\), hence
	\[
		\begin{split}
			2n f &\geq n \sum_{i=1} ^n \alpha _i - \sum_{i=1} ^n (i-1)d(i, j) - 2 \sum_{i=1} ^n (n - i)d(i, j) - \sum_{i=1} ^n i\cdot d(i, j)\\
			&\geq n \sum_{i=1} ^n \alpha _i - \sum_{i=1} ^n id(i, j) - 2 \sum_{i=1} ^n (n - i)d(i, j) - \sum_{i=1} ^n i\cdot d(i, j) = n \sum_{i=1} ^n \alpha _i - \sum\limits_{i=1}^{n} 2n d(i, j)
		\end{split}
	\]
	where we turn the factor \((i-1)\) into \(i\) and gather the terms together. Clean up a bit, we have
	\[
		n \sum\limits_{i=1}^{n} \alpha _i - 2n \sum\limits_{i=1}^{n} d(i, j) \leq 2nf \iff \frac{\sum_{i=1} ^n \alpha _i}{2} - \sum_{i=1} ^n d(i, j) \leq f,
	\]
	finishing the proof.
\end{proof}

From \autoref{lma:lec6}, we immediately have the following.

\begin{theorem}
	\autoref{algo:facility-location-greedy-III} is a \(2\)-\hyperref[def:LMP]{LMP approximation} algorithm w.r.t.\ the original \(f\).
\end{theorem}