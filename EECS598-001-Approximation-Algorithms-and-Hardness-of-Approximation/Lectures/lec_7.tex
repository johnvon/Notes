\lecture{7}{21 Sep. 10:30}{\(k\)-Median and LMP Approximation}
\section{\(k\)-Median}

Let's look at another \hyperref[prb:clustering]{clustering} problem.

\begin{problem}[\(k\)-median]\label{prb:k-median}
Given a \hyperref[def:metric]{metric space} \((X, d)\) and \(P, Q\subseteq X\) with \(k\in \mathbb{\MakeUppercase{n}} \), find \(Q^\prime \subseteq Q\) with \(\left\vert Q^\prime  \right\vert = k\) which minimizes \(\sum_{i\in P} \min _{j\in Q^\prime } d(i, j)\).
\end{problem}

The natural linear programming is the following. Consider \(\left\{ x_{ij}  \right\}_{i\in P, j\in Q} \) and \(\left\{ y_j \right\}_{j\in Q} \), then
\begin{align*}
	\min~ & \sum_{ij} x_{ij} d(i, j)                                             \\
	      & \sum_{j} x_{ij} \geq 1   & \forall i\in P         &  & (\alpha _i)   \\
	      & x_{ij} \leq y_j          & \forall i\in P, j\in Q &  & (\beta _{ij}) \\
	      & \sum_{j} y_j \leq k
	      & x, y\geq 0
\end{align*}

The dual is then
\begin{align*}
	\max~ & \sum_{i} \alpha _i - kf                                    \\
	      & \sum_{i} \beta _{ij} \leq d(i, j) & \forall i\in P, j\in Q \\
	      & \sum_{i} \beta _{ij} \leq f       & \forall j\in Q         \\
	      & \alpha , \beta \geq 0
\end{align*}

\begin{definition}[LMP approximation]\label{def:LMP}\todo{Fix}
	An algorithm is called \emph{\(\gamma \)-Lagrangian multiplier preserving approximation} (LMP-approximation) if
	\[
		\frac{\mathop{\mathrm{conn}}(\text{ALG})}{\gamma }+ k^\prime f \leq \sum_{i} \alpha _i
	\]
	for some \(\gamma > 0\).
\end{definition}

\begin{remark}
	Suppose we guessed \(f\), and after running \(\gamma\)-\hyperref[def:LMP]{LMP approximation} algorithm and somehow get \(k^\prime = k\). Then we have
	\[
		\frac{\mathop{\mathrm{conn}}(\mathrm{ALG})}{\gamma } \leq \sum_{i} \alpha _i - kf \leq \OPT_{\text{\(k\)-med.}},
	\]
	i.e., this is a \(\gamma \)-approximation algorithm.
\end{remark}

Actually, we don't have ideas about the relation between \(k\) and \(f\). In particular, we have very little knowledge about which, the only thing we know is it behaves quite arbitrary. In this case, our previous technique doesn't work anymore, a new idea is then to maintain \([f^2, f^1]\) such that \(f^2 \leq f^1\), where
\begin{itemize}
	\item At \(f^2\), the algorithm opens \(k^2 \geq k\) facilities.
	\item At \(f^1\), the algorithm opens \(k^1 \leq k\) facilities.
\end{itemize}
Then, by binary search, we can get \(f_2 \leq f_1\) such that
\[
	\left\vert f^1 - f^2 \right\vert \leq \frac{\epsilon \OPT}{n}.
\]

\subsection{Bipoint Rounding}

Now, if \(a\in [0, 1]\) and \(b \coloneqq 1 - a\), we have \(k \coloneqq ak^1 + bk^2\) where \(k^1 \leq k \leq k^2\). Denote \(C^i\) as the connection cost with \(f^i\) such that \(C^1 \geq C^2\), we have
\[
	\begin{dcases}
		C^1 + \gamma k^1 f^1 \leq \gamma \sum_{i} \alpha _i^1, & \quad (\times a) \\
		C^2 + \gamma kr2 f^2 \leq \gamma \sum_{i} \alpha _i^2, & \quad (\times b)
	\end{dcases}
\]
hence,
\[
	aC^1 + bC^2
	\leq \gamma \left( a \sum_{i} \alpha _i^1 + b \sum_{i} \alpha _i^2 - ak^1 f^1 - bk^2 f^2 \right)
	\leq \gamma \underbrace{\left( \sum_{i} \alpha _i - kf \right)}_{\leq \OPT_{\text{\(k\)-med.}}} + \underbrace{\vphantom{\left( \sum_{i} \alpha _i - kf \right)}\gamma k \left\vert f^1 - f^2 \right\vert}_{\epsilon \OPT_{\text{\(k\)-med.}}},
\]
where we set \(\alpha \coloneqq a \alpha ^1 + b \alpha ^2\) and \(f \coloneqq \max (f^1, f^2)\).

\begin{note}
	\((\alpha , f)\) is dual-feasible for \autoref{prb:k-median}.
\end{note}

\begin{definition}[Bipoint solution]\label{def:bipoint-solution}
	Given \(F^1\), \(F^2\) with \(\left\vert F^1 \right\vert = k^1\), \(\left\vert F^2 \right\vert = k^2\) and \(k=ak^1 + bk^2\) for \(a, b\in [0, 1]\), \(a + b = 1\). If the connection cost of \(aF^1 + b F^2\) satisfies
	\[
		a\cdot \mathop{\mathrm{conn}}(F^1) + b\cdot \mathop{\mathrm{conn}}(F^2) \leq \gamma \cdot \OPT_{\text{\(k\)-med.}},
	\]
	we then call such a solution a \emph{bipoint solution}.
\end{definition}

\begin{remark}[\(\delta \)-bipoint rounding]
	For \(F^1\) and \(F^2\), output \(F\) with \(\left\vert F \right\vert = k\) such that
	\[
		\mathop{\mathrm{conn}}(F) \leq \delta \cdot (a C^1 + b C^2).
	\]
\end{remark}

\subsubsection{\(2\)-Bipoint Rounding}
For \(j\in F^1\), let \(\pi (j)\)  be the closest facility in \(F^2\) to \(j\). Let \(F^{\ast} = \left\{ j^\prime \in F^2 \colon j^\prime =\pi (j) \text{ for some }j\in F^1  \right\} \). We see that \(\left\vert F^{\ast}  \right\vert \leq k^1\). If \(\left\vert F^{\ast}  \right\vert < k^1\), then we add arbitrary centers so that \(\left\vert F^{\ast}  \right\vert = k^1\).

To open the facilities as what we want, consider the following rounding algorithm.
\begin{enumerate}
	\item Open \(F^1\) with probability \(a\), otherwise open \(F^{\ast} \).
	\item Randomly choose \(k - k_1\) far from \(F^2 \setminus F^{\ast} \) .
\end{enumerate}

Then,
\begin{enumerate}
	\item \(j\in F^1\), \(\Pr(\text{\(j\) open} )= a\)
	\item \(j\in F^{\ast}\), \(\Pr(\text{\(j\) open} )= b\)
	\item \(j\in F^2 \setminus F^{\ast}\), \(\Pr(\text{\(j\) open} )= \frac{k-k_1}{k_2 - k_1}= b\)
\end{enumerate}

Now, fix \(i\in P\), we have
\begin{table}[H]
	\centering
	\begin{tabular}{c|c|c}
		\toprule
		                               & distance       & probability                  \\
		\midrule
		\(j^2\) open                   & \(d_2\)        & \(b\)                        \\
		\(j^2\) not open, \(j^1\) open & \(d_1\)        & \(\geq (a-b)^+ \eqqcolon M\) \\
		none of \(j^1, j^2\) open      & \(2d_1 + d_2\) & \(\leq 1 - b - M\)           \\
		\bottomrule
	\end{tabular}
\end{table}

Then, the expected cost is just
\[
	\mathbb{E}\left[\text{\(i\)'s connected cost}\right] \leq b d_2 + Md_1 + (1 - b - M) (2d_1 + d_2).
\]

\begin{itemize}
	\item If \(b \geq a\),\footnote{We have \(b \geq 1 / 2\).} then \(M = 0\) and
	      \[
		      \mathbb{E}\left[\text{\(i\)'s connected cost}\right] \leq b - d_2 + (1 - b)	(2 d_1 + d_2) = 2ad_1 + d_2 \leq 2(ad_1 + bd_2).
	      \]
	\item If \(a > b\), then \(a > 1 / 2\), \(M = a - b\) and
	      \[
		      \mathbb{E}\left[\text{\(i\)'s connected cost}\right] \leq bd_2 ( (a - b))d_1 + b(2d_1 + d_2) = d_1(a+b)+d_2(b+d) \leq 2(ad_1 + bd_2).
	      \]
\end{itemize}

\begin{remark}[SOTA]
	The SOTA result
\end{remark}