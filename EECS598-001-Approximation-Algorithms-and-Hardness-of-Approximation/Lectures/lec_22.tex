\chapter{Unique Games and the Conjecture}
\lecture{22}{16 Nov. 10:30}{3LIN and BLR Test}

In this section, we're going to study a special problem of \hyperref[prb:label-cover]{label cover}, called the \hyperref[prb:unique-game]{unique games}. As we have already seen, the hardness of \hyperref[prb:label-cover]{label cover} already implies the hardness of problems like \hyperref[prb:max-k-coverage]{max \(k\)-coverage}, as shown in \autoref{thm:max-k-coverage}.

In particular, if we assume a hardness conjecture called \hyperref[conj:unique-game]{unique game conjecture}, we can show the optimal hardness for \hyperref[prb:max-3LIN]{3LIN}, \hyperref[prb:max-3SAT]{3SAT}, and \hyperref[prb:max-cut]{max cut}. Let's first consider the first two.

\section{Optimal Hardness for 3LIN and 3SAT}
Let consider the following problem.
\begin{problem}[MAX-3LIN]\label{prb:max-3LIN}
Given \(X\coloneqq \left\{ x_1, \dots , x_n  \right\} \), \(\Sigma = \mathbb{F} _2\), and a set of \(m\) equations in the form of \(x_i + x_j + x_k = 0\) or \(1\). The problem \emph{MAX-3LIN} asks to find \(\sigma \colon X\to \mathbb{F} _2\) that maximizes the fraction of the satisfied equations.
\end{problem}

\begin{remark}
	We often call \hyperref[prb:max-3LIN]{MAX-3LIN} as 3LIN for brevity.
\end{remark}

We're going to show the hardness of \hyperref[prb:max-3LIN]{3LIN}, but first, note the following.
\begin{claim}
	\hyperref[def:c-s-Gap]{\((1, 1)\)-Gap} \hyperref[prb:max-3LIN]{3LIN} can be solved in polynomial time.
\end{claim}
\begin{explanation}
	Consider solving
	\[
		\begin{bmatrix}
			1      & 0      & \dots  & 1      & 1      \\
			1      & 1      & \dots  & 0      & 1      \\
			\vdots & \vdots & \ddots & \vdots & \vdots \\
			1      & 0      & \dots  & 0      & 0      \\
		\end{bmatrix}_{m\times n} \begin{bmatrix}
			x_1    \\
			x_2    \\
			\vdots \\
			x_n    \\
		\end{bmatrix}_{n\times 1} = \begin{bmatrix}
			b_1    \\
			b_2    \\
			\vdots \\
			b_m    \\
		\end{bmatrix}_{m\times 1},
	\]
	where the coefficient matrix has only three non-zero entries for each row. Then, if this is an accepted instance, this system of equations has a solution, and we can check this just by Gaussian elimination over \(\mathbb{F} _2\).
\end{explanation}

But what if the instance has \(\OPT = 1 - \epsilon \) for some small \(\epsilon > 0\)? In this case, this question is actually hard.

\begin{remark}
	A trivial approximation for \hyperref[prb:max-3LIN]{3LIN} is just to do a random assignment, which gives us a \(1 / 2\)-approximation.
\end{remark}

The hardness result of \hyperref[prb:max-3LIN]{3LIN} we're going see is the following.

\begin{theorem}[\cite{10.1145/258533.258536}]\label{thm:3LIN}
	For every constant \(\epsilon >0\), the \hyperref[def:c-s-Gap]{\((1 - \epsilon , 1 / 2 + \epsilon )\)-Gap} \hyperref[prb:max-3LIN]{3LIN} is \(\NP\)-hard.
\end{theorem}

Which implies the following.

\begin{corollary}
	For every constant \(\epsilon > 0\), the \hyperref[def:c-s-Gap]{\((1-\epsilon , 7 / 8 + \epsilon )\)-Gap} \hyperref[prb:max-3SAT]{3SAT} is \(\NP\)-hard.
\end{corollary}
\begin{proof}
	Given \hyperref[prb:max-3LIN]{3LIN} instance, we create an \hyperref[prb:max-3SAT]{3SAT} instance
	\[
		(x_i + x_j + x_k = 0) \implies \begin{dcases}
			(\overline{x} _i \lor x_j \lor x_k);   \\
			(x _i \lor \overline{x} _j \lor x_k);  \\
			(x _i \lor x _j \lor \overline{x} _k); \\
			(\overline{x} _i \lor \overline{x} _j \lor \overline{x} _k);
		\end{dcases},\quad
		(x_i + x_j + x_k = 1) \implies \begin{dcases}
			(\overline{x} _i \lor \overline{x} _j \lor x_k);   \\
			(x _i \lor \overline{x} _j \lor \overline{x} _k);  \\
			(\overline{x}  _i \lor x _j \lor \overline{x} _k); \\
			(x_i \lor x _j \lor x _k).
		\end{dcases}
	\]
	We see that
	\begin{itemize}
		\item a \hyperref[prb:max-3LIN]{3LIN} equation is satisfied \(\iff \) \(4\) corresponding \hyperref[prb:max-3SAT]{3SAT} clauses are satisfied;
		\item a \hyperref[prb:max-3LIN]{3LIN} equation is unsatisfied \(\iff \) \(3\) corresponding \hyperref[prb:max-3SAT]{3SAT} clauses are satisfied.
	\end{itemize}
	So,
	\[
		\OPT_{\hyperref[prb:max-3SAT]{\text{3SAT}}}
		= \OPT_{\hyperref[prb:max-3LIN]{\text{3LIN}}} \cdot \frac{4}{4} + \left( 1 - \OPT_{\hyperref[prb:max-3LIN]{\text{3LIN}}} \right) \cdot \frac{3}{4}
		= \frac{3}{4} + \frac{1}{4}\cdot \OPT_{\hyperref[prb:max-3LIN]{\text{3LIN}}},
	\]
	and hence the \hyperref[def:c-s-Gap]{\((1 - \epsilon , 1 / 2 + \epsilon )\)-Gap} hardness for \hyperref[prb:max-3LIN]{3LIN} from \autoref{thm:3LIN} implies the \hyperref[def:c-s-Gap]{\((1-\epsilon , 7 / 8 + \epsilon )\)-Gap} hardness for \hyperref[prb:max-3SAT]{3SAT}.
\end{proof}

\begin{remark}
	Actually, with more work, \hyperref[def:c-s-Gap]{\((1 , 7 / 8 + \epsilon )\)-Gap} \hyperref[prb:max-3SAT]{3SAT} is also \(\NP\)-hard.
\end{remark}

\begin{note}
	Recall that a random assignment (such as \autoref{algo:random-max-3SAT}) of \hyperref[prb:max-3SAT]{3SAT} satisfies \(7 / 8\) fraction of clauses from \autoref{lma:random-max-3SAT}. This suggests that both \hyperref[prb:max-3SAT]{3SAT} and \hyperref[prb:max-3LIN]{3LIN} is hard: we can't do better than random assignment.
\end{note}

So, we will embark a long journey to prove \autoref{thm:3LIN} from \hyperref[prb:label-cover]{label cover}, i.e., we again want to find a good assignment \(\sigma \colon U \sqcup V \to L \sqcup R\) by using the hypercube construction. To do this, we need to study Fourier analysis over \(\left\{ \pm 1 \right\} ^n\) of a \hyperref[def:boolean-function]{boolean function}.

\section{Fourier Analysis of Boolean Functions}
We now introduce a powerful tool which is well-known in engineering, Fourier analysis. While it's powerful for infinite-domain function classes, it's not that well-known in the class of finite-domain functions. However, in the latter case, it turns out to be still powerful.

\subsection{Boolean Functions and Boolean-Valued Functions}
Firstly, we introduce the \hyperref[def:boolean-function]{boolean function}.

\begin{definition*}
	Let \(\mathbb{F} _2\) be the additive group over \(\mathbb{F} _2 = \left\{ 0, 1 \right\} \) and consider the conical isomorphism to the multiplicative group \(\left\{ \pm 1 \right\} \).
	\begin{definition}[Boolean funciton]\label{def:boolean-function}
		A function \(f\) is a \emph{boolean function} if \(f\colon \left\{ \pm 1 \right\} ^n \to \mathbb{R} \).
	\end{definition}

	\begin{definition}[Boolean-valued]\label{def:boolean-valued}
		If the range of a \hyperref[def:boolean-function]{boolean function} \(f\) is \(\left\{ \pm 1 \right\} \), we say \(f\) is a \emph{boolean-valued} function.
	\end{definition}
\end{definition*}

\begin{note}
	Since the domain of a \hyperref[def:boolean-function]{boolean function} has cardinality \(2^n\), we can identify it as a \(2^n\)-dimensional vector.
\end{note}

Consider viewing the set of \hyperref[def:boolean-function]{boolean functions} as a Hilbert space, we then define the following inner product between \(f, g\) as
\[
	\left\langle f, g \right\rangle
	= \frac{1}{2^n} \sum_{x\in \left\{ \pm 1 \right\} ^n} f(x) g(x)
	\eqqcolon \mathbb{E}_{x}\left[f(x) g(x) \right] .
\]
\begin{note}
	We have \(\lVert f \rVert _2 ^2 = \left\langle f, f \right\rangle = \mathbb{E}_{x}\left[f(x)^2 \right] \).
\end{note}

Now, we want to know what are the orthonormal basis for the set of \hyperref[def:boolean-function]{boolean functions}. There are two important examples:
\begin{enumerate}[(a)]
	\item Standard basis: For all \(x\in \left\{ \pm 1\right\}^n\),
	      \[
		      f_x \colon \left\{ \pm 1 \right\} ^n \to  \mathbb{R} ,\quad
		      f_x (y) = \begin{dcases}
			      2^{n / 2}, & \text{ if } x=y ;   \\
			      0,         & \text{ otherwise} .
		      \end{dcases}
	      \]
	      We see that \(\left\{ f_x \right\} _{x\in \left\{ \pm 1 \right\} ^n}\) is an orthonormal basis since for all \(x\), \(\lVert f_x \rVert _2^2 = 1\) and \(\left\langle f_x, f_y \right\rangle = 0\) for all \(x \neq y\).
	\item Fourier basis: For all \(S \subseteq [n]\), define \(\chi _\varnothing (x) \coloneqq 1\) and
	      \[
		      \chi _S \colon \left\{ \pm 1 \right\} ^n \to  \left\{ \pm 1 \right\} ,\quad
		      \chi _S (x) = \prod_{i\in S} x_i \eqqcolon x^S.
	      \]
	      We see that \(\left\{ \chi _S \right\}_{S \subseteq [n]} \) is an orthonormal basis since \(\mathbb{E}_{x}\left[\chi _S(x)^2 \right] = 1\), and
	      \[
		      \left\langle \chi _S, \chi _T \right\rangle
		      = \mathbb{E}_{x}\left[\chi _S(x) \chi _T(x) \right]
		      = \mathbb{E}_{x}\left[x^S x^T \right]
		      = \mathbb{E}_{x}\left[x^{S \Delta T} \right]
		      = \begin{dcases}
			      0, & \text{ if } S \neq T ; \\
			      1, & \text{ if } S = T .
		      \end{dcases}
	      \]
\end{enumerate}

\subsection{Fourier Analysis over Boolean Functions}
We'll study the Fourier basis primarily. Firstly, we have the following decomposition of \(f\colon \left\{ \pm 1 \right\} ^n \to  \mathbb{R} \) as
\[
	f = \sum_{S \subseteq [n]} \hat{f} (S) \chi _S,
\]
where we call \(\hat{f} (S)\) the Fourier coefficient. Now, here is some basic facts and theorem.

\begin{proposition}
	Given an orthonormal basis \(\left\{ \chi _S \right\}_{S\subseteq [n]} \) of the space of \hyperref[def:boolean-function]{boolean functions}, then \(\hat{f} (S) = \left\langle f, \chi _S \right\rangle \).
\end{proposition}
\begin{proof}
	Since \(\left\langle f, \chi _S \right\rangle
	= \left\langle \sum_{T \subseteq [n]} \hat{f} (T) \chi _T, \chi _S \right\rangle
	= \sum_{T} \hat{f} (T) \left\langle \chi _T, \chi _S \right\rangle
	= \hat{f} (S)\).
\end{proof}

\begin{theorem}[Plancherel's theorem]\label{thm:Plancherel}
	Given two \hyperref[def:boolean-function]{boolean functions} \(f, g\), we have
	\[
		\left\langle f, g \right\rangle = \mathbb{E}_{x}\left[f(x)g(x) \right] = \sum_{S \subseteq [n]} \hat{f} (S) \hat{g} (S).
	\]
\end{theorem}
\begin{proof}
	Since \(\left\langle f, g \right\rangle
	= \left\langle \sum_{S} \hat{f} (S) \chi _S, \sum_{T} \hat{f} (T) \chi _T \right\rangle
	= \sum_{S, T} \hat{f} (S) \hat{g} (T) \left\langle \chi _S, \chi _T  \right\rangle
	= \sum_S \hat{f}(S)\hat{g}(S)\).
\end{proof}

\begin{theorem}[Parseval's theorem]\label{thm:Parseval}
	Given a \hyperref[def:boolean-function]{boolean function} \(f\), \(\lVert f \rVert _2^2 = \sum_{S \subseteq [n]} \hat{f} (S)^2\).
\end{theorem}
\begin{proof}
	This directly follows from \hyperref[thm:Plancherel]{Plancherel's theorem} with \(\lVert f \rVert _2^2 = \left\langle f, f \right\rangle \).
\end{proof}

\begin{claim}
	Given a \hyperref[def:boolean-function]{boolean function} \(f\), \(\mathbb{E}_{x}\left[f(x) \right] = \hat{f} (\varnothing )\).
\end{claim}
\begin{explanation}
	Since \(\mathbb{E}_{x}\left[f(x) \right]
	= \mathbb{E}_{x}\left[f(x) \cdot 1 \right]
	= \mathbb{E}_{x}\left[\chi _\varnothing (x) f(x) \right]
	= \left\langle \chi _\varnothing , f \right\rangle
	= \hat{f} (\varnothing )\).
\end{explanation}

\begin{claim}
	Given a \hyperref[def:boolean-function]{boolean function} \(f\), \(\Var_{}\left[f \right] = \sum_{S \neq \varnothing } \hat{f} (S)^{2}  \).
\end{claim}
\begin{explanation}
	Since \(\Var_{}\left[f \right]
	= \mathbb{E}_{x}\left[f^2 \right] - \left( \mathbb{E}_{x}\left[f \right]  \right) ^2
	= \sum_{S \subseteq [n]} \hat{f} (S)^2 - \hat{f} (\varnothing )^2
	= \sum_{S \neq \varnothing } \hat{f} (S)^2\).
\end{explanation}

\section{BLR Test and the Noisy BLR Test}
Our goal is to show the \hyperref[def:reduction]{reduction} from \hyperref[prb:label-cover]{label cover} to \hyperref[prb:max-3LIN]{3LIN} by designing a \hyperref[prb:max-3LIN]{3LIN} instance on variables \(U \times \left\{ \pm 1 \right\} ^L\) identified by hypercube. Compared to the \hyperref[prb:max-k-coverage]{max \(k\)-coverage}, now the hypercubes are variables.
\begin{center}
	\incfig{MkC-compare-to-3LIN}
\end{center}

Ideally, \((u, x)\) gets \(x_{\sigma (u)}\), which is just a particular base of the Fourier basis called \hyperref[def:dictation]{dictation}.

\begin{definition}[Dictation]\label{def:dictation}
	The \emph{dictation function} of \(i\) is defined as \(\chi _i(x) \coloneqq \chi _{\left\{ i \right\} } (x) = x_i\).
\end{definition}

For convenience, we sometimes call a \hyperref[def:dictation]{dictation function} as \emph{dictator} for short.

\begin{intuition}
	Consider assignment \(\alpha \colon \left\{ \text{variables of \hyperref[prb:max-3LIN]{3LIN}}\right\} = U \times \left\{ \pm 1 \right\}^L \to \left\{ \pm 1 \right\} \), then fix \(u\in U\), and let \(f\colon \left\{ \pm 1 \right\} ^L \to  \left\{ \pm 1 \right\} \) given by \(f(x) = \alpha (u, x)\). Ideally, \(f(x) = \chi _{\sigma (u)}\), which means \(f\in \left\{ \chi _1, \dots , \chi _L  \right\} \).
\end{intuition}

\begin{problem*}
	But since the number of possible \(f\)'s is \(2^{2^L}\), how can we \emph{force} \(f\) to be in \(\left\{ \chi _i \right\}_{i = 1}^L\), or even \(\left\{ \chi _S \right\}_{S \subseteq [L]} \)?
\end{problem*}
\begin{answer}
	We can design a \hyperref[prb:max-3LIN]{3LIN} instance on variables \(\left\{ \pm 1 \right\} ^n\) such that if the \emph{assignment} \(f\colon \left\{ \pm 1 \right\} ^n \to  \left\{ \pm 1 \right\} \) is \emph{close} to some \(\chi _S\), \(\mathop{\mathrm{obj}}(f)\)\footnote{The \(\mathop{\mathrm{obj}}(f)\) is the fraction of equations satisfied by \(f\).} is large; if it is \emph{far} from any \(\chi _S\), then \(\mathop{\mathrm{obj}}(f)\) is small.
\end{answer}