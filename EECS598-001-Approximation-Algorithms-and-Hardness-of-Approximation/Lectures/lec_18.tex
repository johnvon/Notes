\lecture{18}{2 Nov.\ 10:30}{Graph Coloring via Independent Sets Decomposition}
We're now interested in how large the \hyperref[def:independent-set]{independent set} \autoref{algo:graph-coloring-independent-set} outputs. To start analyzing, since the \(r\) sampled in \autoref{algo:graph-coloring-independent-set-r} is Gaussian, recall the following.

\begin{prev}[\href{https://en.wikipedia.org/wiki/Normal_distribution}{Gaussian distribution}]
	The probability density function for Gaussian distribution is
	\[
		p(x) = \frac{1}{\sqrt{2\pi } }e^{- x^{2} / 2},
	\]
	and the cumulated density function is
	\[
		\Phi (x) = \Pr_{g}(g \leq x) = \int_{-\infty}^{x} p(x) \,\mathrm{d}s, \quad \overline{\Phi }(x) = \Pr_{g}(g \geq x) = 1 - \Phi (x).
	\]
\end{prev}

From the spherical symmetry, we have \(\left\langle r, v_i \right\rangle \sim \mathcal{N} (0, 1)\) for all \(i\in \mathcal{V} \). Moreover, since the probability of \(i\) being in \(S(\epsilon )\) is exactly \(\overline{\Phi }(\epsilon )\), from the linearity of expectation, we have \(\mathbb{E}_{}\left[\left\vert S(\epsilon ) \right\vert\right] = n\cdot \overline{\Phi }(\epsilon )\).

\begin{lemma}\label{lma:lec18}
	\(\Pr_{}(i \notin S^\prime (\epsilon )\mid i\in S(\epsilon )) \leq \Delta \overline{\Phi }(\sqrt{3}\epsilon)\).
\end{lemma}
\begin{proof}
	Fix any \((i, j)\in \mathcal{E} \), it's sufficient to show \(\Pr_{}(j\in S(\epsilon )\mid i\in S(\epsilon )) \leq \overline{\Phi}(\sqrt{3}\epsilon)\). And from the fact that all \(v_j\) are \(120^{\circ } \) apart, we hence can write
	\[
		v_j = - \frac{1}{2} v_i + \frac{\sqrt{3}}{2} u
	\]
	where \(\left\lVert u\right\rVert = 1\) and \(u\perp v_i\). If \(j\in S(\epsilon )\), then
	\[
		\left\langle v_j , r \right\rangle \geq \epsilon
		\implies \underbrace{\left\langle -\frac{1}{2}v_i, r \right\rangle}_{\leq - \epsilon / 2} + \left\langle \frac{\sqrt{3}}{2}u, r \right\rangle \geq \epsilon
		\implies \left\langle \frac{\sqrt{3} }{2} u, r \right\rangle \geq \frac{3\epsilon}{2}
		\implies \left\langle u, r \right\rangle \geq \sqrt{3}\epsilon .
	\]
	Since if \(u\perp v\in \mathbb{R} ^d\), \(\left\langle u, r \right\rangle \) and \(\left\langle v, r \right\rangle \) are independent, so \(\Pr_{}(j\in S(\epsilon )\mid i\in S(\epsilon )) \leq \overline{\Phi }(\sqrt{3} \epsilon )\) as desired.
\end{proof}

We can now prove that the \hyperref[def:independent-set]{independent set} found by \autoref{algo:graph-coloring-independent-set} is large.

\begin{theorem}\label{thm:lec18}
	\autoref{algo:graph-coloring-independent-set} finds an \hyperref[def:independent-set]{independent set} of size \(\Omega (n\cdot \Delta ^{-1 / 3}\log^{-1 / 2} \Delta )\) for any \(3\)-\hyperref[def:coloring]{colorable} \(\mathcal{G} \).
\end{theorem}
\begin{proof}
	We see that
	\[
		\mathbb{E}_{}\left[\left\vert S^\prime (\epsilon ) \right\vert  \right]
		= \sum_{i\in \mathcal{V} } \underbrace{\Pr_{}(i\in S(\epsilon ))}_{\overline{\Phi }(\epsilon )}\cdot \Pr_{}(i\in S^\prime (\epsilon )\mid i\in S(\epsilon ))
		\geq  \sum_{i\in \mathcal{V} } \overline{\Phi }(\epsilon )\cdot \left( 1 - \Delta \overline{\Phi }(\sqrt{3}\epsilon  ) \right),
	\]
	from \autoref{lma:lec18}. Now, observe the following.

	\begin{claim}
		If \(x \geq 10\), \(p(x) / 2x \leq \overline{\Phi}(x) \leq p(x) / x\).
	\end{claim}
	\begin{explanation}
		Since we know that
		\[
			\frac{x}{1+x^{2} }\cdot p(x) \leq \overline{\Phi }(x) \leq \frac{1}{x}\cdot p(x)
		\]
		for all \(x\), if \(x \geq 10\), \(x / (1 + x^{2} ) \geq 1 / 2x\), hence we're done.
	\end{explanation}

	With the above claim, we have \(\overline{\Phi }(\epsilon )\geq p(\epsilon ) / 2\epsilon \) with \(\overline{\Phi }(\sqrt{3}\epsilon) \leq p(\sqrt{3}\epsilon) / 3\epsilon\), hence
	\[
		p(\sqrt{3} \epsilon )
		= \frac{1}{\sqrt{2\pi } } e^{- 3\cdot (2 / 3 \cdot \ln \Delta ) / 2}
		= \frac{1}{\sqrt{2\pi } }\cdot \frac{1}{\Delta }
		\text{ and }p(\epsilon )
		= \frac{1}{\sqrt{2\pi } } e^{- 2 / 3 \cdot \ln \Delta / 2}
		= \frac{1}{\sqrt{2\pi } }\frac{1}{\Delta ^{1 / 3}},
	\]
	leading to
	\[
		\mathbb{E}_{}\left[\left\vert S^\prime (\epsilon ) \right\vert  \right]
		\geq \sum_{i\in \mathcal{V} } \underbrace{\left( \frac{1}{2\epsilon }\cdot p(\epsilon ) \right)}_{\geq \Omega (\frac{1}{\sqrt{\ln \Delta } } \frac{1}{\Delta ^{1 / 3}})} \cdot \underbrace{\left( 1 - \Delta \cdot \frac{1}{3\epsilon }\cdot p(\sqrt{3} \epsilon ) \right)}_{\geq 1 / 2}
		\geq \Omega (n\cdot \Delta ^{-1 / 3} \ln ^{-1 / 2}\Delta ).
	\]
\end{proof}

\subsection{Independent Sets Decomposition}
From \autoref{lma:lec17}, we can iteratively find large \hyperref[def:independent-set]{independent sets} by \autoref{algo:graph-coloring-independent-set} as guaranteed by \autoref{thm:lec18}. But before we see the final algorithm, we introduce a cute trick.

\begin{remark}[Wigderson's trick]\label{rmk:Wigderson-trick}
	We can always \(3\)-\hyperref[def:coloring]{color} \(\left\{ v \right\} \cup \mathcal{N} (v)\) for all \(v\in \mathcal{V} \) if \(\mathcal{G} \) is \(3\)-\hyperref[def:coloring]{colorable}.
\end{remark}
\begin{explanation}
	Since for a \(3\)-\hyperref[def:coloring]{colorable} graph, for all \(v\in \mathcal{V} \), \(\mathcal{G} [N(v)]\) is bipartite (all \(u\in N(v)\) already links with \(v\), so the degree will be at most \(2\) in \(\mathcal{G} [\mathcal{N} (v)]\)). And as mentioned before, we can always \(2\)-\hyperref[def:coloring]{colors} a bipartite graph. And we just use another new color for \(v\) to do the \(3\)-\hyperref[def:coloring]{coloring}.
\end{explanation}

Now, we see the final algorithm.

\begin{algorithm}[H]\label{algo:graph-coloring}
	\DontPrintSemicolon
	\caption{\hyperref[prb:graph-coloring]{Graph Coloring} -- \hyperref[def:independent-set]{Independent Set} Decomposition of \(3\)-\hyperref[def:coloring]{Colorable} Graph}
	\KwData{A \(3\)-\hyperref[def:coloring]{colorable} graph \(\mathcal{G} =(\mathcal{V} , \mathcal{E} )\)}
	\KwResult{A \hyperref[def:coloring]{colored} \(\mathcal{G} \)}
	\SetKwFunction{Ind}{\hyperref[algo:graph-coloring-independent-set]{Independent-Set}}
	\BlankLine
	\(n_0\gets \left\vert \mathcal{V}  \right\vert \)\;
	\;
	\Comment{Phase 1}
	\While(\label{algo:graph-coloring-phase-1}){\(\Delta (\mathcal{G} ) \geq n_0 ^{3 / 4}\)}{
	\(v\gets \argmax_{i\in \mathcal{V} } \deg(i)\)\;
	\(3\) colors \(\left\{ v \right\} \cup N(v)\)\Comment*[r]{\hyperref[rmk:Wigderson-trick]{Wigderson's trick}}
	\(\mathcal{G} \gets \mathcal{G} [\mathcal{V} \setminus (\left\{ v \right\} \cup N(v))]\)\;
	}
	\;
	\Comment{Phase 2}
	\While(\label{algo:graph-coloring-phase-2}\Comment*[f]{\(\Delta (\mathcal{G} ) < n_0^{3 / 4}\)}){\(\Delta (\mathcal{G} ) \geq 100\)}{
		\(S\gets\)\Ind{\(\mathcal{G}\)}\Comment*[r]{\(\left\vert S \right\vert \geq cn\Delta ^{-1 / 3} \ln ^{-1 / 2}\Delta \) from \autoref{thm:lec18}}
		\(1\) colors \(S\)\;
		\(\mathcal{G} \gets \mathcal{G} [\mathcal{V} \setminus S]\)\;
	}
	\;
	\Comment{Phase 3}
	\(\Delta (\mathcal{G} ) + 1\) colors \(\mathcal{G} \)\label{algo:graph-coloring-phase-3}\Comment*[r]{\(\Delta (\mathcal{G} ) < 100\)}
	\Return{\(\mathcal{G} \)}\;
\end{algorithm}

Notice that in \autoref{algo:graph-coloring}, whenever we do a \hyperref[def:coloring]{coloring}, we use a brand-new color to avoid any collision.

\begin{theorem}
	\autoref{algo:graph-coloring} is an \(\widetilde{O} (n^{1 / 4})\)-approximation algorithm for \hyperref[prb:graph-coloring]{graph coloring}.
\end{theorem}
\begin{proof}
	We see that
	\begin{itemize}
		\item \hyperref[algo:graph-coloring-phase-1]{Phase 1}: color at least \(n_0^{3 / 4}\) vertices with \(3\) colors in each iteration, hence need at most \(n_0^{1 / 4} (= 3\cdot n_0 / n_0^{3 / 4})\) colors.
		\item \hyperref[algo:graph-coloring-phase-2]{Phase 2}: from \autoref{thm:lec18}, \(\left\vert S \right\vert \geq n c n_0^{-1 / 4} \log ^{-1 / 2} n_0 \eqqcolon n \gamma \), then the induced graph will have vertices less than \(n(1 - \gamma )\).\footnote{Notice the different between \(n\) and \(n_0\): \(n\) is updating, while \(n_0\) is the original graph size.} Hence, we can run this at most \(k\) iteration since \(n \geq 1\), i.e.,
		      \[
			      1 \leq n \leq n_0 (1 - \gamma )^k \leq n_0 e^{-\gamma k} \implies k \leq \frac{1}{\gamma} \ln n_0 = \frac{1}{c} n_0^{1 / 4}\log ^{1 / 2}n_0\cdot \ln n_0
		      \]
		\item \hyperref[algo:graph-coloring-phase-3]{Phase 3}: Clean-up phase, only uses constant amount (\(< 100\)) more colors.
	\end{itemize}
	In all, \autoref{algo:graph-coloring} uses at most
	\[
		3n_0^{1 / 4} + \frac{1}{c} n_0^{1 / 4} \log ^{3 / 2}n_0 + 100 = \widetilde{O} (n^{1 / 4})
	\]
	colors.
\end{proof}

\begin{remark}
	We can use the similar approach for small constant \(c\), e.g., \(c=4\), \(c=5\), etc.
\end{remark}