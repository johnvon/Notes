\lecture{2}{31 Aug. 10:30}{Linear Programming with Set Covers}

\section{Linear Programming Rounding}

To get a \(d\)-approximation algorithm, instead of seeing the greedy algorithm, we first see the LP\footnote{See \href{https://www.pbb.wtf/posts/Notes\#linear-programming-math561ioe510to518-umich}{MATH561} for a complete reference.} dual method, which turns out to be exactly the same as the greedy algorithm.

\begin{prev}
	Both linear programming and convex programming can be solved in polynomial time.
\end{prev}

Notice that it's more natural to define \hyperref[prb:set-cover]{set cover} in terms of ILP (integer LP). Define our integer variables \(\left\{ x_i \right\} _{i\in [n]}\) such that
\[
	x_i = \begin{dcases}
		1, & \text{ if } S_i\in \mathcal{\MakeUppercase{S}} ^\prime  ; \\
		0, & \text{ otherwise}.
	\end{dcases}
\]

In this way, we have the following ILP formulation for \hyperref[prb:set-cover]{set cover} as

\[
	\begin{aligned}
		\min~            & \sum_{i} w_i \cdot x_i                           \\
		                 & \sum_{S_i\ni e}x_{i} \geq 1     & \forall i\in U \\
		(\text{IP})\quad & x_{i} \in \left\{ 0, 1 \right\} & \forall i.
	\end{aligned}
\]

But we know that this is a \(\NP\)-hard problem, so we relax it to be

\[
	\begin{aligned}
		\min~            & \sum_{i} w_i \cdot x_i                       \\
		                 & \sum_{S_i\ni e}x_{i} \geq 1 & \forall i\in U \\
		(\text{LP})\quad & x_{i} \geq 0                & \forall i.
	\end{aligned}
\]

Write it in a more compact form, we have
\[
	\begin{aligned}
		\min~ & \left\langle w, x \right\rangle \\
		      & Ax \geq \mathbbm{1}             \\
		      & x\geq 0
	\end{aligned}
\]
where \(A\in \mathbb{\MakeUppercase{r}} ^{n\times m}\) such that
\[
	A_{ij} = \begin{dcases}
		1, & \text{ if } e_i\in S_j ; \\
		0, & \text{ otherwise}.
	\end{dcases}
\]

\begin{note}
	Note when we do relaxation, we want \(x\in \mathop{\mathrm{fes}}(\text{IP}) \implies x\in \mathop{\mathrm{fea}}(\text{LP})\), i.e., \(\mathop{\mathrm{fes}}(\text{LP})\supseteq \mathop{\mathrm{fes}}(\text{IP})\). Note that in this case, for a minimization problem, we have
	\[
		f(x) = \OPT_\text{LP} \leq \OPT_\text{IP}.
	\]
\end{note}

In this case, we see that the most natural way to get an integer solution from the fractional solution obtained from the relaxed LP is to \textbf{round} \(x\) to integral solution. This leads to the following \hyperref[algo:set-cover-LP-rounding]{algorithm}.

\par
\begin{algorithm}[H]\label{algo:set-cover-LP-rounding}
	\DontPrintSemicolon
	\caption{\hyperref[prb:set-cover]{Set cover} -- LP Rounding}
	\KwData{A \hyperref[def:set-system]{set system} \((U, \mathcal{\MakeUppercase{S}})\)}
	\KwResult{A \hyperref[def:covering]{covering} \(\mathcal{\MakeUppercase{S}} ^\prime \)}
	\BlankLine
	\(x\gets\mathop{\mathrm{solve}}(\text{LP})\)\Comment*[r]{Solve the relaxed (LP)}
	\(\mathcal{\MakeUppercase{S}} ^\prime \gets \left\{ S_i \colon x_{i} \geq 1 / d \right\}\)\;
	\Return{\(\mathcal{\MakeUppercase{S}} ^\prime \)}\;
\end{algorithm}

We now prove the correctness and \autoref{algo:set-cover-LP-rounding}'s approximation ratio.
\begin{lemma}
	\(\mathcal{\MakeUppercase{S}} ^\prime \) is a covering.
\end{lemma}
\begin{proof}
	Fix \(e\in U\), let \(S_1, \ldots  , S_d\) be the sets containing \(e\). We see that
	\[
		\sum_{i=1}^{d} x_{i} \geq 1\implies \exists j\in[d] \text{ s.t. } x_j \geq \frac{1}{d} \implies S_j\in \mathcal{\MakeUppercase{S}} ^\prime.
	\]
\end{proof}

\begin{theorem}\label{thm:lec2-1}
	\autoref{algo:set-cover-LP-rounding} is \(d\)-approximation algorithm.
\end{theorem}
\begin{proof}
	By comparing \(w(\mathcal{\MakeUppercase{S}} ^\prime )\) and \(\OPT_\text{LP} = \sum_{i=1}^{m} x_{i} w_{i}\), we see that
	\[
		\OPT \leq \sum_{S_i\in \mathcal{\MakeUppercase{S}} ^\prime }w_i \leq d \sum_{S_i\in \mathcal{\MakeUppercase{S}} ^\prime }w_{i} x_{i} \leq d\cdot \OPT_\text{LP} \leq d\cdot \OPT,
	\]
	which implies \(\OPT / d\leq \OPT_\text{LP} \leq \OPT\).

	\begin{note}
		Note that \(\OPT\) is assumed to be \(\OPT_{\text{IP}}\), i.e., the optimum of the original IP formulation of \autoref{prb:set-cover}.
	\end{note}
\end{proof}

\begin{definition}[Intgrality gap]\label{def:integrality-gap}
	Given an integer programming, the \emph{integrality gap} between \(\OPT\) and \(\OPT_{\mathrm{LP}} \) of its LP relaxation is defined as
	\[
		\sup _{\text{input } I}\frac{\OPT(I)}{\OPT_{\mathrm{LP} }(I)}.
	\]
\end{definition}

\begin{remark}
	We see that the \hyperref[def:integrality-gap]{integrality gap} of \autoref{algo:set-cover-LP-rounding} is \(d\) from \autoref{thm:lec2-1}.
\end{remark}

\subsection{Randomized Linear Programming Rounding}

And indeed, we can use a more natural way to do the rounding, i.e., respect to the \(x_i\) value.

\begin{intuition}
	If \(x_i\) is close to \(1\), it's reasonable to include it, vice versa.
\end{intuition}

We see that algorithm first.
\par
\begin{algorithm}[H]\label{algo:set-cover-randomized-LP-rounding}
	\DontPrintSemicolon
	\caption{\hyperref[prb:set-cover]{Set cover} -- Randomized LP Rounding}
	\KwData{A \hyperref[def:set-system]{set system} \((U, \mathcal{\MakeUppercase{S}})\)}
	\KwResult{A (possible) \hyperref[def:covering]{covering} \(\mathcal{\MakeUppercase{S}} ^\prime \)}
	\BlankLine

	\(x\gets\mathop{\mathrm{solve}}(\text{LP})\)\Comment*[r]{Solve the relaxed (LP)}
	\(\mathcal{\MakeUppercase{S}} \gets \varnothing \)\;
	\For(){\(i= 1, \ldots  , m\)}{
		add \(S_i\) to \(\mathcal{\MakeUppercase{S}} ^\prime \) w.p. \(x_i\)\Comment*[r]{independently}
	}
	\Return{\(\mathcal{\MakeUppercase{S}} ^\prime \)}\;
\end{algorithm}

Now, the question is, how is this \(\mathcal{\MakeUppercase{S}} ^\prime \)'s quality? In other words, fix \(e\in U\), what's \(\Pr(\text{\(e\) is covered})\)?

\begin{lemma}
	\(\Pr(\text{\(e\) is covered})\geq 1 - 1/e\approx 0.63\).
\end{lemma}
\begin{proof}
	We bound \(\Pr(\overline{\text{\(e\) is covered}})\) instead. Say \(S_1, \ldots  , S_d\) are the sets containing \(e\), then we see that
	\[
		\Pr(\overline{\text{\(e\) is covered}}) = \prod\limits_{i=1}^{d} (1 - x_i)\leq \prod\limits_{i=1}^{d} e^{- x_i} = e^{- (\overbrace{x_1 + \ldots x_d}^{\geq 1})} \leq e^{-1}.
	\]
	\begin{note}
		For every \(x\), we have \(1 + x \leq e^x\), and this approximation is close when \(\left\vert x \right\vert \) is small.
	\end{note}
\end{proof}

A standard way to boost the correctness of a randomized algorithm is to run it multiple time, which leads to the following.

\par
\begin{algorithm}[H]\label{algo:set-cover-randomized-LP-rounding-multi-time}
	\DontPrintSemicolon
	\caption{\hyperref[prb:set-cover]{Set cover} -- Multi-time Randomized LP Rounding}
	\KwData{A \hyperref[def:set-system]{set system} \((U, \mathcal{\MakeUppercase{S}})\), \(\alpha \)}
	\KwResult{A (possible) \hyperref[def:covering]{covering} \(\mathcal{\MakeUppercase{S}} ^\prime \)}
	\BlankLine


	\(x\gets\mathop{\mathrm{solve}}(\text{LP})\)\Comment*[r]{Solve the relaxed (LP)}
	\(\mathcal{\MakeUppercase{S}} \gets \varnothing \)\;
	\For(\Comment*[f]{independently}){\(t = 1, \ldots  , \alpha \)}{
		\For(){\(i= 1, \ldots  , m\)}{
			add \(S_i\) to \(\mathcal{\MakeUppercase{S}} ^\prime \) w.p. \(x_i\)\Comment*[r]{independently}
		}
	}
	\Return{\(\mathcal{\MakeUppercase{S}} ^\prime \)}\;
\end{algorithm}

\begin{lemma}\label{lma:lec2-2}
	With \(\alpha = 2\ln n\), \(\mathcal{\MakeUppercase{S}} ^\prime\) returned from \autoref{algo:set-cover-randomized-LP-rounding-multi-time} is a \hyperref[def:covering]{covering} w.p. at least \(1 - \frac{1}{n}\).
\end{lemma}
\begin{proof}
	We have \(\Pr(\text{\(e\) is not covered}) \leq e ^{- \alpha }\) from independence of each run. Let \(\alpha = 2\ln n\), then \(\Pr(\text{\(e\) is not covered}) \leq e ^{- \alpha } = 1 / n^{2}\). By union bound,
	\[
		\Pr(\text{some elements is not covered} ) \leq \sum_{e\in U}\Pr(\text{\(e\) not covered} )\leq n\cdot \frac{1}{n^{2} } = \frac{1}{n}.
	\]
	This implies w.p. at least \(1 - 1/n\), \(\mathcal{\MakeUppercase{S}} ^\prime \) is a \hyperref[def:covering]{covering}.
\end{proof}

In other words, with \(\alpha = 2 \ln n\), \autoref{algo:set-cover-randomized-LP-rounding-multi-time} is correct with probability at least \(1 - 1 / n\).

\begin{lemma}\label{lma:lec2-3}
	With \(\alpha = 2 \ln n\), \(\mathcal{\MakeUppercase{S}} ^\prime \) returned from \autoref{algo:set-cover-randomized-LP-rounding-multi-time} has an approximation ratio \(4 \ln n\) w.p. at least \(\frac{1}{2}\).\footnote{Note that \(\mathcal{\MakeUppercase{S}} ^\prime \) is not necessary a \hyperref[def:covering]{covering}.}
\end{lemma}
\begin{proof}
	Since \(\mathbb{\MakeUppercase{e}} [w(\mathcal{\MakeUppercase{S}} ^\prime)] \leq \alpha \sum_{i}w_{i} x_{i} = \alpha \OPT_{\text{LP} }\), we have \(\Pr(w(\mathcal{\MakeUppercase{S}} ^\prime )\geq 2\cdot \alpha \OPT_{\text{LP} })\leq 1 / 2\) from Markov inequality. We see that w.p. \(\geq 1 / 2\), \(w(\mathcal{\MakeUppercase{S}} ^\prime )\leq 2\cdot 2\ln n\cdot \OPT_{\text{LP} } \leq 4 \ln n\OPT\).
\end{proof}

\begin{theorem}\label{thm:lec2-2}
	By running \autoref{algo:set-cover-randomized-LP-rounding-multi-time} many times, we get a \((4\ln n)\)-approximation algorithm with high probability.\footnote{Note that we still need to choose \(\mathcal{\MakeUppercase{S}}^\prime \).}
\end{theorem}
\begin{proof}
	Together with \autoref{lma:lec2-2} and \autoref{lma:lec2-3} and using the union bound, the probability of \(\mathcal{\MakeUppercase{S}} ^\prime \) not being a \hyperref[def:covering]{covering} or with weight higher than \(4\ln n\OPT\) is at most \(\frac{1}{n} + \frac{1}{2}\), which is less than \(1\). Hence, by running \autoref{algo:set-cover-randomized-LP-rounding-multi-time} many times (independently), the failing possibility is exponential small.
\end{proof}

\begin{note}
	With \autoref{thm:lec2-2}, we still need to find a valid \hyperref[def:covering]{covering} with the lowest cost, where a valid \hyperref[def:covering]{covering} with low enough weight is guaranteed to exist with high probability. Note that this is still a polynomial time algorithm since we know that checking \(\mathcal{\MakeUppercase{S}} ^\prime \) is a \hyperref[def:covering]{covering} is just linear.
\end{note}

\begin{remark}
	Indeed, with some smarter algorithm modified from \autoref{algo:set-cover-randomized-LP-rounding-multi-time}, we can get an \(H_k\) approximation ratio.
\end{remark}