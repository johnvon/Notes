\chapter{Review}
\section{Midterm Review}\label{sec:mid-review}
\subsection{Normed Spaces}
Recall the \hyperref[def:normed-vector-space]{normed spaces}, and the properties of which. In particular, focus on \hyperref[def:convex-function]{convexity} and note that \(x\mapsto \left\lVert x\right\rVert \) is a \hyperref[def:convex-function]{convex function}.

\begin{eg}[Normed spaces]
	The spaces \(\ell _p\) for \(1 \leq p \leq \infty \) of sequences and \(L^p(\Omega , \mathcal{F} , \mu )\) of integrable functions. Also, the space of continuous functions on compact \hyperref[def:Hausdorff]{Hausdorff space} with supremum norm \(C(K)\). Notice that
	\[
		C(K) \subseteq L^{\infty} (K, \mathcal{F}).
	\]
\end{eg}

\begin{remark}[Legendre transform]
	Recall the Legendre transform of \hyperref[def:convex-function]{convex functions}. The most general form is that let \(X\) be a \hyperref[def:Banach-space]{Banach space} and \(X^{\ast} \) its \hyperref[def:dual-space]{dual space} with a \hyperref[def:convex-function]{convex function} \(f\colon X\to \mathbb{R} \) and \(f^{\ast} \colon X^{\ast} \to \mathbb{R} \). We have
	\[
		f^{\ast} (y^{\ast} ) = \sup _{x\in X}\left[ y^{\ast} (x) - f^{\ast} (x) \right].
	\]
	Notice that \(f^{\ast} \) is \hyperref[def:convex-function]{convex} and lower semi-continuous where \(f^{\ast} \colon X^{\ast} \to \mathbb{R} \cup \left\{ +\infty  \right\} \).
\end{remark}

\subsection{Quotient Spaces}
Let \(X\) be a \hyperref[def:normed-vector-space]{normed space} and \(E\) be a subspace of \(X\). Then \(\quotient{X}{E} = \left\{ [x] = x+E\colon x\in X \right\}  \) if \(E\) is closed, then \(\quotient{X}{E} \) is also a \hyperref[def:normed-vector-space]{normed space} with the \hyperref[def:norm]{norm} \(\left\lVert [x]\right\rVert \coloneqq \inf _{y\in E}\left\lVert x- y\right\rVert\).

\begin{remark}
	\(E\) need to be closed since we need \(\left\lVert [x]\right\rVert = 0 \implies [x] = 0\).
\end{remark}

\subsection{Banach Spaces}
Any \hyperref[def:normed-vector-space]{normed space} \(E\) can be completed to a \hyperref[def:Banach-space]{Banach space} \(\hat{E} \) by \autoref{thm:completion-of-Banach-space}.

\begin{eg}
	\(\ell _p\) and \(L^p\) are \hyperref[def:Banach-space]{Banach spaces}. For \(x\in \ell _p\), \(x= \left\{ x_n, n\geq 1 \right\} \) with
	\[
		\left\lVert x\right\rVert _p = \left( \sum_{n=1} ^{\infty} \left\vert x_n \right\vert ^p \right) ^{1 / p}.
	\]
\end{eg}

Notice that \hyperref[lma:Minkowski-ineq]{Minkowski inequality} is the triangle inequality for \(\ell _p\) and \(L^p\), and we can prove this using \hyperref[lma:Holder-ineq]{Hölder's inequality} where we have
\[
	\left\lVert fg\right\rVert _1 \leq \left\lVert f\right\rVert _p \left\lVert g\right\rVert _q
\]
for \(1 / p + 1 / q = 1\).

\begin{remark}[Proof of completeness of the \(\ell _p\) spacees]
	This is easy for \(\ell _p\), but for \(L^p\), we need to use \href{https://en.wikipedia.org/wiki/Dominated_convergence_theorem}{dominated convergence theorem}.
\end{remark}

\subsection{Inner Product Spaces and Hilbert Spaces}
Notice that the \hyperref[def:Hilbert-space]{Hilbert spaces} are the completion of \hyperref[def:inner-product-space]{inner product spaces}. Recall the \hyperref[lma:parallelogram-law]{parallelogram law}
\[
	\left\lVert x + y\right\rVert ^{2} + \left\lVert x - y\right\rVert ^{2} = 2\left\lVert x\right\rVert ^{2} + 2 \left\lVert y\right\rVert ^{2}
\]
and the \hyperref[thm:Cauchy-Schwarz-ineq]{Cauchy-Schwarz inequality}
\[
	\left\vert \left\langle x, y \right\rangle  \right\vert \leq \left\lVert x\right\rVert \left\lVert y\right\rVert.
\]

\subsubsection{Orthogonality}
Recall the \hyperref[def:orthogonal-projection]{orthogonal projection} \(P_E\) onto a closed subspace \(E \subseteq \mathcal{H} \) is \(P_E x = x(y)\) where \(x(y)\) is the minimizer of \(\min _{y\in E} \left\lVert x - y\right\rVert \).

\begin{remark}
	\(P_E\) is the projection, i.e., \(P_E ^{2} g P_E\), and \(I-P_E\) is proaction onto the \hyperref[def:orthogonal-complement]{orthogonal complement} \(E^\perp\) of \(E\) in \(\mathcal{H} \) such that \(\mathcal{H} = E \oplus E^{\perp} \). We see that
	\[
		\left\lVert x\right\rVert ^{2} = \left\lVert P_E x\right\rVert ^{2} + \left\lVert (I - P_E)x\right\rVert ^{2}
	\]
	for \(x\in \mathcal{H} \).
\end{remark}

Consider the \hyperref[def:orthogonal-system]{orthogonal} or \hyperref[def:orthonormal-system]{orthonormal} sets of vectors \(x_k\), \(k = 1, 2, \dots  \) in \(\mathcal{H} \) with the corresponding \hyperref[def:Fourier-series]{Fourier series} being
\[
	S_n(x) \coloneqq \sum_{k=1} ^n \left\langle x, x_k \right\rangle x_k
\]
such that
\[
	\left\lVert S_n(x)\right\rVert ^{2} = \sum_{k=1}^n \left\vert \left\langle x, x_k \right\rangle  \right\vert ^{2}.
\]

If the set \(\left\{ x_k \right\} _{k=1}^{\infty} \) is \hyperref[def:orthonormal-system]{orthonormal}, then \(S_n = P_{E_n}\) where \(E_n\) is the span of \(\left\{ x_1, \dots , x_n  \right\} \), and
\[
	\left\lVert S_n x\right\rVert ^{2} = \left\lVert P_{E_n}x\right\rVert ^{2} \leq \left\lVert x\right\rVert ^{2},
\]
which is the \hyperref[thm:Bessel-ineq]{Bessel's inequality}.

\begin{remark}
	\(S_n x \to S_\infty x\) in \(\mathcal{H} \) where \(S_\infty = P_{E_\infty }\) and \(E_\infty \) is the closure of spaces \(E_n\), \(n \geq 1\).
\end{remark}

The \hyperref[def:orthonormal-system]{orthonormal system} \(\{x_k\}_{k \geq 1}\) is complete if \(E_\infty = \mathcal{H}\). In that case, \(\left\lVert x\right\rVert ^{2} = \left\lVert P_{E_\infty }x\right\rVert ^{2} = \sum_{k=1}^{\infty} \left\vert \left\langle x, x_k \right\rangle  \right\vert ^{2} \).

\begin{remark}
	Proving completeness can be difficult.
\end{remark}

\begin{eg}[Haar basis]
	The Haar basis for \(L^2([0, 1])\) is the Fourier basis \(e^{2\pi ni x}\), \(n \in \mathbb{Z } \) for \(L^2([0, 1])\).
\end{eg}
\begin{explanation}
	Let \(x_k\), \(k \geq 1\) be any arbitrary sequence of vectors in \(\mathcal{H} \). We can then construct an \hyperref[def:orthonormal-system]{orthonormal} sequence \(y_k\), \(k \geq 1\) by applying Gram-Schmidt procedure.
\end{explanation}

\subsection{Bounded Linear Functionals}
Consider \hyperref[def:bounded-linear-functional]{bounded linear functionals} on a \hyperref[def:Banach-space]{Banach space} \(E\), \(f\in E^{\ast} \), \(\left\lVert f\right\rVert = \sup _{\left\lVert x\right\rVert = 1}\left\vert f(x) \right\vert \) and \(E^{\ast} \) is a \hyperref[def:Banach-space]{Banach space}. Recall that \(f(\cdot)\) is essentially defined by \(H = \ker(f)\) where \(H\) is a closed subspace of \(E\) with \(\codim(H) = 1\), i.e., \(\dim \quotient{E}{H} = 1 \) and we have
\[
	\widetilde{f} \colon \quotient{E}{H} \to \mathbb{R}
\]
is defined via \(\widetilde{f} ([x]) = f(x)\) for \(x\in E\), and \(\widetilde{f} (a[x]) = ca\)  for some constant \(c\).

\subsection{Representation Theorem}
The important representation theorem for \hyperref[def:bounded-linear-functional]{bounded linear functionals} is the \hyperref[thm:Riesz-representation]{Riesz representation theorem}. The easiest case is \(E = \mathcal{H} \) being a \hyperref[def:Hilbert-space]{Hilbert space} and \(E^{\ast} \equiv \mathcal{H} \). This implies \hyperref[thm:Radon-Nikodym]{Radon-Nikodym theorem}, where if we have \(\nu \ll \mu \), then
\[
	\nu (E) = \int _E f\,\mathrm{d} \mu ,\quad f = \frac{\mathrm{d}\nu }{\mathrm{d}\mu } \in L^1(\mu )
\]
for \(\nu , \mu \) being finite measures. Furthermore, the \hyperref[thm:Radon-Nikodym]{Radon-Nikodym theorem} implies the \hyperref[thm:Riesz-representation]{Riesz representation theorem} for \(\ell _p\) and \(L^p\) with \(1 \leq p < \infty \).

\begin{remark}
	We have \(E^{\ast} = \ell _q\) or \(L^q\) with \(1 / p + 1 / q = 1\) for \(1 \leq p < \infty \), and remarkably, \(\ell _1^{\ast} = \ell _\infty \) but \(\ell _\infty ^{\ast} \neq \ell _1\).
\end{remark}

\begin{remark}
	The \hyperref[thm:Riesz-representation]{Riesz representation theorem} for \(C(K)\) is space of bounded Borel measures where for \(g\in C(K)^{\ast} \),
	\[
		g(f) = \int _K f\,\mathrm{d} \mu
	\]
	for \(f\in C(K)\).
\end{remark}

\subsection{Hahn-Banach Theorem}
Let \(E\) be a \hyperref[def:Banach-space]{Banach space} and \(E_0\) be a subspace such that \(f_0 \colon E_0 \to \mathbb{R} \) a \hyperref[def:bounded-linear-functional]{bounded linear functional} on \(E_0\) such that \(\left\lVert f_0\right\rVert < \infty \). Then there exists an extension \(f\) of \(f_0\) to \(E\) with \(\left\lVert f\right\rVert = \left\lVert f_0\right\rVert \).

\begin{remark}
	\(f\) is not necessary unique. Nevertheless, it is unique for \hyperref[def:Hilbert-space]{Hilbert spaces}, or \(\ell _p\), \(L^p\) with \(1 < p < \infty \).
\end{remark}

\subsubsection{Reflexivity}
Consider the embedding \(E\to E^{\ast\ast} \) such that \(x\mapsto x^{\ast\ast}\), then \(E\) is \hyperref[def:reflexive-space]{reflexive} if the embedding is surjective. Also, \(E\) is \hyperref[def:reflexive-space]{reflexive} implies that
\[
	\left\lVert f\right\rVert = \sup _{\left\lVert x\right\rVert = 1}\left\vert f(x) \right\vert = f(x_f)
\]
for some \(x_f\in E\) with \(\left\lVert x_f\right\rVert = 1\) for every \(f\in E^{\ast} \).

\begin{remark}
	This is one way of showing some spaces is not \hyperref[def:reflexive-space]{reflexive}.
\end{remark}

\subsubsection{Separation Theorem}
Recall the \hyperref[thm:separation-of-a-point-from-a-convex-set]{separation theorem} for \hyperref[def:convex-set]{convex sets} from a point. Given a \hyperref[def:convex-set]{convex set} \(K\) and a point \(x_0 \notin K\), there is a \hyperref[def:hyperplane]{hyperplane} such that \(f(x_0) > f(k)\) for all \(k\in K\). The \hyperref[def:Minkowski-functional]{Minkowski functional} for \hyperref[def:convex-set]{convex set} essentially makes \hyperref[def:convex-set]{convex sets} unit \hyperref[def:ball]{ball} for some semi-norm.