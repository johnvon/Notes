\lecture{7}{20 Sep.\ 14:30}{Hahn-Banach Theorem}

\begin{remark}
	When \(p = 1\), the supremum is not attained necessarily. Consider \(g\in L^{\infty} \), \(F(f) \coloneqq \int fg\,\mathrm{d} \mu \) is \hyperref[def:dual-space]{dual} of \(L^1\). If \(g(\cdot)\) is continuous on \(\mathbb{R} \) with unique maximum, then the supremum \(\sup _{\left\lVert f\right\rVert _1}\left\vert F(f) \right\vert\) is not attained. In all, for \(1 \leq p\leq \infty \), \(L^q\) contained in the \hyperref[def:dual-space]{dual} of \(L^p\). If \(1 < p \leq \infty \), then \(\sup _{\left\lVert f\right\rVert _p = 1} \left\vert F(f) \right\vert \) is attained. For \(p = 1\), the supremum is not necessarily attained.
\end{remark}

Now, we're ready to prove \autoref{thm:lec6}.

\begin{proof}[Proof of \autoref{thm:lec6}]
	To show that the \hyperref[def:dual-space]{dual} of \(L^p\) is \(L^q\) if \(1 \leq p < \infty \) where \(1 / p + 1 / q = 1\), we use \hyperref[thm:Radon-Nikodym]{Radon-Nikodym theorem}. Suppose \(E = L^p(\Omega , \Sigma , \mu )\) with \(1 \leq p < \infty \) and \(f\in E^{\ast} \). Just consider finite measure space, i.e., \(\mu (\Omega ) < \infty \). We define a measure \(\nu \) on \(\Sigma \) by \(\nu (A) \coloneqq F(\chi _A)\) for \(A\in \Sigma \), where \(\chi _A\) is the characteristic function of \(A\). We see that
	\[
		\mu (A) = 0 \implies \nu (A) = 0 \implies \nu \ll \mu,
	\]
	and \hyperref[thm:Radon-Nikodym]{Radon-Nikodym theorem} implies
	\[
		\nu (A) = \int _A g\,\mathrm{d} \mu
	\]
	for some \(g \eqqcolon \frac{\mathrm{d}\nu }{\mathrm{d}\mu } \in L^1(\Omega , \Sigma , \mu )\). Note that \(g\) may not be in \(L^q\) since \(q > 1\). Hence, \(F(f)= \int _\Omega fg \,\mathrm{d} \mu \) for all simple function \(f\) assuming \(g \geq 0\). Set \(f = g^{q - 1}\) with the fact that \(\left\vert F(f) \right\vert  \leq \left\lVert F\right\rVert _p \left\lVert f\right\rVert _p\). Recall that \(q - 1 = q / p\), hence
	\[
		\int g^q\,\mathrm{d} \mu \leq \left\lVert F\right\rVert _p \left( \int g^q \,\mathrm{d} \mu  \right) ^{1 / p} \implies \left\lVert g\right\rVert _q ^q \leq \left\lVert F\right\rVert _p \left\lVert g\right\rVert _q ^{q / p} = \left\lVert F\right\rVert _p \left\lVert g\right\rVert _q ^{q - 1},
	\]
	hence \(\left\lVert g\right\rVert _q \leq \left\lVert F\right\rVert _p\).

	\begin{note}
		We assume \(g \geq 0\) is because \(\nu \) is a sign measure, then if we have a bounded variation function, we can just break it into \(\nu ^+ + \nu ^-\).
	\end{note}
\end{proof}

\begin{remark}
	\(L^1\) is a subset of \({L^{\infty}}^{\ast}\) but not equal to it. If \(F\colon L^{\infty} (\mu )\to \mathbb{C} \) is a \hyperref[def:bounded-linear-functional]{bounded linear functional}, then if \(\Omega = K\) is a compact \hyperref[def:Hausdorff]{Hausdorff space}  \(F\) induces a \hyperref[def:bounded-linear-functional]{bounded linear functional} on \(C(K)\), i.e., the space of continuous functions on \(K\). We see that \(C(K)\subseteq L^{\infty} (K, \Sigma , \mu )\) where \(\Sigma \) is the Borel algebra on \(K\).
\end{remark}

\subsection{Dual of \(C(K)\)}
Finally, we state the following important characterization of \hyperref[def:bounded-linear-functional]{bounded linear functionals} on \(C(K)\).

\begin{theorem}[Riesz representation for \(C(K)\)]\label{thm:Riesz-representation-for-C-K}
	Let \(E = C(K)\) be the space of continuous functions on \hyperref[def:compact]{compact} \hyperref[def:Hausdorff]{Hausdorff space} \(K\). Then we have the following.
	\begin{enumerate}[(a)]
		\item For every Borel regular signed measure on \(K\), the \hyperref[def:linear-functional]{functional} \(F(f) = \int _K f\,\mathrm{d} \mu \) is a \hyperref[def:bounded-linear-functional]{bounded linear functional} on \(K\).
		\item Every \hyperref[def:bounded-linear-functional]{bounded linear functional} on \(C(K)\) can be expressed as \(F(f) = \int _K f\,\mathrm{d} \mu \) for some measure \(\mu \), and \(\left\lVert F\right\rVert = \left\vert \mu  \right\vert (K) \), i.e., \(TV(K)\).
	\end{enumerate}
\end{theorem}
\begin{proof}
	In this case, the proof is much more difficult, and we put the proof in \autoref{pf:Riesz-representation-for-C-K}.
\end{proof}

\section{Hahn-Banach Theorem}
\hyperref[thm:Hahn-Banach]{Hahn-Banach theorem} allows one to extend continuous \hyperref[def:linear-functional]{linear functionals} \(f\) from a subspace to the whole \hyperref[def:normed-vector-space]{normed space}, while preserving the continuity of \(f\). \hyperref[thm:Hahn-Banach]{Hahn-Banach theorem} is a major tool in functional analysis. Together with its variants and consequences, this result has applications in various areas of mathematics, computer science, economics and engineering.

\begin{theorem}[Hahn-Banach theorem]\label{thm:Hahn-Banach}
	Let \(E_0\) be a subspace of a \hyperref[def:Banach-space]{Banach space} \(E\). Then every \(f_0 \colon E_0 \to \mathbb{R} \) or \(\mathbb{C} \) has a continuous extension \(f\colon E\to \mathbb{R} \) or \(\mathbb{C} \) such that \(\left\lVert f\right\rVert = \left\lVert f_0\right\rVert.\)
\end{theorem}
\begin{proof}\let\qed\relax
	We assume \(E\) is \hyperref[def:separable]{separable}, otherwise we need \href{https://en.wikipedia.org/wiki/Transfinite_induction}{transfinite induction}. Let \(\left\{ x_{n} \right\}_{n \geq 1} \) have the property that its span is dense in \(E\).

	\begin{intuition}
		\hyperref[def:separable]{Separability} allows us to extend \(f_0\) one dimension at a time.  Now, if we can extend \(f_0\) such that
		\[
			E_0
			\to E_0 + \operatorname{span}(\left\{ x_1 \right\})
			\to E_0 + \operatorname{span}(\left\{ x_1, x_2 \right\})
			\to \dots
			\to E_0 + \operatorname{span}(\left\{ x_n \right\}_{n \geq 1} ),
		\]
		then \(\left\lVert f\right\rVert = \left\lVert f_0\right\rVert \), with the final space is dense in \(E\), we can extend \(f\) to \(E\) by continuity.
	\end{intuition}
\end{proof}