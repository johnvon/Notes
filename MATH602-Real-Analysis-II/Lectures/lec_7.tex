\lecture{7}{20 Sep. 14:30}{Riesz Representation Theorem II and Hahn-Banach Theorem}
\begin{remark}
	When \(p = 1\), the supremum is not attained necessarily. Consider \(g\in L^{\infty} \), \(F(f) \coloneqq \int fg\,\mathrm{d} \mu \) is dual of \(L^1\). If \(g(\cdot)\) is continuous on \(\mathbb{\MakeUppercase{r}} \) with unique maximum, then the supremum \(\sup _{\left\lVert f\right\rVert _1 \left\vert F(f) \right\vert }\) is not attained.
\end{remark}

\begin{note}
	In all, for \(1 \leq p\leq \infty \), \(L^q\) contained in the dual of \(L^p\). If \(1 < p \leq \infty \), then \(\sup _{\left\lVert f\right\rVert _p = 1} \left\vert F(f) \right\vert \) is attained. For \(p = 1\), the supremum is not necessarily attained.
\end{note}

To show that the dual of \(L^p\) is \(L^q\) if \(1 \leq p < \infty \) where \(1 / p + 1 / q g 1\), we use \autoref{thm:Radon-Nikodym}. Suppose \(E = L^p(\Omega , \Sigma , \mu )\) with \(1 \leq p < \infty \) and \(f\in E^{\ast} \). Just consider finite measure space, i.e., \(\mu (\Omega ) < \infty \). We define a measure \(\nu \) on \(\Sigma \) by \(\nu (A) \coloneqq F(\chi _A)\) for \(A\in \Sigma \), where \(\chi _A\) is the characteristic function of \(A\). We see that
\[
	\mu (A) = 0 \implies \nu (A) = 0 \implies \nu \ll \mu,
\]
and \autoref{thm:Radon-Nikodym} implies
\[
	\nu (A) = \int _A g\,\mathrm{d} \mu
\]
for some \(g \eqqcolon \frac{\mathrm{d}\nu }{\mathrm{d}\mu } \in L^1(\Omega , \Sigma , \mu )\). Note that \(g\) may not be in \(L^q\) since \(q > 1\). Hence, \(F(f)= \int _\Omega fg \,\mathrm{d} \mu \) for all simple function \(f\) assuming \(g \geq 0\). Set \(f = g^{q - 1}\) with the fact that \(\left\lVert F(f)\right\rVert \leq \left\lVert F\right\rVert _p \left\lVert f\right\rVert _p\). Recall that \(q - 1 = q / p\), hence
\[
	\int g^q\,\mathrm{d} \mu \leq \left\lVert F\right\rVert _p \left( \int g^q \,\mathrm{d} \mu  \right) ^{1 / p} \implies \left\lVert g\right\rVert _q ^q \leq \left\lVert g\right\rVert _q^q \leq \left\lVert F\right\rVert _p \left\lVert g\right\rVert _q ^{q / p} = \left\lVert F\right\rVert _p \left\lVert g\right\rVert _q ^{q - 1},
\]
hence \(\left\lVert g\right\rVert _q \leq \left\lVert F\right\rVert _p\).

\begin{note}
	We assume \(g \geq 0\) is because \(\nu \) is a sign measure, then if we have a bounded variation function, we can just break it into \(\nu ^+ + \nu ^-\).
\end{note}

\begin{remark}
	\(L^1\) is a subset of \((L^{\infty})^{\ast}\) but not equal to it. If \(F\colon L^{\infty} (\mu )\to \mathbb{\MakeUppercase{c}} \) is bounded linear functional, then if \(\Omega = K\) is a compact Hausdorff space, \(F\) induces a bounded linear functional on \(C(K)\), i.e., the space of continuous functions on \(K\). We see that \(C(K)\subseteq L^{\infty} (K, \Sigma , \mu )\) where \(\Sigma \) is the Borel algebra on \(K\).
\end{remark}

\begin{theorem}[Riesz representation theorem II]
	Let \(E = C(K)\) be the space of continuous functions on compact Hausdorff space \(K\). Then we have the following.
	\begin{enumerate}[(a)]
		\item For every Borel regular signed measure on \(K\), the functional \(F(f) = \int _K f\,\mathrm{d} \mu \) is a bounded linear functional on \(K\).
		\item Every bounded linear functional on \(C(K)\) can be expressed as \(F(f) = \int _K f\,\mathrm{d} \mu \)  for some measure \(\mu \), and \(\left\lVert F\right\rVert = \left\vert \mu  \right\vert (K) \), i.e., \(TV(K)\).
	\end{enumerate}
\end{theorem}


\begin{theorem}[Hahn-Banach theorem]\label{thm:Hahn-Banach}
	Let \(E_0\) be a subspace of a Banach space \(E\). Then every bounded linear functional \(f_0 \colon E_0 \to \mathbb{\MakeUppercase{r}} \) or \(\mathbb{\MakeUppercase{c}} \) has a continuous extension \(f\colon E\to \mathbb{\MakeUppercase{r}} \) or \(\mathbb{\MakeUppercase{c}} \) such that \(\left\lVert f\right\rVert = \left\lVert f_0\right\rVert.\)
\end{theorem}

Before proving this, let's first see some implications.

\begin{proposition}[Supporting hyperplane theorem]\label{thm:supporting-hyperplane-theorem}
	Let \(E\) be a Banach space. For every \(x\in E\), there exists \(f\in E^{\ast} \)  such that \(\left\lVert f\right\rVert = 1\), \(f(x) = \left\lVert x\right\rVert \). i.e., \(\sup _{\left\lVert y\right\rVert = 1} \left\vert f(y) \right\vert \) attained at \(y = x\).
\end{proposition}
\begin{proof}
	Consider dimension \(1\) space \(E_0 = \mathop{\mathrm{span}}(x) = \left\{ tx,t\in \mathbb{\MakeUppercase{r}} \text{ or }\mathbb{\MakeUppercase{c}}   \right\} \). Define \(f_0\colon E_0 \to \mathbb{\MakeUppercase{r}} \) or \(\mathbb{\MakeUppercase{c}} \) such that \(f_0(tx) = t \left\lVert x\right\rVert \). We see that \(\left\lVert f_0\right\rVert = 1\), and \autoref{thm:Hahn-Banach} implies there exists \(f\in E^{\ast} \), \(f(x) = \left\lVert x\right\rVert \) such that \(\left\lVert f\right\rVert =1\).
\end{proof}

\begin{remark}[Geometric interpretation]
	Let \(B\) be a unit ball \(\left\{ x\in E \colon \left\lVert x\right\rVert \leq 1\right\} \) in a real Banach space \(E\). Choose \(x_0 \in \partial B\) such that \(\left\lVert x_0\right\rVert = 1\). Then there exists \(f\in E^{\ast} \), \(\left\lVert f\right\rVert = 1\), \(f(x) = \left\lVert x\right\rVert \). Let \(H = \ker(f) + x_0\) where \(H\) intersects \(B\) at \(x_0\), we see that \(H\) divides \(E\) into \(2\) disjoint subsets, while \(B\) lies in one of which.
\end{remark}
\begin{explanation}
	Since \(x\in B\) and \(\left\lVert x\right\rVert < 1\) implies \(\left\vert f(x) \right\vert \leq \left\lVert x\right\rVert < 1\), we have \(f(x) < 1\), i.e., \(B\subseteq \left\{ x\colon f(x) < 1 \right\} \) and
	\[
		E = \left\{ x\colon f(x) < 1 \right\} \cup H \cup \left\{ x\colon f(x) > 1 \right\}.
	\]
\end{explanation}

\begin{note}
	Notice that we don't have uniqueness (since we don't have it in \autoref{thm:Hahn-Banach}) since a unit ball in \(L^{\infty } \) has corner, which will give multiple hyperplanes...
\end{note}

Extend this to prove existence of supporting hyperplanes for more general convex sets.