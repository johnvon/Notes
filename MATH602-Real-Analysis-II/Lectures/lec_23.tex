\lecture{23}{17 Nov.\ 14:30}{Spectral Theorems, Continuous Functional Calculus}
\begin{lemma}[Invertibility criterion]\label{lma:invertibility-criterion}
	Let \(T\) be a \hyperref[rmk:bounded-op]{bounded} \hyperref[def:self-adjoint-op]{self-adjoint operator} on \(\mathcal{H} \), then \(T\) is invertible if and only if \(T\) is bounded below, i.e., for all \(x\in \mathcal{H} \), there exists \(c>0\) such that \(\lVert Tx \rVert \geq c\lVert x \rVert \).
\end{lemma}
\begin{proof}
	If \(T\) is invertible, then \(T\) is bounded below with \(c = \lVert T^{-1}  \rVert ^{-1} \).

	Conversely, if \(T\) is bounded below, then \(T\) is injective and \(\im T\) is closed by the \hyperref[prop:isomorphic-embedding]{isomorphic embedding criterion}. Since \(\sigma _r(T)\) is empty, so \(0 \neq \sigma _r(T)\), hence the injectivity of \(T\) implies that \(\im T\) is dense in \(\mathcal{H} \), so \(\im T=\mathcal{H} \), i.e., \(T\) is surjective and injective. With \hyperref[thm:inverse-mapping]{inverse mapping theorem}, \(T\) is invertible with \(\lVert T^{-1}  \rVert \leq c^{-1} \).
\end{proof}

We see that by applying the \hyperref[lma:invertibility-criterion]{invertibility criterion} for the operator \(T - \lambda I\), we immediately obtain the following.

\begin{corollary}[Criterion of spectrum points]\label{col:criterion-of-spectrum-points}
	Let \(T\in \mathcal{L} (\mathcal{H} )\) be a \hyperref[def:self-adjoint-op]{self-adjoint operator}, then \(\lambda \in \sigma (T)\) if and only if the operator \(T - \lambda I\) is not \hyperref[rmk:bounded-op]{bounded} below.
\end{corollary}

Finally, we introduce the following.

\begin{definition*}
	Let \(T\in \mathcal{L} (\mathcal{H} )\) on a \hyperref[def:Hilbert-space]{Hilbert space} \(\mathcal{H} \).
	\begin{definition}[Approximate eigenvalue]\label{def:approximate-eigenvalue}
		A number \(\lambda \in \sigma (T)\) for which \(T - \lambda I\) is not bounded below is called an \emph{approximate eigenvalue} of \(T\).
	\end{definition}
	\begin{definition}[Approximate point spectrum]\label{def:approximate-point-spectrum}
		The set of all \hyperref[def:approximate-eigenvalue]{approximate eigenvalues} of \(T\) is called the \emph{approximate point spectrum} of \(T\).
	\end{definition}
\end{definition*}

\hyperref[col:criterion-of-spectrum-points]{Criterion of spectrum points} states that for \hyperref[def:self-adjoint-op]{self-adjoint operators}, the whole \hyperref[def:spectrum-point]{spectrum} is the \hyperref[def:approximate-point-spectrum]{approximate point spectrum}.

The reason for the name \emph{approximate} is the following. If \(\lambda\) is an eigenvalue, then \((T-\lambda I)x = 0\) for some \(x\) with \(\lVert x \rVert = 1\). If \(\lambda \) is an \hyperref[def:approximate-eigenvalue]{approximate eigenvalue}, then \((T-\lambda I)x\) can be made \emph{arbitrarily close} to \(0\) for some \(x\) with \(\lVert x \rVert = 1\). So, eigenvalues of \(T\) form the \hyperref[def:point-spectrum]{point spectrum} \(\sigma _p(T)\) while the \hyperref[def:approximate-eigenvalue]{approximate eigenvalues} of \(T\) form the \hyperref[def:continuous-spectrum]{continuous spectrum} \(\sigma _c(T)\).

\begin{remark}
	\(\lambda \in \sigma (T)\) is an \hyperref[def:approximate-point-spectrum]{approximate point spectrum} if and only if there exists a sequence \(\left\{ x_n \right\} _{n\geq 1}\) in \(\mathcal{H} \) and \(\lVert x_n \rVert = 1\), and
	\[
		\lim_{n \to \infty} \lVert Tx_n - \lambda x_n \rVert = 0.
	\]
\end{remark}

\subsection{The Spectrum Interval}
We now compute the \emph{tightest} interval that contains the \hyperref[def:spectrum-point]{spectrum} of a \hyperref[def:self-adjoint-op]{self-adjoint operator} \(T\). This interval can be computed from the quadratic form of \(T\) as follows.

\begin{theorem}[Spectrum interval]\label{thm:spectrum-interval}
	Suppose \(T \in \mathcal{L} (\mathcal{H} )\) for \(T\) being a \hyperref[def:self-adjoint-op]{self-adjoint operator}. Then
	\begin{enumerate}[(a)]
		\item \(\sigma (T) \subseteq [m, M]\) for \(m\coloneqq \inf _{\lVert x \rVert = 1}\left\langle Tx, x \right\rangle \), \(M\coloneqq \sup _{\lVert x \rVert = 1} \left\langle Tx, x \right\rangle \).
		\item The endpoints \(m, M\in \sigma (T)\).
	\end{enumerate}
\end{theorem}
\begin{proof}
	We prove this one by one.
	\begin{enumerate}[(a)]
		\item Let \(\lambda \in \mathbb{C} -[m, M]\), and set \(d\) be
		      \[
			      d\coloneqq \mathop{\mathrm{dist}}(\lambda , [m, M])
			      = \inf _{m \leq y\leq M} \vert \lambda -y \vert > 0.
		      \]
		      Then we have
		      \[
			      \lVert (T-\lambda I)x \rVert
			      \geq \vert \left\langle (T-\lambda I)x, x \right\rangle  \vert
			      = \vert \left\langle Tx, x \right\rangle - \lambda  \vert
			      \geq d
		      \]
		      since \(\lVert x \rVert = 1\), which implies \(T - \lambda I\) is bounded below, hence \(\lambda \in \rho (T)\) from \autoref{col:criterion-of-spectrum-points}.
		\item Without loss of generality, assume that \(0 \leq m \leq M\) by considering \(T - mI\) instead of \(T\). Now, choose a sequence \(\left\{ x_n \right\} _{n\geq 1}\) in \(\mathcal{H} \) where \(\lVert x_n \rVert = 1\) such that
		      \[
			      \lim_{n \to \infty} \left\langle Tx_n, x_n \right\rangle = M.
		      \]
		      By the \hyperref[lma:parallelogram-law]{parallelogram law},
		      \[
			      \lVert (T-MI)x_n \rVert ^{2}
			      = \left\langle (T-MI)x_n, (T-MI)x_n \right\rangle
			      = \lVert Tx_n \rVert ^2 - 2M\left\langle Tx_n, x_n \right\rangle + M^{2} \lVert x_n \rVert ^{2}.
		      \]
		      Since we already showed that \(\lVert T \rVert = M\), i.e., \(\lVert T \rVert = \sup _{\lVert x \rVert = 1}\left\langle Tx, x \right\rangle \) from \(\left\langle Tx, x \right\rangle \geq 0\), by letting \(n \to \infty \), the right-hand side goes to \(\leq 0\) since \(\lVert Tx_n \rVert ^{2} \leq M^{2} \lVert x_n \rVert ^{2} \) and \(\left\langle Tx_n, x_n \right\rangle \to M\), we may conclude that
		      \[
			      \lim_{n \to \infty} \lVert (T-MI)x_n \rVert = 0,
		      \]
		      and hence \(M\in \sigma (T)\).
	\end{enumerate}
\end{proof}

As a consequence, \(r(T) = \lVert T \rVert \) for \(T\) being a \hyperref[def:self-adjoint-op]{self-adjoint operator}. This means that \autoref{prop:lec20-2} is tight, while \hyperref[thm:Gelfand-formula]{Gelfand's formula} is useless for \hyperref[def:self-adjoint-op]{self-adjoint operators}! This observation leads to the following.

\begin{corollary}[Spectral radius]\label{col:spectral-radius}
	Let \(T\in \mathcal{L} (\mathcal{H} )\) for \(T\) being a \hyperref[def:self-adjoint-op]{self-adjoint operator}. Then
	\[
		r(T) = \max _{\lambda \in \sigma (T)} \vert \lambda \vert = \lVert T \rVert.
	\]
\end{corollary}
\begin{proof}
	From the property of \hyperref[thm:spectrum-interval]{spectrum interval}, we know that \(r(T) = \max (\vert m \vert , \vert M \vert ) = \lVert T \rVert\).
\end{proof}

\section{Spectral Theorem for Compact Self-Adjoint Operators}
\hyperref[def:compact-op]{Compact} \hyperref[def:self-adjoint-op]{self-adjoint operators} on a \hyperref[def:Hilbert-space]{Hilbert space} \(\mathcal{H} \) are proxies of Hermitian matrices. As we know from linear algebra, every Hermitian matrix has diagonal form in some \hyperref[def:orthonormal-basis]{orthonormal basis} of \(\mathbb{C} ^n\), or equivalently, there exists an \hyperref[def:orthonormal-basis]{orthonormal basis} of \(\mathbb{C} ^n\) consisting of the eigenvectors. In this section, we generalize this fact to infinite dimensions, for all \hyperref[def:compact-op]{compact} \hyperref[def:self-adjoint-op]{self-adjoint operators} on \(\mathcal{H} \).

\subsection{Invariant Subspaces}
\begin{proposition}[Eigenvectors orthogonal]\label{prop:eigenvectors-orthogonal}
	Let \(T\in \mathcal{L} (\mathcal{H} )\) be a \hyperref[def:self-adjoint-op]{self-adjoint operator} on \(\mathcal{H} \). Then its eigenvectors corresponding to distinct eigenvalues are \hyperref[def:orthogonal-system]{orthogonal}.
\end{proposition}
\begin{proof}
	If \(Tx_1 = \lambda _1 x_1\) and \(Tx_2 = \lambda _2 x_2\), then
	\[
		\lambda _1 \left\langle x_1, x_2 \right\rangle
		= \left\langle Tx_1, x_2 \right\rangle
		= \left\langle x_1, Tx_2 \right\rangle
		= \lambda _2 \left\langle x_1, x_2 \right\rangle,
	\]
	so if \(\lambda _1 \neq \lambda _2\), then \(\left\langle x_1, x_2 \right\rangle = 0\), proving the result.
\end{proof}

\begin{definition}[Invariant subspace]\label{def:invariant-subspace}
	A subspace \(E\) of \(\mathcal{H} \) is an \emph{invariant subspace} of \(T\) if \(T(E) \subseteq E\).
\end{definition}

\begin{eg}
	Every eigenspace of \(T\) is \hyperref[def:invariant-subspace]{invariant}. More generally, the linear span of any subset of eigenvectors of \(T\) is an \hyperref[def:invariant-subspace]{invariant subspace}.
\end{eg}

One of the most well-known open problems in functional analysis is the \emph{invariant subspace problem}. It asks whether every operator \(T\in \mathcal{L} (\mathcal{H} )\) has a proper \hyperref[def:invariant-subspace]{invariant subspace}, i.e., different from \(\left\{ 0 \right\} \) and \(\mathcal{H} \). While we don't know the answer for this, we have the following characterization.

\begin{proposition}\label{prop:lec23}
	Let \(T\in \mathcal{L} (\mathcal{H} )\) be a \hyperref[def:self-adjoint-op]{self-adjoint operator} on \(\mathcal{H} \). If \(E \subseteq \mathcal{H} \) is an \hyperref[def:invariant-subspace]{invariant subspace} of \(T\), then \(E^{\perp} \) is also an \hyperref[def:invariant-subspace]{invariant subspace} of \(T\).
\end{proposition}
\begin{proof}
	Let \(x\in E^{\perp} \), and we need to check that \(Tx \in E ^{\perp} \) given \(E\) is \hyperref[def:invariant-subspace]{invariant}. Let's choose \(y\in E\) arbitrarily, then we have
	\[
		\left\langle Tx , y \right\rangle = \left\langle x, Ty \right\rangle = 0
	\]
	since \(x\in E^{\perp} \) and \(y\in E\), hence \(Ty\in E\), as required.
\end{proof}

\subsection{Spectral Theorem}
The following result is known as the Hilbert-Schmidt theorem.

\begin{theorem}[Spectral theorem for compact self-adjoint operator]\label{thm:spectral-theorem-for-compact-self-adjoint-op}
	Let \(T\) be a \hyperref[def:compact-op]{compact} \hyperref[def:self-adjoint-op]{self-adjoint operator} on a \hyperref[def:separable]{separable} \(\mathcal{H} \). Then there exists an \hyperref[def:orthonormal-basis]{orthonormal basis} of \(\mathcal{H} \) consisting of eigenvectors of \(T\).
\end{theorem}
\begin{proof}
	Let's first prove that \(T\) has at least one eigenvector. Firstly, since \(T\) is \hyperref[def:compact-op]{compact}, by \autoref{prop:classification-of-spectrum-of-compact-op},
	\[
		\sigma (T) = \sigma _p(T) \cup \left\{ 0 \right\}.
	\]
	So if \(\sigma (T) \neq \left\{ 0 \right\} \), then \(\sigma _p(T) \neq \varnothing \), i.e., \(T\) has an eigenvector. Otherwise, if \(\sigma (T) = \left\{ 0 \right\} \), then from \autoref{col:spectral-radius}, \(r(T) = \lVert T \rVert = 0\), i.e., \(T \equiv 0\). In this case, any \hyperref[def:orthonormal-basis]{orthonormal basis} gives a basis of eigenvectors,\footnote{Since every vector is an eigenvector of \(T\).} a contradiction.

	Now suppose \(T\) has an eigenvector with \(\sigma_p (T) \neq \varnothing \). The fact that \(\mathcal{H} \) is \hyperref[def:separable]{separable}, all such basis are at most countable, so from \href{https://en.wikipedia.org/wiki/Zorn%27s_lemma}{Zorn's lemma}, this family has a maximal element \(\left\{ \phi _k \right\}_{k\geq 1} \), so the result follows by showing that 
	\[
		E \coloneqq \overline{\mathop{\mathrm{span}}(\left\{ \phi _k \right\}_{k\geq 1} )} = \mathcal{H} .
	\]
	Suppose \(E \neq \mathcal{H} \). Since \(E\) is an \hyperref[def:invariant-subspace]{invariant subspace} of \(T\), \(E^{\perp} \neq \left\{ 0 \right\} \) is also an \hyperref[def:invariant-subspace]{invariant subspace} of \(T\) by \autoref{prop:lec23}. By using the first part of the proof for the restriction \(\at{T}{E^{\perp} }{} \) which is a \hyperref[def:compact-op]{compact} \hyperref[def:self-adjoint-op]{self-adjoint operator} on \(E^{\perp} \). It follows that \(\at{T}{E^{\perp} }{} \), and thus \(T\) itself, has an eigenvector in \(E^{\perp} \). But this contradicts the maximality of \(\left\{ \phi _k \right\}_{k\geq 1} \), so \(E = \mathcal{H} \).
\end{proof}

Finally, we introduce a new kind of operators called \hyperref[def:normal-op]{normal operators}, where the above result generalizes to which.

\begin{definition}[Normal operator]\label{def:normal-op}
	An operator \(T\) is \emph{normal} if \(T T^{\ast} = T^{\ast} T\).
\end{definition}

\begin{remark}
	The \hyperref[thm:spectral-theorem-for-compact-self-adjoint-op]{spectral theorems for compact self-adjoint operators} extend to \hyperref[def:normal-op]{normal operator}.
\end{remark}

However, \hyperref[def:spectrum-point]{spectrum} of \hyperref[def:normal-op]{normal operators} do not have to be real,\footnote{For example, the unitary operators \(U^{\ast} U = U U^{\ast} = I\).} we only have \(\sigma (T) \subseteq \left\{ \lambda \in \mathbb{C} \colon \vert \lambda  \vert =1 \right\} \).

\section{Continuous Functional Calculus}
In this section, we develop the analogy between numbers and operators by introducing a \hyperref[def:partial-order]{partial order} on the set of \hyperref[def:self-adjoint-op]{self-adjoint operators} on \(T\in \mathcal{L} (\mathcal{H} )\), and we define an operator \(f(T) \in \mathcal{L} (\mathcal{H} )\) for every continuous function \(f\colon \mathbb{C} \to \mathbb{C} \). This is the so-called \emph{functional calculus} of operators.

\subsection{Positive Operators}
\begin{definition}[Positive operator]\label{def:positive-op}
	A \hyperref[def:self-adjoint-op]{self-adjoint operator} \(T\in \mathcal{L} (\mathcal{H} )\) is \emph{positive} if for all \(x\in \mathcal{H} \),
	\[
		\left\langle Tx, x \right\rangle \geq 0.
	\]
\end{definition}

\begin{eg}
	\(T^2\) for every \hyperref[def:self-adjoint-op]{self-adjoint} \(T\in \mathcal{L} (\mathcal{H} )\) is \hyperref[def:positive-op]{positive}.
\end{eg}
\begin{explanation}
	Since \(\left\langle T^2 x, x \right\rangle = \left\langle Tx, Tx \right\rangle \geq 0\).
\end{explanation}

\begin{eg}
	Hermitian matrices, or more generally, \hyperref[def:compact-op]{compact} \hyperref[def:self-adjoint-op]{self-adjoint operators} on \(\mathcal{H} \) with non-negative eigenvalues are \hyperref[def:positive-op]{positive}.
\end{eg}

\begin{note}
	\hyperref[def:positive-op]{Positive operators} are generalizations of non-negative numbers, which correspond to operators on one-dimensional space \(\mathbb{C} \).
\end{note}

\begin{remark}[Positive semi-definite]
	In linear algebra, \hyperref[def:positive-op]{positive operators} are called positive semi-definite. Furthermore, we denote \(A\) being positive semi-definite by \(A \succeq 0\), and define a \href{https://en.wikipedia.org/wiki/Partially_ordered_set}{partial order} between \(A\) and \(B\) by  \(A \succeq B \iff A - B \succeq 0\).
\end{remark}

Consider the analogous notion for \hyperref[def:self-adjoint-op]{self-adjoint operators}, where \(T \geq 0\) means \(T\) is \hyperref[def:positive-op]{positive}. Then we have the following.

\begin{definition}[Partial order]\label{def:partial-order}
	For \hyperref[def:self-adjoint-op]{self-adjoint operators} \(S, T\in \mathcal{L} (\mathcal{H} )\), we say \(S \leq T\) if \(T - S \geq 0\).
\end{definition}

\autoref{def:partial-order} defines a \href{https://en.wikipedia.org/wiki/Partially_ordered_set}{partial order} on \(\mathcal{L} (\mathcal{H} )\), and we may restate the \hyperref[thm:spectrum-interval]{spectrum interval theorem} with this new notion.

\begin{theorem}[Spectrum interval]\label{thm:spectrum-interval-2}
	Let \(T\in \mathcal{L} (\mathcal{H} )\) be a \hyperref[def:self-adjoint-op]{self-adjoint operator}, and let \(m, M\) be the smallest and the largest numbers such that \(mI \leq T \leq MI\). Then \(\sigma (T) \subseteq [m, M]\) and \(m, M\in \sigma (T)\).
\end{theorem}

As an immediate corollary, \(T\) is \hyperref[def:positive-op]{positive} if and only if its \hyperref[def:spectrum-point]{spectrum} is positive.

\begin{corollary}\label{col:positive-op-iff-spectrum-positive}
	Let \(T\in \mathcal{L} (\mathcal{H} )\) be a \hyperref[def:self-adjoint-op]{self-adjoint operator}, then \(T \geq 0\) if and only if \(\sigma (T) \subseteq [0, \infty )\).
\end{corollary}

\subsection{Polynomials of Operators}
We start to develop a functional calculus for \hyperref[def:self-adjoint-op]{self-adjoint operators} \(T\in \mathcal{L} (\mathcal{H} )\) by defining polynomials of \(T\), and then we extend the definition to continuous functions of \(T\) by approximation. Working with polynomials is straightforward, and the result of this subsection remain valid for every \hyperref[def:bounded-linear-op]{bounded linear operator} \(T\) on a general \hyperref[def:Banach-space]{Banach space} \(X\).

\begin{definition}[Polynomial operator]\label{def:polynomial-op}
	Consider a polynomial \(p(t) = a_0 + a_1 t + \dots  + a_n t^n\), then for an operator \(T\in \mathcal{L} (\mathcal{H} )\), we define
	\[
		p(T) \coloneqq a_0 I + a_1 T + \dots  + a_n T^n.
	\]
\end{definition}

We first note that if \(T\) is \hyperref[def:self-adjoint-op]{self-adjoint}, then \(p(T)\) is also \hyperref[def:self-adjoint-op]{self-adjoint} if \(p\) is real since
\[
	\begin{split}
		\left\langle p(T) x, y \right\rangle
		&= \left\langle (a_0 I + a_1 T + \dots  + a_n T^n)x, y \right\rangle \\
		&= a_0 \left\langle x, y \right\rangle + a_1 \left\langle Tx, y \right\rangle + \dots + a_n \left\langle T^n x, y \right\rangle \\
		&= \left\langle x, \overline{a_0} y \right\rangle + \left\langle x, \overline{a_1}Ty \right\rangle + \dots + \left\langle x, \overline{a_n}T^n y \right\rangle \\
		&= \left\langle x, a_0 y \right\rangle + \left\langle x, a_1 Ty \right\rangle + \dots + \left\langle x, a_n T^n y \right\rangle \\
		&= \left\langle x, (a_0 I + a_1 T + \dots + a_n T^n) y \right\rangle \\
		&= \left\langle x, p(T)y \right\rangle.
	\end{split}
\]

Moreover, we have the following properties for \hyperref[def:polynomial-op]{polynomial operators}.

\begin{proposition}\label{prop:polynomial-op}
	Let \(p, q\) be complex polynomials and \(T\in \mathcal{L} (\mathcal{H} )\).
	\begin{enumerate}[(a)]
		\item \((ap + bq)(T) = ap(T) + bq(T)\) for \(a, b\in \mathbb{C} \).
		\item \((pq)(T) = p(T) q(T)\).
		\item \(p(T)^{\ast} = \overline{p} (T^{\ast} )\).\footnote{\(\overline{p} \) is the polynomial with coefficients that are complex conjugates of the coefficients of \(p\).}
	\end{enumerate}
\end{proposition}

The following example may serve us as a test case for many future results.

\begin{eg}
	Let \(T\) be a \hyperref[def:self-adjoint-op]{self-adjoint} \hyperref[def:linear-op]{linear operator} on an \(n\)-dimensional \hyperref[def:Hilbert-space]{Hilbert space}. In an \hyperref[def:orthonormal-basis]{orthonormal basis} of eigenvectors, \(T\) can be identified with the \(n \times n\) diagonal matrix
	\[
		T = \diag (\lambda _1, \dots , \lambda _n),
	\]
	where \(\lambda _k\) are the eigenvalues of \(T\). Then for every polynomial \(p(t)\), we have
	\[
		p(T) = \diag (p(\lambda _1), \dots  , p(\lambda _n)).
	\]
	This can be generalized for all \hyperref[def:compact-op]{compact} \hyperref[def:self-adjoint-op]{self-adjoint operators} \(T\) on a general \hyperref[def:Hilbert-space]{Hilbert space} \(\mathcal{H} \).
\end{eg}

\subsection{Spectral Mapping Theorem for Polynomial Operators}
Let's study some important theorem for the \hyperref[def:polynomial-op]{polynomial operators}.

\begin{lemma}[Invertibility for polynomial operator]\label{lma:invertibility-for-polynomial-op}
	Let \(p(t)\) be a polynomial and \(T\in \mathcal{L} (\mathcal{H} )\), then \(p(T)\) is invertible if and only if \(p(t) \neq 0\) for all \(t\in \sigma (T)\).
\end{lemma}
\begin{proof}
	Consider
	\[
		p(t) = a_n (t-t_1)(t-t_2)\dots  (t-t_n)
	\]
	where \(t_1, \dots  , t_n\) are zeros of \(p(\cdot)\), then
	\[
		p(T) = a_n (T-t_1 I)\dots  (T-t_n I).
	\]
	We see that if \(p(t) \neq 0\) for \(t\in \sigma (T)\), \(p(T)\) is clearly invertible.

	Conversely, observe that if \(S, R\in \mathcal{L} (\mathcal{H} )\) and \(SR\) is invertible, then both \(S\) and \(R\) are invertible.\footnote{Since if \(SR\) is invertible, then \(S^{-1} = R(SR)^{-1} \).} Hence, if \(p(T)\) is invertible, then \(T - t_k I\) are all invertible, i.e., \(t_1, \dots  , t_n\in \rho (T)\).
\end{proof}

The \hyperref[def:spectrum-point]{spectrum} of a polynomial \(p(T)\) can be easily computed from the \hyperref[def:spectrum-point]{spectrum} of \(T\).

\begin{theorem}[Spectral mapping theorem for polynomial operator]\label{thm:spectral-mapping-for-polynomial-op}
	Let \(p(t)\) be a polynomial and \(T\in \mathcal{L} (\mathcal{H} )\). Then
	\[
		\sigma (p(T)) = p(\sigma (T))
	\]
	where \(p(\sigma (T)) \coloneqq \left\{ p(t) \colon t\in \sigma (T) \right\} \).
\end{theorem}
\begin{proof}
	Since \(\lambda \in \sigma (p(T))\) if and only if \(p(T) - \lambda I\) is not invertible, which from \autoref{lma:invertibility-for-polynomial-op}, it is equivalent to say \(p(t) - \lambda = 0\) for some \(t\in \sigma (T)\), i.e., \(\lambda \in p(\sigma (T))\).
\end{proof}

Using the \hyperref[thm:spectral-mapping-for-polynomial-op]{spectral mapping theorem for polynomial operator}, one can in particular easily compute the \hyperref[def:norm]{norm} of \hyperref[def:polynomial-op]{polynomial operator}.

\begin{corollary}[Operator norm of polynomial operator]\label{col:op-norm-of-polynomial-op}
	Suppose \(T\) is a \hyperref[rmk:bounded-op]{bounded} \hyperref[def:self-adjoint-op]{self-adjoint operator} on \(\mathcal{H} \) and \(p(t)\) is a polynomial with real coefficients. Then \(p(T)\) is \hyperref[def:self-adjoint-op]{self-adjoint} and
	\[
		\lVert p(T) \rVert = \sup _{t\in \sigma (T)}\vert p(t) \vert.
	\]
\end{corollary}
\begin{proof}
	Firstly, \(p(T)\) is \hyperref[def:self-adjoint-op]{self-adjoint} since \(\overline{p} = p\) as noted in the beginning of the section, with the \autoref{col:spectral-radius} and \hyperref[thm:spectral-mapping-for-polynomial-op]{spectral mapping theorem for polynomial operator},
	\[
		\lVert p(T) \rVert = r(p(T)) = \sup _{t\in \sigma (p(T))} \vert t \vert = \max _{t\in p(\sigma (T))} \vert t \vert = \max _{s\in \sigma (T)} \vert p(s) \vert .
	\]
\end{proof}

\autoref{col:op-norm-of-polynomial-op} generalizes \autoref{col:spectral-radius}, which states that \(r(T) = \lVert T \rVert \) for the \hyperref[def:spectral-radius]{spectral radius} of \hyperref[def:self-adjoint-op]{self-adjoint operators} \(T\).