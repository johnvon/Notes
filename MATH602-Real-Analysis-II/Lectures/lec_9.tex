\lecture{9}{27 Sep. 14:30}{Hahn-Banach Theorem for Sublinear Functions}
From \autoref{prop:lec8}, we see that to show a \hyperref[def:Banach-space]{Banach space} \(E\) is not \hyperref[def:reflexive-space]{reflexive}, it's sufficient to find \(f\in E^{\ast} \) such that \(\sup _{\left\lVert x\right\rVert = 1}\left\vert f(x) \right\vert \) is not attained.

\begin{eg}
	Let \(C([0, 1])\) be the space of continuous functions \(g\colon [0, 1]\to \mathbb{C}\) with \(\left\lVert g\right\rVert \coloneqq \sup _{0 \leq t\leq 1} \left\vert g(t) \right\vert \). Then for \(f\in E^{\ast} \),
	\[
		f(g) = \int _0 ^1 h(x)g(x)\,\mathrm{d} x
	\]
	for
	\[
		h(x) = \begin{dcases}
			-1, & \text{ if } 0 < x < \frac{1}{2}; \\
			1,  & \text{ if } \frac{1}{2} < x < 1.
		\end{dcases}
	\]
	Then \(\left\lVert f\right\rVert = 1 = \sup _{\left\lVert g\right\rVert = 1}\left\vert f(g) \right\vert \), but the supremum is not attained since \(g\) needs to be continuous.
\end{eg}

\section{Separation of Convex Sets}
In this section, we can extend \hyperref[thm:supporting-functional]{supporting functional theorem} such that we now have it for arbitrary \hyperref[def:convex-set]{convex sets} other than the unit \hyperref[def:ball]{ball}. Since \hyperref[thm:supporting-functional]{supporting functional theorem} depends on \hyperref[thm:Hahn-Banach]{Hahn-Banach theorem}, so we should first generalize \hyperref[thm:Hahn-Banach]{Hahn-Banach theorem}.

\subsection{Hahn-Banach Theorem for Sublinear Functions}
By looking into the proof of \hyperref[thm:Hahn-Banach]{Hahn-Banach theorem}, we see that we only used positive homogeneity and triangle inequality of the axiom of \hyperref[def:norm]{norm}, which suggests we define the following.

\begin{definition}[Sublinear]\label{def:sublinear}
	Let \(E\) be a \hyperref[def:linear-vector-space]{linear vector space}. a function \(\left\lVert \cdot\right\rVert \colon E \to [0, \infty )\) is \emph{sublinear} if it satisfies
	\begin{enumerate}[(a)]
		\item \(\left\lVert \lambda x\right\rVert = \lambda \left\lVert x\right\rVert \) for \(\lambda \in \mathbb{R}^+ \), \(x\in E\).
		\item \(\left\lVert x + y\right\rVert \leq \left\lVert x\right\rVert + \left\lVert y\right\rVert\), \(x, y\in E\).
	\end{enumerate}
\end{definition}

\begin{remark}[Differences from norm]
	Note that for a \hyperref[def:sublinear]{sublinear} function to be a \hyperref[def:norm]{norm}, we need
	\begin{enumerate}[(a)]
		\item \(\left\lVert -x\right\rVert =\left\lVert x\right\rVert \), \(x\in E\)
		\item \(\left\lVert x\right\rVert = 0 \implies x = 0\).
	\end{enumerate}
\end{remark}

Now, we can then generalize \hyperref[thm:Hahn-Banach]{Hahn-Banach theorem} to \hyperref[def:sublinear]{sublinear functions}.

\begin{theorem}[Hahn-Banach theorem for sublinear functions]\label{thm:hahn-Banach-Sublinear}
	Let \(E_0\) be a subspace of a \hyperref[def:linear-vector-space]{linear vector space} over \(\mathbb{R} \). Let \(\left\lVert \cdot\right\rVert \) be a \hyperref[def:sublinear]{sublinear functional} on \(E\), and \(f_0 \colon E_0\to \mathbb{R} \) be a \hyperref[def:linear-functional]{linear functional} on \(E_0\) satisfying \(f_0(x) \leq \left\lVert x\right\rVert \) for \(x\in E_0\). Then \(f_0\) admits an extension \(f\) to \(E\) such that \(f(x) \leq \left\lVert x\right\rVert \) for \(x\in E\).
\end{theorem}
\begin{proof}
	The idea is the same from \hyperref[thm:Hahn-Banach]{Hahn-Banach theorem}.
\end{proof}

\subsection{Geometric Properties of Sublinear Functions}
We see that by considering \hyperref[def:sublinear]{sublinear functionals} instead of \hyperref[def:norm]{norms} offers us more flexibility in geometric applications. In particular, \hyperref[def:sublinear]{sublinear functionals} arise as \hyperref[def:Minkowski-functional]{Minkowski functionals} of \hyperref[def:convex-set]{convex sets}.
\begin{definition}[Absorbing]\label{def:absorbing}
	A subset \(K\) of a \hyperref[def:linear-vector-space]{linear vector space} is \emph{absorbing} if
	\[
		E = \bigcup\limits_{t \geq 0} tK
	\]
	where \(tK \coloneqq \left\{ tk \colon k\in K \right\} \).
\end{definition}

\begin{definition}[Minkowski functional]\label{def:Minkowski-functional}
	Let \(K\) be an \hyperref[def:absorbing]{absorbing} \hyperref[def:convex-set]{convex} subset of a \hyperref[def:linear-vector-space]{linear vector space} \(E\) such that \(0\in K\). Then the \emph{Minkowski functional} \(\left\lVert \cdot\right\rVert _K\) is defined as
	\[
		\left\lVert x\right\rVert _K \coloneqq \inf \left\{ t > 0 \colon x / t \in K \right\}.
	\]
\end{definition}

\begin{proposition}
	Let \(K\) be an \hyperref[def:absorbing]{absorbing} \hyperref[def:convex-set]{convex} subset of a \hyperref[def:linear-vector-space]{linear vector space} \(E\) such that \(0\in K\). Then \hyperref[def:Minkowski-functional]{Minkowski functional} \(\left\lVert x\right\rVert _K\) is a \hyperref[def:sublinear]{sublinear functional} on \(E\). Conversely, let \(\left\lVert \cdot\right\rVert \) be a \hyperref[def:sublinear]{sublinear functional} on a \hyperref[def:linear-vector-space]{linear vector space} \(E\), then the sub-level set
	\[
		K = \left\{ x\in E\colon \left\lVert x\right\rVert \leq 1 \right\}
	\]
	is an \hyperref[def:absorbing]{absorbing} \hyperref[def:convex-set]{convex set}, and \(0\in K\).
\end{proposition}
\begin{proof}
	To prove the forward direction, the main observation is that since \(0\in K\) and \(K\) is \hyperref[def:convex-set]{convex}, then \(x\in K \implies tx\in K\) if \(0 \leq t < 1\). To show dilation, for \(\lambda > 0\),
	\[
		\left\lVert \lambda x\right\rVert
		= \inf \left\{ t> 0\colon x \in \frac{t}{\lambda }K \right\}
		= \lambda \inf \left\{ s > 0\colon x \in sK \right\} = \lambda \left\lVert x\right\rVert.
	\]
	To show triangle inequality, suppose \(x\in tK\), \(y\in sK\), then \(x = tk_1\), \(y = sk_2\) for some \(k_1, k_2\in K\). We then have
	\[
		x + y = (t + s) \left( \frac{t}{t+s}k_1 + \frac{s}{t+s}k_2 \right) = (t + s) k
	\]
	for some \(k\in K\) since \(K\) is \hyperref[def:convex-set]{convex}, hence \(x + y \in (t + s) K\), we then have \(\left\lVert x + y\right\rVert \leq \left\lVert x\right\rVert + \left\lVert y\right\rVert \).

	Now, if \(\left\lVert \cdot\right\rVert \) is \hyperref[def:sublinear]{sublinear}, then \(K = \left\{ x\in E\colon \left\lVert x\right\rVert \leq 1 \right\} \) is \hyperref[def:absorbing]{absorbing}, \hyperref[def:convex-set]{convex} and \(0\in K\).\footnote{\(0\in K\) since \(\left\lVert 0\right\rVert = 0\), while the \hyperref[def:convex-set]{convexity} comes from the triangle inequality.}
\end{proof}

\begin{remark}
	If \(K \neq -K\), then \(\exists x\in E\) with \(\left\lVert x\right\rVert \neq \left\lVert -x\right\rVert \). If \(K = E\), then \(\left\lVert \cdot \right\rVert \equiv 0\).
\end{remark}

\subsection{Separation of Convex Sets}
\hyperref[thm:Hahn-Banach]{Hahn-Banach theroem} has some remarkable geometric implications, which are grouped together under the name of \emph{separation theorems}. Under mild topological requirements, these results guarantee that two \hyperref[def:convex-set]{convex sets} \(A\), \(B\) can always be separated by a \hyperref[def:hyperplane]{hyperplane}.

\begin{theorem}[Separation of a point from a convex set]\label{thm:separation-of-a-point-from-a-convex-set}
	Let \(K\) be an open convex subset of a normed space \(E\) and \(x_0 \notin K\). Then there exists a continuous \hyperref[def:linear-functional]{linear functional} \(f\colon E\to \mathbb{R} \) with \(f\neq 0\) and \(f(x) < f(x_0)\) for \(x\in K\).
\end{theorem}
\begin{proof}
	By translation, we can assume without loss of generality that \(0\in K\). Since \(K\) is open, it is \hyperref[def:absorbing]{absorbing}. Now, let \(\left\lVert \cdot\right\rVert _K\) be the \hyperref[def:Minkowski-functional]{Minkowski functional}, then
	\[
		\left\lVert x\right\rVert _K \leq \frac{1}{r}\left\lVert x\right\rVert
	\]
	for \(x\in E\) if \(B(0, r)\subseteq K\).
	\begin{center}
		\incfig{separation-of-a-point-from-convex-set}
	\end{center}
	Proceed as in \hyperref[thm:supporting-functional]{supporting functional theorem} for unit \hyperref[def:ball]{ball}, we define \(f_0\) on \(\mathop{\mathrm{span}}(\left\{ x_0 \right\} )\) by
	\[
		f_0(tx_0) = t\left\lVert x_0\right\rVert_K
	\]
	for \(t\in \mathbb{R} \). Then if \(E_0 = \left\{ \lambda x_0 \colon \lambda \in \mathbb{R}  \right\} \), \(f_0 (x) \leq \left\lVert x\right\rVert _K\) for \(x\in E_0\) (i.e., \(\left\lVert \cdot\right\rVert _K\) dominates \(f_0\)) since for \(t \geq 0\),
	\[
		f_0 (tx_0) = t \left\lVert x_0\right\rVert _K= \left\lVert t x_0\right\rVert_K;
	\]
	while for \(t \leq 0\),
	\[
		f_0 ( tx_0) = t\left\lVert x_0\right\rVert_K \leq 0\leq \left\lVert t x_0\right\rVert _K.
	\]
	Then from \hyperref[thm:Hahn-Banach]{Hahn-Banach theorem}, we can extend \(f_0\) to \(f\colon E\to \mathbb{R} \) such that
	\[
		f(x) \leq \left\lVert x\right\rVert _K \leq \frac{1}{r} \left\lVert x\right\rVert
	\]
	for \(x\in E\), hence \(f\in E^{\ast} \). For separation, we see that if \(x\in K\) (hence in \(E\)),
	\[
		f(x) \leq \left\lVert x\right\rVert _K \leq 1 \leq \left\lVert x_0\right\rVert _K = f_0 (x_0) = f(x_0),
	\]
	hence \(f(x) \leq f(x_0)\). To get a strict separation, since \(K\) is open, so \(x + tv\in K\) for \(x\in K\) and some \(t > 0\) and all \(v\) with \(\left\lVert v\right\rVert = 1\). Hence, for all \(t = t_x > 0\), we have
	\[
		f(x + tv) \leq f(x_0) \implies f(x) + t \sup _{\left\lVert v\right\rVert = 1}f(v) \leq f(x_0).
	\]
	With the fact that \(f\neq 0\), so \(\left\lVert f\right\rVert = \sup _{\left\lVert v\right\rVert = 1}f(v) \neq 0\), we conclude that
	\[
		f(x) < f(x_0).
	\]
\end{proof}

A more general version holds.

\begin{theorem}[Separation of convex sets]\label{thm:separation-of-convex-sets}
	Let \(A, B\) be disjoint \hyperref[def:convex-set]{convex subsets} of a \hyperref[def:Banach-space]{Banach space} \(E\).
	\begin{enumerate}[(a)]
		\item If \(A\) is open, then there \(\exists f\colon E\to \mathbb{R} \) such that \(f(a) < f(b)\) for \(a\in A\), \(b\in B\).
		\item If \(A\), \(B\) are closed and \(B\) is compact, then there \(\exists f\colon E\to \mathbb{R} \) such that \(\sup _{a\in A} f(a) < \inf _{b\in B}f(b)\).
	\end{enumerate}
\end{theorem}
\begin{proof}
	We have the following.
	\begin{enumerate}[(a)]
		\item Let \(K = A - B = \left\{ a - b\colon a\in A, b\in B \right\} \), we then see that \(K\) is open, \hyperref[def:convex-set]{convex} and \(0 \notin K\). Since we can \hyperref[thm:separation-of-a-point-from-a-convex-set]{separate a point from a convex set}, there exists \(f\in E^{\ast} \) such that
		      \[
			      f(a - b) < f(0) = 0
		      \]
		      for \(a\in A\), \(b\in B\), hence \(f(a) < f(b)\) for \(a\in A\), \(b\in B\).
		\item Let \(A\) be closed, \(B\) be compact. Then we have
		      \[
			      d(A, B) = \inf \left\{ \left\lVert x-y\right\rVert \colon x\in A, y\in B \right\} = r > 0.
		      \]
		      Define \(A_\delta \coloneqq \left\{ x\in E\colon d(x, A) < \delta  \right\} \) where \(A_\delta \) is open. By setting \(\delta \coloneqq r / 2\), we have \(A_\delta \cap B = 0\). From (a), we see that there exists \(f\in E^{\ast} \) such that \(f(x) < f(y)\) for \(x\in A_\delta \), \(y\in B\). Then \(a\in A\) implies \(a + \delta /2 v \in A_\delta \) if \(\left\lVert v\right\rVert = 1\), hence
		      \[
			      f(a + \delta / 2 v)< f(b)
		      \]
		      for \(b\in B\). So
		      \[
			      f(a) + \frac{\delta }{2}f(v) < f(b)
		      \]
		      for \(b\in B\), \(\left\lVert v\right\rVert = 1\). Take the supremum over \(\left\lVert v\right\rVert = 1\), we have \(\sup _{\left\lVert v\right\rVert = 1} \left\vert f(v) \right\vert = \delta > 0\), implying \(f(a) < f(b) - \delta \), \(a\in A\), \(b\in B\). Finally, we have
		      \[
			      \sup _{a\in A}f(a) < \inf _{b\in B}f(b).
		      \]
	\end{enumerate}
\end{proof}