\lecture{24}{22 Nov. 14:30}{The Universal Spectral Mapping Theorem}
\subsection{Continuous Functions of Operators}
We're now going to study the \hyperref[thm:spectral-mapping]{spectral mapping theorem}, which is a generalization of finite dimensions \hyperref[def:self-adjoint-op]{self-adjoint} \(T\) on \(\mathbb{R} ^n\), where there exists an \hyperref[def:orthonormal-basis]{orthonormal basis} of eigenvectors. In this basis, the action of \(t\) is just multiplication by scalars. To generalize this to infinite dimensions, we need measure theory. In particular, we need \hyperref[thm:Weierstrass-approximation]{Weierstrass approximation theorem}.

\begin{theorem}[Weierstrass approximation theorem]\label{thm:Weierstrass-approximation}
	Let \(K \subseteq \mathbb{R} \) be \hyperref[def:compact]{compact} and \(f\colon K \to \mathbb{R} \) be continuous, then for any \(\epsilon > 0\), there exists a polynomial \(p^{\epsilon } \) such that
	\[
		\sup_{x\in K} \vert f(x) - p^{\epsilon }(x)  \vert < \epsilon .
	\]
\end{theorem}

Let \(T\) be \hyperref[def:self-adjoint-op]{self-adjoint} on \(\mathcal{H} \) and \(f\colon \sigma (T) \to \mathbb{R} \) be a continuous function, then we want to define a \hyperref[def:self-adjoint-op]{self-adjoint} operator \(f(T)\). To do this, we use \hyperref[thm:Weierstrass-approximation]{Weierstrass approximation theorem} to find polynomials \(p^{\epsilon _n}(t)\) such that
\[
	p^{\epsilon _n}(t) \to f(t)
\]
uniformly on \(\sigma (T)\). This suggests us to define \(f(T)\) as the limit of \hyperref[def:polynomial-op]{operator polynomials} \(p^{\epsilon _n}(T)\).

\begin{definition}[Continuous function operator]\label{def:continuous-function-op}
	The sequence \(p^{\epsilon _n}(T)\) converges in \(\mathcal{L} (\mathcal{H} )\) to a limit that we call \(f(T)\in \mathcal{L} (\mathcal{H} )\), which is also \hyperref[def:self-adjoint-op]{self-adjoint}.
\end{definition}

\begin{note}[Well-defined]
	To show that \autoref{def:continuous-function-op} is well-defined, we need to show that \(p^{\epsilon _n}(T)\) indeed converges to \(f(T)\), and \(f(T)\) is \hyperref[def:self-adjoint-op]{self-adjoint} and does not depend on the choice of the approximating polynomials \(p^{\epsilon _n}\).
\end{note}
\begin{explanation}
	To construct \(f(T)\), choose a sequence \(p^{\epsilon _n}(\cdot)\) of polynomial such that \(\epsilon _n \to 0\) and
	\[
		\sup _{x\in \sigma (T)} \vert f(x) - p^{\epsilon _n}(x)  \vert < \epsilon
		\implies \sup _{x\in \sigma (T)} \vert p^{\epsilon _n}(x) - p^{\epsilon _m}(x) \vert < \epsilon _n + \epsilon _m.
	\]
	From \autoref{col:op-norm-of-polynomial-op}, \(\lVert p(T) \rVert = \sup _{t\in \sigma (T)} \vert p(t) \vert \), so we have
	\[
		\lVert p^{\epsilon _n}(T) - p^{\epsilon _m}(T) \rVert < \epsilon _n + \epsilon _m,
	\]
	so the sequence \(\left\{ p^{\epsilon _n}(T) \right\} _{n\geq 1}\) is in \(\mathcal{L} (\mathcal{H} )\) is Cauchy, so there exists a limit \(f(T)\) with
	\[
		\lim_{n \to \infty} \lVert p^{\epsilon _n}(T) - f(T) \rVert = 0,
	\]
	which shows that \(f(T)\in \mathcal{L} (\mathcal{H} )\) is unique.

	Moreover, \(f(T)\) is \hyperref[def:self-adjoint-op]{self adjoint} since the \hyperref[def:self-adjoint-op]{self adjoint operators} form a closed subset of \(\mathcal{L} (\mathcal{H} )\), and by repeating the above estimation, given two approximating sequences \(p^{\epsilon _n}\) and \(q^{\epsilon _n}\), they will both converge to \(f(T)\).
\end{explanation}

By passing to the limit in the corresponding properties for \hyperref[def:polynomial-op]{polynomial operators} as in \autoref{prop:polynomial-op}, we see that these properties of \(f(T)\) inherited from properties of \(p(T)\) when \(p(\cdot)\) is a polynomial, e.g.,
\begin{enumerate}[(a)]
	\item \((af + bg)(T) = af(T) + bg(T)\);
	\item \((fg)(T) = f(T) g(T)\);
	\item \(f(T)^{\ast} = \overline{f} (T)\) for \(f\colon \sigma (T) \to \mathbb{C}\).
\end{enumerate}

\begin{note}
	The first two are also true for continuous complex functions \(f, g\colon \sigma (T) \to \mathbb{C} \).
\end{note}
\begin{explanation}
	If \(f\colon \sigma (T) \to  \mathbb{C} \) is complex, we can then write \(f = f_1 + if_2\) for \(f_1, f_2 \colon \sigma (T) \to \mathbb{R} \) such that
	\[
		f(T) \coloneqq f_1(T) + if_2(T).
	\]
\end{explanation}

\subsection{Spectral Mapping Theorem}
We will now generalize the \hyperref[thm:spectral-mapping-for-polynomial-op]{spectral mapping theorem for polynomial operators} to \hyperref[def:continuous-function-op]{continuous functions of an operator}. It is based on the straightforward generalization of the \hyperref[lma:invertibility-for-polynomial-op]{invertibility lemma for polynomial operators}.

\begin{lemma}[Invertibility]\label{lma:invertibility}
	Let \(T\in \mathcal{L} (\mathcal{H} )\) be a \hyperref[def:self-adjoint-op]{self-adjoint operator} and \(f\in C(\sigma (T))\). Then the operator \(f(T)\) is invertible if and only if \(f(t) \neq 0\) for all \(t\in \sigma (T)\).
\end{lemma}
\begin{proof}\todo{Add!}
\end{proof}

Now the \hyperref[thm:spectral-mapping]{spectral mapping theorem} follows from \hyperref[lma:invertibility]{invertibility lemma} by the same argument as the corresponding result for \hyperref[def:polynomial-op]{polynomial operators}, i.e., \autoref{thm:spectral-mapping-for-polynomial-op}.

\begin{theorem}[Spectral mapping theorem]\label{thm:spectral-mapping}
	Let \(T\in \mathcal{L} (\mathcal{H} )\) be \hyperref[def:self-adjoint-op]{self-adjoint} and \(f\in C(\sigma (T))\), then
	\[
		\sigma (f(T)) = f(\sigma (T)).
	\]
\end{theorem}

This gives a simple way to create \hyperref[def:positive-op]{positive operators}.

\begin{corollary}\label{col:positive-op-if-f-positive}
	Let \(T\in \mathcal{L} (\mathcal{H} )\) be a \hyperref[def:self-adjoint-op]{self-adjoint operator} and \(f\in C(\sigma (T))\). If \(f(t) \geq 0\) for all \(t\in \sigma (T)\), then \(f(T) \geq 0\).
\end{corollary}
\begin{proof}
	By generalizing \autoref{col:positive-op-iff-spectrum-positive}, it suffices to check \(\sigma (f(T)) \subseteq [0, \infty )\). From \hyperref[thm:spectral-mapping]{spectral mapping theorem},
	\[
		\sigma (f(T)) = f(\sigma (T)) \geq 0
	\]
	as desired.
\end{proof}

\subsection{Square Root of Operators}
Consider a \hyperref[def:positive-op]{positive} \hyperref[def:self-adjoint-op]{self-adjoint operator} \(T\in \mathcal{L} (\mathcal{H} )\), then \(\sigma (T) \subseteq [0, \infty )\). The function \(f(t) = \sqrt{t} \) is continuous on \([0, \infty )\), so we can define \(f(T) = \sqrt{T} \). A simple observation leads to the following.

\begin{proposition}[Square root of operator]\label{prop:square-root-of-op}
	Let \(T\in \mathcal{L} (\mathcal{H} )\) be \hyperref[def:positive-op]{positive} \hyperref[def:self-adjoint-op]{self-adjoint}, i.e., \(\left\langle Tx, x \right\rangle \geq 0\) for \(x\in \mathcal{H} \). Then there exists a unique \hyperref[def:positive-op]{positive} \hyperref[def:self-adjoint-op]{self-adjoint operator} \(\sqrt{T} \in \mathcal{L} (\mathcal{H} )\) such that
	\[
		(\sqrt{T} )^{2} = T.
	\]
\end{proposition}
\begin{proof}
	Let \(f\colon \sigma(T) \to \mathbb{R} \), \(f(t) = \sqrt{t} \) where \(\sigma (T) \subseteq \mathbb{R} ^+\) since \(T\) is \hyperref[def:positive-op]{positive}. Since \(f(t) = \sqrt{t} \) is continuous on \(\sigma (T)\), so we can define \(f(T) = \sqrt{T} \). Furthermore, since \(f(t) \geq 0\) for all \(t\in \sigma (T)\), \autoref{col:positive-op-if-f-positive} states that \(f(T) = \sqrt{T}\geq 0\). Finally, since \(\sqrt{t} ^2 = t\), the algebra homomorphism property implies that \((\sqrt{T} )^2 = T\) as well.\footnote{The uniqueness is left as an exercise.}
\end{proof}

\subsection{Modulus of Operators}
Now, consider an arbitrary operator \(T\in \mathcal{L} (\mathcal{H} )\), which is not necessarily \hyperref[def:self-adjoint-op]{self-adjoint}. Then \(T^{\ast} T\) is a \hyperref[def:positive-op]{positive} \hyperref[def:self-adjoint-op]{self-adjoint operator}, so it has a unique \hyperref[def:positive-op]{positive} square root. This suggests the following definition.

\begin{definition}[Modulus]\label{def:modulus}
	Let \(T\in \mathcal{L} (\mathcal{H} )\), then the \emph{modulus} of \(T\) is defined as \(\vert T \vert \coloneqq \sqrt{T^{\ast} T} \).
\end{definition}

This generalizes the concept of modulus of complex numbers, i.e., \(\vert z \vert = \sqrt{\overline{z} z} \) for \(z\in \mathbb{C} \).

\begin{lemma}\label{lma:lec24}
	For every operator \(T\in \mathcal{L} (\mathcal{H} )\) and vector \(x\in \mathcal{H} \), one has
	\[
		\lVert \vert T \vert x \rVert = \lVert Tx \rVert.
	\]
\end{lemma}
\begin{proof}
	Since
	\[
		\lVert \vert T \vert x \rVert ^2
		= \left\langle \vert T \vert x, \vert T \vert x \right\rangle
		= \left\langle \vert T \vert ^2 x, x \right\rangle
		= \left\langle T^{\ast} Tx, x \right\rangle
		= \left\langle Tx, Tx \right\rangle
		= \lVert Tx \rVert ^2.
	\]
\end{proof}

\autoref{lma:lec24} leads to the \hyperref[thm:polar-decomposition]{polar decomposition theorem}.

\subsection{Polar Decomposition}
\autoref{lma:lec24} motivates us to consider a well-defined \hyperref[def:linear-op]{linear map}
\[
	U\colon \vert T \vert x \mapsto Tx
\]
for \(x\in \mathcal{H} \), and \autoref{lma:lec24} states that \(U\in \mathcal{L} (\mathcal{H} )\) is an isometry.

\begin{theorem}[Polar decomposition]\label{thm:polar-decomposition}
	For every \(T\in \mathcal{L} (\mathcal{H} )\), there exists a unique bijective linear isometry \(U\in \mathcal{L} (\im (\vert T \vert ), \im T)\) such that \(T = U\vert T \vert \).
\end{theorem}
\begin{proof}
	Define \(U\) on \(\im(\vert T \vert )\) by \(U(\vert T \vert x) = Tx\) for all \(x\in \mathcal{H} \).\footnote{Notice that this makes sense since \(\vert Tx \vert = 0 \iff Tx = 0\).} Then, \(U\) is an isometry since \(\vert T \vert = \sqrt{T ^{\ast} T} \), and \(U\) is injective and \(U\) maps onto \(\im (T)\), so \(U\) is surjective.
\end{proof}

\hyperref[thm:polar-decomposition]{Polar decomposition} generalizes the polar decomposition in the complex plane. The latter states that every \(z\in \mathbb{C} \) can be represented as
\[
	z = e^{i \arg (z)} \vert z \vert,
\]
where \(e^{i \arg(z)}\) is a unit scalar (generalized by \(U\)), and \(\vert z \vert \) is the modulus of \(z\) (generalized by \(\vert T \vert\)).

\begin{theorem}[Polar decomposition for invertibile operator]\label{thm:polar-decomposition-for-invertible-op}
	If \(T\in \mathcal{L} (\mathcal{H} )\) is invertible, then there exists a unique \hyperref[def:unitary-op]{unitary} \(U\in \mathcal{L} (\mathcal{H} )\) such that
	\[
		T = U \vert T \vert .
	\]
\end{theorem}
\begin{proof}
	Since \(T\) is invertible, \(T^{\ast} \) is also invertible, hence \(T^{\ast} T\) is invertible. Finally, we can show that \(\vert T \vert = \sqrt{T^{\ast} T} \) is also invertible, therefore \(\im(T) = = \im(\vert T \vert ) = \mathcal{H}\), then the claim follows from \hyperref[thm:polar-decomposition]{polar decomposition}.
\end{proof}

\section{Borel Functional Calculus}
We can extend functional calculus to bounded Borel functions of operators. This is done primarily to define the spectral projections, which are indicator functions of an operator. Once we have spectral projections, we formulate the spectral theorem for general (not necessarily \hyperref[def:compact-op]{compact}) \hyperref[def:self-adjoint-op]{self-adjoint operators}.

As usual, let \(T\in \mathcal{L} (\mathcal{H} )\) be a fixed \hyperref[def:self-adjoint-op]{self-adjoint operator} on a \hyperref[def:Hilbert-space]{Hilbert space} \(\mathcal{H} \). Let's also fix the \hyperref[thm:spectrum-interval-2]{spectrum interval} \([m, M]\).

\subsection{Borel Functional Calculus}
We consider the \hyperref[def:linear-vector-space]{linear space} of bounded Borel complex-valued functions on \([m, M]\), denote this space as \(\mathcal{B} ([m, M])\). We would like to define \(f(T)\) for \(f\in \mathcal{B} ([m, M])\), so that this extends the definition of \(f(T)\) for continuous functions \(f\in C([m, M])\). The difficulty is that Borel functions are only point-wise (but not uniform)limits of continuous functions.

\begin{theorem}[Borel functional calculus]\label{thm:Borel-functional-calculus}
	Let \(\sigma (T) \subseteq [m, M]\) and \(\mathcal{B} ([m, M])\) be the linear space of bounded Borel measurable functions \(f\colon [m, M] \to\mathbb{C}  \) such that
	\[
		\lVert f \rVert _\infty = \sup _{t\in[m, M]} \vert f(t) \vert  < \infty.
	\]
	Then we can define a \hyperref[def:self-adjoint-op]{self-adjoint operator} \(f(T)\in \mathcal{L} (\mathcal{H} )\) with the following properties.
	\begin{enumerate}[(a)]
		\item If \(f(\cdot)\) is real-valued, then \(f(T)\) is \hyperref[def:self-adjoint-op]{self-adjoint}.
		\item If \(f\colon [m, M] \to \mathbb{C} \), \(f\in \mathcal{B} ([m, M])\), then \(\lVert f(T) \rVert \leq \lVert f \rVert _\infty \).
		\item Suppose \(f_n \in \mathcal{B} ([m, M])\) for all \(n\geq 1\) and \(f\in \mathcal{B} ([m, M])\) such that \(\sup _{n\geq 1}\lVert f_n \rVert _\infty < \infty \) and \(f_n \to f\) point-wise, i.e., \(\lim_{n \to \infty} f_n(t) = f(t)\) for all \(t\in [m, M]\). Then \(f_n(T)\) \hyperref[def:strongly-convergence]{converges strongly} to \(f(T)\), i.e.,
		      \[
			      \lim_{n \to \infty} \lVert f_n(T) x - f(T) x \rVert = 0
		      \]
		      for all \(x\in \mathcal{H} \).
		\item Suppose \(T, S \in \mathcal{L} (\mathcal{H} )\) are \hyperref[def:self-adjoint-op]{self-adjoint} and commute, i.e., \(TS = ST\), and assume further that \(\sigma (T), \sigma (S) \subseteq [m, M]\) and \(f, g\in \mathcal{B} ([m, M])\). Then \(f(T)\) and \(g(S)\) commute, i.e., \(f(T) \cdot g(S) = g(S) \cdot f(T)\).
	\end{enumerate}
\end{theorem}