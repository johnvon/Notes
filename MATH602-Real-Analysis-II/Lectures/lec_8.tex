\lecture{8}{20 Sep. 14:30}{Proof of Hahn Banach Theorem and Duality}

We now see the proof of \autoref{thm:Hahn-Banach}.
\begin{proof}[Proof of \autoref{thm:Hahn-Banach}]
	We assume \(E\) is separable, otherwise we need \href{https://en.wikipedia.org/wiki/Transfinite_induction}{transfinite induction}.
	\begin{note}
		Separability allows us to extend \(f_0\) one dimension at a time.
	\end{note}
	Let \(\left\{ x_{n}\colon n \geq 1 \right\} \) have the property that its span is dense in \(E\). Now, if we can extend \(f_0\) such that \(E_0 \to E_0 + \left\{ x_1 \right\} \to E_0 + \left\{ x_1, x_2 \right\} \to \ldots \to E_0 + \mathop{\mathrm{span}}(\left\{ x_n\colon n \geq 1 \right\} ) \), then we can have \(\left\lVert f\right\rVert = \left\lVert f_0\right\rVert \), with the final space is dense in \(E\), we can extend \(f\) to \(E\) by continuity.

	To extend \(f\) by \(1\) dimension, i.e., \(E \to E + \left\{ x_1 \right\} \). Note that extension is determined by a single number \(\gamma = f(x_1)\) since \(f\) is a linear functional. Firstly, we want that \(\left\lVert f\right\rVert = \left\lVert f_0\right\rVert \) such that the linear functional \(f_0 \colon E_0 \to \mathbb{\MakeUppercase{r}} \) extends to \(f\colon D_0 + \left\{ x_1 \right\} \to \mathbb{\MakeUppercase{r}} \), i.e., we want
	\[
		\left\vert f_0 (x_0) + \lambda \gamma \right\vert \leq \left\lVert x_0 + \lambda  x_1\right\rVert
	\]
	for \(x_ 0 \in E\), \(\lambda \in \mathbb{\MakeUppercase{r}} \). By dividing the inequality by \(\lambda \neq 0\), it's sufficient to find \(\gamma \) such that \(\left\vert f_0(x_0) + \gamma  \right\vert \leq \left\lVert x_0 + x_1\right\rVert  \), \(x_0 \in E_0\).

	Suppose \(f_0\) is a real-valued function, we need
	\[
		- \left\lVert x_0 + x_1\right\rVert \leq f_0(x_0) + \gamma \leq \left\lVert x_0 + x_1\right\rVert
	\]
	for all \(x_0 \in E_0\). Such a \(\gamma \) exists, provides \(\left\lVert x_0 + x_1\right\rVert - f_0(x_0) \geq -\left\lVert x_0^\prime e x_1\right\rVert - f_0(x_0^\prime )\) for all \(x_0, x_0^\prime \in E_0\). Furthermore, this is equivalent to write
	\[
		f_0(x_0 - x_0^\prime ) \leq \left\lVert x_0 + x_1\right\rVert + \left\lVert x_0^\prime + x_1\right\rVert
	\]
	for all \(x_0, x_0^\prime \in E_0\), i.e., \(f_0(x_0 - x_0^\prime ) \leq \left\lVert x_0 + x_1\right\rVert + \left\lVert - x_1 - x_0^\prime \right\rVert \) for \(x_0, x_0^\prime \in E_0\). Recall that \(\left\lVert f_0\right\rVert = 1\), we have
	\[
		f_0(x_0 - x_0^\prime ) \leq \left\lVert x_0 - x_0^\prime \right\rVert \leq \left\lVert x_0 + x_1\right\rVert + \left\lVert -x_1 - x_0^\prime \right\rVert.
	\]

	For complex valued \(f\), consider \(f\colon E\to \mathbb{\MakeUppercase{c}} \) be a linear functional over \(\mathbb{\MakeUppercase{c}} \) and let \(g(x) = \Re f(x)\). Then \(g\colon E\to \mathbb{\MakeUppercase{r}} \) is a real-valued linear functional. We see that \(f(x) = g(x) - ig(ix)\) for all \(x\in E\).\footnote{Since \(f(ix) = if(x)\), hence \(g(ix) = -\Im f(x)\).} Conversely, if \(g\colon E\to \mathbb{\MakeUppercase{r}} \) is a real linear functional on Banach space \(E\) over \(\mathbb{\MakeUppercase{c}} \), then \(f\colon E\to \mathbb{\MakeUppercase{c}} \) defined by \(f(x) = g(x) - ig(ix)\), \(x\in E\) is a complex linear functional on \(E\).

	But we need to be a bit careful since when we extend \(f_0 \colon E_0 \to \mathbb{\MakeUppercase{c}} \), we're extending \(2\) real dimensions since for \(g_0 = \Re f_0\), we need to do \(E_0 \to E_0 + \left\{ x_1 \right\} \to E_0 \to \left\{ x_1, x_2 \right\} \). Again, define \(f(\cdot) = g(\cdot) - ig(i\cdot)\), we want to show \(\left\vert f \right\vert = \left\lVert f_0\right\rVert \). We use the fact that for \(x\in E_{0} + \left\{ \lambda x_0\colon \lambda \in \mathbb{\MakeUppercase{c}}  \right\}\),
	\[
		e^{i \theta }f(x) = f(x e^{i \theta })
	\]
	for \(\theta \in \mathbb{\MakeUppercase{r}} \). Choose \(\theta \) such that \(f(x e^{i \theta }) = g(x e^{i \theta })\), and since we already have \(\left\vert g(xe^{i \theta }) \right\vert \leq \left\lVert f_0\right\rVert \left\lVert x e^{i \theta }\right\rVert \), we see that \(\left\vert f(x) \right\vert \leq \left\lVert f_0\right\rVert \left\lVert x\right\rVert \) for \(x\in E_0 + \left\{ \lambda x_1 \colon \lambda \in \mathbb{\MakeUppercase{c}}  \right\} \).
\end{proof}

\chapter{Duality}

Let \(E\) be a Banach space, \(E^{\ast} \) is all bounded linear functionals. Then \(f\in E^{\ast} \), \(\left\lVert f\right\rVert = \sup _{\left\lVert x\right\rVert = 1}\left\vert f(x) \right\vert \). \(E^{\ast} \) is also a Banach space, and we say \(E^{\ast} \) is the dual of \(E\). Now, let \(E^{\ast\ast} \) be the dual of \(E^{\ast} \), turns out that there exists a natural embedding \(E\to E^{\ast\ast}\) such that \(x^{\ast\ast}\in E^{\ast\ast} \) such that
\[
	x^{\ast\ast} (f) = f(x)
\]
for \(f\in E^{\ast} \). Note that
\[
	\left\lVert x^{\ast\ast} \right\rVert = \sup _{\substack{\left\lVert f\right\rVert = 1\\ f\in E^{\ast} }}\left\vert x^{\ast\ast} (f) \right\vert = \sup _{\substack{f\in E^{\ast}\\ \left\lVert f\right\rVert = 1}} \left\vert f(x) \right\vert  \leq \left\lVert x\right\rVert,
\]
implying that \(\left\lVert x^{\ast\ast} \right\rVert \leq \left\lVert x\right\rVert \) for all \(x\in E\). But from \autoref{thm:Hahn-Banach}, \(\left\lVert x^{\ast\ast} \right\rVert = \left\lVert x\right\rVert \), which can be seen from the following. Recall that \(f_0\) such that \(f_0(tx) = t\left\lVert x\right\rVert \) for \(t\in \mathbb{\MakeUppercase{r}} \) or \(\mathbb{\MakeUppercase{c}} \), \(\left\lVert f_0\right\rVert = 1\). To extend \(f_0\) to a functional \(fx \colon E\to \mathbb{\MakeUppercase{c}} \) or \(\mathbb{\MakeUppercase{r}} \) such that \(\left\lVert f_x\right\rVert = 1\) and \(f_x(x) = \left\lVert x\right\rVert \). This implies
\[
	x^{\ast\ast} (f_x) = \left\lVert x\right\rVert \implies \left\lVert x^{\ast\ast} \right\rVert \geq \left\lVert x\right\rVert.
\]

\begin{remark}[Reflexive space]
	The embedding \(E \to E^{\ast\ast} \) is an isometry. If the mapping \(f\) is onto, we say \(E\) is a reflexive space.
\end{remark}

\begin{eg}
	Hilbert spaces.
\end{eg}

\begin{eg}
	\(E = L^p\) spaces for \(1 < p < \infty \).
\end{eg}
\begin{explanation}
	since \(E^{\ast} = L^q\) for \(1 / p + 1 / q = 1\), \(1 < q < \infty \). We then see that \(E^{\ast\ast} = L^p\).
\end{explanation}

\begin{remark}
	An important property of reflexive space \(E\) is the following. If \(E\) is reflexive and \(f\in E^{\ast} \), then \(\exists x_f\in E\) with \(\left\lVert x_f\right\rVert = 1\) and \(\left\lVert f\right\rVert = f(x_f)\), i.e., \(\sup _{\left\lVert x\right\rVert = 1} \left\vert f(x) \right\vert \) is achieved at \(x = x_f\), which follows from \autoref{thm:Hahn-Banach} as follows. Let \(g\in E^{\ast\ast} \), then
	\[
		\left\lVert g\right\rVert = \sup _{\substack{\left\lVert f\right\rVert = 1\\ f\in E^{\ast} }}\left\vert g(f) \right\vert
	\]

	If \(E^{\ast\ast} = E\), then the supremum is achieved since \(g = x^{\ast\ast} \) for some \(x\in E\), so \(x^{\ast\ast} (f) = f(x)\).
\end{remark}

\begin{eg}
	For Banach space \(C([0, 1])\) of continuous function \(g\colon [0, 1]\to \mathbb{\MakeUppercase{c}} \) with supremum norm. Define \(f\colon E\to \mathbb{\MakeUppercase{c}} \) by
	\[
		f(g) = \int _0^1 h(x)g(x)\,\mathrm{d} x
	\]
	where \(h(\cdot)\) is integrable. Then
	\[
		\left\lVert f\right\rVert \leq \left\lVert h\right\rVert _1 = \int _0^1 \left\vert h(x) \right\vert \,\mathrm{d} \times
	\]
	Suppose
	\[
		h(x) \coloneqq \begin{dcases}
			1,  & \text{ if } 0 \leq x < \frac{1}{2} ; \\
			-1, & \text{ if } \frac{1}{2} < x < 1.
		\end{dcases}
	\]
	We see that \(\left\lVert f\right\rVert = 1\). But there does not exist continuous \(g\) such that \(\left\lVert g\right\rVert _\infty = 1\) and \(f(g) = 1\). This implies that the dual space \(E^{\ast} \)  \(C(0, 1)\) is not reflexive since if \(E = C([0, 1]) = E^{\ast\ast} \), then \(E\) is dual of \(E^{\ast} \), but since \(E^{\ast}\) is not reflexive
\end{eg}