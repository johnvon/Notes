\lecture{8}{20 Sep. 14:30}{Proof of Hahn-Banach Theorem and Duality}
Let's first proceed the proof of \hyperref[thm:Hahn-Banach]{Hahn-banach theorem}.

\begin{proof}[Proof of \autoref{thm:Hahn-Banach} (Contd.)]
	To extend \(f\) by \(1\) dimension, i.e., \(E \to E + \mathop{\mathrm{span}}(\left\{ x_1 \right\}) \), first note that extension is determined by a single number \(\gamma = f(x_1)\) since \(f\) is a \hyperref[def:linear-functional]{linear functional}. We want that \(\left\lVert f\right\rVert = \left\lVert f_0\right\rVert \) such that the \hyperref[def:linear-functional]{linear functional} \(f_0 \colon E_0 \to \mathbb{R} \) extends to \(f\colon D_0 + \left\{ x_1 \right\} \to \mathbb{R} \), i.e., we want
	\[
		\left\vert f_0 (x_0) + \lambda \gamma \right\vert \leq \left\lVert x_0 + \lambda  x_1\right\rVert
	\]
	for \(x_ 0 \in E\), \(\lambda \in \mathbb{R} \). By dividing the inequality by \(\lambda \neq 0\), it's sufficient to find \(\gamma \) such that \(\left\vert f_0(x_0) + \gamma  \right\vert \leq \left\lVert x_0 + x_1\right\rVert  \), \(x_0 \in E_0\).

	\begin{itemize}
		\item Suppose \(f_0\) is a real-valued function, we need
		      \[
			      - \left\lVert x_0 + x_1\right\rVert \leq f_0(x_0) + \gamma \leq \left\lVert x_0 + x_1\right\rVert
		      \]
		      for all \(x_0 \in E_0\). Such a \(\gamma \) exists, provides \(\left\lVert x_0 + x_1\right\rVert - f_0(x_0) \geq -\left\lVert x_0^\prime + x_1\right\rVert - f_0(x_0^\prime )\) for all \(x_0, x_0^\prime \in E_0\). Furthermore, this is equivalent to write
		      \[
			      f_0(x_0 - x_0^\prime ) \leq \left\lVert x_0 + x_1\right\rVert + \left\lVert x_0^\prime + x_1\right\rVert
		      \]
		      for all \(x_0, x_0^\prime \in E_0\), i.e., \(f_0(x_0 - x_0^\prime ) \leq \left\lVert x_0 + x_1\right\rVert + \left\lVert - x_1 - x_0^\prime \right\rVert \) for \(x_0, x_0^\prime \in E_0\). Recall that \(\left\lVert f_0\right\rVert = 1\), we have
		      \[
			      f_0(x_0 - x_0^\prime ) \leq \left\lVert x_0 - x_0^\prime \right\rVert \leq \left\lVert x_0 + x_1\right\rVert + \left\lVert -x_1 - x_0^\prime \right\rVert.
		      \]

		\item For complex valued \(f\), consider \(f\colon E\to \mathbb{C} \) be a \hyperref[def:linear-functional]{linear functional} over \(\mathbb{C} \) and let \(g(x) = \Re f(x)\). Then \(g\colon E\to \mathbb{R} \) is a real-valued \hyperref[def:linear-functional]{linear functional}. We see that \(f(x) = g(x) - ig(ix)\) for all \(x\in E\).\footnote{Since \(f(ix) = if(x)\), hence \(g(ix) = -\Im f(x)\).} Conversely, if \(g\colon E\to \mathbb{R} \) is a real \hyperref[def:linear-functional]{linear functional} on \hyperref[def:Banach-space]{Banach space} \(E\) over \(\mathbb{C} \), then \(f\colon E\to \mathbb{C} \) defined by \(f(x) = g(x) - ig(ix)\), \(x\in E\) is a complex \hyperref[def:linear-functional]{linear functional} on \(E\).

		      But we need to be a bit careful since when we extend \(f_0 \colon E_0 \to \mathbb{C} \), we're extending \(2\) real dimensions since for \(g_0 = \Re f_0\), we need to do
		      \[
			      E_0
			      \to E_0 + \mathop{\mathrm{span}}(\left\{ x_1 \right\})
			      \to E_0 \to \mathop{\mathrm{span}}(\left\{ x_1, x_2 \right\}).
		      \]
		      Again, define \(f(\cdot) = g(\cdot) - ig(i\cdot)\), we want to show \(\left\vert f \right\vert = \left\lVert f_0\right\rVert \). We use the fact that for \(x\in E_{0} + \left\{ \lambda x_0\colon \lambda \in \mathbb{C}  \right\}\),
		      \[
			      e^{i \theta }f(x) = f(x e^{i \theta })
		      \]
		      for \(\theta \in \mathbb{R} \). Choose \(\theta \) such that \(f(x e^{i \theta }) = g(x e^{i \theta })\), and since we already have \(\left\vert g(xe^{i \theta }) \right\vert \leq \left\lVert f_0\right\rVert \left\lVert x e^{i \theta }\right\rVert \), we see that \(\left\vert f(x) \right\vert \leq \left\lVert f_0\right\rVert \left\lVert x\right\rVert \) for \(x\in E_0 + \left\{ \lambda x_1 \colon \lambda \in \mathbb{C}  \right\} \).
	\end{itemize}
	The above shows that we can indeed extend one dimension at a time, and the result follows from the fact that the space is \hyperref[def:separable]{separable}.
\end{proof}

\subsection{Supporting Functionals}
\hyperref[thm:Hahn-Banach]{Hahn-Banach theorem} has a variety of analytic and geometric consequences. One of the basic tools guaranteed by \hyperref[thm:Hahn-Banach]{Hahn-Banach theorem} is the existence of a \hyperref[thm:supporting-functional]{supporting functional} \(f\in X^{\ast} \) for every \(x\in X\).

\begin{theorem}[Supporting functional]\label{thm:supporting-functional}
	Let \(E\) be a \hyperref[def:Banach-space]{Banach space}, then for every \(x\in E\), there exists \(f\in E^{\ast} \)  such that \(\left\lVert f\right\rVert = 1\), \(f(x) = \left\lVert x\right\rVert \), i.e., \(\sup _{\left\lVert y\right\rVert = 1} \left\vert f(y) \right\vert \) attained at \(y = x\).
\end{theorem}
\begin{proof}
	Consider dimension \(1\) space \(E_0 = \mathop{\mathrm{span}}(x) = \left\{ tx,t\in \mathbb{R} \text{ or }\mathbb{C}   \right\} \). Define \(f_0\colon E_0 \to \mathbb{R} \) or \(\mathbb{C} \) such that \(f_0(tx) = t \left\lVert x\right\rVert \), then \(\left\lVert f_0\right\rVert = 1\), and \hyperref[thm:Hahn-Banach]{Hahn-Banach theorem} implies there exists \(f\in E^{\ast} \) with \(\left\lVert f\right\rVert = 1\). We see that \(f(x) = \left\lVert x\right\rVert \) explicitly attain the \hyperref[def:norm]{norm} and \(\left\lVert \cdot \right\rVert \) is clearly a continuous extension of \(\left\lVert \cdot \right\rVert _{E_0}= f_0\) as required.
\end{proof}

Notice that we don't have uniqueness (as we don't have it in \hyperref[thm:Hahn-Banach]{Hahn-Banach theorem}) since a unit \hyperref[def:ball]{ball} in \(L^{\infty } \) has corner, which will give multiple \hyperref[def:hyperplane]{hyperplanes}.

\begin{remark}[Geometric interpretation]
	Let \(B\) be a unit \hyperref[def:ball]{ball} \(\left\{ x\in E \colon \left\lVert x\right\rVert \leq 1\right\} \) in a real \hyperref[def:Banach-space]{Banach space} \(E\). Choose \(x_0 \in \partial B\) such that \(\left\lVert x_0\right\rVert = 1\). Then there exists \(f\in E^{\ast} \), \(\left\lVert f\right\rVert = 1\), \(f(x) = \left\lVert x\right\rVert \). Let \(H = \ker(f) + x_0\) where \(H\) intersects \(B\) at \(x_0\), we see that \(H\) divides \(E\) into \(2\) disjoint subsets, while \(B\) lies in one of which.
\end{remark}
\begin{explanation}
	Since \(x\in B\) and \(\left\lVert x\right\rVert < 1\) implies \(\left\vert f(x) \right\vert \leq \left\lVert x\right\rVert < 1\), we have \(f(x) < 1\), i.e., \(B\subseteq \left\{ x\colon f(x) < 1 \right\} \) and \(E = \left\{ x\colon f(x) < 1 \right\} \cup H \cup \left\{ x\colon f(x) > 1 \right\}\).
\end{explanation}

\hyperref[thm:supporting-functional]{supporting-functional theorem} states that for every vector \(x\), we indeed attain its \hyperref[def:norm]{norm} on some \hyperref[def:linear-functional]{functional} \(f\in E^{\ast} \), i.e., their \hyperref[thm:supporting-functional]{supporting functional}. But recall that the \hyperref[def:norm]{norm} of a \hyperref[def:linear-functional]{functional} \(f\in E^{\ast} \) is defined as
\[
	\left\lVert f\right\rVert \coloneqq \sup _{x \neq 0}\frac{\left\vert f(x) \right\vert }{\left\lVert x\right\rVert },
\]
and in general, \(f\) will not attain its \hyperref[def:norm]{norm} on some vector \(x\). This observation leads to the following.

\begin{corollary}
	For every vector \(x\) in a \hyperref[def:normed-vector-space]{normed space} \(E\),
	\[
		\left\lVert x\right\rVert = \max _{f \neq 0}\frac{\left\vert f(x) \right\vert }{\left\lVert f\right\rVert }
	\]
	where the maximum is taken over all non-zero \hyperref[def:linear-functional]{linear functionals} \(f\in E^{\ast} \).
\end{corollary}

\hyperref[thm:Hahn-Banach]{Hahn-Banach theorem} implies that there are enough \hyperref[def:bounded-linear-functional]{bounded linear functionals} \(f\in E^{\ast} \) on every space \(E\). One manifestation of this is the following.
\begin{corollary}[Separation of points]
	For every two vectors \(x_1 \neq x_2\) in a \hyperref[def:normed-vector-space]{normed space} \(E\), there exists a \hyperref[def:linear-functional]{functional} \(f\in E^{\ast} \) such that \(f(x_1) \neq f(x_2)\).
\end{corollary}
\begin{proof}
	The \hyperref[thm:supporting-functional]{supporting functional} \(f\in E^{\ast} \) of the vector \(x = x_1 - x_2\) must satisfy
	\[
		f(x_1 - x_2)= \left\lVert x_1 - x_2\right\rVert \neq 0,
	\]
	as required.
\end{proof}

\subsection{Second Dual Space}
Let \(E\) be a \hyperref[def:normed-vector-space]{normed space}, then the \hyperref[def:linear-functional]{functionals} \(f^{\ast} \) are designed to act on vectors \(x\in E\) via
\[
	f\colon x \mapsto f(x).
\]
But indeed, we can instead say that \emph{vectors \(x\in E\) act on \hyperref[def:linear-functional]{functionals}} \(f\in E^{\ast} \) via
\[
	x\colon f\mapsto f(x).
\]
Thus, a vector \(x\in E\) can itself be considered as a function from \(E^{\ast} \) to \(\mathbb{R} \), i.e., a functional. Furthermore, this function \(x\) is clearly linear, so we may consider \(x\) as a \hyperref[def:linear-functional]{linear functional} on \(E^{\ast} \). Also, the inequality
\[
	\left\vert f(x) \right\vert \leq \left\lVert x\right\rVert \left\lVert f\right\rVert
\]
shows that this \hyperref[def:linear-functional]{functional} is bounded, so \(x\in E^{\ast\ast} \). We may instead write \(x\) as \(x^{\ast\ast} \) for clarity. Note that the \hyperref[def:norm]{norm} of \(x^{\ast\ast} \) as a \hyperref[def:linear-functional]{functional} is \(\left\lVert x^{\ast\ast} \right\rVert _{E^{\ast\ast} } \leq \left\lVert x\right\rVert \) since
\[
	\left\lVert x^{\ast\ast} \right\rVert = \sup _{\substack{\left\lVert f\right\rVert = 1\\ f\in E^{\ast} }}\left\vert x^{\ast\ast} (f) \right\vert = \sup _{\substack{\left\lVert f\right\rVert = 1\\ f\in E^{\ast}}} \left\vert f(x) \right\vert  \leq \left\lVert x\right\rVert,
\]
implying that \(\left\lVert x^{\ast\ast} \right\rVert \leq \left\lVert x\right\rVert \) for all \(x\in E\). But from \hyperref[thm:supporting-functional]{supporting functional} \(f\in E^{\ast} \) of \(x\), we actually have
\[
	\left\lVert x^{\ast\ast} \right\rVert = \left\lVert x\right\rVert,
\]
i.e., we have a \emph{canonical embedding} of \(E\) into \(E^{\ast\ast}\). The above discussion leads to the \hyperref[thm:second-dual-space]{second dual space theorem}.

\begin{theorem}[Second dual space]\label{thm:second-dual-space}
	Let \(E\) be a \hyperref[def:normed-vector-space]{normed space}. Then \(E\) can be considered as a \hyperref[def:linear-vector-space]{linear subspace} of \(E^{\ast\ast}\). For this, a vector \(x\in E\) is considered as a \hyperref[def:bounded-linear-functional]{bounded linear functional} on \(E^{\ast} \) via the action
	\[
		x\colon f\mapsto f(x),\quad f\in E^{\ast}.
	\]
\end{theorem}

To characterize the canonical embedding, we have the following definition.

\begin{definition}[Reflexive space]\label{def:reflexive-space}
	A \hyperref[def:normed-vector-space]{normed space} \(E\) is called \emph{reflexive space} if \(E = E^{\ast\ast} \) under the canonical embedding.
\end{definition}

\begin{eg}
	\(L^p\) spaces for \(1 < p < \infty \) are \hyperref[def:reflexive-space]{reflexive spaces}.
\end{eg}
\begin{explanation}
	We know that \({L^p}^{\ast} = L^q\) where \(1 \leq p < \infty \) for \(q\) being the conjugate index of \(p\).
\end{explanation}
\begin{eg}
	\(L^p\) spaces for \(p = 1\) or \(\infty \) are not \hyperref[def:reflexive-space]{reflexive spaces}
\end{eg}

\begin{proposition}\label{prop:lec8}
	Let \(E\) be a \hyperref[def:reflexive-space]{reflexive space}, then every \hyperref[def:linear-functional]{linear functional} \(f\in E^{\ast} \) attains its \hyperref[def:norm]{norm} on \(E\).
\end{proposition}
\begin{proof}
	By \hyperref[def:reflexive-space]{reflexivity}, the \hyperref[thm:supporting-functional]{supporting functional} of \(f\) is a vector \(x\in E^{\ast\ast} = E\), thus \(\left\lVert x\right\rVert = 1\) and \(f(x) = \left\lVert f\right\rVert\), as required.
\end{proof}

\begin{remark}[\href{https://en.wikipedia.org/wiki/James's_theorem}{James' theorem}]
	The converse of \autoref{prop:lec8} is also true, i.e., if every \hyperref[def:linear-functional]{functional} \(f\in E^{\ast} \) on a \hyperref[def:Banach-space]{Banach space} \(E\) attains its \hyperref[def:norm]{norm}, then \(E\) is \hyperref[def:reflexive-space]{reflexive}.
\end{remark}