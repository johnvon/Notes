\lecture{5}{13 Sep. 14:30}{Abstract Fourier Series}
\subsection{Orthonormal Bases}
It should be easy to identify an extra condition which makes the \hyperref[def:Fourier-series]{Fourier series} of every vector \(x\) converges to \(x\).

\begin{definition}[Complete system]\label{def:complete-system}
	A system of vector \(\left\{ x_k\right\}_k\) in \hyperref[def:Hilbert-space]{Hilbert space} \(\mathcal{\MakeUppercase{h}} \) is \emph{complete} if the space spanned by \(\left\{ x_k\right\}_k\) is dense in \(\mathcal{\MakeUppercase{h}} \), i.e., \(\overline{\mathop{\mathrm{span}}(\left\{ x_k \right\} _k)} = \mathcal{\MakeUppercase{h}} \).
\end{definition}

\begin{definition}[Orthonormal basis]\label{def:orthonormal-basis}
	A \hyperref[def:complete-system]{complete} \hyperref[def:orthonormal-system]{orthonormal system} in a \hyperref[def:Hilbert-space]{Hilbert space} \(\mathcal{\MakeUppercase{h}} \) is called an \emph{orthonormal basis} of \(\mathcal{\MakeUppercase{h}} \).
\end{definition}

\begin{theorem}[Fourier expansions]\label{thm:Fourier-expansion}
	Let \(\left\{ x_k \right\} _k\) be an \hyperref[def:orthonormal-basis]{orthonormal basis} of a \hyperref[def:Hilbert-space]{Hilbert space} \(\mathcal{\MakeUppercase{h}} \). Then every vector \(x\in \mathcal{\MakeUppercase{h}} \) can be expanded in its \hyperref[def:Fourier-series]{Fourier series}
	\[
		x = \sum_{k} \left\langle x, x_k \right\rangle x_k.
	\]
	This is sometimes called \href{https://en.wikipedia.org/wiki/Fourier_inversion_theorem}{Fourier inversion formula}.
\end{theorem}
\begin{proof}
	If an \hyperref[def:orthogonal-system]{orthogonal set} \(\left\{ x_{k} \right\}_k \) is \hyperref[def:complete-system]{complete}, then \(E_\infty = \mathcal{\MakeUppercase{h}} \), \(P_{E_\infty } = I\). This implies \(x = \sum_{k=1}^{\infty} \left\langle x, x_{k}  \right\rangle x_{k}\) for \(x\in \mathcal{\MakeUppercase{h}} \).
\end{proof}

\begin{corollary}[Parseval's identity]\label{col:Parseval}
	Let \(\left\{ x_k \right\} _k\) be an \hyperref[def:orthonormal-basis]{orthonormal basis} of a \hyperref[def:Hilbert-space]{Hilbert space} \(\mathcal{\MakeUppercase{h}} \). Then
	\[
		\left\lVert x\right\rVert ^{2} = \sum_{k=1}^{\infty} \left\vert \left\langle x, x_k \right\rangle  \right\vert ^{2} .
	\]
\end{corollary}
\begin{proof}
	From \hyperref[thm:Fourier-expansion]{Fourier expansion}, we have \(\left\lVert x\right\rVert ^{2} = \left\lVert P_{E_n} x \right\rVert^{2} + \left\lVert (I - P_{E_{n} })\right\rVert ^{2}\). By letting \(n \to \infty \), we have
	\[
		\left\lVert x\right\rVert^{2} = \lim\limits_{n \to \infty} \left\lVert P_{E_n}x\right\rVert ^{2} = \lim\limits_{n \to \infty} \sum_{k=1}^{n} \left\vert \left\langle x, x_{k}  \right\rangle  \right\vert ^{2} = \sum_{k=1}^{\infty} \left\vert \left\langle x, x_{k}  \right\rangle  \right\vert  ^{2}.
	\]
\end{proof}

\subsection{Gram-Schmidt Orthogonalization}

Suppose \(x_1, x_2, \ldots \in \mathcal{\MakeUppercase{h}}\) is a set of vectors and \(E_n = \mathop{\mathrm{span}}(\left\{ x_1, \ldots , x_n \right\} )\). Then we can find an \hyperref[def:orthonormal-system]{orthonormal set} \(\left\{ y_{k} \right\}_{k\geq 1}\) in \(\mathcal{\MakeUppercase{h}}\) such that \(E_n = \mathop{\mathrm{span}}(\left\{ y_1, y_2, \ldots , y_{m(n)}  \right\} )\) where \(m(n) \leq n\).

Firstly, set \(y_1 = x_1 / \left\lVert x_1\right\rVert \), and
\[
	y_n = \frac{(I - P_{E_{n-1}})x_n}{\left\lVert (I - P_{E_{n-1}})x_n \right\rVert}
\]
if \(x_n \notin E_{n-1}\), i.e., \(E_{n-1}\) is properly contained in \(E_n\).

\begin{remark}
	Proving \hyperref[def:complete-system]{completeness} of a set of vectors \(\left\{ x_k \right\}_{k\geq 1} \) in \(\mathcal{\MakeUppercase{h}} \) can be \textbf{non}-trivial.
\end{remark}

We note that we can effectively compute the vectors \(P_{E_n} (x_{n+1}) \) since we know that \(S_n(x)\) is the \hyperref[def:orthogonal-projection]{orthogonal projection} of \(x\) onto \(\mathop{\mathrm{span}}(\left\{ y_k \right\} )\), which is the partial sum of Fourier series
\[
	S_n(x) = \sum_{k=1} ^n \left\langle x, y_k \right\rangle y_k.
\]
As for \(P_n(x)\), we see that it's the \hyperref[def:orthogonal-projection]{orthogonal projection} onto the \hyperref[def:orthogonal-complement]{orthogonal complement}, i.e.,
\[
	P_{E_n}(x) = x - S_n(x) = x - \sum_{k=1} ^n \left\langle x, y_{k}  \right\rangle y_{k} \implies P_{E_n}(x_{n+1} ) = x_{n+1} - \sum\limits_{k=1}^{n} \left\langle x_{n+1, y_{k} }  \right\rangle  y_{k}.
\]

Let's now see some examples.

\begin{eg}[\href{https://en.wikipedia.org/wiki/Haar_wavelet}{Haar basis}]
	We consider the \emph{Haar basis} for \(L^2([0, 1])\). Let \(h\colon (0, 1)\to \mathbb{\MakeUppercase{r}} \) where
	\[
		h(t) = \begin{dcases}
			1,  & \text{ if } 0 < t < 1 / 2; \\
			-1, & \text{ if } 1 / 2 < t < 1.
		\end{dcases}
	\]
	Extend \(h(\cdot)\) by zero outside \((0, 1)\), we get \(h\colon \mathbb{\MakeUppercase{r}} \to \mathbb{\MakeUppercase{r}} \), \(h(t) = 0\) if \(t \notin (0, 1)\), otherwise it's the same as above. The function \(t\mapsto h(2^k t)\) has support in interval \(0 < t < 2^{-k}\). Move the support to interval \(\ell 2^{-k} < t < (\ell +1)2^{-k}\) by translation. Set
	\[
		h_{k, \ell }(t) = h(2^{k}t - \ell ),\quad \ell = 0, 1, \ldots , 2^k - 1.
	\]

	The constant function plus functions \(h_{k, \ell }\), \(k = 0, 1, 2, \ldots  \), \(0 \leq \ell \leq 2^k - 1\) are a \hyperref[def:complete-system]{complete} \hyperref[def:orthogonal-system]{orthogonal set} for \(\mathcal{\MakeUppercase{h}} = L^2([0, 1])\).
\end{eg}
\begin{explanation}
	The span of the Haar functions includes characteristics functions \(\chi _F\) for all dyadic intervals \([2^{-k}\ell , 2^{-k}(\ell +1)]\) for \(\ell = 0, 1, \ldots  , 2^{k-1} \), \(k = 0, 1, \ldots\). If the set is \textbf{not} \hyperref[def:complete-system]{complete}, then there exists \(f\in L^2([0, 1])\) such that
	\[
		\int _F f\,\mathrm{d} t = 0
	\]
	for all dyadic intervals \(F\). Since we can approximate any measurable set \(E\subseteq (0, 1)\) by a union of dyadic intervals.
	\begin{intuition}
		An easy way to see this is to consider
		\[
			\left\{ F\in \mathcal{\MakeUppercase{b}} \colon \int _F f\,\mathrm{d} t = 0 \right\},
		\]
		which is the Borel subalgebra of \(\mathcal{\MakeUppercase{b}} \), which indeed is a Borel algebra on \((0, 1)\). Then observe that dyadic intervals generate all open intervals.
	\end{intuition}

	Hence, we see that \(\int _F f\,\mathrm{d} t= 0\) for all measurable \(F\subseteq (0, 1)\). Let \(F = \left\{ t\in (0, 1)\colon f(t) > 0 \right\} \), if \(m(F) > 0\), then
	\[
		\int _F f\,\mathrm{d} t > 0.
	\]
	Hence, a contradiction, so \(m(F) = 0\).
\end{explanation}

\begin{eg}[Fourier basis]
	Consider the Fourier basis \(e_k (t) = \frac{1}{\sqrt{2\pi }} e^{ikt}\) for \(k\in \mathbb{\MakeUppercase{z}} \), \(-\pi < t < \pi \). This is \hyperref[def:complete-system]{complete} in \(L^2([-\pi , \pi ])\).
\end{eg}
\begin{explanation}
	We use \href{https://en.wikipedia.org/wiki/Stone%E2%80%93Weierstrass_theorem}{Stone-Weierstrass theorem} and apply it to Fourier basis. All \(e_k(\cdot)\) are in \(C[-\pi , \pi ]\), i.e., continuous functions \(f\colon [-\pi , \pi ]\to \mathbb{\MakeUppercase{c}} \). We know that \(C([-\pi , \pi ])\) is a \hyperref[def:Banach-space]{Banach space} with supremum norm \(\left\lVert f\right\rVert \coloneqq \sup _{t\in [-\pi, \pi ]}\left\vert f(t) \right\vert\). \href{https://en.wikipedia.org/wiki/Stone%E2%80%93Weierstrass_theorem}{Stone-Weierstrass theorem} implies density of the space spanned by \(e_k(\cdot)\), \(k\in \mathbb{\MakeUppercase{z}} \) in \(C([-\pi , \pi ])\), hence the completeness in \(L^2([-\pi , \pi ])\) follows from the density of continuous functions in \(L^2([-\pi , \pi ])\). 
\end{explanation}

\begin{proposition}\label{prop:lec5-1}
	Let \(\left\{ x_k \right\} _k\) be a linear independent system in a \hyperref[def:Hilbert-space]{Hilbert space} \(\mathcal{\MakeUppercase{h}} \). Then the system \(\left\{ y_k \right\} _k\) obtained by Gram-Schmidt orthogonalization of \(\left\{ x_k \right\} _k\) is an \hyperref[def:orthonormal-system]{orthonormal system} in \(\mathcal{\MakeUppercase{h}} \), and
	\[
		\mathop{\mathrm{span}}(\left\{ y_k \right\} _{k=1}^n) = \mathop{\mathrm{span}}(\left\{ x_k \right\} _{k=1}^n)
	\]
	for all \(n\in \mathbb{\MakeUppercase{n}} \).
\end{proposition}
\begin{proof}
	The system \(\left\{ y_k \right\} _k\) is \hyperref[def:orthonormal-system]{orthonormal} by construction, and we obviously have the inclusion \(\mathop{\mathrm{span}}(\left\{ y_k \right\} _k) \subseteq \mathop{\mathrm{span}}(\left\{ x_k \right\} _k)\). Furthermore, since the dimensions of these subspaces both equal \(n\) by construction, so they're indeed equal.
\end{proof}

\subsection{Existence of Orthogonal Bases}
We see that from \autoref{prop:lec5-1}, we'll obtain that every \hyperref[def:Hilbert-space]{Hilbert space} that is not \emph{too large} has an \hyperref[def:orthonormal-basis]{orthonormal basis}. We call this \hyperref[def:Hilbert-space]{Hilbert space} \hyperref[def:separable]{separable}.

\begin{definition}[Separable]\label{def:separable}
	A metric space is \emph{separable} if it contains a countable dense subset.
\end{definition}

\begin{remark}[Banach space]
	For \hyperref[def:Banach-space]{Banach space}, \hyperref[def:separable]{separability} follows from finding a countable set of vectors \(\left\{ x_{k}\right\}_k\) such that the span of \(\left\{ x_{k}\right\}_k\) is dense in \(E\).
\end{remark}

\chapter{Bounded Linear Operators}
In this chapter we study certain transformations of \hyperref[def:Banach-space]{Banach spaces}. Because these spaces are linear, the appropriate transformations to study will be \hyperref[def:linear-op]{linear operators}. Furthermore, since \hyperref[def:Banach-space]{Banach spaces} carry topology, it is most appropriate to study continuous transformations, i.e. continuous \hyperref[def:linear-op]{linear operators}. They are also called \hyperref[def:bounded-linear-op]{bounded linear operators} for the reasons that will become clear shortly.

\section{Bounded Linear Functionals}
Turns out that the case when the operators' range is \(\mathbb{\MakeUppercase{r}} \) is interesting enough already, hence we study this case first.
\subsection{Continuity and Boundedness}
\begin{definition*}
	Let \(E\) be a \hyperref[def:linear-vector-space]{linear space} over \(\mathbb{\MakeUppercase{r}} \) or \(\mathbb{\MakeUppercase{c}} \).
	\begin{definition}[Linear functional]\label{def:linear-functional}
		A \emph{linear functional} on \(E\) is a \hyperref[def:linear-op]{linear operator} \(f\colon E\to \mathbb{\MakeUppercase{r}} \) or \(\mathbb{\MakeUppercase{c}} \) such that
		\[
			f(ax + by) = af(x) + bf(y)
		\]
		for \(x, y\in E\), \(a, b\in \mathbb{\MakeUppercase{r}} \) or \(\mathbb{\MakeUppercase{c}} \).
	\end{definition}
	\begin{definition}[Bounded linear functional]\label{def:bounded-linear-functional}
		A \hyperref[def:linear-functional]{linear functional} \(f(\cdot)\) is \emph{bounded} if
		\[
			\left\lVert f\right\rVert \coloneqq \sup _{\left\lVert x\right\rVert = 1}\left\vert f(x) \right\vert  < \infty.
		\]
	\end{definition}
\end{definition*}

Clearly, the boundedness of \(f(\cdot)\) implies \(\left\vert f(x-y) \right\vert \leq \left\lVert f\right\rVert \left\lVert x-y\right\rVert \) for \(x, y\in E\). Hence, \(f(\cdot)\) is continuous and in fact Lipschitz continuous if it's \hyperref[def:bounded-linear-functional]{boounded}.

\begin{remark}
	Conversely, if a \hyperref[def:linear-functional]{linear functional} is continuous then it is bounded.
\end{remark}
\begin{explanation}
	Suppose \(f(\cdot)\) is not bounded, then there exists a sequence \(x_n\in E\) such that \(\left\vert f(x_n) \right\vert \geq n \left\lVert x_n\right\rVert \) for \(n = 1, 2, \ldots  \). By linearity,
	\[
		\left\vert f \left( \frac{x_{n} }{n \left\lVert x_{n} \right\rVert } \right) \right\vert \geq 1, \quad n = 1, 2, \ldots.
	\]
	But we know \(\lim\limits_{n \to \infty} \frac{x_{n} }{n \left\lVert x_{n} \right\rVert }=0\) and \(f(0) = 0\), hence \(f(\cdot)\) is not continuous at \(0\).
\end{explanation}

\subsection{Dual Spaces and Hyperplanes}
Indeed, we have a special name for the space of all \hyperref[def:bounded-linear-functional]{bounded linear functionals} called \hyperref[def:dual-space]{dual spaces} due to its importance.

\begin{definition}[Dual space]\label{def:dual-space}
	Let \(E\) be a \hyperref[def:normed-vector-space]{normed space}, then the space of all \hyperref[def:bounded-linear-functional]{bounded linear functionals} \(f(\cdot)\) on \(E\) is called the \emph{dual space} \(E^{\ast} \) of \(E\).
\end{definition}
The \hyperref[def:dual-space]{dual space} is also a \hyperref[def:normed-vector-space]{normed space} with \hyperref[def:norm]{norm} \(\left\lVert f\right\rVert \coloneqq \sup _{\left\lVert x\right\rVert = 1}\left\vert f(x) \right\vert \), which is in fact a \hyperref[def:Banach-space]{Banach space}. And it is a \hyperref[def:Banach-space]{Banach space} even if the original \(E\) is not. This definition implies \(\left\vert f(x) \right\vert \leq \left\lVert f\right\rVert \left\lVert x\right\rVert \) for \(x\in E\), \(f\in E^{\ast} \). Also, \(\left\lVert f\right\rVert \) is the smallest number in this inequality that makes it valid for all \(x\in X\).

\begin{definition}[Hyperplane]\label{def:hyperplane}
	Let \(E\) be a \hyperref[def:linear-vector-space]{linear space} and \(H\subseteq E\) is a subspace. Say \(H\) is a \emph{hyperplane} if \(\mathop{\mathrm{codim}}(H)=1\), i.e., \(\dim (\quotient{E}{H})=1\).
\end{definition}

The goal is to make an equivalence between \hyperref[def:bounded-linear-functional]{bounded linear functionals} on \(E\) and \emph{closed} \hyperref[def:hyperplane]{hyperplanes} in \(E\).

\begin{problem}
Does there exist a \textbf{non}-closed \hyperref[def:hyperplane]{hyperplane}?
\end{problem}
\begin{answer}
	We know that this is not the case in finite dimension. And this question is analogous to \emph{does there exist a subset \(F\subseteq \mathbb{\MakeUppercase{r}} \) which is \textbf{not} Lebesgue measurable?} The answer to this is yes in both cases. However, construction uses \href{https://en.wikipedia.org/wiki/Axiom_of_choice}{axiom of choice}.
\end{answer}

Turns out that there is a canonical correspondence between the \hyperref[def:linear-functional]{linear functionals} and the \hyperref[def:hyperplane]{hyperplanes} in \(E\). This is clarified in the following.

\begin{proposition}[Linear functionals and hyperplanes]\label{prop:lec5-2}
	Let \(E\) be a \hyperref[def:linear-vector-space]{linear space}.
	\begin{enumerate}[(a)]
		\item For every \hyperref[def:linear-functional]{linear functional} \(f\) on \(E\), \(\ker(f)\) is a \hyperref[def:hyperplane]{hyperplane} in \(E\). If \(E\) is a \hyperref[def:Banach-space]{Banach space}, and \(f(\cdot)\) is bounded, then \(\ker(f) = H\) is closed.
		\item If \(f, g \neq 0\) are \hyperref[def:linear-functional]{linear functionals} on \(E\) such that \(\ker(f) = \ker(g)\), then \(f = ag\) for some \(a \neq 0\).
		\item For every \hyperref[def:hyperplane]{hyperplane} \(H\subseteq E\), there exists a \hyperref[def:linear-functional]{linear functional} \(f \neq 0\) on \(E\) such that \(\ker(f) = H\). If \(E\) is a \hyperref[def:Banach-space]{Banach space} and \(\ker(f) = H\) is closed, then \(f(\cdot)\) is bounded.
	\end{enumerate}
\end{proposition}