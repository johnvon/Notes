\lecture{16}{25 Oct. 14:30}{Strong Convergence \& Weak Topology}
Lastly, we see that \hyperref[def:point-wise-convergence]{point-wise convergence} of operators implies \hyperref[def:uniformly-convergence]{uniformly convergence} on \hyperref[def:compact]{compact sets}.

\begin{definition*}
	Let \(X, Y\) be \hyperref[def:normed-vector-space]{normed spaces}, \(\left\{ T_n \right\}_{n\geq 1}\) be a sequence in \(\mathcal{L} (X, Y)\) and \(T\in \mathcal{L} (X, Y)\).
	\begin{definition}[Point-wise convergence]\label{def:point-wise-convergence}
		The sequence \(\left\{ T_n \right\} _{n \geq 1}\) \emph{converges point-wise to \(T\)} if \(\lim\limits_{n \to \infty} T_n x = Tx\) for all \(x\in X\).
	\end{definition}

	\begin{definition}[Uniformly convergence]\label{def:uniformly-convergence}
		The sequence \(\left\{ T_n \right\} _{n \geq 1}\) \emph{converges uniformly to \(T\) on \(A\subseteq X\)} if \(\lim\limits_{n \to \infty} \left\lVert T_n x - Tx\right\rVert = 0\)  for all \(x\in A\).
	\end{definition}
\end{definition*}

Equivalently, \(\left\{ T_n \right\} _{n\geq }\) \hyperref[def:uniformly-convergence]{converges uniformly} to \(T\) on \(A\) if
\[
	\lim_{n \to \infty} \sup_{x\in A} \left\lVert T_n x - Tx\right\rVert = 0.
\]

\begin{theorem}[Convergence on compact set]\label{thm:convergence-on-compact-set}
	Let \(X, Y\) be \hyperref[def:Banach-space]{Banach spaces}, and \(\left\{ T_n \right\} _{n \geq 1}, T\in \mathcal{L} (X, Y)\). Suppose the sequence \(\left\{ T_n \right\} _{n \geq 1}\) \hyperref[def:point-wise-convergence]{converges point-wise} to \(T\), then \(\left\{ T_n \right\}_{n \geq 1}\) \hyperref[def:uniformly-convergence]{converges uniformly} to \(T\) on all \hyperref[def:precompact]{precompact} subsets \(A \subseteq X\).
\end{theorem}
\begin{proof}
	Since \(\left\{ T_n \right\} \) \hyperref[def:point-wise-convergence]{converges point-wise}, it's \hyperref[def:point-wise-bounded]{point-wise bounded}. From \hyperref[thm:uniform-boundedness]{uniform boundedness theorem}, we know that \(\sup _{n \geq 1} \left\lVert T_n\right\rVert < \infty \), i.e., \(\exists M\) such that \(\left\lVert T_n\right\rVert \leq M\) for all \(n \geq 1\). Let \(\epsilon > 0\) and choose a finite \hyperref[def:eps-net]{\(\epsilon \)-net} \(\Omega _\epsilon \)  for \(A\). Since \(\Omega _{\epsilon }\) is finite, there exists \(N_{\epsilon } \) such that \(\left\lVert T_n y - Ty\right\rVert \leq \epsilon\) for \(n \geq N_{\epsilon } \), \(y\in \Omega _{\epsilon }\), i.e., \(T_n \to T\) \hyperref[def:uniformly-convergence]{uniformly} on \(\Omega _\epsilon \). Now, for an arbitrarily \(x\in A\), there exists \(y\in \Omega _\epsilon \) such that \(\left\lVert x - y\right\rVert < \epsilon \), hence
	\[
		\left\lVert T_n x - Tx\right\rVert \leq \left\lVert T_n y - T y\right\rVert + \left\lVert (T_n - T)(x - y)\right\rVert \leq \epsilon + \left( \left\lVert T_n\right\rVert + \left\lVert T\right\rVert  \right) \left\lVert x - y \right\rVert \leq \epsilon + 2M\epsilon
	\]
	if \(n \geq N_{\epsilon } \). This implies \(\left\lVert T_n x - Tx\right\rVert \leq (2M+1)\epsilon \) for \(n \geq N_{\epsilon } \) for all \(x\in A\), hence we have \hyperref[def:uniformly-convergence]{uniform convergence} on \(A\).
\end{proof}

Finally, we see the criteria of being \hyperref[def:compact]{compact} with \hyperref[def:Schauder-basis]{Schauder basis}.

\begin{corollary}
	Let \(X\) be a \hyperref[def:Banach-space]{Banach space} with \hyperref[def:Schauder-basis]{Schauder basis} \(\left\{ x_k \right\}_{k \geq 1} \). A subset \(A \subseteq X\) is \hyperref[def:precompact]{precompact} if and only if \(A\) is bounded and the basis expansion of vectors \(x\in A\) \hyperref[def:uniformly-convergence]{converges uniformly}, i.e.,
	\[
		\lim_{n \to \infty} \sup _{x\in A} \left\lVert x - S_n x\right\rVert = 0.
	\]
\end{corollary}
\begin{proof}
	From \autoref{thm:convergence-on-compact-set}, since \(\left\{ x_k \right\} _{k\geq 1}\) is a \hyperref[def:Schauder-basis]{Schauder basis}, the projection \(S_n \to I\) \hyperref[def:point-wise-convergence]{point-wise}, implying \hyperref[def:uniformly-convergence]{uniform convergence} since \(A\) is \hyperref[def:precompact]{precompact}.

	Conversely, for any \(\epsilon > 0\), there exists \(n\) such that \(\left\lVert x - S_n x\right\rVert < \epsilon \) for all \(x\in A\), and \(\im (S_n)\)  is finite dimensional and \(\im(S_n A)\) is bounded. Hence, there exists an \hyperref[def:eps-net]{\(\epsilon \)-net} \(\Omega _\epsilon \) for \(\im(S_n A)\), so \(A\) is covered by a finite \hyperref[def:eps-net]{\(2\epsilon \)-net}, i.e., \(A\) is \hyperref[def:precompact]{precompact} from \autoref{lma:appx-by-finite-dim-subspace}.
\end{proof}

Finally, we state without proof one of the most important \hyperref[def:compact]{compactness} criteria in \(C[a, b]\). To start with, we introduce the notion of \hyperref[def:equicontinuous]{equicontinuous}, specifically for real-valued function family.

\begin{definition}[Equicontinuous]\label{def:equicontinuous}
	A real-valued function family \(\mathcal{F} \) is \emph{equicontinuous} if for every \(\epsilon > 0\), there exists \(\delta > 0\) such that
	\[
		\vert f(x) - f(y) \vert \leq \epsilon
	\]
	for all \(f\in \mathcal{F} \) if \(\vert x - y \vert \leq \delta \).
\end{definition}

\begin{remark}[Uniformly Equicontinuous]
	\autoref{def:equicontinuous} is often referred to \emph{uniformly equicontinuous}. There's also a point-wise version of which, but we will not bother introducing it here.
\end{remark}

\begin{theorem}[Arzelà-Ascoli theorem]\label{thm:Arzela-Ascoli}
	A subset \(A \subseteq C[a, b]\) is \hyperref[def:precompact]{precompact} if and only if \(A\) is bounded and \hyperref[def:equicontinuous]{equicontinuous}.
\end{theorem}

\section{Weak Topology}
Every \hyperref[def:normed-vector-space]{normed space} \(X\) is a \hyperref[prev:metric]{metric} space with the \hyperref[prev:metric]{metric} given by \(d(x, y) = \lVert x - y \rVert \) for \(x, y\in X\). This topology on \(X\) is called \emph{strong topology}, i.e., a sequence \(x_n \to x\) \hyperref[def:strongly-convergence]{converges strongly} in \(X\) if \(\lVert x_{n} - x \rVert \to 0\). But actually, in additional to the strong topology, \(X\) carries a different topology called \hyperref[def:weak-topology]{weak topolgy}, as we're going to study it in this section.

\subsection{Weak Convergence}
Let's first formally introduce the definitions.
\begin{definition*}
	Let \(\left\{ x_n \right\} _{n\geq 1}\) be a sequence in a \hyperref[def:normed-vector-space]{normed space} \(X\).
	\begin{definition}[Strongly convergence]\label{def:strongly-convergence}
		The sequence \(\left\{ x_n \right\} _{n\geq 1}\) \emph{converges strongly} to \(x\in X\) if \(\lim\limits_{n \to \infty} \lVert x_n - x \rVert = 0\).
	\end{definition}
	\begin{definition}[Weakly convergence]\label{def:weakly-convergence}
		The sequence \(\left\{ x_n \right\} _{n\geq 1}\) \emph{converges weakly} to \(x\in X\) if \(\lim\limits_{n \to \infty} f(x_n) = f(x)\) for all \(f\in X^{\ast} \).
	\end{definition}
\end{definition*}

\begin{notation}
	If \(\left\{ x_n \right\} _{n \geq 1}\) is \hyperref[def:weakly-convergence]{weakly converging} to \(x\), we write \(x_n \overset{\text{w}}{\to } x\).
\end{notation}

\begin{remark}[Strong and weak]
	As we have seen before (\autoref{def:strongly-bounded}, \autoref{def:weakly-bounded} and \autoref{def:strongly-convergence}, \autoref{def:weakly-convergence}), the convention is that \emph{strong} is for \hyperref[def:norm]{norm}, while \emph{weak} is for \hyperref[def:linear-functional]{functional}.
\end{remark}

We see that \hyperref[def:strongly-convergence]{strongly convergence} implies \hyperref[def:weakly-convergence]{weakly convergence}, while not as before, the converse is often not true. Even with this, there are several useful ties between \hyperref[def:weakly-convergence]{weak} and \hyperref[def:strongly-convergence]{strong} properties.

\begin{proposition}\label{prop:lec16}
	If the sequence \(\left\{ x_n \right\} _{n\geq 1}\) \hyperref[def:weakly-convergence]{converges weakly} to \(x\in X\), then we have the following.
	\begin{enumerate}[(a)]
		\item \(\sup _{n \geq 1} \left\lVert x_n\right\rVert < \infty \).
		\item \(\left\lVert x\right\rVert \leq \liminf_{n \to \infty} \left\lVert x_n\right\rVert \).
		\item \(x\in \overline{\mathop{\mathrm{conv}}(x_k)}\).\footnote{Recall that \(\mathop{\mathrm{conv}}(A)\) is the \href{https://en.wikipedia.org/wiki/Convex_hull}{convex hull} of \(A\), i.e., the smallest closed \hyperref[def:convex-set]{convex set} containing the sequence.}
	\end{enumerate}
\end{proposition}
\begin{proof}
	Let's prove this one by one.
	\begin{enumerate}[(a)]
		\item For \(y\in X\), let \(y^{\ast\ast}\in X^{\ast\ast}\) be from the embedding \(X\to X^{\ast\ast}\), \(y \mapsto y^{\ast\ast}\) such that \(\left\lVert y^{\ast\ast}\right\rVert = \left\lVert y\right\rVert \). Then for \(n \geq 1\), \(x_n \in X\), so \(x_n^{\ast\ast}\in X^{\ast\ast}\), we have \(\sup _{n\geq 1} \left\vert x_n^{\ast\ast}(f) \right\vert < \infty\) since \(f(x_n)\to f(x)\). Then, \hyperref[thm:uniform-boundedness]{uniform boundedness theorem} implies \(\sup _{n\geq 1} \left\lVert x_n^{\ast\ast}\right\rVert < \infty \). Now, since \(\left\lVert x_n\right\rVert = \left\lVert x_n^{\ast\ast}\right\rVert \), we conclude \(\sup _{n\geq 1}\left\lVert x_n\right\rVert < \infty\).
		\item If \(x_n \overset{\text{w}}{\to } x\) in \(X\), by \hyperref[thm:supporting-functional]{supporting functional theorem}, there exists \(f\in X^{\ast} \) with \(\left\lVert f\right\rVert = 1\) and \(f(x) = \left\lVert x\right\rVert\). Since \(\left\lVert f\right\rVert = 1\), \(f(x_n) \leq \left\lVert x_n\right\rVert\) for \(n \geq 1\). And since \(x_n \overset{\text{w}}{\to } x\), we have \(f(x_n)\to f(x)= \left\lVert x\right\rVert \), i.e., \(\liminf_{n \to \infty} \left\lVert x_n\right\rVert \geq \left\lVert x\right\rVert\) as desired.
		\item To show \(x\) lies in the closure of the \href{https://en.wikipedia.org/wiki/Convex_hull}{convex hull} of \(\left\{ x_n \right\} _{n\geq 1}\), denoted it by \(K\), we first note that \(K\) is a closed \hyperref[def:convex-set]{convex set}. If \(x \notin K\), by \hyperref[thm:separation-of-convex-sets]{separating hyperplane theorem}, there exists \(f\in X^{\ast} \) such that \(\sup _{y\in K}f(y) < f(x)\), and hence \(\sup _{n\geq 1}f(x_n) < f(x)\). Since \(\lim_{n \to \infty} f(x_n)=f(x)\), we have a contradiction.
	\end{enumerate}
\end{proof}

There are some known criteria of \hyperref[def:weakly-convergence]{weak convergence} in classical \hyperref[def:normed-vector-space]{normed spaces}, one of them is as follows.
\begin{lemma}[Testing weak convergence on a dense set]\label{lma:testing-weak-convergence-on-a-dense-set}
	Let \(X\) be a \hyperref[def:normed-vector-space]{normed space} and \(A \subseteq X^{\ast} \) be a dense set. Then \(x_n \overset{\text{w}}{\to } x\) if and only if \(\left\{ x_k \right\} _{k\geq 1}\) is bounded and \(\lim\limits_{n \to \infty} f(x_n) = f(x)\) for every \(f\in A\).
\end{lemma}
\begin{proof}
	The necessity follows from \autoref{prop:lec16}. To show the sufficiency, let \(g\in X^{\ast} \), we need to show that \(\lim\limits_{n \to \infty} g(x_n) = g(x)\). Let \(\epsilon > 0\), and \(A\) is dense in \(X^{\ast} \), so there exists \(f\in A\) such that \(\left\lVert g-f\right\rVert < \epsilon \). Then
	\[
		\begin{split}
			\limsup_{n \to \infty} \left\vert g(x_n - x) \right\vert
			&\leq \limsup_{n \to \infty} \left\vert f(x_n - x) \right\vert + \limsup_{n \to \infty} \left\vert (g-f)(x_{n} -x ) \right\vert \\
			&\leq \left\lVert g-f\right\rVert \limsup_{n \to \infty} (\lVert x_n  \rVert + \lVert x \rVert) \\
			&\leq 2M\epsilon
		\end{split}
	\]
	where \(M\coloneqq \sup _{n\geq 1} \lVert x_n\rVert + \lVert x \rVert < \infty\). We see that since \(\epsilon > 0\) is arbitrary, so \(\limsup\limits_{n \to \infty} \left\vert g(x_{n} - x ) \right\vert = 0\) hence \(x_n \overset{\text{w}}{\to} x\).
\end{proof}

\begin{note}
	We see that to show \hyperref[def:weakly-convergence]{weakly convergence}, instead of checking for all \(f\in X^{\ast} \), it's sufficient to check only \(f\in A\) for some dense set \(A\subseteq X^{\ast} \).
\end{note}

\subsection{Weak Topology}
Indeed, \hyperref[def:weakly-convergence]{weak convergence} is just an induced concept from \hyperref[def:weak-topology]{weak topology}, so we now study it directly. And in fact, this allows us to further consider other weak properties as we'll soon see.

\begin{definition}[Weak topology]\label{def:weak-topology}
	The \emph{weak topology} on a \hyperref[def:normed-vector-space]{normed space} \(X\) is the weakest topology such that all maps \(f\in X^{\ast} \) are continuous.
\end{definition}

\begin{note}[Strong topology]\label{note:strong-topology}
	To distinguish two natural topologies on \(X\), the \hyperref[def:norm]{norm} topology is sometimes called \emph{strong topology} on \(X\).
\end{note}

Intuitively, if \(f\) is continuous at \(x_0\), the preimage of an \(\epsilon \)-\hyperref[def:ball]{ball} \(\left\{ x\in X\colon \left\vert f(x) - f(x_0) \right\vert < \epsilon  \right\}\) around \(f(x_0)\) is open, i.e., the base of the \hyperref[def:weak-topology]{weak topology} are \href{https://en.wikipedia.org/wiki/Cylinder_set}{cylinders} of the form
\[
	\left\{ x\in X\colon \left\vert f_k(x - x_0) \right\vert < \epsilon , k = 1, 2, \ldots , N \right\}
\]
where \(x_0\in X\), \(f_k\in X^{\ast} \), \(k = 1, \ldots  , N\) for \(\epsilon > 0\), \(N \geq 1\). In all, these cylinders form a local base of \hyperref[def:weak-topology]{weak topology} at point \(x_0\).

\begin{remark}[\(\mathbb{R} ^{\infty} \) embedding]
	Consider the embedding from \(X\) to an infinite product of \(\mathbb{R} \) such that
	\[
		X\to \mathbb{R} ^{\infty} ,
		\quad x\mapsto (f(x))_{f\in X^{\ast}} = (f_1(x), f_2(x), \ldots )
	\]
	given \(\left\{ f_i \right\} _{i \geq 1}\) in \(X^{\ast}\):\footnote{There may be uncountable many of \(f_i\).} \hyperref[def:weak-topology]{weak topology} is induced from the products of reals.
\end{remark}

\begin{note}[Realtion between CW complex]
	The weak topology in \href{https://www.pbb.wtf/posts/Notes#algebraic-topology-math592-umich}{algebraic topology} is actually the strongest one, i.e., it's the \href{https://en.wikipedia.org/wiki/Final_topology}{final topology}. See \href{https://math.stackexchange.com/questions/921744/any-relations-between-the-weak-topology-on-a-banach-space-and-the-weak-topology}{here} for a further discussion.
\end{note}

Although there are lots of difference between \hyperref[def:weak-topology]{weak} and the corresponding \hyperref[note:strong-topology]{strong} properties, some are equivalent.

\begin{proposition}[Weak closedness]\label{prop:weak-closedness}
	Let \(K\) be a \hyperref[def:convex-set]{convex subset} of a \hyperref[def:Banach-space]{Banach space}. Then \(K\) is \hyperref[def:weak-topology]{weakly} closed if and only if \(K\) is \hyperref[note:strong-topology]{strongly} closed.
\end{proposition}
\begin{proof}
	Clearly, \hyperref[def:weak-topology]{weak} closure implies \hyperref[note:strong-topology]{strong} closure.\footnote{Notice that this case doesn't involve \hyperref[def:convex-set]{convexity} since we only use the coarser relation between topologies.} Conversely, if \(K\) is a \hyperref[note:strong-topology]{strongly} closed \hyperref[def:convex-set]{convex set}, it is the intersection of all \hyperref[def:hyperplane]{hyperplane} containing \(K\) from \autoref{col:lec10}, i.e., \(K = \bigcap_{f, a} A_{f, a}\) where \(A_{f, a}\coloneqq \left\{ x\in X \colon f(x) \leq a, f\in X^{\ast} \right\} \). Since it is \hyperref[def:weak-topology]{weakly} closed from \(A_{f, a}= f^{-1} ((-\infty , a])\),\footnote{Since \(f\) is continuous, \(f^{-1} \) maps closed set in \(\mathbb{R}\) (\((-\infty , a]\)) to (\hyperref[def:weak-topology]{weakly}) closed set in \(X\) with \hyperref[def:weak-topology]{weak topology}.} the intersection of them is also \hyperref[def:weak-topology]{weakly} closed, proving the result.
\end{proof}