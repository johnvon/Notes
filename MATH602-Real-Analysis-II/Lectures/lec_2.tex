\lecture{2}{01 Sep. 14:30}{Banach Spaces and Completion}
Let's first see a proposition.

\begin{proposition}
	Let \(\left\{ E, \left\lVert \cdot\right\rVert  \right\} \) be a \hyperref[def:normed-vector-space]{normed linear space}. Then the norm is \hyperref[def:convex-function]{convex} and continuous.
\end{proposition}
\begin{proof}
	Let \(f\colon E\to \mathbb{\MakeUppercase{r}} \) be \(f(x) = \left\lVert x\right\rVert \). Then \(f(x) - f(y) = \left\lVert x\right\rVert - \left\lVert y\right\rVert  \leq \left\lVert x-y\right\rVert\), which implies \(\left\vert f(x) - f(y) \right\vert \leq \left\lVert x-y\right\rVert \) for \(x, y\in E\), i.e., \(f\) is Lipschitz continuous. For \hyperref[def:convex-function]{convexity}, let \(0 < \lambda < 1\), we have
	\[
		f(\lambda x+(1-\lambda )y) = \left\lVert \lambda x + (1-\lambda )y\right\rVert \leq \left\lVert \lambda x\right\rVert + \left\lVert (1-\lambda )y\right\rVert = \lambda \left\lVert x\right\rVert + (1-\lambda )\left\lVert y\right\rVert  = \lambda f(x) + (1-\lambda )f(y).
	\]
\end{proof}

\begin{note}
	Note that \(f(\cdot)\) is continuous implies the closed \hyperref[def:ball]{ball}
	\[
		B(x_0, r) = \left\{ x\in E\mid \left\lVert x-x_0\right\rVert\leq r\right\} = \left\{ x\in E\mid f(x-x_0)\leq r \right\}
	\]
	is closed in topology of \(E\). Also, \(f(\cdot)\) is \hyperref[def:convex-function]{convex} implies \(B(x_0, r)\) is \hyperref[def:convex-set]{convex}.
\end{note}

\begin{remark}
	If \(f\colon E\to \mathbb{\MakeUppercase{r}} \) is \hyperref[def:convex-function]{convex}, then the sets \(\left\{ x\in E\mid f(x) \leq M \right\}\) is also \hyperref[def:convex-set]{convex}. However, it's possible to have non-\hyperref[def:convex-function]{convex functions} \(f\) such that all sets \(\left\{ x\in E \mid f(x) \leq M \right\} \) are \hyperref[def:convex-set]{convex}.
\end{remark}

\begin{eg}
	Take \(f(x) = \left\vert x \right\vert ^p\) for \(x\in \mathbb{\MakeUppercase{r}} \) and \(p > 0\). We see that \(f\) is \hyperref[def:convex-function]{convex} if \(p>1\), and non-\hyperref[def:convex-function]{convex} if \(p<1\). The sets \(\left\{ x\in \mathbb{\MakeUppercase{R}} \mid f(x) \leq M \right\} \) all \hyperref[def:convex-set]{convex} since it's independent of \(p\).
\end{eg}

\begin{lemma}
	Suppose \(x\mapsto \left\lVert x\right\rVert \) satisfies
	\begin{enumerate}[(a)]
		\item \(\left\lVert x\right\rVert \geq 0\) and \(\left\lVert x\right\rVert =0 \iff x=0\).
		\item \(\left\lVert \lambda x\right\rVert = \left\vert \lambda  \right\vert \left\lVert x\right\rVert\), \(\lambda \in\mathbb{\MakeUppercase{r}} \) or \(\mathbb{\MakeUppercase{c}} \).
		\item The unit \hyperref[def:ball]{ball} \(B(0, 1)\) is \hyperref[def:convex-set]{convex}.
	\end{enumerate}
	Then \(f(x) = \left\lVert x\right\rVert \) satisfies the triangle inequality \(\left\lVert x + y\right\rVert \leq \left\lVert x\right\rVert + \left\lVert y\right\rVert \).
\end{lemma}
\begin{proof}
	We see that if the third condition is true, the for \(u, v\in B(0, 1)\) and \(0<\lambda <1\), we have \(\lambda u + (1 - \lambda )v \in B(0, 1)\). Let \(x, y\in E\), and
	\[
		\lambda = \frac{\left\lVert x\right\rVert }{\left\lVert x\right\rVert + \left\lVert y\right\rVert }\implies 1 - \lambda = \frac{\left\lVert y\right\rVert }{\left\lVert x\right\rVert + \left\lVert y\right\rVert }.
	\]
	By letting \(u = x / \left\lVert x\right\rVert \), \(v = y / \left\lVert y\right\rVert \) we see that
	\[
		\lambda u + (1 - \lambda )v = \frac{\left\lVert x\right\rVert }{\left\lVert x\right\rVert + \left\lVert y\right\rVert } \frac{x}{\left\lVert x\right\rVert } + \frac{\left\lVert y\right\rVert }{\left\lVert x\right\rVert + \left\lVert y\right\rVert }\frac{y}{\left\lVert y\right\rVert }\in B(0, 1) \implies \left\lVert \frac{x}{\left\lVert x\right\rVert + \left\lVert y\right\rVert } + \frac{y}{\left\lVert x\right\rVert + \left\lVert y\right\rVert }\right\rVert \leq 1.
	\]
	From the second condition, it follows that \(\left\lVert x + y\right\rVert \leq \left\lVert x\right\rVert + \left\lVert y\right\rVert \), which is the triangle inequality.
\end{proof}

\begin{remark}
	If \(x\mapsto \left\lVert x\right\rVert \) satisfies the first two condition and is a \hyperref[def:convex-function]{convex}, then it satisfies the triangle inequality.
\end{remark}
\begin{explanation}
	Since \(\frac{1}{2}\left\lVert x + y\right\rVert = \left\lVert \frac{x}{2} + \frac{y}{2}\right\rVert \leq \frac{1}{2}\left\lVert x\right\rVert + \frac{1}{2}\left\lVert y\right\rVert\).
\end{explanation}

Now, given a \hyperref[def:quotient-space]{quotient space} \(\quotient{E}{E_1} \), the question is can we try to define a \hyperref[def:norm]{norm}?

\begin{problem}
On \(\quotient{E}{E_1} \), is \(\left\lVert [x]\right\rVert \coloneqq \inf _{y\in E_1} \left\lVert x+y\right\rVert \) a \hyperref[def:norm]{norm}?
\end{problem}
\begin{answer}
	We see that if \(x\in \overline{E}_1 \setminus E_1 \), then \(\left\lVert [x]\right\rVert = 0\) but \([x] \neq 0\in \quotient{E}{E_1} \).
\end{answer}

\begin{note}
	Notice the difference from finite dimensional situation. All finite dimensional spaces \(E_1\) are closed but not in general if \(E_1\) has \(\infty \) dimensions.
\end{note}
\begin{eg}
	Let \(\ell _1(\mathbb{\MakeUppercase{r}} )\) be the sequence of \(x_n\) for \(n \geq 1\) in \(\mathbb{\MakeUppercase{r}} \) such that \(\sum_{i=1}^{\infty} \left\vert x_i \right\vert \leq \infty\). Define
	\[
		\left\lVert x\right\rVert _1 \coloneqq \sum_{i=1}^{\infty} \left\vert x_i \right\vert ,
	\]
	and let \(E_1\) be all sequences with finite number of the \(x_n\) are nonzero. We see that \(\overline{E}_1= \ell _1(\mathbb{\MakeUppercase{r}} ) \) is infinite dimensional.
\end{eg}

\begin{proposition}
	Let \(\left\{ E, \left\lVert \cdot\right\rVert  \right\} \) be a \hyperref[def:normed-vector-space]{normed space} and \(E_1\subseteq E\), \(E_1\) is closed. Then
	\[
		\left\lVert \cdot\right\rVert \colon \quotient{E}{E_1} \to \mathbb{\MakeUppercase{r}},\quad \left\lVert [x]\right\rVert = \inf _{y\in E_1} \left\lVert x+y\right\rVert
	\]
	is a \hyperref[def:norm]{norm} on \(\quotient{E}{E_1} \).
\end{proposition}
\begin{proof}
	If \(\left\lVert [x]\right\rVert = 0\), then \(\inf _{y\in E_1}\left\lVert x-y\right\rVert = 0\), which implies \(x\in E_1\) since \(E_1\) is closed, so \([x] = 0\). Also, since
	\[
		\left\lVert \lambda [x]\right\rVert = \inf _{y\in E_1}\left\lVert \lambda x + y\right\rVert = \inf _{z\in E_1}\left\lVert \lambda x + \lambda z\right\rVert = \left\vert \lambda  \right\vert \inf _{z\in E_1} \left\lVert x+z\right\rVert = \left\vert \lambda  \right\vert \left\lVert [x]\right\rVert,
	\]
	the dilation property is satisfied. Finally, for triangle inequality, we have
	\[
		\left\lVert [x] + [y]\right\rVert = \inf _{x_1, y_1 \in E} \left\lVert x+y + x_1 + y_1\right\rVert \leq \inf _{x_1\in E_1}\left\lVert x + x_1\right\rVert + \inf _{y_1\in E_1}\left\lVert y + y_1\right\rVert = \left\lVert [x]\right\rVert + \left\lVert [y]\right\rVert.
	\]
\end{proof}
\begin{remark}
	This shows that the only obstacle for this kind of \hyperref[def:norm]{norm} being an actual \hyperref[def:norm]{norm} is the closeness of \(E_1\).
\end{remark}

\chapter{Banach Spaces}
\begin{definition}[Banach space]\label{def:Banach-space}
	A \hyperref[def:normed-vector-space]{linear normed space} is a \emph{Banach space} if it's complete, i.e., every Cauchy sequence converges.
\end{definition}

\begin{note}
	If \(x_n\in E\), \(n\geq 1\) is a sequence with property such that \(\lim\limits_{m \to \infty} \sup\limits_{n\geq m} \left\lVert x_n - x_m\right\rVert > 0 \), then \(\exists x_{\infty }\in E\) such that \(\lim\limits_{n \to \infty} \left\lVert x_n - x_m\right\rVert = 0\).
\end{note}

\begin{eg}
	The spaces \(\ell _1\), \(\ell _{\infty }\) and \(C(K)\) are \hyperref[def:Banach-space]{Banach spaces}.
\end{eg}

We want to give a different criterion for showing \(\left\{ E, \left\lVert \cdot\right\rVert  \right\} \) is \hyperref[def:Banach-space]{Banach}. Let \(E\) be a \hyperref[def:normed-vector-space]{linear normed space} and \(\left\{ x_{\ell } \mid \ell \geq 1 \right\} \) a sequence in \(E\).

\begin{definition}[Absolutely summable]\label{def:absolutely-summable}
	A sequence is \emph{absolutely summable} if \(\sum_{i=1}^{\infty} \left\lVert x_i\right\rVert < \infty \).
\end{definition}

\begin{theorem}[Criterion for completeness]\label{thm:criterion-for-completeness}
	A \hyperref[def:normed-vector-space]{normed space} \(\left\{ E, \left\lVert \cdot\right\rVert  \right\} \) is a \hyperref[def:Banach-space]{Banach space} if and only if every series in \(E\) converges.
\end{theorem}
\begin{proof}
	We need to prove two directions.

	\paragraph{\((\implies )\)}
	Suppose \(E\) is a \hyperref[def:Banach-space]{Banach space} and \(\left\{ x_{k}\mid x \geq 1 \right\} \) an \hyperref[def:absolutely-summable]{absolutely summable} series. Set \(s_n = \sum_{k=1}^{n} x_{k} \), \(n \geq 1\), we want to show \(s_n\) is Cauchy, and if this is the case, completeness of \(E\) implies \(\exists s_{\infty }\) and \(\lim\limits_{n \to \infty} \left\lVert s_n - s_{\infty }\right\rVert = 0\). Let \(n > m\), we see that
	\[
		\left\lVert s_n - s_m\right\rVert = \left\lVert \sum_{k=m+1}^{n} x_k\right\rVert \leq \sum_{k=m+1}^{n} \left\lVert x_k\right\rVert \leq \sum_{k=m+1}^{\infty} \left\lVert x_k\right\rVert.
	\]
	Observe that \(\lim\limits_{m \to \infty} \sum\limits_{k=m+1}^{\infty} \left\lVert x_k\right\rVert = 0\), we see that the sequence \(\left\{ s_n \right\} \) is Cauchy.

	\paragraph{\((\impliedby)\)}
	Conversely, suppose \(E\) is \textbf{not} complete. Then there exists a Cauchy sequence \(\left\{ x_n \mid n \geq 1 \right\} \) which does not converge. Furthermore, no subsequence of \(\left\{ x_n \mid n \geq 1 \right\} \) converges.\footnote{Otherwise, the whole sequence converges by the fact that it's Cauchy.} We now construct an \hyperref[def:absolutely-summable]{absolutely summable} series which does not converge.

	Define \(n(1) \geq 1\) such that \(\left\lVert x_n - x_{n(1)}\right\rVert \leq \frac{1}{2}\) if \(n \geq n(1)\), similarly, let \(n(2) > n(1)\) be such that \(\left\lVert x_n - x_{n(2)}\right\rVert \leq \frac{1}{2^2}\) if \( n > n(2)\). In all, we have \(n(1) < n(2) < n(3) < \ldots  \) such that \(\left\lVert x_n - x_{n(k)}\right\rVert \leq \frac{1}{2^k}\) if \(n > n(k)\). Define \(w_j \coloneqq x_{n(j+1)} - x_{n(j)}\) for \(j = 1, 2, \ldots  \). We see that
	\[
		x_{n(m)} = x_{n(1)} + \sum_{j=1}^{m} w_j
	\]
	for \(m = 1, 2, \ldots\), and \(\left\{ x_{n(m)} \right\} \) does not converge, hence so does the series \(\sum_{j=1}^{\infty} w_j\). However, \(\sum_{j=1}^{\infty} \left\lVert w_j\right\rVert \leq \sum_{j=1}^{\infty} \frac{1}{2^j} = 1\), which implies \(\left\{ w_j \right\} \) is \hyperref[def:absolutely-summable]{absolutely summable}.
\end{proof}

\section{Completion of Normed Space to Banach Space}

\begin{theorem}
	Suppose \(E\) is a \hyperref[def:normed-vector-space]{normed space}. Then there exists a \hyperref[def:Banach-space]{Banach space} \(\hat{E} \) called a completion of \(E\) with the following properties:
	\begin{enumerate}[(a)]
		\item There exists a linear map \(i\colon E \to \hat{E}\) such that \(\left\lVert ix\right\rVert = \left\lVert x\right\rVert \).\footnote{This is called an \emph{isometric embedding} of \(E\) into \(\hat{E} \).}
		\item \(\im(i)\) is dense in \(\hat{E} \), and \(\hat{E} \) is the smallest \hyperref[def:Banach-space]{Banach space} containing image of \(E\).
	\end{enumerate}
\end{theorem}