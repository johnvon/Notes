\lecture{22}{15 Nov. 14:30}{Self-Adjoint Operators on Hilbert Spaces}
\begin{proof}[Proof of \autoref{thm:lec21} (Continued)]
	We now continue to prove \autoref{thm:lec21}.
	\begin{enumerate}
		\item[(b)] Let \(S\in \mathcal{\MakeUppercase{l}} (X, X)\), then we already observed that \((\im S)^{\perp} = \ker(S^{\ast} )\). Assume that \(\overline{\im S} \neq X\), by \hyperref[thm:separation-of-convex-sets]{separation theorem}, there exists \(f_0\in X^{\ast} \), \(f_0 \neq 0\) and \(f_0 \in (\im S)^{\perp} = \ker(S^{\ast} )\).
		\item[(c)] We need to show that if \(S\in \mathcal{\MakeUppercase{l}} (X, X)\) and \(\ker(X) \neq \left\{ 0 \right\} \), then \(\im S\) is not dense in \(X\). Assume \(Sx = 0\) and \(x \neq 0\), since \(\ker S \neq \left\{ 0 \right\} \), then
			\[
				S^{\ast} f(x) = f(Sx) = 0
			\]
			for all \(f\in X^{\ast} \). Hence, if \(g\in \im S^{\ast} \), then \(g(x) = 0\). If \(\im S^{\ast} \) is dense in \(X^{\ast} \), conclude that \(g(x) = 0\) for all \(g\), implying that \(x = 0\), contradiction.
	\end{enumerate}
\end{proof}

To finish this section, we see one final example.

\begin{eg}[Shift operator]
	Let \(T\) be the left shift operator on \(\ell _1\), i.e.,
	\[
		T(a_1, a_2, \ldots  )= (a_2, a_3, \ldots  ).
	\]
	Notice that \(\ell _1^{\ast} = \ell _\infty \),\footnote{While \(\ell ^{\ast} _\infty \neq \ell _1\).} so \(T^{\ast} \) is the right shift operator on \(\ell _\infty \), i.e.,
	\[
		T^{\ast} (a_1, a_2, \ldots  ) = (0, a_1, a_2, \ldots  ).
	\]
	Since \(\lVert T \rVert = \lVert T^{\ast}  \rVert = 1\), hence \(\lambda \in \rho (T)\) if \(\vert \lambda  \vert > 1\). Suppose \(\vert \lambda  \vert < 1\), then the vector \(x_\lambda = (1, \lambda , \lambda ^{2} , \ldots  )\) is in \(\ell _1\) such that \(Tx_\lambda = \lambda x_\lambda \), so \(\lambda \in \sigma _p(T)\). With \(\sigma (T)\) is closed, we have
	\[
		\sigma (T) = \left\{ \lambda \in \mathbb{\MakeUppercase{c}} \colon \vert \lambda  \vert \leq 1 \right\} .
	\]
	Next, we want to show that \(T^{\ast} \) has no \hyperref[def:point-spectrum]{point spectrum}. Suppose \(a=(a_1, a_2, \ldots  \in \ell _\infty )\) and \((T\pm \lambda I)a=0\). Then
	\[
		\lambda a_1 = 0,\quad
		\lambda a_2 - a_1 = 0,\quad
		\lambda a_3 - a_2 = 0,\quad \ldots
	\]
	We then see that if \(\lambda \neq 0\), then \(a\equiv 0\); if \(\lambda = 0\), then \(a\equiv 0\) as well, hence \(\sigma _p(T^{\ast} )\) is empty. By \autoref{thm:lec21}, \(\lambda \in \sigma _p(T) \implies \lambda \notin \sigma _c(T^{\ast} )\). Now, consider
	\begin{itemize}
		\item \(\vert \lambda  \vert < 1\): we further have \(\lambda \in \sigma _p(T)\) and \(\lambda \notin \sigma _p(T^{\ast} )\), hence \(\lambda \in \sigma _r(T^{\ast} )\).
		\item \(\vert \lambda  \vert = 1\): in this case, we want to show \(\lambda \in \sigma (T) = \sigma (T^{\ast} )\). It's clear that \(\lambda \notin \sigma _p(T)\). On the other hand, if \(\lambda \in \sigma _r(T)\), then \(\lambda \in \sigma _p(T^{\ast} )\) by \autoref{thm:lec21}, with the fact that \(\sigma _p(T^{\ast} ) = \varnothing \), so \(\sigma _r(T) = \varnothing \). So we conclude that if \(\vert \lambda \vert = 1\), \(\lambda \in \sigma _c(T)\).

		      Finally, we show that if \(\vert \lambda  \vert = 1\), then \(\lambda \in \sigma _r(T^{\ast} )\). To do this, we shall find an open \hyperref[def:ball]{ball} disjoint from \(\im(T^{\ast} - \lambda I)\). Suppose \(a = \left\{ a_n \right\} _{n\geq 1}\) and \(b = \left\{ b_n \right\} _{n\geq 1}\) are in \(\ell _\infty \) with \(a = (\lambda I-T^{\ast} )b\), hence
		      \[
			      a_1 = \lambda b_1,\quad
			      a_2 = \lambda b_2 - b_1,\quad
			      a_3 = \lambda b_3 - b_3, \quad \ldots
		      \]
		      This is equivalent to write
		      \[
			      \begin{split}
				      b_1 &= \frac{a_1}{\lambda } = \overline{\lambda} a_1, \\
				      b_2 &= \frac{a_1}{\lambda } + \frac{b_1}{\lambda } = \frac{a_2}{\lambda } + \frac{a_1}{\lambda ^{2} } = \overline{\lambda} ^{2} (a_1 + \lambda a_2), \\
				      &\vdots\\
				      b_n &= \overline{\lambda} ^{n+1}\sum_{m=1}^{n} \lambda ^m a_m.
			      \end{split}
		      \]
		      Define \(c = \left\{ c_n \right\} _{n\geq 1}\) such that \(c_n = \overline{\lambda} ^n\). Suppose \(d\in \ell _\infty \), \(\lVert d-c \rVert _\infty < 1/2\). Then
		      \[
			      \Re \left\{ \lambda ^n d_n \right\}
			      \geq \Re \left\{ \lambda ^n c_n \right\} - \lVert d-c \rVert _\infty
			      \geq \frac{1}{2}.
		      \]
		      If \((\lambda I - T^{\ast} ) e = d\) for some \(e = \ell _\infty \), then
		      \[
			      e = \overline{\lambda} ^{n+1} \sum_{m=1}^{n} \lambda ^m d_m,
		      \]
		      implying that \(\vert e_n \vert \geq n / 2\), i.e., \(e \notin \ell _\infty \), contradiction. We then conclude that \(\im(\lambda I - T^{\ast} )\) does not intersect \hyperref[def:ball]{ball} centered at \(c\) with radius \(1 / 2\), hence \(\lambda \in \sigma _r(T^{\ast} )\).
	\end{itemize}
\end{eg}

\chapter{Self-Adjoint Operators on Hilbert Spaces}
Throughout this chapter, \(\mathcal{\MakeUppercase{h}} \) will denote a \hyperref[def:Hilbert-space]{Hilbert space}, and we will study \hyperref[rmk:bounded-op]{bounded} \hyperref[def:self-adjoint-op]{self-adjoint operators} on \(\mathcal{\MakeUppercase{h}} \).

\section{Spectrum of Self-Adjoint Operators}
\begin{definition}[Self-adjoint operator]\label{def:self-adjoint-op}
	Let \(T\in \mathcal{\MakeUppercase{l}} (\mathcal{\MakeUppercase{h}} )\) for \(\mathcal{\MakeUppercase{h}} \) being a \hyperref[def:Hilbert-space]{Hilbert space}. Then \(T\) is \emph{self-adjoint} if for all \(x, y\in \mathcal{\MakeUppercase{h}} \),
	\[
		\left\langle Tx, y \right\rangle = \left\langle x, Ty \right\rangle.
	\]
\end{definition}

There are lots of examples of \hyperref[def:self-adjoint-op]{self-adjoint operators}.

\begin{eg}[Hermitian matrix]
	The linear operators on \(\mathbb{\MakeUppercase{c}} ^n\) given by Hermitian matrices \(A = (a_{ij} )\) is \hyperref[def:self-adjoint-op]{self-adjoint}.
\end{eg}
\begin{explanation}
	Since \(a_{ij} = \overline{a_{ij}}\).
\end{explanation}

\begin{eg}[Integral operator]
	The integral operators \(T\) on \(L^2([0, 1])\) with Hermitian symmetric kernels \(k(s, t)\) given by
	\[
		(Tf)(t) = \int_{0}^{1} k(s, t)f(s) \,\mathrm{d}s
	\]
	is \hyperref[def:self-adjoint-op]{self-adjoint}.
\end{eg}
\begin{explanation}
	Since \(k(s, t) = \overline{k(s, t)}\).
\end{explanation}

Also, the \hyperref[def:orthogonal-projection]{orthogonal projection} \(P\) on \(\mathcal{\MakeUppercase{h}} \) is \hyperref[def:self-adjoint-op]{self-adjoint}.

\begin{remark}
	Every \(A\in \mathcal{\MakeUppercase{l}} (\mathcal{\MakeUppercase{h}} )\) can be represented as \(A = T+iS\) with \(T, S\) \hyperref[def:self-adjoint-op]{self-adjoint}.
\end{remark}
\begin{explanation}
	If \(A = T + iS\), then \(A^{\ast} = T - iS\). Solving these two equations gives
	\[
		T = \frac{A + A^{\ast} }{2},\quad S = \frac{A - A^{\ast} }{2i}.
	\]
\end{explanation}

\subsection{The Quadratic Form and the Norm of a Self-Adjoint Operator}
An important object in the study of \hyperref[def:self-adjoint-op]{self-adjoint operator} is the quadratic form \(f\colon \mathcal{\MakeUppercase{h}} \to \mathbb{\MakeUppercase{r}} \) where
\[
	f(x) = \left\langle Tx, x \right\rangle
\]
for \(x\in \mathcal{\MakeUppercase{h}} \), where \(f(\cdot)\) is real since \(\left\langle Tx, x \right\rangle = \left\langle x, Tx \right\rangle = \overline{\left\langle Tx, x \right\rangle }\). Furthermore, \(f(\cdot)\) determines \(T\) uniquely by the generalized \hyperref[lma:polarization-identity]{polarization identity}\footnote{When \(T = I\), we get back the usual \hyperref[lma:polarization-identity]{polarization identity}.}
\[
	\left\langle Tx, y \right\rangle = \frac{1}{4} \left[ f(x+ y) - f(x-y) + if(x+iy) - if(x-iy) \right].
\]
Since \(f(\cdot)\) determines \(T\) uniquely, we should be able to compute properties of \(T\) using \(f\). The first property is the following.
\begin{proposition}
	For every \hyperref[def:self-adjoint-op]{self-adjoint operator} \(T\in \mathcal{\MakeUppercase{l}} (\mathcal{\MakeUppercase{h}} )\), one has
	\[
		\lVert T \rVert = \sup _{\lVert x \rVert = 1} \vert \left\langle Tx, x \right\rangle  \vert.
	\]
\end{proposition}
\begin{proof}
	Firstly, from \hyperref[thm:Cauchy-Schwarz-ineq]{Cauchy-Schwarz}, we have \(\vert \left\langle Tx, x \right\rangle \vert \leq \lVert Tx \rVert \lVert x \rVert \), implying that
	\[
		\sup _{\lVert x \rVert = 1} \vert \left\langle Tx, x \right\rangle  \vert \leq \sup _{\lVert x \rVert = 1} \lVert Tx \rVert = \lVert T \rVert .
	\]
	To get the equality, we use the \hyperref[lma:polarization-identity]{polarization identity}, where
	\[
		\Re \left\langle Tx, y \right\rangle
		= \frac{1}{4} \left[ \left\langle T(x+ y) , x+y\right\rangle - \left\langle T(x-y), x-y \right\rangle  \right].
	\]
	Let \(M\coloneqq \sup _{\lVert x \rVert = 1} \vert \left\langle Tx, x \right\rangle  \vert \), we have
	\[
		\Re\left\langle Tx, y \right\rangle
		\leq \frac{M}{4} \left[ \lVert x+y \rVert ^{2} + \lVert x-y \rVert ^{2}  \right]
		= \frac{M}{4} \left[ 2\lVert x \rVert ^{2} + 2 \lVert y \rVert ^{2} \right],
	\]
	where the last equality follows from the \hyperref[lma:parallelogram-law]{parallelogram law}. This implies that
	\[
		\lVert T \rVert
		= \sup _{\lVert x \rVert = \lVert y \rVert = 1} \vert \left\langle Tx, y \right\rangle \vert
		= \sup _{\lVert x \rVert = \lVert y \rVert = 1} \Re \left\langle Tx, y \right\rangle
		\leq M.
	\]
\end{proof}

\subsection{Criterion of Spectrum Points}
We now study the \hyperref[def:spectrum-point]{spectrum} of \hyperref[def:self-adjoint-op]{self adjoint operators} \(T\in \mathcal{\MakeUppercase{l}} (\mathcal{\MakeUppercase{h}} )\). We're going to show that the whole \hyperref[def:spectrum-point]{spectrum} of \(T\) is real for \(T\) being a \hyperref[def:self-adjoint-op]{self-adjoint operator}, i.e., \(\sigma (T) \subseteq \mathbb{\MakeUppercase{r}} \). Let's start with some basic facts.

\begin{lemma}
	Let \(T\in \mathcal{\MakeUppercase{l}} (\mathcal{\MakeUppercase{h}} )\) be a \hyperref[def:self-adjoint-op]{self-adjoint}. Then \(\sigma _p(T) \subseteq \mathbb{\MakeUppercase{r}} \) and \(\sigma _r(T) = \varnothing \).
\end{lemma}
\begin{proof}
	Let's first show that \(\sigma _p(T) \subseteq \mathbb{\MakeUppercase{r}} \). Suppose \(Tx=\lambda x\), then
	\[
		\begin{dcases}
			\left\langle Tx, x \right\rangle = \left\langle \lambda x, x \right\rangle = \lambda \lVert x^{2}  \rVert ; \\
			\left\langle x, Tx \right\rangle = \left\langle x, \lambda x \right\rangle = \overline{\lambda} \lVert x^{2}  \rVert
		\end{dcases}
		\implies (\lambda - \overline{\lambda}) \lVert x \rVert ^{2} = 0.
	\]
	Since \(x \neq 0\), this implies that \(\lambda - \overline{\lambda} = 0\), i.e., \(\lambda \in \mathbb{\MakeUppercase{r}} \).

	To show \(\sigma _r(T) = \varnothing \), let \(\lambda \in \sigma _r(T)\), then \(\ker (T-\lambda I) = \left\{ 0 \right\} \) and \(\im (T-\lambda I)\) is not dense in \(\mathcal{\MakeUppercase{h}} \). Notice that since \(\lambda \notin \sigma _p(T)\), then \(\overline{\lambda} \notin \sigma _p(T)\) as we just proved. We then have \(\left\{ 0 \right\} = \ker (T-\overline{\lambda} I) = \ker (T-\lambda I)^{\ast} \) since \(T\) is \hyperref[def:self-adjoint-op]{self-adjoint} and \(T^{\ast} = T\), \(\lambda ^{\ast} = \overline{\lambda} \). But then we have
	\[
		\im (T-\lambda I)^{\perp} = \ker (T-\lambda I)^{\ast} = \ker (T - \overline{\lambda} I) = \left\{ 0 \right\} ,
	\]
	hence \(\im (T - \lambda I)\) is dense in \(\mathcal{\MakeUppercase{h}} \), a contradiction. So \(\lambda \) does not exist, proving that \(\sigma _r(T)= \varnothing \).
\end{proof}