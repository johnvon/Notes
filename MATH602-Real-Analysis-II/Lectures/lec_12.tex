\lecture{12}{6 Oct. 14:30}{Open Mapping Theorem}
Let's start with \hyperref[prop:Baire-category]{Baire category theorem}.

\begin{proposition}[Baire category theorem]\label{prop:Baire-category}
	A complete \hyperref[prev:metric]{metric} space \(M\) is never the union of a countable number of \hyperref[def:nowhere-dense]{nowhere dense} sets.
\end{proposition}
\begin{proof}
	We prove this by contradiction. Assume \(M = \bigcup_{n=1}^{\infty} A_n\) with each \(A_n\) \hyperref[def:nowhere-dense]{nowhere dense}. Since \(A_1\) is \hyperref[def:nowhere-dense]{nowhere dense}, so we can find \(x_1\in M- \overline{A} _1\). Furthermore, since \(\overline{A} _1\) is closed, so we can find open \hyperref[def:ball]{ball} \(B_1\) centered at \(x_1\) with radius less or equal to \(1\) such that \(B_1 \cap A_1= \varnothing \).

	Similarly, \(A_2\) is \hyperref[def:nowhere-dense]{nowhere dense}, so there exists \(x_1\in B_1 - \overline{A} _2\), with \(\overline{A} _2\) closed, we can still find \hyperref[def:ball]{ball} \(B_2\) centered at \(x_2\) with radius less or equal to \(1 / 2\) such that
	\[
		x_2 \in B_2 \subseteq \overline{B} _2 \subseteq B_1
	\]
	and \(B_2 \cap A_2 = \varnothing \). By induction, we can find a sequence \(\left\{ x_n \right\} _{n=1}^{\infty} \) and open \hyperref[def:ball]{balls} \(B_n\) such that
	\[
		x_{n+1} \in B_{n+1} \subseteq \overline{B} _{n+1} \subseteq B_n
	\]
	where \(B_n\) has radius smaller than \(1 / 2^{n-1}\) and \(B_n \cap A_n = \varnothing \).

	Now, since the sequence \(\left\{ x_ n \right\} \) is Cauchy and \(M\) is complete, we know that \(x_n \to x_{\infty }\in M\), so \(x_{\infty }\in B_n\) for all \(n\) and hence \(x_\infty \notin A_n\) for all \(n\). This implies
	\[
		M \neq \bigcup\limits_{n=1}^{\infty} A_n,
	\]
	which is a contradiction \(\conta\)
\end{proof}

We can now prove the central theorem in functional analysis, the \hyperref[thm:open-mapping]{open mapping theorem}.

\begin{theorem}[Open mapping theorem]\label{thm:open-mapping}
	Let \(X, Y\) be \hyperref[def:Banach-space]{Banach spaces} and \(T\in \mathcal{\MakeUppercase{l}} (X, Y)\). Assume \(T\) is surjective, i.e., \(T(X)= Y\), then \(T\) maps open sets in \(X\) to open sets in \(Y\).
\end{theorem}
\begin{proof}
	Let \(B_X \coloneqq \left\{ x\in X \mid \left\lVert x\right\rVert \leq 1 \right\} \) be a unit \hyperref[def:ball]{ball} in \(X\), similarly \(B_Y\) be a unit \hyperref[def:ball]{ball} in \(Y\).

	\begin{claim}
		It's sufficient to show \(T(B_X) \supseteq \epsilon B_Y\) for some \(\epsilon > 0\).
	\end{claim}
	\begin{explanation}
		To see this, let \(U \subseteq X\) be an open set and \(y\in TU\). We need to show \(TU\) contains a neighborhood of \(y\). Let \(x\in U\) such that \(Tx = y\). Since \(U\) is open, so there exists \(\delta > 0\) such that \(U \supseteq x + \delta B_X\), so
		\[
			TU \supseteq T(x + \delta B_X) = y + \delta T(B_X) \supseteq y + \delta \epsilon B_Y,
		\]
		i.e., \(TU\) contains a neighborhood of \(y\).
	\end{explanation}

	We now show \(TB_X \supseteq \epsilon B_Y\) for some \(\epsilon > 0\). Observe that \(X = \bigcup_{n=1}^{\infty} n B_X\), hence
	\[
		Y = TX = \bigcup\limits_{n=1}^{\infty } n T(B_{X}).
	\]
	From \hyperref[prop:Baire-category]{Baire category theorem}, we know that there exists \(n \geq 1\) such that \(\overline{nT(B_X)}\) has non-empty interior, i.e., \(\overline{TB_X}\) has non-empty interior too. Hence, there exists \(y\in Y\), \(\delta > 0\) such that \(y + \delta B_Y \subseteq \overline{TB_X}\). With \(TX=Y\), there exists \(x\in X\) such that \(Tx = y\), hence \(\delta B_Y \subseteq \overline{T(B_X - \left\{ x \right\} )}\). Since \(B_X - \left\{ x \right\} \subseteq n B_X\) for some \(n \geq 1\), meaning that \(\delta B_Y \subseteq n \overline{TB_X}\), implying \(\overline{TB_X}\supseteq \epsilon B_Y\) for some \(\epsilon > 0\). Finally, we show that \(\overline{TB_X} \subseteq T(2 B_X)\), which will imply
	\[
		TB_X \supseteq \frac{1}{2}\overline{TB_X} \supseteq \frac{\epsilon}{2} B_Y,
	\]
	completes the proof. To see this, we use a scaling argument. Let \(y\in \overline{TB_X}\), then there exists \(x_1 \in B_X\) such that
	\[
		y-Tx_1 \in \frac{\epsilon}{2} B_y \subseteq \overline{T \frac{1}{2}B_X}.
	\]
	We can then choose \(x_2\in \frac{1}{2}B_X\) such that
	\[
		y-Tx_1 - Tx_2 \in \frac{\epsilon}{4}B_Y \subseteq \overline{T\frac{1}{2^2}B_X}.
	\]
	By induction, we can construct a sequence \(\left\{ x_n \right\}_{n\geq 1} \) such that
	\[
		x_n \in \frac{1}{2^{n-1}}B_X,\quad y- \sum_{j=1}^n Tx_j\in \frac{\epsilon}{2^n} B_Y.
	\]
	Then, \(x = \sum_{j=1} ^{\infty }x_n \in 2B_X\) where \(Tx = y\).
\end{proof}

\subsection{Inverse Mapping Theorem}
As an immediate consequence of the \hyperref[thm:open-mapping]{open mapping theorem}, we have the \hyperref[thm:inverse-mapping]{inverse mapping theorem}.

\begin{theorem}[Inverse mapping theorem]\label{thm:inverse-mapping}
	Let \(T\colon X\to Y\) be a \hyperref[def:bounded-linear-op]{bounded linear operator} between \hyperref[def:Banach-space]{Banach spaces} \(X\) and \(Y\) which is both injective and surjective. Then \(T\) has a \hyperref[rmk:bounded-op]{bounded} inverse \(T^{-1} \in \mathcal{\MakeUppercase{l}} (Y, X)\).
\end{theorem}
\begin{proof}
	Since \hyperref[thm:open-mapping]{open mapping theorem} states that the preimages of open sets under \(T^{-1} \) are open, hence \(T^{-1} \) is continuous.
\end{proof}

\hyperref[thm:inverse-mapping]{Inverse mapping theorem} is used to establish stability of solutions of linear equations. Consider a linear equation in \(x\) in a \hyperref[def:Banach-space]{Banach space}
\[
	Tx = b
\]
for \(T\in \mathcal{\MakeUppercase{l}} (X, Y)\) and \(b\in Y\). Assume that a solution \(x\) exists and is unique for every \(b\), then, from \hyperref[thm:inverse-mapping]{inverse mapping theorem}, we see that the solution \(x = x(b)\) is continuous w.r.t. \(b\). In other words, the solution is stable under perturbations of \(b\). In case \(T\) is not injective but is surjective, we can still apply \hyperref[thm:inverse-mapping]{inverse mapping theorem} to the injectivization of \(T\) as follows.

\begin{corollary}[Surjective operators are essentially quotient maps]\label{col:surjective-op-are-essentially-quo-maps}
	Let \(X\), \(Y\) be \hyperref[def:Banach-space]{Banach spaces}. Then every surjective \hyperref[def:bounded-linear-op]{bounded linear operator} \(T\in \mathcal{\MakeUppercase{l}} (X, Y)\) is a composition of a quotient map and an isomorphism. Specifically,
	\[
		T = \widetilde{T} q,
	\]
	where \(q\colon X\to \quotient{X}{\ker(T)} \) is the quotient map, \(\widetilde{T} \colon \quotient{X}{\ker(T)} \to Y\) is an isomorphism.
\end{corollary}
\begin{proof}
	Let \(\widetilde{T} \) be the injectivization of \(T\) then by construction, \(T = \widetilde{T} q\) and \(\widetilde{T} \) is injective. Since \(T\) is surjective, \(\widetilde{T} \) is also surjective. Hence, by \hyperref[thm:inverse-mapping]{inverse mapping theorem}, \(\widetilde{T} \) is an isomorphism.
\end{proof}

\subsection{Isomorphic Embeddings}
Finally, as we know, the \hyperref[def:kernel]{kernel} of every \hyperref[def:bounded-linear-op]{bounded linear operators} \(T\in \mathcal{\MakeUppercase{l}} (X, Y)\) is always a closed subspace, while the \hyperref[def:image]{image} of \(T\) may or may not be closed. We can also characterize this.

\begin{proposition}
	Given two \hyperref[def:Banach-space]{Banach spaces} \(X, Y\) and \(T\in \mathcal{\MakeUppercase{l}} (X, Y)\), the following are equivalent.
	\begin{enumerate}[(a)]
		\item \(T\) is injective and \(\im (T)\) is closed.
		\item \(T\) is \hyperref[rmk:bounded-op]{bounded} below, i.e., \(\exists c > 0\), \(\left\lVert Tx\right\rVert \geq c \left\lVert x\right\rVert \) for all \(x\in X\).
	\end{enumerate}
\end{proposition}
\begin{proof}
	To show that (a) implies (b), we see that \(T^{-1} \colon \im(T) \to X\) is \hyperref[rmk:bounded-op]{bounded}  since \(\im(T)\) is \hyperref[def:Banach-space]{Banach space}, from \hyperref[thm:open-mapping]{open mapping theorem},
	\[
		\left\lVert T^{-1} y\right\rVert \leq c^{-1} \left\lVert y\right\rVert
	\]
	for \(y\in \im(T)\), \(c > 0\) being some constant. Set \(y\coloneqq Tx\), then
	\[
		\left\lVert Tx\right\rVert \geq c\left\lVert x\right\rVert
	\]
	for \(x\in X\), we're done. To show another direction, suppose \(T\) is \hyperref[rmk:bounded-op]{bounded} below, then \(T\) is injective since \(Tx= 0\) implies \(x = 0\). To see \(\im(T)\) is closed, let \(x_{n} \in X\) for \(n\geq 1\) be a sequence such that \(\left\{ Tx_n \right\} _{n \geq 1}\) is Cauchy such that \(\left\lVert Tx_n - Tx_m\right\rVert \geq c \left\lVert x_n - x_m\right\rVert \) for all \(n, m\), implying \(\left\{ x_n \right\} _{n\geq 1}\)  is Cauchy, hence \(x_n \to x_\infty \in X\), i.e., \(Tx_n \to Tx_\infty \in \im(T)\), proving the result.
\end{proof}

\section{Closed Graph Theorem}
We now study the second main theorem in functional analysis, which characterizes the property of the \hyperref[def:graph]{graph} of a \hyperref[def:bounded-linear-op]{bounded linear operator}.
\begin{definition}[Graph]\label{def:graph}
	Let \(T\in \mathcal{\MakeUppercase{l}} (X, Y)\) for \(X\), \(Y\) being \hyperref[def:Banach-space]{Banach spaces}. Then the \emph{graph} \(\Gamma (T)\) of \(T\) is defined as
	\[
		\Gamma (T) \coloneqq \left\{ (x, Tx) \in X \times Y\mid x\in X \right\}.
	\]
\end{definition}

Clearly, \(\Gamma (T)\) is a \hyperref[def:linear-vector-space]{linear subspace} of the \hyperref[def:normed-vector-space]{normed space} \(X\oplus Y\).

\begin{definition}[Closed graph]\label{def:closed-graph}
	The \hyperref[def:graph]{graph} \(\Gamma (T)\) of \(T\) is \emph{closed} if it is a closed subspace of \(X \times Y\).
\end{definition}

Hence, if \(\left\{ x_n \right\} _{n\geq 1}\) is a sequence in \(X\) such that both \(\left\{ x_n \right\} _{n\geq 1}\) and \(\left\{ Tx_n \right\}_{n\geq 1} \) are Cauchy, then there exists \(x_\infty \in X\) such that \(x_n \to x_\infty \) and \(Tx_\infty \to y_\infty \) for \(y_\infty = Tx_\infty \).

\begin{theorem}[Closed graph theorem]\label{thm:closed-graph}
	Let \(T\colon X\to Y\) be a \hyperref[def:linear-op]{linear operator} between \hyperref[def:Banach-space]{Banach spaces} \(X\) and \(Y\). Then \(T\) is \hyperref[def:bounded-linear-op]{bounded} (continuous) if and only if \(\Gamma (T)\) is \hyperref[def:closed-graph]{closed}.
\end{theorem}
\begin{proof}
	The forward direction is easy since if \(T\) is \hyperref[def:bounded-linear-op]{bounded}, then \(\Gamma (T)\) is \hyperref[def:closed-graph]{closed}.

	Now assume \(\Gamma (T)\) is \hyperref[def:closed-graph]{closed}, then we see that \(\Gamma (T)\) is a \hyperref[def:Banach-space]{Banach space}, so we can now use \hyperref[thm:open-mapping]{open mapping theorem}. Define a \hyperref[def:norm]{norm} on \(X \times Y\) by
	\[
		\left\lVert (x, y)\right\rVert = \left\lVert x\right\rVert + \left\lVert y\right\rVert,
	\]
	then \(\Gamma (T)\) is a \hyperref[def:Banach-space]{Banach space} with this \hyperref[def:norm]{norm}. Define \(u\colon \Gamma (T)\to X\) by \(u(x, Tx)=x\) for \(x\in X\), then \(u\) is \hyperref[rmk:bounded-op]{bounded} wince \(\left\lVert u\right\rVert \leq 1\). From \hyperref[thm:open-mapping]{open mapping theorem}, we know that \(u\) is surjective and injective implies \(u^{-1} \colon X\to \Gamma (T)\) is \hyperref[rmk:bounded-op]{bounded}, hence
	\[
		\left\lVert u(x, Tx)\right\rVert \geq c \left\lVert (x, Tx)\right\rVert
	\]
	for all \(x\in X\) and some \(c> 0\), i.e.,
	\[
		\left\lVert x\right\rVert \geq c \left( \left\lVert x\right\rVert + \left\lVert Tx\right\rVert  \right) \implies \left\lVert Tx\right\rVert \leq \left( \frac{1}{c}- 1 \right) \left\lVert x\right\rVert
	\]
	for all \(x\in X\), so \(T\) is \hyperref[rmk:bounded-op]{bounded}.
\end{proof}

\subsection{Symmetric Operators on Hilbert Spaces}
One application to self-\hyperref[def:adjoint-op]{adjoint (symmetric) operator}, i.e., \(T^{\ast} = T\), on \hyperref[def:Hilbert-space]{Hilbert space} is the following.

\begin{theorem}[Hellinger-Toeplitz theorem]\label{thm:Hellinger-Toeplitz}
	Let \(T\colon \mathcal{\MakeUppercase{h}} \to \mathcal{\MakeUppercase{h}} \) be a \hyperref[def:linear-op]{linear operator}. If \(T\) is self-\hyperref[def:adjoint-op]{adjoint}, i.e., \(\left\langle Tx, y \right\rangle  = \left\langle x, Ty \right\rangle\) for \(x, y\in \mathcal{\MakeUppercase{h}} \), then \(T\) is \hyperref[rmk:bounded-op]{bounded}.
\end{theorem}
\begin{proof}
	From \hyperref[thm:closed-graph]{closed graph theorem}, it suffices to show that for a self-\hyperref[def:adjoint-op]{adjoint operator} \(T\), \(\Gamma (T)\) is \hyperref[def:closed-graph]{closed}. Let \(\left\{ x_n \right\} _{n\geq 1}\) in \(\mathcal{\MakeUppercase{h}} \) such that \(x_n \to x_\infty \in \mathcal{\MakeUppercase{h}} \) and \(Tx_n \to y_\infty \in\mathcal{\MakeUppercase{h}} \), then we need to show \(Tx_\infty \to y_\infty \). From the self-\hyperref[def:adjoint-op]{adjointness} of \(T\) and the continuity of an \hyperref[def:inner-product]{inner product}, for all \(z\in \mathcal{\MakeUppercase{h}} \),
	\[
		\left\langle z, y_\infty  \right\rangle = \lim\limits_{n \to \infty} \left\langle z, Tx_n \right\rangle = \lim\limits_{n \to \infty} \left\langle Tz, x_n \right\rangle = \left\langle Tz, x_\infty  \right\rangle = \left\langle z, Tx_\infty  \right\rangle.
	\]
	Since this holds for all \(z\in \mathcal{\MakeUppercase{h}} \), we know that \(Tx_\infty = y_\infty \), hence \(\Gamma (T)\) is \hyperref[def:closed-graph]{closed}, so \(T\) is \hyperref[rmk:bounded-op]{bounded}.
\end{proof}

\hyperref[thm:Hellinger-Toeplitz]{Hellinger-Toeplitz theorem} identifies the source of considerable difficulties in mathematical physics since many natural operators such as differential, though satisfy the symmetry condition, but are unbounded, and hence \hyperref[thm:Hellinger-Toeplitz]{Hellinger-Toeplitz theorem} declares that such operators \emph{can not be defined everywhere} on the \hyperref[def:Hilbert-space]{Hilbert space}.

\begin{eg}
	There are no useful notions of differentiation that would make all \(f\in L^2\) differentiable.
\end{eg}