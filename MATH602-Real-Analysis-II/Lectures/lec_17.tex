\lecture{17}{27 Oct. 14:30}{Weak* Topology}
There's another important case regarding \(C(K)\).

\begin{theorem}[Weak convergence in \(C(K)\)]\label{thm:weak-convergence-in-CK}
	Let \(K\) be a \hyperref[def:compact]{compact} Hausdorff space and \(C(K)\) be the space of continuous functions \(f\colon K\to \mathbb{R} \) with \(\left\lVert f\right\rVert = \sup _{t\in K} \left\vert f(t) \right\vert \). Then a sequence \(\left\{ x_n \right\}_{n \geq 1} \) in \(C(K)\) \hyperref[def:weakly-convergence]{converges weakly} to \(x\in C(K)\)  if and only if the sequence \(\left\{ x_n \right\} _{n\geq 1}\) is bounded in \(C(K)\) and \(\lim\limits_{n \to \infty} x_n(t)=x(t)\) for all \(t\in K\).
\end{theorem}
\begin{proof}
	Suppose \(\left\{ x_n \right\} _{n \geq 1}\) \hyperref[def:weakly-convergence]{converges weakly} to \(x\), then \(\sup _{n\geq 1} \left\lVert x_n\right\rVert < \infty \). By \autoref{prop:weak-closedness}, if \(t\in K\), then \(\delta _t\in C(K)^{\ast} \) where \(\delta _t(x) = x(t)\). We see that
	\[
		\lim_{n \to \infty} \delta _t(x_n) = \delta _t(x)\implies \lim_{n \to \infty} x_n(t) = x(t).
	\]

	Conversely, assume \(\sup _{n\geq 1}\left\lVert x_n\right\rVert < \infty \) and \(\lim_{n \to \infty} x_n(t) = x(t)\) for all \(t\in K\). We now need to show that \(\lim_{n \to \infty} f(x_{n} ) = f(x)\) for all \(f\in C(K)^{\ast} \). Recall that \(C(K)^{\ast} \) is the space of bounded signed measures \(\mu \) on \(K\),\footnote{This is just a variant of \hyperref[thm:Riesz-representation]{Riesz representation theorem}.} and \(\left\lVert \mu \right\rVert = TV(\mu )\), we have
	\[
		\lim_{n \to \infty} \int _K x_n \,\mathrm{d} \mu = \int_K x \,\mathrm{d} \mu,
	\]
	from the \href{https://en.wikipedia.org/wiki/Dominated_convergence_theorem}{dominated convergence theorem}, proving the result.
\end{proof}

\section{Weak* Topology}
On \(X^{\ast} \), there are two natural weaker topologies: one is the \hyperref[def:weak-topology]{weak topology} which makes all functionals in \(X^{\ast\ast} \) continuous on \(X^{\ast} \); the other topology, called \hyperref[def:weak*-topology]{weak* topology}, is only concerned with the continuity of functionals that come from \(X \subseteq X^{\ast\ast} \).

\subsection{Weak* Convergence}
We again start from convergence, and in this case, the \hyperref[def:weak*-convergence]{weak* convergence}.

\begin{definition}[Weak* convergence]\label{def:weak*-convergence}
	Let \(X\) be a \hyperref[def:normed-vector-space]{normed space}, a sequence \(\left\{ f_{n} \right\}_{n\geq 1} \) in \(X^{\ast} \) is \emph{weak* converging} to \(f\) in \(X^{\ast}\) if \(\lim\limits_{n \to \infty} f_n(x) = f(x)\) for all \(x\in X\).
\end{definition}

\begin{notation}
	If \(\left\{ f_n \right\}_{n\geq 1} \) is \hyperref[def:weak*-convergence]{weak* converging} to \(f\), we write \(f_n \overset{\text{w*}}{\to } f\).
\end{notation}

We see that if we consider the \hyperref[def:weakly-convergence]{weak convergence} for \(f_k \in X^{\ast} \), we should test on all the double \hyperref[def:dual-space]{dual} elements \(x^{\ast\ast}\in X^{\ast\ast}\); but as said, \hyperref[def:weak*-convergence]{weak* convergence} only test on the subset \(X \subseteq X^{\ast\ast}\), so \autoref{def:weak*-convergence} becomes point-wise convergence.

\begin{remark}
	The \hyperref[def:weakly-convergence]{weak} and \hyperref[def:weak*-convergence]{weak* convergence} coincides on \(X^{\ast} \) if \(X\) is \hyperref[def:reflexive-space]{reflexive}, i.e., \(X^{\ast\ast} \equiv X\).
\end{remark}

\subsection{Weak* Topology}
Similarly to \autoref{def:weak-topology}, we now defined the so-called \hyperref[def:weak*-topology]{weak* topology} on \(X^{\ast} \).

\begin{definition}[Weak* topology]\label{def:weak*-topology}
	Let \(X\) be a \hyperref[def:normed-vector-space]{normed space}. The \emph{weak* topology} on \(X^{\ast} \) is defined as the weakest topology in which point evaluation maps \(f\mapsto f(x)\) from \(X^{\ast} \) to \(\mathbb{R} \) are continuous for all \(x\in X\).
\end{definition}

\begin{note}[Completeness]
	If \(\left\{ f_n \right\}_{n \geq 1}\) is \hyperref[def:weak*-convergence]{weak*} Cauchy, then \hyperref[thm:uniform-boundedness]{uniform boundedness principle} implies that there exists \(f\in X^{\ast} \) such that \(f_n \overset{\text{w*}}{\to } f\).
\end{note}

Equivalently, the base of the \hyperref[def:weak*-topology]{weak* topology} is given by the \href{https://en.wikipedia.org/wiki/Cylinder_set}{cylinders}
\[
	\left\{ f\in X^{\ast} \colon \left\vert (f-f_0)(x_k) \right\vert < \epsilon , k = 1, \ldots , N \right\}
\]
where \(f_0\in X^{\ast} \), \(x_k\in X\), \(\epsilon > 0\), and \(N\in \mathbb{N} \). So again, these \href{https://en.wikipedia.org/wiki/Cylinder_set}{cylinders} form a local base of \hyperref[def:weak*-topology]{weak* topology} at \(x^{\ast\ast}_k(f_0) = x_k(f_0)\).

\begin{remark}[\(\mathbb{R} ^{\infty} \) embedding]
	Consider the embedding from \(X^{\ast} \) to an infinite product of \(\mathbb{R} \) such that
	\[
		X^{\ast} \to \mathbb{R} ^{\infty },\quad f\mapsto (f(x))_{x\in X \subseteq X^{\ast\ast}} = (f(x_1), f(x_2), \ldots )
	\]
	given \(\left\{ x_i \right\} _{i\geq 1}\) in \(X \subseteq X^{\ast\ast} \): \hyperref[def:weak*-topology]{weak* topology} is again induced from the products of reals!
\end{remark}

As we mentioned before, since \(X \subseteq X^{\ast\ast}\), \hyperref[def:weak*-topology]{weak* topology} is weaker than the \hyperref[def:weak-topology]{weak topology} on \(X^{\ast} \). However, for \hyperref[def:reflexive-space]{reflexive spaces}, these two topologies are of course equivalent.

The main result on \hyperref[def:weak*-topology]{weak* topology} is \hyperref[thm:Banach-Alaoglu]{Banach-Alaoglu theorem}, which allows us to still have a weaker notion of (\hyperref[def:weak*-topology]{weak*}) \hyperref[def:compact]{compactness} of unit \hyperref[def:ball]{balls} even though we know that it's not (strongly) \hyperref[def:compact]{compact} from \hyperref[thm:Riesz]{Riesz's theorem}. The result depends on \hyperref[thm:Tychonoff]{Tychonoff's theorem}.

\begin{theorem}[\href{https://en.wikipedia.org/wiki/Tychonoff's_theorem}{Tychonoff's theorem}]\label{thm:Tychonoff}
	Let \(\left\{ X_\gamma\right\}_{\gamma \in \Gamma}\) be a collection of any number\footnote{May be countable or uncountable.} of topological spaces \(X_\gamma \). The Cartesian product \(\prod_{\gamma \in \Gamma } X_\gamma \) can be equipped with the product topology whose base is formed by the sets of the form
	\[
		\left\{ \prod_{\gamma \in \Gamma } A_\gamma \colon \text{\(A_\gamma\) is open in \(X_\gamma\); all but finitely many of \(A_\gamma\) equal \(X_\gamma\)}\right\}.
	\]
	Then, if each \(X_\gamma \) is \hyperref[def:compact]{compact}, then \(\prod_{\gamma \in \Gamma }X_\gamma \) is \hyperref[def:compact]{compact} in the product topology.
\end{theorem}

The proof is omitted here, but the upshot is that although the \(\mathbb{R} ^{\infty} \) embedding may involve uncountably many \(\mathbb{R} \), \hyperref[thm:Tychonoff]{Tychonoff's theorem} ensures that the \hyperref[def:compact]{compactness} is still preserved.

\begin{theorem}[Banach-Alaoglu theorem]\label{thm:Banach-Alaoglu}
	Let \(X\) be a \hyperref[def:Banach-space]{Banach space} with \hyperref[def:dual-space]{dual} \(X^{\ast} \), then the unit \hyperref[def:ball]{ball} \(B_{X^{\ast} }= \left\{ f\in X^{\ast} \colon \left\lVert f\right\rVert \leq 1 \right\}\) in \(X^{\ast} \) is \hyperref[def:compact]{compact} in the \hyperref[def:weak*-topology]{weak* topology}.
\end{theorem}
\begin{proof}
	Since \(f\in B_{X^{\ast} }\implies \left\vert f(x) \right\vert \leq \left\lVert x\right\rVert\) for \(x\in X\), so we can embed \(B_{X^{\ast} }\) into \(\prod_{x\in X} [-\left\lVert x\right\rVert , \left\lVert x\right\rVert]\).\footnote{Recall the \(\mathbb{R} ^{\infty} \) embedding: we identify \(f\in X^{\ast} \) by \((f(x))_{x\in X}\).} Note that this is the product of \hyperref[def:compact]{compact} spaces, so \hyperref[thm:Tychonoff]{Tychonoff's theorem} implies that this product is \hyperref[def:compact]{compact}, and we only need to show \(B_{X^{\ast} }\) is \hyperref[def:weak*-topology]{weak*} closed in \(\prod_{x\in X} [-\left\lVert x\right\rVert , \left\lVert x\right\rVert]\). Observe that
	\[
		B_{X^{\ast} } = \bigcap_{\substack{x, y\in X\\ a, b\in \mathbb{R} }}B_{x, y, a, b}, \text{ where } B_{x, y, a, b} = \left\{ f\in K\colon f(ax+by) = af(x) + bf(y) \right\}
	\]
	for \(K \coloneqq \prod_{x\in X}[-\left\lVert x\right\rVert , \left\lVert x\right\rVert ]\),\footnote{What we're claiming is that \(B_{X^{\ast} }\) consists of linear functions in \(K\).} so it's sufficient to show \(B_{x, y, a, b}\) is \hyperref[def:weak*-topology]{weak*} closed in \(K\). But since \(B_{x, y, a, b}\) is the preimage of (a closed set) \(\left\{ 0 \right\} \) under mapping \(f \mapsto f(ax+by) - af(x) - bf(y)\), which is continuous from the definition of \hyperref[def:weak*-topology]{weak* topology}, hence we know all \(B_{x, y, a, b}\) is (\hyperref[def:weak*-topology]{weak*}) closed as well, so is their intersection \(B_{X^{\ast} }\).
\end{proof}
Let's see some applications of \hyperref[thm:Banach-Alaoglu]{Banach-Alaoglu's theorem}. Consider spaces \(L^p(\mu )\) with \(1 < p < \infty\), we know that these are \hyperref[def:reflexive-space]{reflexive}, then the unit \hyperref[def:ball]{ball} in \(L^p(\mu )\) is \hyperref[def:compact]{compact} in the \hyperref[def:weak-topology]{weak topology}. Namely, let \(f_n\in L^p(\mu )\) for \(n \geq 1\) be bounded, then there exists a subsequence \(f_{n_k}\) for \(k\geq 1\) and \(f\in L^p(\mu )\) such that
\[
	\lim_{k \to \infty} \int f_{n_k}g\,\mathrm{d} \mu = \int fg\,\mathrm{d} \mu
\]
for all \(g\in L^{q}(\mu )\) with \(1 / p + 1 / q = 1\).

Another example is that let the \hyperref[def:Banach-space]{Banach space} be \(C(K)\), the \hyperref[def:dual-space]{dual} \(C(K)^{\ast} \) is the space of finite signed measure \(\mu\) on \(K\) with \(TV(\mu )< \infty \). Let \(\mu _n\) be a sequence of measures on \(K\) such that \(\sup _{n\geq 1} TV(\mu _n) < \infty \), then \(\left\{ \mu _n \right\}_{n\geq 1} \) is bounded in \(C(K)^{\ast} \). We see that there exists a subsequence \(\mu _{n_k}\) for \(k\geq 1\) such that \(\mu _{n_k}\overset{\text{w*}}{\to } \mu \in C(K)^{\ast} \), i.e.,
\[
	\lim_{n \to \infty} \int _K f\,\mathrm{d} \mu _{n_k} = \int _K f\,\mathrm{d} \mu
\]
for all \(f\in C(K)\).

\begin{note}
	This is generally referred to as \emph{weak convergence of measures}.
\end{note}

\chapter{Compact Operators and Spectral Theory}

\section{Compact Operators}
\hyperref[def:compact-op]{Compact operators} form an important class of \hyperref[def:bounded-linear-op]{bounded linear operators}. On the one hand, they are \emph{almost} \hyperref[rmk:finite-rank-op]{finite rank operators},\footnote{In the same way as \hyperref[def:compact]{compact sets} are \emph{almost} finite dimensional.} so they share some properties of \hyperref[rmk:finite-rank-op]{finite rank operators}, which facilitates their study. On the other hand, the class of \hyperref[def:compact-op]{compact operators} is wide enough to include integral and \hyperref[def:Hilbert-Schmidt-op]{Hilbert-Schmidt operators}, which are important in many applications.

\begin{definition}[Compact operator]\label{def:compact-op}
	Let \(X, Y\) be \hyperref[def:Banach-space]{Banach spaces}, we say a \hyperref[def:bounded-linear-op]{bounded linear operator} \(T\colon X\to Y\) is \emph{compact} if it maps bounded sets in \(X\) to \hyperref[def:precompact]{precompact sets} in \(Y\).
\end{definition}

\begin{notation}
	The set of \hyperref[def:compact-op]{compact operators} \(T\colon X\to Y\) is denoted as \(\mathcal{K} (X, Y)\).
\end{notation}

In other words, the closure of \(T(B_X)\) is \hyperref[def:compact]{compact} for \(T\) being \hyperref[def:compact-op]{compact}. Notice that since \hyperref[def:precompact]{precompact sets} are bounded, \hyperref[def:compact-op]{compact operators} are always \hyperref[def:bounded-map]{bounded}, i.e., \(\mathcal{K} (X, Y)\subseteq \mathcal{L} (X, Y)\). So indeed, we may remove the \hyperref[def:bounded-map]{bounded} condition in \autoref{def:compact-op}.

\begin{remark}[Finite rank operator]\label{rmk:finite-rank-op}
	An operator \(T\) is with \emph{finite rank} if \(\dim \im T < \infty \). And we see that every finite rank operator \(T\in \mathcal{L} (X, Y)\) is \hyperref[def:compact-op]{compact}.
\end{remark}
\begin{explanation}
	Since \(T(B_X)\) 	is a bounded subset of a finite dimensional \hyperref[def:normed-vector-space]{normed space} \(\im T \subseteq Y\), so \(T(B_X)\) is \hyperref[def:precompact]{precompact} by \hyperref[thm:Heine-Borel]{Heine-Borel theorem}.
\end{explanation}

\subsection{Properties of Compact Operators}
Let's study some basic properties of \hyperref[def:compact-op]{compact operators}.

\begin{proposition}[Peroperties of \(\mathcal{K}(X, Y)\)]\label{prop:compact-op}
	Let \(X, Y\) be \hyperref[def:Banach-space]{Banach spaces}.
	\begin{enumerate}[(a)]
		\item \(\mathcal{K}(X, Y)\) is a closed \hyperref[def:linear-vector-space]{linear subspace} of \(\mathcal{L} (X, Y)\).
		\item If \(T\in \mathcal{K}(X, Y)\) and \(S\) is \hyperref[def:bounded-linear-op]{bounded linear}, then both \(TS\) and \(ST\) are \hyperref[def:compact-op]{compact}.\footnote{\(S\) needs to have proper domain and range, i.e., \(TS\) is \hyperref[def:compact-op]{compact} if \(S\colon Z\to X\); \(ST\) is \hyperref[def:compact-op]{compact} if \(S\colon Y\to Z\).}
	\end{enumerate}
\end{proposition}
\begin{proof}
	Let's prove this one by one.
	\begin{enumerate}[(a)]
		\item Linearity follows since the sum of two \hyperref[def:precompact]{precompact sets} is \hyperref[def:precompact]{precompact}. For closedness, let \(T_n\in \mathcal{K}(X, Y)\), \(T\in \mathcal{L} (X, Y)\) with \(\lim_{n \to \infty} \left\lVert T_n - T\right\rVert = 0\), we need to show \(T\) is \hyperref[def:compact-op]{compact}, i.e., \(T\in \mathcal{K}(X, Y)\). This can be done by showing that we can cover \(T(B_X)\) by a finite \hyperref[def:eps-net]{\(\epsilon \)-net}. Given any \(\epsilon >0\), choose \(N\) such that \(\left\lVert T_N - T\right\rVert < \epsilon / 2\), i.e., \(\lVert T_N x - Tx \rVert \leq \epsilon / 2\) for every \(x\in B_X\). This means \(T_N(B_X)\) is an \hyperref[def:eps-net]{\(\epsilon /2\)-net} which covers \(T(B_X)\). Now, since \(T_N (B_X)\) is itself \hyperref[def:precompact]{precompact}, we can find another finite \hyperref[def:eps-net]{\(\epsilon / 2\)-net} \(\Omega _{\epsilon /2}\) covering \(T_N(B_X)\), hence \(\Omega _{\epsilon / 2}\) is a finite \hyperref[def:eps-net]{\(\epsilon \)-net} of \(T(B_X)\).
		\item Consider \(TS\) first. Since \(S\) maps unit \hyperref[def:ball]{balls} to bounded sets, and \(T\) maps bounded sets to \hyperref[def:precompact]{precompact sets}, hence \(TS\) is \hyperref[def:compact-op]{compact}. As for \(ST\), since \(T\) maps unit \hyperref[def:ball]{balls} to \hyperref[def:precompact]{precompact sets}, and because \(S\) is continuous, \(S\) maps \hyperref[def:precompact]{precompact sets} to \hyperref[def:precompact]{precompact sets}.
	\end{enumerate}
\end{proof}

\begin{remark}[\href{https://en.wikipedia.org/wiki/Operator_ideal}{Operator ideal}]
	If \(\mathcal{K}(X, Y)\) satisfies the second property in \autoref{prop:compact-op}, we call it an \emph{operator ideal}.
\end{remark}