\chapter{Banach and Hilbert Spaces}
\lecture{1}{30 Aug. 14:30}{Introduction}

We first briefly review different kinds of vector spaces.
\section{Linear Spaces}
Let's first see the simplest (i.e., without structures) vector space called \hyperref[def:linear-vector-space]{linear vector space}.
\begin{definition}[Linear vector space]\label{def:linear-vector-space}
	A \emph{linear vector space} \(E\) over a field \(\mathbb{\MakeUppercase{f}} \) is a set with operations of addition and multiplication (by a scalar) such that it's closed under operations, and also the addition and scalar multiplication obey
	\begin{enumerate}[(a)]
		\item \(u + v = v + u\) for \(u, v\in E\)
		\item \(u + (v + w) = (u + v) + w\) for \(u, v, w\in E\)
		\item \(\exists 0\in E\) such that \(0 + u = u + 0 = u\) for \(u\in E\)
		\item \(\forall u\in E\), \(\exists -u\in E\) such that \(u + (- u) = 0\)
		\item \(\lambda (u + v) = \lambda u + \lambda v\) for \(u, v\in E\), \(\lambda \in \mathbb{\MakeUppercase{f}} \)
		\item \((\lambda + \mu ) u = \lambda u + \mu u\) for \(u\in E\), \(\lambda , \mu \in \mathbb{\MakeUppercase{f}} \)
		\item \(\lambda (\mu u) = (\lambda \mu )u\) for \(u\in E\), \(\lambda , \mu \in \mathbb{\MakeUppercase{f}} \)
	\end{enumerate}
\end{definition}

\begin{remark}
	If \(v, w\in E\), \(\lambda , \mu \in \mathbb{\MakeUppercase{r}} \) (or \(\mathbb{\MakeUppercase{c}} \)), then \(\lambda v + \mu w\in E\).
\end{remark}

\begin{notation}[Real and complex vector space]
	If \(E\) is over \(\mathbb{\MakeUppercase{f}} = \mathbb{\MakeUppercase{c}} \), we usually call \(E\) a \emph{complex vector space}; if \(\mathbb{\MakeUppercase{f}} = \mathbb{\MakeUppercase{r}} \), we say \(E\) is a \emph{real vector space}.
\end{notation}

\begin{eg}
	\(\mathbb{\MakeUppercase{r}} ^n\) an \(n\) dimensional real \hyperref[def:linear-vector-space]{linear vector space}, \(\mathbb{\MakeUppercase{c}} ^n\) an \(n\) dimensional complex \hyperref[def:linear-vector-space]{linear vector space}.
\end{eg}

We concentrate on \(\infty \) dimensional \hyperref[def:linear-vector-space]{linear vector space}.

\begin{eg}
	Let \(K\) is a compact Hausdorff space, then
	\[
		E = \left\{ f\colon K\to \mathbb{\MakeUppercase{r}} \mid f(\cdot) \text{ is continuous}  \right\}
	\]
	is a \(\infty\) dimensional \textbf{real} \hyperref[def:linear-vector-space]{linear vector space}.
\end{eg}

\begin{notation}[Subspace]
	If \(E\) is a \hyperref[def:linear-vector-space]{linear vector space}, then we say \(E_1 \subseteq E\) is a \emph{subspace} if \(E_1 \subseteq E\) and \(E_1\) is itself a \hyperref[def:linear-vector-space]{linear vector space}. Moreover, if \(E_1 \subsetneq E\), we say \(E_1\) is a \emph{proper subspace}.
\end{notation}

Observe that a \hyperref[def:linear-vector-space]{linear vector space} can have many subspaces.

\section{Quotient Spacesl}
Sometimes we don't care about vectors in some directions, hence we introduce the notion of \hyperref[def:quotient-space]{quotient space}.

\begin{definition}[Quotient Space]\label{def:quotient-space}
	The \emph{quotient space} \(\quotient{E}{E_1} \) of two \hyperref[def:linear-vector-space]{linear vector spaces} \(E, E_1\) such that \(E_1 \subseteq E\) is the set of equivalence classes of vectors in \(E\) where equivalence is given by \(x\sim y\) if \(x - y\in E_1\). Additionally, denote \([x]\) as the equivalence class of \(x\in E\), i.e., \([x] = x + E_1\).
\end{definition}

One can see that \hyperref[def:quotient-space]{quotient space} \(\quotient{E}{E_1} \) is a \hyperref[def:linear-vector-space]{linear vector space} since if \(x_1 + x_2\in E\), \([x_1] + [x_2] = [x_1 + x_2]\), and also, \(\lambda [x] = [\lambda x]\) for \(\lambda \in \mathbb{\MakeUppercase{r}} \) or \(\mathbb{\MakeUppercase{c}} \), i.e., \(v, w\in \quotient{E}{E_1} \), \(\lambda , \mu \in\mathbb{\MakeUppercase{r}} \) or \(\mathbb{\MakeUppercase{c}} \) implies \(\lambda v + \mu w\in E\).

The dimension of a \hyperref[def:quotient-space]{quotient space} is defined as follows.

\begin{definition}[Codimension]\label{def:codimension}
	If \(\quotient{E}{E_1} \) has finite dimension, then the dimension of \(\quotient{E}{E_1} \) is called the \emph{codimension} of \(E_1\) in \(E\), denoted as \(\mathop{\mathrm{codim}}(E_1)\).
\end{definition}

\autoref{def:codimension} is introduced since the way of defining dimensions for finite dimensional \hyperref[def:linear-vector-space]{vector spaces} doesn't work here. Recall \autoref{thm:lec1} in the finite dimension case.

\begin{theorem}\label{thm:lec1}
	If \(E\) is finite dimensional, then \(\mathop{\mathrm{codim}}(E_1) + \dim (E_1) = \dim(E)\)
\end{theorem}

We see that we may encounter something like \(\infty - \infty \) if we define \(\mathop{\mathrm{codim}}(E_1) \coloneqq \dim(E) - \dim(E_1)\), and indeed, \autoref{def:codimension} is well-defined in this sense.

\begin{eg}
	There exists the case that \(\dim(E) = \infty \), \(\dim(E_1) < \infty\) where \(\dim(\quotient{E}{E_1} )<\infty \).
\end{eg}
\begin{explanation}
	Let \(E = \left\{ f\colon K\to \mathbb{\MakeUppercase{r}} \mid f(\cdot) \text{ continuous}  \right\} \) and \(E_1 = \left\{ f\in E\colon f(k_1) = 0 \right\} \) for a fixed \(k_1\in K\). We see that the dimension of \(\quotient{E}{E_1} \) is exactly \(1\) since \(\quotient{E}{E_1} \) is the set of constant functions.
\end{explanation}

\begin{definition}[Linear operator]\label{def:linear-op}
	A map \(T\colon E\to F\) between \hyperref[def:linear-vector-space]{linear spaces} \(E\) and \(F\) is a \emph{linear operator} if it preserves the properties of addition and multiplication by a scalar, i.e., for \(v, w\in E\) and \(\lambda , \mu \in \mathbb{\MakeUppercase{r}}\) or \(\mathbb{\MakeUppercase{c}}\),
	\[
		T(\lambda v + \mu w) = \lambda T(v) + \mu T(w).
	\]
\end{definition}

\begin{definition*}
	Given a \hyperref[def:linear-op]{linear operator} \(T\colon E \to F\) we have the following.
	\begin{definition}[Kernel]
		The \emph{kernel} of \(T\) is the subspace \(\ker(T) = \left\{ x\in E\mid Tx=0 \right\} \).
	\end{definition}

	\begin{definition}[Image]
		The \emph{image} of \(T\) is the subspace \(\im(T) = \left\{ Tx\in F\mid x\in E \right\} \).
	\end{definition}
\end{definition*}

\section{Normed Spaces}
Given a vector, we want to measure the length of which. This suggests the following definitions.

\begin{definition}[Norm]\label{def:norm}
	Let \(E\) be a \hyperref[def:linear-vector-space]{linear vector space}. A \emph{norm} \(\left\lVert \cdot \right\rVert \colon E \to \mathbb{\MakeUppercase{r}} \) on \(E\) is a function from \(E\) to \(\mathbb{\MakeUppercase{r}} \) with the properties:
	\begin{enumerate}[(a)]
		\item \(\left\lVert x\right\rVert \geq 0\) and \(\left\lVert x\right\rVert =0 \iff x=0\).
		\item \(\left\lVert \lambda x\right\rVert = \left\vert \lambda  \right\vert \left\lVert x\right\rVert\), \(\lambda \in\mathbb{\MakeUppercase{r}} \) or \(\mathbb{\MakeUppercase{c}} \).
		\item \(\left\lVert x+y\right\rVert \leq \left\lVert x\right\rVert + \left\lVert y\right\rVert \).
	\end{enumerate}
\end{definition}

\begin{notation}[Dilation]
	We say that the second condition is the \emph{dilation} property.
\end{notation}

\begin{definition}[Normed vector space]\label{def:normed-vector-space}
	A \hyperref[def:linear-vector-space]{linear vector space} \(E\) equipped with a \hyperref[def:norm]{norm} \(\left\lVert \cdot\right\rVert \) is called a \emph{normed vector space}, denoted by \((E, \left\lVert \cdot\right\rVert )\).
\end{definition}

A similar notion called \hyperref[def:metric]{metric} is also widely used, though the structure is slightly coarser.

\begin{prev}[Metric]\label{def:metric}
	Given a \hyperref[def:linear-vector-space]{vector space} \(E\), the \emph{metric} \(d(\cdot, \cdot)\colon E\times E\to \mathbb{\MakeUppercase{r}} \) on \(E\) is a function form \(E\times E\) to \(\mathbb{\MakeUppercase{r}} \) with the properties:
	\begin{enumerate}[(a)]
		\item \(d(x, y) \geq 0\). Also, \(d(x, x) = 0\) and \(d(x, y)\) implies \(x =y\).
		\item \(d(x, y) = d(y, x)\).
		\item \(d(x, z) \leq d(x, y) + d(y, z)\).
	\end{enumerate}
\end{prev}

As one can imagine, if we can measure the length of a vector (by a \hyperref[def:norm]{norm}), we can also measure the distance between vectors (by a \hyperref[def:metric]{metric}).

\begin{remark}[Induced metric space]
	A \hyperref[def:normed-vector-space]{normed vector space} \((E, \left\lVert \cdot\right\rVert )\) induces a \emph{metric space} \((E, d)\) with the induced \hyperref[def:metric]{metric} \(d(x, y) = \left\lVert x- y\right\rVert \).
\end{remark}

Now we give some well-known examples of \hyperref[def:normed-vector-space]{normed spaces}.

\begin{eg}[Bounded sequences \(\ell^\infty\)]
	Let \(\ell _\infty \) be the space of bounded sequences \(x = (x_1, x_2, \ldots )\) with \(x_i\in \mathbb{\MakeUppercase{r}} \) for \(i = 1, 2, \ldots \). Then we define \(\left\lVert x\right\rVert = \left\lVert x\right\rVert _\infty = \sup _{i \geq 1}\left\vert x_i \right\vert \).
\end{eg}

\begin{eg}[Absolutely summable sequences \(\ell _1\)]
	Let \(\ell _1\) be the space of absolutely summable sequences \(x = (x_1, x_2, \ldots)\) and \(\sum_{i=1}^{\infty} \left\vert x_i \right\vert < \infty\). Then we define \(\left\lVert x\right\rVert = \left\lVert x\right\rVert _1 = \sum_{i=1}^{\infty} \left\vert x_i \right\vert < \infty\).
\end{eg}

\begin{eg}[Continuous functions \(C(k)\)]
	The space \(C(k)\) of continuous functions \(f\colon K\to \mathbb{\MakeUppercase{r}} \) where \(K\) is compact Hausdorff. Then we define \(\left\lVert f\right\rVert = \left\lVert f\right\rVert _\infty = \sup _{x\in K}\left\vert f(x) \right\vert \).
\end{eg}

\subsection{Geometry of Normed Spaces}
Now we can look into the structure of a \hyperref[def:normed-vector-space]{normed space} we're referring to without actually explaining what this really means previously. Intuitively, it's about the geometric properties of the spaces like how do \hyperref[def:ball]{balls}, \hyperref[def:sphere]{spheres} and other shapes look like in that space when defining these shapes with \autoref{def:norm}.

\begin{definition}[Ball]\label{def:ball}
	A (closed) \emph{ball} centered at a point \(x_0\in E\) with radius \(r>0\) is the set \(B(x_0, r) = \left\{ x\in E\mid \left\lVert x - x_0\right\rVert \leq r \right\} \).
\end{definition}

\begin{definition}[Sphere]\label{def:sphere}
	The \emph{sphere} centered at \(x_0\) with radius \(r>0\) is the set \(S(x_0, r) = \left\{ x\in E\mid \left\lVert x - x_0\right\rVert = r\right\} \).
\end{definition}

\begin{note}
	We see that \(S(x_0, r)\) is the \textbf{boundary} of \(B(x_0, r)\), i.e., \(S(x_0, r) = \partial B(x_0, r)\).
\end{note}

Let's first look at \hyperref[def:ball]{balls}. In finite dimensional, all \hyperref[def:norm]{norms} are equivalent, which is not true for infinite dimensional \hyperref[def:normed-vector-space]{vector spaces}. This has something to do with the geometry of \hyperref[def:ball]{balls}.

Explicitly, \hyperref[def:ball]{balls} can have different geometries depending on the properties of the \hyperref[def:norm]{norms}. We see that a \(\left\lVert \cdot\right\rVert _{\infty}\) can have multiple supporting \hyperref[def:hyperplane]{hyperplane} at the corner, while for a \(\left\lVert \cdot\right\rVert _2\) can have only one at each point.
\begin{remark}
	The unit \hyperref[def:ball]{balls} for \(\left\lVert \cdot\right\rVert _1\) looks like \textbf{squares}, where we have
	\[
		B(0, 1) = \left\{ x = (x_1, x_2, \ldots)\mid -1 < y_{\epsilon } < 1 \text{ for all } \epsilon\right\}
	\]
	such that \(y_{\epsilon } = \sum_{i=1}^{\infty} \epsilon _i x_i \), \(\epsilon _i = \pm 1\) and \(\epsilon = (\epsilon _1, \epsilon _2, \ldots  )\).
\end{remark}

We see that different \hyperref[def:norm]{norms} give different geometry, but they have important common features, most notably, \hyperref[def:convex-function]{convexity} properties.

\begin{definition}[Convex set]\label{def:convex-set}
	Given \(E\) a \hyperref[def:linear-vector-space]{linear vector space}, a set \(K\subset E\) is \emph{convex} if for \(x, y\in K\) and \(0 \leq \lambda \leq 1\),
	\[
		\lambda x + (1 - \lambda )y\in K.
	\]
\end{definition}

\begin{definition}[Convex function]\label{def:convex-function}
	Given \(E\) a \hyperref[def:linear-vector-space]{linear vector space}, a function \(f\colon E\to \mathbb{\MakeUppercase{r}} \) is called \emph{convex} if for \(x, y\in E\) and \(0 \leq \lambda \leq 1\),
	\[
		f(\lambda x + (1 - \lambda )y) \leq \lambda f(x) + (1 - \lambda )f(y).
	\]
\end{definition}

\begin{remark}[Sublevel set]
	If \(f\colon E \to \mathbb{\MakeUppercase{r}} \) is a \hyperref[def:convex-function]{convex function}, then for any \(M\in \mathbb{\MakeUppercase{r}} \) the \emph{sublevel set} \(\left\{ x\in E\mid f(x) \leq M \right\} \) is \hyperref[def:convex-set]{convex}.
\end{remark}

The upshot is that \hyperref[def:norm]{norms} are \hyperref[def:convex-function]{convex}, and the unit \hyperref[def:ball]{balls} are \hyperref[def:convex-set]{convex} as well.
