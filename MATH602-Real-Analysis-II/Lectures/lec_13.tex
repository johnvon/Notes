\lecture{13}{11 Oct. 14:30}{Principle of Uniform Boundedness}
\section{Principle of Uniform Boundedness}
The final consequence of \hyperref[thm:open-mapping]{open mapping theorem} is the following, which completes the whole picture of functional analysis. We first see some definitions.

\begin{definition*}
	Let \(X, Y\) be \hyperref[def:Banach-space]{Banach spaces} and let \(\mathcal{\MakeUppercase{T}} \subseteq \mathcal{\MakeUppercase{l}} (X, Y)\) be a family of \hyperref[def:bounded-linear-op]{bounded linear operator} from \(X\) to \(Y\).
	\begin{definition}[Point-wise bounded]\label{def:point-wise-bounded}
		\(\mathcal{\MakeUppercase{t}} \) is \emph{point-wise bounded} if \(\sup _{T\in \mathcal{\MakeUppercase{t}} } \left\lVert Tx\right\rVert < \infty\) for all \(x\in X\).
	\end{definition}
	\begin{definition}[Uniformly bounded]\label{def:uniformly-bounded}
		\(\mathcal{\MakeUppercase{t}} \) is \emph{uniformly bounded} if \(\sup _{T\in \mathcal{\MakeUppercase{t}} }\left\lVert T\right\rVert < \infty \).
	\end{definition}
\end{definition*}

\begin{theorem}[Uniform boundedness theorem]\label{thm:uniform-boundedness}
	Let \(X, Y\) be \hyperref[def:Banach-space]{Banach spaces} and let \(\mathcal{\MakeUppercase{T}} \subseteq \mathcal{\MakeUppercase{l}} (X, Y)\) be a family of \hyperref[def:bounded-linear-op]{bounded linear operator} from \(X\) to \(Y\) such that it's \hyperref[def:point-wise-bounded]{point-wise bounded}, then it's \hyperref[def:uniformly-bounded]{uniformly bounded}.
\end{theorem}
\begin{proof}
	Define \(M\colon X\to \mathbb{\MakeUppercase{r}} \) by \(M(x) = \sup _{T\in \mathcal{\MakeUppercase{t}} }\left\lVert Tx\right\rVert \) for \(x\in X\), also, let \(X_n \coloneqq \left\{ x\in X\colon M(x) \leq n \right\}\), we can then write \(X = \bigcup_{n=1}^{\infty} X_n\). From \hyperref[prop:Baire-category]{Baire category theorem}, there exists \(n \geq 1\) such that \(\overline{X} _n\) has non-empty interior.
	\begin{claim}
		\(X_n\) is closed.
	\end{claim}
	\begin{explanation}
		Note that the function \(x \mapsto M(x)\) for \(x\in X\) is lower semi-continuous, i.e., \(M(x) \leq \liminf_{x_n \to x} M(x_n)\) since
		\[
			\left\lVert Tx\right\rVert \leq \lim\limits_{n \to \infty} \left\lVert Tx_n\right\rVert \leq \liminf_{n \to \infty} M(x_n),
		\]
		and by taking supremum over \(x\), we have \(M(x) \leq \liminf_{n \to \infty} M(x_n)\). Hence, we see that \(X_n\) is closed, i.e., \(\overline{X} _n = X_n\).
	\end{explanation}

	Hence, \(X_n\) itself has non-empty interior, so \(X_n \supseteq x_{0} + \epsilon B_X\) for some \(\epsilon > 0\) and \(B_X\coloneqq \left\{ x\in X\colon \left\lVert x\right\rVert \leq 1 \right\} \).

	Since \(M(\cdot)\) is symmetric and \hyperref[def:convex-function]{convex}, i.e., \(M(-x) = M(x)\) for \(x\in X\) and \(M(\lambda x + (1 - \lambda )y) \leq \lambda M(x) + (1 - \lambda )M(y)\) for \(x, y\in X\), \(0 < \lambda < 1\), we see that \(X_n \supseteq x_0 + \epsilon B_X\). From symmetric, we also have \(X_n \supseteq -x_0 + \epsilon B_X\). Then by \hyperref[def:convex-function]{convexity}, we together have \(X_n \supseteq \epsilon B_X\), hence
	\[
		\left\lVert x\right\rVert \leq \epsilon \implies \sup _{T\in \mathcal{\MakeUppercase{t}} } \left\lVert Tx\right\rVert \leq n\implies \sup _{T\in \mathcal{\MakeUppercase{t}} } \left\lVert T\right\rVert \leq n / \epsilon  < \infty.
	\]
\end{proof}

\begin{remark}[Completeness]
	\hyperref[thm:uniform-boundedness]{Uniform boundedness theorem} still holds if \(X\) is a \hyperref[def:Banach-space]{Banach space} while \(Y\) is only a \hyperref[def:normed-vector-space]{normed space}.
\end{remark}
\begin{explanation}
	In the above proof, we only use the completeness of \(X\), not \(Y\).
\end{explanation}

\begin{note}[Principle of condensation of singularities]
	The \hyperref[thm:uniform-boundedness]{uniform Boundedness theorem} is called \emph{principle of condensation of singularities} by Banach and Steinhaus initially.
\end{note}
\begin{explanation}
	Suppose a family \(\mathcal{\MakeUppercase{t}} \subseteq \mathcal{\MakeUppercase{l}} (X, Y)\) is not \hyperref[def:uniformly-bounded]{uniformly bounded}, then the set of vectors
	\[
		\left\{ Tx\colon x\in B_X, T\in \mathcal{\MakeUppercase{t}}  \right\}
	\]
	is unbounded. We see that from the \hyperref[thm:uniform-boundedness]{uniform boundedness theorem} is not even \hyperref[def:point-wise-bounded]{point-wise bounded}, so there exists \emph{one} vector \(x\in X\) with unbounded trajectory \(\left\{ Tx\colon T\in \mathcal{\MakeUppercase{t}}  \right\} \). One can say that the unboundedness of the family \(\mathcal{\MakeUppercase{t}} \) is condensated in a single \emph{singularity} vector \(x\).
\end{explanation}

\subsection{Weak and Strong Boundedness}
\hyperref[thm:uniform-boundedness]{Principle of uniform boundedness} can be used to check whether a given set in a \hyperref[def:Banach-space]{Banach space} is bounded in the following way. Firstly, let's see some definitions.

\begin{definition*}
	Let \(A \subseteq X\) where \(X\) is a \hyperref[def:Banach-space]{Banach space}.
	\begin{definition}[Weakly bounded]\label{def:weakly-bounded}
		\(A\) is \emph{weakly bounded} if \(\sup _{f\in X^{\ast} }\left\vert f(x) \right\vert < \infty\) for all \(x\in A\).
	\end{definition}
	\begin{definition}[Strongly bounded]\label{def:strongly-bounded}
		\(A\) is \emph{strongly bounded} if \(\sup _{x\in A}\left\lVert x\right\rVert < \infty\).
	\end{definition}
\end{definition*}

Clearly, \hyperref[def:strongly-bounded]{strong boundedness} trivially implies \hyperref[def:weakly-bounded]{weak boundedness}. Indeed, the converse is also true.

\begin{corollary}[Weak boundedness implies strong boundedness]\label{col:weak-bd-implies-strong-bd}
	Let \(A \subseteq X\) for \(X\) being a \hyperref[def:Banach-space]{Banach space}. If \(A\) is \hyperref[def:weakly-bounded]{weakly bounded}, then \(A\) is \hyperref[def:strongly-bounded]{strongly bounded}.
\end{corollary}
\begin{proof}
	We embed \(A\) into \(A^{\ast\ast} \subseteq X^{\ast\ast} \) by considering the conical embedding \(X\to X^{\ast\ast}\), and we see that
	\[
		\sup _{x^{\ast\ast}\in A^{\ast\ast}} \left\vert x^{\ast\ast} (f) \right\vert < \infty
	\]
	for all \(f\in X^{\ast} \). From the \hyperref[thm:uniform-boundedness]{uniform boundedness theorem}, we have \(\sup _{x^{\ast\ast} \in A^{\ast\ast} }\left\lVert x^{\ast\ast} \right\rVert < \infty \), and with \hyperref[thm:Hahn-Banach]{Hahn-Banach theorem}, we have \(\left\lVert x^{\ast\ast} \right\rVert = \left\lVert x\right\rVert \) for all \(x\in X\), proving the result.
\end{proof}

We now review the midterm in \autoref{sec:mid-review}.