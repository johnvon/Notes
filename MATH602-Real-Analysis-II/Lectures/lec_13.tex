\lecture{13}{11 Oct. 14:30}{Open Mapping}
\subsection{Principle of Uniform Boundedness}
The final consequence of \autoref{thm:open-mapping} is the following.
\begin{proposition}[Uniform boundedness theorem]\label{prop:uniform-boundedness}
	Let \(X, Y\) be \hyperref[def:Banach-space]{Banach spaces} and let \(\mathcal{\MakeUppercase{T}} \subseteq \mathcal{\MakeUppercase{l}} (X, Y)\) be a family of \hyperref[def:bounded-linear-op]{bounded linear operator} from \(X\) to \(Y\) such that \(\sup _{T\in \mathcal{\MakeUppercase{t}} } \left\lVert Tx\right\rVert < \infty\) for all \(x\in X\). Then \(\sup _{T\in \mathcal{\MakeUppercase{t}} }\left\lVert T\right\rVert < \infty \).
\end{proposition}
\begin{proof}
	Define \(M\colon X\to \mathbb{\MakeUppercase{r}} \) by \(M(x) = \sup _{T\in \mathcal{\MakeUppercase{t}} }\left\lVert Tx\right\rVert \) for \(x\in X\). Then
	\[
		X = \bigcup\limits_{n=1}^{\infty} X_n,\quad X_n \coloneqq \left\{ x\in X\colon M(x) \leq n \right\}.
	\]
	From \autoref{prop:Baire-category}, there exists \(n \geq 1\) such that \(\overline{X} _n\) has non-empty interior. Note that the function \(x \mapsto M(x)\) for \(x\in X\) is lower semi-continuous, i.e.,
	\[
		M(x) \leq \liminf_{x_n \to x} M(x_n)
	\]
	since
	\[
		\left\lVert Tx\right\rVert \leq \lim\limits_{n \to \infty} \left\lVert Tx_n\right\rVert \leq \liminf_{n \to \infty} M(x_n),
	\]
	and by taking supremum over \(x\), we have \(M(x) \leq \liminf_{n \to \infty} M(x_n)\). Hence, we see that \(X_n = \left\{ x\in X\colon M(x)\leq n \right\} \) is closed, i.e., \(\overline{X} _n = X_n\), and we conclude \(X_n\) has non-empty interior. This implies \(X_n \supseteq x_{0} + \epsilon B_X\) for some \(\epsilon > 0\) and \(B_X\coloneqq \left\{ x\in X\colon \left\lVert x\right\rVert \leq 1 \right\} \). And since \(M(\cdot)\) is symmetric and \hyperref[def:convex-function]{convex}, i.e., \(M(-x) = M(x)\) for \(x\in X\) and
	\[
		M(\lambda x + (1 - \lambda )y) \leq \lambda M(x) + (1 - \lambda )M(y)
	\]
	for \(x, y\in X\), \(0 < \lambda < 1\), we see that \(X_n \supseteq x_0 + \epsilon B_X\). From symmetric, we also have \(X_n \supseteq -x_0 + \epsilon B_X\). Then by \hyperref[def:convex-function]{convexity}, we together have \(X_n \supseteq \epsilon B_X\), hence
	\[
		\left\lVert x\right\rVert \leq \epsilon \implies \sup _{T\in \mathcal{\MakeUppercase{t}} } \left\lVert Tx\right\rVert \leq n\implies \sup _{T\in \mathcal{\MakeUppercase{t}} } \left\lVert T\right\rVert \leq \frac{n}{\epsilon }.
	\]
\end{proof}

\begin{definition}[Weakly bounded]\label{def:weakly-bounded}
	Let \(A \subseteq X\), we say \(A\) is \emph{weakly bounded} if \(\sup _{f\in X^{\ast} }\left\vert f(x) \right\vert < \infty\) for all \(x\in A\).
\end{definition}

\begin{corollary}[Weak boundedness implies strong boundedness]\label{col:weak-bd-implies-strong-bd}
	Let \(A \subseteq X\) and suppose \(A\) is \hyperref[def:weakly-bounded]{weakly bounded}, then \(A\) is strongly bounded, i.e., \(\sup _{x\in A} \left\lVert x\right\rVert < \infty \).
\end{corollary}
\begin{proof}
	Firstly, we embed \(A\) into \(A^{\ast\ast} \subseteq X^{\ast\ast} \) by considering the conical embedding \(X\to X^{\ast\ast}\), and we see that
	\[
		\sup _{x^{\ast\ast}\in A^{\ast\ast}} \left\vert x^{\ast\ast} (f) \right\vert < \infty
	\]
	for all \(f\in X^{\ast} \). From \autoref{prop:uniform-boundedness}, we have \(\sup _{x^{\ast\ast} \in A^{\ast\ast} }\left\lVert x^{\ast\ast} \right\rVert < \infty \), and with \autoref{thm:Hahn-Banach}, we have \(\left\lVert x^{\ast\ast} \right\rVert = \left\lVert x\right\rVert \) for all \(x\in X\), proving the result.
\end{proof}