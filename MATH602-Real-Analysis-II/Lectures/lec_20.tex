\lecture{20}{8 Nov. 14:30}{Spectrum Theory}
\hyperref[thm:Fredholm-alternative]{Fredholm alternative} does not hold if \(T\) is not \hyperref[def:compact-op]{compact}.
\begin{eg}[Shift operator]
	Consider the shift operators on \(\ell _2 = \left\{ (a_1, a_2, \ldots ), a_j\in \mathbb{\MakeUppercase{r}}, \sum_{j=1}^{\infty}  a_j^2 < \infty \right\} \). The right shift \(T_r\) is defined as
	\[
		T_r(a_1, a_2, \ldots) = (0, a_1, a_2, \ldots ),
	\]
	which is injective but not surjective; while the left shift \(T_l\) is defined as
	\[
		T_l(a_1, a_2, \ldots ) = (a_2, a_3, \ldots ),
	\]
	which is surjective but not injective.
\end{eg}

\section{Spectrum of Bounded Linear Operators}

\begin{definition*}
	Let \(X\) be a complex \hyperref[def:Banach-space]{Banach space}, and \(T\colon X \to X\) be a \hyperref[def:bounded-linear-op]{bounded linear operator}.
	\begin{definition}[Regular point]\label{def:regular-point}
		A number \(\lambda \in \mathbb{\MakeUppercase{c}} \) is called a \emph{regular point} of \(T\) if the operator \(T-\lambda I\) is invertible, i.e., \((T-\lambda I)^{-1} \in \mathcal{\MakeUppercase{l}} (X, X)\).
	\end{definition}
	\begin{definition}[Spectrum point]\label{def:spectrum-point}
		A number \(\lambda \in \mathbb{\MakeUppercase{c}} \) is called a \emph{spectrum point} of \(T\) if it's not a \hyperref[def:regular-point]{regular point}.
	\end{definition}
\end{definition*}

\begin{notation}
	The set of \hyperref[def:regular-point]{regular points} for \(T\) is denoted as \(\rho (T)\), while the set of \hyperref[def:spectrum-point]{spectrum points} is denoted as \(\sigma (T)\).
\end{notation}

\begin{remark}[Resolvent point]
	We sometimes called a \hyperref[def:regular-point]{regular point} as a \emph{resolvent point}.
\end{remark}

From definitions, we know that \(\sigma (T) = \mathbb{\MakeUppercase{c}} - \rho (T)\).

\begin{definition*}[Classification of spectrum]
	Let \(T\) be a \hyperref[def:bounded-linear-op]{bounded linear operator} on \(X\) with \(\sigma (T)\).
	\begin{definition}[Point spectrum]\label{def:point-spectrum}
		The \emph{point spectrum} \(\sigma _p(T)\) contains \(\lambda \) such that \(\ker(T - \lambda I) \neq \left\{ 0 \right\} \), i.e., \(T - \lambda I\) is not injective, i.e., \(\lambda \in \sigma _p(T)\) is \(\lambda \) is an eigenvalue of \(T\).
	\end{definition}
	\begin{definition}[Continuous spectrum]\label{def:continuous-spectrum}
		The \emph{continuous spectrums} \(\sigma _c(T)\) contains \(\lambda \) such that \(\ker(T - \lambda I) = \left\{ 0 \right\} \), i.e., \(\lambda \notin \sigma _p(T)\) and \(\im(T-\lambda I)\) is dense in \(X\), i.e., \(\lambda \in \sigma _c(T)\) if \(\lambda \) is not an eigenvalue, \(\im(T-\lambda I)\neq X\) but \(\overline{\im(T-\lambda I)}=X\).\footnote{Note that by \hyperref[thm:open-mapping]{open mapping theorem}, if \(\im(T - \lambda I)=X\), then \(\lambda \in \rho (T)\).}
	\end{definition}
	\begin{definition}[Residual spectrum]\label{def:residual-spectrum}
		The \emph{residual spectrum} \(\sigma _r(T)\) is defined as \(\sigma _r(T)\coloneqq \sigma (T) - \sigma _p(T) - \sigma _c(T)\), i.e., \(\lambda \in \sigma _r(T)\) if \(\lambda \) is not an eigenvalue, i.e., \(T-\lambda I\) is injective and \(\overline{\im(T-\lambda I)}\neq X\).
	\end{definition}
\end{definition*}

\begin{eg}[Diagonal operator]
	Let \(\left\{ \lambda _k \right\} _{k \geq 1}\) be a sequence in \(\mathbb{\MakeUppercase{c}} \) such that \(\lim_{k \to \infty} \lambda _k = 0\). Define \(T\colon \ell _2 \to \ell _2\) by
	\[
		T(\left\{ x_k \right\} _{k\geq 1})= \left\{ \lambda_k x_k \right\} _{k\geq 1}
	\]
	where the sequence \(\left\{ \lambda _k \right\} _{k\geq 1}\) is bounded. Hence, \(T\) is a \hyperref[def:bounded-linear-op]{bounded linear operator}. Then, \((T-\lambda I) x = \left\{ (\lambda _k - \lambda )x_k \right\} _{k\geq 1}\), so given \(y = \left\{ y_k \right\} _{k\geq 1}\),
	\[
		(T-\lambda I)^{-1} y = \left\{ \frac{y_k}{\lambda _k - \lambda } \right\} _{k\geq 1},
	\]
	which implies \((T-\lambda I)^{-1} \) is \hyperref[rmk:bounded-op]{bounded} on \(\ell _2\) if \(\sup _{k\geq 1} \vert \lambda _k - \lambda \vert ^{-1} < \infty \). Indeed, since \(\lim_{k \to \infty} \lambda _k = 0\),
	\[
		\sup _{k\geq 1} \frac{1}{\vert \lambda _k - \lambda  \vert }< \infty
	\]
	if \(\lambda \notin \left\{ \lambda _k \right\}_{k\geq 1} \cup \left\{ 0 \right\}\). Further, since
	\begin{itemize}
		\item if \(\lambda =\lambda _k\) then \(\ker(T - \lambda I) \neq \left\{ 0 \right\} \);
		\item if \(\lambda=0\), then \(\im(T - \lambda I)\) is dense in \(\ell _2\) but \(\ker(T-\lambda I) = \left\{ 0 \right\} \).
	\end{itemize}

	So \(\im(T)\) contains all sequences \(x = \left\{ x_k \right\} _{k\geq 1}\) with only finite number of \(k\) such that \(x_k \neq 0\), i.e., \(\im(T)\) is dense in \(\ell _2\). Hence, \(\sigma (T) = \left\{ \lambda _k \right\}_{k\geq 1} \cup \left\{ 0 \right\}\) with \(\sigma _p(T) = \left\{ \lambda _k \right\}_{k\geq 1} \), \(\sigma _c(T) = \left\{ 0 \right\} \) with \(\sigma _r(T) = \varnothing \).
\end{eg}

\begin{eg}[Multiplication operator]
	Consider the multiplication on \(L^2([0, 1])\) such that \(Tf(t) = tf(t)\) for \(0 < t < 1\). Then
	\[
		(T-\lambda I)f(t) = (t - \lambda )f(t) \implies (T-\lambda I)^{-1} g(t) = \frac{g(t)}{t - \lambda }
	\]
	for \(0 < t < 1\). Hence, \(T-\lambda I\) is invertible if \(\lambda \notin [0, 1]\). And if \(\lambda \in [0, 1]\), then \(\ker(T - \lambda I) = 0\), i.e., \(\sigma _p(T)=\varnothing \). Lastly, since \(\im(T-\lambda I)\) is dense in \(L^2([0, 1])\) if \(\lambda \in [0, 1]\), hence \(\sigma _c(T) = [0, 1]\) and \(\sigma _r(T)\) is empty.
\end{eg}

\begin{definition}[Resolvent operator]\label{def:resolvent-op}
	To each \hyperref[def:regular-point]{regular point} \(\lambda \in \rho (T)\) for \(T\in \mathcal{\MakeUppercase{l}} (X, X)\), the associated \emph{resolvent operator} \(R(\lambda )\colon \rho (T) \to \mathcal{\MakeUppercase{l}} (X, X)\) is defined as \((T-\lambda I)^{-1} \).
\end{definition}

\begin{lemma}
	Let \(S\in \mathcal{\MakeUppercase{l}} (X, X)\) such that \(\lVert S \rVert < 1\), then \(I-S\) is invertible and \((I-S)^{-1} \) is given by the geometric series
	\[
		(I-S)^{-1} = \sum_{j=0} ^{\infty} S^j \text{ and }\lVert (I-S)^{-1} \rVert \leq \frac{1}{1 - \lVert S \rVert} .
	\]
\end{lemma}
\begin{proof}
	From the inequality \(\lVert S^j \rVert \leq \lVert S \rVert ^j\) for all \(j\geq 1\).
\end{proof}

\begin{proposition}\label{prop:lec20-1}
	The \hyperref[def:regular-point]{resolvent} set \(\rho (T)\subseteq \mathbb{\MakeUppercase{c}} \) is open and contains the disk \(\left\{ \lambda \in \mathbb{\MakeUppercase{c}} \colon \vert \lambda  \vert > \lVert T \rVert  \right\} \), and \(\lVert R(\lambda ) \rVert \leq 1 / (\vert \lambda  \vert - \lVert T \rVert )\).
\end{proposition}
\begin{proof}
	Since
	\[
		(T-\lambda I)^{-1}
		= - \lambda ^{-1} (I-\lambda ^{-1} T)^{-1}
		= - \lambda ^{-1} (I-S),
	\]
	by letting \(S = T / \lambda \), we have \(\lVert S \rVert < 1\) if \(\vert \lambda \vert > \lVert T \rVert \), hence
	\[
		\lVert (T-\lambda I)^{-1} \rVert
		\leq \frac{1}{\vert \lambda  \vert } \frac{1}{1 - \lVert S \rVert }
		= \frac{1}{\vert \lambda  \vert } \frac{1}{1 - \vert \lambda  \vert^{-1} \lVert T \rVert  }
		= \frac{1}{\vert \lambda  \vert - \lVert T \rVert },
	\]
	i.e., if \(\vert \lambda  \vert > \lVert T \rVert \), then \(\lambda \in \rho (T)\). We now show \(\rho (T)\) is open. Since
	\[
		\frac{1}{x - \lambda } - \frac{1}{x-\mu } = \frac{\lambda -\mu }{(x-\lambda )(x-\mu )}
	\]
	for all \(\mu , \lambda \in \mathbb{\MakeUppercase{c}} \) and \(x\in \mathbb{\MakeUppercase{c}} \), we can generalize this to
	\[
		R(\lambda ) - R(\mu ) = (\lambda -\mu )R(\lambda )R(\mu )
	\]
	since \(R(\lambda ), R(\mu )\) commutes. Hence,
	\[
		R(\mu )
		= \left[ I-(\mu -\lambda )R(\lambda ) \right] ^{-1} R(\lambda )
		= (I-S)^{-1} R(\lambda ),
	\]
	so \(R(\mu )\) is \hyperref[rmk:bounded-op]{bounded} if \(\lVert S \rVert < 1\), i.e., \(\vert \mu -\lambda  \vert \lVert R(\lambda ) \rVert < 1\). We see that \(\lambda \in \rho (T)\) implies that the disk \(D(\lambda , r) \subseteq \rho (T)\) if \(r \lVert R(\lambda ) \rVert < 1\), i.e., \(\rho (T)\) is open. We have shown that \(\sigma (T)\) is a closed set and \(\sigma (T)\) is bounded with \(\sigma (T) \subseteq \overline{D(0, \lVert T \rVert )}\).
\end{proof}

\begin{definition}[Spectral radius]\label{def:spectral-radius}
	The \emph{spectral radius} of an operator \(T\in \mathcal{\MakeUppercase{l}} (X, X)\) is defined as
	\[
		r(T) \coloneqq \max _{\lambda\in \sigma (T) }\vert \lambda \vert .
	\]
\end{definition}

From \autoref{prop:lec20-1}, \(r(T) \leq \lVert T \rVert \), but actually the equality can be achieved.

\begin{proposition}
	Let \(T\) be a \hyperref[def:bounded-linear-op]{bounded linear operator} on \hyperref[def:Banach-space]{Banach space} \(X\) with \(r(T)\). Then
	\[
		r(T) = \lim_{n \to \infty} \lVert T^n \rVert ^{1 / n} = \inf _{n\geq 1}\lVert T^n \rVert ^{1 / n}.
	\]
\end{proposition}
\begin{proof}
	Let \(r\) be a large integer and \(m\geq 1\) an integer. Then \(r = am+b\) for \(a \geq 0\), \(0 \leq b < m\). Then
	\[
		\lVert T^r \rVert
		= \lVert T^{am} T^b\rVert
		\leq \lVert T^m \rVert ^a \lVert T^b \rVert,
	\]
	hence
	\[
		\lVert T^r \rVert ^{1 / r}
		\leq \lVert T^m \rVert ^{1 / r} \lVert T^b \rVert ^{1 / r}
		= \left( \lVert T^m \rVert ^{1 / m} \right) ^{\frac{a}{a+ b / m}} \lVert T^b \rVert ^{1 / r}.
	\]
	Let \(r \to \infty \), we have
	\[
		\limsup_{r \to \infty} \lVert T^r \rVert ^{1 / r}
		\leq \lVert T^m \rVert ^{1 / m}
	\]
	for all \(m \geq 1\), concluding that
	\[
		\limsup_{r \to \infty} \lVert T^r \rVert ^{1 / r}
		= \liminf_{r \to \infty} \lVert T^r \rVert ^{1 / r}
		= \inf _{m\geq 1}\lVert T^m \rVert ^{1 / m}.
	\]
\end{proof}

Now, the next goal is to show that \(\sigma (T)\) is nonempty.