\lecture{19}{3 Nov. 14:30}{Fredholm Alternative}
\section{Fredholm Theory}
Fredholm theory studies operators of the form \emph{identity plus \hyperref[def:compact-op]{compact}}. They are conveniently put in the form \(I - T\) for \(I\) being the identity operator on some \hyperref[def:Banach-space]{Banach space} \(X\) and \(T\in \mathcal{K}(X, X)\).

Furthermore, Fredholm theory is motivated by two applications. One is for solving linear equations \(\lambda x - Tx = b\), and in particular, integral equations.\footnote{\(T\) being an integral operator.} Another is in spectral theory, where the spectrum of \(T\) consists of numbers \(\lambda \) for which the operator \(\lambda I - T\) is invertible.
\subsection{Closed Image}
\begin{lemma}\label{lma:lec19}
	Let \(X\) be a \hyperref[def:Banach-space]{Banach space} and \(T\in \mathcal{K}(X, X)\), then the operator \(I-T\) has closed image \(\im(I-T)\).
\end{lemma}
\begin{proof}
	Let \(A=I-T\), since \(\ker (A)\) is closed, we consider the injectivization \(\widetilde{A} \colon \quotient{X}{\ker(A)}\to X\) be the induced operator \(\widetilde{A} ([x])=Ax\) for all \(x\in X\). Since \(\im(A) = \im(\widetilde{A} )\), it's sufficient to show \(\im(\widetilde{A} )\) is closed, which is equivalent to show that \(\widetilde{A} \) is bounded below from \autoref{prop:isomorphic-embedding}, i.e., \(\exists c>0\) such that
	\[
		\lVert \widetilde{A} [x]\rVert \geq c\left\lVert [x]\right\rVert
	\]
	for all \(x\in X\). Toward a contradiction, suppose \(\widetilde{A} \) is not bounded below, then there exists a sequence \(\left\{ x_k \right\} _{k\geq 1}\) in \(X\) such that \(\left\lVert [x_{k} ]\right\rVert = 1\) for all \(k\geq 1\) and \(\lVert \widetilde{A} [x_k] \rVert \to 0\) as \(k \to \infty \). We can then choose \(\left\{ x_{k}  \right\}_{k\geq 1} \) such that \(\left\lVert x_{k} \right\rVert \leq 2\), with \(\inf_{y\in \ker(A)} \left\lVert x_{k} -y\right\rVert=1 \). So as \(k\to \infty \),
	\[
		x_{k} - Tx_{k} = Ax_{k} = \widetilde{A} ([x_{k} ])\to 0.
	\]

	From the fact that \(T\) is \hyperref[def:compact-op]{compact}, \(\left\{ Tx_{k}  \right\}_{k\geq 1} \) has a converging subsequence, so we can assume \(Tx_{k} \to z\) as \(k\to \infty \), with \(x_{k} - Tx_{k} \to 0\), we have \(x_{k} \to z\) as \(k \to \infty \) and \(Az = 0\), i.e., \(z\in \ker(A)\). But from \(\inf _{y\in \ker(A)} \left\lVert z - y\right\rVert = 1\), a contradiction since this distance should be \(0\) if \(z\in \ker(A)\).
\end{proof}

\subsection{Fredholm Alternative}
We now study a partial case of the so-called \hyperref[thm:Fredholm-alternative]{Fredholm alternative}.

\begin{theorem}[Fredholm alternative]\label{thm:Fredholm-alternative}
	Let \(X\) be a \hyperref[def:Banach-space]{Banach space}, \(T\colon X\to  X\) be \hyperref[def:compact-op]{compact}. Then \(I - T\) is injective if and only if \(I - T\) is surjective.
\end{theorem}
\begin{proof}
	Assume that \(A\coloneqq I - T\) is injective but not surjective, to have a contradiction, we just need to find a sequence \(\left\{ f_{n}  \right\}_{n\geq 1} \) in \(X^{\ast} \) with \(\lVert f_{n}  \rVert = 1\) for all \(n\geq 1\) such that the sequence \(\left\{ T^{\ast} f_n \right\}_{n\geq 1} \) has no converging subsequence, since it will contradict the fact that \(T^{\ast} (B_{X^{\ast} })\) is \hyperref[def:precompact]{precompact}.

	\begin{claim}
		Let \(Y_n \coloneqq \im(A^n)\) for \(n\geq 1\), then \(Y_{n+ 1} \subsetneq Y_n\) for all \(n\).
	\end{claim}
	\begin{explanation}
		Since \(A\) is not surjective, \(\im(A) \neq X\), i.e., \(Y_1 \subsetneq Y_0 = X\). Consider \(y \notin \im(A)\), and suppose \(\im(A^{n+1})= \im(A^n)\), then there exists \(x\) such that \(A^{n+1}x = A^n y\), i.e., \(A^n (Ax - y) = 0\). From the injectivity of \(A\), \(\ker(A^n) = 0\), so \(Ax - y = 0\), implying \(y\in \im(A)\) \(\conta\) So \(Y_{n+1}\) is properly contained in \(Y_n\) for all \(n\geq 1\).
	\end{explanation}

	\begin{claim}
		\(Y_n\) are all closed.
	\end{claim}
	\begin{explanation}
		Since \(Y_n = \im(A^n) = \im((I-T)^n)\), where \((I-T)^n = I - ST = I - \widetilde{T}\) for \(S\) being \hyperref[def:bounded-map]{bounded} and \(T\) being \hyperref[def:compact-op]{compact}, hence \(\widetilde{T} = ST\) is \hyperref[def:compact-op]{compact} from \autoref{prop:compact-op}. The result follows from \autoref{lma:lec19}.
	\end{explanation}

	Now, since \(\quotient{Y_n}{Y_{n+1}} \) is a \hyperref[def:Banach-space]{Banach space}, let \(\widetilde{f} _n \colon \quotient{Y_n}{Y_{n+1}} \to  \mathbb{R} \) be a \hyperref[def:bounded-linear-functional]{bounded linear functional} with \(\lVert \widetilde{f} _n \rVert=1 \), which can be found by the \hyperref[thm:supporting-functional]{supporting functional theorem}. Define \(f_{n} \colon Y_n \to  \mathbb{R} \) by \(f_n (y) = \widetilde{f} _n([y])\) for \(y\in Y_n\) with \(f_n(y) = 0\) when \(y\in Y_{n+1}\), implying that \(f_n \in Y _{n+1}^{\perp}\) from the \hyperref[thm:Riesz-representation]{Riesz representation theorem}. Finally, we extend \(f_n\) to \(f_{n} \) to \(f_{n} \colon X\to \mathbb{R} \) with \(\lVert f_{n} \rVert = 1\) by \hyperref[thm:Hahn-Banach]{Hahn Banach theorem} to avoid any domain issue.

	We now show the sequence \(\left\{ T^{\ast} f_{n}  \right\} _{n\geq 1}\) has no converging subsequence by shown that for \(n > m \geq 1\), \(T^{\ast} f_n\) and \(T^{\ast} f_m\) are pairwise separated.

	\begin{claim}
		For \(n > m \geq 1\), \((T^{\ast} f_n - T^{\ast} f_m)(x) = f_n (x)\).
	\end{claim}
	\begin{explanation}
		We have
		\[
			T^{\ast} f_{n} - T^{\ast} f_m= T^{\ast} (f_{n} - f_{m}) = (I-T^{\ast} )(f_m-f_{n})+ (f_{n} -f_{m})
		\]
		where \(f_n\in Y^{\perp} _{n+1}\) and \(f_m\in Y^{\perp} _{m+1} \subseteq Y^{\perp} _{n+1}\), so \(f_n - f_m\in Y^{\perp} _{n+1}\). Now, observe that \((I-T)x = Ax\in Y_{n+1}\), hence \((I-T^{\ast} )(f_n - f_m) = (f_n - f_m)(I-T) = 0\). In all, we have that for \(x\in Y_n\), \(T^{\ast} (f_{n} - f_m)(x) = (f_n - f_m)(x)=f_n(x)\) from \(m < n\)
	\end{explanation}
	This implies that
	\[
		\lVert T^{\ast} f_n - T^{\ast} f_m \rVert
		= \sup _{\lVert x \rVert = 1} \lVert [T^{\ast} f_n - T^{\ast} f_m] (x)\rVert
		= \sup _{\lVert x \rVert = 1} \lVert f_n(x) \rVert
		= \lVert f_n \rVert
		= 1,
	\]
	i.e., all terms in the sequence \(\left\{ T^{\ast} f_n \right\} \) are pairwise separated as desired.

	Conversely, assume \(I-T\) is surjective, we want to prove that \(I-T\) is injective. Again, let \(A\coloneqq I-T\), we have \(\im(A) = X\) and \(\ker(A^{\ast} )= (\im A)^{\perp} \), implying \(\ker(A^{\ast} ) = \left\{ 0\right\} \), i.e., \(A^{\ast} \) is injective. Since \(A^{\ast} = I-T^{\ast} \), where \(T^{\ast} \) is \hyperref[def:compact-op]{compact}, the previous result implies \(A^{\ast} \) is surjective. Note that \((\ker A)^{\perp} \supseteq (\im A^{\ast} )= X^{\ast}\), hence \(\ker(A) = \left\{ 0 \right\} \), so \(A\) is injective.
\end{proof}

We see that the \hyperref[def:spectrum-point]{spectrum} in infinite dimension is much more complicated compared to the finite dimension case.