\lecture{18}{1 Nov. 14:30}{Hilbert-Schmidt Operators}
\begin{corollary}[Isomorphisms are not compact]\label{col:iso-are-not-compact}
	Let \(X\) be an infinite dimension \hyperref[def:normed-vector-space]{normed space}, then the identity map \(I\colon X\to X\) is not \hyperref[def:compact-op]{compact}. More generally, an invertible operator (i.e., isomorphism) \(T\in \mathcal{\MakeUppercase{l}} (X, Y)\) is not \hyperref[def:compact-op]{compact}.
\end{corollary}
\begin{proof}
	\(I\) is not \hyperref[def:compact-op]{compact} since \hyperref[thm:Riesz]{Riesz theorem} implies that \(B_X\) is not \hyperref[def:compact]{compact}. And in general, if \(T\in \mathcal{\MakeUppercase{l}} (X, Y)\) is \hyperref[def:compact-op]{compact} and invertible, \autoref{prop:compact-op} implies that \(T^{-1} T = I\colon X\to X\) is \hyperref[def:compact-op]{compact} since \(T^{-1} \) is bounded and \(T\) is \hyperref[def:compact-op]{compact}, contradiction.
\end{proof}

\subsection{Hilbert-Schmidt Operators}
This is a most frequently used class of \hyperref[def:compact-op]{compact operators} in \hyperref[def:Hilbert-space]{Hilbert spaces}. As we will see, it covers the class of important operators, i.e., integral operators, motivating us to study this class.

\begin{definition}[Hilbert-Schmidt operators]\label{def:Hilbert-Schmidt-op}
	Let \(\mathcal{\MakeUppercase{h}} \) be a \hyperref[def:Hilbert-space]{Hilbert space} which is \hyperref[def:separable]{separable} and \(\left\{ x_k \right\} _{k\geq 1}\) is an \hyperref[def:orthonormal-system]{orthonormal} basis in \(\mathcal{\MakeUppercase{h}} \). A \hyperref[def:linear-map]{linear operator} \(T\colon \mathcal{\MakeUppercase{h}} \to \mathcal{\MakeUppercase{h}} \) is \emph{Hilbert-Schmidt} if
	\[
		\sum_{k=1}^{\infty} \left\lVert T x_{k} \right\rVert ^{2} < \infty.
	\]
\end{definition}

From \autoref{def:Hilbert-Schmidt-op}, the following naturally induced \hyperref[def:norm]{norm} is defined.

\begin{definition}[Hilbert-Schmidt norm]\label{def:Hilbert-Schmidt-norm}
	The \emph{Hilbert-Schmidt norm} of a \hyperref[def:linear-map]{linear operator} \(T\) is
	\[
		\left\lVert T\right\rVert _{\mathrm{HS} } = \left( \sum_{k=1}^{\infty} \left\lVert T x_{k} \right\rVert ^2 \right) ^{1 / 2}.
	\]
\end{definition}

To characterize \hyperref[def:Hilbert-Schmidt-op]{Hilbert-Schmidt operators}, we have the following.

\begin{proposition}
	Let \(T\colon \mathcal{\MakeUppercase{h}} \to \mathcal{\MakeUppercase{h}} \) be a \hyperref[def:linear-op]{linear operator} on the \hyperref[def:Hilbert-space]{Hilbert space} \(\mathcal{\MakeUppercase{h}} \).
	\begin{enumerate}[(a)]
		\item The \hyperref[def:Hilbert-Schmidt-norm]{Hilbert-Schmidt norm} of \(T\) is independent of the choice of \hyperref[def:orthonormal-system]{orthonormal} basis.
		\item If \(T\) is \hyperref[def:Hilbert-Schmidt-op]{Hilbert-Schmidt}, then \(T^{\ast} \) is \hyperref[def:Hilbert-Schmidt-op]{Hilbert-Schmidt} and \(\left\lVert T\right\rVert _{\mathrm{HS} } = \left\lVert T^{\ast} \right\rVert _{\mathrm{HS} } \).
		\item If \(T\) is \hyperref[def:Hilbert-Schmidt-op]{Hilbert-Schmidt}, then \(T\) is \hyperref[def:bounded-linear-op]{bounded} on \(\mathcal{\MakeUppercase{h}} \) and \(\left\lVert T\right\rVert \leq \left\lVert T\right\rVert _{\mathrm{HS} }\).
		\item If \(T\) is \hyperref[def:Hilbert-Schmidt-op]{Hilbert-Schmidt}, then \(T\) is \hyperref[def:compact-op]{compact}.
	\end{enumerate}
\end{proposition}
\begin{proof}
	We prove this in the following order.
	\begin{enumerate}
		\item[(c)] Let \(x\in \mathcal{\MakeUppercase{h}} \) such that \(x = \sum_{k=1}^{\infty} a_{k} x_{k} \), \(a_{k} \in \mathbb{\MakeUppercase{c}} \). We see that \(\left\lVert x\right\rVert ^{2} = \sum_{k=1}^{\infty}\left\vert a_{k}  \right\vert ^{2} \) from the \hyperref[col:Parseval]{Parseval identity} with \(a_{k} = \left\langle x, x_{k}  \right\rangle \). Then,
			\[
				\left\lVert Tx\right\rVert
				= \left\lVert \sum_{k=1}^{\infty} a_{k} Tx_{k} \right\rVert
				\leq \sum_{k=1}^{\infty} \left\vert a_{k}  \right\vert \left\lVert T x_{k} \right\rVert
				\leq \left( \sum_{k=1}^{\infty} \left\vert a_{k} \right\vert ^2 \right) ^{1 / 2} \left( \sum_{k=1}^{\infty} \left\lVert T x_{k} \right\rVert ^{2} \right) ^{1 / 2}
				= \left\lVert x\right\rVert \left\lVert T\right\rVert _{\mathrm{HS}},
			\]
			so \(T\) is \hyperref[def:bounded-linear-op]{bounded} with \(\left\lVert T\right\rVert \leq \left\lVert T\right\rVert _{\mathrm{HS} }\), hence \(T^{\ast} \) is well-defined.
		\item[(b)] Show \(\left\lVert T\right\rVert _{\mathrm{HS} }\) under of an \hyperref[def:orthonormal-system]{orthonormal} basis and \(\left\lVert T\right\rVert _{\mathrm{HS} }= \left\lVert T^{\ast} \right\rVert _{\mathrm{HS} } \). Since
			\[
				\sum_{k=1}^{\infty} \left\lVert Tx_{k} \right\rVert ^{2}
				= \sum_{k, j} \left\vert \left( T x_{k} , x_{j}  \right)  \right\vert ^{2}
				= \sum_{k, j} \left\vert \left\langle x_{k} , T^{\ast} x_{j}  \right\rangle  \right\vert ^{2}
				= \sum_{j} \left\lVert T^{\ast} x_{j} \right\rVert ^{2}
				= \left\lVert T^{\ast} \right\rVert ^{2} _{\mathrm{HS} },
			\]
			hence \(\left\lVert T\right\rVert _{\mathrm{HS} } = \left\lVert T^{\ast} \right\rVert _{\mathrm{HS} }\).
		\item[(a)] Let \(\left\{ x^\prime _k \right\}_{k\geq 1} \) be a \hyperref[def:separable]{separate} \hyperref[def:orthonormal-system]{orthonormal} basis, Then
			\[
				\sum_{j} \left\lVert T^{\ast} x_{j} \right\rVert ^{2}
				= \sum_{j, k} \left\vert \left\langle x^\prime _{k} , T^{\ast} x_{j}  \right\rangle  \right\vert ^{2}
				= \sum_{j, k} \left\vert \left\langle T x^\prime _{k} , x_{j}  \right\rangle  \right\vert ^{2}
				= \sum_{k} \left\lVert T x^\prime _{k} \right\rVert ^{2},
			\]
			we see that the norm is independent of the choice of basis.
		\item[(d)] If \(T\) is \hyperref[def:Hilbert-Schmidt-op]{Hilbert-Schmidt}, then \(T\) is a limit of finite rank operators. Define for \(N \geq 1\), \(T_N\) by \(T_N x_K = Tx_k\) for \(k = 1, 2, \ldots  , N\), \(T_{N}x_k = 0 \) for \(k > N\), hence \(T_N\) is finite rank. We then have
			\[
				\left\lVert T - T_{N} \right\rVert ^{2} _{\mathrm{HS} }= \sum_{k=N+1}^{\infty} \left\lVert T x_{k} \right\rVert ^{2}
				\implies \lim_{N \to \infty} \left\lVert T - T_{N} \right\rVert _{\mathrm{HS} }= 0
				\implies \lim_{N \to \infty} \left\lVert T - T_N\right\rVert = 0.
			\]
	\end{enumerate}
\end{proof}

One important motivation of studying \hyperref[def:compact-op]{compact operators} is because integral operators are \hyperref[def:compact-op]{compact}, and furthermore, is \hyperref[def:Hilbert-Schmidt-op]{Hilbert-Schmidt}.
\begin{proposition}[Hilbert-Schmidt integral operator]\label{prop:Hilbert-Schmidt-integral-op}
	Let \(k\in L^2([0, 1]\times [0, 1])\) such that
	\[
		\int _0^1 \int _0^1 \left\vert k(t, s) \right\vert ^{2} \,\mathrm{d} t\,\mathrm{d} s < \infty.
	\]
	Then define an integral operator \(T\colon L^2([0, 1])\to L^2([0, 1])\) by
	\[
		Tf(t) = \int _0^1 k(t, s)f(s)\,\mathrm{d} s
	\]
	for \(0 < t < 1\) and \(f\in L^2([0, 1])= \mathcal{\MakeUppercase{h}} \). Then \(T\) is \hyperref[def:Hilbert-Schmidt-op]{Hilbert-Schmidt} on \(\mathcal{\MakeUppercase{h}} \) and \(\left\lVert T\right\rVert _{\mathrm{HS} }= \left\lVert k\right\rVert _{L^2([0, 1]^2)}\).
\end{proposition}
\begin{proof}
	Let \(K_t(s) \coloneqq k(t, x)\) for \(0<s, t<1\), then \(Tf(t) = \left\langle K_t, f \right\rangle \). Note that \(\left\lVert Tf(t)\right\rVert \leq \left\lVert K_t\right\rVert _2 \left\lVert f\right\rVert _2\),
	\[
		\int_{0}^{1} \left\lVert K_t\right\rVert ^2_2 \,\mathrm{d}t = \left\lVert k\right\rVert _2^2 \implies \left\lVert K_t\right\rVert _2<\infty \text{ a.e. }t\in [0,1].
	\]
	Hence, \(Tf(t)\) is defined for a.e. \(t\). Let \(\left\{ x_k \right\} _{k \geq 1}\) be an \hyperref[def:orthonormal-system]{orthonormal} basis for \(\mathcal{\MakeUppercase{h}} =L^2([0, 1])\),
	\[
		\left\lVert T\right\rVert ^2 _{\mathrm{HS} }
		= \sum_{k=1}^{\infty} \left\lVert T x_{k} \right\rVert _2^2
		= \sum_{i=1}^{\infty} \int _0^1 \left\vert T x_{k} (t) \right\vert ^2 \,\mathrm{d} t
		= \sum_{k=1}^{\infty} \int _0^1 \left\vert \left\langle K_{t} , x_{t}  \right\rangle  \right\vert ^2 \,\mathrm{d} t
		= \int _0^1 \sum_{k=1}^{\infty}\left\vert \left\langle K_{t} , x_{t}  \right\rangle  \right\vert ^2 \,\mathrm{d} t ,
	\]
	where the last equality follows from the \href{https://en.wikipedia.org/wiki/Monotone_convergence_theorem}{monotone convergence theorem}. Further, from \hyperref[col:Parseval]{Parseval},
	\[
		\left\lVert T\right\rVert ^2 _{\mathrm{HS} }
		= \int _0^1 \sum_{k=1}^{\infty}\left\vert \left\langle K_{t} , x_{t}  \right\rangle  \right\vert ^2 \,\mathrm{d} t
		= \int _0^1 \left\lVert K_t\right\rVert _2^2 \,\mathrm{d} t
		= \left\lVert k\right\rVert _2^2
	\]
	by the definition of \(K_t\) and \href{https://en.wikipedia.org/wiki/Fubini's_theorem}{Fubini's theorem}
\end{proof}

\subsection{Compactness of the Adjoint Operators}
Recall the basic \hyperref[def:dual-space]{duality} property for \hyperref[def:bounded-linear-op]{bounded linear operators}.

\begin{prev}
	If \(T\in \mathcal{\MakeUppercase{l}} (X, Y)\), then \(T^{\ast} \in \mathcal{\MakeUppercase{l}} (Y^{\ast} , X^{\ast} )\) and \(\lVert T^{\ast}  \rVert = \lVert T \rVert \).
\end{prev}

Indeed, a similar \hyperref[def:dual-space]{duality} principle holds for \hyperref[def:compact-op]{compact operators} as guaranteed from \autoref{thm:Schauder}.

\begin{theorem}[Schauder]\label{thm:Schauder}
	Let \(X, Y\) be \hyperref[def:Banach-space]{Banach spaces}, then if \(T\in K(X, Y)\), \(T^{\ast} \in K(Y^{\ast} , X^{\ast} )\).
\end{theorem}
\begin{proof}
	\(T\) is \hyperref[def:bounded-map]{bounded} implies \(T^{\ast} \) is \hyperref[def:bounded-map]{bounded} and \(\left\lVert T\right\rVert = \left\lVert T^{\ast} \right\rVert \). To prove this, we need the \href{https://en.wikipedia.org/wiki/Arzel%C3%A0%E2%80%93Ascoli_theorem}{Arzelà-Ascoli theorem}. Let \(K\) be a \hyperref[def:compact]{compact set}, and \(\mathcal{\MakeUppercase{f}} \subseteq C(K)\). \(\mathcal{\MakeUppercase{f}} \) is \hyperref[def:precompact]{precompact} in \(C(K)\) provided that 
	\begin{itemize}
		\item \(\mathcal{\MakeUppercase{f}} \) is bounded in \(C(K)\).
		\item \(\mathcal{\MakeUppercase{f}} \) is equicontinuous, i.e., for any \(\epsilon >0\), \(y\in K\), there exists an open neighborhood \(U_{\epsilon , y}\) of \(y\) such that \(\left\vert f(x) - f(x^\prime ) \right\vert< \epsilon \) if \(x, x^\prime \in U_{\epsilon , y}\), \(f\in \mathcal{\MakeUppercase{f}} \).
	\end{itemize}
	We want to show \(T^{\ast} B_{Y^{\ast} }\) is \hyperref[def:precompact]{precompact} in \(B_{X^{\ast} }\). Let \(f_{n} \in B_{Y^{\ast} }\), \(n \geq 1\) be a sequence, i.e., \(\left\lVert f_n \right\rVert _{Y^{\ast} } \leq 1\) for all \(n\geq 1\) and \(T^{\ast} f_{n} \in X^{\ast} \), and we need to show \(\left\{ T^{\ast} f_n \right\}_{n\geq 1} \) has a converging subsequence. Regard \(T^{\ast} f_{n} \colon B_X \to \mathbb{\MakeUppercase{r}} \) a function on \(\overline{TB_{X} } = K\) with \(T^{\ast} f_{n} (x) = f_{n} (Tx)\), we have
	\[
		\sup _{y\in K} \left\vert f_{n} (y)\leq \left\lVert f_{n} \right\rVert  \right\vert \left\lVert T\right\rVert \leq \left\lVert T\right\rVert,
	\]
	hence \(\left\{ T^{\ast} f_{n}  \right\}_{n\geq 1} \) is a bounded sequence of functions \(K\to \mathbb{\MakeUppercase{r}} \). Next, to show the equicontinuity, we have
	\[
		\left\vert f_{n} (y) - f_{n} (y^\prime ) \right\vert
		\leq \left\lVert f_{n} \right\rVert \left\lVert y-y^\prime \right\rVert
		\leq \left\lVert y - y^\prime \right\rVert
	\]
	for all \(y, y^\prime \in Y\), hence we have that \(\left\{ T^{\ast} f_{n}  \right\}_{n\geq 1} \) is equicontinuous on \(K\) concluding that there's a subsequence \(f{n_{k} }\), \(k \geq 1\) and \(f_{\infty }\colon K\to \mathbb{\MakeUppercase{r}} \) such that
	\[
		\lim_{k \to \infty} \sup _{y\in K}\left\vert f_{n_{k} } (y) - f_\infty (y) \right\vert = 0 ,
	\]
	implying that the sequence \(T^{\ast} f_{n_{k} }\) is Cauchy in \(X^{\ast} \), hence we have \hyperref[def:precompact]{precompactness}.
\end{proof}