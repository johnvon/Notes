\lecture{18}{1 Nov. 14:30}{Hilbert-Schmidt Operators}
\begin{corollary}[Isomorphisms are not compact]\label{col:iso-are-not-compact}
	Let \(X\) be an infinite dimension \hyperref[def:normed-vector-space]{normed space}, then the identity map \(I\colon X\to X\) is not \hyperref[def:compact-op]{compact}. More generally, an invertible operator (i.e., isomorphism) \(T\in \mathcal{\MakeUppercase{l}} (X, Y)\) is not \hyperref[def:compact-op]{compact}.
\end{corollary}
\begin{proof}
	\(I\) is not \hyperref[def:compact-op]{compact} since \hyperref[thm:Riesz]{Riesz theorem} implies that \(B_X\) is not \hyperref[def:compact]{compact}. And in general, if \(T\in \mathcal{\MakeUppercase{l}} (X, Y)\) is \hyperref[def:compact-op]{compact} and invertible, \autoref{prop:compact-op} implies that \(T^{-1} T = I\colon X\to X\) is \hyperref[def:compact-op]{compact} since \(T^{-1} \) is bounded and \(T\) is \hyperref[def:compact-op]{compact}, contradiction.
\end{proof}

Finally, an important observation is that since we know \(\mathcal{K}(X, Y)\) is closed from \autoref{prop:compact-op}, and any \hyperref[rmk:finite-rank-op]{finite rank operators} are \hyperref[def:compact-op]{compact}, hence for any operator that can be approximated by \hyperref[rmk:finite-rank-op]{finite rank operators} is also \hyperref[def:compact-op]{compact}, leading to the following.

\begin{corollary}[Almost finite rank operators are compact]\label{col:almost-finite-rank-op-are-compact}
	Suppose a \hyperref[def:linear-map]{linear operator} \(T\colon X\to Y\) can be approximated by \hyperref[rmk:finite-rank-op]{finite rank operators} \(T_{n} \in \mathcal{\MakeUppercase{l}} (X, Y)\) as \(\lim_{n \to \infty} \lVert T_{n} - T \rVert = 0\), then \(T\) is \hyperref[def:compact-op]{compact}.
\end{corollary}

\subsection{Hilbert-Schmidt Operators}
This is a most frequently used class of \hyperref[def:compact-op]{compact operators} in \hyperref[def:Hilbert-space]{Hilbert spaces}. As we will see, it covers the class of important operators, i.e., integral operators, motivating us to study this class.

\begin{definition}[Hilbert-Schmidt operators]\label{def:Hilbert-Schmidt-op}
	Let \(\mathcal{\MakeUppercase{h}} \) be a \hyperref[def:Hilbert-space]{Hilbert space} which is \hyperref[def:separable]{separable} and \(\left\{ x_k \right\} _{k\geq 1}\) is an \hyperref[def:orthonormal-system]{orthonormal} basis in \(\mathcal{\MakeUppercase{h}} \). A \hyperref[def:linear-map]{linear operator} \(T\colon \mathcal{\MakeUppercase{h}} \to \mathcal{\MakeUppercase{h}} \) is \emph{Hilbert-Schmidt} if
	\[
		\sum_{k=1}^{\infty} \left\lVert T x_{k} \right\rVert ^{2} < \infty.
	\]
\end{definition}

From \autoref{def:Hilbert-Schmidt-op}, the following naturally induced \hyperref[def:norm]{norm} is defined.

\begin{definition}[Hilbert-Schmidt norm]\label{def:Hilbert-Schmidt-norm}
	The \emph{Hilbert-Schmidt norm} of a \hyperref[def:linear-map]{linear operator} \(T\) is
	\[
		\left\lVert T\right\rVert _{\mathrm{HS} } = \left( \sum_{k=1}^{\infty} \left\lVert T x_{k} \right\rVert ^2 \right) ^{1 / 2}.
	\]
\end{definition}

To characterize \hyperref[def:Hilbert-Schmidt-op]{Hilbert-Schmidt operators}, we have the following.

\begin{proposition}
	Let \(T\colon \mathcal{\MakeUppercase{h}} \to \mathcal{\MakeUppercase{h}} \) be a \hyperref[def:linear-op]{linear operator} on the \hyperref[def:Hilbert-space]{Hilbert space} \(\mathcal{\MakeUppercase{h}} \) with a \hyperref[def:separable]{separate} \hyperref[def:orthonormal-system]{orthonormal} basis \(\left\{ x^\prime _k \right\}_{k\geq 1} \).
	\begin{enumerate}[(a)]
		\item The \hyperref[def:Hilbert-Schmidt-norm]{Hilbert-Schmidt norm} of \(T\) is independent of the choice of \hyperref[def:orthonormal-system]{orthonormal} basis.
		\item If \(T\) is \hyperref[def:Hilbert-Schmidt-op]{Hilbert-Schmidt}, then \(T^{\ast} \) is \hyperref[def:Hilbert-Schmidt-op]{Hilbert-Schmidt} and \(\left\lVert T\right\rVert _{\mathrm{HS} } = \left\lVert T^{\ast} \right\rVert _{\mathrm{HS} } \).
		\item If \(T\) is \hyperref[def:Hilbert-Schmidt-op]{Hilbert-Schmidt}, then \(T\) is \hyperref[def:bounded-linear-op]{bounded} on \(\mathcal{\MakeUppercase{h}} \) and \(\left\lVert T\right\rVert \leq \left\lVert T\right\rVert _{\mathrm{HS} }\).
		\item If \(T\) is \hyperref[def:Hilbert-Schmidt-op]{Hilbert-Schmidt}, then \(T\) is \hyperref[def:compact-op]{compact}.
	\end{enumerate}
\end{proposition}
\begin{proof}
	We prove this in the following order.
	\begin{enumerate}
		\item[(c)] Let \(x\in \mathcal{\MakeUppercase{h}} \) such that \(x = \sum_{k=1}^{\infty} a_{k} x_{k} \), \(a_{k} \in \mathbb{\MakeUppercase{c}} \). From the \hyperref[col:Parseval]{Parseval identity}, \(\left\lVert x\right\rVert ^{2} = \sum_{k=1}^{\infty}\left\vert a_{k}  \right\vert ^{2} \) where \(a_{k} = \left\langle x, x_{k}  \right\rangle \), then
			\[
				\left\lVert Tx\right\rVert
				= \left\lVert \sum_{k=1}^{\infty} a_{k} Tx_{k} \right\rVert
				\leq \sum_{k=1}^{\infty} \left\vert a_{k}  \right\vert \left\lVert T x_{k} \right\rVert
				\leq \left( \sum_{k=1}^{\infty} \left\vert a_{k} \right\vert ^2 \right) ^{1 / 2} \left( \sum_{k=1}^{\infty} \left\lVert T x_{k} \right\rVert ^{2} \right) ^{1 / 2}
				= \left\lVert x\right\rVert \left\lVert T\right\rVert _{\mathrm{HS}},
			\]
			so \(T\) is \hyperref[def:bounded-linear-op]{bounded} with \(\left\lVert T\right\rVert \leq \left\lVert T\right\rVert _{\mathrm{HS} }\).\footnote{This implies that \(T^{\ast} \) is well-defined.}
		\item[(b)] Since
			\[
				\sum_{k=1}^{\infty} \left\lVert Tx_{k} \right\rVert ^{2}
				= \sum_{k, j} \left\vert \left\langle  T x_{k} , x_{j} \right\rangle \right\vert ^{2}
				= \sum_{k, j} \left\vert \left\langle x_{k} , T^{\ast} x_{j}  \right\rangle  \right\vert ^{2}
				= \sum_{j} \left\lVert T^{\ast} x_{j} \right\rVert ^{2}
				= \left\lVert T^{\ast} \right\rVert ^{2} _{\mathrm{HS} },
			\]
			hence \(\left\lVert T\right\rVert _{\mathrm{HS} } = \left\lVert T^{\ast} \right\rVert _{\mathrm{HS} }\).
		\item[(a)] Let \(\left\{ x^\prime _k \right\}_{k\geq 1} \) be another \hyperref[def:separable]{separate} \hyperref[def:orthonormal-system]{orthonormal} basis, then from the \hyperref[col:Parseval]{Parseval identity}, and \(\lVert T \rVert _{\mathrm{HS} }= \lVert T^{\ast}  \rVert_{\mathrm{HS}}\),
			\[
				\lVert T \rVert _{\mathrm{HS} }
				= \lVert T^{\ast}  \rVert_{\mathrm{HS} }
				= \sum_{j} \left\lVert T^{\ast} x_{j} \right\rVert ^{2}
				= \sum_{j, k} \left\vert \left\langle x^\prime _{k} , T^{\ast} x_{j}  \right\rangle  \right\vert ^{2}
				= \sum_{j, k} \left\vert \left\langle T x^\prime _{k} , x_{j}  \right\rangle  \right\vert ^{2}
				= \sum_{k} \left\lVert T x^\prime _{k} \right\rVert ^{2},
			\]
			i.e., the \hyperref[def:Hilbert-Schmidt-norm]{Hilbert-Schmidt norm} is independent of the choice of basis.
		\item[(d)] From \autoref{col:almost-finite-rank-op-are-compact}, we show that \(T\) is a limit of \hyperref[rmk:finite-rank-op]{finite rank operators}. Define for \(N \geq 1\), \(T_N\) by \(T_N x_k = Tx_k\) for \(k = 1, 2, \ldots  , N\), \(T_{N}x_k = 0 \) for \(k > N\), hence \(T_N\) is \hyperref[rmk:finite-rank-op]{finite rank}. We then have
			\[
				\left\lVert T - T_{N} \right\rVert ^{2} _{\mathrm{HS} }= \sum_{k=N+1}^{\infty} \left\lVert T x_{k} \right\rVert ^{2}
				\implies \lim_{N \to \infty} \left\lVert T - T_{N} \right\rVert _{\mathrm{HS} }= 0.
			\]
			Since \(\lVert \cdot \rVert \leq \lVert \cdot \rVert _{\mathrm{HS} }\), \(\lim_{N \to \infty} \left\lVert T - T_N\right\rVert = 0\) as desired.
	\end{enumerate}
\end{proof}

One important motivation of studying \hyperref[def:compact-op]{compact operators} is because integral operators are \hyperref[def:compact-op]{compact}, and furthermore, is \hyperref[def:Hilbert-Schmidt-op]{Hilbert-Schmidt}.
\begin{proposition}[Hilbert-Schmidt integral operator]\label{prop:Hilbert-Schmidt-integral-op}
	Let \(k\in L^2([0, 1]\times [0, 1])\) such that
	\[
		\int _0^1 \int _0^1 \left\vert k(t, s) \right\vert ^{2} \,\mathrm{d} t\,\mathrm{d} s < \infty,
	\]
	and define an integral operator \(T\colon L^2([0, 1])\to L^2([0, 1])\) by
	\[
		Tf(t) = \int _0^1 k(t, s)f(s)\,\mathrm{d} s
	\]
	for \(0 < t < 1\) and \(f\in L^2([0, 1])= \mathcal{\MakeUppercase{h}} \). Then \(T\) is \hyperref[def:Hilbert-Schmidt-op]{Hilbert-Schmidt} on \(\mathcal{\MakeUppercase{h}} \) and \(\left\lVert T\right\rVert _{\mathrm{HS} }= \left\lVert k\right\rVert _{L^2([0, 1]^2)}\).
\end{proposition}
\begin{proof}
	Let \(K_t(s) \coloneqq k(t, s)\) for \(0<s, t<1\), then \(Tf(t) = \left\langle K_t, f \right\rangle \). Note that \(\left\lVert Tf(t)\right\rVert \leq \left\lVert K_t\right\rVert _2 \left\lVert f\right\rVert _2\) from \hyperref[thm:Cauchy-Schwarz-ineq]{Cauchy-Schwarz}, and also, notice that
	\[
		\int_{0}^{1} \left\lVert K_t\right\rVert ^2_2 \,\mathrm{d}t = \left\lVert k\right\rVert _2^2 \implies \left\lVert K_t\right\rVert _2<\infty \text{ a.e. }t\in [0,1].
	\]
	Hence, \(Tf(t)\) is defined for a.e. \(t\). Let \(\left\{ x_k \right\} _{k \geq 1}\) be an \hyperref[def:orthonormal-system]{orthonormal} basis for \(\mathcal{\MakeUppercase{h}} =L^2([0, 1])\),
	\[
		\left\lVert T\right\rVert ^2 _{\mathrm{HS} }
		= \sum_{k=1}^{\infty} \left\lVert T x_{k} \right\rVert _2^2
		= \sum_{i=1}^{\infty} \int _0^1 \left\vert T x_{k} (t) \right\vert ^2 \,\mathrm{d} t
		= \sum_{k=1}^{\infty} \int _0^1 \left\vert \left\langle K_{t} , x_{t}  \right\rangle  \right\vert ^2 \,\mathrm{d} t
		= \int _0^1 \sum_{k=1}^{\infty}\left\vert \left\langle K_{t} , x_{t}  \right\rangle  \right\vert ^2 \,\mathrm{d} t ,
	\]
	where the last equality follows from the \href{https://en.wikipedia.org/wiki/Monotone_convergence_theorem}{monotone convergence theorem}. Further, from \hyperref[col:Parseval]{Parseval},
	\[
		\left\lVert T\right\rVert ^2 _{\mathrm{HS} }
		= \int _0^1 \sum_{k=1}^{\infty}\left\vert \left\langle K_{t} , x_{t}  \right\rangle  \right\vert ^2 \,\mathrm{d} t
		= \int _0^1 \left\lVert K_t\right\rVert _2^2 \,\mathrm{d} t
		= \left\lVert k\right\rVert _2^2
	\]
	by the definition of \(K_t\) and \href{https://en.wikipedia.org/wiki/Fubini's_theorem}{Fubini's theorem}
\end{proof}

\subsection{Compactness of the Adjoint Operators}
Recall the basic \hyperref[def:dual-space]{duality} property for \hyperref[def:bounded-linear-op]{bounded linear operators}.

\begin{prev}
	If \(T\in \mathcal{\MakeUppercase{l}} (X, Y)\), then \(T^{\ast} \in \mathcal{\MakeUppercase{l}} (Y^{\ast} , X^{\ast} )\) and \(\lVert T^{\ast}  \rVert = \lVert T \rVert \).
\end{prev}

Indeed, a similar \hyperref[def:dual-space]{duality} principle holds for \hyperref[def:compact-op]{compact operators} as guaranteed by \autoref{thm:Schauder}.

\begin{theorem}[Schauder's theorem]\label{thm:Schauder}
	Let \(X, Y\) be \hyperref[def:Banach-space]{Banach spaces}, then if \(T\in \mathcal{K}(X, Y)\), \(T^{\ast} \in \mathcal{K}(Y^{\ast} , X^{\ast} )\).
\end{theorem}
\begin{proof}
	Without loss of generality, we prove that \(T^{\ast} B_{Y^{\ast} }\) is \hyperref[def:precompact]{precompact} in \(X^{\ast}\) given \(T\in \mathcal{K}(X, Y)\), i.e., consider \(f_n \in B_{Y^{\ast} }\) for \(n \geq 1\) such that \(\{T^{\ast} f_n\}_{n\geq 1}\) has a convergent subsequence.

	\begin{claim}
		\(B_{Y^{\ast} }\) is \hyperref[def:precompact]{precompact} in \(Y^{\ast} \), i.e., there is a convergent subsequence \(\left\{ f_{n_k} \right\} _{k\geq 1}\) of \(\left\{ f_n \right\} _{n\geq 1}\).
	\end{claim}
	\begin{explanation}
		This can be done by embedding \(B_{Y^{\ast} }\) into \(C(K)\) with \(K\coloneqq \overline{TB_X}\) by considering \(\at{f_n}{K}{} \in C(K)\). Observe that \(\{\at{f_n}{K}{}\}_{n\geq 1}\) is bounded and \hyperref[def:equicontinuous]{equicontinuity}: First, \(\left\{ \at{f_n}{K}{}  \right\} _{n\geq 1}\) is bounded since \(f_{n} \in B_{Y^{\ast} }\) implying \(\left\lVert f_n \right\rVert _{Y^{\ast} } \leq 1\) for all \(n\geq 1\), and so is \(\at{f_n}{K}{} \); while for \hyperref[def:equicontinuous]{equicontinuity}, we have
		\[
			\left\vert \at{f_{n}}{K}{} (y) - \at{f_{n}}{K}{} (y^\prime ) \right\vert
			= \vert f(y_1 - y_2) \vert
			\leq \left\lVert f_{n} \right\rVert \left\lVert y-y^\prime \right\rVert
			\leq \left\lVert y - y^\prime \right\rVert
		\]
		for all \(y, y^\prime \in Y\). In all, since \(K\) is \hyperref[def:compact]{compact}, with \hyperref[thm:Arzela-Ascoli]{Arzelà-Ascoli theorem}, \(\left\{ \at{f_n}{K}{} \right\}_{n\geq 1} \) has a convergent subsequence \(\left\{ \at{f_{n_k}}{K}{}\right\}_{k\geq 1} \) in \(C(K)\).\footnote{Same as saying \(B_{Y^{\ast} }\) is \hyperref[def:precompact]{precompact} as the original statement in \hyperref[thm:Arzela-Ascoli]{Arzelà-Ascoli theorem}.}
	\end{explanation}

	Hence, to show that \(\left\{ T^{\ast} f_n \right\}_{n\geq 1} \) has a convergent subsequence, consider \(\left\{ T^{\ast} f_{n_k} \right\}_{k\geq 1} \), we have\footnote{The last equality follows from the fact that \(TB_X\) is dense in \(K\) (i.e., \(\overline{TB_X} = K\)).}
	\[
		\lVert T^{\ast} f_{n_i} - T^{\ast} f_{n_j} \rVert_{X^{\ast} }
		= \sup _{x\in B_X} \vert f_{n_i}(Tx) - f_{n_j}(Tx)\vert
		= \sup _{y\in K} \left\vert \at{f_{n_i}}{K}{} (y) - \at{f_{n_j}}{K}{} (y) \right\vert \to 0
	\]
	as \(n_i, n_j \to \infty \) since \(f_{n_k}\) converges, i.e., \(\{T^{\ast} f_{n_{k} }\}_{k\geq 1}\) is Cauchy in \(X^{\ast} \) hence converges.
\end{proof}