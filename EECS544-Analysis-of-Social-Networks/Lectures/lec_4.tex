\lecture{4}{13 Sep. 12:30}{Network Properties}
We now try to further characterize the connectedness of a network. We first give a common notion.
\begin{definition}[Complete graph]\label{def:complete-graph}
	A \hyperref[def:graph]{graph} \(\mathcal{G} = (\mathcal{V} , \mathcal{E} )\) is \emph{complete} if \(\mathcal{E} \) contains every possible edges.
\end{definition}

\begin{notation}
	We often denote a \hyperref[def:complete-graph]{complete graph} with \(n\) nodes as \(K_n\).
\end{notation}

\subsection{Bridges and Local Bridges}
\begin{definition}[Bridge]\label{def:bridge}
	A \emph{bridge} is the only edge that connects together two \hyperref[def:subgraph]{subgraphs} that are themselves \hyperref[def:connected]{connected}.
\end{definition}

\begin{remark}
	Deleting a bridge may change the distance between two nodes from some finite value to \(\infty\). (disconnects those two nodes)
\end{remark}

\begin{definition}[Local bridge]\label{def:local-bridge}
	An edge \((i, j)\) is said to be a \emph{local bridge} if \(i, j\) have no neighbors in common.
\end{definition}

\begin{note}
	Equivalently, a \hyperref[def:local-bridge]{local bridge} is an edge that is not included in any triangles.
\end{note}

\begin{remark}
	If \((i, j)\) is a \hyperref[def:local-bridge]{local bridge}, then we have
	\[
		d_{\mathcal{G}}(i, j) \geq  3
	\]
	after removing this \hyperref[def:local-bridge]{local bridge}. Furthermore, a \hyperref[def:bridge]{bridge} must also be a \hyperref[def:local-bridge]{local bridge} by definition.
\end{remark}

\begin{definition}[Span of a local bridge]
	The \emph{span of a \hyperref[def:local-bridge]{local bridge}} \(b\) is defined as \(d_{\mathcal{G}[\mathcal{E} \setminus \{b\}]}\).
\end{definition}

\section{Triadic Closure}
\begin{intuition}[\cite{easley2010networks} pp.62]
	If two people in a social network have a friend in common, then there is an increased likelihood that they will become friends themselves at some point in the future.
\end{intuition}

\begin{eg}[Triadic closure]
	The \emph{triadic closure} says that if a network has this property, then given \(a, b, c\in \mathcal{V}\), if \(\{a, b\}, \{b, c\}\in \mathcal{E} \), it's very likely that \(\{a, c\} \in \mathcal{E} \) as well.
	\begin{center}
		\incfig{triadic-closure}
	\end{center}
\end{eg}

This motivates us that the number of triangles maybe can be used to quantify the connectedness of a network.

\begin{definition}[Clustering coefficient]\label{def:clustering-coefficient}
	For each node \(i\), define the \emph{clustering coefficient}
	\[
		c_{i} = \dfrac{\#\text{ of triangles in the \hyperref[def:graph]{graph} that include node }i}{d_{i}(d_{i}-1) / 2},
	\]
	where \(d_i\) is the \hyperref[def:degree]{degree}\footnote{We assume our network is an \hyperref[def:undirected-graph]{undirected} for simplicity.} of \(i\).
\end{definition}

\begin{remark}
	We have, for any node \(i\), \(c_{i}\in \left[ 0, 1 \right].\) If \(\forall i\ c_{i} = 1\), then this \hyperref[def:graph]{graph} is \hyperref[def:complete-graph]{complete}. If \(\forall i\ c_{i} = 0\), then this \hyperref[def:graph]{graph} is a tree.
\end{remark}

The above intuition suggests the following characteristics of a network.
\begin{definition}[Strong triadic closure]\label{def:strong-triadic-closure}
	If \(a\) is a node and there are two nodes \(b, c\) such that \((a, b)\) and \((a, c)\) are strong ties, then \((b, c)\) will form an edge. If for any nodes \(a, b, c\) in this graph \(\mathcal{G}\) satisfy this property, then we say this graph has \emph{strong triadic closure property} (STC).
\end{definition}

\begin{definition}[Embeddedness]\label{def:embeddedness}
	The \emph{embeddedness} of an edge is defined as the number of common neighbors shared by the end points.
\end{definition}