\lecture{3}{8 Sep. 12:30}{Network Property}
A network is essentially a \hyperref[def:graph]{graph} with some interesting properties. We start with connectivity.

\section{Measuring Connectivity}
\begin{intuition}
	If there are more \hyperref[def:path]{paths} in the \hyperref[def:graph]{graph} between different parts of the \hyperref[def:graph]{graph}, then this \hyperref[def:graph]{graph} is \emph{more connected}.
\end{intuition}

To characterize \hyperref[def:path]{paths}, we have the following definitions.
\begin{definition}[Edge independent]\label{def:edge-independent}
	Two \hyperref[def:path]{paths} are said to be \emph{edge independent} (EI) if they do not share a common edge.
\end{definition}

\begin{definition}[Vertex independent]\label{def:vertex-independent}
	To \hyperref[def:path]{paths} are said to be \emph{vertex independent} (VI) if they do not share a common vertex except the starting vertex and the ending vertex.
\end{definition}

\subsection{Cut Set}
Before we formulate the definition of cut set, we'll need to define so-called \hyperref[def:induced-subgraph]{induced subgraph} since they're highly related.

\begin{definition}[Induced subgraph]\label{def:induced-subgraph}
	Given a \hyperref[def:graph]{graph} \(\mathcal{\MakeUppercase{g}} = (\mathcal{\MakeUppercase{v}} , \mathcal{\MakeUppercase{e}} )\), the \emph{induced subgraph} with respect to
	\begin{itemize}
		\item \(\mathcal{\MakeUppercase{v}}^\prime \subseteq \mathcal{\MakeUppercase{v}} \) is defined as \(\mathcal{\MakeUppercase{g}} [\mathcal{\MakeUppercase{v}} ^\prime ] \coloneqq (\mathcal{\MakeUppercase{v}} ^\prime, \hat{\mathcal{\MakeUppercase{e}}})\), where \(\hat{\mathcal{\MakeUppercase{e}} }\coloneqq \{e\in \mathcal{\MakeUppercase{e}} \colon e = (i, j), i, j\in \mathcal{\MakeUppercase{v}} ^\prime\}\).\footnote{This is for \hyperref[def:directed-graph]{directed graph}, but the case for \hyperref[def:undirected-graph]{undirected graph} is similar, just change \((i, j)\) to \(\{i, j\}\).}
		\item \(\mathcal{\MakeUppercase{e}}^\prime \subseteq \mathcal{\MakeUppercase{e}} \) is defined as \(\mathcal{\MakeUppercase{g}} [\mathcal{\MakeUppercase{e}} ^\prime ] \coloneqq (\mathcal{\MakeUppercase{v}} , \mathcal{\MakeUppercase{e}}^\prime )\).
	\end{itemize}
\end{definition}

\begin{intuition}
	This is just a convenient notion for \hyperref[def:graph]{graph} subtraction which is essentially just the subtraction between the edge set and the vertex set. Notice that the case of edge-subtraction is trivial, while node-subtraction needs to be well-defined, e.g, it's nonsense to say \((i, j)\) is still in the \hyperref[def:induced-subgraph]{induced subgraph} if one end of the edge is not in the \hyperref[def:vertex-set]{node set} anymore.
\end{intuition}

\begin{definition}[Edge cut set]\label{def:edge-cut-set}
	Given \(u, v\in \mathcal{\MakeUppercase{v}} \), a set of edges \(\mathcal{\MakeUppercase{e}} ^\prime \subseteq \mathcal{\MakeUppercase{e}} \) is called an \emph{edge cut set} if \(u, v\) are in different \hyperref[def:connected]{connected components} in the \hyperref[def:induced-subgraph]{subgraph induced from \(\mathcal{\MakeUppercase{e}} ^\prime \)}.
\end{definition}

\begin{definition}[Vertex cut set]\label{def:vertex-cut-set}
	Given \(u, v\in \mathcal{\MakeUppercase{v}} \), a set of vertices \(\mathcal{\MakeUppercase{v}} ^\prime \subseteq \mathcal{\MakeUppercase{v}}\) is called a \emph{vertex cut set} if \(u, v\) are in different \hyperref[def:connected]{connected components} in the \hyperref[def:induced-subgraph]{subgraph induced from \(\mathcal{\MakeUppercase{v}} ^\prime \)}.
\end{definition}

\begin{note}
	It's worth noting that in the case of \hyperref[def:vertex-cut-set]{vertex cut set}, if \(u\) or \(v\) is not in \(\mathcal{\MakeUppercase{v}} ^\prime\), then we still say \(u\) and \(v\) are not in the same \hyperref[def:connected]{connected component} (even if neither \(u\) nor \(v\) is in \(\mathcal{\MakeUppercase{v}} ^\prime \)).
\end{note}

\begin{remark}
	Note that the above definitions are for \hyperref[def:undirected-graph]{undirected graph} since we only consider \hyperref[def:connected]{connected} but not \hyperref[def:strongly-connected]{strongly connected}. But it's obvious that one can generalize the notion to \hyperref[def:directed-graph]{directed graph}.
\end{remark}


\begin{theorem}[Mengur's theorem]\label{thm:Mengur-theorem}
	If for any pair of nodes \((i, j)\) we have \(\left\vert \mathrm{ECS}(i, j) \right\vert > n\) where \(\mathrm{ECS} (i, j)\) denotes the collection of \hyperref[def:edge-cut-set]{edge cut set} respect to \((i, j)\), then for any pair of nodes \((i, j)\),
	\[
		\# \text{ \hyperref[def:edge-independent]{edge independent} \hyperref[def:path]{paths} between } i \text{ and } j > n.
	\]
\end{theorem}
\begin{proof}
	A nice proof can be found \href{https://en.wikipedia.org/wiki/Menger%27s_theorem}{here}.
\end{proof}