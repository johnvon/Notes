\lecture{16}{1 Nov.\ 12:30}{Random Graph}
\section{Erdős-Rényi Random Graphs Family}\label{sec:Erdos-Renyi-random-graphs-family}
We introduce two main models for generating random \hyperref[def:graph]{graphs}.
\begin{itemize}
	\item Erdős-Rényi. The random \hyperref[def:graph]{graph} is given by \(G(n, M)\). A \hyperref[def:graph]{graph} is chosen uniformly at random from the collection of all \hyperref[def:graph]{graphs} with \(n\) nodes and \(M\) edges.
	\item Edward-Gilbert. The random \hyperref[def:graph]{graph} is given by \(G(n, p)\). The \hyperref[def:graph]{graph} is constructed by connecting nodes randomly, with each edge is included in the \hyperref[def:graph]{graph} with probability \(p\) \hyperref[def:independent]{independently}.
\end{itemize}

\begin{remark}
	We compare these two models.
	\begin{itemize}
		\item The number of expected edges in \(G(n, p)\) is \(\binom{n}{2}p\). Hence, by \hyperref[thm:SLLN]{SLLN}, nearly all \hyperref[def:graph]{graphs} in \(G(n, p)\) will have \(\binom{n}{2}p\) edges. Then if \(pn^{2} \to \infty \) and \(M = \binom{n}{2}p\), then as \(n\) grows, the behavior of \(G(n, p)\) should be similar with \(G(n, M)\).
		\item The most commonly used model is \(G(n, p)\) since the \hyperref[def:independent]{independent} property of edges simplifies lots of analysis.
	\end{itemize}
\end{remark}

We first focus on Erdős-Rényi random \hyperref[def:graph]{graphs} family.

\subsection{Uniform and Undirected Model}\label{subsec:Uniform-and-Undirected-model}
In this case, the \hyperref[def:adjacency-matrix]{adjacency matrix} \(A\) is \hyperref[def:symmetric-matrix]{symmetric}. This can be seen from constructing the \hyperref[def:adjacency-matrix]{adjacency matrix} \(A\) such that
\[
	A = \begin{pmatrix}
		0 &        & U \\
		  & \ddots &   \\
		L &        & 0 \\
	\end{pmatrix},
\]
where we need to specify the upper triangular part \(U\), and the lower triangular part \(L\) can then be obtained by \hyperref[def:symmetric-matrix]{symmetry}.

Now, let \(V\) denotes the number of nodes of the \hyperref[def:graph]{graph} we're going to generate, also let \(p\in(0, 1)\), we have the following algorithm to generate a random \hyperref[def:graph]{graph}.

\begin{algorithm}[H]\label{algo:uniform-and-undirected-model-algo}
	\DontPrintSemicolon
	\caption{Uniform and Undirected Random Graph Generator}
	\SetKwData{t}{t}
	\KwData{\(p\in(0, 1)\), \(V\)}
	\KwResult{\hyperref[def:adjacency-matrix]{Adjacency matrix} \(A\)}
	\SetKwFunction{bern}{Bernoulli}
	\BlankLine

	\(A_{V\times V} = 0_{V \times V}\) \Comment*[r]{Initialize}
	\;
	\For(){\(i= 1, \dots  , V-1\)}{
		\For(){\(j=i+1, \dots  , V\)}{
			\(A_{ij} \gets \)\bern{\(p\)}\Comment*[r]{\(\Pr(A_{ij} = 1) = p\), \(\Pr(A_{ij} = 0) = 1-p\)}
		}
	}
	\Return{\(A\)}\;
\end{algorithm}

Soon, we will study some properties as \(V\to \infty\).

\subsection{Non-Uniform and Undirected Model}\label{subsec:non-uniform-and-undirected-model}
Given a non-uniform distribution matrix \(P\) with \(P = P^{\top}\), where \(P\) is the probability matrix with diagonal entries being \(0\). Then, we have the following algorithm to generate a random \hyperref[def:graph]{graph}.

\begin{algorithm}[H]\label{algo:non-uniform-and-undirected-model-algo}
	\DontPrintSemicolon
	\caption{Non-Uniform and Undirected Random Graph Generator}
	\SetKwData{t}{t}
	\KwData{\(P\), \(V\)}
	\KwResult{\hyperref[def:adjacency-matrix]{Adjacency matrix} \(A\)}
	\SetKwFunction{bern}{Bernoulli}
	\BlankLine

	\(A_{V\times V} = 0_{V \times V}\) \Comment*[r]{Initialize}
	\;
	\For(){\(i= 1, \dots  , V-1\)}{
	\For(){\(j=i+1, \dots  , V\)}{
	\(A_{ij} \gets \)\bern{\(P_{ij}\)}\Comment*[r]{\(\Pr(A_{ij} = 1) = P_{ij}\), \(\Pr(A_{ij} = 0) = 1-P_{ij}\)}
	}
	}
	\Return{\(A\)}\;
\end{algorithm}

\subsection{Uniform and Directed Model}\label{subsec:uniform-and-directed-model}
Again, with \(p\in(0, 1)\), we have the following algorithm to generate a random \hyperref[def:graph]{graph}.

\begin{algorithm}[H]\label{algo:uniform-and-directed-model-algo}
	\DontPrintSemicolon
	\caption{Uniform and Directed Random Graph Generator}
	\SetKwData{t}{t}
	\KwData{\(p\in(0, 1)\), \(V\)}
	\KwResult{\hyperref[def:adjacency-matrix]{Adjacency matrix} \(A\)}
	\SetKwFunction{bern}{Bernoulli}
	\BlankLine

	\(A_{V\times V} = 0_{V \times V}\) \Comment*[r]{Initialize}
	\;
	\For(){\(i= 1, \dots  , V-1\)}{
		\For(){\(j=1, \dots  , V\)}{
			\(A_{ij} \gets \)\bern{\(p\)}\Comment*[r]{\(\Pr(A_{ij} = 1) = p\), \(\Pr(A_{ij} = 0) = 1-p\)}
		}
	}
	\Return{\(A\)}\;
\end{algorithm}

We see that for all \(i \neq j\),
\[
	i\neq j\quad A_{ij}= \mathrm{Bernoulli}(p)
\]
chosen \hyperref[def:independent]{independently}. And because of \hyperref[def:independent]{independence},
\(A_{ij}\neq A_{ji}\) can occur and
\[
	\Pr(A_{ij} = A_{ji} = 1) = \Pr(A_{ij} = 1) \Pr( A_{ji} = 1) = p^2.
\]
Compared to \hyperref[subsec:Uniform-and-Undirected-model]{undirected} case, we see that
\[
	\Pr(A_{ij} = A_{ji} = 1) = p
\]
in an \hyperref[def:undirected-graph]{undirected graph}.

\subsection{Non-Uniform and Directed Model}\label{subsec:non-uniform-and-directed-model}
Denote \(P\) as the probability matrix with diagonal entries being \(0\). Then we have the following algorithm to generate a random \hyperref[def:graph]{graph}.

\begin{algorithm}[H]\label{algo:non-uniform-and-directed-model-algo}
	\DontPrintSemicolon
	\caption{Non-Uniform and Directed Random Graph Generator}
	\SetKwData{t}{t}
	\KwData{\(P\), \(V\)}
	\KwResult{\hyperref[def:adjacency-matrix]{Adjacency matrix} \(A\)}
	\SetKwFunction{bern}{Bernoulli}
	\BlankLine

	\(A_{V\times V} = 0_{V \times V}\) \Comment*[r]{Initialize}
	\;
	\For(){\(i= 1, \dots  , V-1\)}{
	\For(){\(j= 1, \dots  , V\)}{
	\(A_{ij} \gets \)\bern{\(P_{ij}\)}\Comment*[r]{\(\Pr(A_{ij} = 1) = P_{ij}\), \(\Pr(A_{ij} = 0) = 1-P_{ij}\)}
	}
	}
	\Return{\(A\)}\;
\end{algorithm}

We see that for all \(i\neq j\),
\[
	A_{ij} = \mathrm{Bernoulli}(P_{ij})
\]
chosen \hyperref[def:independent]{independently}.

\begin{eg}[Stochastic block model]
	We consider a specific model called \emph{stochastic block model} (SBM) with the probability matrix being:
	\[
		P = \substack{V_1 \\ \\ V_2}\overset{V_1\qquad V_2}{\begin{bmatrix}
				P_{11} & P_{12} \\
				P_{21} & P_{22} \\
			\end{bmatrix}}
	\]
	with \(P_{12} = P_{21}^{\top}\), \(P_{11} = P_{11}^{\top}\) and \(P_{22} = P_{22}^{\top}\). In particular,
	\[
		\begin{split}
			P_{11} = p_{11}\begin{pmatrix}
				0      & 1      & \dots  & 1      & 1      \\
				1      & 0      & \dots  & 1      & 1      \\
				\vdots & \vdots & \ddots & \vdots & \vdots \\
				1      & 1      & \dots  & 0      & 1      \\
				1      & 1      & \dots  & 1      & 0      \\
			\end{pmatrix}_{V_1 \times V_1}, &
			P_{22} = p_{22}\begin{pmatrix}
				0      & 1      & \dots  & 1      & 1      \\
				1      & 0      & \dots  & 1      & 1      \\
				\vdots & \vdots & \ddots & \vdots & \vdots \\
				1      & 1      & \dots  & 0      & 1      \\
				1      & 1      & \dots  & 1      & 0      \\
			\end{pmatrix}_{V_2 \times V_2}\\
			P_{12} = p_{12}\begin{pmatrix}
				1      & 1      & \dots  & 1      & 1      \\
				1      & 1      & \dots  & 1      & 1      \\
				\vdots & \vdots & \ddots & \vdots & \vdots \\
				1      & 1      & \dots  & 1      & 1      \\
				1      & 1      & \dots  & 1      & 1      \\
			\end{pmatrix}_{V_1 \times V_2}, &
			P_{21} = p_{21}\begin{pmatrix}
				1      & 1      & \dots  & 1      & 1      \\
				1      & 1      & \dots  & 1      & 1      \\
				\vdots & \vdots & \ddots & \vdots & \vdots \\
				1      & 1      & \dots  & 1      & 1      \\
				1      & 1      & \dots  & 1      & 1      \\
			\end{pmatrix}_{V_2 \times V_1},
		\end{split}
	\]
	where each block matrix is uniform within the block. In the typical set up, we will have \(p_{11}, p_{22} \gg p_{12}\). This creates a community structure such that
	\begin{itemize}
		\item \(p_{11}\) governs connectivity in community \(1\).
		\item \(p_{22}\) governs connectivity in community \(2\).
		\item \(p_{12}\) governs the \emph{cross} connectivity structure.
	\end{itemize}

	\begin{remark}[Bipartite graph generalization]
		If we instead set
		\[
			p_{12}\gg p_{11}, p_{22},
		\]
		The generated random \hyperref[def:graph]{graph} generalizes to multiple communities by an appropriate block structure.

		\begin{figure}[H]
			\centering
			\incfig{bipartite-graph-generalization}
			\caption{For \(p_{12}=0\) and \(p_{12}, p_{22}>0\)}
			\label{fig:bipartite-graph-generalization}
		\end{figure}

		Also, this relates to the community detection algorithm. We look at the \hyperref[def:Laplacian]{Laplacian} \(L = D - A\) and find \hyperref[thm:eigen-decomposition]{spectral decomposition} and find the smallest \hyperref[def:eigenvalue]{eigenvalues} and \hyperref[def:eigenvector]{eigenvectors} to study community structure. See \autoref{sec:network-decomposition} for detailed discussion.
	\end{remark}
\end{eg}

\section{Analysis of Erdős-Rényi Random Graphs Family}
\begin{prev}
	To summarize all cases we have seen, we have
	\begin{itemize}
		\item Uniform case. With \(p\in(0, 1)\),
		      \begin{itemize}
			      \item \(p = 0\) - \(V\) has isolated nodes.
			      \item \(p = 1\) - \hyperref[def:complete-graph]{Complete graph}.
		      \end{itemize}
		\item Non-uniform case. Some values can be \(1\) or \(0\), but not all.
	\end{itemize}
\end{prev}

Now, let's look at the properties follows the above discussion. In particular, we consider the \hyperref[subsec:uniform-and-directed-model]{uniform directed} case with \(p\in(0, 1)\). We then see that each \hyperref[def:undirected-graph]{undirected} edge is chosen with probability \(p\). For node \(i\), we have
\begin{itemize}
	\item The \hyperref[def:degree]{degree} is associated with
	      \(\mathrm{Binomial}(V-1, p)\) such that
	      \[
		      \Pr(\deg(i) = k) = \binom{V-1}{k}p^k (1 - p)^{V-1-k}.
	      \]
	\item The mean \hyperref[def:degree]{degree} is just \((V-1)p\).
	\item The total number of edges is associated with \(\mathrm{Binomial}\left(\frac{V(V-1)}{2}, p\right)\).
	\item The mean number of edges is \(\frac{V(V-1)}{2}p\).
\end{itemize}

Now, we can discuss different regimes.

\subsection{Extremely Sparse Regime}\label{subsec:extremely-sparse-regime}
With \(p = \min\left\{c / V, 1\right\}\) for some \(c>0\). Then for node \(i\), we have
\begin{itemize}
	\item The \hyperref[def:degree]{degree} is associated with \(\mathrm{Binomial}(V - 1, \frac{c}{V}) \to \mathrm{Poisson}(c) \) as \(V\to \infty \).
	\item The mean \hyperref[def:degree]{degree} is \(c \frac{V-1}{V}\to c\) as \(V\to \infty.\)
	      \begin{note}[Poisson distribution]
		      Recall the Poisson distribution is defined as
		      \[
			      \Pr( \mathrm{Poisson}(c)=k ) = e^{-c} \frac{c^k}{k!}, \qquad k = 0, 1, \dots
		      \]
	      \end{note}
	      We see that for an isolated node \(i\), \(\Pr( \deg(i) =0) = e^{-c} > 0\), hence as \(V\to \infty \), \(e^{-c}\) fraction of nodes will be isolated since
	      \[
		      \Pr( \text{isolated nodes}) = (1 - p)^{V - 1} = \left(1 - \frac{c}{v}\right)^{V-1} \to e^{-c},
	      \]
	      where \(\lim_{n \to \infty} \left(1 - \frac{1}{x}\right)^x = e^{-1}\).
	\item The total number of edges is associated with \(\mathrm{Binomial}\left(\frac{V(V-1)}{2}, \frac{c}{V}\right)\).
	\item The mean number of edges is \(\frac{V(V - 1)}{2}\frac{c}{V} = \frac{c(V - 1)}{2}\sim \Theta(\frac{c}{2}V)\), which is linear in \(V\).
\end{itemize}

\begin{remark}
	We can further divide this cases into following subcases:
	\begin{enumerate}
		\item Sub-critical regime with \(0<c<1\). Essentially \textbf{isolated nodes}. The largest \hyperref[def:connected]{connected} component will have size \(O(\log V)\) as \(V\to \infty \), and the number of \hyperref[def:connected]{connected} components will be
		      \[
			      \Omega\left(\frac{V}{\log V}\right)
		      \]
		      as \(V\to \infty \).
		      \begin{figure}[H]
			      \centering
			      \incfig{sub-critical-regime}
			      \caption{Sub-critical regime, at least some islands are \(O(\log V)\) size}
			      \label{fig:sub-critical-regime}
		      \end{figure}
		\item Super-critical regime with \(c>1\). This essentially leads to a (single) \hyperref[def:giant-component]{giant component}. There will exist a \(c_1\) associated with the largest component such that
		      \[
			      \left\vert c_1 \right\vert \sim f(c_1)V, \qquad f\in(0, 1)
		      \]
		      where \(f\) is an increasing function of \(c\) with \(f(0) = 0\) and \(\lim\limits_{c\to \infty }f(c) = 1\). And since \(e^{-c}>0\), so isolated nodes exist. Further, the second-largest component will have
		      \[
			      \left\vert c_2 \right\vert \sim O(\log V),
		      \]
		      while all other components are very small.
		      \begin{figure}[H]
			      \centering
			      \incfig{super-critical-regime}
			      \caption{Super-critical regime, \(e^{-c}>0\), so isolated nodes exist}
			      \label{fig:super-critical-regime}
		      \end{figure}
		\item Critical regime such that \(c = 1\). We see that \(\left\vert c_1 \right\vert \sim \Theta(V^{\frac{2}{3}})\), \(\left\vert c_2 \right\vert O(\log V)\).
	\end{enumerate}
	\begin{remark}[Branching processes]
		These analyses are using so-called \href{https://en.wikipedia.org/wiki/Branching_process}{\emph{branching process}}.
		\begin{figure}[H]
			\centering
			\incfig{branching-processes}
			\caption{Branching processes.}
			\label{fig:branching-processes}
		\end{figure}

		Let \(c\) be the mean number of children. Then we have
		\begin{itemize}
			\item \(c<1\): dies out fast
			\item \(c>1\): can keep growing
			\item \(c=1\): This is the critical case: dies out but goes much further compare to the first case
		\end{itemize}
	\end{remark}
\end{remark}

\subsection{Sparse Regime}\label{subsec:sparse-regime}
This is also known as connectivity regime. With \(p = \frac{c\log V + a}{V}\) where \(a\in\mathbb{R}\) and \(c>0\), more precisely, we have
\[
	p\sim \frac{c\log V + a}{V} \text{ as }V\to \infty.
\]

Asymptotically, as \(V\to \infty \), node \hyperref[def:degree]{degrees} go to infinity (not necessarily for all nodes). Now, for node \(i\), we have
\begin{itemize}
	\item The mean \hyperref[def:degree]{degree} is
	      \[
		      (V - 1)p = \frac{V-1}{V}(c\log V + a)\sim c\log V + a = \Theta(\log V).
	      \]
	\item The number of edges is
	      \[
		      \frac{V(V-1)}{2}p = (V - 1)(c\log V + a) = \Theta(V\log V).
	      \]
\end{itemize}

\begin{remark}
	Now consider following subcases.
	\begin{enumerate}
		\item \(0<c<1\): As \(V\to \infty \), there will always be isolated nodes. There will be one \hyperref[def:giant-component]{giant component} with almost all nodes.
		\item \(c>1\): As \(V\to \infty \), the \hyperref[def:graph]{graph} is \textbf{always connected}.
		\item \(c = 1\):
		      \begin{itemize}
			      \item With probability \(e^{-e^{-a}}\), there are isolated nodes.
			      \item With probability \(1 - e^{-e^{-a}}\), the \hyperref[def:graph]{graph} is \hyperref[def:connected]{connected}.
		      \end{itemize}
	\end{enumerate}
	For these \hyperref[def:graph]{graphs}, isolated nodes are the ones stopping, preventing \hyperref[def:connected]{connectivity}. The moment you eliminate isolated nodes, connectivity emerges.
\end{remark}

\subsection{Extremely Dense Regime}\label{subsec:extremely-dense-regime}
Now, if \(p\sim \Omega(\frac{\log V}{V})\), the \hyperref[def:graph]{graph} is extremely dense with \hyperref[def:connected]{connectivity}. In this case, the \hyperref[def:degree]{degree} distributions as Poisson
\[
	\Pr(\mathrm{Poisson}(c) > k) \sim e^{-g(c)k},
\]
which drop exponentially or geometrically in \(k\) since its has geometric tail.
\begin{remark}
	In practice, we have
	\[
		\Pr(\deg(i) > k) \sim \frac{1}{k^\alpha }
	\]
	for \(\alpha > 1\), which has only power-law tail.
\end{remark}