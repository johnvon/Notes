\lecture{6}{20 Sep. 12:30}{Girvan-Newman Algorithm Analysis}
\begin{problem}
But how do we calculate \hyperref[def:betweenness]{betweenness} in reality?
\end{problem}
\begin{answer}
	We see that the only obstacle we're facing is the fact that we don't know how to calculate the number of \hyperref[def:shortest-path]{shortest paths}. And to calculate how many shortest paths are there between nodes, this can be done by using \hyperref[algo:BFS]{breadth-first search} in a slightly different manner.
\end{answer}

The key is that any sub-\hyperref[def:path]{path} to destination is also a \hyperref[def:shortest-path]{shortest path}. To calculate this with an algorithmic procedure, we use \hyperref[algo:BFS]{BFS}. In the \hyperref[algo:BFS]{BFS} exploration from \(v^{\ast} \) to \(u^{\ast} \), we have
\[
	\#\text{\hyperref[def:shortest-path]{shortest path} from \(v^{\ast} \) to \(u^{\ast}\)} =  \#\text{\hyperref[def:shortest-path]{shortest path} from \(v^{\ast} \) to \(j\)} +
	\#\text{\hyperref[def:shortest-path]{shortest path} from \(v^{\ast} \) to \(k\)} + \dots
\]
suppose that there are edges between \(j, k, \dots  \) to \(u^{\ast} \).

\begin{remark}
	In this case, \(j, k\) are at same level \(n-1\), and \(u^{\ast} \) is at level \(n\) if the \hyperref[def:distance-between-nodes]{distance}
	between \(v^{\ast} \) and \(u^{\ast} \) is \(n\).
\end{remark}

Finally, to calculate this, we can add \emph{another} counter when running \hyperref[algo:BFS]{BFS}. This is certainly a different type of counter from \(c\), which we call it \(d\). This time, the counter \(d\) is induced by the above equation, and that implies the updating rule being
\[
	d_{v}^{n} = \sum_{(u, v)\in \mathcal{E} }d^{n-1}_{u}
\]
where \(d_{v}^{n}\) maintains the number of \hyperref[def:shortest-path]{shortest path} to \(v\) where \(v\) is at layer \(n\), i.e., \(d_{\mathcal{G}}(v^{\ast} , v)=n\).\footnote{Note that only those \(d_v^n\) such that \(n=d_{\mathcal{G} }(v^{\ast} , v)=n \) is defined. It's nonsense to define \(d_v^{n^\prime }\) where \(d_{\mathcal{G} }(v^{\ast} , v)\neq n^\prime\) from the meaning we're trying to put on \(d\) and maintained.}

The unit flow flows from nodes back from the \hyperref[def:path]{path} it is added into the \hyperref[algo:BFS]{BFS} algorithm needs to keep its own \(1\), and then distribute.

\begin{remark}
	By adding this counter \(d\), in the end, \(d_v\) (regardless of the supscript, since only one such \(d_v^{n^\prime }\) for any \(n^\prime \) will be defined) is the number of shortest path from \(v^{\ast} \) to \(v\). Then we can just calculate the \hyperref[def:betweenness]{betweenness} from its definition.
\end{remark}

\section{Homophily}
Links in a (social) network are formed between people that are \emph{similar}. This is an important feature that leads to the formation of communities.

\begin{problem}
How can we define the notion of \emph{like} and \emph{opposite}?
\end{problem}
\begin{answer}
	Consider a network with two types of individuals.
\end{answer}

\begin{eg}[Middle school]
	Consider a school of boys and girls. Suppose the fraction of boy is $p$ and the fraction of girls is $q = 1 - p$. We have
	\[
		\frac{\binom{np}{2}}{\binom{n}{2}} \cong p^2
	\]
	and also
	\[
		1 - p^2 - q^2 = 2pq.
	\]

	If the fraction of cross connection $\ll 2pq$, homophily is considered likely. Otherwise, if the fraction $\gg 2pq$, inverse-homophily is likely to occur.
\end{eg}

We see that some social implications arise from this\todo{This part is unclear, need to polish.}
\begin{enumerate}
	\item schelling model
	\item parameters
	      \begin{enumerate}
		      \item satisfaction threshold
		      \item population of each partition
		      \item vacancy
	      \end{enumerate}
	\item a grid of X and O (denoting partition of agents).
\end{enumerate}

\begin{itemize}
	\item Satisfied Agent: A satisfied agent is one that is surrounded by at least \(t\) percent of agents that are like itself.
	\item Dynamics: When an agent is not satisfied, it can be moved to any vacant location in the grid.
\end{itemize}

