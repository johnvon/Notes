\lecture{23}{01 Dec. 12:30}{Bayesian Nash Equilibrium}
Now we look at the expected \hyperref[def:reward]{utility}  of \hyperref[def:player]{agent} \(i\) over other \hyperref[def:player]{agents}' valuation and bids. From the monotonically increasing property of \(\sigma(\cdot)\), the expected \hyperref[def:reward]{utility}  of \(i\) is
\[
	\begin{split}
		\mathbb{E}_{-i}\left[u_{i}(v_{i}, \sigma_{i}(v_{i}), v_{-i}, \sigma_{-i}(v_{-i})) \right]
		= &\mathbb{E}_{-i}\left[(v_{i} - \sigma_{i}(v_{i})) \prod\limits_{j\in -i}\mathbbm{1}_{\{ \sigma_{i}(v_{i})\geq \sigma_{j}(v_{j}) \}} \right]\\
		= &(v_{i} - \sigma_{i}(v_{i}))\prod\limits_{j\in -i}\Pr(\sigma_{j}(v_{j})\leq \sigma_{i}(v_{i}))\\
		= &(v_{i} - \sigma_{i}(v_{i}))\prod\limits_{j\in -i}\Pr(v_{j}\leq v_{i})\\
		= &(v_{i} - \sigma_{i}(v_{i}))F^{I-1}(v_{i})\\
		= &(v_{i} - \sigma_{i}(v_{i}))v_{i}^{I-1}.
	\end{split}
\]

Now the question is how should an \hyperref[def:player]{agent} bid? Namely, how to find \(\sigma_{i}(\cdot)\)?
\subsection{Reason of Response}
To reason about the bidding strategy, we first see an important principle.
\begin{theorem}[Revelation Principle]\label{thm:revelation-principle}
	Let \(\sigma\) be the \hyperref[def:Nash-equilibrium]{equilibrium} bidding function. Then any deviation can be decomposed into two steps:
	\begin{enumerate}[(a)]
		\item \(v_{i}\to v\): Map value to same \textbf{false} value.
		\item Apply \(\sigma\) to the \textbf{false} value, so that the bid is \(\sigma(v)\).
	\end{enumerate}
\end{theorem}

\begin{intuition}
	\autoref{thm:revelation-principle} simply tells us that we don't need to perturb \(\sigma\) itself to analyze incentive of deviation, we can simply perturb \(v\) to achieve the same result.
\end{intuition}

\begin{remark}[Equilibrium bidding]\label{rmk:equilibrium-bidding}
	The \emph{\hyperref[def:Nash-equilibrium]{equilibrium} bidding} is
	\begin{itemize}
		\item Step 1: \(v_{i}\to v_{i}\): identical mapping.
		\item Step 2: The same, so bid is \(\sigma(v_{i})\).
	\end{itemize}
\end{remark}

Assume a deviation of \(v_{i}\to v\) and the bid is \(\sigma(v)\). Then the \hyperref[def:reward]{utility}  becomes
\[
	\begin{split}
		u_{i} &= (v_{i} - \sigma(v))\Pr( \text{Winning with bid } \sigma(v))\\
		&= (v_{i} - \sigma(v))\prod\limits_{j\in -i}\Pr(\sigma(v_{j})\leq \sigma(v))\\
		&= (v_{i} - \sigma(v))F^{I-1}(v)\\
		&= (v_{i} - \sigma(v))v^{I-1}\\
		&\eqqcolon f_{\sigma}(v; v_{i}).
	\end{split}
\]

Then the \hyperref[def:best-response]{best response} calculation for \(i\) is as follows. Given a \(\sigma\), maximize \hyperref[def:reward]{utility} by properly choosing \(v\) for each \(v_{i}\). We simply differentiate the function and set the value be zero
\[
	\frac{\mathrm{d}}{\mathrm{d}v}f_{\sigma}(v;v_{i}) = v_{i}(I - 1)v^{I-2} - (I - 1)v^{I-2}\sigma(v) - \sigma^1(v)v^{I-1} \coloneqq 0
\]
for the \underline{first order condition}. It's equivalent to solve the following O.D.E.:
\[
	v_{i}(I - 1)v^{I-2} = \frac{\mathrm{d}}{\mathrm{d}v} (\sigma(v)v^{I-1}).
\]

We see that \(\sigma(v)\leq v_{i}\) is needed as you never bid higher than your valuation in a \hyperref[eg:first-price-auction]{first-price auction}.

For the \hyperref[rmk:equilibrium-bidding]{equilibrium bidding} \(v_{i}\to v_{i}\), the maximizer should be \(v_{i}\) itself. For the \hyperref[rmk:equilibrium-bidding]{equilibrium bidding} function \(\sigma^{\ast}\),
\[
	\at{\frac{\mathrm{d}}{\mathrm{d}v} f_{\sigma^{\ast}}(v;v_{i})}{v = v_{i}}{} = 0.
\]
Expand it out, we have
\[
	\frac{\mathrm{d}}{\mathrm{d}v} f_{\sigma^{\ast}}(v;v_{i}) = v_{i}(I - 1)v^{I-2} - \left((I - 1)v^{I-2}\sigma^{\ast}(v) + \frac{\mathrm{d}\sigma^{\ast}(v)}{\mathrm{d}v} v^{I-1}\right).
\]
By setting \(v = v_{i}\), we should get zero!

\[
	(I - 1)v_{i}^{I-1} = (I - 1)v_{i}^{I-2}\sigma^{\ast}(v_{i})+\at{\frac{\mathrm{d}\sigma^{\ast}(v)}{\mathrm{d}v}}{v = v_{i}}{}{v_{i}}^{I-1}.
\]
This equation has to hold for every \(v_{i}\) in \([0, 1]\).

Now, for a \(\overline{v}\), \(\sigma^{\ast}(\overline{v})\) should be
\[
	(I - 1)v_{i}^{I-1} = \frac{\mathrm{d}}{\mathrm{d}v_{i}} \left(v_{i}^{I-1}\sigma^{\ast}(v_{i})\right).
\]
Integrate on both sides, we have
\[
	\int_0^{\overline{v}}(I - 1)v_{i}^{I-1}\,\mathrm{d} v_{i} = \at{v_{i}^{I-1}\sigma^{\ast}(v_{i})}{0}{\overline{v}}.
\]
Solve for the left-hand side, we have
\[
	\frac{I - 1}{I}\overline{v}^I = \overline{v}^{I-1}\sigma^{\ast}(\overline{v}).
\]

Hence,
\[
	\sigma^{\ast}(\overline{v}) = \frac{I-1}{I}\overline{v} = \left( 1 - \frac{1}{I} \right) \overline{v},
\]
which means that \hyperref[def:player]{agent} bids \(\frac{I - 1}{I}\) times the value who estimate!

\begin{note}
	We see that
	\begin{itemize}
		\item \hyperref[def:player]{Agents} will shade their bid, hence it's not truthful.
		\item The good goes to the \hyperref[def:player]{agent} with the highest valuation, hence it's a \hyperref[def:social-welfare]{social-welfare} maximizer.
	\end{itemize}
\end{note}

\begin{remark}
	If an \hyperref[def:player]{agent} whose valuation is zero, then from the function, we see that he would bid zero, hence the \hyperref[def:reward]{utility} is zero as well.
\end{remark}

\subsection{Revenue of the Auctioneer}
The expected revenue for the auctioneer is equal to
\[
	\frac{I - 1}{I}\mathbb{E}\left[\widetilde{v}_I \right]  = \frac{I - 1}{I}\frac{I}{I+1} = \frac{I - 1}{I+1}.
\]

\begin{note}
	Recall that the expected revenue for \hyperref[eg:second-price-auction]{second-price auction} for the auctioneer is also
	\[
		\mathbb{E}\left[\widetilde{v}_{I-2} \right] = \frac{I-1}{I+1}.
	\]
	We see that both auctions give the same expected revenue.
\end{note}

\begin{remark}[Revenue equivalence]\label{rmk:revenue-equivalence}
	If two auctions implement the same outcome in \hyperref[def:mathematical-Bayesian-game]{Bayesian} \hyperref[def:Nash-equilibrium]{Nash equilibrium}, and \hyperref[def:player]{agents} with value \(0\) will bid \(0\), then the expected revenue will be the same. This is so-called \emph{revenue equivalence}. And this equivalence relation holds for more general kinds of auctions, namely under some conditions of auctions, the revenue for auctioneer is always the same. We'll come back to this relation soon.
\end{remark}

\section{Mechanism Design of Auctions}
After seeing two kinds of auctions, we now try to give a general framework of auctions and try to design a mechanism of such \hyperref[def:mathematical-Bayesian-game]{games}. Assume that the auctioneer doesn't know values of \hyperref[def:player]{agents}.

\begin{problem}
How the auction will be conducted?
\end{problem}
\begin{answer}
	Given \(b_1, \dots , b_I\), we need to answer the following two questions:
	\begin{enumerate}[(a)]
		\item Who wins the auction?
		\item What that person pays?
	\end{enumerate}
	The mechanism design of auctions can be though as answering these two questions.

	\begin{eg}[All-pay auction]
		We first see an example called \emph{all-pay auction}. For bids \(b_1, \dots , b_I\),
		\begin{itemize}
			\item The highest bidder wins.
			\item Everyone pays their bid.
		\end{itemize}
	\end{eg}
\end{answer}

For a general auction, the bids are just messages. Auctioneer is specifying two things.
\begin{itemize}
	\item Allocation rule: \((x_1, \dots , x_I)\), where \(x_{i}\) is the amount of good goes to \hyperref[def:player]{agent} \(i\).
	\item Payment rule: \((p_1, \dots , p_I)\), where \(p_{i}\) to be paid by \hyperref[def:player]{agent} \(i\).
\end{itemize}

This suggests the following definition.
\begin{definition}[Direct mechanism]\label{def:direct-mechanism}
	In the specification, if the messages are in the same space as values, then we call such a mechanism as a \emph{direct mechanism}.
\end{definition}

\begin{remark}
	From \hyperref[thm:revelation-principle]{revelation principle}, \hyperref[def:direct-mechanism]{direct mechanisms} are sufficient. Since bids are messages, which can be anything, hence it suffices to consider messages to be in the same space as the values.
\end{remark}

Hence, we assume that the message space is the same as value space, namely we are designing a \hyperref[def:direct-mechanism]{direct mechanism}. For every message vector, specifies the allocation rule and the payment rule. Furthermore, we let
\[
	\text{\hyperref[def:reward]{utility}} = \text{Value} - \text{Payment},
\]
which turns out to be so-called \href{https://en.wikipedia.org/wiki/Quasilinear_utility}{quasilinear utility}.

\begin{note}
	There are things that a mechanism designer may want.
	\begin{itemize}
		\item \hyperref[rmk:incentive-compatible]{Incentive compatible} (IC).
		\item \hyperref[rmk:voluntary-participation]{Individual rationality} (IR).
	\end{itemize}

	Also, a mechanism designer may have a goal in mind.
	\begin{itemize}
		\item \hyperref[def:social-welfare]{Social welfare} maximization (Efficiency).
		\item Revenue maximization.
		\item Budget-Balance: Payments should sum to \(0\).
	\end{itemize}

	\begin{remark}
		All can't be met simultaneously.
	\end{remark}
\end{note}

Assume every \hyperref[def:player]{agent} \(i\) values \(v_1, \dots , v_{I}\), and bids \(b_1, \dots , b_{I}\) respectively. With the specification being
\[
	\begin{split}
		\text{Allocations}&: x_1, \dots , x_I;\\
		\text{Payments}&:p_1, \dots , p_I,
	\end{split}
\]
where all are functions of \(b_{i}\). Then the \hyperref[def:reward]{utility} is
\[
	u_{i}(v_{i}, n_{i}(\vec{v})) - p_{i}(\vec{v}).
\]
With \hyperref[def:interim]{interim} setup, we average across all other \hyperref[def:player]{agents}. Now, assume that \(x_{i}(\vec{v})\in\{0, 1\}\) and will allow multiple \hyperref[def:player]{agents} to get
\[
	x_{i}(\vec{v}) = 1.
\]

If \(x_{i}(\vec{v})=1\), then \hyperref[def:reward]{utility}  is \(v_{i}\), else \(0\). We then see that
\[
	u_{i}(v_{i}, x_{i}(\vec{v})) = v_{i}x_{i}(\vec{v}) - p_{i}(\vec{v}).
\]
Then the expected value is
\[
	\begin{split}
		\mathbb{E}_{-i}\left[u_{i}(v_{i}, x_{i}(\vec{v})) \right]  &= v_{i} \mathbb{E}_{-i}\left[x_{i}(\vec{v})\right] - \mathbb{E}_{-i}\left[p_{i}(\vec{v})\right]\\
		&= v_{i} \mathbb{E}_{-i}\left[x_{i}(v_{i}, v_{-i})\right] - \mathbb{E}_{-i}\left[p_{i}(v_{i}, v_{-i})\right]\\
		&= v_{i} \overline{x}_{i}(v_{i})-\overline{p}_{i}(v_{i}).
	\end{split}
\]

Now we see a theorem.
\begin{theorem}[Bayesian Nash equilibrium characterization]\label{thm:Bayesian-Nash-equilibrium-characterization}
	When the \hyperref[def:type]{types} are drawn from a continuous joint distribution \(\vec{F}\) with \hyperref[def:independent]{independent} \hyperref[def:type]{types}, \hyperref[def:strategy]{strategies}
	\[
		\vec{\sigma}^{\ast} = (\sigma^{\ast}_1, \sigma^{\ast}_2, \dots , \sigma^{\ast}_I)\qquad (b_{i} = \sigma^{\ast}_{i}(v_{i}))
	\]
	are in a \hyperref[def:mathematical-Bayesian-game]{Bayesian} \hyperref[def:Nash-equilibrium]{Nash equilibrium} if and only if
	\begin{itemize}
		\item Monotonicity. \(\overline{x}_{i}(v_{i})\) is monotonically non-decreasing.
		\item Payment identity.
		      \[
			      \overline{p}_{i}(v_{i}) = v_{i}\overline{x}_{i}(v_{i}) - \int_0^{v_{i}} \overline{x}_{i}(z)\,\mathrm{d}z + \overline{p}_{i}(0),
		      \]
		      where often \(\overline{p}_{i}(0) = 0\).
	\end{itemize}
	If the \hyperref[def:strategy]{strategy} profile is onto, the converse is true as well.
\end{theorem}