\chapter{Additional Proofs}\label{Apx:Additional-Proofs}
\section{\hyperref[ch:Measure]{Measure}}
This section gives all additional proofs in \autoref{ch:Measure}.
\begin{theorem}[\autoref{thm:Caratheodory-extension-Thm} 3.]\label{thm:Caratheodory-extension-Thm:3.}
	Under the setup of \autoref{thm:Caratheodory-extension-Thm}, \((X, \mathcal{\MakeUppercase{a}} , \mu )\) is a \hyperref[def:complete-measure-space]{complete measure space}.
\end{theorem}
\begin{proof}
	We see this in two parts.
	\begin{claim}
		If \(A\subset X\) satisfies \(\mu ^{*} (A) = 0\), then \(A\) is \hyperref[def:C-measurable]{Carathéodory measurable} with respect to \(\mu ^{\ast} \).
	\end{claim}
	\begin{explanation}
		If \(A\subset X\) and \(\mu^{*} (A) = 0\), where \(\mu^{*} \) is an outer measure on \(X\), we'll show that \(A\) is \hyperref[def:C-measurable]{Carathéodory measurable}
		with respect to \(\mu^{*} \).

		\par Equivalently, we want to show that for any \(E\subset X\),
		\[
			\mu^{*} (E) = \mu^{*} (E\cap A) + \mu^{*} (E \setminus A).
		\]
		Firstly, noting that \((E\cap A)\subset A\), and by \hyperref[def:outer-measure-montonicity]{monotonicity} of \(\mu^{*} \), we see that
		\[
			\mu^{*} (E\cap A)\leq \mu^{*} (A) = 0,
		\]
		and since \(\mu^{*} \geq 0\), hence \(\mu^{*} (E\cap A) = 0\). Now, we only need to show that
		\[
			\mu^{*} (E) = \mu^{*} (E\setminus A).
		\]
		Since \(E\setminus A = E\cap A^{c} \), and hence we have \(E\cap A^{c} \subset E\), so
		\[
			\mu^{*} (E)\geq \mu^{*} (E\setminus A).
		\]
		To show another direction, we note that
		\[
			\mu^{*} (E)\leq \mu^{*} (E\cup A) = \mu^{*} ((E\setminus A) \cup A) \leq \mu^{*} (E\setminus A),
		\]
		hence we conclude that \(A\) is \hyperref[def:C-measurable]{Carathéodory measurable} with respect to \(\mu^{*} \) if \(\mu^{*} (A)=0\).
	\end{explanation}

	\begin{claim}
		If \(A\) is \hyperref[def:mu-subnull-set]{\(\mu\)-subnull}, then \(A\in \mathcal{\MakeUppercase{a}} \).
	\end{claim}
	\begin{explanation}
		Let \(\mathcal{A} \) denotes the \hyperref[thm:Caratheodory-extension-Thm]{Carathéodory \(\sigma\)-algebra}, and \(\mu\coloneqq \at{\mu^{*} }{\mathcal{A} }{} \). We want to show if
		\(A\) is \hyperref[def:mu-subnull-set]{\(\mu\)-subnull}, then \(A\in\mathcal{A} \).

		\par Firstly, if \(A\) is \hyperref[def:mu-subnull-set]{\(\mu \)-subnull}, then there exists a \(B\in \mathcal{A} \) such that \(A\subset B\) and \(\mu (B) = 0\). But since from
		the \hyperref[def:outer-measure-montonicity]{monotonicity} of \(\mu ^{*} \), we further have
		\[
			0 = \mu(B) = \mu ^{*} (B) \geq \mu ^{*} (A),
		\]
		hence \(\mu ^{*} (A) = 0\).

		\par  From the first claim, we immediately see that \(A\) is \hyperref[def:C-measurable]{Carathéodory measurable} with respect to \(\mu ^{*}\),
		which implies \(A\) is in \hyperref[thm:Caratheodory-extension-Thm]{Carathéodory \(\sigma\)-algebra}, hence \(A\in \mathcal{A} \).
	\end{explanation}

	We see that the second claim directly proves that \((X, \mathcal{\MakeUppercase{a}} , \mu )\) is a \hyperref[def:complete-measure-space]{complete measure space}.
\end{proof}

\begin{lemma}\label{lma:lec7-eg:4-is-distribution-function}
	The function \(F\) defined in \hyperref[eg:lec8-3]{this example} is a \hyperref[def:distribution-function]{distribution function}
\end{lemma}
\begin{proof}
	\par We define
	\[
		F_{n}(x) = \begin{dcases}
			1, & \text{ if } x\geq r_{n} ; \\
			0, & \text{ if } x<r_{n}
		\end{dcases}
	\]
	where \(\{r_1, r_2, \ldots  \}= \mathbb{\MakeUppercase{q}} \), and
	\[
		F(x) = \sum\limits_{n=1}^{\infty} \frac{F_{n}(x)}{2^n} = \sum\limits_{n;r_{n}\leq x}\frac{1}{2^n}
	\]
	is both increasing and right-continuous.

	\begin{itemize}
		\item Increasing. Consider \(x<y\). We see that
		      \[
			      F(x) = \sum\limits_{n;r_{n}\leq x} \frac{1}{2^n} \leq \sum\limits_{n;r_{n}\leq y} \frac{1}{2^n} = F(y)
		      \]
		      clearly.\footnote{This is trivial since we're always going to sum more strictly positive terms in \(F(y)\) than in \(F(x)\).}
		\item Right-continuous. We want to show \(F(x^+) = F(x)\). Let \(x^+(\epsilon )\coloneqq x + \epsilon \) with \(\epsilon >0\), we'll show that
		      \[
			      \lim\limits_{\epsilon\to 0}F(x^+(\epsilon )) =  \lim\limits_{\epsilon \to 0} F(x + \epsilon ) = F(x).
		      \]

		      Firstly, we have
		      \[
			      F(x^+(\epsilon )) - F(x) = \sum\limits_{n;r_{n}\leq x+\epsilon } \frac{1}{2^n} - \sum\limits_{n;r_{n}\leq x}\frac{1}{2^n} = \sum\limits_{n;x<r_{n}\leq x+\epsilon}\frac{1}{2^n},
		      \]
		      and we want to show
		      \[
			      \lim\limits_{\epsilon \to 0}F(x^+(\epsilon )) - F(x) = \lim\limits_{\epsilon \to 0}\sum\limits_{n;x< r_{n}\leq x+\epsilon }\frac{1}{2^n} = 0.
		      \]
		      \begin{remark}
			      The strict is crucial to show the result, since if \(x = r_k\) for some fixed \(k\), then we can't make the summation arbitrarily small.
		      \end{remark}

		      \par Before we show how we choose \(\epsilon \),\footnote{To be precise, how \(\epsilon \) depends on \(r_n\).} we see that
		      \[
			      \sum\limits_{n=k}^{\infty }\frac{1}{2^n} = 2^{1-k}.
		      \]
		      Now, we observe that
		      \[
			      \sum\limits_{n;x< r_{n}\leq x+\epsilon }\frac{1}{2^n}\leq \sum\limits_{n=\underset{k}{\arg\mathop{\min}}\ x< r_{k}\leq x+\epsilon }^{\infty }\frac{1}{2^n} = 2^{1-k}.
		      \]
		      With this observation, it should be fairly easy to see that we can choose \(\epsilon \) based on how small we want to make \(2^{1-k}\) be,\footnote{We're referring to \(\epsilon -\delta \) proof approach.}
		      and we indeed see that
		      \[
			      \lim\limits_{k \to \infty} 2^{1-k} = 0,
		      \]
		      which implies that \(F\) is right-continuous by squeeze theorem.
	\end{itemize}
\end{proof}

\begin{lemma}\label{lma:lec8-eg:3}
	The function \(F\) defined in \hyperref[eg:lec8-3]{this example} satisfies
	\begin{itemize}
		\item \(\mu _F(\{r_{i}\})>0\) for all \(r_{i}\in\mathbb{\MakeUppercase{q}} \).
		\item \(\mu _F(\mathbb{\MakeUppercase{r}} \setminus \mathbb{\MakeUppercase{q}} ) = 0\)
	\end{itemize}
	given in \hyperref[eg:lec8-3]{this example}.
\end{lemma}
\begin{proof}
	We prove them one by one. And notice that \(F\) is indeed a distribution function as we proved in \autoref{lma:lec7-eg:4-is-distribution-function}.
	\begin{enumerate}[(1)]
		\item To show \(\mu _{F}(\{r\}) > 0\) for every \(r\in \mathbb{\MakeUppercase{q}} \), we first note that
		      \[
			      \{r\} = \bigcap\limits_{a-1\leq x< r}(x, r].
		      \]
		      Then, we see that
		      \[
			      \mu _{F}(\{r\}) = \mu _{F}\left(\bigcap\limits_{a-1\leq x<a} (x, r]\right),
		      \]
		      where each \((x, r]\in\mathcal{\MakeUppercase{a}} \) and \((x, r]\supset (y, r]\) whenever \(r-1\leq x \leq y<r\). Notice that we implicitly
		      assign the order of the index by the order of \(x\). Then, we see that \(\mu _{F}(r-1, r]<\infty \).\footnote{This will be \(\mu (A_1)\) in the condition of \hyperref[thm:measure-space-continuity-from-above]{continuity from above}. Furthermore, since \(\mathbb{\MakeUppercase{q}} \) is countable, hence \(F(x)<\infty \) is promised.}
		      Then, from \hyperref[thm:measure-space-continuity-from-above]{continuity from above}, we see that
		      \[
			      \mu _{F}(\{r\}) = \lim\limits_{i \to \infty} \mu _{F}((x_{i}, r]),
		      \]
		      where we again implicitly assign an order to \(x\) as the usual order on \(\mathbb{\MakeUppercase{r}}\) by given index \(i\). It's then clear
		      that as \(i\to \infty \), \(x_{i}\to r\). From the definition of \(F\), we see that
		      \[
			      F((x_{i}, r]) = F(r) - F(x_{i}) = \sum\limits_{n;r_{n}\leq r} \frac{1}{2^n} - \sum\limits_{n;r_{n}\leq x_{i}} \frac{1}{2^n} = \sum\limits_{n;x_{i}<r_{n}\leq r}\frac{1}{2^n}.
		      \]
		      It's then clear that since \(r\in \mathbb{\MakeUppercase{q}} \), there exists an \(i ^\prime \) such that \(r_{i ^\prime } = r\). Then, we immediately see
		      that no matter how close \(x_{i}\to r\), this sum is at least
		      \[
			      \frac{1}{2^{i ^\prime }}
		      \]
		      for a fixed \(i ^\prime \). Hence, we conclude that \(\mu _{F}(\{r\}) > 0\) for every \(r\in\mathbb{\MakeUppercase{q}} \).
		\item Now, we show \(\mu _{F}(\mathbb{\MakeUppercase{r}} \setminus \mathbb{\MakeUppercase{q}} ) = 0\). Firstly, we claim that
		      \[
			      \mu _{F}(\mathbb{\MakeUppercase{q}} ) = 1
		      \]
		      and
		      \[
			      \mu _{F}(\mathbb{\MakeUppercase{r}} ) = 1
		      \]
		      as well. Since \(\mu _{F}(\mathbb{\MakeUppercase{q}} ) = 1\) is clear, we note that the second one essentially follows from the fact that we can write
		      \[
			      \mathbb{\MakeUppercase{r}} = \lim\limits_{N \to \infty} \bigcup\limits_{i=1}^{N} (a-i, a+i]
		      \]
		      for any \(a\in\mathbb{\MakeUppercase{r}} \), say \(0\). From \hyperref[thm:measure-space-continuity-from-below]{continuity from below}, we have
		      \[
			      \mu _{F}\left(\bigcup\limits_{i=1}^{\infty} (-i, +i]\right) = \lim\limits_{n \to \infty} \mu _{F}((-n, n]) = \sum\limits_{n;r_{n}\in\mathbb{\MakeUppercase{q}} }\frac{1}{2^n} = 1.
		      \]

		      \par Given the above, from countable additivity of \(\mu _{F}\), we have
		      \[
			      \mu _{F}(\mathbb{\MakeUppercase{r}} \setminus \mathbb{\MakeUppercase{q}} ) + \underbrace{\mu _{F}(\mathbb{\MakeUppercase{q}} )}_{1} = \underbrace{\mu _{F}(\mathbb{\MakeUppercase{r}} )}_{1} \implies \mu _{F}(\mathbb{\MakeUppercase{r}} \setminus \mathbb{\MakeUppercase{q}} ) = 0
		      \]
		      as we desired.
	\end{enumerate}
\end{proof}

\begin{lemma}[Cantor set has measure \(0\)]\label{lma:Cantor-set-has-measure-0}
	Let \(C\) denotes the \hyperref[eg:lec8:Cantor-set]{middle thirds Cantor set}, then the \hyperref[def:Lebesgue-measure]{Lebesgue measure} of \(C\) is \(0\). i.e.,
	\[
		m(C) = 0.
	\]
\end{lemma}
\begin{proof}
	Since we're removing \(\frac{1}{3}\) of the whole interval at each \(n\), we see that the measure of those removing parts, denoted by \(r\), is
	\[
		m(r) = \sum\limits_{n=1}^{\infty} \frac{2^{n-1}}{3^n} = \frac{1}{2}\sum\limits_{n=1}^{\infty} \left(\frac{2}{3}\right)^n = 1.
	\]
	Then, by \hyperref[def:measure]{countable additivity} of \(m\), we see that
	\[
		m(C) = m([0, 1]) - m(r) = 1 - 1 = 0.
	\]
\end{proof}

\section{\hyperref[ch:Integration]{Integration}}

\section{\hyperref[ch:Product-Measure]{Produce Measure}}

\section{\hyperref[ch:Differentiation-on-Euclidean-Space]{Differentiation on Euclidean Space}}

\section{\hyperref[ch:Normed-Vector-Space]{Normed-Vector-Space}}

\section{\hyperref[ch:Signed-and-Complex-Measures]{Signed and Complex Measures}}
\subsection{Uniqueness of \autoref{thm:nbv-measures}}\label{pf:thm:nbv-measures-uniqueness}
\begin{claim}
	We show that if \(F\in NBV\), then the unique
\end{claim}
\begin{proof}[Proof of Uniqueness in \autoref{thm:nbv-measures}]
	We observe that
	\[
		\begin{split}
			\mu ((-\infty , x]) &= \mu^+ ((-\infty , x]) - \mu ^- ((-\infty , x])\\
			&= \nu ^+((-\infty , x]) - \nu ^-((-\infty , x]) = \nu ((-\infty , x]).
		\end{split}
	\]
	Then, since we know that \(\mu ^\pm\) is uniquely defined, hence we define
	\[
		F^\pm (x) \coloneqq  \mu ^\pm ((-\infty , x]),\quad G^\pm (x) \coloneqq \nu ^\pm ((-\infty , x]).
	\]
	Then we see that
	\[
		F^+ + G^- = G^+ + F^-,
	\]
	while both left-hand side and right-hand sides are right-continuous and increasing. We hence associate it with the Lebesgue-Stieljes measure, which is
	uniquely defined, denoted it as \(\lambda \). Hence, we see that
	\[
		\lambda ((a, b]) = (F^+ + G^-)(b) - (F^+ + G^-)(a) = (G^+ + F^-)(b) - (G^+ + F^-)(a)
	\]
	which implies
	\[
		\begin{split}
			(\mu ^+((-\infty , b]) + \nu ^-((-\infty &, b])) - (\mu ^+((-\infty , a]) + \nu ^-((-\infty , a]))\\
			= &(\nu ^+((-\infty , b]) + \mu ^-((-\infty , b])) - (\nu ^+((-\infty , a]) + \mu ^-((-\infty , a])),
		\end{split}
	\]
	i.e.,
	\[
		\mu ^+((a, b]) + \nu ^-((a, b]) = \nu ^+((a, b]) + \mu ^-((a, b])
	\]
	which further implies
	\[
		\underbrace{\mu ^+((a, b]) - \mu ^-((a, b])}_{\mu ((a, b])} = \underbrace{\nu ^+((a, b]) - \nu ^-((a, b])}_{\nu ((a, b])}.
	\]
	Since this holds for all \((a, b]\), while \((a, b]\) generates \(\mathcal{\MakeUppercase{b}} (\mathbb{\MakeUppercase{r}} )\), hence
	we have that \(\mu = \nu \).
\end{proof}
\section{\hyperref[ch:Hilbert-Spaces]{Hilbert Spaces}}

\section{\hyperref[ch:Introduction-to-Fourier-Analysis]{Introduction to Fourier Analysis}}