\lecture{16}{11 Feb.\ 11:00}{Modes of Convergence}
Let's start with a proposition.

\begin{proposition}[Fast \(L^1\) convergence leads to a.e. convergence]\label{prop:lec16:1}
	Let \((X, \mathcal{A} , \mu )\) be a \hyperref[def:measure-space]{measure space}, and \(f_{n}, f\) are all \hyperref[def:measurable-function]{measurable}
	functions on \(X\). Then
	\[
		\sum_{n=1}^{\infty} \left\lVert f_{n} - f\right\rVert_1 < \infty \implies \hyperref[def:converge-almost-everywhere]{f_{n}\to f\ \text{ a.e.}}
	\]
\end{proposition}
\begin{proof}
	Let
	\[
		E \coloneqq \bigcup_{k=1}^{\infty} \bigcap_{N=1}^{\infty} \bigcup_{n=N}^{\infty} B_{n, k}^{c} = \{x\in X \mid f_{n}(x)\nrightarrow f(x)\}.
	\]
	By \autoref{lma:Markov-inequality}, we see that
	\[
		\underset{k}{\forall }\ \underset{N}{\forall }\ \mu \left(B_{n, k}^{c} \right) \leq k \int \left\vert f_{n} - f \right\vert \implies \underset{k}{\forall }\ \mu \left(\bigcup_{n=N}^{\infty} B_{n, k}^{c} \right)\leq \sum_{n=N}^{\infty} k\left\lVert f_{n} - f\right\rVert_1 \to 0
	\]
	as \(N\to \infty \). Now, by \hyperref[thm:measure-space-continuity-from-above]{continuity of measure from above},
	\[
		\underset{k}{\forall }\ \mu \left(\bigcap_{N=1}^{\infty} \bigcup_{n=N}^{\infty} B_{n, k}^{c} \right)= \lim_{N \to \infty} \mu \left(\bigcup_{n=N}^{\infty} B_{n, k}^{c} \right) = 0 \implies \mu (E) = 0
	\]
	since \(f_{n}\to f\) \hyperref[def:pointwise-convergence]{pointwise} on \(E^{c} \).
\end{proof}

\begin{corollary}
	Given \(\{f_{n}\}_n\) such that \hyperref[def:converge-in-L-1]{\(f_{n}\to f\) in \(L^1\)}, there exists a subsequence \(\{f_{n_{j}}\}_{n_{j}}\) where
	\hyperref[def:converge-almost-everywhere]{\(f_{n_{j}}\to f\) a.e.}
\end{corollary}
\begin{proof}
	Since
	\[
		\underset{j\in \mathbb{N} }{\forall }\ \underset{n_{j}\in \mathbb{N} }{\forall }\ \left\lVert f_{n_{j}} - f\right\rVert _1 \leq \frac{1}{j^{2} }.
	\]
	Then,
	\[
		\sum_{j=1}^{\infty} \left\lVert f_{n_{j}} - f\right\rVert _1 < \infty.
	\]
	From \autoref{prop:lec16:1}, we have the desired result.
\end{proof}

\begin{definition}[Converge in measure]\label{def:converge-in-measure}
	Let \(f_{n}, f\) be \hyperref[def:measurable-function]{measurable functions} on \((X, \mathcal{A} , \mu )\). Then
	\emph{\(f_{n}\to f\) in measure} if
	\[
		\underset{\epsilon >0}{\forall }\ \lim_{n \to \infty} \mu \left(\left\{x\in X \mid \left\vert f_{n}(x) - f(x) \right\vert \geq \epsilon \right\}\right) = 0.
	\]
\end{definition}
\begin{eg}
	Let \(f_{n} = n \mathbbm{1}_{(0, \frac{1}{n})} \) and \(f = 0\) on \((\mathbb{R} , \mathcal{B}(\mathbb{R} ), m)\),
	then \hyperref[def:converge-in-measure]{\(f_{n}\to f\) in measure}.
\end{eg}
\begin{explanation}
	We see that
	\[
		\forall \epsilon >0\ \left\{x\in X \mid \left\vert f_{n}(x) - f(x) \right\vert > \epsilon \right\} = \left(0, \frac{1}{n}\right),
	\]
	\hyperref[def:converge-in-measure]{\(f_{n}\to 0\) in measure}. (Recall that \(f_{n}\nrightarrow 0\) in \(L^1\))
\end{explanation}
\begin{remark}
	We see that
	\[\begin{tikzcd}
			& {\hyperref[def:converge-in-L-1]{f_n\to f \text{ in }L^1}} & {\hyperref[def:converge-in-measure]{f_n\to f \text{ in measure}}} \\
			{\hyperref[prop:lec16:1]{f_n\to f \text{ fast in } L^1}} \\
			& {\hyperref[def:converge-almost-everywhere]{f_n\to f\text{ a.e.}}} & {\exists \hyperref[def:converge-almost-everywhere]{f_{n_j}\to f \text{ a.e.}}}
			\arrow[Leftrightarrow, "/"{marking}, from=1-2, to=3-2]
			\arrow[Rightarrow, "{\text{Check}}", shift left=1, from=1-2, to=1-3]
			\arrow[Rightarrow, "/"{marking}, shift left=1, from=1-3, to=1-2]
			\arrow[Leftrightarrow, "/"{marking}, from=1-3, to=3-2]
			\arrow[Rightarrow, from=2-1, to=1-2]
			\arrow[Rightarrow, from=2-1, to=3-2]
			\arrow[Rightarrow, "{\text{Read}}", from=1-3, to=3-3]
		\end{tikzcd}\]
\end{remark}

Finally, we have the following.
\begin{definition*}
	Let \(f_{n}, f\) be \hyperref[def:measurable-function]{measurable functions} on \((X, \mathcal{A} , \mu )\).
	\begin{definition}[Uniformly almost everywhere]\label{def:uniformly-almost-everywhere}
		\(f_{n}\to f\) \emph{uniformly almost everywhere} if \(\exists \)\hyperref[def:mu-null-set]{null set} \(F\) such that
		\(f_{n}\to f\) \hyperref[def:uniformly-convergence]{uniformly} on \(F^{c} \).
	\end{definition}
	\begin{definition}[Almost uniformly]\label{def:almost-uniformly}
		\(f_{n}\to f\) \emph{almost uniformly} if \(\forall \epsilon >0\ \exists F\in \mathcal{A} \) such that
		\(\mu (F)<\epsilon \), \(f_{n}\to f\)  \hyperref[def:uniformly-convergence]{uniformly} on \(F^{c} \).
	\end{definition}
\end{definition*}

\begin{lemma}
	We have
	\[
		f_{n}\to f \text{ \hyperref[def:uniformly-convergence]{uniformly} on } S \iff \exists N_1, N_2, \dots\in \mathbb{N}\ S\subset \bigcap_{k=1}^{\infty} \bigcap_{n=N_k}^{\infty} B_{n, k}.
	\]
\end{lemma}

\begin{theorem}[Egorov's Theorem]\label{thm:Egorov-theorem}
	Let \(f_{n}, f\) be \hyperref[def:measurable-function]{measurable functions} on \((X, \mathcal{A} , \mu )\). Suppose \(\mu (X)<\infty \), then
	\[
		f_{n}\to f\ \hyperref[def:mu-almost-everywhere]{a.e.} \iff f_{n}\to f\text{ \hyperref[def:almost-uniformly]{almost uniformly}}.
	\]
\end{theorem}
\begin{proof}
	We prove two directions.
	\paragraph{\((\impliedby)\)}\todo{DIY}

	\paragraph{\((\implies )\)}
	Fix \(\epsilon > 0\). We see that
	\[
		\begin{split}
			f_{n}\to f\ \hyperref[def:mu-almost-everywhere]{a.e.}
			\implies \mu \left(\bigcup_{k=1}^{\infty} \bigcap_{N=1}^{\infty} \bigcup_{n=N}^{\infty} B_{n, k}^{c} \right) = 0
			\implies \underset{k}{\forall}\ \mu \left(\bigcap_{N=1}^{\infty} \bigcup_{n=N}^{\infty} B_{n, k}^{c} \right) = 0.
		\end{split}
	\]
	From \hyperref[thm:measure-space-continuity-from-above]{continuity of measure from above} and \(\mu (X)<\infty \), we further have
	\[
		\underset{k}{\forall}\ \lim_{N \to \infty} \mu \left(\bigcup_{n=N}^{\infty} B_{n, k}^{c} \right) = 0 \implies \underset{k}{\forall}\ \underset{ N_{k}\in\mathbb{N}}{\exists}\ \mu \left(\bigcup_{n=N_{k}}^{\infty} B_{n, k}^{c} \right)< \frac{\epsilon}{2^k}.
	\]
	Now, let
	\[
		F \coloneqq \bigcup_{k=1}^{\infty} \bigcup_{n=N_{k}}^{\infty} B_{n, k}^{c} ,
	\]
	we see that \(\mu (F) < \epsilon \), hence \(f_{n}\to f\) \hyperref[def:uniformly-convergence]{uniformly}.
\end{proof}

\chapter{Product Measure}\label{ch:Product-Measure}
\section{Product \(\sigma\)-algebra}
Before we start, we see the setup.
\begin{itemize}
	\item Product space.
	      \[
		      X = \prod_{\alpha \in I} X_\alpha
	      \]
	      where \(x = (x_\alpha )_{\alpha \in I}\in X\).
	\item Coordinate map.
	      \[
		      \pi _\alpha \colon X\to X_\alpha.
	      \]
\end{itemize}

Now we see the formal definition.
\begin{definition}[Product \(\sigma\)-algebra]\label{def:product-sigma-algebra}
	Let \((X_\alpha , \mathcal{A} _\alpha )\) be a \hyperref[def:measurable-space]{measurable space} for all \(\alpha \in I\).
	Then a \emph{product \(\sigma\)-algebra} on \(X = \prod_{\alpha \in I} X_\alpha \) is
	\[
		\bigotimes_{\alpha \in I}\mathcal{A} _\alpha = \left< \bigcup_{\alpha \in I} \pi ^{-1} _{\alpha }\left(\mathcal{A} _\alpha \right)  \right>,
	\]
	where \(\pi ^{-1} _\alpha \left(\mathcal{A} _\alpha \right) = \left\{\pi ^{-1} _\alpha (E) \mid E\in \mathcal{A} _\alpha \right\}\).
\end{definition}

\begin{notation}
	We denote \(I = \{1, \dots , d\} \implies X = \prod_{i=1}^{d} X_{i}, x = (x_1, \dots , x_d )\). Also,
	\[
		\bigotimes_{i = 1}^d \mathcal{A} _{i} = \mathcal{A} _1\otimes \dots \otimes \mathcal{A} _d.
	\]
\end{notation}

\begin{lemma}
	If \(I\) is countable, then
	\[
		\bigotimes_{i = 1}^{\infty}\mathcal{A} _{i} = \left< \left\{\prod_{i = 1}^{\infty} E_{i} \mid \underset{i}{\forall }\ E_{i}\in \mathcal{A} _{i}\right\} \right>.
	\]
\end{lemma}
\begin{proof}
	If \(E_i \in \mathcal{A} _{i}\), then \(\pi ^{-1} _{i} (E_{i} ) = \prod_{j=1}^{\infty } E_{j}\), where \(E_{j} = X_j\) for \(j\neq i\). On the other hand, since
	\[
		\prod_{i=1}^{\infty } E_{i} = \bigcap_{i=1}^{\infty } \pi ^{-1} _{i} (E_{i} ),
	\]
	from \autoref{lma:lec2-1}, the result follows.
\end{proof}