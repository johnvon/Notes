\lecture{2}{07 Jan. 11:00}{Measure}
\begin{eg}
	Again, we first see some examples.
	\begin{enumerate}
		\item Let \(\mathcal{A} = \mathcal{P} (X)\), which is the power \hyperref[def:sigma-algebra]{\(\sigma\)-algebra}.
		\item Let \(\mathcal{A} = \{\varnothing , X\}\), which is a trivial \hyperref[def:sigma-algebra]{\(\sigma\)-algebra}.
		\item Let \(B\subset X\), \(B\neq \varnothing \), \(B\neq X\). Then we see that \(\mathcal{A} = \{\varnothing , B, B^{c}, X\}\) is a \hyperref[def:sigma-algebra]{\(\sigma\)-algebra}.
	\end{enumerate}
\end{eg}

\begin{lemma}
	Let \(\mathcal{A}_{\alpha}\), \(\alpha\in I\), be a family of \hyperref[def:sigma-algebra]{\(\sigma\)-algebra} on \(X\). Then
	\[
		\bigcap\limits_{\alpha\in I} \mathcal{A}_{\alpha}
	\]
	is a \hyperref[def:sigma-algebra]{\(\sigma\)-algebra} on \(X\).
\end{lemma}
\begin{remark}
	Notice that \(I\) may be an uncountable intersection.
\end{remark}
\begin{proof}
	A simple proof can be made as follows. Firstly, \(\varnothing \in \mathcal{A}_{\alpha}\) for every \(\alpha\) clearly.
	Moreover, closure under complement and countable unions for every \(\mathcal{A}_{\alpha}\) implies the same must be true for \(\bigcap\limits_{\alpha\in I} \mathcal{A}_{\alpha}\).
	Hence, \(\bigcap\limits_{\alpha\in I} \mathcal{A}_{\alpha}\) is a \hyperref[def:sigma-algebra]{\(\sigma\)-algebra}.
\end{proof}

The above allows us to give the following definition.
\begin{definition}[Generation of \(\sigma\)-algebra]
	Given \(\mathcal{E} \subset \mathcal{P} (X) \), where \(\mathcal{E}\) is not necessarily a \hyperref[def:sigma-algebra]{\(\sigma\)-algebra}. Let
	\(\left<\mathcal{E}\right>\) be the intersection of all \hyperref[def:sigma-algebra]{\(\sigma\)-algebras} on \(X\) containing \(\mathcal{E}\), then
	we call \(\left<\mathcal{E} \right>\) the \emph{\hyperref[def:sigma-algebra]{\(\sigma\)-algebra} generated by \(\mathcal{E}\)}.
\end{definition}

\begin{remark}
	Clearly, \(\left<\mathcal{E} \right>\) is the smallest \hyperref[def:sigma-algebra]{\(\sigma\)-algebra} containing \(\mathcal{E}\), and it is unique.
	To check the uniqueness, we suppose there are two different \(\left<\mathcal{E}\right>_1\) and \(\left<\mathcal{E}\right>_2\)
	generated from \(\mathcal{E} \). It's easy to show
	\[
		\left<\mathcal{E}\right>_1\subseteq \left<\mathcal{E}\right>_2,
	\]
	and by symmetry, they are equal.
\end{remark}
\begin{eg}
	We see that \(\{\varnothing , B, B^{c} , X\}= \left<\{B\}\right> = \left<\{B^{c} \}\right>\).
\end{eg}

\begin{lemma}\label{lma:lec2-1}
	We have
	\begin{enumerate}
		\item Given \(\mathcal{A}\) a \hyperref[def:sigma-algebra]{\(\sigma\)-algebra}, \(\mathcal{E} \subset \mathcal{A} \subset \mathcal{P} (X)\implies \left<\mathcal{E}\right> \subset \mathcal{A} \)
		\item \(\mathcal{E} \subset \mathcal{F} \subset \mathcal{P} (X)\implies \left<\mathcal{E}\right> \subset\left<\mathcal{F} \right>\)
	\end{enumerate}
\end{lemma}
\begin{proof}
	We'll see that after proving the first claim, the second follows smoothly.
	\begin{enumerate}
		\item The first claim is trivial, since we know that \(\left<\mathcal{E} \right>\) is the smallest \hyperref[def:sigma-algebra]{\(\sigma\)-algebra} containing \(\mathcal{E} \),
		      then if \(\mathcal{E} \subset \mathcal{A} \), we clearly have \(\left<\mathcal{E} \right>\subset \mathcal{A} \) by the definition.
		\item The second claim is also easy. From the first claim and the definition, we have
		      \[
			      \mathcal{E} \subset \mathcal{F} \subset \left< \mathcal{F} \right> \implies \left< \mathcal{E} \right>\subset \left< \mathcal{F} \right>.
		      \]
	\end{enumerate}
\end{proof}

At this point, we haven't put any specific structure on \(X\). Now we try to describe those spaces with good structure, which will give the space some nice properties.

\begin{definition}[Borel \(\sigma\)-algebra]
	For a topological space \(X\), the \emph{Borel \hyperref[def:sigma-algebra]{\(\sigma\)-algebra} on \(X\)}, denotes as \(\mathcal{B}(X)\),
	is the \hyperref[def:sigma-algebra]{\(\sigma\)-algebra} generated by the collection of all open sets in \(X\).
\end{definition}

\begin{eg}
	We see that \(\mathcal{B} (\mathbb{\MakeUppercase{R}} )\) contains
	\begin{itemize}
		\item \(\mathcal{E}_1 = \left\{(a, b) \mid a < b; a, b\in \mathbb{\MakeUppercase{R}} \right\}\).
		\item \(\mathcal{E}_2 = \left\{[a, b] \mid a < b; a, b\in \mathbb{\MakeUppercase{R}} \right\}\) since \([a, b] = \bigcap\limits_{n=1}^{\infty} (a - \frac{1}{n}, b + \frac{1}{n})\).
		\item \(\mathcal{E}_3 = \left((a, b]  \mid a < b; a, b\in \mathbb{\MakeUppercase{R}} \right)\) since \((a, b] = \bigcap\limits_{n=1}^{\infty} (a, b + \frac{1}{n})\).
		\item \(\mathcal{E}_4 = \left([a, b)  \mid a < b; a, b\in \mathbb{\MakeUppercase{R}} \right)\) since \([a, b) = \bigcap\limits_{n=1}^{\infty} (a - \frac{1}{n}, b)\).
		\item \(\mathcal{E}_5 = \left((a, \infty)  \mid a\in \mathbb{\MakeUppercase{R}} \right)\) since \((a, \infty) = \bigcup\limits_{n=1}^{\infty} (a , a + n)\).
		\item \(\mathcal{E}_6 = \left([a, \infty)  \mid a\in \mathbb{\MakeUppercase{R}} \right)\) since \([a, \infty) = \bigcup\limits_{n=1}^{\infty} [a , a + n)\).
		\item \(\mathcal{E}_7 = \left((-\infty, b) \mid b\in \mathbb{\MakeUppercase{R}} \right)\) since \((-\infty, b) = \bigcup\limits_{n=1}^{\infty} (b - n, b)\).
		\item \(\mathcal{E}_8 = \left((-\infty, b] \mid a\in \mathbb{\MakeUppercase{R}} \right)\) since \((-\infty, b] = \bigcup\limits_{n=1}^{\infty} (b - n, b]\).
	\end{itemize}
\end{eg}

\begin{proposition}
	\(\mathcal{B} (\mathbb{\MakeUppercase{R}} ) = \left<\mathcal{E}_{i}\right> \) for each \(i = 1, \ldots, 8\).
\end{proposition}
\begin{proof}
	Firstly, we see that \(\mathcal{E} _{i}\subset \mathcal{B} (\mathbb{\MakeUppercase{R}} ) \implies \left<\mathcal{E} _{i}\right>\subset \mathcal{B} (\mathbb{\MakeUppercase{R}} )\) by \autoref{lma:lec2-1}.
	Secondly, by definition, \(\mathcal{B} (\mathbb{\MakeUppercase{R}} ) = \left<\mathcal{E} \right>\) where
	\[
		\mathcal{E} = \left\{O\subseteq \mathbb{\MakeUppercase{R}}\mid O \text{ is open in \(\mathbb{\MakeUppercase{R}}\) } \right\}.
	\]
	It's enough to show \(\mathcal{E} \subset \left<\mathcal{E} _{i}\right>\) since if so, \(\left<\mathcal{E} \right>\subseteq \left<\mathcal{E} _{i}\right>\), and clearly
	\(\left< \mathcal{E}  \right> \supseteq \left< \mathcal{E}_{i} \right> = \mathcal{B} (\mathbb{\MakeUppercase{R}}) \), then we will have
	\(\left< \mathcal{E}\right> = \left< \mathcal{E} _{i} \right> \). Let \(O\subset \mathbb{\MakeUppercase{R}} \) be an open set, i.e., \(O\in\mathcal{E}\).
	We claim that every open set in \(\mathbb{\MakeUppercase{R}} \) is a countable union of disjoint open intervals.\footnote{\url{https://math.stackexchange.com/questions/318299/any-open-subset-of-bbb-r-is-a-countable-union-of-disjoint-open-intervals}}

	Thus,
	\[
		O = \bigcup\limits_{j=1}^{\infty} I_{j},
	\]
	where \(I_{j}\) open interval with the form of \((a, b), (-\infty, b), (a, \infty), (-\infty, \infty)\).

	For example, \(\mathcal{E} _1\) is trivially true, and
	\[
		(a, b) = \underbrace{\bigcup\limits_{n=1}^{\infty} \underbrace{[a+\frac{1}{n}, b-\frac{1}{n}]}_{\in \mathcal{E}_2}}_{\in\left<\mathcal{E} _2\right>}
	\]
	shows the case for \(\mathcal{E} _2\) and
	\[
		(a, \infty) = \bigcup\limits_{k=1}^{\infty} (a, a+k)
	\]
	shows the case for \(\mathcal{E} _5\). It's now straightforward to check open intervals are in \(\left<\mathcal{E} _{i}\right>\) for every \(i\).
\end{proof}

Now, to put a structure on a space, we define the following.
\begin{definition}[Measurable space]
	\((X, \mathcal{A})\) is called a \emph{measurable space}, and \(E\in \mathcal{A} \) is called an \emph{\(\mathcal{A}-\)measurable set}.
\end{definition}

\subsection{Measures}
With the definition of measurable space, we now can refine our measure function \(\mu\) as follows.
\begin{definition}[Measure, Measure space]\label{def:measure}
	Given a measurable space on \((X, \mathcal{A}) \), a \emph{measure} is a function \(\mu\) such that
	\[
		\mu\colon \mathcal{A} \to [0, \infty]
	\]
	with
	\begin{enumerate}
		\item\label{def:measure-empty-measure} \(\mu(\varnothing ) = 0\)
		\item\label{def:measure-countable-additivity} \(\mu\left(\bigcup\limits_{i=1}^{\infty} A_{i}\right) = \sum\limits_{i=1}^{\infty}\mu(A_{i})\) if \(A_1, A_2, \ldots \in \mathcal{A}\)
		are \textbf{disjoint}. We call this \emph{Countable additivity}.
	\end{enumerate}
	We denote \((X, \mathcal{A} , \mu)\) a \emph{measure space}\label{def:measure-space}.
\end{definition}

\begin{notation}
	We denote \([0, \infty] \coloneqq [0, \infty) \cup \{\infty\}\).
\end{notation}

\begin{remark}
	The motivation of why we only want \emph{countable additivity} but not uncountable additivity can be seen by the following example. We'll consider
	the most intuitive measure on \(\mathbb{\MakeUppercase{R}} , \mathcal{B} (\mathbb{\MakeUppercase{R}})\).

	Since we have
	\[
		(0, 1] = \left.\left(\frac{1}{2}, 1\right.\right]\cup \left.\left(\frac{1}{4}, \frac{1}{2}\right.\right]\cup \left.\left(\frac{1}{8}, \frac{1}{4}\right.\right]\cup \ldots
	\]
	and also
	\[
		(0, 1] = \bigcup\limits_{x\in (0, 1]}\{x\}.
	\]

	Specifically, in the first case, we are claiming that
	\[
		1 = \underbrace{\frac{1}{2}}_{\mu((\frac{1}{2}, 1])} + \underbrace{\frac{1}{4}}_{\mu((\frac{1}{4}, \frac{1}{2}])} + \underbrace{\frac{1}{8}}_{\mu((\frac{1}{8}, \frac{1}{4}])}  + \ldots  ;
	\]
	while in the second case, we are claiming that
	\[
		1 = \sum\limits_{x\in(0, 1]} 0
	\]
	since \(\mu(x) = 0\) for \(x\in \mathbb{\MakeUppercase{R}} \), which is clearly not what we want.
\end{remark}

\begin{eg}
	We see some examples.
	\begin{enumerate}
		\item\label{eg:counting-measure} For any \((X, \mathcal{A})\), we let \(\mu(A) \coloneqq \# A\). This is called \emph{counting measure}.
		\item Let \(x_0\in X\). For any \((X, \mathcal{A})\), the \emph{Dirac measure at \(x_0\)} is
		      \[
			      \mu(A) = \begin{dcases}
				      1, & \text{ if } x_0\in A;    \\
				      0, & \text{ if } x_0\notin A.
			      \end{dcases}
		      \]
		\item For \((\mathbb{\MakeUppercase{N}}, \mathcal{P} (\mathbb{\MakeUppercase{N}} ))\),
		      \[
			      \mu(A) = \sum\limits_{i\in A}a_{i},
		      \]
		      where \(a_1, a_2, \ldots \in [0, \infty) \).
	\end{enumerate}
\end{eg}



