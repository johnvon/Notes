\lecture{18}{16 Feb. 11:00}{Monotone Class}
Let's start with a theorem.
\begin{theorem}
	Let \((X, \mathcal{\MakeUppercase{a}} , \mu ), (Y, \mathcal{\MakeUppercase{b}} , \nu )\) be \hyperref[def:measure-space]{measure spaces}. Then
	\begin{enumerate}[(1)]
		\item There is a \hyperref[def:measure]{measure} \(\mu \times \nu \) on \(\mathcal{\MakeUppercase{a}} \otimes \mathcal{\MakeUppercase{b}} \) satisfying
		      \[
			      (\mu \times \nu )(A\times B) = \mu (A)\nu (B)
		      \]
		      for every \(A\in \mathcal{\MakeUppercase{a}} , B\in \mathcal{\MakeUppercase{b}} \).
		\item If \(\mu , \nu \) are \hyperref[def:sigma-finite-measure]{\(\sigma\)-finite }, then \(\mu \times \nu \) is unique.
	\end{enumerate}
\end{theorem}
\begin{proof}
	We prove this one by one.
	\begin{enumerate}[(1)]
		\item Define \(\pi \colon \mathcal{\MakeUppercase{r}} \to [0, \infty ]\) by \(\pi (A \times B) = \mu (A)\nu (B)\), and extending linearly, we have
		      \[
			      \pi (A\times B) = \mu (A)\nu (B),
		      \]
		      hence
		      \[
			      \pi \left(\coprod\limits_{i=1}^{N} A_{i} \times B_{i} \right) = \sum\limits_{i=1}^{n} \pi (A_{i} \times B_{i}).
		      \]

		      We claim that \(\pi \) is a \hyperref[def:pre-measure]{pre-measure}. To show this, it's enough to check that \(\pi (A\times B) = \sum\limits_{n=1}^{\infty} \pi (A_{n} \times B_{n} )\)
		      if \(A\times B = \coprod\limits_{n}A_{n} \times B_{n}  \). Since \(A_{n} \times B_{n} \) are disjoint, so
		      \[
			      \mathbbm{1}_{A\times B}(x, y) = \sum\limits_{n=1}^{\infty} \mathbbm{1}_{A_{n} \times B_{n} }(x, y).
		      \]
		      Thus,
		      \[
			      \mathbbm{1}_{A} (x)\mathbbm{1}_{B} (y) = \sum\limits_{n=1}^{\infty} \mathbbm{1}_{A_{n} }(x)\mathbbm{1}_{B_{n}}(y).
		      \]
		      Integrating with respect to \(x\), and applying \autoref{thm:Tonelli-theorem-for-series}, we have
		      \[
			      \int _X \mathbbm{1}_{A} (x)\mathbbm{1}_{B}(y)\,\mathrm{d} \mu (x) = \sum\limits_{n=1}^{\infty} \int _X \mathbbm{1}_{A_{n} }(x)\mathbbm{1}_{B_{n} }(y)\,\mathrm{d} \mu (x),
		      \]
		      which implies
		      \[
			      \mu (A)\mathbbm{1}_{B} (y) = \sum\limits_{n=1}^{\infty} \mu (A_{n} )\mathbbm{1}_{B_{n} }(y)
		      \]
		      for every \(y\). We can then integrate again with respect to \(y\) and apply \autoref{thm:Tonelli-theorem-for-series}, we have
		      \[
			      \int _Y \mu (A)\mathbbm{1}_{B}(y)\,\mathrm{d} \nu (y) = \sum\limits_{n=1}^{\infty} \int _Y \mu (A_{n} )\mathbbm{1}_{B_{n} }(y)\,\mathrm{d} \nu (y),
		      \]
		      which gives us
		      \[
			      \mu (A)\nu (B) = \sum\limits_{n=1}^{\infty} \mu (A_{n} )\nu (B_{n} ).
		      \]
		      Hence, we see that \(\mu\) is indeed a \hyperref[def:pre-measure]{pre-measure}, so  \autoref{thm:Hahn-Kolmogorov-Thm} gives \(\mu \times \nu \) on \(\left< \mathcal{\MakeUppercase{r}}  \right> = \mathcal{\MakeUppercase{a}} \otimes \mathcal{\MakeUppercase{b}}  \)
		      extending \(\pi \) on \(\mathcal{\MakeUppercase{r}} \).
		\item If \(\mu , \nu \) are \hyperref[def:sigma-finite-measure]{\(\sigma\)-finite}, then \(\pi \) is  \hyperref[def:sigma-finite-measure]{\(\sigma\)-finite} on \(\mathcal{\MakeUppercase{r}} \), then
		      \autoref{thm:uniqueness-of-HK-extension} applies. Moreover, we have that
		      \[
			      (\mu \times \nu )(E) = \inf \left\{\sum\limits_{i=1}^{\infty} \mu (A_{i})\nu (B_{i})\mid E\subset \bigcup\limits_{i=1}^{\infty} A_{i} \times B_{i}, A_{i} \in \mathcal{\MakeUppercase{a}} , B_{i} \in \mathcal{\MakeUppercase{b}} \right\}.
		      \]
	\end{enumerate}
\end{proof}

\section{Monotone Class Lemma}
Let's start with a definition.
\begin{definition}[Monotone Class]\label{def:monotone-class}
	If \(X\) is a set, and \(C\subset \mathcal{\MakeUppercase{p}} (X)\), we say that \(C\) is a \emph{monotone class} on \(X\) if
	\begin{itemize}
		\item \(C\) is closed under countable increasing unions.
		\item \(C\) is closed under countable decreasing intersections.
	\end{itemize}
\end{definition}

\begin{eg}
	Every \hyperref[def:sigma-algebra]{\(\sigma\)-algebra} is a \hyperref[def:monotone-class]{monotone class}.
\end{eg}

\begin{eg}
	If \(C_\alpha \) are (arbitrarily many) \hyperref[def:monotone-class]{monotone classes} on a set \(X\), then \(\bigcap\limits_{\alpha}C_\alpha  \)
	is a \hyperref[def:monotone-class]{monotone class}. Furthermore, if \(\mathcal{\MakeUppercase{e}} \subset \mathcal{\MakeUppercase{p}} (X)\), there is a unique smallest
	\hyperref[def:monotone-class]{monotone class} containing \(\mathcal{\MakeUppercase{e}}\), denoted by \(\left< \mathcal{\MakeUppercase{e}}  \right> \), which follows the same
	idea as in \autoref{def:generation-of-sigma-algebra}.
\end{eg}

\begin{theorem}[Monotone Class Lemma]\label{thm:monotone-class-lemma}
	Suppose \(\mathcal{\MakeUppercase{a}} _0\) is an \hyperref[def:algebra]{algebra} on \(X\). Then \(\left< \mathcal{\MakeUppercase{a}} _0 \right>\)\footnote{\(\left< \mathcal{\MakeUppercase{a}} _0 \right> \) is
		the \hyperref[def:sigma-algebra]{\(\sigma\)-algebra} generated by \(\mathcal{\MakeUppercase{a}} _0\) by \autoref{def:generation-of-sigma-algebra}.}
	is the \hyperref[def:monotone-class]{monotone class} generated by \(\mathcal{\MakeUppercase{a}} _0\).
\end{theorem}
\begin{proof}
	Let \(\mathcal{\MakeUppercase{a}}  = \left< \mathcal{\MakeUppercase{a}} _0 \right> \) and let \(\mathcal{\MakeUppercase{C}} \) be the \hyperref[def:monotone-class]{monotone class}
	generated by \(\mathcal{\MakeUppercase{a}} _0\). Since \(\mathcal{\MakeUppercase{a}} \) is a \hyperref[def:sigma-algebra]{\(\sigma\)-algebra}, it's a
	\hyperref[def:monotone-class]{monotone class}. Note that it contains \(\mathcal{\MakeUppercase{a}} _0\), hence \(\mathcal{\MakeUppercase{a}} \supset \mathcal{\MakeUppercase{C}} \).

	\par To show \(\mathcal{\MakeUppercase{C}} \supset \mathcal{\MakeUppercase{a}} \), it's enough to show that \(\mathcal{\MakeUppercase{C}} \) is a \hyperref[def:sigma-algebra]{\(\sigma\)-algebra}. We
	check that
	\begin{enumerate}
		\item \(\varnothing \in \mathcal{\MakeUppercase{a}} _0 \subseteq \mathcal{\MakeUppercase{C}} \).
		\item Let \(\mathcal{\MakeUppercase{C}} ^\prime = \{E\subset X \mid E^{c} \in \mathcal{\MakeUppercase{C}} \}\).
		      \begin{itemize}
			      \item \(\mathcal{\MakeUppercase{C}} ^\prime \) is a \hyperref[def:monotone-class]{monotone class}.
			      \item \(\mathcal{\MakeUppercase{a}} _0\subset \mathcal{\MakeUppercase{C}} ^\prime \) because if \(E\in \mathcal{\MakeUppercase{a}} _0\), then \(E^{c} \in \mathcal{\MakeUppercase{a}} _0\), so
			            \(E^{c} \in \mathcal{\MakeUppercase{C}} \), thus \(E\in \mathcal{\MakeUppercase{C}} ^\prime \).
		      \end{itemize}
		      We see that \(\mathcal{\MakeUppercase{C}} ^\prime \subset \mathcal{\MakeUppercase{C}}^\prime \), so \(\mathcal{\MakeUppercase{C}} \) is closed under complements.
		\item For \(E\subset X\), let \(\mathcal{\MakeUppercase{d}} (E) = \{F\in \mathcal{\MakeUppercase{C}} \mid E \cup F\in \mathcal{\MakeUppercase{C}} \}\).
		      \begin{itemize}
			      \item \(\mathcal{\MakeUppercase{d}} (E)\subset \mathcal{\MakeUppercase{C}} \).
			      \item \(\mathcal{\MakeUppercase{D}} (E)\) is a \hyperref[def:monotone-class]{monotone class}.
			      \item If \(E\in \mathcal{\MakeUppercase{a}} _0\), then \(\mathcal{\MakeUppercase{a}} _0\subset \mathcal{\MakeUppercase{D}} (E)\). We see this by picking \(F\in \mathcal{\MakeUppercase{a}} _0\), then
			            \(E\cup F\in \mathcal{\MakeUppercase{a}} _0\supset \mathcal{\MakeUppercase{C}} \).
		      \end{itemize}
		      Hence, \(C = \mathcal{\MakeUppercase{D}} (E)\) if \(E\in \mathcal{\MakeUppercase{a}} _0\).
		\item Let \(\mathcal{\MakeUppercase{d}} = \{E\in \mathcal{\MakeUppercase{C}} \mid \mathcal{\MakeUppercase{d}} (E) = \mathcal{\MakeUppercase{C}} \}\). That is \(\mathcal{\MakeUppercase{d}}  = \{E\in \mathcal{\MakeUppercase{C}} \mid E\cup F, \forall F\in \mathcal{\MakeUppercase{C}} \}\).
		      Then we have
		      \begin{itemize}
			      \item \(A_0\subset \mathcal{\MakeUppercase{d}} \) by 3.
			      \item \(\mathcal{\MakeUppercase{d}} \) is a \hyperref[def:monotone-class]{monotone class}.
			      \item \(\mathcal{\MakeUppercase{D}} \subset \mathcal{\MakeUppercase{C}} \) by definition.
		      \end{itemize}
		      Thus, \(\mathcal{\MakeUppercase{d}}  = \mathcal{\MakeUppercase{C}} \), so if \(E, F\in \mathcal{\MakeUppercase{C}} \), then \(E\cup F\in \mathcal{\MakeUppercase{C}} \). This implies that \(\mathcal{\MakeUppercase{C}} \)
		      is closed under finite unions.
		\item Now to show that \(\mathcal{\MakeUppercase{C}}\) is closed under countable unions, let \(E_1, E_2, \ldots \in \mathcal{\MakeUppercase{C}}  \). We may then define
		      \[
			      F_{N} = \bigcup\limits_{n=1}^{N} E_{n} \in \mathcal{\MakeUppercase{C}} .
		      \]
		      Then we see that \(F_1\subset F_2\subset \ldots  \), hence \(\bigcup\limits_{N} F_{N} \in \mathcal{\MakeUppercase{C}}\). But this simply implies
		      \[
			      \bigcup\limits_{N}F_{N} = \bigcup\limits_{n}E_{n} ,
		      \]
		      so we're done.
	\end{enumerate}
\end{proof}