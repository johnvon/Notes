\lecture{33}{1 Apr. 11:00}{Monotone Differentiation Theorem}
From last time we have that if \(\rho \ll m\) is \hyperref[def:regular]{regular} then
\[
	\lim_{r \to 0} \frac{\rho(E_r)}{m(E_r)} = \frac{\,\mathrm{d}\rho}{\,\mathrm{d}m}(x)
\]
for \hyperref[def:mu-almost-everywhere]{Lebesgue almost every \(x\)}, where \(E_r\) \hyperref[def:shrink-nicely]{shrinks nicely} to \(x\).
Likewise, if \(\lambda \perp m\) \hyperref[def:regular]{regular} (\hyperref[def:signed-measure]{positive measure}) then
\[
	\lim_{r \to 0} \frac{\lambda(E_r)}{m(E_r)} = 0
\]
for \hyperref[def:mu-almost-everywhere]{Lebesgue almost every \(x\)}, where \(E_r\) \hyperref[def:shrink-nicely]{shrinks nicely} to \(x\).

\begin{theorem}[Lebesgue differentiation theorem for regular measures]\label{thm:Lebesgue-differentiation-theorem-for-regular-measure}
	Let \(\nu\) be a \hyperref[def:regular]{regular} Borel \hyperref[def:signed-measure]{signed measure} on \(\mathbb{\MakeUppercase{r}} ^d\).
	Then \(\,\mathrm{d}\nu = \,\mathrm{d}\lambda + f \,\mathrm{d}m\), \(\lambda \perp m\) by \autoref{thm:Lebesgue-Radon-Nikodym-theorem}.

	Then for \hyperref[def:mu-almost-everywhere]{Lebesgue almost every \(x\)},
	\[
		\lim_{r \to 0} \frac{\nu(E_r)}{m(E_r)} = f(x)
	\]
	for every \(E_r\) \hyperref[def:shrink-nicely]{shrinks nicely} to \(x\).
\end{theorem}
\begin{proof}
	It must be checked that \(\nu\) \hyperref[def:regular]{regular} implies \(\lambda, f \,\mathrm{d}m\) are \hyperref[def:regular]{regular}.\todo{Check!}
\end{proof}

\subsection{Monotone Differentiation Theorem}
\begin{definition}[]\label{def:}
	For \(F \colon  \mathbb{R} \to \mathbb{R}\) that is monotonically increasing, denote
	\[
		F(x^+) = \lim_{y \to x^+} F(y), \quad F(x^-) = \lim_{y \to x^-} F(y).
	\]

	These exist and are
	\[
		\inf_{y > x} F(y),\quad \sup_{y < x} F(y).
	\]

	So they always exist (being bounded below/above respectively by \(F(x)\)).
\end{definition}

\begin{lemma}\label{lma:lec-33}
	If \(F\) is monotonically increasing, then
	\[
		D = \{x \mid F \text{ is discontinuous at } x\}
	\]
	is a countable set.
\end{lemma}

\begin{proof}
	\(x \in D\) if and only if \(F(x^+) > F(x^-)\). For each \(x \in D\), let \(I_x = (F(x^-),F(x^+))\), not empty.

	Also, if \(x,y \in D\), \(x \neq y\), then \(I_x,I_y\) are disjoint. Say if \(x < y\) then
	\[
		F(x^-) < F(x^+) \leq F(x) \leq F(y) \leq F(y^-) < F(y^+).
	\]
	Taking a rational number in each interval gives an injective map \(D \to \mathbb{\MakeUppercase{q}}\), so \(D\) is countable.
\end{proof}

\begin{theorem}[Monotone Differentiation Theorem]\label{thm:monotone-differentiation}
	Let \(F\) be increasing. Then
	\begin{itemize}
		\item \(F\) is differentiable \hyperref[def:mu-almost-everywhere]{Lebesgue almost everywhere}.
		\item \(G(x) = F(x^+)\) (which is right-continuous) is differentiable \hyperref[def:mu-almost-everywhere]{almost everywhere}.
		\item \(G^\prime  = F^\prime \) \hyperref[def:mu-almost-everywhere]{almost everywhere}
	\end{itemize}
\end{theorem}

\begin{proof}
	Start with \(G\), which is increasing and right-continuous on \(\mathbb{R}\). There is then a \hyperref[def:Lebesgue-Stieltjes-measure]{Lebesgue-Stieltjes measure}
	\(\mu_G\) on \(\mathbb{R}\). Thus, it is \hyperref[def:regular]{regular} on \(\mathbb{R}\). We see
	\[
		\frac{G(x+h) - G(x)}{h} = \begin{dcases}
			\frac{\mu _{G} ((x, x+h])}{m((x, x+h])}, & \text{ if } h>0 ; \\
			\frac{\mu _{G} ((x+h, x])}{m((x+h, x])}, & \text{ if } h<0 .
		\end{dcases}
	\]
	These both \hyperref[def:shrink-nicely]{shrink nicely} to \(x\). By \autoref{thm:Lebesgue-differentiation-theorem-for-regular-measure} (since
	these \hyperref[def:shrink-nicely]{shrink nicely}), we know then that these both converge for \hyperref[def:mu-almost-everywhere]{Lebesgue almost every \(x\)}
	to some common limit \(f(x)\). Hence, \(G^\prime\) exists \hyperref[def:mu-almost-everywhere]{Lebesgue almost everywhere}.

	Define \(H(x) = G(x) - F(x) \geq 0\). We see that
	\[
		\{x \mid H(x) > 0\} \subseteq \{x \mid F \text{ is discontinuous at } x\}
	\]
	This is then countable by \autoref{lma:lec-33}, and we can write \(\{x \mid H(x) > 0\} = \{x_n\}\). Then let
	\[
		\mu \coloneqq \sum_n H(x_n) \delta_{x_n}.
	\]
	This is a \hyperref[def:Borel-measure]{Borel measure}, so we check if it is \hyperref[def:locally-finite]{locally finite}. That is we check
	\[
		\mu((-N,N)) = \sum_{-N < x_n < N} H(x_n) \leq G(N) - F(-N) < \infty
	\]
	checking the inequality just consists of seeing that the intervals \((F(x_n),G(x_n))\) are disjoint and a subset of \((F(-N),G(N))\) so
	\[
		\sum_{-N < x_n < N} H(x_n) = \mu\left( \bigcup_n (F(x_n), G(x_n)) \right) \leq \mu((F(-N),G(N))).
	\]

	Thus, \(\mu\) is a \hyperref[def:Lebesgue-Stieltjes-measure]{Lebesgue-Stieltjes measure} on \(\mathbb{R}\), so it is \hyperref[def:regular]{regular}.
	\begin{remark}
		Special to \(\mathbb{R}\), we have
		\[
			\begin{split}
				\text{that \hyperref[def:locally-finite]{locally finite} \hyperref[def:Borel-measure]{Borel}}
				&\implies \text{\hyperref[def:Lebesgue-Stieltjes-measure]{Lebesgue-Stieltjes}}\\
				&\implies \text{\hyperref[def:regular]{regular}}\\
				&\implies \text{\hyperref[def:regular-outer-regularity]{outer regularity}}.
			\end{split}
		\]
	\end{remark}
	Then we have that
	\[
		\left\vert \frac{H(x + h) - H(x)}{h} \right\vert \leq \frac{H(x + h) + H(x)}{\left\vert h \right\vert} \leq \frac{\mu((x-2h,x+2h))}{\left\vert h \right\vert}.
	\]

	This goes to \(0\) for \hyperref[def:mu-almost-everywhere]{Lebesgue almost every \(x\)} by \autoref{thm:Lebesgue-differentiation-theorem-for-regular-measure}
	and that \(\mu \perp m\).\todo{Check!}

	Thus, \(H\) is differentiable \hyperref[def:mu-almost-everywhere]{almost everywhere} and \(H^\prime = 0\) \hyperref[def:mu-almost-everywhere]{almost everywhere}.
	Thus, \(F\) is differentiable \hyperref[def:mu-almost-everywhere]{almost everywhere} and \(F^\prime = G^\prime\) \hyperref[def:mu-almost-everywhere]{almost everywhere}.
\end{proof}

\begin{proposition}
	Suppose \(F\) is an increasing function. Then \(F^\prime\) exists \hyperref[def:mu-almost-everywhere]{almost everywhere} and is
	\hyperref[def:measurable-function]{measurable}, then
	\[
		\int_a^b F^\prime (x) \,\mathrm{d}x \leq F(b) - F(a).
	\]
\end{proposition}

\begin{eg}
	Take \(F(x)\) to be \(0\) on \(x \leq 0\), \(1\) on \(x > 0\). Then \(F^\prime(x) = 0\) \hyperref[def:mu-almost-everywhere]{almost everywhere}. So
	\[
		\int_{-1}^1 F^\prime(x) \,\mathrm{d}x = 0 < 1 = F(1) - F(-1).
	\]

	Even if \(F\) is continuous we might not have equality. Take \(F(x)\) to be the \hyperref[sssec:Cantor-Function]{Cantor function}.
	Then \(F^\prime(x) = 0\) \hyperref[def:mu-almost-everywhere]{almost everywhere}, but
	\[
		\int_0^1 F^\prime(x) \,\mathrm{d}x = 0 < 1 = F(1) - F(0).
	\]
\end{eg}
