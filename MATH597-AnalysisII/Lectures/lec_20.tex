\lecture{20}{21 Feb. 11:00}{Lebesgue Measure on \(\mathbb{R}^d \)}
\section{Lebesgue Measure on \(\mathbb{R} ^d\)}
\begin{eg}
	\((\mathbb{\MakeUppercase{r}} ^{2} , \mathcal{\MakeUppercase{l}} \otimes \mathcal{\MakeUppercase{l}}, m\times m)\) is not
	\hyperref[def:complete-measure-space]{complete}.
\end{eg}
\begin{explanation}
	We see that
	\begin{itemize}
		\item Let \(A\in \mathcal{\MakeUppercase{l}} \), \(A\neq \varnothing \), \(m(A) = 0\).
		\item Let \(B\subset [0, 1]\), \(B \notin \mathcal{\MakeUppercase{l}} \) (Vital set for example).
		\item Let \(E = A\times B\), \(F = A \times [0, 1]\).
	\end{itemize}
	We see that \(E\subset F\), \(F \in \mathcal{\MakeUppercase{l}} \otimes \mathcal{\MakeUppercase{l}} \), \((m\times m)(F)=m(A)m([0, 1]) = 0\), i.e.,
	\(F\) is a \hyperref[def:mu-null-set]{null} set. But \(E\) is \textbf{not}
	\hyperref[def:A-measurable-function]{\(\mathcal{\MakeUppercase{l}} \otimes \mathcal{\MakeUppercase{l}} \)-measurable-function} since otherwise,
	its sections are all \hyperref[def:A-measurable-function]{measurable}.
\end{explanation}
\begin{definition}
	Let \((\mathbb{\MakeUppercase{r}} ^d, \mathcal{\MakeUppercase{l}} ^d, m^d)\) be the \emph{completion} of
	\[
		(\mathbb{\MakeUppercase{r}} ^d, \mathcal{\MakeUppercase{b}} (\mathbb{\MakeUppercase{r}} ^d), m\times \ldots \times m ),
	\]
	which is \underline{same} as the \emph{completion} of
	\[
		(\mathbb{\MakeUppercase{r}} ^d, \mathcal{\MakeUppercase{l}} \otimes \ldots \otimes \mathcal{\MakeUppercase{l}}, m\times \ldots \times m).
	\]
\end{definition}

\begin{remark}
	We see that
	\[
		\mathcal{\MakeUppercase{l}} ^d \supsetneq \mathcal{\MakeUppercase{l}} \otimes \ldots \otimes \mathcal{\MakeUppercase{l}}
		= \left< \left\{ \prod\limits_{i=1}^{d}E_{i} \mid E_{i} \in \mathcal{\MakeUppercase{l}}  \right\} \right>.
	\]
\end{remark}

\begin{definition}[Rectangle in \(\mathbb{\MakeUppercase{r}} ^d\)]\label{def:rectangle-in-Rd}
	A \emph{rectangle in \(\mathbb{\MakeUppercase{r}} ^d\)} is \(R = \prod\limits_{i=1}^{d} E_{i} \) where \(E_{i}\in \mathcal{\MakeUppercase{b}} (\mathbb{\MakeUppercase{r}} )\).
\end{definition}

\begin{definition}[Lebesgue measure in \(\mathbb{\MakeUppercase{r}} ^d\)]
	We let the \emph{\hyperref[def:Lebesgue-measure]{Lebesgue measure} in \(\mathbb{\MakeUppercase{r}} ^d\)}, denoted as \(m^d\), defined as
	\[
		m^d(E) \coloneqq \inf \left\{\sum\limits_{k=1}^{\infty} m^d(R_{k} )\mid E\subset \bigcup\limits_{k=1}^{\infty} R_{k} , R_{k} \text{ is \hyperref[def:rectangle-in-Rd]{rectangles}}\right\}.
	\]
\end{definition}
\begin{theorem}\label{thm:lec-20}
	Let \(E\subset \mathcal{\MakeUppercase{l}} ^d\). Then
	\begin{enumerate}[(1)]
		\item \(m^d(E) = \inf \left\{m^d (0) \mid \text{open } O\supset E \right\} = \sup \left\{m^d (K)\mid \text{compact }K\subset E \right\}\).
		\item \(E = A_1 \cup N_1 = A_2 \setminus N_2\), where \(A_1\) is \hyperref[def:F-sigma-set]{\(F_\sigma \)}, \(A_2\) is \hyperref[def:G-delta-set]{\(G_\delta \) }, and \(N_{i} \) are \hyperref[def:mu-null-set]{null}.
		\item If \(m^d(E)<\infty \), \(\forall \epsilon >0\), \(\exists R_1, \ldots , R_m \) \hyperref[def:rectangle-in-Rd]{rectangles} whose sides are \underline{intervals} such that
		      \[
			      m^d \left(E\triangle \left(\bigcup\limits_{i=1}^{m} R_{i} \right)\right)< \epsilon .
		      \]
	\end{enumerate}
\end{theorem}
\begin{proof}
	Similar to \(d = 1\) case.
\end{proof}

\begin{theorem}
	\hyperref[def:integrable]{Integrable} \hyperref[def:step-function]{step functions} and \(C_c(\mathbb{\MakeUppercase{r}} ^d)\), the collection
	of continuous functions, are dense in \(L^1(\mathbb{\MakeUppercase{r}} ^d, \mathcal{\MakeUppercase{l}} ^d, m^d)\)
\end{theorem}
\begin{proof}
	See~\cite{folland1999real}.
\end{proof}

\begin{theorem}
	\hyperref[def:Lebesgue-measure]{Lebesgue measure} in \(\mathbb{\MakeUppercase{r}} ^d\) is translation-invariant.
\end{theorem}
\begin{proof}
	See~\cite{folland1999real}.
\end{proof}

\begin{theorem}[Effect of linear transformation on Lebesgue measure]\label{thm:effect-of-linear-transformation-on-Lebesgue-measure}
	If \(T\in \mathrm{GL} (\mathbb{\MakeUppercase{r}} ^d)\), \(e\in \mathcal{\MakeUppercase{l}} ^d\), then \(T(E)\) is \hyperref[def:measurable-function]{measurable} and
	\[
		m(T(E)) = \left\vert \det T \right\vert \cdot m(E).
	\]
\end{theorem}
\begin{proof}
	See~\cite{folland1999real}.
\end{proof}

\chapter{Differentiation on Euclidean Space}\label{ch:Differentiation-on-Euclidean-Space}
\begin{prev}
	Given \(f\colon [a, b]\to \mathbb{\MakeUppercase{r}} \), there are two versions of \textbf{fundamental theorem of calculus}:
	\begin{enumerate}[(1)]
		\item
		      \[
			      \int_{a}^{b} f^\prime (x) \,\mathrm{d}x = f(b) - f(a).
		      \]
		\item
		      \[
			      \frac{\mathrm{d}}{\mathrm{d}x} \int_{a}^{x} f(t) \,\mathrm{d}t = f(x),
		      \]
		      which follows from
		      \[
			      \lim\limits_{r \to 0^+} \frac{1}{r}\int_{x}^{x+r} f(t) \,\mathrm{d}t = f(x) = \lim\limits_{r \to 0^+} \frac{1}{r}\int_{x-r}^{x} f(t) \,\mathrm{d}t.
		      \]
	\end{enumerate}
\end{prev}
\begin{remark}
	We see that
	\[
		\lim\limits_{r \to 0^+} \frac{1}{r}\int_{x}^{x+r} \left(f(t)-f(x)\right) \,\mathrm{d}t = 0 = \lim\limits_{r \to 0^+} \frac{1}{r}\int_{x-r}^{x} \left(f(t)-f(x)\right) \,\mathrm{d}t,
	\]
	where we have
	\[
		f(x) = \frac{1}{r}\int_{x}^{x+r} f(t) \,\mathrm{d}t.
	\]
	This generalized to \(f\colon \mathbb{\MakeUppercase{r}} ^d\to \mathbb{\MakeUppercase{r}} \), namely
	\[
		\lim\limits_{r \to 0^+} \frac{1}{\mathrm{vol} \left(B(x, r)\right)}\int_{B(x, r)}\left(f(t)-f(x)\right)\,\underbrace{\mathrm{d}t}_{\mathbb{\MakeUppercase{r}} ^d} \overset{?}{=} 0.
	\]
\end{remark}
\section{Hardy-Littlewood Maximal Function}
We first see our notation.
\begin{notation}
	Given a(n) (open) ball in \(\mathbb{\MakeUppercase{r}} ^d\), \(B = B(a, r)\), denote \(cB = B(a, cr)\) for \(c>0\).
\end{notation}

\begin{lemma}[Vitali-type covering lemma]\label{lma:Vitali-type-covering-lemma}
	Let \(B_1, \ldots , B_k \) be a finite collection of open balls in \(\mathbb{\MakeUppercase{r}} ^d\). Then there
	exists a sub-collection \(B^\prime _1, \ldots , B^\prime _m \) of \underline{disjoint} open balls such that
	\[
		\bigcup\limits_{i=1}^{m} \left(3 B_{j} ^\prime \right)\supset \bigcup\limits_{i=1}^{k} B_{i}.
	\]
\end{lemma}
\begin{proof}
	Greedy Algorithm.
\end{proof}