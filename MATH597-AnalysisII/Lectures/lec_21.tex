\lecture{21}{25 Feb. 11:00}{}
\begin{notation}
	We let
	\[
		\int _E f \,\mathrm{d} m = \int _E f(x) \,\mathrm{d} x.
	\]
\end{notation}

The problem we're working on is
\[
	\frac{1}{m(B(w, r))}\int _{B(x, r)}f(y)\,\mathrm{d} y \overset{r\to 0}{\underset{?}{\longrightarrow}}f(x).
\]
\begin{definition}[Locally integrable]\label{def:locally-integrable}
	Given \(f\colon \mathbb{\MakeUppercase{r}} ^d \to \mathbb{\MakeUppercase{c}} \) be \hyperref[def:Lebesgue-measurable-function]{Lebesgue measurable}
	function. Then we say \(f\) is \emph{locally integrable} if
	\[
		\int _K \left\vert f \right\vert \,\mathrm{d} m < \infty ,\forall \text{compact } K\subset \mathbb{\MakeUppercase{r}} ^d.
	\]
	We write \(f\in L^1_{\text{loc} }(\mathbb{\MakeUppercase{r}} ^d)\).
\end{definition}

\begin{definition}[Hardy-Littlewood maximal function]\label{def:HL-maximal-function}
	Given \(f\in L^1_{\text{loc} }(\mathbb{\MakeUppercase{r}} ^d)\), the \emph{Hardy-Littlewood maximal function} for \(f\) is defined as
	\[
		\mathrm{Hf}(x) \coloneqq \sup \left\{\mathrm{A}_r(x)\mid r > 0\right\},
	\]
	where
	\[
		\mathrm{A}_r(x) \coloneqq \frac{1}{m(B(x, r))}\int _{B(x, r)} \left\vert f(y) \right\vert \,\mathrm{d}y.
	\]
\end{definition}
\begin{note}
	We note that \(\mathrm{A}_r(\cdot) \) means \emph{averaging function}.
\end{note}
\begin{lemma}\label{lma:lec21}
	Let \(f\in L^1_{\text{loc} }(\mathbb{\MakeUppercase{r}} ^d)\), then
	\begin{enumerate}
		\item \(\mathrm{A}_r(x)\) is jointly continuous for \((x, r)\in \mathbb{\MakeUppercase{r}} ^d \times (0, \infty )\).
		\item \(\mathrm{Hf}(x) \) is \hyperref[def:Borel-measurable]{Borel measurable}.
	\end{enumerate}
\end{lemma}
\begin{proof}
	We outline the proof.
	\begin{enumerate}
		\item Let \((x, r)\to (x^\ast, r^\ast)\implies \mathrm{A}_r(x)\to \mathrm{A}_{r^\ast}(x^\ast)\). Let \((x_{n} , r_{n} )\) be any sequence which converges to
		      \(x^\ast, r^\ast\), then we consider \(\lim\limits_{n \to \infty} \mathrm{A}_{r_{n} }(x_{n} )\) and we can calculate
		      \[
			      \int\underbrace{\left\vert f(y) \right\vert \mathbbm{1}_{B(x_{n} , r_{n} )} (y)}_{\coloneqq h_{n} (y)},
		      \]
		      then we apply \autoref{thm:dominated-convergence-theorem} to \(h_{n} \).
		\item Observe that
		      \[
			      (\mathrm{Hf})^{-1} (\underbrace{(a, \infty )}_{\text{open} }) = \bigcup\limits_{r>0} \mathrm{A}_{r} ^{-1} \left((a, \infty )\right)
		      \]
		      is open, since \(\mathrm{A} _{r} ^{-1} \left((a, \infty )\right)\) is open from the 1. Note that the equality comes from the fact that \(\mathrm{Hf} = \sup_{r} \mathrm{A} _{r} \).
	\end{enumerate}
\end{proof}

\begin{theorem}[Hardy-Littlewood maximal inequality]\label{thm:HL-maximal-inequality}
	There exists \(C_{d} >0\) such that for every \(f\in L^1(\mathbb{\MakeUppercase{r}} ^d)\),
	\[
		\underset{\alpha >0}{\forall }\ m\left(\left\{x\in \mathbb{\MakeUppercase{r}} ^d\mid \mathrm{Hf}(x) > \alpha  \right\}\right) \leq \frac{C_{d} }{\alpha }\int \left\vert f(x) \right\vert \,\mathrm{d} x.
	\]
\end{theorem}
\begin{proof}
	We first fix \(f\in L^1\) and \(\alpha >0\). We define
	\[
		E\coloneqq \left\{x\mid \mathrm{Hf} (x) > \alpha \right\},
	\]
	which is a \hyperref[def:A-measurable-set]{Borel measurable set} by \autoref{lma:lec21}.then
	\[
		x\in E\implies \underset{r_{x} >0}{\exists } \ \mathrm{A} _{r_{x} }(x)> \alpha \implies m(B(x, r_{x} ))< \frac{1}{\alpha }\int _{B(x, r_{x} )}\left\vert f(y) \right\vert \,\mathrm{d} y.
	\]
	From \hyperref[thm:inner-regularity]{inner regularity}, we have
	\[
		m(E) = \sup \left\{m(K)\mid \text{compact }K\subset E \right\}.
	\]
	Let \(K\subset E\) be compact, then
	\[
		K\subset \bigcup\limits_{x\in K}B(x, r_{x} ) \overset{K \text{ compact} }{\implies } K\subset \bigcup\limits_{i=1}^{N} B_{i} \overset{\hyperref[lma:Vitali-type-covering-lemma]{!}}{\implies} K\subset \bigcup\limits_{i=1}^{m} \left\{3 B_{j} ^\prime \right\}.
	\]
	From here, we further have
	\[
		m(K) \leq \sum\limits_{i=1}^{m} m(3 B_{j} ^\prime ) = 3^d \sum\limits_{j=1}^{m} m(B_{j} ^\prime ) \leq \frac{3^d}{\alpha }\sum\limits_{j=1}^{m} \int _{B_{j} ^\prime }\left\vert f(y) \right\vert \,\mathrm{d}y.
	\]
	Now, since \(B^\prime _{i} , \ldots , B^\prime _{m}  \) are disjoint, hence we finally have
	\[
		m(K) \leq \frac{3^d}{\alpha }\int _{\mathbb{\MakeUppercase{r}} ^d}\left\vert f(y) \right\vert \,\mathrm{d} y.
	\]
\end{proof}

\subsection{Lebesgue Differentiation Theorem}
We start with a theorem!
\begin{theorem}[Lebesgue Differentiation Theore]\label{thm:lec21}
	Let \(f\in L^1\), then
	\[
		\lim\limits_{r \to 0} \frac{1}{m(B(x, r))}\int_{B(x, r)}\left\vert f(y) - f(x) \right\vert   \,\mathrm{d}y = 0
	\]
	for \hyperref[def:mu-almost-everywhere]{a.e.} \(x\).
\end{theorem}

\begin{corollary}
	\autoref{thm:lec21} also holds for \(f\in L^1_{\text{loc}}(\mathbb{\MakeUppercase{r}} ^d)\).
\end{corollary}
\begin{proof}
	Using the fact that \(m^d\) is \hyperref[def:finite-measure]{\(\sigma \)-finite}, and apply \autoref{thm:lec21}.
\end{proof}

\begin{corollary}
	For \(f\in L^1_{\text{loc} }\), we have
	\[
		\lim\limits_{r \to 0} \frac{1}{m(B(x, r))}\int_{B(x, r)}^{} f(y) \,\mathrm{d}y = f(x)
	\]
	for \hyperref[def:mu-almost-everywhere]{a.e.} \(x\).
\end{corollary}
\begin{proof}
	\todo{DIY}
\end{proof}

\begin{definition}[Shrink nicely]\label{def:shrink-nicely}
	We say that \(\left\{E_{r} \right\}_{r>0}\) \emph{shrinks nicely} to \(x\) as \(r\to 0\) if \(E_{r} \subset B(x, r)\) and
	\[
		\underset{c>0}{\exists } \ c\cdot m(B(x, r)) \leq m(E_{r} ).
	\]
\end{definition}