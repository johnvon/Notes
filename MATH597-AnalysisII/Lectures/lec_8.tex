\lecture{8}{24 Jan. 11:00}{Lebesgue-Stieltjes Measure on \(\mathbb{R} \)}
To classify all measures, we now see this last theorem to complete the task.
\begin{theorem}[Locally finite Borel measures on \(\mathbb{\MakeUppercase{r}} \)]\label{thm:locally-finite-borel-measure-on-R}
	We have
	\begin{enumerate}
		\item \(F\colon \mathbb{\MakeUppercase{r}} \to \mathbb{\MakeUppercase{r}} \) a \hyperref[def:distribution-function]{distribution function}, then there exists
		      a \textbf{unique} \emph{\hyperref[def:locally-finite]{locally finite} \hyperref[def:Borel-measure]{Borel measure}} \(\mu _F\) on \(\mathbb{\MakeUppercase{r}} \)
		      satisfying
		      \[
			      \mu _F((a, b]) = F(b) - F(a)
		      \]
		      for every \(a < b\).
		\item Suppose \(F, G\colon \mathbb{\MakeUppercase{r}} \to \mathbb{\MakeUppercase{r}} \) are \hyperref[def:distribution-function]{distribution functions}. Then,
		      \[
			      \mu _F = \mu _G
		      \]
		      on \(\mathcal{B} (\mathbb{\MakeUppercase{r}} )\) if and only if \(F-G\) is a constant function.
	\end{enumerate}
\end{theorem}
\begin{proof}
	\todo{HW.}
\end{proof}
\begin{remark}
	\autoref{thm:locally-finite-borel-measure-on-R} simply states that given a \hyperref[def:distribution-function]{distribution function},
	if we restrict our attention on \hyperref[def:locally-finite]{locally finite} measures on \(\mathbb{\MakeUppercase{r}} \) following our
	usual convention, then it defines the \hyperref[def:measure]{measure} on \(\mathcal{B}(\mathbb{\MakeUppercase{r}} )\) uniquely up to a \emph{constant shift}.
\end{remark}

\subsection{Lebesgue-Stieltjes Measure on \(\mathbb{R}\)}
We see that
\[
	F \text{ \hyperref[def:distribution-function]{distribution function}}
	\overset{\hyperref[thm:Hahn-Kolmogorov-Thm]{!}}{\implies }
	\mu _F \text{ on \hyperref[thm:Caratheodory-extension-Thm]{Carathéodory \(\sigma\)-algebra}}\ \mathcal{A}_{\mu _F}\supset \mathcal{B} (\mathbb{\MakeUppercase{r}}) .
\]

Furthermore, we have
\[
	(\mathcal{A}_{\mu _F}, \mu _F) = \overline{(\mathcal{B} (\mathbb{\MakeUppercase{r}} ), \mu _F)}.
\]
\begin{definition}[Lebesgue-Stieltjes measure]\label{def:Lebesgue-Stieltjes-measure}
	Given a \hyperref[def:distribution-function]{distribution function} \(F\), we say \(\mu _F\) on \(\mathcal{A} _{\mu _F}\) is called the
	\emph{Lebesgue-Stieltjes measure} corresponding to \(F\).
\end{definition}

\begin{definition}[Lebesgue measure]\label{def:Lebesgue-measure}
	From \autoref{def:Lebesgue-Stieltjes-measure}, if \(F(x)=x\), then the induced \((\mathcal{\MakeUppercase{a}} _{\mu _F}, \mu _{F})\) is denoted as \((\mathcal{L} , m)\), where
	\(\mathcal{L} \) is called \emph{Lebesgue \(\sigma\)-algebra}, and \(m\) is called \emph{Lebesgue measure}.
\end{definition}

\begin{remark}
	Recall that \(\mathcal{\MakeUppercase{l}} \) is induced by \autoref{thm:Caratheodory-extension-Thm}, namely given \(m\), for all \(A\subset \mathbb{\MakeUppercase{r}} \),
	we have
	\[
		\mathcal{\MakeUppercase{l}} \coloneqq \left\{A\subset \mathbb{\MakeUppercase{r}} \mid \underset{E\subset \mathbb{\MakeUppercase{r}} }{\forall }\ m(A) = m(A\cap E) + m(A\setminus E)\right\}
	\]
\end{remark}

\begin{note}
	We see that since \(F\) is right-continuous and increasing, hence
	\[
		F(x^-)\leq F(x) = F(x^+).\footnotemark
	\]\footnotetext{Some text will use \(x-\) and \(x+\) instead of \(x^-\) and \(x^+\), respectively.}
\end{note}

\begin{eg}
	We first see some examples.
	\begin{enumerate}
		\item \(\mu _F((a, b]) = F(b) - F(a)\). Then
		      \begin{itemize}
			      \item \(\mu _F(\{a\}) = F(a) - F(a^-)\)
			      \item \(\mu _F([a, b]) = F(b) - F(a^-)\)
			      \item \(\mu _F((a, b)) = F(b^-) - F(a)\)
		      \end{itemize}
		\item We define
		      \[
			      F(x) = \begin{dcases}
				      1, & \text{ if } x\geq 0 ; \\
				      0, & \text{ if } x<0.
			      \end{dcases}
		      \]
		      Then
		      \begin{itemize}
			      \item \(\mu _F(\{0\})=1\)
			      \item \(\mu _F(\mathbb{\MakeUppercase{r}} ) = 1\)
			      \item \(\mu _F(\mathbb{\MakeUppercase{r}}\setminus \{0\} ) = 0\). This is easy to see since \(\mathbb{\MakeUppercase{r}} \setminus \{0\} = (-\infty , 0)\cup (0, \infty )\), hence
			            \[
				            \begin{split}
					            \mu _F(\mathbb{\MakeUppercase{r}} \setminus \{0\}) &= \mu _F((-\infty , 0)\cup (0, \infty ))\\
					            &= \underbrace{\mu _F((-\infty , 0))}_{0 - 0\footnotemark } + \underbrace{\mu _F((0, \infty ))}_{1 -1 \footnotemark }= 0.
				            \end{split}
			            \]
			            \addtocounter{footnote}{-2}
			            \stepcounter{footnote}\footnotetext{It follows from \(F(0^-) - F(-\infty ) = 0 - 0 = 0\).}
			            \stepcounter{footnote}\footnotetext{It follows from \(F(\infty ) - F(0) = 1 - 1 = 0\).}
		      \end{itemize}
		      We call that \(\mu _F\) is the \emph{Dirac measure} at \(0\).
		      \item\label{eg:lec8-3} Denote \(\mathbb{\MakeUppercase{q}} = \{r_1, r_2, \ldots  \}\), and we define
		      \[
			      F(x) = \sum\limits_{n=1}^{\infty} \frac{F_n(x)}{2^n} \text{ where }  F_{n}(x) = \begin{dcases}
				      1, & \text{ if } x\geq r_n; \\
				      0, & \text{ if } x<r.       \\
			      \end{dcases}
		      \]
		      Then
		      \begin{itemize}
			      \item \(\mu _F(\{r_{i}\})>0\) for all \(r_{i}\in\mathbb{\MakeUppercase{q}} \).
			      \item \(\mu _F(\mathbb{\MakeUppercase{r}} \setminus \mathbb{\MakeUppercase{q}} ) = 0\).
		      \end{itemize}
		      This is shown in \autoref{lma:lec8-eg:3}.
		\item If \(F\) is continuous at \(a\), then \(\mu _F(\{a\}) = 0\).
		\item \(F(x) = x\), then recall that we denote \(\mu _F \coloneqq m\), and we have
		      \begin{itemize}
			      \item \( m((a, b]) = m((a, b)) = m([a, b]) = b - a\).
		      \end{itemize}
		\item \(F(x) = e^x\)
		      \begin{itemize}
			      \item \(\mu _F((a, b]) = \mu _F((a, b)) = e^b - e^a\).
		      \end{itemize}
	\end{enumerate}
	\begin{remark}
		We see that the first two examples are \emph{discrete measures}.
	\end{remark}
\end{eg}

\begin{eg}[Middle thirds Cantor set]\label{eg:lec8:Cantor-set}
	Let \(C \coloneqq \bigcap\limits_{n=1}^{\infty} K_{n}\), where we have
	\[
		\begin{split}
			K_0 &\coloneqq [0, 1]\\
			K_1 &\coloneqq K_{0}\setminus \left(\frac{1}{3}, \frac{2}{3}\right)\\
			K_2 &\coloneqq K_{1}\setminus \left(\frac{1}{9}, \frac{2}{9}\right) \cup \left(\frac{7}{9}, \frac{8}{9}\right)\\
			&\vdots\\
			K_n &\coloneqq K_{n-1}\setminus \bigcup\limits_{k=1}^{3^n - 1}\left(\frac{3k+1}{3^{n+1}}, \frac{3^{k+2}}{3^{n+1}}\right).
		\end{split}
	\]
	We see that \(C\) is uncountable and with \(m(C) = 0\). And observe that \(x\in C\) if and only if \(x = \sum\limits_{n=1}^{\infty} \frac{a_n}{3}\) for some \(a_n\in\{0, 2\}\).
	Hence, we can instead formulate \(K_{n}\) by
	\[
		K_{n} = \bigcup\limits_{\substack{a_{i}\in\{0, 2\}\\ 1\leq i\leq n}}\left[\sum\limits_{i=1}^{\infty} \frac{a_{i}}{3^i}, \sum\limits_{i=1}^{\infty} \frac{a_{i}}{3^i} + \frac{1}{3^n}\right].
	\]

	\begin{figure}[H]
		\centering
		\begin{tikzpicture}
			\foreach \order in {0,...,4}
			\draw[yshift=-\order*10pt]  l-system[l-system={cantor set, axiom=F, order=\order, step=100pt/(3^\order)}];
		\end{tikzpicture}
		\caption{The top line corresponds to \(K_0\), and then \(K_1\), etc.}
	\end{figure}
	The proof of \(m(C) = 0\) is given in \autoref{lma:Cantor-set-has-measure-0}.
\end{eg}

\subsubsection{Cantor Function}\label{sssec:Cantor-Function}
Consider \(F\) as follows. We define a function \(F\) to be \(0\) to the left of \(0\),  and \(1\) to the right of \(1\). Then, define \(F\) to be \(\frac{1}{2}\)
on \(\left(\frac{1}{3}, \frac{2}{3}\right)\), \(\frac{1}{4}\) on \(\left(\frac{1}{9}, \frac{2}{9}\right)\), \(\frac{3}{4}\) on \(\left(\frac{7}{9}, \frac{8}{9}\right)\)
and so on. This is so-called \emph{Cantor Function}. We can show \(F\) is continuous and increasing, which makes \(F\) a distribution function.
Also, we see that the measure this \(F\) induced is called \emph{Cantor measure}\label{def:Cantor-measure}.

\begin{figure}[H]
	\centering
	\begin{tikzpicture}
		\tikzset{
			if/.code n args=3{\pgfmathparse{#1}\ifnum\pgfmathresult=0\pgfkeysalso{#3}\else\pgfkeysalso{#2}\fi},
			lower cantor/.initial=.3333, upper cantor/.initial=.6667, y cantor/.initial=.5,
			declare function={
					cantor_l(\lowerBound,\upperBound)=(\pgfkeysvalueof{/tikz/lower\space cantor})*(\upperBound-\lowerBound)+\lowerBound;
					cantor_u(\lowerBound,\upperBound)=(\pgfkeysvalueof{/tikz/upper\space cantor})*(\upperBound-\lowerBound)+\lowerBound;
					cantor(\lowerBound,\upperBound)=(\pgfkeysvalueof{/tikz/y\space cantor})*(\upperBound-\lowerBound)+\lowerBound;},
			cantor start/.style n args=5{%
					insert path={(#1,#3)},
					cantor={#1}{#2}{#3}{#4}{#5}{0},
					insert path={to[every cantor edge/.try, cantor 1 edge/.try] (#2,#4)}},
			cantor/.style n args=6{%
					/utils/exec=%
					\pgfmathsetmacro\lBx{cantor_l(#1,#2)}%
					\pgfmathsetmacro\uBx{cantor_u(#1,#2)}%
					\pgfmathsetmacro\y{cantor(#3,#4)},% fun
					style/.expanded={
							if={#6<#5}{cantor={#1}{\lBx}{#3}{\y}{#5}{#6+1}}{},
							insert path={to[every cantor edge/.try, cantor 1 edge/.try] (\lBx,\y)to[every cantor edge/.try, cantor 2 edge/.try] (\uBx,\y)},
							if={#6<#5}{cantor={\uBx}{#2}{\y}{#4}{#5}{#6+1}}{}}}
		}
		\foreach \level in {5}{
				\begin{tikzpicture}[line join=round, scale=0.8] % cantor 1 edge/.style={move to}
					(0,0) grid[xstep=1/9, ystep=.25] (1,1);
					\draw[thick, cantor start={0}{6}{0}{6}{\level}{0}];
				\end{tikzpicture}}
	\end{tikzpicture}
	\caption{Cantor Function (Devil's Staircase).}
\end{figure}

We see that \(F\) is \emph{continuous} and increasing. Furthermore,
\begin{table}[H]
	\centering
	\begin{tabular}{lcl}
		\toprule
		\hyperref[def:Cantor-measure]{Cantor Measure} \(\mu _F\) &           & \hyperref[def:Lebesgue-measure]{Lebesgue Measure} \(m\) \\
		\midrule
		\(\mu _F(\mathbb{\MakeUppercase{r}} \setminus C) = 0\)   &           & \(m(\mathbb{\MakeUppercase{r}} \setminus C)=\infty >0\) \\
		\(\mu _F(C) = 1 \)                                       & \(\iff \) & \(m(C) = 0\)                                            \\
		\(\mu _F(\{a\}) = 0\)                                    &           & \(m(\{a\})=0\)                                          \\
		\bottomrule
	\end{tabular}
\end{table}
\begin{remark}
	\(\mu _F\) and \(m\) are said to be \textbf{singular} to each other.
\end{remark}

\subsection{Regularity Properties of Lebesgue-Stieltjes Measures}
We first see a lemma.
\begin{lemma}
	Let \(\mu\) be \hyperref[def:Lebesgue-Stieltjes-measure]{Lebesgue-Stieltjes measure} on \(\mathbb{\MakeUppercase{r}} \). Then we have
	\[
		\begin{split}
			\mu (A) &\overset{\hyperref[prop:outer-measure]{!}}{=}\inf\left\{\sum\limits_{i=1}^{\infty} \mu ((a_{i}, b_{i}]) \mid \bigcup\limits_{i=1}^{\infty} (a_{i}, b_{i}]\supset A\right\}\\
			&= \inf\left\{\sum\limits_{i=1}^{\infty} \mu ((a_{i}, b_{i})) \mid \bigcup\limits_{i=1}^{\infty} (a_{i}, b_{i})\supset A\right\}
		\end{split}
	\]
	for every \(A\in \mathcal{A} _\mu \)
\end{lemma}
\begin{proof}
	The second equality follows from the continuity of the \hyperref[def:measure]{measure}.
\end{proof}

\begin{remark}
	This is similar to
	\[
		(a, b) = \bigcup\limits_{n=1}^{\infty} (a, b - 1/n],\quad (a, b] = \bigcap\limits_{n=1}^{\infty} (a, b+1/n].
	\]
\end{remark}