\lecture{27}{18 Mar.\ 11:00}{Dual Space}
\begin{prev}
	From \autoref{def:operator-norm}, we have that
	\[
		\left\lVert Tv\right\rVert ^{\prime\prime} \leq \left\lVert T\right\rVert \left\lVert v\right\rVert ^\prime .
	\]
\end{prev}

\begin{remark}
	Notice that this \autoref{def:operator-norm} is only for \hyperref[def:bounded-linear-transformation]{bounded linear transformation}.
\end{remark}

\begin{theorem}\label{thm:blts-complete}
	If \(W\) is \hyperref[def:complete]{complete}, then \(L(V,W)\) is \hyperref[def:complete]{complete}.
\end{theorem}

\begin{proof}
	Suppose \(T_n\) is a \hyperref[def:Cauchy-sequence]{Cauchy sequence} in \(L(V,W)\). Fix \(v \in V\) and let \(w_n \coloneqq T_n v \in W\), we then have
	\[
		\left\lVert w_n - w_m\right\rVert = \left\lVert T_n v - T_m v\right\rVert = \left\lVert (T_n - T_m)v\right\rVert
		\leq \underbrace{\left\lVert T_n - T_m\right\rVert}_{\to 0} \underbrace{\left\lVert v\right\rVert}_{\text{fixed}\atop \text{value}}.
	\]
	Thus, \(w_n\) is \hyperref[def:Cauchy-sequence]{Cauchy}, so it converges since \(W\) is \hyperref[def:complete]{complete}. We call its unique
	limit \(Tv\). This makes \(T \colon V \to W\) into a function. We must show it is a \hyperref[def:bounded-linear-transformation]{bounded linear transformation} and that
	\(\left\lVert T_n - T\right\rVert \to 0\).\todo{DIY}
\end{proof}

\section{Dual of \(L^p\) Spaces}
\begin{eg}
	Let \(w \in \mathbb{R} ^d\), and denote the \hyperref[def:inner-product]{inner product} between \(w\) and \(v\in \mathbb{R} ^d\) by
	\[
		v\cdot w \coloneqq \left< v, w \right>.
	\]
	Then we can consider
	\[
		\max\{v \cdot w \mid \left\lVert v\right\rVert_2 = 1\} = \left\lVert w\right\rVert_2.
	\]
	If \(w \in \mathbb{C}^d\), this is similar since
	\[
		\max\{\left\vert v \cdot w \right\vert  \mid \left\lVert v\right\rVert_2 = 1\} = \left\lVert w\right\rVert_2.
	\]
	These maximums are achieved by \(v = \frac{\overline{w} }{\left\lVert w\right\rVert}\) if \(w \neq 0\).
\end{eg}

\begin{proposition}\label{prop:Lp-dual-formula}
	Let \(1/p + 1/q = 1\) with \(1 \leq q < \infty\). For every \(g \in L^q\),
	\[
		\left\lVert g\right\rVert_q = \sup\left\{\left\vert \int fg \right\vert  \mid \left\lVert f\right\rVert_p = 1\right\}.
	\]

	Suppose \(\mu \) is \hyperref[def:sigma-finite-measure]{\(\sigma\)-finite}, then the result also holds for \(q = \infty\), \(p = 1\).
\end{proposition}
\begin{proof}
	By \hyperref[thm:Holder-inequality]{Hölder's inequality}, we know that
	\[
		\left\vert \int fg \right\vert  \leq \int \left\vert fg \right\vert  = \left\lVert fg\right\rVert_1 \leq \left\lVert f\right\rVert _p \left\lVert g\right\rVert_q = \left\lVert g\right\rVert_q.
	\]

	Thus, the supremum is less or equal to \(\left\lVert g\right\rVert_q\).

	\begin{enumerate}[(a)]
		\item Let
		      \[
			      f(x) = \frac{\left\vert g(x) \right\vert ^{q-1}\cdot \overline{\sgn(g(x))}}{\left\lVert g\right\rVert_q^{q-1}}
		      \]
		      Then \(\int \left\vert f \right\vert ^p = 1\), and \(\int fg = \left\lVert g\right\rVert _q\).\todo{Check}
		      \begin{note}
			      For \(\alpha \in \mathbb{C}\), \(\sgn (\alpha) \coloneqq e^{i\theta}\) where \(\alpha = \left\vert \alpha \right\vert e^{i\theta}\).
		      \end{note}
		\item The case that \(\mu\) is \hyperref[def:sigma-finite-measure]{\(\sigma\)-finite} and \(q = \infty, p = 1\) can be shown.\todo{DIY}
	\end{enumerate}
\end{proof}

\begin{remark}
	One could use the above to prove \hyperref[thm:Minkowski-inequality]{Minkowski's inequality} (as it only uses \hyperref[thm:Holder-inequality]{Hölder's inequality}).
\end{remark}

\begin{definition}[Dual space]\label{def:dual-space}
	For a \hyperref[def:norm]{normed} space \((V, \left\lVert \cdot\right\rVert )\), its \emph{dual space} is \(V^\ast = L(V, \mathbb{R} )\) or
	\(V^\ast = L(V, \mathbb{C})\).
\end{definition}
\begin{remark}
	Namely, the \hyperref[def:dual-space]{dual space} of \(V\) contains \hyperref[def:bounded-linear-transformation]{bounded linear transformations} with codomain being the scalar field.
\end{remark}

\begin{definition}[Linear functional]\label{def:linear-functional}
	Given a \hyperref[def:norm]{normed} space \((V, \left\lVert \cdot\right\rVert )\), \(\ell \in V^\ast\) is called a \emph{linear functional} on \(V\). i.e.,
	\begin{itemize}
		\item \(\ell \colon V \to \mathbb{R}\) (or \(\mathbb{C} \)).
		\item \(\ell \) is linear.
		\item There exists a \(c \geq 0\) such that \(\left\vert \ell(v) \right\vert  = c\left\lVert v\right\rVert\).
	\end{itemize}
\end{definition}

\begin{note}
	\(V^\ast\) is always a \hyperref[def:Banach-space]{Banach space} (even if \(V\) is not \hyperref[def:complete]{complete}).
\end{note}

\begin{corollary}\label{col:lp-lq-isometric}
	We have the following.
	\begin{enumerate}[(a)]
		\item Let \(1/p + 1/q = 1, 1 \leq q < \infty\). For \(g \in L^q\) define \(\ell_g \in L^p \to \mathbb{C}\) by
		      \[
			      \ell_g(f) = \int fg.
		      \]
		      Then \(\ell_g \in (L^p)^\ast\). Furthermore, \(\left\lVert \ell_g\right\rVert = \left\lVert g\right\rVert_q\).
		\item If \(\mu\) is \hyperref[def:finite-measure]{\(\sigma \)-finite}, then this also holds for \(q = \infty, p = 1\).
	\end{enumerate}
\end{corollary}
\begin{proof}
	\(\ell_g\) is clearly linear in \(f\) because the integral is linear. Then \autoref{prop:Lp-dual-formula} gives in both (1) and (2) that
	\[
		\left\lVert g\right\rVert_q = \sup\{\left\vert \ell_g(f) \right\vert  \mid \left\lVert g\right\rVert _p = 1 \} = \left\lVert \ell_g\right\rVert
	\]
	and so \(\ell_g\) is a \hyperref[def:bounded-linear-transformation]{bounded linear transformation} with the desired properties.
\end{proof}

\begin{theorem}
	We have the following.
	\begin{enumerate}[(a)]
		\item Let \(1/p + 1/q = 1\), \(1 \leq q < \infty\). The map \(T \colon L^q \to (L^p)^\ast\) given by \(Tg = \ell_g\) is an isometric\footnote{A map \(T\) is called isometric if for a given \(g\), \(\left\lVert Tg\right\rVert = \left\lVert g\right\rVert \).}
		      linear isomorphism. This means that
		      \begin{itemize}
			      \item \(T\) is a \hyperref[def:bounded-linear-transformation]{bounded linear transformation}.
			      \item \(T\) is bijective.
			      \item \(T\) is \hyperref[def:norm]{norm}-preserving.
		      \end{itemize}
		\item If \(\mu\) is \hyperref[def:sigma-finite-measure]{\(\sigma\)-finite} then this also holds for \(q = \infty, p = 1\).
	\end{enumerate}
\end{theorem}
\begin{proof}
	We have already proved this is isometric in \autoref{col:lp-lq-isometric}, it is clearly linear, and isometry implies injectivity.

	We will prove that it is surjective later.\todo{Fix!!!}
\end{proof}
\begin{note}
	Even if \(\mu\) is \hyperref[def:sigma-finite-measure]{\(\sigma\)-finite} we might not have \(L^1 \cong (L^\infty)^\ast\).
	Also note that \(L^2 \cong (L^2)^\ast\), and for all \(1 < p < \infty\) we have \((L^p)^{\ast\ast} \cong L^p\).
\end{note}
