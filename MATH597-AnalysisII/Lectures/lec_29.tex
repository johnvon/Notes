\lecture{29}{23 Mar. 11:00}{Hahn and Jordan Decomposition Theorem}
We now prove \autoref{thm:Hahn-decomposition-theorem}.
\begin{proof}[Proof of \autoref{thm:Hahn-decomposition-theorem}]
  We first show the uniqueness. We see that \(P \setminus P^\prime \subseteq P, P \setminus P^\prime  \subseteq N^\prime \). Thus, \(P \setminus P^\prime  \subseteq P \cap N^\prime \)
  is both \hyperref[def:positive-set-for-a-signed-measure]{positive} and \hyperref[def:negative-set-for-a-signed-measure]{negative}, hence \(P \setminus P^\prime\) is \hyperref[def:null-set-for-a-signed-measure]{null}.

  Similarly, for \(P^\prime \setminus P\), and then their union \(P \triangle P^\prime\) is \hyperref[def:null-set-for-a-signed-measure]{null} as well.

  \par To show the existence, without loss of generality suppose \(\nu \colon \mathcal{A} \to [-\infty,\infty)\). If not, consider \(-\nu\). Let
  \[
    s \coloneqq \sup\{\nu(E) \mid E \in \mathcal{A} \text{ is a \hyperref[def:positive-set-for-a-signed-measure]{positive set}}\},
  \]
  which is a nonempty supremum because \(\varnothing\) is \hyperref[def:positive-set-for-a-signed-measure]{positive}. Then there exist
  \(P_1,P_2,\ldots\) \hyperref[def:positive-set-for-a-signed-measure]{positive sets} such that \(\lim\limits_{n\to \infty } \nu(P_n) = s\).

  Then we have that \(P = \bigcup_n P_n\) is \hyperref[def:positive-set-for-a-signed-measure]{positive} by \autoref{lma:lec28-1}.
  We then have \(\nu(P) \leq s\) and \(\nu(P) = \nu(P_n) + \nu(P \setminus P_n) \geq \nu(P_n)\). Thus,
  \[
    \nu(P) \geq \lim_{n \to \infty} \nu(P_n) = s.
  \]
  Hence, \(\nu(P) = s\) and the supremum is in fact a max. We then know that \(s = \nu(P) < \infty\) because \(\nu\) does not attain the value infinity.

  Now let \(N = X \setminus P\). We claim that \(N\) is \hyperref[def:negative-set-for-a-signed-measure]{negative}. If not then there exists a
  \hyperref[def:measurable-set]{measurable} \(E \subseteq N\) with \(\nu(E) > 0\).
  By assumption, \(\nu(E) < \infty\). Then \(0 < \nu(E) < \infty\), so by \autoref{lma:lec28-2} there exists a \hyperref[def:measurable-set]{measurable}
  \(A \subseteq E\) such that \(A\) is \hyperref[def:positive-set-for-a-signed-measure]{positive} and \(\nu(A) > 0\).

  But we then know that
  \[
    \nu(P \cup A) = \nu(P) + \nu(A) > \nu(P)
  \]
  which is a contradiction since \(P \cup A\) is a \hyperref[def:positive-set-for-a-signed-measure]{positive set}, and \(\nu(P)\) is maximal. Therefore,
  \(N\) is \hyperref[def:negative-set-for-a-signed-measure]{negative}, and the theorem holds.

  \par Finally, if \(P^\prime , N^\prime \) is another pair of sets as in the statement of the theorem, we have
  \[
    P\setminus P^\prime \subset P,\quad P\setminus P^\prime \subset N^\prime,
  \]
  so that \(P\setminus P^\prime \) is both positive and negative, hence null; likewise for \(P^\prime \setminus P\).
\end{proof}

\begin{definition}[Singular]\label{def:singular}
  If \(\mu,\nu\) are \hyperref[def:signed-measure]{signed measures} on \((X, \mathcal{A})\), then we say \(\mu\) and \(\nu \) are \emph{singular to each other}, denoted as
  \(\mu \perp \nu\), if there exists \(E,F \in \mathcal{A}\) such that \(E \cap F = \varnothing, E \cup F = X\), \(F\) is \hyperref[def:null-set-for-a-signed-measure]{null}
  for \(\mu\), \(E\) is \hyperref[def:null-set-for-a-signed-measure]{null} for \(\nu\).
\end{definition}

\begin{eg}
  Consider \((\mathbb{R}, \mathcal{B}(\mathbb{R}))\) with
  \begin{itemize}
    \item The \hyperref[def:Lebesgue-measure]{Lebesgue measure} \(m\).
    \item The \hyperref[def:Cantor-measure]{Cantor measure} \(\mu_C\) \hyperref[def:Lebesgue-Stieltjes-measure]{induced} by the \hyperref[sssec:Cantor-Function]{Cantor function}.
    \item The \hyperref[eg:discrete-measure]{discrete measure} \(\mu_D = \delta_1 + 2\delta_{-1}\).
  \end{itemize}

  We then see that
  \begin{enumerate*}
    \item \(m\perp \mu _D\).
    \item \(m\perp \mu _c\).
    \item \(\mu _C \perp \mu _D\).
  \end{enumerate*}
\end{eg}
\begin{explanation}
  We see them as follows.
  \begin{enumerate}
    \item Take \(E = \mathbb{R} \setminus \{-1,1\}, F = \{1,-1\}\) to see that \(m \perp \mu_D\).
    \item Take \(E = \mathbb{R} \setminus K\) and \(F = K\) where \(K\) is the \hyperref[eg:lec8:Cantor-set]{Cantor set} to see that \(m \perp \mu_C\).
    \item We can also see that \(\mu_C \perp \mu_D\).
  \end{enumerate}
\end{explanation}

\begin{theorem}[Jordan decomposition theorem]\label{thm:Jordan-decomposition-theorem}
  Let \(\nu\) be a \hyperref[def:signed-measure]{signed measure} on \((X, \mathcal{A})\). Then there exists unique \hyperref[def:measure]{positive measures} \(\nu^+,\nu^-\)
  on \((X, \mathcal{A})\) such that for all \(E \in \mathcal{A}\) we have
  \[
    \nu(E) = \nu^+(E) - \nu^-(E)
  \]
  and \(\nu^+ \perp \nu^-\).
\end{theorem}
\begin{proof}
  For existence, we take \(\nu^+(E) \coloneqq \nu(E \cap P), \nu^-(E) \coloneqq -\nu(E \cap N)\) where \(P, N\) is the
  \hyperref[thm:Hahn-decomposition-theorem]{Hahn decomposition} of \(X\).

  If there exists \(\mu ^+, \mu ^-\) such that \(\nu = \mu ^+ + \mu ^-\) and \(\mu ^+ \perp \mu ^-\), let \(E, F\in \mathcal{\MakeUppercase{a}} \) be
  such that \(E\cap F = \varnothing \), \(E\cup F= X\), and \(\mu ^+(F) = \mu ^-(E) = 0\). Then we have that \(X = E \cup F\) is another
  \hyperref[thm:Hahn-decomposition-theorem]{Hahn decomposition} for \(\nu \), so \(P\triangle E\) is \(\nu\)-null. Therefore, for any
  \(A \in \mathcal{\MakeUppercase{a}} \), \(\mu ^+(A) = \mu ^+(A \cap E) = \nu (A \cap E) = \nu (A \cap P) = \nu ^+(A)\), hence \(\mu ^+ = \nu ^+\).
  Likewise, we have \(\nu ^- = \mu ^-\).
\end{proof}