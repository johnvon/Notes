\lecture{17}{14 Feb.\ 11:00}{Product Measure}
We now see a lemma.
\begin{lemma}
	Suppose \(\mathcal{A} _\alpha  = \left< \mathcal{E} _\alpha  \right> \) for every \(\alpha \in I\). Then
	\begin{enumerate}[(1)]
		\item \(\pi _\alpha ^{-1} (\mathcal{A} _\alpha ) = \left< \pi ^{-1} _\alpha (\mathcal{E} _\alpha ) \right> \)
		\item \(\bigotimes_\alpha  \mathcal{A} _\alpha = \left< \bigcup_{\alpha} \pi _\alpha ^{-1} (\mathcal{E} _\alpha )\right> \)
		\item If \(I\) is countable, then
		      \[
			      \bigotimes_{i=1}^{\infty }\mathcal{A} _i = \left< \left\{\prod_{i=1}^{\infty} E_{i} \mid \underset{i}{\forall}\ E_{i}\in \mathcal{E} _{i}\right\} \right>
		      \]
	\end{enumerate}
\end{lemma}
\begin{proof}
	We prove this one by one.
	\begin{enumerate}[(1)]
		\item Note that for \(f\colon Y\to Z\), and \(\mathcal{B} \) be a \hyperref[def:sigma-algebra]{\(\sigma\)-algebra} on \(Z\), then
		      \(f^{-1} (\mathcal{B} )\) is also a \hyperref[def:sigma-algebra]{\(\sigma\)-algebra} since
		      \(f^{-1} (\varnothing ) = \varnothing\), \(f^{-1} (B)^{c} = f^{-1} (B^{c} )\), and \(\bigcup_{n} f^{-1} (B_{n}) = f^{-1} (\bigcup_{n}B_{n})\).
		      Hence, \(\pi ^{-1}_\alpha \) is a \hyperref[def:sigma-algebra]{\(\sigma\)-algebra} on \(X\), i.e.,
		      \[
			      \pi _\alpha ^{-1} (\mathcal{E} _\alpha )\subset \pi _\alpha ^{-1} (\mathcal{A} _\alpha )\overset{\hyperref[lma:lec2-1]{!}}{\implies} \left< \pi ^{-1} _\alpha (\mathcal{E} _\alpha ) \right> \subset \pi _\alpha ^{-1} (\mathcal{A} _\alpha ).
		      \]
		      To show the other direction, let \(\mathcal{M} \) being
		      \[
			      \mathcal{M} = \left\{B\subset X_\alpha  \mid \pi _\alpha ^{-1} (B)\in \left< \pi _\alpha ^{-1} (\mathcal{E} _\alpha ) \right> \right\}.
		      \]
		      We now check
		      \begin{itemize}
			      \item \(\mathcal{M} \) is a \hyperref[def:sigma-algebra]{\(\sigma\)-algebra}. \todo{Check (Easy)!}
			      \item \(\mathcal{E} _\alpha \subset \mathcal{M}\). This is true by definition of \(\mathcal{M} \).
		      \end{itemize}
		      Thus, \(\left< \mathcal{E} _\alpha \right> = \mathcal{A} _\alpha \subset \mathcal{M} \). Hence,
		      if \(E\in \mathcal{A} _\alpha \), \(E\in \mathcal{M} \), implying
		      \[
			      \pi _\alpha ^{-1} (E)\in \left< \pi _\alpha ^{-1} (\mathcal{E} _\alpha ) \right>,
		      \]
		      i.e., \(\mathcal{A} _\alpha \subset \left< \pi_\alpha ^{-1} (\mathcal{E} _\alpha ) \right> \).
		\item \todo{DIY}
		\item \todo{DIY}
	\end{enumerate}
\end{proof}

\begin{theorem}
	Suppose \(X_1, \dots , X_d \) are metric spaces. Let \(X = \prod_{i=1}^{d} X_{i}\) with product metric defined as
	\[
		\rho (x, y) = \sum_{i=1}^{d} \rho _{i}(x_{i}, y_{i}).
	\]
	Then,
	\begin{enumerate}[(1)]
		\item \(\bigotimes_{i=1}^{d}\mathcal{B} (X_{i})\subset \mathcal{B} (X)\)
		\item If in addition, each \(X_{i}\) has a countable dense subset,
		      \[
			      \bigoplus_{i=1}^{d}\mathcal{B} (X_{i}) = \mathcal{B} (X).
		      \]
	\end{enumerate}
\end{theorem}
\begin{proof}
	\todo{DIY}
\end{proof}

\begin{remark}
	We see that
	\begin{itemize}
		\item \(\mathcal{B} (\mathbb{R} ^d) = \mathcal{B} (\mathbb{R} )\otimes \dots \otimes \mathcal{B} (\mathbb{R} ) \)
		\item let \(f = u + iv\colon X\to \mathbb{C} \), and \(\mathcal{A} \) be a \hyperref[def:sigma-algebra]{\(\sigma\)-algebra} on \(X\). Then
		      \[
			      \underset{E\in \mathcal{B} (\mathbb{R} )}{\forall }\ u^{-1} (E), v^{-1} (E)\in \mathcal{A}
			      \iff
			      f^{-1} (F)\in \mathcal{A}, \forall\ F\in \mathcal{B} (\mathbb{C} )
		      \]
		      with \(\mathcal{B} (\mathbb{C} ) = \mathcal{B} (\mathbb{R} ^2) = \mathcal{B} (\mathbb{R} )\otimes \mathcal{B} (\mathbb{R} )\).
	\end{itemize}
\end{remark}

We first focus on \(2\) dimensional case. Specifically, we think of our coordinate is \(x\) and \(y\) on \(\mathbb{R} ^2\).

\begin{definition*}
	Let \(X, Y\) be two sets, then we have the following.
	\begin{definition}[\(x\)-section, \(y\)-section for set]
		For \(E\subset X\times Y\),
		\[
			E_x = \left\{y\in Y \mid (x, y)\in E\right\},\quad E^y = \left\{x\in X \mid (x, y)\in E\right\},
		\]
		where \(E_x\) is called the \emph{\(x\)-section of \(E\)}, while \(E_y\) is called the \emph{\(y\)-section of \(E\)}.
	\end{definition}
	\begin{definition}[\(x\)-section, \(y\)-section for function]
		For \(f\colon X\times Y\to \mathbb{C} \), define \(f_x\colon Y\to \mathbb{C} ,\quad f^y\colon X\to \mathbb{C}\)
		by
		\[
			f_x(y) = f(x, y) = f^y(x),
		\]
		where \(f_x(y)\) is called the \emph{\(x\)-section of \(f\)}, while \(f_y(x)\) is called the \emph{\(y\)-section of \(f\)}.
	\end{definition}
\end{definition*}
\begin{eg}
	We see that
	\[
		\left(\mathbbm{1}_{E} \right)_x = \mathbbm{1}_{E_x}
	\]
	and
	\[
		\left(\mathbbm{1}_{E} \right)^y = \mathbbm{1}_{E^y}.
	\]
\end{eg}

\begin{proposition}
	Given two \hyperref[def:measurable-space]{measurable spaces} \((X, \mathcal{A} )\) and \((Y, \mathcal{B} )\), then
	\begin{enumerate}[(1)]
		\item If \(E\in \mathcal{A} \otimes \mathcal{B}\), then
		      \[
			      \underset{x\in X}{\forall }\ \underset{y\in Y}{\forall }\ E_x\in \mathcal{B} , E^y\in \mathcal{A}.
		      \]
		\item If \(f\colon X\times Y\to \mathbb{C} \) is \hyperref[def:A-measurable-function]{\(\mathcal{A} \otimes \mathcal{B} \)-measurable}, then
		      \[
			      \underset{x\in X}{\forall }\ \underset{y\in Y}{\forall }\ f_x \text{ is \hyperref[def:A-measurable-function]{\(\mathcal{B}\)-measurable}}, f^y \text{ is \hyperref[def:A-measurable-function]{\(\mathcal{A}\)-measurable}}.
		      \]
	\end{enumerate}
\end{proposition}
\begin{proof}
	We prove this one by one.
	\begin{enumerate}[(1)]
		\item Let \(\mathcal{F} \coloneqq \left\{E\subset X\times Y \mid \underset{x\in X}{\forall }\ \underset{y\in Y}{\forall }\ E_x\in \mathcal{B} , E^y\in \mathcal{A}\right\}\), then
		      \begin{itemize}
			      \item \(\mathcal{F} \) is a \hyperref[def:sigma-algebra]{\(\sigma\)-algebra}.
			            \begin{itemize}
				            \item \(\varnothing \in \mathcal{F} \).
				            \item \((E^{c} )_x = E_{x}^{c}\).
				            \item \(\left(\bigcup_{j=1}^{\infty} E_{j} \right)_x = \bigcup_{j=1}^{\infty} (E_{j} )_{x} \).
			            \end{itemize}
			            And the same is true for \(y\).
			      \item Let \(\mathcal{R} _0 \coloneqq \left\{A\times B \mid A\in \mathcal{A} , B\in \mathcal{B} \right\}\subset \mathcal{F}\), which is again
			            easy to show from definition.
		      \end{itemize}
		      Hence, we see that \(\left< R_0 \right> = \mathcal{A} \otimes \mathcal{B} \subset \mathcal{F} \).
		\item Since
		      \[
			      (f_{x} )^{-1} (B) = (f^{-1} (B))_{x}
		      \]
		      and
		      \[
			      (f^{y} )^{-1} ({B} ) = (f^{-1} (B))^{y},
		      \]
		      the result follows from (1).
	\end{enumerate}
\end{proof}

\section{Product Measures}
We start with the definition.
\begin{definition}[Rectangle]\label{def:rectangle}
	Given two \hyperref[def:measurable-space]{measurable spaces} \((X, \mathcal{A} ), (Y, \mathcal{B} )\),
	a \emph{(measurable) rectangle} is \(R = A\times B\) where \(A\in \mathcal{A} \)
	and \(B\in \mathcal{B} \). Furthermore, we let
	\[
		\mathcal{R} _0 \coloneqq \left\{R = A\times B \mid A\in \mathcal{A} , B\in \mathcal{B} \right\},
	\]
	and
	\[
		\mathcal{R} \coloneqq \left\{\bigcup_{i=1}^{N} R_{i} \mid N\in \mathbb{N} , R_1, \dots , R_N \text{ disjoint \hyperref[def:rectangle]{rectangles}}\right\}.
	\]
\end{definition}
\begin{note}
	Whenever we're talking about \hyperref[def:rectangle]{rectangle}, they're always \hyperref[def:measurable-set]{measurable}.
\end{note}

\begin{lemma}
	\(\mathcal{R} \) is an \hyperref[def:algebra]{algebra}, and
	\[
		\left< \mathcal{R} _0 \right> = \left< \mathcal{R}  \right> = \mathcal{A} \otimes \mathcal{B}.
	\]
\end{lemma}
\begin{proof}
	Simply observe that
	\[
		(A\times B)^{c} = (A^{c} \times Y)\cup (X\times B^{c} ).
	\]\todo{DIY}
\end{proof}