\lecture{4}{12 Jan. 11:00}{Carathéodory extension Theorem}
We now prove \autoref{prop:outer-measure}.
\begin{proof}[Proof of \autoref{prop:outer-measure}]
	We need to prove
	\begin{itemize}
		\item \(\mu^{*} \) is well-defined. i.e., \(\inf\) is taken over a non-empty set. This is trivial since \(X\in \mathcal{E}\) and \(X\supset A\) for any \(A\in \mathcal{E}\).
		\item \(\mu^{*} (\varnothing ) = 0\). Since \(\varnothing \in \mathcal{E}\) and
		      \[
			      \mu^{*} (\varnothing ) = \inf \left\{\sum\limits_{i=1}^{\infty} \rho(E_{i})\colon \underset{i\in \mathbb{\MakeUppercase{N}} }{\forall}E_{i}\in \mathcal{E} , \bigcup\limits_{i=1}^{\infty} E_{i}\supset \varnothing \right\} = 0
		      \]
		      since \(\rho(\varnothing ) = 0\) for all \(i\) and further, by Squeeze Theorem, we see that \(\lim\limits_{n \to \infty} \sum\limits_{i=1}^{n} \rho(\varnothing) = 0\).
		\item \(A\subset B\implies \mu^{*} (A)\leq \mu^{*} (B)\). We simply show this by contradiction. Suppose \(A\subset B\) and \(\mu^{*} (A)>  \mu^{*} (B)\), then by
		      definition of \(\mu^{*} \), we have
		      \[
			      \begin{split}
				      \mu^{*} (A) &= \inf\left\{\sum\limits_{i=1}^{\infty} \rho(E_{i})\colon \underset{i\in\mathbb{\MakeUppercase{N}}}{\forall}E_{i}\in \mathcal{E},\bigcup\limits_{i=1}^{\infty} E_{i}\supset A \right\}\\
				      &>\inf\left\{\sum\limits_{i=1}^{\infty} \rho(E_{i})\colon \underset{i\in\mathbb{\MakeUppercase{N}}}{\forall}E_{i}\in \mathcal{E},\bigcup\limits_{i=1}^{\infty} E_{i}\supset B \right\} = \mu^{*} (B).
			      \end{split}
		      \]
		      Now, let \(B \eqqcolon (B\setminus A)\cup A\), then we have
		      \[
			      \begin{split}
				      \mu^{*} (A) &= \inf\left\{\sum\limits_{i=1}^{\infty} \rho(E_{i})\colon \underset{i\in\mathbb{\MakeUppercase{N}}}{\forall}E_{i}\in \mathcal{E},\bigcup\limits_{i=1}^{\infty} E_{i}\supset A \right\}\\
				      &>\inf\left\{\sum\limits_{i=1}^{\infty} \rho(E_{i})\colon \underset{i\in\mathbb{\MakeUppercase{N}}}{\forall}E_{i}\in \mathcal{E},\bigcup\limits_{i=1}^{\infty} E_{i}\supset (B\setminus A)\cup A \right\} = \mu^{*} (B).
			      \end{split}
		      \]
		      Now, since \(B\setminus A\supseteq \varnothing \), then this inequality can't hold, hence a contradiction\conta.
		\item Countable subadditivity. Let \(A_1, A_2, \ldots \in X \). If one of \(\mu^{*} (A_{n}) = \infty \), then result holds.
		      So we may assume \(\mu^{*} (A_n)<\infty \) for all \(n\in \mathbb{\MakeUppercase{N}} \). Now, fix any \(\epsilon >0\),
		      we will show that
		      \[
			      \mu^{*} \left(\bigcup\limits_{n=1}^{\infty} A_n\right)\leq \sum\limits_{n=1}^{\infty} \mu^{*} (A_n)+\epsilon.
		      \]
		      For each \(n\in\mathbb{\MakeUppercase{N}} \), \(\exists E_{n, 1}, E_{n, 2}, \ldots \in \mathcal{E}\) such that
		      \[
			      \bigcup\limits_{k=1}^{\infty} E_{n, k}\supset A_n
		      \]
		      and
		      \[
			      \mu^{*} (A_n)+\frac{\epsilon}{2^n} \geq \sum\limits_{k=1}^{\infty} \rho(E_{n, k}).
		      \]
		      \begin{remark}
			      This is an important trick! We often set the error term as \(\epsilon / 2^n\) instead of \(\epsilon \) as in above.
		      \end{remark}

		      Then we see that
		      \[
			      \bigcup\limits_{k=1}^{\infty} A_n \subset \bigcup\limits_{n=1}^{\infty} \bigcup\limits_{k=1}^{\infty} E_{k, n} = \bigcup\limits_{(n, k)\in\mathbb{\MakeUppercase{N}} ^2} E_{k, n},
		      \]
		      which implies
		      \[
			      \mu^{*} \left(\bigcup\limits_{n=1}^{\infty} A_{n}\right)
			      \leq \sum\limits_{(n, k)\in\mathbb{\MakeUppercase{N}} ^2} \rho\left(E_{k, n}\right)
			      \overset{\hyperref[thm:Tonelli-theorem-for-series]{\text{!}}}{=} \sum\limits_{n=1}^{\infty} \sum\limits_{k=1}^{\infty} \rho(E_{k, n})
			      \leq \sum\limits_{n=1}^{\infty} \left(\mu^{*} (A_n) + \frac{\epsilon}{2^n}\right)
		      \]
		      from the inequality just derived. Now, since the last term is just
		      \[
			      \sum\limits_{n=1}^{\infty} \left(\mu^{*} (A_n) + \frac{\epsilon}{2^n}\right) = \sum\limits_{n=1}^{\infty} \mu^{*} (A_n)+\epsilon,
		      \]
		      hence we finally have
		      \[
			      \mu^{*} \left(\bigcup\limits_{n=1}^{\infty} A_{n}\right)\leq \sum\limits_{n=1}^{\infty} \mu^{*} (A_n)+\epsilon
		      \]
		      for arbitrarily small fixed \(\epsilon >0\), hence the subadditivity is proved.
	\end{itemize}
\end{proof}

\begin{definition}[Carathéodory measurable]\label{def:C-measurable}
	Let \(\mu^{*} \) be an \hyperref[def:outer-measure]{outer measure} on \(X\). We say \(A\subset X\) is \emph{Carathéodory measurable (C-measurable) with respect to \(\mu^{*} \)} if
	\[
		\forall E\subset X,\ \mu^{*} (E) = \mu^{*} \left(E\cap A\right) + \mu^{*} \left(E\setminus A\right).
	\]
\end{definition}

\begin{lemma}\label{lma:disjoint-C-measurable-finite-additive}
	Let \(\mu^{*} \) be an \hyperref[def:outer-measure]{outer measure} on \(X\). Suppose \(B_1, \ldots , B_N \) are \underline{disjoint} \hyperref[def:C-measurable]{C-measurable} sets.
	Then,
	\[
		\forall E\subset X,\ \mu^{*} \left(E\cap \left(\bigcup\limits_{i=1}^{N} B_{i}\right)\right) = \sum\limits_{i=1}^{N} \mu^{*} \left(E\cap B_i\right).
	\]
\end{lemma}
\begin{proof}
	Since we have
	\[
		\begin{split}
			\mu^{*} \left(E\cap \left(\bigcup\limits_{i=1}^{N} B_{i}\right)\right)
			&= \mu^{*} \left( E^\prime  \cap B_1\right) + \mu^{*} \left(E ^\prime \setminus B_1\right)\\
			&= \mu^{*} \left(E\cap \left(\bigcup\limits_{i=1}^{N} B_{i}\cap B_1\right)\right) + \mu^{*} \left(E\cap \left(\bigcup\limits_{i=1}^{N} B_{i}\right)\cap B^{c}_1\right)\\
			&= \mu^{*} (E\cap B_1) + \mu^{*} \left(E\cap \left(\bigcup\limits_{i=2}^{N} B_{i}\right)\right)
		\end{split}
	\]
	where the equality comes from the fact that \(B_1 \) is \hyperref[def:C-measurable]{C-measurable} and disjoint from \(B_{i}\), \(i\neq 1\). Then, we simply iterate this argument
	and have the result. Note that in the first inequality, we define \(E ^\prime\coloneqq E\cap \left(\bigcup\limits_{i=1}^{N} B_{i}\right)\) for the simplicity of notation.
\end{proof}
\begin{remark}
	This implies that if we restrict an \hyperref[def:outer-measure]{outer measure} on a \hyperref[def:C-measurable]{C-measurable} set, then it becomes \underline{finite additive}.
\end{remark}

\begin{theorem}[Carathéodory extension Theorem]\label{thm:Caratheodory-extension-Thm}
	Let \(\mu^{*} \) be an \hyperref[def:outer-measure]{outer measure} on \(X\). Let \(\mathcal{A} \) be the collection of \hyperref[def:C-measurable]{C-measurable} sets (with respect to \(\mu^{*} \)).
	Then,
	\begin{enumerate}
		\item \(\mathcal{A}\) is a \hyperref[def:sigma-algebra]{\(\sigma\)-algebra} on \(X\).
		\item \(\mu = \at{\mu^{*} }{\mathcal{A} }{}\) is a \hyperref[def:measure]{measure} on \((X, \mathcal{A})\).
		\item \((X, \mathcal{A} , \mu)\) is a \hyperref[def:complete-measure-space]{complete measure space}.
	\end{enumerate}
\end{theorem}

\begin{proof}\let\qed\relax
	We divide the proof in several steps.
	\begin{enumerate}
		\item We show \(\mathcal{A}\) is a \hyperref[def:sigma-algebra]{\(\sigma\)-algebra} by showing
		      \begin{enumerate}
			      \item \(\varnothing \in \mathcal{A} \). To show this, we simply check that \(\varnothing \) is \hyperref[def:C-measurable]{C-measurable}. We see that
			            \[
				            \underset{E\subset X}{\forall}\ \mu^{*} (E) = \mu^{*} (E\cap \varnothing ) + \mu^{*} (E \setminus \varnothing ) = \mu^{*} (E),
			            \]
			            which just shows \(\varnothing \in \mathcal{A}\).
			      \item \(\mathcal{A} \) closed under complements. This is equivalent to say that if \(A\) is \hyperref[def:C-measurable]{C-measurable}, so is \(A^{c}\).
			            We see that if \(A\) is \hyperref[def:C-measurable]{C-measurable}, then for every \(E\subset X\),
			            \[
				            \mu^{*} (E) = \mu^{*} (E\cap A) + \mu^{*} (E\setminus A).
			            \]
			            Observing that \(E\cap A = E\setminus A^{c} \) and \(E\setminus A = E\cap A^{c} \), hence
			            \[
				            \mu^{*} (E) = \mu^{*} (E\setminus A^{c} ) + \mu^{*} (E\cap A^{c} ).
			            \]
			            \par We immediately see that above implies \(A^{c} \in\mathcal{A} \).
			      \item \(\mathcal{A} \) closed under countable unions.
			            \begin{note}
				            To show \(\mathcal{A} \) closed under countable unions, we show that \(\mathcal{A}\) is closed under:
				            \begin{center}
					            finite unions \(\overset{\text{then}}{\implies}\) countable \underline{disjoint} unions \(\overset{\text{then}}{\implies}\) countable unions.
				            \end{center}
			            \end{note}
			            \begin{itemize}
				            \item\label{pf:Caratheodory-extension-Thm-1-finite-unions} We show \(\mathcal{A} \) is closed under finite unions.

				            \textbf{Claim.} \(A, B\in \mathcal{A} \implies A\cup B\in \mathcal{A}\).
				            \par Fix \(E\subset X\) arbitrary. We need to show that
				            \[
					            \mu^{*} (E) = \mu^{*} (E\cap (A\cup B)) + \mu^{*} (E\setminus (A\cup B)),
				            \]
				            i.e.,
				            \[
					            \mu^{*} (1\cup 2\cup 3\cup 4) = \mu^{*} (1\cup 2\cup 3) + \mu^{*} (4)
				            \]
				            given \(A, B\in\mathcal{A}\).
				            \begin{figure}[H]
					            \centering
					            \incfig{thm:Caratheodory-extension-Thm-1a}
					            \label{fig:thm:Caratheodory-extension-Thm-1a}
				            \end{figure}

				            \begin{itemize}
					            \item Since \(A\) is \hyperref[def:C-measurable]{C-measurable},
					                  \begin{itemize}
						                  \item \(\mu^{*} (1\cup 2\cup 3\cup 4) = \mu^{*} (1\cup 2) + \mu^{*} (3\cup 4)\)
						                  \item \(\mu^{*} (1\cup 2\cup 3) = \mu^{*} (1\cup 2) + \mu^{*} (3)\)
					                  \end{itemize}
					            \item Since \(B\) is \hyperref[def:C-measurable]{C-measurable},
					                  \begin{itemize}
						                  \item \(\mu^{*} (3\cup 4) = \mu^{*} (3) + \mu^{*} (4)\)
					                  \end{itemize}
				            \end{itemize}
				            Hence, we have
				            \[
					            \begin{split}
						            \mu^{*} (1\cup 2\cup 3\cup 4) &= \mu^{*} (1\cup 2)+\mu^{*} (3\cup 4)\\
						            &= \mu^{*} (1\cup 2) + \mu^{*} (3) + \mu^{*} (4)\\
						            &= \mu^{*}(1\cup 2\cup 3) + \mu^{*} (4).
					            \end{split}
				            \]
				            \item We show \(\mathcal{A} \) is closed under countable \underline{disjoint} unions.

				                  \par Let \(A_1, A_2, \ldots \in\mathcal{A}\) and \underline{disjoint}.
				                  Fix \(E\subset X\) arbitrary. Since \(\mu^{*} \) is countably subadditive,
				                  \[
					                  \mu^{*} (E) \leq  \mu^{*} \left(E\cap \bigcup\limits_{i=1}^{\infty} A_{i}\right) + \mu^{*} \left(E\setminus \bigcup\limits_{i=1}^{\infty} A_{i}\right),
				                  \]
				                  hence we only need to show another way around.

				                  Fix \(N\in\mathbb{\MakeUppercase{N}} \), we have \(\bigcup\limits_{n=1}^{N} A_{n}\in \mathcal{A} \) since \hyperref[pf:Caratheodory-extension-Thm-1-finite-unions]{\(N\) is finite}, and
				                  \[
					                  \begin{split}
						                  \mu^{*} (E) &= \mu^{*} \left(E\cap \left(\bigcup\limits_{n=1}^{N} A_{n}\right)\right) + \mu^{*} \left(E\setminus \left(\bigcup\limits_{n=1}^{N} A_{n}\right)\right)\\
						                  &\geq \underbrace{\sum\limits_{n=1}^{N} \mu^{*} (E\cap A_{n})}_{ \overset{\hyperref[lma:disjoint-C-measurable-finite-additive]{!}}{=} \mu^{*} \left(E\cap \left(\bigcup\limits_{n=1}^{N} A_{n}\right)\right)} + \underbrace{\mu^{*} \left(E\setminus \bigcup\limits_{n=1}^{\infty} A_{n}\right)}_{\leq \mu^{*} \left(E\setminus \left(\bigcup\limits_{n=1}^{N} A_{n}\right)\right)}.
					                  \end{split}
				                  \]
				                  Now, take \(N\to \infty \) then we are done.
				            \item We show \(\mathcal{A} \) is closed under countable unions.\todo{DIY}
			            \end{itemize}
		      \end{enumerate}
	\end{enumerate}
	\emph{The proof will be \hyperref[pf:Caratheodory-extension-Thm-cont]{continued}\ldots}
\end{proof}