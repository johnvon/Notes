\lecture{4}{12 Jan.\ 11:00}{Carathéodory Extension Theorem}
\begin{prev}
	Last time we skip the proof of \autoref{prop:outer-measure}, which is a quite important theorem for building a \hyperref[def:measure]{measure}.
\end{prev}

To see this, we note that from \autoref{prop:outer-measure}, given a positive function \(\rho \) defined on a subset of the power set of \(X\) with \(\rho (\varnothing ) = 0\), we can induce an \hyperref[def:outer-measure]{outer measure} from \(\rho \).

\begin{note}
	Later, we can further induce a natural \hyperref[def:measure]{measure} from the induced \hyperref[def:outer-measure]{outer measure}.
\end{note}

We now prove \autoref{prop:outer-measure}.
\begin{proof}[Proof of \autoref{prop:outer-measure}]
	We need to prove the following.

	\begin{claim}
		\(\mu ^{\ast} \) is well-defined, i.e., \(\inf\) is taken over a non-empty set.
	\end{claim}
	\begin{explanation}
		This is trivial since \(X\in \mathcal{E}\) and \(X\supset A\) for any \(A\in \mathcal{E}\).
	\end{explanation}

	\begin{claim}
		\hyperref[def:outer-measure-null-empty-set]{Null empty set} holds, i.e., \(\mu ^{\ast} (\varnothing ) = 0\).
	\end{claim}
	\begin{explanation}
		Since \(\varnothing \in \mathcal{E}\) and
		\[
			\mu^{\ast} (\varnothing ) = \inf \left\{\sum_{i=1}^{\infty} \rho(E_{i})\colon \underset{i\in \mathbb{N} }{\forall}E_{i}\in \mathcal{E} , \bigcup_{i=1}^{\infty} E_{i}\supset \varnothing \right\} = 0
		\]
		since \(\rho(\varnothing ) = 0\) for all \(i\) and further, by Squeeze Theorem, \(\lim_{n \to \infty} \sum_{i=1}^{n} \rho(\varnothing) = 0\).
	\end{explanation}

	\begin{claim}
		\hyperref[def:outer-measure-montonicity]{Monotonicity} holds, i.e., \(A\subset B\implies \mu^{\ast} (A)\leq \mu^{\ast} (B)\).
	\end{claim}
	\begin{explanation}
		Suppose for a contradiction that \(A\subset B\) and \(\mu^{\ast} (A)>  \mu^{\ast} (B)\), then by definition of \(\mu^{\ast} \),
		\[
			\begin{split}
				\mu^{\ast} (A)
				 & = \inf\left\{\sum_{i=1}^{\infty} \rho(E_{i})\colon \underset{i\in\mathbb{N}}{\forall}E_{i}\in \mathcal{E},\bigcup_{i=1}^{\infty} E_{i}\supset A \right\} \\
				 & >\inf\left\{\sum_{i=1}^{\infty} \rho(E_{i})\colon \underset{i\in\mathbb{N}}{\forall}E_{i}\in \mathcal{E},\bigcup_{i=1}^{\infty} E_{i}\supset B \right\}
				= \mu^{\ast} (B).
			\end{split}
		\]
		Now, let \(B \eqqcolon (B\setminus A)\cup A\), then we have
		\[
			\begin{split}
				\mu^{\ast} (A)
				 & = \inf\left\{\sum_{i=1}^{\infty} \rho(E_{i})\colon \underset{i\in\mathbb{N}}{\forall}E_{i}\in \mathcal{E},\bigcup_{i=1}^{\infty} E_{i}\supset A \right\}                   \\
				 & >\inf\left\{\sum_{i=1}^{\infty} \rho(E_{i})\colon \underset{i\in\mathbb{N}}{\forall}E_{i}\in \mathcal{E},\bigcup_{i=1}^{\infty} E_{i}\supset (B\setminus A)\cup A \right\}
				= \mu^{\ast} (B).
			\end{split}
		\]
		Now, since \(B\setminus A\supseteq \varnothing \), then this inequality can't hold, hence a contradiction.
	\end{explanation}

	\begin{claim}
		\hyperref[def:outer-measure-countable-subadditivity]{Countable subadditivity} holds, i.e., \(\mu^{\ast} \left(\bigcup_{i=1}^{\infty} A_{i}\right) \leq \sum_{i=1}^{\infty} \mu^{\ast} (A_{i})\) for every \(A_{i} \subset X \).
	\end{claim}
	\begin{explanation}
		Let \(A_1, A_2, \dots \in X \). If one of \(\mu^{\ast} (A_{n}) = \infty \), then result holds.
		So we may assume \(\mu^{\ast} (A_n)<\infty \) for all \(n\in \mathbb{N} \). Now, fix any \(\epsilon >0\),
		we will show that
		\[
			\mu^{\ast} \left(\bigcup_{n=1}^{\infty} A_n\right)\leq \sum_{n=1}^{\infty} \mu^{\ast} (A_n)+\epsilon.
		\]
		For each \(n\in\mathbb{N} \), \(\exists E_{n, 1}, E_{n, 2}, \dots \in \mathcal{E}\) such that
		\(\bigcup_{k=1}^{\infty} E_{n, k}\supset A_n\) and \(\mu^{\ast} (A_n) + \epsilon /2^n \geq \sum_{k=1}^{\infty} \rho(E_{n, k})\).
		\begin{remark}
			This is an important trick! We often set the error term as \(\epsilon / 2^n\) instead of \(\epsilon \) as in above to accommodate the summation over
			a countable set.
		\end{remark}

		Then we see that
		\[
			\bigcup_{k=1}^{\infty} A_n \subset \bigcup_{n=1}^{\infty} \bigcup_{k=1}^{\infty} E_{k, n} = \bigcup_{(n, k)\in\mathbb{N} ^2} E_{k, n},
		\]
		which implies
		\[
			\mu^{\ast} \left(\bigcup_{n=1}^{\infty} A_{n}\right)
			\leq \sum_{(n, k)\in\mathbb{N} ^2} \rho\left(E_{k, n}\right)
			\overset{\hyperref[thm:Tonelli-theorem-for-series]{\text{!}}}{=} \sum_{n=1}^{\infty} \sum_{k=1}^{\infty} \rho(E_{k, n})
			\leq \sum_{n=1}^{\infty} \left(\mu^{\ast} (A_n) + \frac{\epsilon}{2^n}\right)
		\]
		from the inequality just derived. Now, since the last term is just
		\[
			\sum_{n=1}^{\infty} \left(\mu^{\ast} (A_n) + \frac{\epsilon}{2^n}\right) = \sum_{n=1}^{\infty} \mu^{\ast} (A_n)+\epsilon,
		\]
		hence we finally have
		\[
			\mu^{\ast} \left(\bigcup_{n=1}^{\infty} A_{n}\right)\leq \sum_{n=1}^{\infty} \mu^{\ast} (A_n)+\epsilon
		\]
		for arbitrarily small fixed \(\epsilon >0\), hence the subadditivity is proved.
	\end{explanation}
\end{proof}

\begin{definition}[Carathéodory measurable]\label{def:C-measurable}
	Let \(\mu^{\ast} \) be an \hyperref[def:outer-measure]{outer measure} on \(X\). We say \(A\subset X\) is \emph{Carathéodory measurable with respect to \(\mu^{\ast} \)} if
	\[
		\forall E\subset X,\ \mu^{\ast} (E) = \mu^{\ast} \left(E\cap A\right) + \mu^{\ast} \left(E\setminus A\right).
	\]
\end{definition}
\begin{note}
	We sometimes write \emph{C-measurable} instead of \hyperref[def:C-measurable]{Carathéodory measurable} for convenience.
\end{note}
\begin{lemma}\label{lma:disjoint-C-measurable-finite-additive}
	Let \(\mu^{\ast} \) be an \hyperref[def:outer-measure]{outer measure} on \(X\). Suppose \(B_1, \dots , B_N \) are \underline{disjoint} \hyperref[def:C-measurable]{C-measurable} sets.
	Then,
	\[
		\forall E\subset X,\ \mu^{\ast} \left(E\cap \left(\bigcup_{i=1}^{N} B_{i}\right)\right) = \sum_{i=1}^{N} \mu^{\ast} \left(E\cap B_i\right).
	\]
\end{lemma}
\begin{proof}
	Since we have
	\[
		\begin{split}
			\mu^{\ast} \left(E\cap \left(\bigcup_{i=1}^{N} B_{i}\right)\right)
			 & = \mu^{\ast} \left( E^\prime  \cap B_1\right) + \mu^{\ast} \left(E ^\prime \setminus B_1\right)                                                               \\
			 & = \mu^{\ast} \left(E\cap \left(\bigcup_{i=1}^{N} B_{i}\cap B_1\right)\right) + \mu^{\ast} \left(E\cap \left(\bigcup_{i=1}^{N} B_{i}\right)\cap B^{c}_1\right) \\
			 & = \mu^{\ast} (E\cap B_1) + \mu^{\ast} \left(E\cap \left(\bigcup_{i=2}^{N} B_{i}\right)\right)
		\end{split}
	\]
	where the equality comes from the fact that \(B_1 \) is \hyperref[def:C-measurable]{C-measurable} and disjoint from \(B_{i}\), \(i\neq 1\). Then, we simply iterate this argument
	and have the result. Note that in the first inequality, we define \(E ^\prime\coloneqq E\cap \left(\bigcup_{i=1}^{N} B_{i}\right)\) for the simplicity of notation.
\end{proof}
\begin{remark}
	This implies that if we restrict an \hyperref[def:outer-measure]{outer measure} on a \hyperref[def:C-measurable]{C-measurable} set, then it becomes \underline{finite additive}.
\end{remark}

\begin{theorem}[Carathéodory extension Theorem]\label{thm:Caratheodory-extension-Thm}
	Let \(\mu^{\ast} \) be an \hyperref[def:outer-measure]{outer measure} on \(X\). Let \(\mathcal{A} \) be the collection of \hyperref[def:C-measurable]{C-measurable} sets (with respect to \(\mu^{\ast} \)).
	Then,
	\begin{enumerate}[(a)]
		\item \(\mathcal{A}\) is a \hyperref[def:sigma-algebra]{\(\sigma\)-algebra} on \(X\).
		\item \(\mu = \at{\mu^{\ast} }{\mathcal{A} }{}\) is a \hyperref[def:measure]{measure} on \((X, \mathcal{A})\).
		\item \((X, \mathcal{A} , \mu)\) is a \hyperref[def:complete-measure-space]{complete measure space}.
	\end{enumerate}
\end{theorem}

\begin{proof}\let\qed\relax
	We divide the proof in several steps.
	\begin{enumerate}[(a)]
		\item We show \(\mathcal{A}\) is a \hyperref[def:sigma-algebra]{\(\sigma\)-algebra} by showing
		      \begin{itemize}
			      \item We first show \(\varnothing \in \mathcal{A} \).
			            \begin{claim}
				            \(\varnothing \in \mathcal{A} \).
			            \end{claim}
			            \begin{explanation}
				            To show this, we simply check that \(\varnothing \) is \hyperref[def:C-measurable]{C-measurable}. We see that
				            \[
					            \underset{E\subset X}{\forall}\ \mu^{\ast} (E) = \mu^{\ast} (E\cap \varnothing ) + \mu^{\ast} (E \setminus \varnothing ) = \mu^{\ast} (E),
				            \]
				            which just shows \(\varnothing \in \mathcal{A}\).
			            \end{explanation}
			      \item Then we show \(\mathcal{A} \) is closed under complements.
			            \begin{claim}
				            \(\mathcal{A} \) closed under complements.
			            \end{claim}
			            \begin{explanation}
				            This is equivalent to say that if \(A\) is \hyperref[def:C-measurable]{C-measurable}, so is \(A^{c}\).
				            We see that if \(A\) is \hyperref[def:C-measurable]{C-measurable}, then for every \(E\subset X\),
				            \[
					            \mu^{\ast} (E) = \mu^{\ast} (E\cap A) + \mu^{\ast} (E\setminus A).
				            \]
				            Observing that \(E\cap A = E\setminus A^{c} \) and \(E\setminus A = E\cap A^{c} \), hence
				            \[
					            \mu^{\ast} (E) = \mu^{\ast} (E\setminus A^{c} ) + \mu^{\ast} (E\cap A^{c} ).
				            \]

				            We immediately see that above implies \(A^{c} \in\mathcal{A} \).
			            \end{explanation}
			      \item We now show \(\mathcal{A} \) is closed under countable unions.
			            \begin{note}
				            To show \(\mathcal{A} \) closed under countable unions, we show that \(\mathcal{A}\) is closed under:
				            \begin{center}
					            finite unions \(\overset{\text{then}}{\implies}\) countable \underline{disjoint} unions \(\overset{\text{then}}{\implies}\) countable unions.
				            \end{center}
			            \end{note}
			            Hence, we first show \(\mathcal{A} \) is closed under finite unions.
			            \begin{claim}\label{pf:Caratheodory-extension-Thm-1-finite-unions}
				            \(A, B\in \mathcal{A} \implies A\cup B\in \mathcal{A}\).
			            \end{claim}
			            \begin{explanation}
				            Fix \(E\subset X\) arbitrary. We need to show that
				            \[
					            \mu^{\ast} (E) = \mu^{\ast} (E\cap (A\cup B)) + \mu^{\ast} (E\setminus (A\cup B)),
				            \]
				            i.e.,
				            \(\mu^{\ast} (1\cup 2\cup 3\cup 4) = \mu^{\ast} (1\cup 2\cup 3) + \mu^{\ast} (4)\) given \(A, B\in\mathcal{A}\) and the following figure.
				            \begin{figure}[H]
					            \centering
					            \incfig{thm:Caratheodory-extension-Thm-1a}
					            \label{fig:thm:Caratheodory-extension-Thm-1a}
				            \end{figure}

				            \begin{itemize}
					            \item Since \(A\) is \hyperref[def:C-measurable]{C-measurable},
					                  \begin{itemize}
						                  \item \(\mu^{\ast} (1\cup 2\cup 3\cup 4) = \mu^{\ast} (1\cup 2) + \mu^{\ast} (3\cup 4)\)
						                  \item \(\mu^{\ast} (1\cup 2\cup 3) = \mu^{\ast} (1\cup 2) + \mu^{\ast} (3)\)
					                  \end{itemize}
					            \item Since \(B\) is \hyperref[def:C-measurable]{C-measurable},
					                  \begin{itemize}
						                  \item \(\mu^{\ast} (3\cup 4) = \mu^{\ast} (3) + \mu^{\ast} (4)\)
					                  \end{itemize}
				            \end{itemize}
				            Hence, we have
				            \[
					            \begin{split}
						            \mu^{\ast} (1\cup 2\cup 3\cup 4) & = \mu^{\ast} (1\cup 2)+\mu^{\ast} (3\cup 4)              \\
						                                             & = \mu^{\ast} (1\cup 2) + \mu^{\ast} (3) + \mu^{\ast} (4)
						            = \mu^{\ast}(1\cup 2\cup 3) + \mu^{\ast} (4).
					            \end{split}
				            \]
			            \end{explanation}
			            We now show \(\mathcal{A} \) is closed under countable \underline{disjoint} unions.
			            \begin{claim}
				            \(\mathcal{A} \) is closed under countable disjoint unions.
			            \end{claim}
			            \begin{explanation}
				            Let \(A_1, A_2, \dots \in\mathcal{A}\) and disjoint.
				            Fix \(E\subset X\) arbitrary. Since \(\mu^{\ast} \) is countably subadditive,
				            \[
					            \mu^{\ast} (E) \leq  \mu^{\ast} \left(E\cap \bigcup_{i=1}^{\infty} A_{i}\right) + \mu^{\ast} \left(E\setminus \bigcup_{i=1}^{\infty} A_{i}\right),
				            \]
				            hence we only need to show another way around.

				            Fix \(N\in\mathbb{N} \), we have \(\bigcup_{n=1}^{N} A_{n}\in \mathcal{A} \) since \hyperref[pf:Caratheodory-extension-Thm-1-finite-unions]{\(N\) is finite}, and
				            \[
					            \begin{split}
						            \mu^{\ast} (E) & = \mu^{\ast} \left(E\cap \left(\bigcup_{n=1}^{N} A_{n}\right)\right) + \mu^{\ast} \left(E\setminus \left(\bigcup_{n=1}^{N} A_{n}\right)\right)                                                                                                                                                                                                                      \\
						                           & \geq \underbrace{\sum_{n=1}^{N} \mu^{\ast} (E\cap A_{n})}_{ \overset{\hyperref[lma:disjoint-C-measurable-finite-additive]{!}}{=} \mu^{\ast} \left(E\cap \left(\bigcup_{n=1}^{N} A_{n}\right)\right)} + \underbrace{\mu^{\ast} \left(E\setminus \bigcup_{n=1}^{\infty} A_{n}\right)}_{\leq \mu^{\ast} \left(E\setminus \left(\bigcup_{n=1}^{N} A_{n}\right)\right)}.
					            \end{split}
				            \]
				            Now, take \(N\to \infty \) then we are done.
			            \end{explanation}
			            We can then extend this to the case of countable unions.
			            \begin{exercise}
				            Show \(\mathcal{A} \) is closed under countable unions.
			            \end{exercise}
			            \begin{answer}
				            \todo{DIY}
			            \end{answer}
		      \end{itemize}
		      Above shows that \(\mathcal{A} \) is a \hyperref[def:sigma-algebra]{\(\sigma\)-algebra}.
	\end{enumerate}
	\emph{The proof will be \hyperref[pf:Caratheodory-extension-Thm-cont]{continued}\dots}
\end{proof}