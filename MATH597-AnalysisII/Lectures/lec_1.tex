\lecture{1}{05 Jan. 11:00}{\(\sigma\)-algebra}
\section{Measure}\label{sec:Measure}
\begin{eg}
	Before we start, we first see some examples.
	\begin{enumerate}
		\item Let \(X = \{a, b, c\}\). Then
		      \[
			      \mathcal{P}(X)\coloneqq \left\{\varnothing , \{a\}, \{b\}, \{c\}, \{a, b\}, \{a, c\}, \{b, c\}, \{a, b, c\}\right\},
		      \]
		      which is the \emph{power set} of \(X\). We see that
		      \[
			      \# X = n \implies \# \mathcal{P}(X) = 2^n
		      \]
		      for \(n< \infty\).
		\item If \(n = \infty\), say \(X = \mathbb{\MakeUppercase{N}} \), then
		      \[
			      \mathcal{P} (\mathbb{\MakeUppercase{N}} )
		      \]
		      is an uncountable set while \(\mathbb{\MakeUppercase{N}}\) is a countable set. We can see this as follows. Consider
		      \[
			      \phi\colon \mathcal{P}(\mathbb{\MakeUppercase{N}}) \to \left[0, 1\right],\quad A\mapsto 0.a_1 a_2 a_3 \ldots (\text{base }2),
		      \]
		      where
		      \[
			      a_i = \begin{dcases}
				      1, & \text{ if } i\in A     \\
				      0, & \text{ if } i\notin A,
			      \end{dcases}
		      \]
		      and for example, \(A\) can be
		      \(A = \{2, 3, 6, \ldots  \}\subseteq \mathbb{\MakeUppercase{N}}\). Note that \(\phi\) is surjective, hence we have
		      \[
			      \# \mathcal{P}(\mathbb{\MakeUppercase{N}} ) \geq \#\left[0, 1\right].
		      \]
		      But since \(\left[0, 1\right]\) is uncountable, so is \(\mathcal{P} (\mathbb{\MakeUppercase{N}} )\).
	\end{enumerate}
\end{eg}

We like to \emph{measure} the \emph{size} of subsets of \(X\). Hence, we are intriguing to define a map \(\mu\) such that
\[
	\mu\colon \mathcal{P}(X) \to \left[0, \infty\right].
\]
\begin{eg}
	We first see some examples.
	\begin{enumerate}
		\item Let \(X = \{0, 1, 2\}\). Then we want to define \(\mu\colon \mathcal{P}(X)\to \left[0, \infty\right] \), we can have
		      \begin{itemize}
			      \item \(\mu(A) = \# A\). Then we have
			            \begin{itemize}
				            \item \(\mu(\{0, 1\}) = 2\)
				            \item \(\mu(\{0\}) = 1\)
			            \end{itemize}
			      \item \(\mu(A) = \sum\limits_{i\in A} 2^i\). Then we have
			            \begin{itemize}
				            \item \(\mu(\{0, 1\}) = 2^0 + 2^1 = 3\)
			            \end{itemize}
		      \end{itemize}
		\item Let \(X = \{0\}\cup \mathbb{\MakeUppercase{N}} \). Then we want to define \(\mu\colon \mathcal{P}(\mathbb{\MakeUppercase{N}} )\to \left[0, \infty\right] \), we can have
		      \begin{itemize}
			      \item \(\mu(A) = \# A\). Then we have
			            \begin{itemize}
				            \item \(\mu(\{2, 3, 4, 5, \ldots  \}) = \infty = \mu(\{\text{even numbers} \})\)
			            \end{itemize}
			      \item \(\mu(A) = e^{-1}\sum\limits_{i\in A}\frac{1}{i!}\). Then we have
			            \begin{itemize}
				            \item \(\mu(\{0, 2, 4, 6, \ldots \}) = e^{-1} \left(1+\frac{1}{2!}+\frac{1}{3!}+\ldots  \right)\)
			            \end{itemize}
			      \item \(\mu(A) = \sum\limits_{i\in A} a_{i}\)
		      \end{itemize}
		\item Let \(X = \mathbb{\MakeUppercase{R}} \). Then we want to define \(\mu\colon \mathcal{P}(\mathbb{\MakeUppercase{R}} )\to \left[0, \infty\right] \), we can have
		      \begin{itemize}
			      \item \(\mu(A) = \# A\)
			      \item \(\mu(\left(a, b\right)) = b-a\).
			            \begin{problem}
			            Can we extend this map to all of \(\mathcal{P}(\mathbb{\MakeUppercase{R}} )\)?
			            \begin{answer}
				            No!
			            \end{answer}
			            \end{problem}
			      \item \(\mu(\left(a, b\right)) = e^b - e^a\).
			            \begin{problem}
			            Can we extend this map to all of \(\mathcal{P}(\mathbb{\MakeUppercase{R}} )\)?
			            \begin{answer}
				            No!
			            \end{answer}
			            \end{problem}
		      \end{itemize}
	\end{enumerate}
\end{eg}

We immediately see the problems. To extend our native measure method into \(\mathbb{\MakeUppercase{R}} \) is hard and will cause something counter-intuitive!\footnote{\url{https://en.wikipedia.org/wiki/Banach-Tarski_paradox}}
Hence, rather than define measurement on \emph{all} subsets in the power set of \(X\), we only focus on \emph{some} subsets. In other words, we
want to define
\[
	\mu\colon \mathcal{P}(\mathbb{\MakeUppercase{R}} )\supset\mathcal{A} \to \left[0, \infty\right].
\]

\subsection{\(\sigma\)-algebras}
We start from the definition of the most fundamental element in measure theory.
\begin{definition}[\(\sigma\)-algebra]\label{def:sigma-algebra}
	Let \(X\) be a set. A collection \(\mathcal{A} \) of subsets of \(X\), i.e., \(\mathcal{A}\subset \mathcal{P} (X) \) is called a \emph{\(\sigma\)-algebra on \(X\)} if
	\begin{itemize}
		\item \(\varnothing \in \mathcal{A} \).
		\item \(\mathcal{A} \) is closed under complements. i.e., if \(A\in \mathcal{A} \), \(A^c = X\setminus A\in \mathcal{A} \).
		\item \(\mathcal{A} \) is closed under countable unions. i.e., if \(A_i\in \mathcal{A} \), then \(\bigcup\limits_{i=1}^{\infty} A_{i}\in \mathcal{A} \).
	\end{itemize}
\end{definition}

\begin{remark}
	There are some easy properties we can immediately derive.
	\begin{itemize}
		\item \(X\in \mathcal{A} \) from \(X = X\setminus \underbrace{\varnothing}_{\in \mathcal{A}} \) and \(\mathcal{A}\) is closed under complement.
		\item \(\bigcap\limits_{i=1}^{\infty} A_{i} = \left(\bigcup\limits_{i=1}^{\infty} A_{i}^{c} \right)^c\), namely \(\mathcal{A} \) is \underline{closed under countable intersections}.
		\item \(A_1\cup A_2 \cup \ldots \cup A_n = A_1\cup A_2 \cup \ldots \cup A_n \cup \varnothing \cup \varnothing \cup\ldots\), hence \(\mathcal{A} \) is closed under finite unions and intersections.
	\end{itemize}
\end{remark}

An immediate definition can be given. We now define so-called \emph{Borel set}.
\begin{definition}[Borel set]\label{def:Borel-set}
	Given a topological space \(X\), a \emph{Borel set} is any set in \(X\) that can be formed from open sets through the operations of countable union, countable intersection and relative complement.
\end{definition}