\lecture{36}{8 Apr. 11:00}{Absolutely Continuous Functions}

\subsection{Absolutely Continuous Functions}
We start with a definition.
\begin{definition}[Absolutely continuous function]\label{def:absolutely-continuous-function}
	We say that \(F \colon \mathbb{R} \to \mathbb{R}\) is \emph{absolutely continuous}, denoted as \(F \in AC\), if for all \(\epsilon > 0\), there exists a
	\(\delta > 0\) such that if \((a_1,b_1),\ldots,(a_N,b_N)\) are finitely many disjoint open intervals satisfying \(\sum_{n=1}^N (b_n-a_n) < \delta\), then
	\[
		\sum_{n=1}^N \left\vert F(b_n) - F(a_n) \right\vert < \epsilon.
	\]
\end{definition}
\begin{note}
	Do not be confused between \autoref{def:absolutely-continuous-function} and \autoref{def:absolutely-continuous-measure}.
\end{note}
\begin{lemma}
	We have that
	\begin{enumerate}[(1)]
		\item If \(F\) is \hyperref[def:absolutely-continuous-function]{absolutely continuous}, then it is uniformly continuous.
		\item If \(F\) is \hyperref[def:Lipschitz]{Lipschitz}, then \(F\) is \hyperref[def:absolutely-continuous-function]{absolutely continuous}.
		\item \(F(x) = \int_{-\infty}^x f(t) \,\mathrm{d}t\), \(f \in L^1\), is \hyperref[def:absolutely-continuous-function]{absolutely continuous}.
	\end{enumerate}
\end{lemma}
\begin{proof}
	We prove this one by one.
	\begin{enumerate}[(1)]
		\item We simply take \(N = 1\).
		\item This is trivial.
		\item We write this out as
		      \[
			      \sum_{n=1}^N \left\vert F(b_n) - F(a_n) \right\vert = \sum_{n=1}^N \left\vert \int_{a_n}^{b_n} f(t) \,\mathrm{d}t \right\vert
			      \leq \sum_{n=1}^N \int_{a_n}^{b_n} \left\vert f(t) \right\vert \,\mathrm{d}t
			      = \int_E \left\vert f(t) \right\vert \,\mathrm{d}t
		      \]
		      where \(E = \bigcup_{n=1}^N (a_n,b_n)\), so \(m(E) = \sum_{n=1}^N (b_n-a_n)\). From Midterm Q1, if \(f \in L^1(X,\mu)\), for all \(\epsilon > 0\),
		      there is a \(\delta > 0\) such that \(\mu(E) < \delta\) implies \(\int_E \left\vert f \right\vert < \epsilon\).
		      This directly implies that this function is \hyperref[def:absolutely-continuous-function]{absolutely continuous}.
	\end{enumerate}
\end{proof}

\begin{eg}
	The \hyperref[sssec:Cantor-Function]{Cantor function} \(F\) is uniformly continuous.
	However, we will see that it is not \hyperref[def:absolutely-continuous-function]{absolutely continuous}.
\end{eg}

\begin{proposition}\label{prop:abs-cts}
	Suppose \(F \in NBV\), then \(F\) is \hyperref[def:absolutely-continuous-function]{absolutely continuous} if and only if \(\mu_F \ll m\).
\end{proposition}
\begin{proof}
	We prove two directions.
	\paragraph{\((\impliedby)\)}
	Suppose \(\mu_F \ll m\). Then \(F(x) = \int_{-\infty}^x F^\prime (t) \,\mathrm{d}t\), and \(F^\prime \in L^1(\mathbb{R},m)\), by \autoref{prop:nbv-ftoc}. Therefore, \(F \in AC\).

	\paragraph{\((\implies)\)}
	Now suppose \(F \in AC\). Note that since \(F\) is continuous,
	\[
		\mu_F((a,b)) = \lim_{n \to \infty} \mu_F((a,b-1/n]) = \lim_{n \to \infty} F(b-1/n) - F(a) = F(b) - F(a).
	\]
	We let \(E\) be a \hyperref[def:Borel-set]{Borel set} with \(m(E) = 0\). Fix \(\epsilon > 0\), we will show \(\left\vert \mu_F(E) \right\vert \leq \epsilon \).
	Let \(\delta > 0\) be the constant from \(F \in AC\).

	We know that there exists open \(U_1 \supseteq U_2 \supseteq \cdots \supseteq E\) such that \(\lim\limits_{n \to \infty} m(U_n) = m(E) = 0\), and open
	\(V_1 \supseteq V_2 \supseteq \cdots \supseteq E\) such that \(\lim\limits_{n \to \infty} \mu_F(V_n) = \mu_F(E)\) by regularity.

	Let \(O_n = U_n \cap V_n\), then \(O_1 \supseteq O_2 \supseteq \cdots \supseteq E\), and by monotonicity (for \(\mu_F\) decomposing into pos/neg first)
	\[
		\lim_{n \to \infty} m(O_n) = m(E) = 0,\quad \lim_{n \to \infty} \mu_F(O_n) = \mu_F(E).
	\]
	Thus without loss of generality, we may assume \(m(O_1) < \delta\). Each \(O_n\) is a countable union of disjoint intervals
	\[
		O_n = \bigcup_{k=1}^\infty \left( a_k^n, b_k^n \right).
	\]

	For any \(N\) we also have
	\[
		\sum_{k=1}^N (b_k^n - a_k^n) \leq m(O_n) \leq m(O_1) < \delta.
	\]
	Therefore,
	\[
		\left\vert \mu_F\left( \bigcup_{k=1}^N (a_k^n,b_k^n) \right) \right\vert
		= \left\vert \sum_{k=1}^N \mu_F((a_k^n,b_k^n)) \right\vert
		\leq \sum_{k=1}^N \left\vert \mu_F((a_k^n,b_k^n)) \right\vert
		\leq \sum_{k=1}^N \left\vert F(b_k^n) - F(a_k^n) \right\vert < \epsilon,
	\]
	which implies
	\[
		\left\vert \mu_F(O_n) \right\vert = \lim_{N \to \infty} \left\vert \mu_F\left( \bigcup_{k=1}^N (a_k^n,b_k^n) \right) \right\vert \leq \epsilon \\
	\]
	and
	\[
		\left\vert \mu_F(E) \right\vert = \lim_{n \to \infty} \left\vert \mu_F(O_n) \right\vert \leq \epsilon.
	\]
	By taking \(\epsilon \to 0\) we have \(\mu_F(E) = 0\), hence \(\mu_F \ll m\).
\end{proof}

\begin{corollary}
	\(F \in NBV \cap AC\) if and only if \(F(x) = \int_{-\infty}^x f(t) \,\mathrm{d}t\) for some \(f \in L^1(\mathbb{R},m)\). If this holds, we have \(f = F^\prime \)
	\hyperref[def:Lebesgue-measure]{Lebesgue} \hyperref[def:mu-almost-everywhere]{almost everywhere}.
\end{corollary}

\begin{lemma}
	If \(F \in AC([a,b])\), then \(F \in NBV([a,b])\).
\end{lemma}

\begin{proof}
	\todo{DIY}
\end{proof}

\begin{theorem}[Fundamental Theorem of Calculus]\label{thm:FTC}
	For \(F \in [a,b] \to \mathbb{R}\), the following are equivalent.
	\begin{enumerate}[(1)]
		\item \(F \in AC([a,b])\).
		\item \(F(x) - F(a) = \int_a^x f(t) \,\mathrm{d}t\) for some \(f \in L^1([a,b],m)\).
		\item \(F\) is differentiable \hyperref[def:mu-almost-everywhere]{almost everywhere} on \([a,b]\) and \(F(x) - F(a) = \int_a^b F^\prime (t) \,\mathrm{d}t\).
	\end{enumerate}
\end{theorem}
\begin{proof}
	This follows directly from \autoref{}.
\end{proof}


\begin{definition*}
	Let \(\mu\) be a \hyperref[def:finite-signed-measure]{finite signed \hyperref[def:Borel-measure]{Borel measure}} on \(\mathbb{R}\).
	\begin{definition}[Discrete measure]\label{def:discrete-measure}
		\(\mu\) is called a \emph{discrete measure} if there is a countable set \(\{x_n\}\) and \(c_n \neq 0\) such that \(\sum_{n=1}^\infty \left\vert c_n \right\vert < \infty\)
		and \(\mu = \sum_n c_n \delta_{x_n}\).
		\begin{note}
			\(\delta_{x_n}\) is the \hyperref[eg:Dirac-Delta measure]{Dirac delta measure} at \(x_n\).
		\end{note}
	\end{definition}
	\begin{definition}[Continuous measure]\label{def:continuous-measure}
		\(\mu\) is called a \emph{continuous measure} if \(\mu(\{a\}) = 0\) for all \(a \in \mathbb{R}\).
	\end{definition}
\end{definition*}


\begin{lemma}
	Given a \hyperref[def:finite-signed-measure]{finite signed \hyperref[def:Borel-measure]{Borel measure}} \(\mu\),
	\begin{enumerate}[(1)]
		\item Any \(\mu = \mu_d + \mu_c\), where \(\mu_d\) is \hyperref[def:discrete-measure]{discrete}, \(\mu_c\) is \hyperref[def:continuous-measure]{continuous}
		      are uniquely determined.
		\item \(\mu\) \hyperref[def:discrete-measure]{discrete} implies \(\mu \perp m\).
		\item \(\mu \ll m\) implies \(\mu\) is \hyperref[def:continuous-measure]{continuous}.
	\end{enumerate}
\end{lemma}

\begin{corollary}
	For \(\mu \) a \hyperref[def:finite-signed-measure]{finite signed \hyperref[def:Borel-measure]{Borel measure}} on \(\mathbb{R}\), we have that
	\[
		\mu = \mu_d + \mu_{ac} + \mu_{sc}
	\]
	where \(\mu_d\) is \hyperref[def:discrete-measure]{discrete}, \(\mu_{ac}\) is \hyperref[def:absolutely-continuous-function]{absolutely continuous},
	and \(\mu_{sc}\) is \hyperref[def:continuous-measure]{continuous} and \hyperref[def:singular]{singularly} to \(m\).
\end{corollary}
