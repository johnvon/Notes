\lecture{9}{26 Jan.\ 11:00}{Properties of Lebesgue-Stieltjes measure}
\begin{prev}
	Let \(X\subset [0, \infty ]\). Recall that
	\begin{itemize}
		\item Finite supremum.
		      \[
			      \alpha = \sup X < \infty \iff \begin{dcases}
				      \underset{x\in X}{\forall }\ \alpha \geq x \\
				      \underset{\epsilon >0}{\forall }\ \underset{x\in X}{\exists } \ x+\epsilon \geq \alpha.
			      \end{dcases}
		      \]
		\item Infinite supremum.
		      \[
			      \alpha  = \sup X = \infty \iff \underset{L>0}{\forall }\ \underset{x\in X}{\exists }\ x\geq L.
		      \]
	\end{itemize}
	This should be useful latter on.
\end{prev}

\begin{theorem}[Regularity]\label{thm:regularity}
	Let \(\mu\) be \hyperref[def:Lebesgue-Stieltjes-measure]{Lebesgue-Stieltjes measure}. Then, for all \(A\in \mathcal{A} _\mu \),
	\begin{enumerate}[(1)]
		\item\label{thm:outer-regularity} (outer regularity) \(\mu (A) = \inf \{\mu (O) \mid O\supset A, O\text{ is open}\}\)
		\item\label{thm:inner-regularity} (inner regularity) \(\mu (A) = \sup \{\mu (K) \mid K\subset A, K\text{ is compact}\}\)
	\end{enumerate}
\end{theorem}
\begin{proof}
	We check them separately.
	\begin{enumerate}[(1)]
		\item \todo{DIY}
		\item Let \(s \coloneqq \sup \{\mu (K) \mid K\subset A, K\text{ is compact} \}\), then by \hyperref[thm:measure-space-monotonicity]{monotonicity}, we have \(\mu (A)\geq s\).
		      To show the other direction, we consider
		      \begin{claim}
			      \hyperref[thm:inner-regularity]{Inner regularity} holds if \(A\) is a bounded set.
		      \end{claim}
		      \begin{explanation}
			      Then \(\overline{A} \in \mathcal{B} (\mathbb{R} )\subset \mathcal{A} _\mu \),
			      \(\overline{A} \) is also bounded \(\implies \mu (\overline{A} ) < \infty \). Fix \(\epsilon >0\), then by \hyperref[thm:outer-regularity]{outer regularity},
			      there exists an open \(O\supset \overline{A} \setminus A\) , and \(\mu (O) - \mu (\overline{A} \setminus A) = \mu (O\setminus (\overline{A} \setminus A))\leq \epsilon \).
			      Let \(K\coloneqq \underbrace{A\setminus O}_{K\subset A} = \underbrace{\overline{A} \setminus O}_{\text{compact}}\), we show that
			      \[
				      \mu (K)\geq \mu (A) - \epsilon .
			      \]\todo{DIY}
		      \end{explanation}
		      \begin{claim}
			      \hyperref[thm:inner-regularity]{Inner regularity} holds if \(A\) is an unbounded set with \(\mu (A)<\infty \).
		      \end{claim}
		      \begin{explanation}
			      Let \(A = \bigcup_{n=1}^{\infty} A_{n}\), \(A_{n} = A\cap [-n, n]\) where
			      \(A_1\subset A_2\subset \dots  \), then
			      \[
				      \lim_{n \to \infty} \mu (A_{n}) = \mu (A) < \infty .
			      \]
		      \end{explanation}
		      \begin{claim}
			      \hyperref[thm:inner-regularity]{Inner regularity} holds if \(A\) is an unbounded set with \(\mu (A) = \infty \).
		      \end{claim}
		      \begin{explanation}
			      We can show that
			      \[
				      \lim_{n \to \infty} \mu (A_{n}) = \mu (A) = \infty.
			      \]
			      Fix \(L>0\), then \(\exists N\) such that \(\mu (A_{N})\geq L\).
		      \end{explanation}
	\end{enumerate}
\end{proof}

\begin{definition*}
	Let \(X\) be a topological space. Then
	\begin{definition}[\(G_\delta \)-set]\label{def:G-delta-set}
		A \emph{\(G_{\delta}\)-set} is \(G = \bigcap_{i=1}^{\infty} O_{i}\), \(O_{i}\) open.
	\end{definition}

	\begin{definition}[\(F_\sigma \)-set]\label{def:F-sigma-set}
		A \emph{\(F_{\sigma}\)-set} is \(F = \bigcup_{i=1}^{\infty} F_{i}\), \(F_{i}\) closed.
	\end{definition}
\end{definition*}

\begin{theorem}
	Let \(\mu \) be a \hyperref[def:Lebesgue-Stieltjes-measure]{Lebesgue-Stieltjes measure}. Then \emph{TFAE}\footnote{\emph{TFAE}: The following are equivalent.}:
	\begin{enumerate}[(1)]
		\item \(A\in \mathcal{A} _\mu \)
		\item \(A = G\setminus M\), \(G\) is a \hyperref[def:G-delta-set]{\(G_{\delta}\)-set}, \(M\) is a \hyperref[def:mu-null-set]{\(\mu\)-null set}.
		\item \(A = F\setminus N\), \(F\) is a \hyperref[def:F-sigma-set]{\(F_{\sigma}\)-set}, \(N\) is a \hyperref[def:mu-null-set]{\(\mu\)-null set}.
	\end{enumerate}
\end{theorem}
\begin{proof}
	We see that \((2) \implies (1)\) and \((3)\implies (1)\) are clear.
	\begin{claim}
		\((1)\implies (3)\).
	\end{claim}
	\begin{explanation}
		We consider two cases.
		\begin{itemize}
			\item Assume \(\mu (A)<\infty \). From the \hyperref[thm:inner-regularity]{inner regularity}, we have
			      \[
				      \forall n\in\mathbb{N} \ \exists \text{compact }K_{n}\subset A\text{ such that } \mu (K_{n}) + \frac{1}{n} \geq \mu (A).
			      \]
			      Let \(F = \bigcup_{n=1}^{\infty} K_{n}\), then \(N = A\setminus F\) is \hyperref[def:mu-null-set]{\(\mu\)-null}.\todo{Check!}
			\item Assume \(\mu (A) = \infty \). Let \(A = \bigcup_{k\in\mathbb{Z} } A_{k}\), \(A_{k} = A\cap (k, k+1]\). From what we
			      have just shown above,
			      \[
				      \forall k\in\mathbb{Z}\ A_{k} = F_{k}\cup N_{k},\ A = \underbrace{\left(\bigcup_{k}F_{k} \right)}_{\hyperref[def:F-sigma-set]{F_{\sigma}\text{-set}}}\cup \underbrace{\left(\bigcup_{k}N_{k}\right)}_{\hyperref[def:mu-null-set]{\mu\text{-null}}}.
			      \]
		\end{itemize}
	\end{explanation}
	\begin{claim}
		\((1)\implies (2)\).
	\end{claim}
	\begin{explanation}
		We see that
		\[
			A^{c} = F\cup N,\quad A = F^{c} \cap N^{c} = F^{c} \setminus N.
		\]
	\end{explanation}
\end{proof}

\begin{proposition}
	Let \(\mu \) be a \hyperref[def:Lebesgue-Stieltjes-measure]{Lebesgue-Stieltjes measure}, and \(A\in \mathcal{A} _\mu \), \(\mu (A)<\infty \). Then we have
	\[
		\forall \epsilon >0\ \exists I = \bigcup_{i=1}^{N(\epsilon )} I_{i}
	\]
	\underline{disjoint open intervals} such that \(\mu (A\triangle  I)\leq \epsilon \).
\end{proposition}
\begin{proof}
	Using \hyperref[thm:outer-regularity]{outer regularity} and the fact that every open set is \(\bigcup_{i=1}^{\infty} I_{i}\), where \(I_{i}\) are disjoint open intervals.
	\todo{DIY}
\end{proof}

We now see some properties of \hyperref[def:Lebesgue-measure]{Lebesgue measure}.

\begin{theorem}
	Let \(A\in \mathcal{L}\), then we have \(A + s\in \mathcal{L}\), \(rA\in\mathcal{L}  \) for all \(r, s\in\mathbb{R} \). i.e.,
	\[
		m(A+s) = m(A),\quad m(rA) = \left\vert r \right\vert \cdot m(A).
	\]
\end{theorem}
\begin{proof}
	\todo{DIY}
\end{proof}

\begin{eg}
	We now see some examples.
	\begin{enumerate}[(1)]
		\item Let \(\mathbb{Q}\eqqcolon\{r_{i}\}_{i=1}^{\infty}\) which is dense in \(\mathbb{R} \). Let \(\epsilon >0\), and
		      \[
			      O = \bigcup_{i=1}^{\infty} \left(r_{i} - \frac{\epsilon}{2^i}, r_{i} + \frac{\epsilon}{2^i}\right).
		      \]
		      We see that \(O\) is open and dense\footnote{\url{https://en.wikipedia.org/wiki/Dense_set}} in \(\mathbb{R} \). But we see
		      \[
			      m(O) \leq \sum_{i=1}^{\infty} \frac{2\epsilon }{2^i} = 2\epsilon.
		      \]
		      Furthermore, \(\partial O = \overline{O} \setminus O\), \(m(\partial O) = \infty \)
		\item There exists uncountable set \(A\) with \(m(A)=0\).
		\item There exists \(A\) with \(m(A)>0\) but \(A\) contains no non-empty open intervals.
		\item There exists \(A\notin \mathcal{L} \). e.g. Vitali set.\footnote{\url{https://en.wikipedia.org/wiki/Vitali_set}}
		\item There exists \(A\in\mathcal{L} \setminus \mathcal{B} (\mathbb{R} )\).
	\end{enumerate}
\end{eg}