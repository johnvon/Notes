\lecture{7}{21 Jan. 11:00}{Borel Measures}
\subsection{Borel Measures on \(\mathbb{R}\)}
We first introduce so-called \emph{distribution function}.

\begin{definition}[Distribution function]\label{def:distribution-function}
	An \underline{increasing}\footnote{Here, increasing means \(F(x)\leq F(y)\) for \(x<y\).} function
	\[
		F\colon \mathbb{\MakeUppercase{R}} \to \mathbb{\MakeUppercase{R}}
	\]
	and \underline{right-continuous}. \(F\) is then a \emph{distribution function}.
\end{definition}

\begin{eg}
	Here are some examples of right-continuous functions.
	\begin{enumerate}
		\item \(F(x) = x\).
		\item \(F(x) = e^x\).
		\item Define
		      \[
			      F(x) = \begin{dcases}
				      1, & \text{ if } x\geq 0 \\
				      0, & \text{ if } x < 0.
			      \end{dcases}
		      \]
		\item Let \(\mathbb{\MakeUppercase{Q}} \coloneqq \{r_1, r_2, \ldots  \}\). Define
		      \[
			      F_n(x) = \begin{dcases}
				      1, & \text{ if } x\geq r_n \\
				      0, & \text{ if } x < r_n,
			      \end{dcases}
		      \]
		      and
		      \[
			      F(x) \coloneqq \sum\limits_{n=1}^{\infty} \frac{F_n(x)}{2^n}.
		      \]

		      Then \(F\) is a distribution function (hence right-continuous).
	\end{enumerate}
\end{eg}
\begin{note}
	If \(F\) is increasing, and
	\[
		F(\infty )\coloneqq \lim\limits_{x \nearrow \infty} F(x),\quad F(-\infty ) \coloneqq \lim\limits_{x \searrow \infty} F(x)
	\]
	exist in \([-\infty , \infty ]\).

	In probability theory, cumulative distribution function (CDF) is a distribution function with \(F(\infty ) = 1\), \(F(-\infty ) = 0\).\footnote{There are \underline{distributions} \cite{folland1999real} Ch9., but these are different from distribution functions.}
\end{note}

\hr

Now, we can define a \emph{Borel measure} on \((X, \mathcal{\MakeUppercase{B}} (\mathbb{\MakeUppercase{r}} ))\).
\begin{definition}[Borel mesaure]\label{def:Borel-measure}
	A \emph{Borel measure} is any \hyperref[def:measure]{measure} \(\mu\) defined on the \hyperref[def:sigma-algebra]{\(\sigma\)-algebra} of \hyperref[def:Borel-set]{Borel sets}.
\end{definition}

\begin{definition}[Locally finite]\label{def:locally-finite}
	Let \(X\) be a Hausdorff topological space, \(\mu\) on \((X, \mathcal{B} (X))\) is called \emph{locally finite} if \(\mu (K)<\infty \)
	for every compact set \(K\subset X\).
\end{definition}

\begin{note}
	Some authors will require a \hyperref[def:Borel-measure]{Borel measure} equipped with the \hyperref[def:locally-finite]{locally finite} property.
	But formally, this is not so common.
\end{note}

\begin{lemma}
	Let \(\mu \) be a \hyperref[def:locally-finite]{locally finite} \hyperref[def:Borel-measure]{Borel measure} on
	\(\mathbb{\MakeUppercase{R}} \), then
	\[
		F_{\mu }(x) = \begin{dcases}
			\mu \left((0, x]\right),  & \text{ if } x > 0 \\
			0,                        & \text{ if } x = 0 \\
			-\mu \left((x, 0]\right), & \text{ if } x < 0
		\end{dcases}
	\]
	is a \hyperref[def:distribution-function]{distribution function}.
\end{lemma}
\begin{proof}
	To show \(F_\mu\) is increasing, consider \(x<y\) such that
	\[
		F_\mu (x) \leq F_\mu (y)
	\]
	by considering
	\begin{itemize}
		\item \(x>0\): Then \(F_\mu (x) = \mu ((0, x])\) and
		      \[
			      F_\mu (y) = \mu ((0, y]) = \mu ((0, x]\cup (x, y]) \geq \mu ((0, x]) = F_\mu (x).
		      \]
		\item \(x=0\): Then \(F_\mu (x) = 0\) and
		      \[
			      F_\mu (y) = \mu ((0, y])\geq 0 = F_\mu (0)
		      \]
		      since \(y>0\).
		\item \(x<0\): Follows the same argument with \(x>0\).
	\end{itemize}

	\par Now, we need to show \(F_\mu \) is right-continuous. Firstly, assume that \(x \geq 0\), then we see that
	\[
		F_\mu(x) = \mu ((0, x]) = \mu ((0, x^+])
	\]
	from the fact that a measure is right-continuous.\footnote{Actually, a measure is always continuous.} Now, if \(x\leq 0\),
	the same argument follows since multiplying \(-1\) will not change the fact that a measure is continuous.
\end{proof}

\begin{definition}[Half intervals]\label{def:half-intervals}
	We call
	\[
		\varnothing , (a, b], (a, \infty ), (-\infty , b], (-\infty , \infty )
	\]
	\emph{half-intervals}.
\end{definition}

\begin{lemma}
	Let \(\mathcal{H} \) be the collection of \underline{finite disjoint} unions of \hyperref[def:half-intervals]{half-intervals}. Then, \(\mathcal{H} \)
	is an \hyperref[def:algebra]{algebra} on \(\mathbb{\MakeUppercase{R}} \).
\end{lemma}
\begin{proof}
	We see that
	\begin{itemize}
		\item \(\varnothing \in \mathcal{H} \). Clearly.
		\item To show \(\mathcal{H} \) is closed under complements, we have
		      \begin{itemize}
			      \item \(\varnothing ^{c} = \mathbb{\MakeUppercase{r}} = (-\infty , \infty )\in \mathcal{H}\).
			      \item \((a, b]^{c} = (-\infty , a] \cup (a, \infty )\in\mathcal{H} \).\footnote{Since it's a two disjoint union of half intervals.}
			      \item \((a, \infty )^{c} = (-\infty , a]\in\mathcal{H}\).
			      \item \((-\infty , b]^{c} = (b, \infty )\in\mathcal{H} \).
			      \item \((-\infty , \infty )^{c} = \varnothing \in\mathcal{H}\).
		      \end{itemize}
		\item \(\mathcal{H} \) is closed under finite unions, clearly.
	\end{itemize}
\end{proof}

\begin{proposition}[Distribution function defines a pre-measure]
	Let \(F\colon \mathbb{\MakeUppercase{R}}\to \mathbb{\MakeUppercase{R}}  \) be a \hyperref[def:distribution-function]{distribution function}. For a
	\hyperref[def:half-intervals]{half interval} \(I\), define
	\[
		\ell(I) \coloneqq \ell_F(I) = \begin{dcases}
			0,                        & \text{ if } I = \varnothing;         \\
			F(b) - F(a),              & \text{ if } I = (a, b];              \\
			F(\infty ) - F(a),        & \text{ if } I = (a, \infty];         \\
			F(b) - F(-\infty ),       & \text{ if } I = (-\infty , b];       \\
			F(\infty ) - F(-\infty ), & \text{ if } I = (-\infty , \infty ).
		\end{dcases}
	\]
	Define \(\mu _0 \coloneqq \mu _{0, F}\) as
	\[
		\mu _{0, F}\colon \mathcal{H} \to [0, \infty ]
	\]
	by
	\[
		\mu _0(A) = \sum\limits_{k=1}^{N} \ell (I_{k}) \text{ if }A = \bigcup\limits_{k=1}^{N} I_{k},
	\]
	where \(A\) is a finite \underline{disjoint} union of \hyperref[def:half-intervals]{half intervals} \(I_1, \ldots , I_{N} \).
	Then, \(\mu _0\) is a \hyperref[def:pre-measure]{pre-measure} on \(\mathcal{H}\).
\end{proposition}
\begin{proof}
	We see that
	\begin{enumerate}
		\item \(\mu _0\) is well-defined.
		\item \(\mu _0(\varnothing ) = 0\).
		\item \(\mu _0\) is finite additive.
		\item \(\mu _0\) is countable additive within \(\mathcal{H}\).
		      \par Suppose \(A\in \mathcal{H} \) where \(A = \bigcup\limits_{i=1}^{\infty} A_{i}\) is a countable
		      disjoint union. It is enough to consider the case that \(A = I\), \(A_{k} = I_{k}\) are all
		      half-intervals.\footnote{Since \(\mathcal{\MakeUppercase{H}}\) is only a collection of \emph{finite} disjoint \hyperref[def:half-intervals]{half intervals}, hence
			      after considering \(A = I\), we can apply the same argument iteratively and stop in finite steps. Formally, we can consider \(H\in \mathcal{\MakeUppercase{H}} \),
			      \(H = \bigcup\limits_{i=1}^{\infty} A^{i}\), where \(A^i\) being a \hyperref[def:half-intervals]{half interval}. Then by the above argument, we have \(A^i = I^i\) and so on.}

		      Focus on the case \(I = (a, b]\). Let
		      \[
			      (a, b] = \bigcup\limits_{n=1}^{\infty} (a_{n}, b_{n}],
		      \]
		      which is a disjoint union. Then we only need to check
		      \[
			      F(b) - F(a) = \sum\limits_{n=1}^{\infty} \left(F(b_{n}) - F(a_{n})\right).
		      \]
		      \begin{itemize}
			      \item Since \((a, b]\supset \bigcup\limits_{n=1}^{N} (a_{n}, b_{n}]\) for any fixed \(N\in\mathbb{\MakeUppercase{N}} \), hence
			            \[
				            \underset{N\in\mathbb{\MakeUppercase{N}} }{\forall}\ F(b) - F(a) \geq \sum\limits_{n=1}^{N} \left(F(b_{n}) - F(a_{n})\right).
			            \]
			            By letting \(N \to \infty\), we have
			            \[
				            F(b) - F(a) \geq \sum\limits_{n=1}^{\infty} \left(F(b_{n}) - F(a_{n})\right).
			            \]
			      \item Fix \(\epsilon >0\). Since \(F\) is right-continuous, \(\exists a ^\prime > a\) such that
			            \[
				            F(a ^\prime ) - F(a) <\epsilon.
			            \]
			            For each \(n\in\mathbb{\MakeUppercase{N}} \), \(\exists b_{n} ^\prime > b_{n}\) such that
			            \[
				            F(b_{n} ^\prime ) - F(b_{n})<\frac{\epsilon }{2^n}.
			            \]
			            Then, we have
			            \[
				            [a ^\prime , b] \subset \bigcup\limits_{n=1}^{\infty} (a_{n}, b_{n} ^\prime ),
			            \]
			            hence
			            \[
				            \underset{ N\in\mathbb{\MakeUppercase{N}}}{\exists}\ [a ^\prime , b]\subset \bigcup\limits_{n=1}^{N} (a_{n}, b_{n} ^\prime ),\footnotemark
			            \]
			            which is only finitely many unions now. In this case, we have
			            \[
				            F(b) - F(a ^\prime ) \leq \sum\limits_{n=1}^{N} F(b_{n} ^\prime ) - F(a_{n}).
			            \]
			            \footnotetext{This essentially follows from the fact that open sets are closed under countable unions, hence the equality will not hold, even after taking the limit.}

			            Finally, we see that
			            \[
				            \begin{split}
					            F(b) - F(a)&\leq F(b) - F(a ^\prime )+\epsilon \\
					            &\leq \sum\limits_{n=1}^{\infty} \left(F(b_{n} ^\prime ) - F(a_{n})\right) + \epsilon \\
					            &\leq \sum\limits_{n=1}^{\infty} \left(F(b_{n}) - F(a_{n}) + \frac{\epsilon }{2^n}\right) + \epsilon\\
					            &= \sum\limits_{n=1}^{\infty} \left(F(b_{n}) - F(a_{n})\right) + 2\epsilon
				            \end{split}
			            \]
			            for any fixed \(\epsilon > 0\), hence
			            \[
				            F(b) - F(a) \leq \sum\limits_{n=1}^{\infty} (F(b_{n}) - F(a_{n})).
			            \]
		      \end{itemize}
		      Combine these two inequalities, we have
		      \[
			      F(b) - F(a) = \sum\limits_{n=1}^{\infty} (F(b_{n}) - F(a_{n}))
		      \]
		      as we desired.
	\end{enumerate}
\end{proof}
\begin{remark}
	It's again the \(\frac{\epsilon}{2^n}\) trick we saw before!
\end{remark}