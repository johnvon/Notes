\lecture{7}{21 Jan.\ 11:00}{Borel Measures}
\section{Borel Measures on \(\mathbb{R}\)}
\begin{prev}
	Recall that when we say something is \textbf{Borel}, we assume some sort of topological structure on the underlying space implicitly.
	In our context, we're considering the usual topology on \(\mathbb{R}^n \) specifically. We'll focus on one dimensional case
	for now.
\end{prev}

\begin{definition}[Distribution function]\label{def:distribution-function}
	An \underline{increasing}\footnote{Here, increasing means \(F(x)\leq F(y)\) for \(x<y\).} function
	\[
		F\colon \mathbb{R} \to \mathbb{R}
	\]
	and \underline{right-continuous}. \(F\) is then a \emph{distribution function}.
\end{definition}

\begin{eg}
	Here are some examples of right-continuous functions.
	\begin{enumerate}[(1)]
		\item \(F(x) = x\).
		\item \(F(x) = e^x\).
		\item \(F(x) = \begin{dcases}
			      1, & \text{ if } x\geq 0; \\
			      0, & \text{ if } x < 0.
		      \end{dcases}\)
		\item \label{eg:lec7-4} Let \(\mathbb{Q} \coloneqq \{r_1, r_2, \dots  \}\). Define
		      \(
		      F_n(x) = \begin{dcases}
			      1, & \text{ if } x\geq r_n; \\
			      0, & \text{ if } x < r_n,
		      \end{dcases}
		      \) and \( F(x) \coloneqq \sum\limits_{n=1}^{\infty} \frac{F_n(x)}{2^n}\).

		      Then \(F\) is a distribution function (hence right-continuous). This is shown in \autoref{lma:lec7-eg:4-is-distribution-function}.
	\end{enumerate}
\end{eg}
\begin{note}
	If \(F\) is increasing, and
	\[
		F(\infty )\coloneqq \lim\limits_{x \nearrow \infty} F(x),\quad F(-\infty ) \coloneqq \lim\limits_{x \searrow \infty} F(x)
	\]
	exist in \([-\infty , \infty ]\).

	In probability theory, cumulative distribution function (CDF) is a distribution function with \(F(\infty ) = 1\), \(F(-\infty ) = 0\).\footnote{There are \underline{distributions}~\cite{folland1999real} Ch9., but these are different from distribution functions.}
\end{note}

Now, we see a new definition which is essential to our discussion.
\begin{definition}[Borel mesaure]\label{def:Borel-measure}
	A \emph{Borel measure} is any \hyperref[def:measure]{measure} \(\mu\) defined on the \hyperref[def:sigma-algebra]{\(\sigma\)-algebra} of \hyperref[def:Borel-set]{Borel sets}.
\end{definition}

Since we're now considering a topological space, hence it's reasonable to define the following because we have the concept of compact set now.

\begin{definition}[Locally finite]\label{def:locally-finite}
	Let \(X\) be a Hausdorff topological space, \(\mu\) on \((X, \mathcal{B} (X))\) is called \emph{locally finite} if \(\mu (K)<\infty \)
	for every compact set \(K\subset X\).
\end{definition}

\begin{note}
	Some authors will require a \hyperref[def:Borel-measure]{Borel measure} equipped with the \hyperref[def:locally-finite]{locally finite} property.
	But formally, this is not so common.
\end{note}

\begin{lemma}
	Let \(\mu \) be a \hyperref[def:locally-finite]{locally finite} \hyperref[def:Borel-measure]{Borel measure} on
	\(\mathbb{R} \), then
	\[
		F_{\mu }(x) = \begin{dcases}
			\mu \left((0, x]\right),  & \text{ if } x > 0 \\
			0,                        & \text{ if } x = 0 \\
			-\mu \left((x, 0]\right), & \text{ if } x < 0
		\end{dcases}
	\]
	is a \hyperref[def:distribution-function]{distribution function}.
\end{lemma}
\begin{proof}
	We need to show two things.
	\begin{claim}
		\(F_\mu\) is increasing.
	\end{claim}
	\begin{explanation}
		To show \(F_\mu\) is increasing, consider \(x<y\) such that
		\[
			F_\mu (x) \leq F_\mu (y)
		\]
		by considering
		\begin{itemize}
			\item \(x>0\): Then \(F_\mu (x) = \mu ((0, x])\) and
			      \[
				      F_\mu (y) = \mu ((0, y]) = \mu ((0, x]\cup (x, y]) \geq \mu ((0, x]) = F_\mu (x).
			      \]
			\item \(x=0\): Then \(F_\mu (x) = 0\) and
			      \[
				      F_\mu (y) = \mu ((0, y])\geq 0 = F_\mu (0)
			      \]
			      since \(y>0\).
			\item \(x<0\): Follows the same argument with \(x>0\).
		\end{itemize}
	\end{explanation}

	We now show \(F_\mu \) is right-continuous.
	\begin{claim}
		\(F_\mu \) is right-continuous.
	\end{claim}
	\begin{explanation}
		Firstly, assume that \(x \geq 0\), then we see that
		\[
			F_\mu(x) = \mu ((0, x]) = \mu ((0, x^+])
		\]
		from the fact that a measure is right-continuous.\footnote{Actually, a measure is always continuous.} Now, if \(x\leq 0\),
		the same argument follows since multiplying \(-1\) will not change the fact that a \hyperref[def:measure]{measure} is continuous.
	\end{explanation}
\end{proof}

\begin{definition}[Half intervals]\label{def:half-intervals}
	We call \(\varnothing , (a, b], (a, \infty ), (-\infty , b],\) and \((-\infty , \infty )\) \emph{half-intervals}.
\end{definition}

\begin{lemma}
	Let \(\mathcal{H} \) be the collection of \underline{finite disjoint} unions of \hyperref[def:half-intervals]{half-intervals}. Then, \(\mathcal{H} \)
	is an \hyperref[def:algebra]{algebra} on \(\mathbb{R} \).
\end{lemma}
\begin{proof}
	We observe that \(\varnothing \in \mathcal{H} \) and \(\mathcal{H} \) is closed under finite unions is obvious, hence
	we only need to show that \(\mathcal{H} \) is closed under complements.
	\begin{claim}
		\(\mathcal{H} \) is closed under complements.
	\end{claim}
	\begin{explanation}
		We have
		\begin{itemize}
			\item \(\varnothing ^{c} = \mathbb{R} = (-\infty , \infty )\in \mathcal{H}\).
			\item \((a, b]^{c} = (-\infty , a] \cup (a, \infty )\in\mathcal{H} \) since it's a two disjoint union of half intervals.
			\item \((a, \infty )^{c} = (-\infty , a]\in\mathcal{H}\).
			\item \((-\infty , b]^{c} = (b, \infty )\in\mathcal{H} \).
			\item \((-\infty , \infty )^{c} = \varnothing \in\mathcal{H}\).
		\end{itemize}
	\end{explanation}
\end{proof}

\begin{proposition}[Distribution function defines a pre-measure]
	Let \(F\colon \mathbb{R}\to \mathbb{R}  \) be a \hyperref[def:distribution-function]{distribution function}. For a
	\hyperref[def:half-intervals]{half interval} \(I\), define
	\[
		\ell(I) \coloneqq \ell_F(I) = \begin{dcases}
			0,                        & \text{ if } I = \varnothing;         \\
			F(b) - F(a),              & \text{ if } I = (a, b];              \\
			F(\infty ) - F(a),        & \text{ if } I = (a, \infty];         \\
			F(b) - F(-\infty ),       & \text{ if } I = (-\infty , b];       \\
			F(\infty ) - F(-\infty ), & \text{ if } I = (-\infty , \infty ).
		\end{dcases}
	\]
	Define \(\mu _0 \coloneqq \mu _{0, F}\colon \mathcal{H} \to [0, \infty ]\) by
	\[
		\mu _0(A) = \sum\limits_{k=1}^{N} \ell (I_{k}) \text{ if }A = \bigcup\limits_{k=1}^{N} I_{k},
	\]
	where \(A\) is a finite \underline{disjoint} union of \hyperref[def:half-intervals]{half intervals} \(I_1, \dots , I_{N} \).
	Then, \(\mu _0\) is a \hyperref[def:pre-measure]{pre-measure} on \(\mathcal{H}\).
\end{proposition}
\begin{proof}
	Firstly, we note that \(\mu _0\) is well-defined. And also, \(\mu _0\) satisfies \hyperref[def:pre-measure-null-empty-set]{null empty set} and
	\hyperref[def:pre-measure-finite-additivity]{Finite additivity} properties clearly. The only nontrivial part needs a proof is the
	\hyperref[def:pre-measure-countable-additivity-within-the-algebra]{Countable additivity within \(\mathcal{H}\)} properties.
	To show that \hyperref[def:pre-measure-countable-additivity-within-the-algebra]{Countable additivity within \(\mathcal{H}\)} holds, we proceed as follows.

	Suppose \(A\in \mathcal{H} \) where \(A = \bigcup\limits_{i=1}^{\infty} A_{i}\) is a countable
	disjoint union. It is enough to consider the case that \(A = I\), \(A_{k} = I_{k}\) are all
	half-intervals.
	\begin{remark}
		Since \(\mathcal{H}\) is only a collection of \emph{finite} disjoint \hyperref[def:half-intervals]{half intervals}, hence
		after considering \(A = I\), we can apply the same argument iteratively and stop in finite steps. Formally, we can consider \(H\in \mathcal{H} \),
		\(H = \bigcup_{i=1}^{\infty} A^{i}\), where \(A^i\) being a \hyperref[def:half-intervals]{half interval}. Then by the above argument, we have \(A^i = I^i\) and so on.
	\end{remark}

	Focus on the case \(I = (a, b]\). Let \((a, b] = \bigcup_{n=1}^{\infty} (a_{n}, b_{n}]\),
	which is a disjoint union. Then we only need to check \(F(b) - F(a) = \sum\limits_{n=1}^{\infty} \left(F(b_{n}) - F(a_{n})\right)\).
	\begin{claim}
		\(F(b) - F(a) \geq \sum\limits_{n=1}^{\infty} (F(b_{n} ) - F(a_{n} ))\).
	\end{claim}
	\begin{explanation}
		Since \((a, b]\supset \bigcup_{n=1}^{N} (a_{n}, b_{n}]\) for any fixed \(N\in\mathbb{N} \), hence
		\[
			\underset{N\in\mathbb{N} }{\forall}\ F(b) - F(a) \geq \sum\limits_{n=1}^{N} \left(F(b_{n}) - F(a_{n})\right).
		\]
		By letting \(N \to \infty\), we have \(F(b) - F(a) \geq \sum\limits_{n=1}^{\infty} \left(F(b_{n}) - F(a_{n})\right)\).
	\end{explanation}
	\begin{claim}
		\(F(b) - F(a) \leq \sum\limits_{n=1}^{\infty} (F(b_{n} ) - F(a_{n} ))\).
	\end{claim}
	\begin{explanation}
		Fix \(\epsilon >0\). Since \(F\) is right-continuous, \(\exists a ^\prime > a\) such that \(F(a ^\prime ) - F(a) <\epsilon\).
		For each \(n\in\mathbb{N} \), \(\exists b_{n} ^\prime > b_{n}\) such that
		\[
			F(b_{n} ^\prime ) - F(b_{n})<\frac{\epsilon }{2^n}.
		\]
		Then, we have \([a ^\prime , b] \subset \bigcup\limits_{n=1}^{\infty} (a_{n}, b_{n} ^\prime )\), hence
		\(\underset{ N\in\mathbb{N}}{\exists}\ [a ^\prime , b]\subset \bigcup\limits_{n=1}^{N} (a_{n}, b_{n} ^\prime )\),
		which is only finitely many unions now.
		\begin{remark}
			This essentially follows from the fact that open sets are closed under countable unions, hence the equality will not hold, even after taking the limit.
		\end{remark}

		In this case, we have
		\[
			F(b) - F(a ^\prime ) \leq \sum\limits_{n=1}^{N} F(b_{n} ^\prime ) - F(a_{n}).
		\]


		Finally, we see that
		\[
			\begin{split}
				F(b) - F(a)&\leq F(b) - F(a ^\prime )+\epsilon \\
				&\leq \sum\limits_{n=1}^{\infty} \left(F(b_{n} ^\prime ) - F(a_{n})\right) + \epsilon \\
				&\leq \sum\limits_{n=1}^{\infty} \left(F(b_{n}) - F(a_{n}) + \frac{\epsilon }{2^n}\right) + \epsilon
				= \sum\limits_{n=1}^{\infty} \left(F(b_{n}) - F(a_{n})\right) + 2\epsilon
			\end{split}
		\]
		for any fixed \(\epsilon > 0\), hence
		\[
			F(b) - F(a) \leq \sum\limits_{n=1}^{\infty} (F(b_{n}) - F(a_{n})).
		\]
		\begin{remark}
			It's again the \(\epsilon /2^n\) trick we saw before!
		\end{remark}
	\end{explanation}

	Combine these two inequalities, we have
	\[
		F(b) - F(a) = \sum\limits_{n=1}^{\infty} (F(b_{n}) - F(a_{n}))
	\]
	as we desired.
\end{proof}