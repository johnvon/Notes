\lecture{7}{21 Jan. 11:00}{Borel Measures}
\subsection{Borel Measures on \(\mathbb{R}\)}
We first introduce so-called \emph{distribution function}.

\begin{definition}[Distribution function]
	An \underline{increasing}\footnote{Here, increasing means \(F(x)\leq F(y)\) for \(x<y\).} function
	\[
		F\colon \mathbb{\MakeUppercase{R}} \to \mathbb{\MakeUppercase{R}}
	\]
	and \underline{right-continuous}. \(F\) is then a \emph{distribution function}.
\end{definition}

\begin{eg}
	Here are some examples of right-continuous functions.
	\begin{enumerate}
		\item \(F(x) = x\).
		\item \(F(x) = e^x\).
		\item Define
		      \[
			      F(x) = \begin{dcases}
				      1, & \text{ if } x\geq 0 \\
				      0, & \text{ if } x < 0.
			      \end{dcases}
		      \]
		\item Let \(\mathbb{\MakeUppercase{Q}} \coloneqq \{r_1, r_2, \ldots  \}\). Define
		      \[
			      F_n(x) = \begin{dcases}
				      1, & \text{ if } x\geq r_n \\
				      0, & \text{ if } x < r_n,
			      \end{dcases}
		      \]
		      and
		      \[
			      F(x) \coloneqq \sum\limits_{n=1}^{\infty} \frac{F_n(x)}{2^n}.
		      \]

		      Then \(F\) is a distribution function (hence right-continuous).
	\end{enumerate}
\end{eg}
\begin{note}
	If \(F\) is increasing, and
	\[
		F(\infty )\coloneqq \lim\limits_{x \nearrow \infty} F(x),\quad F(-\infty ) \coloneqq \lim\limits_{x \searrow \infty} F(x)
	\]
	exist in \([-\infty , \infty ]\).

	In probability theory, cumulative distribution function (CDF) is a distribution function with \(F(\infty ) = 1\), \(F(-\infty ) = 0\).\footnote{There are \underline{distributions} \cite{folland1999real} Ch9., but these are different from distribution functions.}
\end{note}

\begin{definition}[Locally finite]
	Let \(X\) be a topological space, \(\mu\) on \((X, \mathcal{B} (X))\) is called \emph{locally finite} if \(\mu (K)<\infty \)
	for every compact set \(K\subset X\).
\end{definition}

\begin{lemma}
	Let \(\mu \) be a locally finite Borel measure on \(\mathbb{\MakeUppercase{R}} \), then
	\[
		F_{\mu }(x) = \begin{dcases}
			\mu \left((0, x]\right),  & \text{ if } x > 0 \\
			0,                        & \text{ if } x = 0 \\
			-\mu \left((x, 0]\right), & \text{ if } x < 0
		\end{dcases}
	\]
	is a distribution function.
\end{lemma}
\begin{proof}
	\todo{DIY, use continuity of measure}
\end{proof}

\begin{definition}[Half intervals]
	We call
	\[
		\varnothing , (a, b], (a, \infty ), (-\infty , b], (-\infty , \infty )
	\]
	\emph{half-intervals}.
\end{definition}

\begin{lemma}
	Let \(\mathcal{H} \) be the collection of \underline{finite disjoint} unions of half-intervals. Then, \(\mathcal{H} \)
	is an algebra on \(\mathbb{\MakeUppercase{R}} \).
\end{lemma}
\begin{proof}
	\todo{DIY}
\end{proof}

\begin{proposition}[Distribution function defines a pre-measure]
	Let \(F\colon \mathbb{\MakeUppercase{R}}\to \mathbb{\MakeUppercase{R}}  \) be a distribution function. For a
	half-interval \(I\), define
	\[
		\ell(I) \coloneqq \ell_F(I) = \begin{dcases}
			0,                        & \text{ if } I = \varnothing          \\
			F(b) - F(a),              & \text{ if } I = (a, b]               \\
			F(\infty ) - F(a),        & \text{ if } I = (a, \infty]          \\
			F(b) - F(-\infty ),       & \text{ if } I = (-\infty , b]        \\
			F(\infty ) - F(-\infty ), & \text{ if } I = (-\infty , \infty ).
		\end{dcases}
	\]
	Define \(\mu _0 \coloneqq \mu _{0, F}\) as
	\[
		\mu _{0, F}\colon \mathcal{H} \to [0, \infty ]
	\]
	by
	\[
		\mu _0(A) = \sum\limits_{k=1}^{N} \ell (I_{k}) \text{ if }A = \bigcup\limits_{k=1}^{N} I_{k},
	\]
	where \(A\) is a finite \underline{disjoint} union of half-intervals \(I_1, \ldots , I_{N} \).
	Then, \(\mu _0\) is a pre-measure on \(\mathcal{H}\).
\end{proposition}
\begin{proof}
	We see that
	\begin{enumerate}
		\item \(\mu _0\) is well-defined.
		\item \(\mu _0(\varnothing ) = 0\).
		\item \(\mu _0\) is finite additive.
		\item \(\mu _0\) is countable additive within \(\mathcal{H}\).
		      \par Suppose \(A\in \mathcal{H} \) where \(A = \bigcup\limits_{i=1}^{\infty} A_{i}\) is a countable
		      disjoint union. It is enough to consider the case that \(A = I\), \(A_{k} = I_{k}\) are all
		      half-intervals.\footnote{why?}

		      Focus on the case \(I = (a, b]\). Let
		      \[
			      (a, b] = \bigcup\limits_{n=1}^{\infty} (a_{n}, b_{n}],
		      \]
		      which is a disjoint union. Then we only need to check
		      \[
			      F(b) - F(a) = \sum\limits_{n=1}^{\infty} \left(F(b_{n}) - F(a_{n})\right).
		      \]
		      \begin{itemize}
			      \item Since \((a, b]\supset \bigcup\limits_{n=1}^{N} (a_{n}, b_{n}]\), hence
			            \[
				            \underset{N\in\mathbb{\MakeUppercase{N}} }{\forall}\ F(b) - F(a) \geq \sum\limits_{n=1}^{N} \left(F(b_{n}) - F(a_{n})\right).
			            \]
			            By letting \(N \to \infty\), we have
			            \[
				            F(b) - F(a) \geq \sum\limits_{n=1}^{\infty} \left(F(b_{n}) - F(a_{n})\right).
			            \]
			      \item Fix \(\epsilon >0\). Since \(F\) is right-continuous, \(\exists a ^\prime > a\) such that
			            \[
				            F(a ^\prime ) - F(a) <\epsilon.
			            \]
			            For each \(n\in\mathbb{\MakeUppercase{N}} \), \(\exists b_{n} ^\prime > b_{n}\) such that
			            \[
				            F(b_{n} ^\prime ) - F(b_{n})<\frac{\epsilon }{2^n}.
			            \]
			            Then, we have
			            \[
				            [a ^\prime , b] \subset \bigcup\limits_{n=1}^{\infty} (a_{n}, b_{n} ^\prime ),
			            \]
			            hence
			            \[
				            \underset{ N\in\mathbb{\MakeUppercase{N}}}{\exists}\ [a ^\prime , b]\subset \bigcup\limits_{n=1}^{N} (a_{n}, b_{n} ^\prime ),
			            \]
			            which is only finitely many unions now. In this case, we have
			            \[
				            F(b) - F(a ^\prime ) \leq \sum\limits_{n=1}^{N} F(b_{n} ^\prime ) - F(a_{n}).
			            \]

			            Finally, we see that
			            \[
				            \begin{split}
					            F(b) - F(a)&\leq F(b) - F(a ^\prime )+\epsilon \\
					            &\leq \sum\limits_{n=1}^{\infty} \left(F(b_{n} ^\prime ) - F(a_{n})\right) + \epsilon \\
					            &\leq \sum\limits_{n=1}^{\infty} \left(F(b_{n}) - F(a_{n}) + \frac{\epsilon }{2^n}\right) + \epsilon .
				            \end{split}
			            \]
		      \end{itemize}
	\end{enumerate}
\end{proof}
\begin{remark}
	It's again the \(\frac{\epsilon}{2^n}\) trick we saw before!
\end{remark}