\lecture{30}{25 Mar. 11:00}{Absolutely Continuous Measures}
\begin{eg}\label{eg:lec30}
	For an example of \autoref{thm:Jordan-decomposition-theorem}, let \((X, \mathcal{A}, \mu)\) be a \hyperref[def:measure-space]{measure space},
	\(f \colon X \to \overline{\mathbb{R}}\), and \(\nu(E) = \int_E f \,\mathrm{d} \mu\). Then
	\[
		\nu^+(E) = \int_E f^+ \,\mathrm{d} \mu,\quad \nu^-(E) = \int_E f^- \,\mathrm{d} \mu.
	\]
\end{eg}

\begin{definition*}
	Given a \hyperref[def:signed-measure]{signed measure} \(\nu \) on \((X, \mathcal{\MakeUppercase{a}} )\) and its
	\hyperref[thm:Jordan-decomposition-theorem]{Jordan decomposition} \(\nu = \nu ^+ - \nu ^-\).
	\begin{definition}[Positive variation]\label{def:positive-variation-measure}
		We call \(\nu ^+\) the \emph{positive variation} of \(\nu \).
	\end{definition}
	\begin{definition}[Negative variation]\label{def:negative-variation-measure}
		We call \(\nu ^-\) the \emph{negative variation} of \(\nu \).
	\end{definition}
	\begin{definition}[Total variation]\label{def:total-variation}
		The \emph{total variation measure of \(\nu\)}, denoted as \(\left\vert \nu \right\vert \), is defined as \(\left\vert \nu \right\vert \coloneqq \nu^+ + \nu^-\).
	\end{definition}
\end{definition*}

\begin{remark}
	There is always a \hyperref[def:signed-measure]{positive measure} on \(X\).
\end{remark}
\begin{explanation}
	Consider the \hyperref[def:total-variation]{total variation} \(\left\vert \nu \right\vert \) for an arbitrary \hyperref[def:signed-measure]{signed measure} \(\nu \).
\end{explanation}

\begin{eg}
	In the above \hyperref[eg:lec30]{example}, \(\left\vert \nu \right\vert (E) = \int_E \left\vert f \right\vert \,\mathrm{d} \mu\).
\end{eg}

\begin{lemma}
	We have the following
	\begin{enumerate}[(1)]
		\item \(\left\vert \nu(E) \right\vert \leq \left\vert \nu \right\vert(E)\).
		\item \(E\) is \hyperref[def:null-set-for-a-signed-measure]{\(\nu\)-null} if and only if \(E\) is \hyperref[def:null-set-for-a-signed-measure]{\(\left\vert \nu \right\vert\)-null}.
		\item If \(\kappa\) is another \hyperref[def:signed-measure]{signed measure}, then \(\kappa \perp \nu\)
		      if and only if \(\kappa \perp \left\vert \nu \right\vert\) if and only if \(\kappa \perp \nu^+\) and \(\kappa \perp \nu^-\).
	\end{enumerate}
\end{lemma}
\begin{proof}
	\todo{DIY}
\end{proof}

\begin{definition}[Finite signed measure]\label{def:finite-signed-measure}
	A \hyperref[def:signed-measure]{signed measure} \(\nu\) is \emph{finite} if \(\left\vert \nu \right\vert\) is a \hyperref[def:finite-measure]{finite measure},
	and similarly for \hyperref[def:sigma-finite-measure]{\(\sigma\)-finite}.
\end{definition}
\begin{remark}
	This holds if and only if \(\nu^+,\nu^-\) are both \hyperref[def:finite-measure]{finite} (resp. \hyperref[def:sigma-finite-measure]{\(\sigma\)-finite})
	\hyperref[def:measure]{measures}.
\end{remark}

\section{Absolutely Continuous Measures}
\begin{definition}[Absolutely continuous measure]\label{def:absolutely-continuous-measure}
	Let \(\mu\) be a \hyperref[def:signed-measure]{positive measure}, \(\nu\) be a \hyperref[def:signed-measure]{signed measure}, both on \((X, \mathcal{A})\). We say that \(\nu\) is
	\emph{absolutely continuous with respect to \(\mu\)}, denoted as \(\nu \ll \mu\), provided that for all \(E \in \mathcal{A}\), \(\mu(E) = 0\) implies \(\nu(E) = 0\).
\end{definition}
\begin{remark}
	This is equivalent to every \hyperref[def:null-set-for-a-signed-measure]{\(\mu\)-null} set being \hyperref[def:null-set-for-a-signed-measure]{\(\nu\)-null}.
\end{remark}
\begin{eg}
	If \((X ,\mathcal{A}, \mu)\), \(f \colon X \to \overline{\mathbb{R}}\), \(\nu(E) = \int_E f \,\mathrm{d} \mu\), then \(\nu \ll \mu\).
\end{eg}

\begin{notation}
	\(\,\mathrm{d} \nu = f \,\mathrm{d} \mu\) means \(\nu\) is a \hyperref[def:signed-measure]{signed measure} defined by
	\[
		\nu(E) = \int_E f \,\mathrm{d} \mu.
	\]
\end{notation}

\begin{lemma}
	If \(\mu\) is a \hyperref[def:signed-measure]{positive measure}, \(\nu\) is a \hyperref[def:signed-measure]{signed measure} on \((X, \mathcal{A})\), then
	\begin{enumerate}[(1)]
		\item \(\nu \ll \mu\) if and only if \(\left\vert \nu \right\vert \ll \mu\) if and only if \(\nu^+ \ll \mu\) and \(\nu^- \ll \mu\).
		\item \(\nu \ll \mu\) and \(\nu \perp \mu\) implies \(\nu = 0\).
	\end{enumerate}
\end{lemma}

\begin{proof}
	\todo{DIY (1)}

	For (2), write \(X = A \cup B\), \(A \cap B = \varnothing\), \(A\) \hyperref[def:null-set-for-a-signed-measure]{\(\mu\)-null}, \(B\) \hyperref[def:null-set-for-a-signed-measure]{\(\nu\)-null}.
	Then
	\[
		\nu(E) = \nu(E \cap A) + \nu(E \cap B) = \nu(E \cap A).
	\]
	Then \(E \cap A \subseteq A\), so \(\nu(E \cap A) = 0\). By \hyperref[def:absolutely-continuous-measure]{absolute continuity}, \(\nu(E \cap A) = 0\), thus \(\nu(E) = 0\).
\end{proof}

\begin{theorem}[Radon-Nikodym theorem]\label{thm:Radon-Nikodym-theorem}
	Suppose \(\mu\) is a \hyperref[def:finite-signed-measure]{\(\sigma\)-finite positive measure}, \(\nu\) is a \hyperref[def:finite-signed-measure]{\(\sigma\)-finite signed measure},
	and suppose \(\nu \ll \mu\). Then there exists \(f \colon X \to \overline{\mathbb{R}}\) such that \(\,\mathrm{d} \nu = f \,\mathrm{d} \mu\), in other words \(\nu(E) = \int_E f \,\mathrm{d} \mu\).

	If \(g\) is another such function with \(\,\mathrm{d} \nu = g \,\mathrm{d} \mu\) then \(f = g\) \hyperref[def:mu-almost-everywhere]{\(\mu\)-a.e.}.
\end{theorem}
\begin{proof}
	We'll prove a more general form called \hyperref[thm:Lebesgue-Radon-Nikodym-theorem]{Lebesgue Radon Nikodym theorem}, which is a more general
	theorem compare to this theorem.
\end{proof}

\begin{definition}[Randon-Nikodym derivative]\label{def:Radon-Nikodym-derivative}
	Suppose \(\nu \ll \mu\). The \emph{Radon-Nikodym derivative of \(\nu\) with respect to \(\mu\)} is a function
	\[
		\frac{\,\mathrm{d} \nu}{\,\mathrm{d} \mu} \colon X \to \overline{\mathbb{R}}
	\]
	such that
	\[
		\nu(E) = \int_E \frac{\,\mathrm{d} \nu}{\,\mathrm{d} \mu} \,\mathrm{d} \mu
	\]
	for all \(E \in \mathcal{A}\).
\end{definition}
\begin{remark}
	i.e. we have \(\,\mathrm{d} \nu = \frac{\,\mathrm{d} \nu}{\,\mathrm{d} \mu} \,\mathrm{d} \mu\) in our notation.
\end{remark}
\begin{note}
	By \autoref{thm:Radon-Nikodym-theorem}, such a function exists and is unique up to equivalence \hyperref[def:mu-almost-everywhere]{\(\mu\)-a.e.}
	in the \hyperref[def:finite-signed-measure]{\(\sigma\)-finite} case.
\end{note}
\begin{eg}
	Let \(F(x) = e^{2x} \colon \mathbb{R} \to \mathbb{R}\), then
	\[
		\frac{\,\mathrm{d} F}{\,\mathrm{d} m} = 2e^{2x}.
	\]
\end{eg}
\begin{explanation}
	Since \(F\) is continuous and strictly increasing, so we may define a \hyperref[def:Lebesgue-Stieltjes-measure]{Lebesgue-Stieltjes measure}
	\(\mu_F\) on \((\mathbb{R}, \mathcal{B}(\mathbb{R}))\).

	This is defined to be the unique \hyperref[def:locally-finite]{locally finite} \hyperref[def:measure]{measure} satisfying
	\[
		\mu_F([a,b]) = F(b) - F(a) = e^{2b} - e^{2a}.
	\]
	Then one can check that
	\[
		\mu_F(E) = \int_E 2e^{2x} \,\mathrm{d} x.
	\]
	By uniqueness and the classical \underline{fundamental theorem of calculus}, since the right-hand side is a \hyperref[def:locally-finite]{locally finite}
	\hyperref[def:Borel-measure]{Borel measure}, and \(\kappa([a,b]) = e^{2b} - e^{2a}\), thus \(\mu_F = \kappa\). Therefore, \(\mu_F \ll m\) and we have
	\[
		\frac{\,\mathrm{d} \mu_F}{\,\mathrm{d} m} = 2e^{2x} = \frac{\,\mathrm{d} F}{\,\mathrm{d} x}.
	\]
\end{explanation}

\begin{eg}
	Let \(\mu _C = C(X) \colon \mathbb{R} \to \mathbb{R}\) be the \hyperref[sssec:Cantor-Function]{Cantor function}. Then \(C^\prime(x) = 0\)
	outside the \hyperref[eg:lec8:Cantor-set]{Cantor set}, but we don't always have
	\[
		\mu_C(E) \neq \int_E 0 \,\mathrm{d} x.
	\]
	So the candidate derivative is \(0\), but this fails.
\end{eg}
\begin{explanation}
	In particular,
	\[
		C(b) - C(a) \neq \int_a^b C'(x) \,\mathrm{d} x.
	\]
	In fact, \(\mu_C \not\ll m\) because \(\mu_C \perp m\) and \(\mu_C \neq 0\).

	Thus, the existence of a derivative \hyperref[def:mu-almost-everywhere]{almost everywhere} and continuity is not enough to guarantee
	a version of the \underline{fundamental theorem of calculus} to hold.
\end{explanation}