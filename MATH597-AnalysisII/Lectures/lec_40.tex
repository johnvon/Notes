\lecture{40}{18 Apr.\ 11:00}{Fourier Analysis}
\begin{prev}
	We have from last time that for \(0 \leq r < 1\),
	\[
		\begin{split}
			e_n(x)                & \coloneqq \frac{1}{\sqrt{2\pi} }e^{inx}                                                                                    \\
			P_r(t)                & \coloneqq \sum_{n=-\infty}^\infty e_n(t)r^{\left\vert n \right\vert }                                                      \\
			f \in L^1([-\pi,\pi]) & \implies \sum_{n=-\infty}^\infty \hat{f}_ne_n(x)r^{\left\vert n \right\vert } = \int_{-\pi}^\pi P_r(x-y)f(y) \,\mathrm{d}y
		\end{split}
	\]
	uniformly in \(x \in [-\pi,\pi]\). Furthermore, if \(f \in C([-\pi,\pi])\) and \(f(\pi) = f(-\pi)\) then
	\[
		\lim_{r \to 1} \int_{-\pi}^\pi P_r(x-y)f(y) \,\mathrm{d}y = f(x)
	\]
	uniformly in \(x \in [-\pi,\pi]\).

	Finally, we also have
	\[
		\left\lVert f\right\rVert_1 \leq \sqrt{2\pi} \left\lVert f\right\rVert_2 \leq 2\pi \left\lVert f\right\rVert_\infty.
	\]
\end{prev}

\begin{theorem}[Fourier Series]\label{thm:fourier-series}
	The set of \(e_n(x) = \frac{1}{\sqrt{2\pi} }e^{inx}\) for \(n \in \mathbb{Z} \) is an
	\hyperref[def:orthonormal-basis]{orthonormal basis} of \(L^2([-\pi,\pi])\).
\end{theorem}
\begin{proof}
	From \autoref{lma:lec-39}, this is an \hyperref[def:orthonormal-set]{orthonormal set}. Thus, we just need to show that for
	all \(f \in L^2([-\pi,\pi])\), that there exists \(h = \sum_{n=-M}^N \beta_n e_n(x)\) such that
	\(\left\lVert f-h\right\rVert _2 < \epsilon \). That is we need to show that the span of the \(e_n\) is dense.

	Let \(f \in L^2([-\pi,\pi])\) and fix an \(\epsilon > 0\). Then there is a function \(g \in C([-\pi,\pi])\) with \(g(\pi) = g(-\pi)\)
	such that \(\left\lVert f-g\right\rVert_2 < \frac{\epsilon}{3}\) which follows from a simple density argument.

	Let \(g_r(x) = \int_{-\pi}^\pi P_r(x-y)g(y) \,\mathrm{d}y\). By the above, there exists an \(r \in [0,1)\) such that
	\(\left\lVert g_r - g\right\rVert_\infty < \frac{\epsilon }{3\sqrt{2\pi}}\).

	Therefore, \(\left\lVert g_r - g\right\rVert_2 < \frac{\epsilon }{3}\). Consider
	\(g_{r,N}(x) = \sum_{n=-N}^N \hat{g} _n e_n(x)r^{\left\vert n \right\vert }\). By the above there exists an
	\(N \in \mathbb{N} \) such that \(\left\lVert g_{r,N} - g_r\right\rVert _\infty < \frac{\epsilon }{3\sqrt{2\pi}}\),
	thus \(\left\lVert g_{r,N} - g_r\right\rVert_2 < \frac{\epsilon}{3}\). Therefore,
	\[
		\left\lVert f - g_{r,N}\right\rVert < \epsilon
	\]
	where \(g_{r,N}\) is a finite linear combination of \(e_n\)'s, so these form an \hyperref[def:orthonormal-basis]{orthonormal basis}
	as desired.
\end{proof}

\begin{eg}[Plancherel's identity]\label{eg:Plancherel-identity}
	We have \(\left\lVert f\right\rVert _2^2 = \sum_{n=-\infty}^\infty \vert \hat{f}_n \vert ^2\).
\end{eg}
\begin{explanation}
	For \(f(x) = x\), we have
	\[
		\hat{f}_n = \frac{1}{\sqrt{2\pi}}\int_{-\pi}^\pi x e^{-inx} \,\mathrm{d}x = \begin{dcases}
			0,                            & \text{ if } n=0 ;     \\
			\frac{(-1)^ni\sqrt{2\pi}}{n}, & \text{ if } n\geq 0 .
		\end{dcases}
	\]
	Together these imply that
	\[
		\sum_{n=1}^\infty \frac{1}{n^2} = \frac{\pi^2}{6}.
	\]
\end{explanation}
\begin{eg}[Isoperimetric inequality.]
	Consider a parametric curve \((x(t),y(t)) \in \mathbb{R}^2\) with \(t \in [-\pi,\pi]\). Assume that
	\begin{enumerate}[(a)]
		\item This is a closed curve, so \((x(-\pi),y(-\pi)) = (x(\pi),y(\pi))\).
		\item We assume these are smooth, but in fact we just need \(x,y\) are \(C^1\) functions.
		\item The curve is simple.
	\end{enumerate}
	Suppose that
	\[
		L = \int_{-\pi}^\pi \sqrt{x^\prime(t)^2 + y^\prime(t)^2}  \,\mathrm{d}t = 2\pi,
	\]
	then the largest area \(A\) this curve can enclose is \(\pi \).
\end{eg}
\begin{explanation}
	By \href{https://en.wikipedia.org/wiki/Green's_theorem}{Green's theorem},
	\[
		A = \frac{1}{2} \oint_C (x \,\mathrm{d}y - y \,\mathrm{d}x) = \frac{1}{2} \int_{-\pi}^\pi (x(t)y'(t) - x'(t)y(t)) \,\mathrm{d}t.
	\]
	Arc length parametrization, so that \(x^\prime (t)^2 + y^\prime (t)^2 = 1\) for all \(t\). Then the condition \(L = 2 \pi\) is automatically satisfied and can be written as
	\[
		\int_{-\pi}^\pi (x'(t)^2 + y^\prime(t)^2) \,\mathrm{d}t = 2\pi.
	\]

	For ease of computation, rewrite using complex numbers, i.e., \(z(t) = x(t) + i y(t)\). This is then subject to
	\[
		\int_{-\pi}^\pi \left\vert z^\prime(t) \right\vert^2 \,\mathrm{d}t = 2 \pi i.
	\]
	Rewriting our above formula for area, we need to find the maximum of
	\[
		A = \frac{1}{4i}\int_{-\pi}^\pi (\overline{z(t)}z^\prime(t) - z(t)\overline{z^\prime(t)}) \,\mathrm{d}t.
	\]
	Note \(z\) is \(C^1\) and \(z(\pi) = z(-\pi)\). Denote \(\hat{z}_n = \alpha_n\), then we now have
	\[
		\widehat{z^\prime}_n
		= \frac{1}{\sqrt{2\pi} } \int_{-\pi}^\pi z^\prime(t) e^{-in t} \,\mathrm{d}t
		= \left(z(t)e^{-i n t}\right)^\pi_\pi + in\int_{-\pi}^\pi z(t) e^{-in t} \,\mathrm{d}t
		= in\alpha_n.
	\]
	By \hyperref[eg:Plancherel-identity]{Plancherel's identity}, we have that
	\begin{align*}
		\int_{-\pi}^\pi \left\vert z^\prime (t) \right\vert ^2 \,\mathrm{d}t
		= \sum_{n=-\infty}^\infty \left\vert in\alpha_n \right\vert ^2
		= \sum_{n=-\infty}^\infty n^2 \left\vert \alpha_n \right\vert ^2 = 2\pi.
	\end{align*}
	We know \hyperref[eg:Plancherel-identity]{Plancherel's identity} says \(\left< f,g \right> = \sum_{n=-\infty}^\infty \hat{f}_n\overline{\hat{g}_n}\). Therefore,
	\[
		A = \frac{1}{4i} \sum_{n=-\infty}\infty \overline{\alpha_n}(in\alpha_n) - \alpha_n\overline{(in\alpha_n)}
		= \frac{1}{2} \sum_{n=-\infty}^\infty n \left\vert \alpha_n \right\vert ^2.
	\]
	Then the question becomes what is the maximum of \(\frac{1}{2}\sum_{n=-\infty}^\infty n\left\vert \alpha_n \right\vert^2\) subject to \(\sum_{n=-\infty}^\infty n^2\left\vert \alpha_n \right\vert^2 = 2\pi\).

	We will show that
	\[
		2 \pi - \sum_{-\infty}^\infty n\left\vert \alpha_n \right\vert^2 \geq 0
	\]
	as we guess the maximum should be \(\pi\), by virtue of the circle being theoretically optimal. Then we have
	\[
		2 \pi - \sum_{-\infty}^\infty n\left\vert \alpha_n \right\vert^2 = \sum_{n=-\infty}^\infty (n^2-n)\left\vert \alpha_n \right\vert^2 \geq 0
	\]
	because term by term every term is non-negative.

	Thus, the area cannot be more than \(\pi\). We have \(A = \pi\) if and only if equality holds above, that is \(\alpha_n = 0\) for \(n \neq 0,1\). This means \(z(t) = \alpha_0 + \alpha_1 e_1(x)\), that is
	\[
		z(t) = \alpha_0 + Ce^{it}.
	\]
	This means that \(z(t)\) is a circle about \(\alpha_0\).
\end{explanation}

\begin{note}
	Professor Baik likes Dym and McKean's Fourier Series \& Integrals~\cite{dym1972fourier}!
\end{note}
