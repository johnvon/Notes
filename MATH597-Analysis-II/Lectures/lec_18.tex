\lecture{18}{16 Feb.\ 11:00}{Monotone Class}
Let's start with a theorem.
\begin{theorem}
	Let \((X, \mathcal{A} , \mu ), (Y, \mathcal{B} , \nu )\) be \hyperref[def:measure-space]{measure spaces}.
	\begin{enumerate}[(a)]
		\item There is a \hyperref[def:measure]{measure} \(\mu \times \nu \) on \(\mathcal{A} \otimes \mathcal{B} \) such that for every \(A\in \mathcal{A} , B\in \mathcal{B} \),
		      \[
			      (\mu \times \nu )(A\times B) = \mu (A)\nu (B).
		      \]
		\item If \(\mu , \nu \) are \hyperref[def:sigma-finite-measure]{\(\sigma\)-finite }, then \(\mu \times \nu \) is unique.
	\end{enumerate}
\end{theorem}
\begin{proof}
	We prove this one by one.
	\begin{enumerate}[(a)]
		\item Define \(\pi \colon \mathcal{R} \to [0, \infty ]\) by \(\pi (A \times B) = \mu (A)\nu (B)\), and extending linearly, we have
		      \[
			      \pi (A\times B) = \mu (A)\nu (B),
		      \]
		      hence
		      \[
			      \pi \left(\coprod_{i=1}^{N} A_{i} \times B_{i} \right) = \sum_{i=1}^{n} \pi (A_{i} \times B_{i}).
		      \]

		      We claim that \(\pi \) is a \hyperref[def:pre-measure]{pre-measure}. To show this, it's enough to check that \(\pi (A\times B) = \sum_{n=1}^{\infty} \pi (A_{n} \times B_{n} )\)
		      if \(A\times B = \coprod_{n}A_{n} \times B_{n}  \). Since \(A_{n} \times B_{n} \) are disjoint, so
		      \[
			      \mathbbm{1}_{A\times B}(x, y) = \sum_{n=1}^{\infty} \mathbbm{1}_{A_{n} \times B_{n} }(x, y).
		      \]
		      Thus,
		      \[
			      \mathbbm{1}_{A} (x)\mathbbm{1}_{B} (y) = \sum_{n=1}^{\infty} \mathbbm{1}_{A_{n} }(x)\mathbbm{1}_{B_{n}}(y).
		      \]
		      Integrating w.r.t.\ \(x\), and applying \autoref{thm:Tonelli-for-series}, we have
		      \[
			      \int _X \mathbbm{1}_{A} (x)\mathbbm{1}_{B}(y)\,\mathrm{d} \mu (x) = \sum_{n=1}^{\infty} \int _X \mathbbm{1}_{A_{n} }(x)\mathbbm{1}_{B_{n} }(y)\,\mathrm{d} \mu (x),
		      \]
		      which implies
		      \[
			      \mu (A)\mathbbm{1}_{B} (y) = \sum_{n=1}^{\infty} \mu (A_{n} )\mathbbm{1}_{B_{n} }(y)
		      \]
		      for every \(y\). We can then integrate again w.r.t.\ \(y\) and apply \autoref{thm:Tonelli-for-series}, we have
		      \[
			      \int _Y \mu (A)\mathbbm{1}_{B}(y)\,\mathrm{d} \nu (y) = \sum_{n=1}^{\infty} \int _Y \mu (A_{n} )\mathbbm{1}_{B_{n} }(y)\,\mathrm{d} \nu (y),
		      \]
		      which gives us
		      \[
			      \mu (A)\nu (B) = \sum_{n=1}^{\infty} \mu (A_{n} )\nu (B_{n} ).
		      \]
		      Hence, we see that \(\mu\) is indeed a \hyperref[def:pre-measure]{pre-measure}, so \autoref{thm:Hahn-Kolmogorov} gives \(\mu \times \nu \) on \(\left< \mathcal{R}  \right> = \mathcal{A} \otimes \mathcal{B} \) extending \(\pi \) on \(\mathcal{R} \).
		\item If \(\mu , \nu \) are \hyperref[def:sigma-finite-measure]{\(\sigma\)-finite}, then \(\pi \) is \hyperref[def:sigma-finite-measure]{\(\sigma\)-finite} on \(\mathcal{R} \), then \autoref{thm:uniqueness-of-HK-extension} applies. Moreover, we have
		      \[
			      (\mu \times \nu )(E) = \inf \left\{\sum_{i=1}^{\infty} \mu (A_{i})\nu (B_{i})\mid E\subset \bigcup_{i=1}^{\infty} A_{i} \times B_{i}, A_{i} \in \mathcal{A} , B_{i} \in \mathcal{B} \right\}.
		      \]
	\end{enumerate}
\end{proof}

\section{Monotone Class Lemma}
Let's start with a definition.
\begin{definition}[Monotone Class]\label{def:monotone-class}
	Given a set \(X\)< \(C\subset \mathcal{P} (X)\) is a \emph{monotone class} on \(X\) if
	\begin{itemize}
		\item \(C\) is closed under countable increasing unions, and
		\item \(C\) is closed under countable decreasing intersections.
	\end{itemize}
\end{definition}

\begin{eg}
	Every \hyperref[def:sigma-algebra]{\(\sigma\)-algebra} is a \hyperref[def:monotone-class]{monotone class}.
\end{eg}

\begin{eg}
	If \(C_\alpha \) are (arbitrarily many) \hyperref[def:monotone-class]{monotone classes} on a set \(X\), then \(\bigcap_{\alpha}C_\alpha \) is a \hyperref[def:monotone-class]{monotone class}. Furthermore, if \(\mathcal{E} \subset \mathcal{P} (X)\), there is a unique smallest \hyperref[def:monotone-class]{monotone class} containing \(\mathcal{E}\), denoted by \(\left< \mathcal{E}  \right> \), which follows the same idea as in \autoref{def:generation-of-sigma-algebra}.
\end{eg}

\begin{theorem}[Monotone class lemma]\label{thm:monotone-class-lemma}
	Suppose \(\mathcal{A} _0\) is an \hyperref[def:algebra]{algebra} on \(X\). Then \(\left< \mathcal{A} _0 \right>\)\footnote{\(\left< \mathcal{A} _0 \right> \) is
		the \hyperref[def:sigma-algebra]{\(\sigma\)-algebra} generated by \(\mathcal{A} _0\) by \autoref{def:generation-of-sigma-algebra}.} is the \hyperref[def:monotone-class]{monotone class} generated by \(\mathcal{A} _0\).
\end{theorem}
\begin{proof}
	Let \(\mathcal{A}  = \left< \mathcal{A} _0 \right> \) and let \(\mathcal{C} \) be the \hyperref[def:monotone-class]{monotone class} generated by \(\mathcal{A} _0\). Since \(\mathcal{A} \) is a \hyperref[def:sigma-algebra]{\(\sigma\)-algebra}, it's a \hyperref[def:monotone-class]{monotone class}. Note that it contains \(\mathcal{A} _0\), hence \(\mathcal{A} \supset \mathcal{C} \).

	To show \(\mathcal{C} \supset \mathcal{A} \), it's enough to show that \(\mathcal{C} \) is a \hyperref[def:sigma-algebra]{\(\sigma\)-algebra}. We check that
	\begin{enumerate}
		\item \(\varnothing \in \mathcal{A} _0 \subseteq \mathcal{C} \).
		\item Let \(\mathcal{C} ^\prime = \{E\subset X \mid E^{c} \in \mathcal{C} \}\).
		      \begin{itemize}
			      \item \(\mathcal{C} ^\prime \) is a \hyperref[def:monotone-class]{monotone class}.
			      \item \(\mathcal{A} _0\subset \mathcal{C} ^\prime \) because if \(E\in \mathcal{A} _0\), then \(E^{c} \in \mathcal{A} _0\), so
			            \(E^{c} \in \mathcal{C} \), thus \(E\in \mathcal{C} ^\prime \).
		      \end{itemize}
		      We see that \(\mathcal{C} ^\prime \subset \mathcal{C}^\prime \), so \(\mathcal{C} \) is closed under complements.
		\item For \(E\subset X\), let \(\mathcal{D} (E) = \{F\in \mathcal{C} \mid E \cup F\in \mathcal{C} \}\).
		      \begin{itemize}
			      \item \(\mathcal{D} (E)\subset \mathcal{C} \).
			      \item \(\mathcal{D} (E)\) is a \hyperref[def:monotone-class]{monotone class}.
			      \item If \(E\in \mathcal{A} _0\), then \(\mathcal{A} _0\subset \mathcal{D} (E)\). We see this by picking \(F\in \mathcal{A} _0\), then
			            \(E\cup F\in \mathcal{A} _0\supset \mathcal{C} \).
		      \end{itemize}
		      Hence, \(C = \mathcal{D} (E)\) if \(E\in \mathcal{A} _0\).
		\item Let \(\mathcal{D} = \{E\in \mathcal{C} \mid \mathcal{D} (E) = \mathcal{C} \}\). That is \(\mathcal{D}  = \{E\in \mathcal{C} \mid E\cup F, \forall F\in \mathcal{C} \}\).
		      Then we have
		      \begin{itemize}
			      \item \(A_0\subset \mathcal{D} \) by 3.
			      \item \(\mathcal{D} \) is a \hyperref[def:monotone-class]{monotone class}.
			      \item \(\mathcal{D} \subset \mathcal{C} \) by definition.
		      \end{itemize}
		      Thus, \(\mathcal{D}  = \mathcal{C} \), so if \(E, F\in \mathcal{C} \), then \(E\cup F\in \mathcal{C} \). This implies that \(\mathcal{C} \) is closed under finite unions.
		\item Now to show that \(\mathcal{C}\) is closed under countable unions, let \(E_1, E_2, \dots \in \mathcal{C}  \). We may then define
		      \[
			      F_{N} = \bigcup_{n=1}^{N} E_{n} \in \mathcal{C} .
		      \]
		      Then we see that \(F_1\subset F_2\subset \dots  \), hence \(\bigcup_{N} F_{N} \in \mathcal{C}\). But this simply implies
		      \[
			      \bigcup_{N}F_{N} = \bigcup_{n}E_{n} ,
		      \]
		      so we're done.
	\end{enumerate}
\end{proof}