\lecture{19}{16 Mar. 14:30}{\(\acl\) in Algebraically Closed Fields}
\begin{theorem}
	Let \(K \models \ACF\), and \(A \subseteq K\). The \(\acl(A)\)  is the set of all elements \(b\in K\) which are \hyperref[def:algebraic]{algebraic} over \(A\).\footnote{I.e., that satisfy a non-zero polynomial with coefficients in the ring generated by \(A\).}
\end{theorem}
\begin{proof}
	If \(p(x)\) is a non-zero polynomial over \(A\), with \(p(b) = 0\), then \(\left\{ c \mid p(c) = 0 \right\} \), so \(p(x) = 0\) witnesses that \(b\in \acl(A)\).

	Suppose that \(b\in \acl(A)\), as witnessed by \(\varphi (x, \overline{a} )\) for \(\overline{a} \in A\). We may assume that \(\varphi \) is \hyperref[not:quantifier-free]{quantifier-free}. Write
	\[
		\varphi (x, \overline{a} ) = \psi _1(x, \overline{a} ) \lor \dots \lor \psi _n(x, \overline{a} ),
	\]
	where each \(\psi _i\) is a conjunction of \hyperref[not:atomic]{atomic} or negated \hyperref[not:atomic]{atomic} \hyperref[def:formula]{formulas}. We may replace \(\varphi \) with some \(\psi _i\), choosing one that \(b\) satisfies. Then,
	\[
		\varphi (x, \overline{a} ) = p_1(x, \overline{a} ) \land \dots \land p_k(x, \overline{a} ) = 0 \land q_1(x, \overline{a} ) \neq 0 \land \dots \land q_{\ell}(x, \overline{a} ) \neq 0.
	\]
	In order for this to have finitely many solutions, we must have \(k \geq 1\) and some \(p_i(x, \overline{a} )\) non-zero polynomial in \(x\). But then \(b\) satisfies a polynomial over \(A\).
\end{proof}

\section{Types}
Now, unless we specify, we will now work in a \hyperref[def:model]{model} of a \hyperref[def:strongly-minimal]{strongly minimal} \hyperref[def:theory]{theory}.

\begin{definition}[Independent]\label{def:independent}
	A set \(X\) is \emph{independent} if for all \(a\in X\), \(a \notin \acl(X - \left\{ a \right\} )\).
\end{definition}

\begin{intuition}
	Think about the case of vector spaces.
\end{intuition}

\begin{definition}[Basis]\label{def:basis}
	A set \(X\) is a \emph{basis} of \(Y\) if it's the maximal \hyperref[def:independent]{independent} set in \(Y\).
\end{definition}

\begin{note}
	In homework, we will show that each set \(Y\) always contains a \hyperref[def:basis]{basis}.
\end{note}

\begin{definition}[Dimension]\label{def:dimension}
	The \emph{dimension} \(\dim Y\) of \(Y\) is this cardinality of its \hyperref[def:basis]{basis}.
\end{definition}

\begin{remark}
	If \(X_1, X_2\) are two \hyperref[def:basis]{bases} for \(Y\), then \(\vert X_1 \vert = \vert X_2 \vert \), i.e., \autoref{def:dimension} is well-defined.
\end{remark}

\begin{definition}[Type]\label{def:type}
	Let \(\mathcal{M} \) be an \hyperref[def:structure]{\(\mathcal{L} \)-structure}, \(\overline{c} \in M\), \(A \subseteq M\). The \emph{type} of \(\overline{c} \) over \(A\) (in \(\mathcal{M} \)) is
	\[
		\tp^{\mathcal{M} }\left( \overline{c} / A \right) = \left\{ \varphi (\overline{x} , \overline{a} ) \mid \mathcal{M} \models \varphi (\overline{c} , \overline{a} ) \text{ and } \overline{a} \in A \right\}.
	\]
\end{definition}

\begin{notation}
	Where we omit \(/ A\), we just mean \(/ \varnothing\).
\end{notation}

\begin{lemma}\label{lma:lec19-1}
	Let \(T\) be a \hyperref[def:theory-complete]{complete} \hyperref[def:strongly-minimal]{strongly minimal} \hyperref[def:theory]{theory}, and \(\mathcal{M} , \mathcal{N} \models T\). Let \(\overline{a} \in M\) and \(\overline{b} \in N\) be \hyperref[def:independent]{independent} tuples of the same size, then \(\tp^{\mathcal{M} }(\overline{a} ) = \tp^{\mathcal{N} }(\overline{b} )\), i.e., \(\mathcal{M} \models \varphi (\overline{a} ) \iff \mathcal{N} \models \varphi (\overline{b} )\).
\end{lemma}
\begin{proof}
	We do an induction on the lengths of \(\overline{a} , \overline{b} \). Start with \(n = 1\): let \(a, b\) be \hyperref[def:independent]{independent}, then
	\[
		\tp(a) = \left\{ \varphi (x) \mid \text{\(\varphi (x)\) has \hyperref[def:cofinite]{cofinitely} many solutions in \(\mathcal{M} \) }\right\}
	\]
	since \(a\) is \hyperref[def:independent]{independent}, \(a \notin \acl(\varnothing )\), i.e., \(\varphi (\mathcal{M} ) \coloneqq \left\{ c\in M \mid \mathcal{M} \models \varphi (c) \right\} \) is infinite for all \(\varphi \). And since \(\mathcal{M} \) is strongly minimal, it's \hyperref[def:cofinite]{cofinite}.

	\begin{eg}
		In \(\ACF\), \(\tp(a) = \left\{ p(a) \neq 0 \mid p\in \mathbb{Q} [x], p \neq 0 \right\} \) (or ``generated by'' these). Also, \((1 + 1) x\) being \(0\) depends on the \hyperref[def:characteristic]{characteristic}.
	\end{eg}

	Similarly, we have
	\[
		\tp(b) = \left\{ \varphi (x) \mid \text{\(\varphi (x)\) has \hyperref[def:cofinite]{cofinitely} many solutions in \(\mathcal{N} \) }\right\}.
	\]
	Now, suppose that \(\varphi \) has \(k\) non-solutions in \(\mathcal{M} \), \(\mathcal{M} \models \exists ^{=k} x\ \lnot \varphi (x)\), with \(T\) being \hyperref[def:theory-complete]{complete}, \(T \models \exists ^{=k} x\ \lnot \varphi (x)\), hence \(\mathcal{N} \models \exists ^{=k} x\ \lnot \varphi (x)\). So if \(\varphi \) has \hyperref[def:cofinite]{cofinitely} many (all but \(k\)) solutions in \(\mathcal{M} \), the same is true in \(\mathcal{N} \). Hence, \(\tp(a) = \tp(b)\).

	\begin{eg}
		The \hyperref[def:theory-complete]{completeness} is important: \(\ACF\) is a non-example.
	\end{eg}

	For \(n + 1\), let \(\overline{a} a^{\prime} \) and \(\overline{b} b^{\prime} \) be \hyperref[def:independent]{independent} \((n+1)\)-tuples with \(\tp(\overline{a} ) = \tp(\overline{b} )\). Suppose \(\mathcal{M} \models \varphi (\overline{a}, a^{\prime} )\). Since \(\mathcal{M} \models T\) is \hyperref[def:strongly-minimal]{strongly minimal} and \(a^{\prime} \notin\acl(\overline{a} )\), \(\varphi (\overline{a} , \mathcal{M} ) = \left\{ c\in M \mid \mathcal{M} \models \varphi (\overline{a} , c) \right\} \) is \hyperref[def:cofinite]{cofinite} with complement of size \(k\). Then \(\mathcal{M} \models \exists ^{=k}x\ \lnot \varphi (\overline{a} , x)\), so \(\exists ^{=k}x\ \lnot \varphi (\overline{y} , x) \in \tp(\overline{a} ) = \tp(\overline{b} )\), i.e., \(\mathcal{N} \models \exists ^{=k}x\ \lnot \varphi(\overline{b} , x) \). Then, since \(b^{\prime} \notin \acl(\overline{b} )\), so \(b^{\prime} \in \varphi (\overline{b} , \mathcal{N} )\),\footnote{Since \(\varphi (\overline{b} , \mathcal{N} )\) is \hyperref[def:cofinite]{cofinite}, and if \(b^{\prime} \) is in the complement of \(\varphi (\overline{b} , \mathcal{N} )\), which is finite, \(b^{\prime} \in \acl(\overline{b} )\) by \(\lnot \varphi (\overline{b} , x) \conta\)} hence \(\mathcal{N} \models \varphi (\overline{b} , b^{\prime} )\).
\end{proof}

\begin{lemma}\label{lma:lec19-2}
	Let \(T\) be a \hyperref[def:strongly-minimal]{strongly minimal} \hyperref[def:theory]{theory}. If \(\mathcal{M} \models T\) and \(c\in \acl(A)\), then there is a \hyperref[def:formula]{formula} \(\varphi (x, \overline{a} ) \in \tp(c / A)\) such that for any other \(c^{\prime} \in M\) with \(\mathcal{M} \models \varphi (c^{\prime} , \overline{a} )\), \(\tp(c / A) = \tp(c^{\prime} / A)\).
\end{lemma}
\begin{proof}
	Let \(\varphi (x, \overline{a} )\in \tp(c / A)\) be such that
	\[
		\vert \left\{ x\in M \mid \mathcal{M} \models \varphi (x, \overline{a} ) \right\}  \vert = k
	\]
	is minimal. We claim that \(\varphi (x, \overline{a} )\) witnesses the statement. If not, there are some \(c^{\prime} \) and a \hyperref[def:formula]{formula} \(\psi (x, \overline{a} ^{\prime} )\in \tp(c / A)\) with \(\mathcal{M} \models \varphi (c^{\prime} , \overline{a} ) \land \lnot \psi (c^{\prime} , \overline{a} ^{\prime} )\). So
	\[
		\left\{ x\in M \mid \mathcal{M} \models \varphi (x, \overline{a} ) \land \psi (x, \overline{a} ^{\prime} ) \right\}
		\subsetneq \left\{ x \in M \mid \mathcal{M} \models \varphi (d, \overline{a} ) \right\}
	\]
	since \(c^{\prime} \) is in the right-hand side but not the left-hand side. This implies that the left-hand side has cardinality \(< k\), a contradiction, so \(\varphi \) does imply \(\psi \).
\end{proof}

\begin{theorem}
	Let \(T\) be a \hyperref[def:theory-complete]{complete} \hyperref[def:strongly-minimal]{strongly minimal} \hyperref[def:theory]{theory}. If \(\mathcal{M} , \mathcal{N} \models T\), then \(\mathcal{M} \cong \mathcal{N} \) if and only if \(\dim \mathcal{M} = \dim \mathcal{N} \).
\end{theorem}
\begin{proof}
	Suppose \(\dim \mathcal{M} = \dim \mathcal{N} \) with \(A, B\) being the \hyperref[def:basis]{bases} for \(\mathcal{M} , \mathcal{N} \). Then, we have \(\vert A \vert = \vert B \vert \), so there exists a bijection \(f\colon A \to B\), which is a \hyperref[def:partial-elementary-map]{partial elementary map} since for \(f\), \(\overline{a} \in A\), \(\tp(\overline{a} ) = \tp(f(\overline{a} ))\) from \autoref{lma:lec19-1}.

	\begin{definition}[Paritla elementary map]\label{def:partial-elementary-map}
		A map \(g \colon U \subseteq M \to N\) is a \emph{partial elementary map} if \(\mathcal{M} \models \varphi (\overline{a} ) \iff \mathcal{N} \models \varphi (g(\overline{a} ))\) for \(\overline{a} \in \dom g\).\footnote{This implies injectivity since if \(\overline{a} \neq \overline{a} ^{\prime} \) with \(g(\overline{a} ) = g(\overline{a} ^{\prime} )\), there exists some \(\varphi \) differentiate them.}
	\end{definition}

	By \hyperref[thm:Zorn]{Zorn's lemma}, there exists a maximal \hyperref[def:partial-elementary-map]{partial elementary map} \(g\) from \(\mathcal{M} \to  \mathcal{N} \) extending \(f\).

	\begin{claim}
		\(g\) is an \hyperref[def:isomorphism]{isomorphism}, i.e., \(\dom g = M \) (\(\im g = N\) is automatic).
	\end{claim}
	\begin{explanation}
		Assume that \(c \in M - \dom g\). We know that \(c\in \acl(A) \subseteq \acl(\dom g)\) since \(A\) is a \hyperref[def:basis]{basis}, so let \(\varphi (x, \overline{d} )\) be the \hyperref[def:formula]{formula} from \autoref{lma:lec19-2} that isolates \(\tp(c / \dom g)\), i.e., whenever \(\mathcal{M} \models \varphi (c^{\prime} , \overline{d} )\), \(\tp(c / \dom g) = \tp(c ^{\prime} / \dom g)\). Then, \(\mathcal{M} \models \exists x\ \varphi (x, \overline{d} )\), implying \(\mathcal{N} \models \exists x\ \varphi (x, g(\overline{d} ))\). Let \(c^{\prime} \in N\) witness this. Then \(\tp^{\mathcal{M} } (c / \dom g) = \tp^{\mathcal{N} } (c^{\prime} / \im g)\) after identifying \(\dom g\) and \(\im g\).\footnote{Suppose \(\mathcal{M} \models \psi (c, \overline{b} )\) for \(\overline{b} \in \dom g\), then in \(\mathcal{M} \), \(\mathcal{M} \models \forall x\ \varphi (x, \overline{a} )\to \psi (x, \overline{b} )\). Since \(g\) is \hyperref[def:partial-elementary-map]{elementary}, \(\mathcal{N} \models \forall x\ \varphi (x, g(\overline{a} )) \to \psi (x, g(\overline{b} ))\), so \(\mathcal{N} \models \psi (c^{\prime} , \overline{b} )\).} Then we can define \(g(c) = c^{\prime} \) to extend \(g\) but remain \hyperref[def:partial-elementary-map]{partial elementary}, contradicting to the maximality of \(g\), so \(\dom g = M\).

		Following the same argument, we can show that \(\im g = N\), hence \(g\) is an \hyperref[def:isomorphism]{isomorphism}.
	\end{explanation}
	The other direction is trivial, since if \(\mathcal{M} \cong \mathcal{N} \), then clearly \(\dim \mathcal{M} = \dim \mathcal{N} \).
\end{proof}