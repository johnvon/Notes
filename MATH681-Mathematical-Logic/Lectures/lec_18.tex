\lecture{18}{14 Mar. 14:30}{Quantifier Elimination for Algebraically Closed Fields}
\begin{remark}
	We see that
	\[
		\varphi (\overline{x} ) \coloneqq \exists y\ p_n(\overline{x} ) y^n + \ldots + p_1(\overline{x} )y + p_0(\overline{x} ) = 0
	\]
	is equivalent to
	\[
		\psi (\overline{x} ) \coloneqq p_n(\overline{x} ) \neq 0 \lor \ldots \lor p_1(\overline{x} ) \neq 0 \lor p_0(\overline{x} ) = 0.
	\]
\end{remark}

\begin{problem}
What do we get from \hyperref[def:quantifier-elimination]{quantifier elimination}?
\end{problem}
\begin{answer}
	Understand the \hyperref[def:definable]{definable} sets, they are defined by \hyperref[not:quantifier-free]{quantifier-free} \hyperref[def:formula]{formulas}.
\end{answer}

\begin{definition}[Cofinite]\label{def:cofinite}
	A \emph{cofinite} subset of a set \(X\) is a subset \(A\) such that \(\vert A^{c} \vert < \infty \).
\end{definition}

\begin{proposition}\label{prop:lec18}
	The \hyperref[def:definable]{definable}\footnote{Using parameters.} subsets of an \hyperref[def:algebraically-closed]{algebraically closed} field \(K\) are exactly the finite and \hyperref[def:cofinite]{cofinite} sets.
\end{proposition}
\begin{proof}
	Let \(\left\{ a_1, \ldots , a_n \right\} \) be a finite set. This is \hyperref[def:definable]{definably} by \(x=a_1 \lor \ldots \lor x=a_n\) using \(a_1, \ldots , a_n\) as parameters. The complement of \(\left\{ a_1, \ldots , a_n  \right\} \) is \hyperref[def:definable]{definable} by \(x \neq a_1 \land \ldots \land x\neq a_n\).\footnote{We did not use anything about fields here.}

	Let \(X=\left\{ x\in K \mid K \models \varphi (x, \overline{a} ) \right\} \) be a \hyperref[def:definable]{definable} subset of \(K\). By \hyperref[def:quantifier-elimination]{quantifier elimination}, we may assume that \(\varphi \) is \hyperref[not:quantifier-free]{quantifier-free}, so \(\varphi \) is a boolean combination of \hyperref[not:atomic]{atomic} and negated \hyperref[not:atomic]{atomic} \hyperref[def:formula]{formulas}. Notice that an \hyperref[not:atomic]{atomic} \hyperref[def:formula]{formula} is of the form
	\[
		p_n(\overline{a} ) x^n + \ldots + p_1(\overline{a} ) x + p_0(\overline{a} ) = 0.
	\]
	Hence, this \hyperref[not:atomic]{atomic} \hyperref[def:formula]{formula} \hyperref[def:definable]{defines} either a finite set or all of \(K\). A negated \hyperref[not:atomic]{atomic} \hyperref[def:formula]{formula} \hyperref[def:definable]{defines} a \hyperref[def:cofinite]{cofinite} set or \(\varnothing \). Boolean combinations of finite and \hyperref[def:cofinite]{cofinite} sets are finite or \hyperref[def:cofinite]{cofinite}, so \(X\) is finite or \hyperref[def:cofinite]{cofinite}.
\end{proof}

\begin{remark}
	\autoref{prop:lec18} is not true for \(X \subseteq K^2\).
\end{remark}
\begin{explanation}
	\(X = \{ (x, y) \mid x^2 + y^2 + 1 \} \).
\end{explanation}

\begin{definition}[Strongly minimal]\label{def:strongly-minimal}
	A \hyperref[def:theory]{theory} \(T\) is \emph{strongly minimal} if for any \(\mathcal{M} \models T\), and \(X \subseteq M\) \hyperref[def:definable]{definable}, \(X\) is either finite or \hyperref[def:cofinite]{cofinite}.
\end{definition}

\begin{definition}[Algebraic]\label{def:algebraic-set}
	Let \(K\) be a field, and \(X \subseteq K^n\). We say that \(X\) is \emph{algebraic} if there is a set \(S\) of polynomials over \(K\) such that \(X\) is the zero set of \(S\).
\end{definition}

\begin{prev}
	Recall \autoref{def:algebraic} and compared it to the above.
\end{prev}

\begin{eg}
	\(X = \{(x, y) \mid x^2 + y^2 = 1\}\) is \hyperref[def:algebraic]{algebraic} since \(S = \{ x^2 + y^2 -1 \} \).
\end{eg}

The complement of an \hyperref[def:algebraic-set]{algebraic set} is usually not \hyperref[def:algebraic-set]{algebraic}.

\begin{definition}[Constructible]\label{def:constructible}
	The \emph{constructible} sets are the boolean combinations of \hyperref[def:algebraic-set]{algebraic set}.
\end{definition}

\begin{remark}
	The \hyperref[def:constructible]{constructible} sets are exactly the \hyperref[def:definable]{definable} sets in \(K \models \ACF\).
\end{remark}
\begin{explanation}
	The \hyperref[def:definable]{definable} sets, by \hyperref[def:quantifier-elimination]{quantifier elimination}, boolean combinations of sets \hyperref[def:definable]{defined} by \hyperref[not:atomic]{atomic} \hyperref[def:formula]{formulas}, which are \hyperref[def:algebraic-set]{algebraic}. So \hyperref[def:definable]{definable} implies \hyperref[def:constructible]{constructible}.

	On the other hand, it is enough to see that \hyperref[def:algebraic-set]{algebraic set} are \hyperref[def:definable]{definable}. The issue is that \(S \subseteq K[\overline{X} ]\) might be infinite. Let \(I\) be the \hyperref[def:ideal]{ideal} \hyperref[def:ideal-generation]{generated} by \(S\). Then the set \(X \subseteq K^n\) of common zeros of \(S\) is also the set of common zeros of \(I\).\footnote{Each \(f\in I\) is \(f = r_1 g_1 + \ldots + r_n g_n\), where \(g_1, \ldots , g_n \in S\).} By the \hyperref[thm:Hilbert-basis]{Hilbert's basis theorem}, \(I = (f_1, \ldots , f_m)\) is finitely \hyperref[def:ideal-generation]{generated}. Hence,
	\[
		X = \left\{ \overline{a} \in K^n \mid f_1(\overline{a} ) = 0 \land \ldots \land f_{m}(\overline{a} ) = 0  \right\},
	\]
	hence it's \hyperref[def:definable]{definable}.
\end{explanation}

\begin{theorem}[Chevalley's theorem]\label{def:Chevalley}
	Let \(K\) be an \hyperref[def:algebraically-closed]{algebraically closed} field. Let \(X \subseteq K^n\) be \hyperref[def:constructible]{constructible}. Let \(p \colon K^n \to K^m\) be a polynomial map, i.e., \(p(\overline{x} ) = (q_1(\overline{x} ), \ldots , q_n(\overline{x} ))\) for polynomials \(q_i\). Then, \(p(X)\), the image of \(X\) under \(p\), is also \hyperref[def:constructible]{constructible}.
\end{theorem}
\begin{proof}
	Since we know that \hyperref[def:constructible]{constructible} is the same as \hyperref[def:definable]{definable}, so
	\[
		p(X) = \left\{ \overline{y} \mid \exists \overline{x} \ (\overline{x} \in X \land p(\overline{x} ) = \overline{y} ) \right\}
	\]
	where for \(\overline{x} \in X\), there is a \hyperref[def:formula]{formula} expressing this, and for \(p(\overline{x} ) = \overline{y} \), \(y_1 = q_1(\overline{x} ) \land \ldots \land y_m=q_m(\overline{x} )\). Hence, \(p(X)\) is \hyperref[def:definable]{definable} (hence \hyperref[def:constructible]{constructible}), since \(X\) was.
\end{proof}

\begin{eg}
	\(p(x_1, x_2, x_3) = (x_1, x_3)\).
\end{eg}

\begin{theorem}[Weak Hilbert's Nullstellensatz]\label{thm:weak-Hilbert-Nullstellensatz}
	Let \(K\) be \hyperref[def:algebraically-closed]{algebraically closed}, and \(f_1, \ldots , f_n\in K[\overline{x} ]\). Then there is \(\overline{a} \in K^m\) sch that \(f_1(\overline{x} ) = \ldots = f_n(\overline{x} ) = 0\) if and only if \(1 \notin (f_1, \ldots , f_n)\), i.e., there are no \(r_1, \ldots , r_n\in K[\overline{x} ]\) such that \(1 = r_1 f_1 + \ldots + r_n f_n\).
\end{theorem}
\begin{proof}
	If \(1\in (f_1, \ldots , f_n)\), there are \(r_1, \ldots , r_n\in K[\overline{x} ]\) such that \(1 = r_1 f_1 + \ldots + r_n f_n\). If \(\overline{a} \) was a common zero of \(f_1, \ldots , f_n\), then
	\[
		1 = r_1(\overline{a} ) \cancel{f_1(\overline{a} )} + \ldots + r_n(\overline{a} ) \cancel{f_n(\overline{a} )} = 0,
	\]
	so no such \(\overline{a} \) exists.

	Now, suppose \(1 \notin (f_1, \ldots , f_n)\), so \((f_1, \ldots , f_n) \neq K[\overline{x} ]\). Let \(I\) be a \hyperref[def:maximal]{maximal} \hyperref[def:ideal]{ideal} containing \(f_1, \ldots , f_n\), and let \(L = \quotient{K[\overline{x} ]}{I} \), which is a \hyperref[def:field-extension]{field extension} of \(K\), \(K\hookrightarrow L\). There is \(\overline{a} \in L^m\) which is a common root of \(f_1, \ldots , f_n\), namely \(a_i = x_i + I\). Let \(M\) be the \hyperref[def:algebraic-closure]{algebraic closure} of \(L\), with \(\overline{a} \in M^n\). By \hyperref[def:quantifier-elimination]{quantifier elimination}, \(K \succeq M\). \(M \models \exists \overline{y} \ f_1(\overline{y} ) = 0 \land \ldots \land f_n(\overline{y} ) = 0\), which is a \hyperref[def:formula]{formula} about elements of \(K\) (the coefficients). Because \(K \succeq M\), \(K \models \exists \overline{y} \ f_1(\overline{y} ) = 0 \land \ldots \land f_n(\overline{y} ) = 0\), which says that \(f_1, \ldots , f_n\) have a common zero of \(K\).
\end{proof}

\begin{note}
	We leave the full version in the note, which relates to algebraic geometry (if you care).
\end{note}

\begin{remark}
	The \hyperref[thm:weak-Hilbert-Nullstellensatz]{weak Hilbert's Nullstellensatz} says that whether \(1\in (f_1, \ldots , f_n)\) is the only barrier for \(f_1, \ldots , f_n\) having a common zero.
\end{remark}

\begin{prev}[Definable closure]
	The \emph{definable closure} \(\mathop{\mathrm{dcl}}(A) \) of \(A\) is the set of all \(b\in M\) which are \hyperref[def:definable]{definable} over \(A\).
\end{prev}

On the homework, we looked at \(\mathop{\mathrm{dcl}}\), which is not really useful. In \hyperref[def:strongly-minimal]{strongly minimal} \hyperref[def:theory]{theories}, we look at \(\mathop{\mathrm{acl}} \) instead, which is very important.

\begin{definition}[(Model theoretical) algebraic closure]\label{def:model-algebraic-closure}
	Let \(\mathcal{M} \) be a \hyperref[def:structure]{structure} and \(A \subseteq M\). Then the \emph{(model-theoretic) algebraic closure} \(\mathop{\mathrm{acl}}(A) \) of \(A\) is the set of all \(a\in M\) such that there are \(\overline{b} \in A\) and a \hyperref[def:formula]{formula} \(\varphi (x, \overline{b} )\) such that \(\mathcal{M} \models \varphi (a, \overline{b} )\) and there are only finitely many other \(a^{\prime} \) with \(\mathcal{M} \models \varphi (a^{\prime} , \overline{b} )\).
\end{definition}

\begin{note}
	There are some properties that \(\mathop{\mathrm{acl}} \) always satisfies.
	\begin{itemize}
		\item \(\mathop{\mathrm{dcl}}(A) \subseteq \mathop{\mathrm{acl}}(A) \).
		\item \(A \subseteq \mathop{\mathrm{acl}}(A) \).
		\item If \(A \subseteq B\), then \(\mathop{\mathrm{acl}}(A) \subseteq \mathop{\mathrm{acl}}(B) \).
		\item \(\mathop{\mathrm{acl}}(A) = \mathop{\mathrm{acl}}(\mathop{\mathrm{acl}}(A)) \).\footnote{This is a good exercise!}
		\item If \(a\in \mathop{\mathrm{acl}}(A) \), then there is a finite \(F \subseteq A\) such that \(a\in \mathop{\mathrm{acl}}(F) \).
	\end{itemize}
\end{note}

On homework 5, we will show that in \hyperref[def:model]{models} of a \hyperref[def:strongly-minimal]{strongly minimal} \hyperref[def:theory]{theory}, \(\mathop{\mathrm{acl}}\) also satisfies the ``exchange property'':

\begin{remark}[Exchange property]
	If \(a\in \mathop{\mathrm{acl}}(X \cup \left\{ b \right\}) \) and \(a \notin \mathop{\mathrm{acl}}(X) \), then \(b\in \mathop{\mathrm{acl}}(X \cup \left\{ a \right\}) \). This makes \(\mathop{\mathrm{acl}} \) a \href{https://en.wikipedia.org/wiki/Pregeometry_(model_theory)}{\emph{pregeometry}} or \href{https://en.wikipedia.org/wiki/Matroid}{\emph{matroid}}.
\end{remark}

In \hyperref[def:algebraically-closed]{algebraically closed} fields, \(a\in \mathop{\mathrm{acl}}(X) \) just means that \(a\) is \hyperref[def:algebraic]{algebraic} over (the field generated by) \(X\) , i.e., there is \(p(y) \in F_X[\overline{x} ]\) such that \(p(a) = 0\). In vector spaces, \(a\in \mathop{\mathrm{acl}}(X) \) if and only if \(a\in \mathop{\mathrm{span}}(X) \).

\begin{intuition}
	This makes \hyperref[def:model]{model} has notions like dimensions, independence, etc.
\end{intuition}