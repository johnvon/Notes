\lecture{13}{16 Feb. 14:30}{Upward Löwenheim-Skolem Theorem}
\begin{corollary}[Leftschetz principle]\label{col:Leftschetz-principle}
	Let \(\mathcal{L} \) be the \hyperref[def:language]{language} of rings. For an \hyperref[def:sentence]{\(\mathcal{L}\)-sentence} \(\varphi \), the following are equivalent:
	\begin{enumerate}[(i)]
		\item \(\mathbb{C} \models \varphi \)
		\item every \hyperref[def:algebraically-closed]{algebraically closed} field of \hyperref[def:characteristic]{characteristic} \(0 \models \varphi \)
		\item some \hyperref[def:algebraically-closed]{algebraically closed} fields of \hyperref[def:characteristic]{characteristic} \(0 \models \varphi \)
		\item for all sufficient large positive \(p\), \(\varphi \) is \hyperref[def:truth]{true} in all \hyperref[def:algebraically-closed]{algebraically closed} fields of \hyperref[def:characteristic]{characteristic} \(p\)
		\item for infinitely many positive \(p\), \(\varphi \) is \hyperref[def:truth]{true} in all \hyperref[def:algebraically-closed]{algebraically closed} fields oof \hyperref[def:characteristic]{characteristic} \(p\)
	\end{enumerate}
\end{corollary}
\begin{proof}
	We only show the first three, others are left as homework. Let \(K \models \ACF_0\), then since it's \hyperref[def:theory-complete]{complete},
	\[
		K \models \varphi \iff \ACF_0 \models \varphi .
	\]
\end{proof}

\begin{theorem}[Ax-Grothendieck theorem]\label{thm:Ax-Grothendieck}
	Let \(f\colon \mathbb{C} ^n \to \mathbb{C} ^n\) be a polynomial map.\footnote{I.e., \(f(\overline{x} ) = (f_1(\overline{x} ), \ldots , f_n(\overline{x} ))\) where \(f_1, \ldots , f_n\) are polynomials.} If \(f\) is injective, then it's surjective. More generally, this is true for any \(K \models \ACF_p\) for any \(p\).
\end{theorem}
\begin{proof}
	The claim can be expressed by the \hyperref[def:sentence]{sentences},
	% e.g., for \(n = 3\) with quadratic polynomials,
	% \[
	% 	\begin{split}
	% 		\forall a_{1, 1, 1} \forall a_{1, 1, 2} \ldots \forall y_1 \forall y_2 \forall y_3 \exists x_1 \exists x_2 \exists x_3\
	% 		&\left( y_1 = a_{1, 1, 1}x_1 x_1 + a_{1, 1, 2} x_1 x_2 + \ldots + \land \right. \\
	% 		& y_2 = a_{2, 1, 1}x_1 x_1 + a_{2, 1, 2} x_1 x_2 + \ldots + \ldots \land \\
	% 		\left. \right)
	% 	\end{split}
	% \]
	so by \hyperref[col:Leftschetz-principle]{Leftschetz principle}, it's enough to prove that if for \(K = \overline{\mathbb{F}_p}\), for each \(p > 0\). Let \(f\colon \overline{\mathbb{F}}_p^n \to \overline{\mathbb{F}}_p^n\) be an injective polynomial map and \(\overline{y} \in \overline{\mathbb{F}}_p^n\). Then there is a finite subfield \(L \subseteq \overline{\mathbb{F}} _p \) which contains \(\overline{y}\) and the coefficients of \(f\).then, \(f\) restricts to an injective function \(L^n \to L^n\), which is surjective because \(L^n\) is finite, so \(\exists \overline{x} \in L^n\) such that \(f(\overline{x} ) = \overline{y} \).
\end{proof}

\section{Up and Down}
\begin{definition*}
	Let \(\mathcal{M} \) be an \hyperref[def:structure]{\(\mathcal{L}\)-structure}. Let \(\mathcal{L} _M \supseteq \mathcal{L} \)  be the expanded \hyperref[def:language]{language} with a new constant symbol \(\underline{a}\) for each \(a\in M\).

	\begin{definition}[Atomic diagram]\label{def:atomic-diagram}
		The \emph{atomic diagram} of \(\mathcal{M} \) is the \hyperref[def:theory]{\(\mathcal{L} _{M}\)-theory}
		\[
			\Diag(\mathcal{M} ) \coloneqq \left\{ \varphi (\underline{a}_1, \ldots , \underline{a}_n) \mid \mathcal{M} \models \varphi \text{ and \(\varphi \) is atomic or negated of atomic} \right\}.
		\]
	\end{definition}

	\begin{definition}[Elementary diagram]\label{def:elementary-diagram}
		The \emph{elementary diagram} of \(\mathcal{M} \) is the \hyperref[def:theory]{\(\mathcal{L} _{M}\)-theory}
		\[
			\Diag_{\mathrm{el}}(\mathcal{M} ) \coloneqq \left\{ \varphi (\underline{a}_1, \ldots , \underline{a}_n) \mid \mathcal{M} \models \varphi \text{ and \(\varphi \) an \hyperref[def:formula]{\(\mathcal{L}\)-formula}} \right\}.
		\]
	\end{definition}
\end{definition*}

\begin{note}
	There's a canonical way of \hyperref[not:expansion]{expanding} \(\mathcal{M} \) to an \hyperref[def:structure]{\(\mathcal{L} _M\)-structure} with \(\underline{a}^{\mathcal{M}} \coloneqq a\), i.e., we write \(a\) for both the symbol and the element.
\end{note}

\begin{lemma}\label{lma:lec13}
	Let \(\mathcal{N} \) be an \hyperref[def:structure]{\(\mathcal{L} _M\)-structure}.
	\begin{enumerate}[(a)]
		\item If \(\mathcal{N} \models \Diag(\mathcal{M} )\) then, viewing \(\mathcal{N} \) as an \hyperref[def:structure]{\(\mathcal{L} \)-structure}, there is an \hyperref[def:embedding]{embedding} \(f\colon \mathcal{M} \to \mathcal{N} \).
		\item If \(\mathcal{N} \models \Diag_{\mathrm{el} }(\mathcal{M} )\), then there is an \hyperref[def:elementary-embedding]{elementary \(\mathcal{L} \)-embedding} of \(\mathcal{M} \) into \(\mathcal{N} \).
	\end{enumerate}
\end{lemma}
\begin{proof}
	Take \(f(a) = \underline{a}^{\mathcal{N} } \), then \(\mathcal{N} \models \Diag(\mathcal{M} )\) means exactly that \(f\) is an \hyperref[def:embedding]{embedding}, and \(\mathcal{N} \models \Diag_{\mathrm{el} }(\mathcal{M} )\) means that \(f\) is an \hyperref[def:elementary-embedding]{elementary embedding}.
\end{proof}

\begin{theorem}[Upward Löwenheim-Skolem theorem]\label{thm:upward-Lowenheim-Skolem}
	Let \(\mathcal{M} \) be an infinite \hyperref[def:structure]{\(\mathcal{L} \)-structure} and let \(\kappa \) be an infinite cardinal \(\kappa \geq \vert \mathcal{M}  \vert + \vert \mathcal{L} \vert \). Then there is an \hyperref[def:structure]{\(\mathcal{L} \)-structure} \(\mathcal{N} \) of cardinality \(\kappa \) such that \(j\colon \mathcal{M} \to \mathcal{N} \) is \hyperref[def:elementary-embedding]{elementary}.
\end{theorem}
\begin{proof}
	\(\Diag_{\mathrm{el} }(\mathcal{M} )\) is \hyperref[def:satisfiable]{satisfiable} since \(\mathcal{M} \models \Diag_{\mathrm{el} }(\mathcal{M} )\), so by \autoref{prop:lec10}, it has a \hyperref[def:model]{model} \(\mathcal{N} \) of cardinality \(\kappa \geq \vert \mathcal{L} _M \vert \), and by \autoref{lma:lec13}, there is an \hyperref[def:elementary-embedding]{elementary embedding} \(\mathcal{M} \to \mathcal{N} \).
\end{proof}

\begin{proposition}[Tarski-Vaught Test]\label{prop:Tarski-Vaught-test}
	Let \(\mathcal{M} \) be a \hyperref[def:substructure]{substructure} of \(\mathcal{N} \). Then \(\mathcal{M} \) is an \hyperref[def:elementary-substructure]{elementary substructure} of \(\mathcal{N} \) if and only if for any \hyperref[def:formula]{formula} \(\varphi (x, \overline{y} )\) and \(\overline{a} \in M^n\), if there is \(b\in N\) such that \(\mathcal{N} \models \varphi (b, \overline{a} )\), then there is \(c\in M\) such that \(\mathcal{N} \models \varphi (c, \overline{a} )\).
\end{proposition}
\begin{proof}
	The forward direction follows from the fact that \(\mathcal{M} \) is an \hyperref[def:elementary-substructure]{elementary substructure}, so the \hyperref[def:truth]{truth} of \(\exists x\ \varphi (x, \overline{y} )\) is proved.

	For the backward direction, suppose the condition holds. We show that \(\mathcal{M} \models \varphi (\overline{a} )\iff \mathcal{N} \models \varphi (\overline{a} )\) by induction on \(\varphi \). Suppose the claim holds for \(\varphi , \psi \). Then,
	\[
		\mathcal{M} \models (\varphi \land \psi )(\overline{a} )
		\iff \mathcal{M} \models \varphi (\overline{a} ) \text{ and } \mathcal{M} \models \psi (\overline{a} )
		\iff \mathcal{N} \models \varphi (\overline{a} ) \text{ and } \mathcal{N} \models \psi (\overline{a} )
		\iff \mathcal{N} \models (\varphi \land \psi )(\overline{a} ).
	\]
	Finally, suppose the claim holds for \(\varphi (x, \overline{y} )\), then
	\[
		\mathcal{M} \models \exists x\ \varphi (x, \overline{a} )
		\iff \exists b\in M\ \mathcal{M} \models \varphi (b, \overline{a} )
		\iff \exists b\in M\ \mathcal{N} \models \varphi (b, \overline{a} )
	\]
	by induction hypotheses. Conversely, \(\mathcal{N} \models \exists x\ \varphi (x, \overline{a} )\), then \(\exists b\in N\) such that \(\mathcal{N} \models \varphi (b, \overline{a} )\) by the condition from the statement, so \(\exists c\in M\) such that \(\mathcal{N} \models \varphi (c, \overline{a} )\). By the induction hypotheses, we further have \(\mathcal{M} \models \varphi (c, \overline{a} )\), hence \(\mathcal{M} \models \exists x\ \varphi (x, \overline{a} )\).
\end{proof}

\begin{eg}
	The ring \(\mathbb{Z} \) is a \hyperref[def:substructure]{substructure} of \(\mathbb{Q} \), but \(\mathbb{Q} \models \exists x\ (x+x=1)\) while \(\mathbb{Z} \not \models \exists x\ (x+x=1)\).
\end{eg}