\lecture{13}{16 Feb. 14:30}{Upward Löwenheim-Skolem Theorem}
Another consequence of \hyperref[def:theory-complete]{completeness} is that since \(\mathbb{C} \models \ACF_0\), if \(K\) is any \hyperref[def:algebraically-closed]{algebraically closed} field of \hyperref[def:characteristic]{characteristic} \(0\), \(\mathbb{C} \equiv K\), i.e., we have the following.

\begin{remark}
	The \hyperref[def:sentence]{sentences} \hyperref[def:truth]{true} of \(\mathbb{C} \) are exactly the same as the \hyperref[def:sentence]{sentences} \hyperref[def:truth]{true} of any \hyperref[def:algebraically-closed]{algebraically closed} field.
\end{remark}

Essentially, the idea is that if one proves an algebraic statement about the complex numbers by analytic techniques of complex analysis, then there will be a \hyperref[def:proof]{proof} of the same algebraic statement using purely algebraic tools, which works in any \hyperref[def:algebraically-closed]{algebraically closed} field of \hyperref[def:characteristic]{characteristic} \(0\). The \hyperref[thm:compactness]{compactness theorem} also gives connections to fields of finite \hyperref[def:characteristic]{characteristic}.

\begin{theorem}[Leftschetz principle]\label{thm:Leftschetz-principle}
	Let \(\mathcal{L} \) be the \hyperref[def:language]{language} of rings. For an \hyperref[def:sentence]{\(\mathcal{L}\)-sentence} \(\varphi \), the following are equivalent:
	\begin{enumerate}[(i)]
		\item \(\varphi \) is \hyperref[def:truth]{true} in \(\mathbb{C}\);
		\item \(\varphi \) is \hyperref[def:truth]{true} in every \hyperref[def:algebraically-closed]{algebraically closed} field of \hyperref[def:characteristic]{characteristic} \(0\);
		\item \(\varphi \) is \hyperref[def:truth]{true} in some \hyperref[def:algebraically-closed]{algebraically closed} fields of \hyperref[def:characteristic]{characteristic} \(0\);
		\item there is a number \(n\) such that \(\varphi \) is \hyperref[def:truth]{true} in all \hyperref[def:algebraically-closed]{algebraically closed} fields of \hyperref[def:characteristic]{characteristic} \(p > n\);
		\item for each number \(n\), \(\varphi \) is \hyperref[def:truth]{true} in all \hyperref[def:algebraically-closed]{algebraically closed} fields of \hyperref[def:characteristic]{characteristic} \(p>n\).
	\end{enumerate}
\end{theorem}
\begin{proof}
	Let \(K \models \ACF_0\), then since it's \hyperref[def:theory-complete]{complete}, we know that \(K \models \varphi \iff \ACF_0 \models \varphi\), which proves the first three. Others are left as homework.
\end{proof}

We can use the \hyperref[thm:Leftschetz-principle]{Leftschetz principle} to prove the following.

\begin{theorem}[Ax-Grothendieck theorem]\label{thm:Ax-Grothendieck}
	Let \(f\colon \mathbb{C} ^n \to \mathbb{C} ^n\) be a polynomial map.\footnote{I.e., \(f(\overline{x} ) = (f_1(\overline{x} ), \ldots , f_n(\overline{x} ))\) where \(f_1, \ldots , f_n\) are polynomials.} If \(f\) is injective, then it's surjective. More generally, this is true for any \(K \models \ACF_p\) for any \(p\).
\end{theorem}
\begin{proof}
	The claim can be expressed by the \hyperref[def:sentence]{sentences}, so by \hyperref[thm:Leftschetz-principle]{Leftschetz principle}, it's enough to prove that if for \(K = \overline{\mathbb{F}_p}\), for each \(p > 0\).

	Let \(f\colon \overline{\mathbb{F}}_p^n \to \overline{\mathbb{F}}_p^n\) be an injective polynomial map and \(\overline{y} \in \overline{\mathbb{F}}_p^n\). Then there is a finite subfield \(L \subseteq \overline{\mathbb{F}} _p \) which contains \(\overline{y}\) and the coefficients of \(f\). Then, \(f\) restricts to an injective function \(L^n \to L^n\), which is surjective because \(L^n\) is finite, so \(\exists \overline{x} \in L^n\) such that \(f(\overline{x} ) = \overline{y} \).
\end{proof}

\section{Up and Down}
\begin{definition*}
	Let \(\mathcal{M} \) be an \hyperref[def:structure]{\(\mathcal{L}\)-structure}. Let \(\mathcal{L} _M \supseteq \mathcal{L} \)  be the expanded \hyperref[def:language]{language} with a new constant symbol \(\underline{a}\) for each \(a\in M\).

	\begin{definition}[Atomic diagram]\label{def:atomic-diagram}
		The \emph{atomic diagram} of \(\mathcal{M} \) is the \hyperref[def:theory]{\(\mathcal{L} _{M}\)-theory}
		\[
			\Diag(\mathcal{M} ) \coloneqq \left\{ \varphi (\underline{a}_1, \ldots , \underline{a}_n) \mid \mathcal{M} \models \varphi (m_1, \ldots , m_n) \text{ and \(\varphi \) is \hyperref[not:atomic]{atomic} or negated of \hyperref[not:atomic]{atomic}} \right\}.
		\]
	\end{definition}

	\begin{definition}[Elementary diagram]\label{def:elementary-diagram}
		The \emph{elementary diagram} of \(\mathcal{M} \) is the \hyperref[def:theory]{\(\mathcal{L} _{M}\)-theory}
		\[
			\Diag_{\mathrm{el}}(\mathcal{M} ) \coloneqq \left\{ \varphi (\underline{a}_1, \ldots , \underline{a}_n) \mid \mathcal{M} \models \varphi (m_1, \ldots , m_n) \text{ and \(\varphi \) an \hyperref[def:formula]{\(\mathcal{L}\)-formula}} \right\}.
		\]
	\end{definition}
\end{definition*}

\begin{intuition}
	Basically both \(\Diag(\mathcal{M} )\) and \(\Diag_{\mathrm{el} }(\mathcal{M} )\) contain the information about the \hyperref[def:structure]{structure} but in the form of a \hyperref[def:theory]{theory}.
\end{intuition}

\begin{notation}
	There's a canonical way of \hyperref[not:expansion]{expanding} \(\mathcal{M} \) to an \hyperref[def:structure]{\(\mathcal{L} _M\)-structure} with \(\underline{a}^{\mathcal{M}} \coloneqq a\), i.e., we write \(a\) for both the symbol and the element.
\end{notation}

\begin{lemma}\label{lma:lec13}
	Let \(\mathcal{N} \) be an \hyperref[def:structure]{\(\mathcal{L} _M\)-structure}.
	\begin{enumerate}[(a)]
		\item If \(\mathcal{N} \models \Diag(\mathcal{M} )\) then, viewing \(\mathcal{N} \) as an \hyperref[def:structure]{\(\mathcal{L} \)-structure}, there is an \hyperref[def:embedding]{embedding} \(f\colon \mathcal{M} \to \mathcal{N} \).
		\item If \(\mathcal{N} \models \Diag_{\mathrm{el} }(\mathcal{M} )\), then there is an \hyperref[def:elementary-embedding]{elementary \(\mathcal{L} \)-embedding} of \(\mathcal{M} \) into \(\mathcal{N} \).
	\end{enumerate}
\end{lemma}
\begin{proof}
	Take \(f(a) = \underline{a}^{\mathcal{N} } \), then \(\mathcal{N} \models \Diag(\mathcal{M} )\) means exactly that \(f\) is an \hyperref[def:embedding]{embedding}, and \(\mathcal{N} \models \Diag_{\mathrm{el} }(\mathcal{M} )\) means that \(f\) is an \hyperref[def:elementary-embedding]{elementary embedding}.
\end{proof}

\begin{theorem}[Upward Löwenheim-Skolem theorem]\label{thm:upward-Lowenheim-Skolem}
	Let \(\mathcal{M} \) be an infinite \hyperref[def:structure]{\(\mathcal{L} \)-structure} and let \(\kappa \) be an infinite cardinal \(\kappa \geq \vert \mathcal{M}  \vert + \vert \mathcal{L} \vert \). Then there is an \hyperref[def:structure]{\(\mathcal{L} \)-structure} \(\mathcal{N} \) of cardinality \(\kappa \) such that \(j\colon \mathcal{M} \to \mathcal{N} \) is \hyperref[def:elementary-embedding]{elementary}.
\end{theorem}
\begin{proof}
	\(\Diag_{\mathrm{el} }(\mathcal{M} )\) is \hyperref[def:satisfiable]{satisfiable} since \(\mathcal{M} \models \Diag_{\mathrm{el} }(\mathcal{M} )\), by \autoref{prop:lec10} it has a \hyperref[def:model]{model} \(\mathcal{N} \) of cardinality \(\kappa \geq \vert \mathcal{L} _M \vert = \vert \mathcal{M} \vert + \vert \mathcal{L} \vert \), so an \hyperref[def:elementary-embedding]{elementary embedding} exists \(\mathcal{M} \to \mathcal{N} \) by \autoref{lma:lec13}.
\end{proof}

\begin{intuition}
	The \hyperref[thm:upward-Lowenheim-Skolem]{upward Löwenheim-Skolem theorem} says that every \hyperref[def:structure]{structure} is an \hyperref[def:elementary-substructure]{elementary substructure} of many much bigger \hyperref[def:structure]{structures}.
\end{intuition}

\begin{prev}
	Our very first application of the \hyperref[thm:compactness]{compactness theorem}: in \hyperref[eg:construction-non-standard-model-of-arithmetic]{the construction of the non-standard model of arithmetic}, we built \(\mathcal{N} \models \mathop{\mathrm{Th}}(\mathbb{N} ) \) not \hyperref[def:isomorphism]{isomorphic} to \(\mathbb{N} \), which is exactly like this. Every element of \(\mathbb{N} \) can already be expressed as a \hyperref[def:term]{term} of the form \(1+\ldots +1\) without adding any new constants.
\end{prev}

A ``downward'' version of the \hyperref[thm:upward-Lowenheim-Skolem]{upward Löwenheim-Skolem theorem} also exists, which says that big \hyperref[def:model]{models} contain smaller \hyperref[def:elementary-substructure]{elementary substructures}. this will take some more work to prove, so we first need a test for this.

\begin{proposition}[Tarski-Vaught test]\label{prop:Tarski-Vaught-test}
	Let \(\mathcal{M} \) be a \hyperref[def:substructure]{substructure} of \(\mathcal{N} \). Then \(\mathcal{M} \) is an \hyperref[def:elementary-substructure]{elementary substructure} of \(\mathcal{N} \) if and only if for any \hyperref[def:formula]{formula} \(\varphi (x, \overline{y} )\) and \(\overline{a} \in M^n\), if there is \(b\in N\) such that \(\mathcal{N} \models \varphi (b, \overline{a} )\), then there is \(c\in M\) such that \(\mathcal{N} \models \varphi (c, \overline{a} )\).
\end{proposition}
\begin{proof}
	The forward direction follows from the fact that \(\mathcal{M} \) is an \hyperref[def:elementary-substructure]{elementary substructure}, so the \hyperref[def:truth]{truth} of \(\exists x\ \varphi (x, \overline{y} )\) is proved since \(\mathcal{M} \models \exists x\ \varphi (x, \overline{a} )\) if and only if \(\mathcal{N} \models \exists x\ \varphi (x, \overline{a} )\).

	For the backward direction, suppose the condition holds. We show that \(\mathcal{M} \models \varphi (\overline{a} )\iff \mathcal{N} \models \varphi (\overline{a} )\) by induction on \(\varphi \). Since \(\mathcal{M} \) is a \hyperref[def:substructure]{substructure} of \(\mathcal{N} \), this is true for all \hyperref[not:quantifier-free]{quantifier-free} \hyperref[def:formula]{formulas}, and in particular for the \hyperref[not:atomic]{atomic} \hyperref[def:formula]{formulas}.

	Suppose that the claim is true for \(\psi \), then
	\[
		\mathcal{M} \models \lnot \psi (\overline{a} )
		\iff \mathcal{M} \not \models \psi (\overline{a} )
		\iff \mathcal{N} \not \models \psi (\overline{a} )
		\iff \mathcal{N} \models \lnot \psi (\overline{a} ),
	\]
	so the claim is also true for \(\lnot \psi \). Similarly, suppose the claim holds for \(\varphi , \psi \). Then,
	\[
		\mathcal{M} \models (\varphi \land \psi )(\overline{a} )
		\iff \mathcal{M} \models \varphi (\overline{a} ) \text{ and } \mathcal{M} \models \psi (\overline{a} )
		\iff \mathcal{N} \models \varphi (\overline{a} ) \text{ and } \mathcal{N} \models \psi (\overline{a} )
		\iff \mathcal{N} \models (\varphi \land \psi )(\overline{a} ).
	\]
	Finally, suppose the claim holds for \(\varphi (x, \overline{y} )\), then
	\[
		\mathcal{M} \models \exists x\ \varphi (x, \overline{a} )
		\implies \exists b\in M \ \mathcal{M} \models \varphi (b, \overline{a} )
		\implies \exists b\in M \ \mathcal{N} \models \varphi (b, \overline{a} )
		\implies \mathcal{N} \models \exists x\ \varphi (x, \overline{a} ),
	\]
	by induction hypotheses. Conversely, \(\mathcal{N} \models \exists x\ \varphi (x, \overline{a} )\), then \(\exists b\in N\) such that \(\mathcal{N} \models \varphi (b, \overline{a} )\) by the condition from the statement, so \(\exists c\in M\) such that \(\mathcal{N} \models \varphi (c, \overline{a} )\). By the induction hypotheses, we further have \(\mathcal{M} \models \varphi (c, \overline{a} )\), hence \(\mathcal{M} \models \exists x\ \varphi (x, \overline{a} )\).
\end{proof}

\begin{eg}
	The ring \(\mathbb{Z} \) is a \hyperref[def:substructure]{substructure} of \(\mathbb{Q} \), but \(\mathbb{Q} \models \exists x\ (x+x=1)\) while \(\mathbb{Z} \not \models \exists x\ (x+x=1)\).
\end{eg}