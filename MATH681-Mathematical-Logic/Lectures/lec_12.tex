\lecture{12}{14 Feb. 14:30}{The \(\ACF\) Theory and Categorical}
\begin{definition}[Characteristic]\label{def:characteristic}
	A field \(F\) has finite \emph{characteristic} \(p > 0\) if \(\underbrace{1 + \ldots + 1}_{\text{\(p\) times}} = 0\).
\end{definition}

\begin{remark}
	\(p\) is always prime, otherwise, \(F\) has \hyperref[def:characteristic]{characteristic} \(p = 0\), i.e., \(1 + \ldots + 1 \neq 0\), always.
\end{remark}

It's now easy to come up with the following notion naturally.

\begin{definition}[Prime field]\label{def:prime-field}
	The \emph{prime field} \(\mathbb{F} _p\) in \hyperref[def:characteristic]{characteristic} \(p\) such that \(\mathbb{F} _p = \mathbb{Q} \) if \(p = 0\), \(\mathbb{F} _p = \quotient{\mathbb{Z} }{p\mathbb{Z} } \) if \(p > 0\).
\end{definition}

\begin{definition}[Transcendence basis]\label{def:transcendence-basis}
	Let \(\quotient{L}{K} \) be a \hyperref[def:field-extension]{field extension}. A set \(S \subseteq L\) is called a \emph{transcendence basis} of \(\quotient{L}{K} \) if \(S\) is algebraically independent\footnote{No \(a_1, \ldots , a_n\in S\) have non-zero polynomial \(f(x_1, \ldots , x_n) \in K[\overline{x} ]\) with \(f(a_1, \ldots , a_n) = 0\).} and \(L\) is an \hyperref[def:algebraic-extension]{algebraic extension} of \(K(S)\), i.e., \(S\) is maximal.
\end{definition}

\begin{remark}
	Every \hyperref[def:field-extension]{field extension} has a \hyperref[def:transcendence-basis]{transcendence basis}, and any two \hyperref[def:transcendence-basis]{transcendence basis} have the same size.
\end{remark}
\begin{explanation}
	On a combinatorial level, this is exactly the same as the proof that any two bases for a vector space have the same cardinality.
\end{explanation}

\begin{eg}
	Let \(K(t_1, \ldots , t_n)\) be the fraction field of \(K[x_1, \ldots , x_n]\), then \(\left\{ t_1, \ldots , t_n \right\} \) is a \hyperref[def:transcendence-basis]{transcendence basis} for \(K(t_1, \ldots , t_n)\) over \(K\).
\end{eg}

\begin{definition}[Transcendence degree]\label{def:transcendence-degree}
	The \emph{transcendence degree} of \(L\) over \(K\) is the cardinality of any \hyperref[def:transcendence-basis]{transcendence basis}.
\end{definition}

If we do not specify \(K\), then \(K\) is the \hyperref[def:prime-field]{prime field} \(K = \mathbb{F} _p\).

\begin{theorem}\label{thm:same-transcendence-degree-isomorphic}
	Any two \hyperref[def:algebraically-closed]{algebraically closed} fields of the same \hyperref[def:characteristic]{characteristic} \(p\) and \hyperref[def:transcendence-degree]{transcendence degree} are isomorphic.
\end{theorem}
\begin{proof}
	Let \(L, K\) be those fields, with \hyperref[def:transcendence-basis]{transcendence basis} \(S, T\) over \(\mathbb{F} _p\) with \(\vert S \vert = \vert T \vert \). \(L\) is the \hyperref[def:algebraic-closure]{algebraic closure} of \(\mathbb{F} _p(S)\) and \(K\) is the \hyperref[def:algebraic-closure]{algebraic closure} of \(\mathbb{F} _p(T)\). There is a bijection \(f\colon S\to T\), and then \(f\) extends to \(\overline{f} \colon \mathbb{F} _p(S) \to \mathbb{F} _p(T)\) such that
	\[
		\overline{f} \left( \frac{\sum_{\alpha } r_\alpha \overline{x} ^\alpha }{\sum_{\alpha }s_\alpha \overline{x} ^\alpha  } \right)
		= \frac{\sum_{\alpha } r_\alpha f(\overline{x} )^\alpha }{\sum_{\alpha } s_\alpha f(\overline{x} )^\alpha },
	\]
	where \(r_\alpha , s_\alpha \in \mathbb{F} _p\) and \(\overline{x} ^\alpha \) is some monomial from \(S\), e.g., \(x_1^2 x_2\) for \(x_1, x_2\in S\).\footnote{\(\alpha \) can be thought as a tuple, in the case of \(x_1^2 x_2\), \(\alpha = (2, 1)\).}

	\(\mathbb{F} _p(S)\) and \(\mathbb{F} _p(T)\) are the same (up to isomorphism), but the \hyperref[def:algebraic-closure]{algebraic closures} are unique from \autoref{thm:algebraic-closures-isomorphism}, so \(K \cong L\) via an isomorphism extending \(\overline{f} \).
\end{proof}

The above proof actually shows more.

\begin{corollary}
	If \(\quotient{L}{K} \) and \(\quotient{M}{K} \) are \hyperref[def:field-extension]{field extensions} with \hyperref[def:transcendence-basis]{transcendence bases} \(S\) and \(T\), and \(\alpha \colon S \to T\) is a bijection, then \(\alpha \) extends to an isomorphism \(L \cong _K M\).
\end{corollary}

If we apply this inside a single \hyperref[def:algebraically-closed]{algebraically closed} field, we have the following.

\begin{theorem}
	Let \(K\) be the \hyperref[def:algebraic-closure]{algebraic closure} of \(k\), and \(L, M\) be subfields of \(K\) which \hyperref[def:field-extension]{extend} \(k\). Suppose that \(\alpha \colon M \to L\) is an isomorphism fixing \(k\), then \(\alpha \) extends to an automorphism of \(K\).
\end{theorem}

\subsection{The \(\ACF\) Theory}
Finally, we are ready to introduce the \hyperref[def:theory]{theory} we're going to study, which is called \(\ACF\). It turns out that the \hyperref[def:model]{models} of which are exactly the \hyperref[def:algebraically-closed]{algebraically closed} fields with nice properties we're going to discuss.

\begin{definition}[\(\ACF\)]\label{def:ACF}
	\(\ACF\) is the \hyperref[def:theory]{theory} of \hyperref[def:algebraically-closed]{algebraically closed} fields consists of field axioms and \hyperref[def:formula]{formulas} that for every \(n \geq 1\),
	\[
		\forall a_0 \ldots \forall a_n \left( a_n \neq 0 \to \exists b\ a_n b^n + a_{n-1} b^{n-1} + \ldots + a_0 = 0\right) .
	\]
\end{definition}

\begin{remark}
	The \hyperref[def:model]{models} of \hyperref[def:ACF]{ACF} are exactly the \hyperref[def:algebraically-closed]{algebraically closed} fields, and the \hyperref[def:language]{language} \(\mathcal{L} = \mathcal{L} _{\text{ring} } = \left\{ 0, 1, +, -, \cdot \right\} \).
\end{remark}

\begin{notation}[\(\ACF_p\)]
	For a prime \(p > 0\), let \(\ACF_p \coloneqq \ACF \cup \{\underbrace{1+\ldots +1}_{p} = 0\}\).
\end{notation}

\begin{notation}[\(\ACF_0\)]
	Let \(\ACF_0 \coloneqq \ACF \cup \{\underbrace{1+\ldots +1}_n \neq 0 \mid n\in \mathbb{N} \}\).
\end{notation}

\begin{definition}[Categorical]\label{def:categorical}
	Let \(\kappa \) be an infinite cardinal and \(T\) be an \hyperref[def:theory]{\(\mathcal{L} \)-theory}. \(T\) is \emph{\(\kappa \)-categorical} if any \(\mathcal{M} , \mathcal{N} \models T\) of size \(\kappa \) have \(\mathcal{M} \cong \mathcal{N} \).

	\begin{definition}[Countably categorical]\label{def:countably-categorical}
		If \(\kappa \) is countable, then \(T\) is \emph{countably categorical}.
	\end{definition}
	\begin{definition}[Uncountably categorical]\label{def:uncountably-categorical}
		If \(\kappa \) is uncountable, then \(T\) is \emph{uncountably categorical}.
	\end{definition}
\end{definition}

We see that for being \hyperref[def:uncountably-categorical]{uncountably categorical}, we only need one uncountable \(\kappa \).

\begin{eg}
	\((\mathbb{Q} , \leq )\) is \hyperref[def:countably-categorical]{countably categorical}.
\end{eg}

\begin{lemma}\label{lma:lec12}
	If \(K\) has \hyperref[def:transcendence-degree]{transcendence degree} \(\lambda \), then \(\vert K \vert = \lambda + \aleph_0\).
\end{lemma}
\begin{proof}
	\(K\) is \hyperref[def:algebraic]{algebraic} over \(\mathbb{F} _p(S)\), where \(S\) is a \hyperref[def:transcendence-basis]{transcendence basis} of size \(\lambda \). By counting, \(\vert \mathbb{F} _p(S) \vert = \lambda + \aleph_0\), so \(\vert \mathbb{F} _p(S)[x] \vert = \lambda + \aleph_0\). But since each element of \(K\) satisfies some polynomials, and each polynomial has finitely many roots in \(K\), so \(\vert K \vert = \lambda +\aleph_0\).
\end{proof}

\begin{theorem}
	Fix \(p\). \(\ACF_p \) is \hyperref[def:categorical]{\(\kappa \)-categorical} for every uncountable \(\kappa \).
\end{theorem}
\begin{proof}
	Let \(L, K\) be \(\ACF_p \) for size \(\kappa \), then \(L, K\) have \hyperref[def:transcendence-degree]{transcendence degree} \(\kappa \), and hence are isomorphic from \autoref{thm:same-transcendence-degree-isomorphic}. With the application of \autoref{lma:lec12}, we're done.
\end{proof}

\begin{eg}
	\(\mathbb{Q} ^\text{alg} \), the \hyperref[def:algebraic-closure]{algebraic closure} of \(\mathbb{Q} \), has size \(\aleph_0\), and has \hyperref[def:transcendence-degree]{transcendence degree} is \(0\).
\end{eg}

\begin{eg}
	\(\mathbb{Q} (t)^\text{alg} \), the \hyperref[def:algebraic-closure]{algebraic closure} of \(\mathbb{Q} (t) \cong \mathbb{Q} (\pi )\), has size \(\aleph_0\), and has \hyperref[def:transcendence-degree]{transcendence degree} is \(1\).
\end{eg}
\begin{explanation}
	We see that
	\[
		\mathbb{Q} (t)^{\text{alg} } = \left\{ z\in \mathbb{C} \mid \text{\(z\) is \hyperref[def:algebraic]{algebraic} over \(\mathbb{Q} (\pi )\) }  \right\}.
	\]
	These are countable, but not isomorphic. \(\ACF_0\) is not \hyperref[def:countably-categorical]{countably categorical}. The same with \(\ACF_0\) for \(p>0\).
\end{explanation}

\begin{note}
	\(\ACF\) is not \hyperref[def:uncountably-categorical]{uncountably categorical}.
\end{note}

\begin{theorem}[Vaught's test]\label{thm:Vaught-test}
	Let \(T\) be a \hyperref[def:satisfiable]{satisfiable} \hyperref[def:theory]{\(\mathcal{L} \)-theory} with no finite \hyperref[def:model]{models}. If \(T\) is \hyperref[def:categorical]{\(\kappa \)-categorical} for some infinite \(\kappa \geq \vert \mathcal{L}  \vert \), then \(T\) is \hyperref[def:theory-complete]{complete}.
\end{theorem}
\begin{proof}
	Suppose \(T\) was not \hyperref[def:theory-complete]{complete}, so pick \(\varphi \) with \(T \not \models \varphi \) and \(T \not \models \lnot \varphi \), and hence \(T \cup \left\{ \varphi  \right\} \) and \(T \cup \left\{ \lnot \varphi  \right\} \) are \hyperref[def:satisfiable]{satisfiable}. By a consequence of the proof of \hyperref[thm:completeness]{completeness theorem} (with a \hyperref[thm:compactness]{compactness} argument),
	\begin{itemize}
		\item \(T \cup \left\{ \varphi  \right\} \) has a \hyperref[def:model]{model} \(\mathcal{M} \) of size \(\kappa \), and
		\item \(T \cup \left\{ \lnot \varphi  \right\} \) has a \hyperref[def:model]{model} \(\mathcal{N} \) of size \(\kappa \).
	\end{itemize}
	But \(T\) is \hyperref[def:categorical]{\(\kappa \)-categorical}, so \(\mathcal{M} \cong \mathcal{N} \), which is a contradiction \(\conta\)
\end{proof}

\begin{corollary}
	\(\ACF_p\) is \hyperref[def:theory-complete]{complete} for each \(p\).
\end{corollary}

The axioms for \(\ACF_p \) completely determines all first-order facts about \hyperref[def:algebraically-closed]{algebraically closed} fields of \hyperref[def:characteristic]{characteristic} \(p\).

\begin{remark}[Fact]
	The axioms for \(\ACF \) or \(\ACF_p \) can be \hyperref[def:computably-enumerable]{listed computably}. So \(\left\{ \varphi \mid \ACF \models \varphi \right\} \) and \(\left\{ \varphi \mid \ACF_p \models \varphi \right\} \) can be \hyperref[def:computably-enumerable]{listed computably}.
\end{remark}

\begin{definition}[Decidable]\label{def:decidable}
	A \hyperref[def:theory]{theory} \(T\) is \emph{decidable} if there is a program that given \(\varphi \), it determines whether \(T \models \varphi \).
\end{definition}

\begin{remark}
	\(\ACF_p \) is \hyperref[def:decidable]{decidable}.
\end{remark}
\begin{explanation}
	Given \(\varphi \), either \(\ACF_p \models \varphi \) or \(\ACF_p \models \lnot \varphi \) since \(\ACF_p \) is \hyperref[def:theory-complete]{complete}. By looking for a \hyperref[def:proof]{proof} of \(\varphi \) and a \hyperref[def:proof]{proof} of \(\lnot \varphi \), eventually we will find one, telling us whether \(\ACF_p \models \varphi \).
\end{explanation}

\begin{theorem}
	\(\ACF \) is \hyperref[def:decidable]{decidable}.
\end{theorem}
\begin{proof}
	Given \(\varphi \), simultaneously
	\begin{enumerate}[(a)]
		\item Look for a \hyperref[def:proof]{proof} of \(\ACF \vdash \varphi \), and
		\item Look for \(p\) such that \(\ACF_p \vdash \lnot \varphi \) (so \(\ACF \not \models \varphi \)).\footnote{We don't know \(\ACF \models \lnot \varphi \).}
	\end{enumerate}
	The first case is fine. Suppose \(\ACF \not \models \varphi \), so there is \(\mathcal{M} \models \ACF\), \(\mathrm{\mathcal{M} \models \lnot \varphi } \). There is \(p\) such that \(\mathcal{M} \models \ACF_p\). Since \(\ACF_p\) is \hyperref[def:theory-complete]{complete}, \(\ACF_p \models \lnot \varphi \), so the search of the second case will half, so the whole search will eventually halt.
\end{proof}