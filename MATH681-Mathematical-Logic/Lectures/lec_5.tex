\lecture{5}{19 Jan. 14:30}{Hilbert-Style Deductive System}
We start by asking whether \(\mathbb{R} \) is \hyperref[def:definable]{definable} in \(\mathbb{C} = (\mathbb{C} , 0, 1, +, \cdot, -)\)?

\begin{proposition}
	Let \(\mathcal{M} \) be an \hyperref[def:structure]{\(\mathcal{L} \)-structure}, and let \(X \subseteq M^n\) be a set which is \hyperref[def:definable]{definable} over \(\overline{a} \). Then any \hyperref[def:automorphism]{automorphism} of \(\mathcal{M} \) that fixes \(\overline{a} \) pointwise\footnote{If \(\overline{a} =(a_1, \dots , a_m)\), then \(f(a_i) = a_i\).} fixes \(X\) setwise.\footnote{If \(b\in X\), then \(f(b)\in X\).}
\end{proposition}
\begin{proof}
	Let \(f\) be an \hyperref[def:automorphism]{automorphism} of \(\mathcal{M} \) fixing \(\overline{a} \) pointwise, and \(X = \left\{ \overline{b} \in M^n\colon \mathcal{M} \models \varphi (\overline{b}, \overline{a} ) \right\} \). Fix \(\overline{b} \), and suppose \(\overline{b} \in X\), so \(\mathcal{M} \models \varphi (\overline{b} , \overline{a} )\). Because \(f\) is an \hyperref[def:elementary-embedding]{elementary embedding} from \autoref{prop:isomorphism-is-elementary-embedding},
	\[
		\mathcal{M} \models \varphi (f(\overline{b} ), f(\overline{a} ))
		\implies \mathcal{M} \models \varphi (f(\overline{b} ), \overline{a} ),
	\]
	hence \(f(\overline{b} )\in X\). Similarly, if \(\overline{b} \notin X\), \(\mathcal{M} \models \lnot \varphi (\overline{b} , \overline{a} ) \implies \mathcal{M} \models \lnot \varphi (f(\overline{b} , \overline{a} ))\), so \(f(\overline{b} ) \notin X\).
\end{proof}

\begin{remark}
	If \(X\) is \hyperref[def:definable]{\(\varnothing \)-definable}, it is fixed setwise by any \hyperref[def:automorphism]{automorphism}.
\end{remark}

\begin{eg}
	\(\mathbb{N} \) is fixed setwise by any \hyperref[def:automorphism]{automorphism} of the ring \(\mathbb{Z} \). In fact, the only \hyperref[def:automorphism]{automorphism} of \(\mathbb{Z} \) is the identity.
\end{eg}

\begin{eg}
	\(\mathbb{N} \) is not \hyperref[def:definable]{\(\varnothing \)-definable} in \(\mathbb{Z} =(\mathbb{Z} , 0, +)\).
\end{eg}
\begin{explanation}
	Consider an \hyperref[def:automorphism]{automorphism} \(f(x) = -x\) of the group \(\mathbb{Z} \), which does not fix \(\mathbb{N} \) setwise.
\end{explanation}

\begin{problem*}
	Is \(\mathbb{N} \) \hyperref[def:definable]{definable} in \(\mathbb{Z} =(\mathbb{Z} , 0, +)\) over some parameters \(\overline{a} \)?
\end{problem*}
\begin{answer}
	For example, if \(\overline{a} =(1)\), then \(f\) does not fix \(1\). In fact, any \hyperref[def:automorphism]{automorphism} fixing \(1\) also fixes all of \(\mathbb{Z} \), but \(\mathbb{N} \) is not \hyperref[def:definable]{definable} in \((\mathbb{Z} , 0, +)\). To prove this we need \hyperref[thm:compactness]{compactness}.
\end{answer}

\begin{prev}
	Given a field \(F\), then \(F(a) \cong F(b)\) if \(a\) and \(b\) have the same minimal polynomial over \(F\) or if both do not satisfy any polynomial over \(F\).
\end{prev}

\begin{eg}
	\(\mathbb{Q} (\pi ) \cong \mathbb{Q} (e)\) because \(\pi \) and \(e\) are both transcendental.
\end{eg}

We now return to the big question: is \(\mathbb{R} \) \hyperref[def:definable]{definable} in \(\mathbb{C} = (\mathbb{C} , 0, 1, +, \cdot, -)\)? If \(f\colon \mathbb{Q} (a) \to  \mathbb{Q} (b)\) such that \(a\mapsto b\), then there is an \hyperref[def:automorphism]{automorphism} \(\hat{f} \colon \mathbb{C} \to  \mathbb{C} \) such that \(a\mapsto b\), i.e., \(\hat{f} \) extends \(f\). In other words, we need to find such an \(f\) with \(a\in \mathbb{R} \) and \(b\notin \mathbb{R} \).

\begin{eg}
	\(a = \pi \), \(b = i \pi \) are both transcendental.
\end{eg}

\begin{eg}
	\(a\) is a real \(\sqrt[4]{2}\), \(b\) is a complex \(\sqrt[4]{2} \).
\end{eg}

The above two examples show that \(\mathbb{R} \) is not \hyperref[def:definable]{\(\varnothing \)-definable} in \(\mathbb{C} \). In fact, \(\mathbb{R} \) is not \hyperref[def:definable]{definable} over any \(\overline{a} \) because there are elements of \(\mathbb{R} \) and \(\mathbb{C} \setminus \mathbb{R} \) transcendental over any \(\overline{a} \).

\begin{intuition}
	There are so many \(a, b\) such that given any \(\overline{a} \), we can still find a pair that works.
\end{intuition}

\section{Proofs}
There are all sorts of different proof systems, and the one we use is the so-called Hilbert-style deductive system. Before that, we first see some common notions.

\begin{notation}[Schema]\label{not:schema}
	A \emph{schema} is written in symbols for \hyperref[def:formula]{formulas}, variables, etc.
\end{notation}

\begin{eg}
	\(\varphi \to (\psi \to  \varphi )\) is a \hyperref[not:schema]{schema}, i.e., an infinite set with all possible choices of \(\varphi \) and \(\psi \).
\end{eg}

Specifically, every \hyperref[def:logical-axioms]{logical axiom} is written in \hyperref[not:schema]{schema}, meaning that any instance of a symbol for a \hyperref[def:formula]{formula}, e.g., \(\varphi \), can be replaced by any \hyperref[def:formula]{formula}.

\begin{definition}[Generalization]\label{def:generalization}
	A \hyperref[def:formula]{formula} \(\varphi \) is a \emph{generalization} of a \hyperref[def:formula]{formula} \(\psi \) if \(\varphi \) is \(\forall x_1 \dots \forall x_n\ \psi \) where \(x_1, \dots , x_n\) are variables.
\end{definition}

\begin{notation}[Hypothesis]\label{not:hypothesis}
	\emph{Hypotheses} are \hyperref[def:formula]{formulas} that we may assume in a \hyperref[def:proof]{proof}.
\end{notation}

\begin{definition}[Proof]\label{def:proof}
	A \emph{proof} is a sequence of \hyperref[def:formula]{formulas} \(\left\{ \varphi _i \right\} _{i=1}^n\) such that \(\varphi _n\) is the conclusion, and each \hyperref[def:formula]{formula} is either an \hyperref[def:logical-axioms]{axiom} or is obtained from the previous \hyperref[def:formula]{formulas} by a \hyperref[def:rule-of-inference]{rule of inference}.

	Moreover, for a \hyperref[def:proof]{proof} based on a set of \hyperref[not:hypothesis]{hypotheses} \(\Gamma \), then in addition to a \hyperref[def:logical-axioms]{logical axiom}, we can assert a \hyperref[def:formula]{formula} \(\varphi \in \Gamma \). If we prove \(\psi \) using \(\Gamma \) as \hyperref[not:hypothesis]{hypotheses}, we write \(\Gamma \vdash \psi \).

	\begin{definition}[Valid]\label{def:valid}
		If we \hyperref[def:proof]{prove} \(\psi \) without \hyperref[not:hypothesis]{hypotheses}, we write \(\vdash \psi \) and say \(\psi \) is \emph{valid}.
	\end{definition}

	\begin{definition}[Logical axioms]\label{def:logical-axioms}
		The \emph{logical axioms} are the following \hyperref[def:formula]{formulas} written in \hyperref[not:schema]{schema}, as well as all of their \hyperref[def:generalization]{generalizations}:
		\begin{definition}[Propositional axioms]\label{def:propositional-axioms}
			The \emph{propositional axioms} are
			\begin{enumerate}[label=(A\arabic*), resume*]
				\item\label{A1} \(\varphi \to  (\psi \to \varphi )\).
				\item\label{A2} \((\varphi \to (\psi \to \theta )) \to ((\varphi \to  \psi ) \to (\varphi \to \theta ))\).
				\item\label{A3} \((\lnot \varphi \to \lnot \psi ) \to  ((\lnot \varphi \to \psi ) \to \varphi )\).
			\end{enumerate}
		\end{definition}

		\begin{enumerate}[label=(A\arabic*), resume]
			\item\label{A4} \(\forall x\ \varphi (x, \dots ) \to \varphi (t, \dots )\) where \(t\) is any \hyperref[def:term]{term}.
			\item\label{A5} \(\left[ \forall x\ (\varphi \to \psi ) \right] \to \left[ (\forall x\ \varphi ) \to (\forall x\ \psi ) \right] \).
			\item\label{A6} \(\varphi \to  \forall x\ \varphi \), where \(x\) is not \hyperref[def:free-variable]{free} in \(\varphi \).
		\end{enumerate}

		\begin{definition}[Axioms for equality]\label{def:axioms-for-equality}
			The \emph{axioms for equality} is
			\begin{enumerate}[label=(A\arabic*), resume]
				\item\label{A7} for any \hyperref[def:term]{terms} \(t, u, v, \dots \), function symbols \(f\), and relation symbols \(R\),
				\begin{enumerate}[(a)]
					\item\label{def:axioms-for-equality-a} \(t = t\).
					\item\label{def:axioms-for-equality-b} \(t = u \to  u = t\).
					\item\label{def:axioms-for-equality-c} \((t=u \land u = v) \to  (t = v)\).
					\item\label{def:axioms-for-equality-d} \((u_1 = t_1 \land \dots \land u_{n_f} = t_{n_f}) \to f(u_1, \dots , u_{n_f}) = f(t_1, \dots , t_{n_f})\).
					\item\label{def:axioms-for-equality-e} \((u_1 = t_1 \land \dots \land u_{n_R} = t_{n_R}) \to (R(u_1, \dots , u_{n_R}) \leftrightarrow R(t_1, \dots , t_{n_R}))\).
				\end{enumerate}
			\end{enumerate}
		\end{definition}
	\end{definition}

	\begin{definition}[Rule of inference]\label{def:rule-of-inference}
		From \(\varphi \) and \(\varphi \to \psi\), deduces \(\psi \).\footnote{This is called \href{https://en.wikipedia.org/wiki/Modus_ponens}{modus ponens}.}
	\end{definition}
\end{definition}

These \hyperref[def:formula]{formulas} might have \hyperref[def:free-variable]{free variables}.

\begin{eg}
	A \hyperref[def:proof]{proof} from calculus of a limit, e.g., \(\forall \epsilon \exists \delta \dots \). And we start by stating
	\begin{enumerate}
		\item let \(\epsilon > 0\),
		\item choose \(\delta =\epsilon \),
		\item[] \(\vdots\)
		\item[\(n\).] \(\vert f(x) - f(y) \vert < \epsilon \).
	\end{enumerate}
	We should interpret \hyperref[def:free-variable]{free variables} as anything.
\end{eg}

\begin{prev}[Propositional logic]
	\((p \land q) \lor (r \land \lnot q)\).
\end{prev}

\begin{remark}
	We can check whether the \hyperref[def:propositional-axioms]{propositional axioms} are \hyperref[def:truth]{true} with a truth table.
\end{remark}

\begin{definition}[Propositional tautology]\label{def:propositional-tautology}
	A \emph{propositional tautology} is a boolean combination \(\lor , \land , \lnot \) of \hyperref[def:formula]{formulas} \(\varphi _1, \dots , \varphi _n\) which is \hyperref[def:truth]{true} via a truth table assigning true or false to each of \(\varphi _1, \dots , \varphi _n\).
\end{definition}

So instead of using \hyperref[def:propositional-axioms]{propositional axioms}, we could instead allow as \hyperref[def:logical-axioms]{logical axioms} any \hyperref[def:propositional-tautology]{propositional tautology}. To prove \hyperref[def:complete]{completeness}, we will need \(~5\) \hyperref[def:propositional-tautology]{propositional tautologies}. We will \hyperref[def:proof]{prove} some of these, but take others on faith.

\begin{remark}
	\hyperref[def:propositional-axioms]{Propositional axioms} are enough to \hyperref[def:proof]{prove} all \hyperref[def:propositional-tautology]{propositional tautologies}.
\end{remark}

\begin{notation}
	We write \(\Gamma \vdash_\mathcal{L} \varphi \) if there is a \hyperref[def:proof]{proof} of \(\varphi \) from \(\Gamma \) in the \hyperref[def:language]{language} \(\mathcal{L} \).
\end{notation}

\begin{note}
	Passing to a larger \hyperref[def:language]{language} will not let you \hyperref[def:proof]{prove} more, so we can just write \(\vdash \).
\end{note}