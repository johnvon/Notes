\lecture{14}{21 Feb. 14:30}{Downward Löwenheim-Skolem theorem}
\begin{definition}[Built-in Skolem functions]\label{def:built-in-Skolem-function}
	We say an \hyperref[def:theory]{\(\mathcal{L} \)-theory} \(T\) has \emph{built-in Skolem functions} if for all \hyperref[def:formula]{\(\mathcal{L} \)-formulas} \(\varphi (x, y_1, \ldots , y_n)\), there is a function symbol \(f\) such that
	\[
		T\models \forall \overline{y} \ (\exists x\ \varphi (x, \overline{y} ) \to \varphi (f(\overline{y} ), \overline{y} )).
	\]
\end{definition}

\begin{lemma}\label{lma:Skolemization}
	Let \(T\) be an \hyperref[def:theory]{\(\mathcal{L} \)-theory}, then there are \(\mathcal{L} ^{\ast} \supseteq \mathcal{L} \) and \(T^{\ast} \supseteq T\) an \hyperref[def:theory]{\(\mathcal{L} ^{\ast} \)-theory} such that \(T^{\ast} \) has \hyperref[def:built-in-Skolem-function]{built-in Skolem functions}. Moreover, if \(\mathcal{M} \models T\), then we can expand \(\mathcal{M} \) to \(\mathcal{M} ^{\ast} \models T^{\ast} \).
\end{lemma}
\begin{proof}
	Start with \(\mathcal{L} _0 = \mathcal{L} \) and \(T_0 = T\), we build \(\mathcal{L} ^{\ast} = \bigcup_{i} \mathcal{L} _i\) and \(T^{\ast} = \bigcup_{i} T_i\). Given \(\mathcal{L} _i\) and \(T_i\), define
	\[
		\mathcal{L} _{i+1} = \mathcal{L} _i \cup \left\{ f_\varphi \mid \varphi (x, \overline{y} ) \text{ is an \hyperref[def:formula]{\(\mathcal{L} _i \)-formulas}} \right\}
	\]
	where the arity of \(f_\varphi \) is the same as \(\overline{y} \), and
	\[
		T_{i+1} = Tvi \cup \left\{ \forall \overline{y} \ (\exists x\ \varphi (x, \overline{y} ) \to \varphi (f_\varphi (\overline{y} ) , \overline{y} ) ) \right\}.
	\]
	Now, we argue that if \(\mathcal{M} \models T_i\), we can expand it to a \hyperref[def:model]{model} \(\mathcal{M} ^{\ast} \) of \(T_{i+1} \). Pick \(c\in M\) a ``default value.'' Given \(\varphi \) and \(\overline{a} \), define \(f_\varphi ^{\mathcal{M} ^{\ast} }(\overline{a} )\) to be some \(b\) with \(\mathcal{M} \models \varphi (b, \overline{a} )\) if such a \(b\) exists, or \(c\) otherwise. Then, \(\mathcal{M} ^{\ast} \models T_{i+1} \).

	Now, \(T^{\ast} \) has \hyperref[def:built-in-Skolem-function]{built-in Skolem functions}. Suppose \(\mathcal{M} \models T\), i.e., \(\mathcal{M} = \mathcal{M} _0 \models T_0\). Then \(\mathcal{M} _0\) has an \hyperref[not:expansion]{expansion} \(\mathcal{M} _1 \models T_1\), which has an \hyperref[not:expansion]{expansion} \(\mathcal{M} _2 \models T_2\), etc. Then, \(\mathcal{M} ^{\ast} = \bigcup_{i} \mathcal{M} _i\) is a \hyperref[def:model]{model} of \(T^{\ast} \), and by counting, we have \(\vert \mathcal{L} ^{\ast} \vert = \vert \mathcal{L} \vert + \aleph_0\).
\end{proof}

\begin{notation}[Skolemization]\label{not:Skolemization}
	We call \(T^{\ast} \) in \autoref{lma:Skolemization} a \emph{Skolemization} of \(T\).
\end{notation}

\begin{theorem}[Downward Löwenheim-Skolem theorem]\label{thm:downward-Lowenheim-Skolem}
	Let \(\mathcal{M} \) be an \hyperref[def:structure]{\(\mathcal{L} \)-structure} and \(X \subseteq M\). Then there is an \hyperref[def:elementary-substructure]{elementary substructure} \(X \subseteq \mathcal{N} \) of \(\mathcal{M} \) of cardinality \(\vert \mathcal{N} \vert \leq \vert \mathcal{L} \vert + \aleph_0 + \vert X \vert \).
\end{theorem}
\begin{proof}
	By \hyperref[not:expansion]{expanding} the \hyperref[def:language]{language}, we get \(\mathcal{M} ^{\ast} \) and \hyperref[def:structure]{\(\mathcal{L} ^{\ast} \)-structure} with \(\mathop{\mathrm{Th}}(\mathcal{M} ^{\ast} )\) has \hyperref[def:built-in-Skolem-function]{built-in Skolem functions} (where \(T = \mathop{\mathrm{Th}}(\mathcal{M} ) \) in \autoref{lma:Skolemization}). Replacing \(\mathcal{M} \) by \(\mathcal{M} ^{\ast} \), etc., we may assume that we already had \hyperref[def:built-in-Skolem-function]{built-in Skolem functions}.

	Start with \(X_0 = X \cup \left\{ c^{\mathcal{M} } \mid \text{\(c\) a constant symbol}\right\} \). Given \(X_i\), define \(X_{i+1} \) as
	\[
		X_{i+1} = X_i \cup \left\{ f(\overline{a} ) \mid \text{\(f\) a function symbol, \(\overline{a} \in X_i \)} \right\}.
	\]
	Let \(N = \bigcup_{i} X_i\), let \(\mathcal{N} \) be the \hyperref[def:substructure]{substructure} of \(\mathcal{M} \) with domain \(N\).\footnote{Here, \(\mathcal{N} \) is called the \emph{substructure generated by \(X\)}.} Now, to show that \(\mathcal{N} \) is an \hyperref[def:elementary-substructure]{elementary substructure} of \(\mathcal{M} \), we use the \hyperref[prop:Tarski-Vaught-test]{Tarski-Vaught test}. Suppose that we have an \hyperref[def:formula]{\(\mathcal{L} \)-formula} \(\varphi (x, \overline{y} )\), \(\overline{a} \in N\), \(b\in M\) such that \(\mathcal{M} \models \varphi (b, \overline{a} )\) (we must replace \(b\) by \(c\in N\)). Since \(\mathcal{M} \models \exists x\ \varphi (x, \overline{a} )\), so \(\mathcal{M} \models \varphi (f_\varphi (\overline{a} ), \overline{a} )\). But since \(\overline{a} \in N\), so \(f_\varphi (\overline{a} )\in N\), so the \hyperref[prop:Tarski-Vaught-test]{Tarski-Vaught test} says \(\mathcal{N} \) is an \hyperref[def:elementary-substructure]{elementary substructure} of \(\mathcal{M} \).

	Finally, we see that \(\vert X_0 \vert \leq \vert X \vert + \vert \mathcal{L} \vert + \aleph_0\), and since \(N\) is a countable union, so
	\[
		\vert N \vert \leq \vert X \vert + \vert \mathcal{L}  \vert + \aleph_0.
	\]
\end{proof}

\begin{eg}[Countable real closed filed]
	Consider \(\mathbb{R} =(\mathbb{R} , 0, 1, +, -, \cdot, \leq )\). Let \(X \subseteq \mathbb{R} \) be countable, e.g., \(X = \varnothing \) or \(X = \left\{ \pi , e \right\} \). Then there is \(X \subseteq R \preceq \mathbb{R} \) such that \(R\) is countable. In particular, \(\mathop{\mathrm{Th}}(R) = \mathop{\mathrm{Th}}(\mathbb{R})\). In this way, \(R\) is a \emph{countable real closed filed}:
	\begin{enumerate}[(a)]
		\item \(-1\) is not a sum of squares;
		\item for all \(a\), there is \(b\) such that \(a = b^2\) or \(a=-b^2\);
		\item every odd degree polynomial has a root.
	\end{enumerate}
\end{eg}

\begin{eg}[Skolem's paradox]
	Let \(\mathcal{L} = \left\{ \in\right\} \), where \(\in\) a binary relation symbol. Let \(T=\mathrm{ZFC} \). Suppose that \(\mathrm{ZFC} \) is a \hyperref[def:satisfiable]{satisfiable},\footnote{From \href{https://en.wikipedia.org/wiki/Gödel's_incompleteness_theorems}{Gödel's incompleteness theorem}, in \(\mathrm{ZFC} \), one can't prove that \(\mathrm{ZFC} \) is \hyperref[def:consistent]{consistent}.} and let \(\mathcal{M} \models T\). Then there is \(\mathcal{N} \preceq \mathcal{M} \) with \(\mathcal{N} \) countable. Then,
	\[
		\mathcal{N} \models \text{``there is no bijection between \(\mathbb{R} ^\mathcal{N} \) and \(\mathbb{N} ^\mathcal{N} \)''}.
	\]
	\(\mathcal{N} \) thinks that it contains an uncountable set \(\mathbb{R} ^\mathcal{N} \), but \(\left\{ a\in N \mid \mathcal{N} \models a\in \mathbb{R} ^\mathcal{N} \right\} \subseteq N\) is countable. This is called \emph{Skolem's paradox}.
\end{eg}

\begin{definition}[Universallly axiomatizable]\label{def:universallly-axiomatizable}
	Let \(T\) be an \hyperref[def:theory]{\(\mathcal{L} \)-theory}, then \(T\) is \emph{universallly axiomatizable} if there is a set \(\Gamma \) of \hyperref[not:universal]{universal} \hyperref[def:sentence]{sentences} such that \(T \models \Gamma \) and \(\Gamma \models T\).
\end{definition}

\begin{theorem}
	Let \(T\) be an \hyperref[def:theory]{\(\mathcal{L} \)-theory}. \(T\) is \hyperref[def:universallly-axiomatizable]{universallly axiomatized} if and only if whenever \(\mathcal{N} \models T\) and \(\mathcal{M} \subseteq \mathcal{N} \), then \(\mathcal{M} \models T\).
\end{theorem}
\begin{proof}
	We already know the forward direction. Now, to prove the backward direction, suppose that if \(\mathcal{N} \models T\), \(\mathcal{M} \subseteq \mathcal{N} \), then \(\mathcal{M} \models T\). Define
	\[
		\Gamma = \left\{ \varphi \text{ \hyperref[not:universal]{universal}} \mid T \models \varphi \right\},
	\]
	then \(T \models \Gamma \). Now, we show that \(\Gamma \models T\). We may assume that \(T\) is \hyperref[def:satisfiable]{satisfiable}\footnote{Since otherwise \(\Gamma \ni \forall x\ x \neq x\).} and let \(\mathcal{M} \models \Gamma \).we must prove that \(\mathcal{M} \models T\).  We will do this by finding \(\mathcal{N} \supseteq \mathcal{M} \), \(\mathcal{N} \models T\), which implies \(\mathcal{M} \models T\). We build such an \(\mathcal{N} \) by showing that \(\Diag(\mathcal{M} ) \cup T\) is \hyperref[def:satisfiable]{satisfiable} with \hyperref[thm:compactness]{compactness theorem}. Let \(\Delta \subseteq \Diag(\mathcal{M} ) \cup T\)  be finite, then there is a finite set of \hyperref[not:atomic-formula]{atomic} or negated \hyperref[not:atomic-formula]{atomic formulas} \(\varphi _1, \ldots , \varphi _\ell \) and \(m_1, \ldots , m_k \in M\) such that
	\[
		\Delta \subseteq \left\{ \varphi _1(\overline{m} ), \ldots , \varphi _\ell (\overline{m} ) \right\} \cup T.
	\]
	We may assume that they are actually equal. To show that \(\Delta \) is \hyperref[def:satisfiable]{satisfiable}, it is enough to show that
	\[
		\left\{ \exists x_1 \ldots \exists x_k\ (\varphi _1(\overline{x} ) \land \ldots \land \varphi _\ell (\overline{x} )) \right\} \cup T
	\]
	is \hyperref[def:satisfiable]{satisfiable}. If not, then \(T\models \forall x_1 \ldots \forall x_k\ \lnot (\varphi _1(\overline{x} ) \land \ldots \land \varphi _\ell (\overline{x} ))\). But this is \hyperref[not:universal]{universal}, hence in \(\Gamma \), so it is \hyperref[def:truth]{true} in \(\mathcal{M} \), i.e., \(\mathcal{M} \models \varphi _1(\overline{m} ) \land \ldots \land \varphi _\ell (\overline{m} )\) and \(\mathcal{M} \models \forall x_1 \ldots \forall x_k\ \lnot (\varphi _1(\overline{x} ) \land \ldots \land \varphi _\ell (\overline{x} ))\), a contradiction \(\conta\) Hence, \(\Delta \) is \hyperref[def:satisfiable]{satisfiable}, so any finite subset is \hyperref[def:satisfiable]{satisfiable}, by \hyperref[thm:compactness]{compactness theorem}, we're done.
\end{proof}