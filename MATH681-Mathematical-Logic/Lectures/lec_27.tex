\lecture{27}{13 Apr. 11:30}{Proving the Algebraic Bounds}
\begin{prev}[Goal]\label{prev:lec27-goal}
	If \(I \subseteq K[x_1, \dots , x_n]\) then \(IR \cap K[x_1, \dots , x_n] = I\).
\end{prev}

\begin{theorem}
	If \(m\) is a \hyperref[def:proper-ideal-maximal]{maximal} \hyperref[def:ideal]{ideal} of \(K[x_1, \dots , x_n]\), then \(mR \neq R\).
\end{theorem}
\begin{proof}
	Let \(m=(f_1, \dots , f_{\ell } )\). Then \(f_1, \dots , f_{\ell } \) have a common zero in an \hyperref[def:field-extension]{extension} of \(K\), namely, \(\quotient{K[m_1, \dots , x_n]}{m} \). So, \(f_1, \dots , f_{\ell } \) have a zero in any \hyperref[def:algebraic-closure]{algebraic closure} of \(K\).

	Now if \(mR = R\), \(1 = h_1 f_1 + \dots + h_{\ell } f_{\ell } \) where \(h_1, \dots , h_{\ell } \) in \(R\). So for almost every \(i\), \(1 = h_1^i f_1^i + \dots + h_{\ell }^i f_{\ell }^i \) in \(K_i[x_1, \dots , x_n]\). Then \(f_1^i, \dots , f_{\ell }^i \) can't have a common zero in any \hyperref[def:field-extension]{extension} of \(K_i\), including in \(\overline{K} _i\), the \hyperref[def:algebraic-closure]{algebraic closure} of \(K_i\).

	Let \(\overline{K} = \prod \quotient{\overline{K} _i}{\mathcal{U} } \). This is an \hyperref[def:algebraically-closed]{algebraically closed} field containing \(K\). Then, \(f_1, \dots , f_{\ell } \) have a common zero \(\overline{a} = [\overline{a} ^i]\) in \(\overline{K} \), i.e., \(f_1(\overline{a} ) = \dots = f_{\ell } (\overline{a} ) = 0\). So for almost every \(i\), \(f_1^i(\overline{a} ^i) = \dots = f_{\ell }^i (\overline{a} ^i) = 0\) in \(\overline{K} _i\). Hence, for almost every \(i\), \(f_1^i, \dots , f_{\ell }^i \) both have and don't have a common zero, a contradiction, implying \(mR \neq R\).
\end{proof}

\begin{theorem}\label{thm:lec27}
	Let \(A\) be a subring of \(B\) such that
	\begin{enumerate}[(a)]
		\item if \(f_1, \dots , f_{\ell }\in A \), any solution in \(B\) of \(f_1 y_1 + \dots + f_{\ell }y_{\ell } = 0\) is a linear combination of solutions in \(A\);
		\item if \(m\) a \hyperref[def:proper-ideal-maximal]{maximal} \hyperref[def:ideal]{ideal} of \(A\), then \(mB \neq B\),
	\end{enumerate}
	then if \(I\) is an \hyperref[def:ideal]{ideal} of \(A\), \(I B \cap A = I\).
\end{theorem}
\begin{proof}
	Consider the inclusion map \(\quotient{A}{I} \to \quotient{B}{I B}\), which is well-defined since \(I \subseteq I B\). If this is injective, then \(I = IB \cap A\).

	Take \(x\in A\), \(x \notin I\). Consider \(f\colon A \to \quotient{A}{I} \) such that \(r \mapsto rx\). Observe that \(f(1) \notin I\), i.e., \(1 \notin \ker f\). Let \(J = \{ a\in A \mid f(a) = 0 \} = \{ a\in A \mid ax \in I \} \), then \(1 \notin J\) since \(x \notin I\). Moreover, \(J\) is an \hyperref[def:ideal]{ideal} of \(A\), so \(J \subsetneq A\). Then, \(JB \neq B\) because \(J \subseteq m\) a \hyperref[def:proper-ideal-maximal]{maximal} \hyperref[def:ideal]{ideal} of \(A\), and \(mB \neq B\).

	Now, consider the following, where we want to show that \(\quotient{B}{BJ} \to \quotient{Bx}{BI} \) defined by \(r \mapsto rx\) is a bijection.
	% https://q.uiver.app/?q=WzAsNCxbMCwwLCJcXHF1b3RpZW50e0F9e0p9Il0sWzEsMCwiXFxxdW90aWVudHtBeH17SX0iXSxbMSwxLCJcXHF1b3RpZW50e0J4fXtJQn0iXSxbMCwxLCJcXHF1b3RpZW50e0J9e0pCfSJdLFswLDEsIiIsMCx7InN0eWxlIjp7InRhaWwiOnsibmFtZSI6Imhvb2siLCJzaWRlIjoidG9wIn0sImhlYWQiOnsibmFtZSI6ImVwaSJ9fX1dLFsxLDJdLFswLDNdLFszLDIsIiIsMix7InN0eWxlIjp7InRhaWwiOnsibmFtZSI6Imhvb2siLCJzaWRlIjoidG9wIn0sImJvZHkiOnsibmFtZSI6ImRhc2hlZCJ9LCJoZWFkIjp7Im5hbWUiOiJlcGkifX19XV0=
	\[\begin{tikzcd}
			{\quotient{A}{J}} & {\quotient{Ax}{I}} \\
			{\quotient{B}{JB}} & {\quotient{Bx}{IB}}
			\arrow[hook, two heads, from=1-1, to=1-2]
			\arrow[from=1-2, to=2-2]
			\arrow[from=1-1, to=2-1]
			\arrow[dashed, hook, two heads, from=2-1, to=2-2]
		\end{tikzcd}\]
	where \(\quotient{Ax}{I} = \quotient{\{rx \mid r\in A \}}{I}\). Note that \(\quotient{B}{BJ} \) is non-trivial since \(B \neq J B\). We need to show that this is well-defined, i.e., \(JB \subseteq IB\). If \(r\in BJ\), \(r= b_1 j_1 + \dots + b_n j_n\) for \(j_1, \dots , j_n\in J\), then \(rx = b_1 j_1 x + \dots + b_n j_n x\) where we know that \(j_i x\in I\) for all \(i\), so \(rx\in BI\).

	This is also injective. Suppose \(r\in B\) is such that \(rx\in IB\), we want to show that \(r\in JB\). Since \(rx \in I B\), there are \(f_1, \dots , f_{\ell } \in I\) and \(h_1, \dots , h_{\ell } \in B\) with
	\[
		rx = f_1 h_1 + \dots + f_{\ell } h_{\ell }.
	\]
	Think of \((r, h_1, \dots , h_{\ell } )\) as a solution of \(- y_0 x + y_1 f_1 + \dots + y_{\ell } f_{\ell } = 0\), which is a homogeneous equation over \(A\).so \((r, h_1, \dots , h_{\ell } )\) is a \(B\)-linear combination of \((r^i, h_1^i, \dots , h_{\ell }^i )\in A^{\ell + 1}\), which are solution of the same equation. For each \(i\), \(r^1 x = f_1 h_1^i + \dots + h_{\ell }h_{\ell } ^i\in I\). Since \(r^i x \in I\), \(r^i \in J\). Then \(r = \sum_{i} g_i r^i\) where \(g^i \in B\), so \(r\in JB\). This shows that the diagram does commute.

	Finally, since \(JB \neq B\) and \(\quotient{B}{JB} \cong \quotient{Bx}{IB} \), \(x \notin IB\). Recall that \(x\in A\) was an arbitrary element \(\notin I\), so \(\quotient{A}{I} \to \quotient{B}{BI} \) is injective.
\end{proof}

From \autoref{thm:lec27}, take \(A = K[x_1, \dots , x_n]\) and \(B = R\), we have the following.

\begin{corollary}
	If \(I \subseteq K[x_1, \dots , x_n]\) then \(IR \cap K[x_1, \dots , x_n] = I\).
\end{corollary}

This is exactly \autoref{thm:lec25} \autoref{thm:lec25-b}! From \autoref{thm:lec25}, we can finally put \autoref{thm:lec25} \autoref{thm:lec25-a} as a theorem.

\begin{theorem}
	Given \(n\) and \(d\), there is \(m = m(n, d)\) such that for any field \(K\) and \(f_1, \dots , f_{\ell }, g \in K[x_1, \dots , x_n]\) of \(\deg \leq d\), if \(g\in (f_1, \dots , f_{\ell } )\), then there are \(h_1, \dots , h_{\ell } \in K[x_1, \dots , x_n]\) of \(\deg \leq m\) such that \(g = h_1 f_1 + \dots + h_{\ell } f_{\ell } \).
\end{theorem}

This solves \autoref{prb:algebraic-bound}.

\begin{problem*}
	We're proved these bounds exist, but can we compute them?
\end{problem*}
\begin{answer}
	Yes! This can be done by
	\begin{enumerate}[(a)]
		\item an algebraic proof, or
		\item \emph{proof mining}: if we look very carefully at the \hyperref[def:ultraproduct]{ultraproduct} proof, we can find bounds.
	\end{enumerate}
\end{answer}

\begin{remark}
	If we can compute bounds, we can solve these problems using linear algebra.
\end{remark}

\chapter{Epilogue}
\section{An Review: Map of the Universe}
To wrap up this course, we review what we have learned in this course, especially various \hyperref[def:model]{models}. Specifically, we have a glance on \href{www.forkinganddividing.com}{forkinganddividing}, and discuss some \hyperref[def:model]{models}.