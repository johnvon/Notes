\lecture{2}{10 Jan. 14:30}{Formulas and First-Order Logic}
We start by asking that given a function symbol \(f\) of arity \(n\), could we replace \(f\) with an \((n+1)\)-ary \(R\) relation to represent its graph?

\begin{eg}
	Let \(\mathcal{L} \) be a \hyperref[def:language]{language} with only relation symbols. Let \(\mathcal{A} \) be an \hyperref[def:structure]{\(\mathcal{L} \)-structure}. For any \(B \subseteq A\), there is a \hyperref[def:substructure]{substructure} \(\mathcal{B} \) of \(\mathcal{A} \) with domain \(B\).
\end{eg}
\begin{explanation}
	For each relation symbol \(R\), leting \(R^{\mathcal{B} } =R^{\mathcal{A} }  \cap B^{n_R}\) will make \(\mathcal{B} \) a \hyperref[def:substructure]{substructure} of \(\mathcal{A} \).
\end{explanation}

The above is not true for function symbols though.

\begin{eg}
	If \(G=(\mathbb{Z} , 0, +)\), then \(\mathbb{N} \) is not the domain of a subgroup. So if we took \(\mathcal{L} =\left\{ 0, +, ^{-1}  \right\} \), where \(0\) is the unary relation, \(+\) is the ternary relation, and \(^{-1} \) is the binary relation, an \hyperref[def:substructure]{\(\mathcal{L} \)-substructure} of a group might not be a subgroup.
\end{eg}

\section{Terms}
Intuitive, an \hyperref[def:formula]{\(\mathcal{L} \)-formula} is an expression built using the symbols in a \hyperref[def:language]{language} \(\mathcal{L} \), \(=\), the logical connectives \(\land, \lor, \lnot\), and variable symbols \(v_1, v_2, \ldots , x, y, z\), and also quantifiers \(\exists \) and \(\forall \).

\begin{definition}[Term]\label{def:term}
	Given a \hyperref[def:language]{language} \(\mathcal{L} \), the set of \emph{\(\mathcal{L} \)-terms} are defined inductively by:
	\begin{enumerate}[(a)]
		\item each constant symbol is a \emph{term};
		\item each variable symbol \(v_1, \ldots \) is a \emph{term};
		\item if \(f\) is a function symbol, and \(t_1, \ldots , t_{n_f}\) are \hyperref[def:term]{terms}, then \(f(t_1, \ldots , t_{n_f})\) is a \emph{term}.
	\end{enumerate}
\end{definition}

If \(\mathcal{M}\) is an \hyperref[def:structure]{\(\mathcal{L} \)-structure}, and \(t\) is a \hyperref[def:term]{term} involving only variables among \(v_1, \ldots , v_n\), then \(t\) has an \hyperref[not:interpretation]{interpretation} \(t^{\mathcal{M}} \colon M^n \to M\) as a function as follows. On input \(a_1, \ldots , a_n \in M\),
\begin{enumerate}[(a)]
	\item if \(t\) is a constant \(c\), \(t^{\mathcal{M} } (a_1, \ldots , a_n) = c^{\mathcal{M} }\).
	\item if \(t\) is a variable \(v_i\), \(t^{\mathcal{M} } (a_1, \ldots , a_n) = v_i\);
	\item if \(t\) is \(f(s_1, \ldots , s_k)\), then \(t^{\mathcal{M} } (a_1, \ldots , a_n) = f^{\mathcal{M} } \big(s_1^{\mathcal{M} }(a_1, \ldots , a_n), \ldots , s_k^{\mathcal{M}}(a_1, \ldots , a_n) \big)\).
\end{enumerate}

\begin{intuition}
	We are basically substituting for variables and evaluating the expression.
\end{intuition}

\begin{eg}
	In \((\mathbb{R} , 0, 1, +, \cdot, -)\), a \hyperref[def:term]{term} is essentially just a polynomial with integer coefficients, assuming we interpret them in a ring. Technically, a \hyperref[def:term]{term} looks like
	\[
		\cdot(+(1, 1), +(x, y)),
	\]
	but we will write \hyperref[def:term]{terms} the natural way, i.e.,
	\[
		(1+1) (x+y).
	\]
	Also, we will use \(\underline{n}\) or \(n\) to represent the \hyperref[def:term]{term} \(\underline{n} = \underbrace{1+1+\ldots +1}_{n\text{ times}}\). So we could write the above \hyperref[def:term]{term} as \(2\cdot(x+y)\).
\end{eg}

\section{Formulas}
A \hyperref[def:term]{term} is just a building block of \hyperref[def:formula]{formulas}, as we now see.

\begin{definition}[Formula]\label{def:formula}
	The set of \emph{\(\mathcal{L} \)-formulas} is defined inductively:
	\begin{enumerate}[(a)]
		\item\label{def:formula(a)} If \(s, t\) are \hyperref[def:term]{terms}, then \(s=t\) is a \emph{formula}.
		\item\label{def:formula(b)} If \(R\) is a relation symbol of arity \(n_R\) and \(s_1, \ldots , s_{n_R}\) are \hyperref[def:term]{terms}, then \(R(s_1, \ldots , s_{n_R})\) is a \emph{formula}.
		\item\label{def:formula(c)} If \(f\) is a \hyperref[def:formula]{formula}, then \(\lnot f\) is a \emph{formula}.
		\item\label{def:formula(d)} If \(\varphi\) and \(\psi\) are \hyperref[def:formula]{formulas}, then \(\varphi \land \psi \) and \(\varphi \lor \psi \) are \emph{formulas}.
		\item\label{def:formula(e)} If \(\varphi \) is a \hyperref[def:formula]{formula} and \(v_i\) are variables, then \(\exists v_i \ \varphi \) and \(\forall v_i \ \varphi \) are \emph{formulas}.
	\end{enumerate}
\end{definition}

\begin{notation}[Atomic]\label{not:atomic}
	\autoref{def:formula} \autoref{def:formula(a)} and \autoref{def:formula(b)} are called \emph{atomic}.
\end{notation}

\begin{notation}[Quantifier-free]\label{not:quantifier-free}
	\autoref{def:formula} \autoref{def:formula(a)}, \autoref{def:formula(b)}, \autoref{def:formula(c)}, and \autoref{def:formula(d)} are called \emph{quantifier-free}.
\end{notation}

This logic is called \emph{first-order logic} (FO logic), since the quantifiers range over elements of the \hyperref[def:structure]{structures}, but not over, e.g., subsets.

\begin{eg}
	We can say that an element \(x\) of a ring has a square root by \(\exists y\ y^2 = x\).
\end{eg}

\begin{eg}
	A group is torsion of order \(2\) can be said by \(\forall x\ x\cdot x = e\).
\end{eg}

\begin{eg}
	We can write down all the field/group/... axioms as \hyperref[def:formula]{formulas}.
\end{eg}

\subsection{Bounded and Free Variables}
Notice that for the first example, the \hyperref[def:formula]{formula} \(\exists y\ y^2 = x\) only has meaning if we assign what \(x\) is. In this case, we say that \(y\) is \emph{bound} by \(\exists y\). But this is local:

\begin{eg}
	Consider
	\[
		y=1 \land \exists y\ y^2 = x,
	\]
	while the first appearance of \(y\) is free, the second appearance of \(y\) is bound by (in the scope of) \(\exists y\).
\end{eg}

While our definitions work perfectly fine with the above example, but sometimes we don't want this to happen. In such a case, we simply replace the bound instances of \(y\) with a new variable \(z\). This idea of variables being free or bound is defined formally as follows.

\begin{definition}[Free variable]\label{def:free-variable}
	The \emph{free variables} \(\mathrm{FV}(\varphi )\) of a \hyperref[def:formula]{formula} \(\varphi \) are defined inductively:
	\begin{enumerate}[(a)]
		\item \(\mathrm{FV}(s=t) \) is the set of variables showing up in \(s\) or \(t\).
		\item \(\mathrm{FV}(R(s_1, \ldots , s_{n_R})) \) is the set of variables showing up in \(s_1, \ldots , s_{n_R}\).
		\item \(\mathrm{FV}(\lnot \varphi ) = \mathrm{FV} (\varphi )\).
		\item \(\mathrm{FV} (\varphi \land \psi ) = \mathrm{FV} (\varphi \lor \psi ) = \mathrm{FV}(\varphi ) \cup \mathrm{FV} (\psi ) \).
		\item \(\mathrm{FV} (\exists x\ \varphi ) = \mathrm{FV} (\forall x\ \varphi ) = \mathrm{FV} (\varphi ) \setminus \left\{ x \right\} \).
	\end{enumerate}
\end{definition}

\begin{eg}
	\(\mathrm{FV} (\exists y\ y^2 = x) = \left\{ x \right\} \).
\end{eg}

\begin{eg}
	\(\mathrm{FV} (\forall x\ x\cdot x = e) = \varnothing \).
\end{eg}

\begin{definition}[Sentence]\label{def:sentence}
	A \hyperref[def:formula]{formula} \(\varphi \) is called a \emph{sentence} if it has no \hyperref[def:free-variable]{free variables}.
\end{definition}

\begin{notation}
	If \(\varphi \) is a \hyperref[def:formula]{formula} with \hyperref[def:free-variable]{free variables} among \(x_1, \ldots , x_n\) we often write \(\varphi (x_1, \ldots , x_n)\).
\end{notation}

\begin{remark}
	So given \(\varphi (x_1, \ldots , x_n)\), we know that \(\varphi \) has no other \hyperref[def:free-variable]{free variables} than \(x_1, \ldots , x_n\).
\end{remark}

\begin{eg}
	It's valid to write \(\varphi (x, y, z) \coloneqq x=y\).
\end{eg}

\section{Truths}
Finally, we define the notion of \hyperref[def:truth]{truth}.

\begin{definition}[Truth]\label{def:truth}
	Given an \hyperref[def:structure]{\(\mathcal{L} \)-structure} \(\mathcal{M} \), let \(\varphi (x_1, \ldots , x_n)\) be an \hyperref[def:formula]{\(\mathcal{L} \)-formula} and let \(a_1, \ldots , a_n\in M\). Then we say \(\varphi \) is \emph{true} of \(\overline{a} \) in \(\mathcal{M} \),\footnote{Or \(\mathcal{M} \) satisfies \(\varphi (\overline{a} )\).} denoted as \(\mathcal{M} \models \varphi (\overline{a} )\), as follows:
	\begin{enumerate}[(a)]
		\item If \(\varphi \) is \(s=t\), then \(\mathcal{M} \models \varphi (\overline{a} )\) if \(s^{\mathcal{M}} (\overline{a} ) = t^{\mathcal{M} } (\overline{a} )\).
		\item If \(\varphi \) is \(R(t_1, \ldots , t_{n_R})\), then \(\mathcal{M} \models \varphi (\overline{a} )\) if \(\big( t^{\mathcal{M} }_1(\overline{a} ), \ldots , t^{\mathcal{M} }_{n_R}(\overline{a} )  \big)\in R^{\mathcal{M} }\).
		\item If \(\varphi \) is \(\lnot \psi \), then \(\mathcal{M} \models \varphi (\overline{a} )\) if \(\mathcal{M} \not\models \psi (\overline{a} )\).
		\item If \(\varphi \) is \(\psi _1\land \psi _2\), then \(\mathcal{M} \models \varphi (\overline{a} )\) if \(\mathcal{M} \models \psi _1(\overline{a} )\) and \(\mathcal{M} \models \psi _2(\overline{a} )\).
		\item If \(\varphi \) is \(\psi _1\lor \psi _2\), then \(\mathcal{M} \models \varphi (\overline{a} )\) if \(\mathcal{M} \models \psi _1(\overline{a} )\) or \(\mathcal{M} \models \psi _2(\overline{a} )\).
		\item If \(\varphi \) is \(\exists y\ \psi (\overline{x} , y)\), then \(\mathcal{M} \models \varphi (\overline{a} )\) if there's \(b\in M\) such that \(\mathcal{M} \models \psi (\overline{a} , b)\).
		\item If \(\varphi \) is \(\forall y\ \psi (\overline{x} , y)\), then \(\mathcal{M} \models \varphi (\overline{a} )\) if for all \(b\in M\) such that \(\mathcal{M} \models \psi (\overline{a} , b)\).
	\end{enumerate}
\end{definition}

\begin{remark}
	Every \hyperref[def:formula]{formula} is \hyperref[def:truth]{true}, or its negation is.
\end{remark}