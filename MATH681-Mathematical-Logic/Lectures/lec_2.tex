\lecture{2}{10 Jan. 14:30}{Formulas and First-Order Logic}
We start by asking that given a function symbol \(f\) of arity \(n\), could we replace \(f\) with an \((n+1)\)-ary \(R\) relation for its graph?

\begin{eg}
	Let \(\mathcal{L} \) be a \hyperref[def:language]{language} with only relation symbols. Let \(\mathcal{A} \) be an \hyperref[def:structure]{\(\mathcal{L} \)-structure}. For any \(B \subseteq A\), there is a \hyperref[def:substructure]{substructure} \(B\) of \(A\) with domain \(B\).

	What is \(B\)? For each relation symbol \(R\), let \(R^B=R^A \cap B^{n_R}\), and this will make \(B\) a \hyperref[def:substructure]{substructure} of \(A\).
\end{eg}

The above is not true for function symbols though.

\begin{eg}
	If \(G=(\mathbb{Z} , 0, +)\), then \(\mathbb{N} \) is not the domain of a subgroup. So if we took \(\mathcal{L} =\left\{ 0, +, ^{-1}  \right\} \), where \(0\) is the unary relation, \(+\) is the ternary relation, and \(^{-1} \) is the binary relation, an \hyperref[def:substructure]{\(\mathcal{L} \)-substructure} of a group might not be a subgroup.
\end{eg}

\section{First-Order Logic}
\subsection{Terms, Formulas, and Truths}
Intuitive, an \hyperref[def:formula]{\(\mathcal{L} \)-formula} is an expression built using the symbols in a \hyperref[def:language]{language} \(\mathcal{L} \), \(=\), the logical connectives \(\land, \lor, \lnot\), and variable symbols \(v_1, v_2, \ldots , x, y, z\), and also quantifiers \(\exists \) and \(\forall \).

\begin{definition}[Term]\label{def:term}
	Given a \hyperref[def:language]{language} \(\mathcal{L} \), the set of \emph{\(\mathcal{L} \)-terms} are defined inductively by:
	\begin{enumerate}[(a)]
		\item Each constant symbol is a \emph{term}.
		\item Each variable symbol \(v_1, \ldots \) is a \emph{term}.
		\item If \(f\) is a function symbol, and \(t_1, \ldots , t_{n_f}\) are \hyperref[def:term]{terms}, then \(f(t_1, \ldots , t_{n_f})\) is a \emph{term}.
	\end{enumerate}
\end{definition}

If \(\mathcal{M}\) is an \hyperref[def:structure]{\(\mathcal{L} \)-structure}, and \(t\) is a \hyperref[def:term]{term} involving only variables among \(v_1, \ldots , v_n\), then \(t\) has an interpretation \(t^{\mathcal{M}} \colon M^n \to M\).

Then, we define \(t^m\) inductively as follows: On input \(a_1, \ldots , a_n \in M\)
\begin{enumerate}[(a)]
	\item If \(t\) is a constant \(c\),
	      \[
		      t^{\mathcal{M} } (a_1, \ldots , a_n) = c^{\mathcal{M} }.
	      \]
	\item If \(t\) is a variable \(v_i\),
	      \[
		      t^{\mathcal{M} } (a_1, \ldots , a_n) = a_i.
	      \]
	\item If \(t\) is \(f(s_1, \ldots , s_k)\), then
	      \[
		      t^{\mathcal{M} } (a_1, \ldots , a_n) = f^{\mathcal{M} } \big(s_1^{\mathcal{M} }(a_1, \ldots , a_n), \ldots , s_k^{\mathcal{M}}(a_1, \ldots , a_n) \big).
	      \]
\end{enumerate}

\begin{intuition}
	We are basically substituting for variables and evaluating the expression.
\end{intuition}

\begin{eg}
	In \((\mathbb{R} , 0, 1, +, \cdot, -)\), technically, a \hyperref[def:term]{term} looks like
	\[
		\cdot(+(1, 1), +(x, y)),
	\]
	but we will write \hyperref[def:term]{terms} the natural way, i.e.,
	\[
		(1+1) (x+y).
	\]
	Also, we will use \(\underline{n}\) or \(n\) to represent the \hyperref[def:term]{term}
	\[
		\underline{n} = \underbrace{1+1+\ldots +1}_{n\text{ times}}.
	\]
	So we could write the above \hyperref[def:term]{term} as
	\[
		2\cdot(x+y)
	\]
\end{eg}

Then, what do the \hyperref[def:term]{terms} in the \hyperref[eg:language-ring]{ring language} look like? They are the polynomials with integer coefficients, assuming we interpret  them in a ring.

\begin{definition}[Formula]\label{def:formula}
	Given a \hyperref[def:language]{language} \(\mathcal{L} \), the \emph{\(\mathcal{L} \)-formulas} are defined inductively:
	\begin{enumerate}[(a)]
		\item If \(s, t\) are \hyperref[def:term]{terms}, \(s=t\) is a \emph{formula}.
		\item If \(R\) is a relation symbol of arity \(n_R\), and \(s_1, \ldots , s_{n_R}\) are \hyperref[def:term]{term}, then \(R(s_1, \ldots , s_{n_R})\) is a \emph{formula}.
		\item If \(f\) is a \hyperref[def:formula]{formula}, then \(\lnot f\) is a \emph{formula}.
		\item If \(\varphi\) and \(\psi\) are \hyperref[def:formula]{formulas}, then \(\varphi \land \psi \) and \(\varphi \lor \psi \) are \emph{formulas}.
		\item If \(\varphi \) is a \hyperref[def:formula]{formula}, and \(v_1\) is a variable, \(\exists v_i \ \varphi \) and \(\forall v_i \ \varphi \) are \emph{formulas}.
	\end{enumerate}
\end{definition}

\begin{notation}[Atomic formula]\label{not:atomic-formula}
	\hyperref[def:formula]{Formulas} of the form (a) and (b) in \autoref{def:formula} are called \emph{atomic formulas}.
\end{notation}

\begin{notation}[Quantifier-free formula]\label{not:quantifiers-free-formula}
	\hyperref[def:formula]{Formulas} of the form (a), (b), (c), and (d) in \autoref{def:formula} are called \emph{quantifier-free formulas}.
\end{notation}

\begin{eg}
	We can say that an element \(x\) of a ring has a square root by
	\[
		\exists y\ y^2 = x
	\]
\end{eg}

\begin{eg}
	A group is torsion of order \(2\) can be said by
	\[
		\forall x\ x\cdot x = e.
	\]
\end{eg}

\begin{eg}
	We can write down all the field/group/... axioms as \hyperref[def:formula]{formulas}.
\end{eg}

Notice that for the first example, the \hyperref[def:formula]{formula} \(\exists y\ y^2 = x\) only has meaning if we assign what \(x\) is. In this case, we say that \(y\) is \emph{bound} by \(\exists y\). But this is local:

\begin{eg}
	Consider
	\[
		y=1 \land \exists y\ y^2 = x,
	\]
	while the first appearance of \(y\) is free, the second appearance of \(y\) is bound by \(\exists y\), or we say that \(y\) is in the scope of \(\exists y\).
\end{eg}

While our definitions work perfectly fine with the above example, but sometimes we don't want this to happen. In such a case, we simply replace the bound instances of \(y\) with a new variable \(z\).

\begin{definition}[Free variable]\label{def:free-variable}
	The \emph{free variables} \(\mathrm{FV}(\varphi )\) of a \hyperref[def:formula]{formula} \(\varphi \) are defined inductively:
	\begin{enumerate}[(a)]
		\item \(\mathrm{FV}(s=t) \) is the set of variables showing up in \(s\) or \(t\).
		\item \(\mathrm{FV}(R(s_1, \ldots , s_{n_R})) \) is the set of variables showing up in \(s_1, \ldots , s_{n_R}\).
		\item \(\mathrm{FV}(\lnot \varphi ) = \mathrm{FV} (\varphi )\).
		\item \(\mathrm{FV} (\varphi \land \psi ) = \mathrm{FV} (\varphi \lor \psi ) = \mathrm{FV}(\varphi ) \cup \mathrm{FV} (\psi ) \).
		\item \(\mathrm{FV} (\exists x\ \varphi ) = \mathrm{FV} (\forall x\ \varphi ) = \mathrm{FV} (\varphi ) \setminus \left\{ x \right\} \).
	\end{enumerate}
\end{definition}

\begin{eg}
	\(\mathrm{FV} (\exists y\ y^2 = x) = \left\{ x \right\} \).
\end{eg}

\begin{eg}
	\(\mathrm{FV} (\forall x\ x\cdot x = e) = \varnothing \).
\end{eg}

\begin{definition}[Sentence]\label{def:sentence}
	A \hyperref[def:formula]{formula} \(\varphi \) is called a \emph{sentence} if it has no \hyperref[def:free-variable]{free variables}.
\end{definition}

\begin{notation}
	If \(\varphi \) is a \hyperref[def:formula]{formula} with \hyperref[def:free-variable]{free variables} among \(x_1, \ldots , x_n\) we often write \(\varphi (x_1, \ldots , x_n)\).
\end{notation}

\begin{remark}
	So given \(\varphi (x_1, \ldots , x_n)\), we know that \(\varphi \) has no other \hyperref[def:free-variable]{free variables}.
\end{remark}

\begin{eg}
	It's valid to write \(\varphi (x, y, z) \coloneqq x=y\).
\end{eg}

\begin{definition}[Truth]\label{def:truth}
	Given a \hyperref[def:language]{language} \(\mathcal{L} \) and an \hyperref[def:structure]{\(\mathcal{L} \)-structure} \(\mathcal{M} \), let \(\varphi (x_1, \ldots , x_n)\) be an \hyperref[def:formula]{\(\mathcal{L} \)-formula}. Let \(a_1, \ldots , a_n\in \mathcal{M} \). Define \(\mathcal{M} \models \varphi (\overline{a} )\)\footnote{We read this as \emph{\(\varphi \) is true of \(\overline{a} \) in \(\mathcal{M} \)}.} as follows:
	\begin{enumerate}[(a)]
		\item If \(\varphi \) is \(s=t\), then \(m\models \varphi (\overline{a} )\) if \(s^{\mathcal{M}} (\overline{a} ) = t^{\mathcal{M} } (\overline{a} )\).
		\item If \(\varphi \) is \(R(t_1, \ldots , t_{n_R})\), then \(\mathcal{M} \models \varphi (\overline{a} )\) if \(\big( t^{\mathcal{M} }_1(\overline{a} ), \ldots , t^{\mathcal{M} }_{n_R}(\overline{a} )  \big)\in R^{\mathcal{M} }\).
		\item If \(\varphi \) is \(\lnot \psi \), then \(\mathcal{M} \models \varphi (\overline{a} )\) if \(\mathcal{M} \not\models \psi (\overline{a} )\).
		\item If \(\varphi \) is \(\psi _1\land \psi _2\), then \(\mathcal{M} \models \varphi (\overline{a} )\) if \(\mathcal{M} \models \psi _1(\overline{a} )\) and \(\mathcal{M} \models \psi _2(\overline{a} )\).
		\item If \(\varphi \) is \(\psi _1\lor \psi _2\), then \(\mathcal{M} \models \varphi (\overline{a} )\) if \(\mathcal{M} \models \psi _1(\overline{a} )\) or \(\mathcal{M} \models \psi _2(\overline{a} )\).
		\item If \(\varphi \) is \(\exists y\ \psi (\overline{x} , y)\),\footnote{Recall that \(\overline{x} = (x_1, \ldots , x_n)\).} then \(\mathcal{M} \models \varphi (\overline{a} )\) if there's \(b\in \mathcal{M} \) such that \(\mathcal{M} \models \psi (\overline{a} , b)\).
		\item If \(\varphi \) is \(\forall y\ \psi (\overline{x} , y)\), then \(\mathcal{M} \models \varphi (\overline{a} )\) if for all \(b\in \mathcal{M} \) such that \(\mathcal{M} \models \psi (\overline{a} , b)\).
	\end{enumerate}
\end{definition}