\lecture{3}{12 Jan. 14:30}{Logical Consequence and Equivalence}
\begin{notation}[Material implication]
	The \emph{material implication} \(\varphi \to \psi \) between two \hyperref[def:formula]{formulas} \(\varphi, \psi \) is an abbreviation of \(\lnot \varphi \lor \psi\).
\end{notation}

\begin{notation}
	We use \(\varphi \leftrightarrow \psi \) as an abbreviation of \(((\varphi \to \psi ) \land (\psi \to \varphi ))\).
\end{notation}

Essentially, \(\to \) and \(\leftrightarrow\) is different from \(\implies \) and \(\iff \), where the former are only shown in \hyperref[def:formula]{formula}. Now, consider the \hyperref[eg:language-graph]{language of graphs} \(\mathcal{L} _{\text{graph} } = \left\{ E\right\} \), let's see some examples.
\begin{eg}
	An undirected graph can be written as
	\[
		\forall x\forall y\ (xEy \to yEx).
	\]
\end{eg}

\begin{eg}
	A vertex has at least three neighbors can be written as
	\[
		\varphi (x) \coloneqq \exists u\exists v\exists w\ (xEu \land xEv \land xEw \land u \neq v\land v \neq w\land u \neq w)
	\]
	in non-reflexive graphs.
\end{eg}

\begin{eg}
	For a vertex has exactly three neighbors,
	\[
		\psi (x)\coloneqq \exists u \exists v \exists w \forall y\ \big(xEu \land xEv \land xEw \land u \neq v\land v \neq w\land u \neq w \land (y = u \lor y = v \lor y = w \lor \lnot y E x)\big).
	\]
\end{eg}

\begin{problem*}
	Can we say that \(x\) has an even number of neighbors?
\end{problem*}
\begin{answer}
	We can't. Some things are not expressible in FO logic.
\end{answer}

\begin{eg}
	For a vertex \(x\) has a path of length \(4\) to \(y\),
	\[
		\Theta (x, y) \coloneqq \exists u \exists v \exists w\ (xEu \land uEv \land vEw \land wEy).
	\]
	We can also express that there is a path of length at most \(4\).
\end{eg}

\begin{problem*}
	Can we say that there is a path from \(x\) to \(y\)?
\end{problem*}
\begin{answer}
	We still can't! Not in FO logic (using \hyperref[thm:compactness]{compactness theorem}).
\end{answer}

\begin{remark}
	When we prove results by induction on \hyperref[def:formula]{formulas}, we only need to prove for \(\lnot , \land , \exists \), instead of for both \(\land , \lor \), and both \(\exists \) and \(\forall \).
\end{remark}
\begin{explanation}
	Since we can view \(\varphi \lor \psi \) as an abbreviation for \(\lnot (\lnot \varphi \land \lnot \psi ) \) and \(\forall x\ \varphi \) as an abbreviation for \(\lnot (\exists x\ \lnot \varphi )\).
\end{explanation}

\begin{remark}[\href{https://en.wikipedia.org/wiki/Sheffer_stroke}{Sheffer stroke}]
	In fact, we can get \(\land , \lor , \lnot\) from one logical connective, e.g., the \emph{sheffer stroke} \(\uparrow \), which is defined as
	\[
		\varphi \uparrow \psi \coloneqq \lnot(\varphi \land \psi ),
	\]
	and we can use \(\uparrow \) to define \(\lnot, \lor , \land \).
\end{remark}

\begin{notation}
	Let \(\Phi \) be a (possibly infinite) set of \hyperref[def:sentence]{sentences}, we write \(\mathcal{M} \models \Phi \) if \(\mathcal{M} \models \varphi \) for all \(\varphi \in \Phi \).
\end{notation}

\begin{definition}[Logical consequence]\label{def:logical-consequence}
	Let \(\Phi \) be a set of \hyperref[def:sentence]{sentences}, and \(\varphi \) be a \hyperref[def:sentence]{sentence}. We say that \(\varphi \) is a \emph{logical consequence} of \(\Phi \), written \(\Phi \models \varphi \), if \(\mathcal{M} \models \varphi \) whenever \(\mathcal{M} \models \Phi \).
\end{definition}

If \(\Phi = \varnothing \) is the empty set, \autoref{def:logical-consequence} is written as \(\models \varphi \), i.e., \(\varphi \) is \hyperref[def:truth]{true} in all \hyperref[def:structure]{\(\mathcal{L} \)-structures}.\footnote{Recall that we always have a \hyperref[def:language]{language} \(\mathcal{L} \) implicitly.}

\begin{definition}[Equivalent]\label{def:equivalent}
	Given two \hyperref[def:formula]{formulas} \(\varphi , \psi \), \(\varphi (\overline{x} )\) and \(\psi (\overline{x} )\) are \emph{equivalent} if
	\[
		\models \forall \overline{x} \ \big(\varphi (\overline{x} ) \leftrightarrow \psi (\overline{x} )\big).
	\]
\end{definition}

\begin{problem*}
	Two \hyperref[def:sentence]{sentences} \(\varphi \) and \(\psi \) are \hyperref[def:equivalent]{equivalent} if and only if \(\varphi \models \psi \) and \(\psi \models \varphi \).\todo{DIY}
\end{problem*}

\begin{prev}
	\(\mathcal{A} \) is a \hyperref[def:substructure]{substructure} of \(\mathcal{B} \), or \(\mathcal{A} \subseteq \mathcal{B} \), means that \(A \subseteq B\) and \(\identity\colon A \hookrightarrow B \) is an \hyperref[def:embedding]{\(\mathcal{L} \)-embedding}.
\end{prev}

\begin{proposition}\label{prop:lec3-1}
	Suppose that \(\mathcal{A} \) is a \hyperref[def:substructure]{substructure} of \(\mathcal{B} \), and \(\varphi (\overline{x} )\) is a \hyperref[not:quantifier-free-formula]{quantifier-free formula}. Let \(\overline{a} \in \mathcal{A} \),\footnote{Formally, we need to write \(\mathcal{A} \) to be the Cartesian product with a fixed length.} then \(\mathcal{A} \models \varphi (\overline{a} )\) if and only if \(\mathcal{B} \models \varphi (\overline{a} )\).
\end{proposition}
\begin{proof}
	We start with \hyperref[def:term]{terms} by proving that if \(t\) is a \hyperref[def:term]{term} and \(\overline{b} \in \mathcal{A} \), then \(t^{\mathcal{A} }(\overline{b} )=t^{\mathcal{B} }(\overline{B} ) \). The proof is induction on \hyperref[def:term]{terms}.
	\begin{enumerate}[(a)]
		\item If \(t\) is a constant symbol \(c\), then \(t^{\mathcal{A} }(\overline{b} ) = c^{\mathcal{A} } = c^{\mathcal{B} } = t^{\mathcal{B} } (\overline{b} )\).
		\item If \(t\) is a variable \(x_i\), then \(t^{\mathcal{A} }(\overline{b} ) = b_i = t^{\mathcal{B}} (\overline{b} )\).
		\item If \(t\) is a function symbol \(f(s_1, \ldots , s_n)\) where \(s_i\) are \hyperref[def:term]{terms}, then \(t^{\mathcal{A} }(\overline{b} )=f^{\mathcal{A} }\big(s_1^{\mathcal{A} }(\overline{b} ), \ldots , s_n^{\mathcal{A} }(\overline{b} ) \big) \). By the induction hypothesis, \(s_i^{\mathcal{A} }(\overline{b} )=s_i^{\mathcal{B} }(\overline{b} )\in \mathcal{A} \), and hence
		      \[
			      t^{\mathcal{B} }(\overline{b} )
			      = f^{\mathcal{B} }\big( s_1^{\mathcal{B} }(\overline{b} ), \ldots , s_n^{\mathcal{B} }(\overline{b} ) \big)
			      = f^{\mathcal{A} }\big( s_1^{\mathcal{A} }(\overline{b} ), \ldots , s_n^{\mathcal{A} }(\overline{b} ) \big)
			      = t^{\mathcal{A} }(\mathcal{B} ),
		      \]
		      i.e., \(f^{\mathcal{B} }\upharpoonright _{\mathcal{A} } = f^{\mathcal{A} } \), so \(t^{\mathcal{A} }(\overline{b} ) = t^{\mathcal{B} }(\overline{b} ) \).
	\end{enumerate}
	Now we turn to \hyperref[def:formula]{formulas}, and prove that for \(\varphi \) \hyperref[not:quantifier-free-formula]{quantifier-free}, then \(\mathcal{A} \models \varphi (\overline{a} ) \iff \mathcal{B} \models \varphi (\overline{a} )\) for \(\overline{a} \in \mathcal{A} \). The proof is, again, induction on \hyperref[def:formula]{formulas}.\footnote{Recall that we only need to show one of \(\lor \) or \(\land \), and here we pick \(\lor \) and treat \(\land \) as an abbreviation.}
	\begin{enumerate}[(a)]
		\item If \(\varphi \) is \(s=t\), then \(s^{\mathcal{A} }(\overline{a} )= s^{\mathcal{B} }(\overline{a} ) \) and \(t^{\mathcal{A} }(\overline{a} ) = t^{\mathcal{B} }(\overline{a} ) \), so
		      \[
			      \mathcal{A} \models \varphi (\overline{a} )
			      \iff s^{\mathcal{A} }(\overline{a} ) = t^{\mathcal{A} }(\overline{a} )
			      \iff s^{\mathcal{B} }(\overline{a} ) = t^{\mathcal{B} }(\overline{a} )
			      \iff \mathcal{B} \models \varphi (\overline{a} ).
		      \]
		\item If \(\varphi \) is \(R(s_1, \ldots , s_n)\), then
		      \[
			      A\models \varphi (\overline{a} )
			      \iff \big(s_1^{\mathcal{A} }(\overline{a} ), \ldots , s_n^{\mathcal{A} }(\overline{a} ) \big) \in R^{\mathcal{A} }
			      \iff \big(s_1^{\mathcal{B} }(\overline{a} ), \ldots , s_n^{\mathcal{B} }(\overline{a} )  \big)\in R^{\mathcal{B} }
			      \iff \mathcal{B} \models \varphi (\overline{a} ).
		      \]
		\item If \(\varphi \) is \(\lnot \psi \),
		      \[
			      \mathcal{A} \models \varphi (\overline{a} )
			      \iff \mathcal{A} \not\models \psi (\overline{a} )
			      \iff \mathcal{B} \not\models \psi (\overline{a} )
			      \iff \mathcal{B} \models \varphi (\overline{a} ),
		      \]
		      where we use the induction hypothesis in the second \(\iff \).
		\item If \(\varphi \) is \(\psi _1 \lor \psi _2\),
		      \[
			      \mathcal{A} \models \varphi (\overline{a} )
			      \iff \mathcal{A} \models \psi _1(\overline{a} ) \text{ or } \mathcal{A} \models \psi _2(\overline{a} )
			      \iff \mathcal{B} \models \psi _1(\overline{a} ) \text{ or } \mathcal{B} \models \psi _2(\overline{a} )
			      \iff \mathcal{B} \models \varphi (\overline{a} ),
		      \]
		      where we use the induction hypothesis in the second \(\iff \).
	\end{enumerate}
\end{proof}

\begin{prev}[Characteristic]
	Given a field \(K\), the \emph{characteristic} \(p\) of \(K\) is the number of \(1\) you need to add \(1\) in order to get \(0\), i.e., \(\underbrace{1+1+\ldots +1}_{p} = 0\).
\end{prev}

\begin{eg}
	Let \(L\) be a subfield of \(K\), for each \(p > 0\), \(\varphi _p \coloneqq \underbrace{1+1+\ldots +1}_{p} = 0\), which says the characteristic \(p\). \(\varphi _p\) is \hyperref[not:quantifier-free-formula]{quantifier-free}, so
	\[
		L \models \varphi _p \iff K \models \varphi _p.
	\]
\end{eg}

\begin{eg}
	Consider \(\mathbb{Z} =(\mathbb{Z} , 0, 1, +, -, \cdot)\), and let \(\varphi (x)\coloneqq \lnot \exists y\ y + y = x\). We see that \(\mathbb{Z} \models \varphi (1)\) but \(\mathbb{Q} \models \lnot \varphi (1)\).
\end{eg}

\begin{proposition}\label{prop:lec3-2}
	Suppose that \(\mathcal{A} \) is a \hyperref[def:substructure]{substructure} of \(\mathcal{B} \), and \(\varphi (\overline{x}, y_1, \ldots , y_n )\) is a \hyperref[not:quantifier-free-formula]{quantifier-free} \hyperref[def:formula]{formula}. Let \(\overline{a} \in \mathcal{A} \), then
	\begin{enumerate}[(a)]
		\item\label{prop:lec3-2(a)} if \(\mathcal{A} \models \exists y_1 \ldots \exists y_n\ \varphi (\overline{a} , y_1, \ldots , y_n)\), then \(\mathcal{B} \models \exists y_1\ldots \exists y_n\ \varphi (\overline{a} , y_1, \ldots , y_n)\);
		\item\label{prop:lec3-2(b)} if \(\mathcal{B} \models \forall y_1 \ldots \forall y_n\ \varphi (\overline{a} , y_1, \ldots , y_n)\), then \(\mathcal{A} \models \forall y_1 \ldots \forall y_n\ \varphi (\overline{a} , y_1, \ldots , y_n)\).
	\end{enumerate}
\end{proposition}
\begin{proof}
	Suppose that \(\mathcal{A} \models \exists y_1 \ldots \exists y_n\ \varphi (\overline{a} , y_1, \ldots , y_n)\), so there are \(b_1, \ldots , b_n \in \mathcal{A} \) such that \(\mathcal{A} \models \varphi (\overline{a} , b_1, \ldots , b_n)\). Since \(\varphi \) is \hyperref[not:quantifier-free-formula]{quantifier-free}, so \(\mathcal{B} \models \varphi (\overline{a} , b_1, \ldots , b_n)\) from \autoref{prop:lec3-1}, and hence \(\mathcal{B} \models \exists y_1 \ldots \exists y_n \ \varphi (\overline{a} , y_1, \ldots , y_n)\).

	On the other hand, it's easy to see that \autoref{prop:lec3-2(b)} is implied by \autoref{prop:lec3-2(a)}.
\end{proof}

\begin{notation}
	In \autoref{prop:lec3-2}, \hyperref[def:formula]{formulas} as in \autoref{prop:lec3-2(a)} are called \emph{existential (\(\exists _1\) or \(\exists \)) formulas}; and \hyperref[def:formula]{formulas} as in \autoref{prop:lec3-2(b)} are called \emph{universal (\(\forall _1\) or \(\forall \)) formulas}.
\end{notation}

\begin{eg}
	Recall \(\mathcal{L} _1 = \left\{ e, \cdot \right\} \), \(\mathcal{L} _2=\left\{ e, \cdot, ^{-1} \right\} \).
	\begin{itemize}
		\item Associativity: \(\forall x \forall y \forall z\ (xy)z = x(yz)\).
		\item Identity: \(\forall x\ ex = xe\).
	\end{itemize}
	These are \(\forall \)-\hyperref[def:formula]{formulas} in either \hyperref[def:language]{language}.
	\begin{itemize}
		\item Inverses in \(\mathcal{L} _1\): \(\forall x \exists y\ xy = yx = e\), which is \textbf{not} an \(\forall \)-\hyperref[def:formula]{formula}.
		\item Inverses in \(\mathcal{L} _2\): \(\forall x\ x x ^{-1} = x ^{-1} x = e\), which is an \(\forall \)-\hyperref[def:formula]{formula}.
	\end{itemize}
	Hence, group axioms in \(\mathcal{L} _1\) are not universal, but in \(\mathcal{L} _2\) they are.
\end{eg}

The above discrepancy is the reason why \(\mathcal{L} _2\) is better than \(\mathcal{L} _1\), i.e., \hyperref[def:substructure]{\(\mathcal{L} _1\)-substructure} might not be a group.

\begin{problem*}
	Show that \(\forall x \exists y\ xy = yx = e\) in the above example is not \hyperref[def:equivalent]{equivalent} to an \(\forall \)-\hyperref[def:formula]{formula}.
\end{problem*}