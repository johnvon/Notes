\lecture{17}{9 Mar. 14:30}{Quantifier Elimination}
\begin{eg}
	\(\varphi (x, y) \coloneqq \exists u\ (u > x \land u < y)\). This is equivalent, in \(\DLO\), to \(\psi (x, y) \coloneqq x < y\).
\end{eg}

\begin{eg}
	In Problem Set 2, we have looked at \(\mathcal{L} = \varnothing \). There, we showed that if \(A \subseteq B\) and both infinite, then \(A \preceq B\). The same idea show that \(T = \left\{ \exists x_1 \ldots \exists x_{n} \bigwedge_{i\neq j} x_i \neq x_j \mid n \in \mathbb{N} \right\} \) admits \hyperref[def:quantifier-elimination]{quantifier elimination}.
\end{eg}

\begin{proposition}
	Let \(T\) be a \hyperref[def:theory]{theory} that admits \hyperref[def:quantifier-elimination]{quantifier elimination}. If \(A, B \models T\) and \(A \subseteq B\), then \(A \preceq B\).
\end{proposition}
\begin{proof}
	Given \(\phi (\overline{x} )\) and \(\overline{a} \in A\), then \(\phi (x)\) is equivalent to a \hyperref[not:quantifier-free]{quantifier free} \(\psi (\overline{x} )\) (modulo \(T\)). Then,
	\[
		A \models \phi (\overline{a} )
		\iff A \models \psi (\overline{a} )
		\iff B \models \psi (\overline{a} )
		\iff B \models \phi (\overline{a} ).
	\]
\end{proof}

Next, we want to show that \(\ACF\) admits \hyperref[def:quantifier-elimination]{quantifier elimination}. First, we need a test for \hyperref[def:quantifier-elimination]{quantifier elimination}.

\begin{theorem}\label{thm:QE-test}
	Let \(\mathcal{L} \) include at least one constant symbol \(c\). Let \(T\) be an \hyperref[def:theory]{\(\mathcal{L} \)-theory}, and \(\phi (\overline{x} )\) an \hyperref[def:formula]{\(\mathcal{L} \)-formula}. Then the following are equivalent.
	\begin{enumerate}[(a)]
		\item\label{thm:QE-test-a} There is a \hyperref[not:quantifier-free]{quantifier free} \(\psi (\overline{x} )\) such that \(T \models \forall \overline{x} \ (\phi (\overline{x} ) \leftrightarrow \psi (\overline{x} ))\).
		\item\label{thm:QE-test-b} If \(\mathcal{M} , \mathcal{N} \models T\), and \(\mathcal{A} \) is a common \hyperref[def:substructure]{substructure}, then for all \(\overline{a} \in A\), \(\mathcal{M} \models \phi (\overline{a} )\) if and only if \(\mathcal{N} \models \phi (\overline{a} )\).
		% https://q.uiver.app/?q=WzAsMyxbMCwwLCJcXG1hdGhjYWx7TX0gXFxtb2RlbHMgVCJdLFsyLDAsIlxcbWF0aGNhbHtOfSBcXG1vZGVscyBUIl0sWzEsMSwiXFxtYXRoY2Fse0F9XFxuaSBcXG92ZXJsaW5le2F9Il0sWzAsMiwiIiwwLHsic3R5bGUiOnsiaGVhZCI6eyJuYW1lIjoibm9uZSJ9fX1dLFsyLDEsIiIsMCx7InN0eWxlIjp7ImhlYWQiOnsibmFtZSI6Im5vbmUifX19XV0=
		\[\begin{tikzcd}
				{\mathcal{M} \models T} && {\mathcal{N} \models T} \\
				& {\mathcal{A}\ni \overline{a}}
				\arrow[no head, from=1-1, to=2-2]
				\arrow[no head, from=2-2, to=1-3]
			\end{tikzcd}\]
	\end{enumerate}
\end{theorem}
\begin{proof}
	\autoref{thm:QE-test-a} implies \autoref{thm:QE-test-b} is easy, since we have
	\[
		\mathcal{M} \models \phi (\overline{a} )
		\iff \mathcal{M} \models \psi (\overline{a} )
		\iff \mathcal{A} \models \psi (\overline{a} )
		\iff \mathcal{N} \psi (\overline{a} )
		\iff \mathcal{N} \models \phi (\overline{a} ).
	\]
	To show \autoref{thm:QE-test-b} implies \autoref{thm:QE-test-a}, we see that first, deal with
	\begin{itemize}
		\item \(T \models \forall \overline{x} \ \phi (\overline{x} )\): take \(\psi \) to be \(c = c\);
		\item \(T \models \forall \overline{x} \ \lnot \phi (\overline{x} )\): take \(\psi \) to be \(c \neq c\).
	\end{itemize}
	Now, suppose we are not in the above two cases. Let \(\Gamma (\overline{x} )\) be the set of all \hyperref[not:quantifier-free]{quantifier free} \hyperref[def:formula]{formulas} \(\psi (\overline{x} )\) such that \(T \models \forall \overline{x} \ (\phi (\overline{x} ) \to \psi (\overline{x} ))\), and let \(\overline{d} = (d_1, \ldots , d_n)\) be new constant symbols.
	\begin{claim}
		It's enough to show \(T \cup \Gamma (\overline{d} ) \models \phi (\overline{d} )\).
	\end{claim}
	\begin{explanation}
		If we can do this, then by \hyperref[thm:compactness]{compactness}, there are \(\psi _1(\overline{x}) , \ldots , \psi _m(\overline{x} )\) such that
		\[
			T \models \left[ \psi _1(\overline{d} ) \land \ldots \land \psi _m(\overline{d} ) \right] \to \phi (\overline{d} )
		\]
		By the choice of \(\Gamma \), the \(\gets\) direction also holds, hence
		\[
			T \models \left[ \psi _1(\overline{d} ) \land \ldots \land \psi _m(\overline{d} ) \right] \leftrightarrow \phi (\overline{d} ),
		\]
		so take \(\psi (x) \coloneqq \psi _1(\overline{x} ) \land \ldots \land \psi _m(\overline{x} )\).
	\end{explanation}

	Now, we show \(T\cup \Gamma (\overline{d} ) \models \phi (\overline{d} )\). Suppose not. Then \(T \cup \Gamma (\overline{d} ) \cup \left\{ \lnot \phi (\overline{d} ) \right\} \) is \hyperref[def:satisfiable]{satisfiable}.\footnote{Notice that \(T \cup \Gamma (\overline{d} )\) needs to be \hyperref[def:satisfiable]{satisfiable}, which is true from our assumption.} Let \(\mathcal{M}\) be a \hyperref[def:model]{model} of \(T \cup \Gamma (\overline{d} ) \cup \left\{ \lnot \phi (\overline{d} ) \right\} \), and let \(\mathcal{A} \subseteq \mathcal{M} \) be the ``submodel''\footnote{\(\mathcal{A} \) is the smallest \hyperref[def:model]{model} containing \(\overline{d} \), and \(\overline{a} \) might not be a \hyperref[def:model]{model} of \(T\).} generated by \(\overline{d} \). Notice that since \(c^{\mathcal{A} } \in \mathcal{A} \), so \(\mathcal{A} \neq \varnothing \). Every element of \(\mathcal{A} \) is \(t^{\mathcal{A} }(\overline{d} ) = t^{\mathcal{M} }(\overline{d} ) \) for some \hyperref[def:term]{term} \(t\). Also, \(\mathcal{A} \models \Gamma (\overline{d} )\) because \(\mathcal{M} \models \Gamma (\overline{d} )\) and \(\Gamma (\overline{d} )\) is \hyperref[not:quantifier-free]{quantifier free}. We now show that \(T \cup \mathop{\mathrm{Diag}}(\mathcal{A} ) \cup \left\{ \phi (\overline{d} ) \right\} \) is \hyperref[def:satisfiable]{satisfiable}, and take \(\mathcal{N} \) to be a \hyperref[def:model]{model}. If not, some finite subset is not \hyperref[def:satisfiable]{satisfiable}. There are \hyperref[not:quantifier-free]{quantifier free} \hyperref[def:formula]{formulas} \(\gamma _1(\overline{d} ), \ldots , \gamma _\ell (\overline{d} )\) in \(\mathop{\mathrm{Diag}}(\mathcal{A} ) \) such that\footnote{We can take \(\gamma _i(\overline{d} )\) to be just about \(\overline{d} \) but not \(\gamma _i(\overline{a} )\) for \(\overline{a} \in A\) is because each \(\overline{a} \in A\) is \(t^{\mathcal{A} } (\overline{d} )\), so we can replace \(\overline{a} \) by this \hyperref[def:term]{term}.}
	\[
		T \models \left[ \gamma _1(\overline{d} ) \land \ldots \land \gamma _\ell (\overline{d} ) \right] \to \lnot \phi (\overline{d} ).
	\]
	Equivalently, \(T \models \phi (\overline{d} ) \to \lnot \left[ \gamma _1(\overline{d} ) \land \ldots \land \gamma _\ell (\overline{d} ) \right] \), so \(T \models \forall \overline{x}\ \left( \phi (\overline{x} ) \to \left[ \lnot \gamma _1(\overline{x} ) \lor \ldots \lor \lnot \gamma _\ell (\overline{x} ) \right]  \right) \). This implies that \(\lnot \gamma _1(\overline{x} ) \lor \ldots \lor \lnot \gamma _\ell (\overline{x} )\) is in \(\Gamma (\overline{x} )\), hence \(\mathcal{A} \models \lnot \gamma _1(\overline{d} ) \lor \ldots \lor \lnot \gamma _\ell (\overline{d} )\). But \(\mathcal{A} \models \gamma _1(\overline{d} ) \land \ldots \land \gamma _\ell (\overline{d} )\), contradicts to the fact that \(\gamma _1(\overline{d} ), \ldots , \gamma _\ell (\overline{d} )\) is in \(\mathop{\mathrm{Diag}}(\mathcal{A} ) \).

	Hence, there is \(\mathcal{N} \models T \cup \mathop{\mathrm{Diag}}(\mathcal{A} ) \cup \left\{ \phi (\overline{d} ) \right\} \). This contradicts to \autoref{thm:QE-test-b}:
	% https://q.uiver.app/?q=WzAsMyxbMCwwLCJcXG1hdGhjYWx7TX1cXG1vZGVscyBcXGxub3RcXHBoaShcXG92ZXJsaW5le2R9KVxcdGV4dHsgYW5kIH1cXG1hdGhjYWx7TX1cXG1vZGVscyBUIl0sWzIsMCwiXFxtYXRoY2Fse059XFxtb2RlbHMgXFxsbm90XFxwaGkoXFxvdmVybGluZXtkfSlcXHRleHR7IGFuZCB9XFxtYXRoY2Fse059XFxtb2RlbHMgVCJdLFsxLDEsIlxcbWF0aGNhbHtBfVxcbmkgXFxvdmVybGluZXtkfSJdLFsyLDEsIiIsMCx7InN0eWxlIjp7InRhaWwiOnsibmFtZSI6Imhvb2siLCJzaWRlIjoidG9wIn19fV0sWzIsMCwiXFxzdWJzZXRlcSIsMl1d
	\[\begin{tikzcd}
			{\mathcal{M}\models \lnot\phi(\overline{d})\text{ and }\mathcal{M}\models T} && {\mathcal{N}\models \lnot\phi(\overline{d})\text{ and }\mathcal{N}\models T} \\
			& {\mathcal{A}\ni \overline{d}}
			\arrow[hook, from=2-2, to=1-3]
			\arrow["\subseteq"', from=2-2, to=1-1]
		\end{tikzcd}\]
	Hence, \(T \cup \Gamma (\overline{d} ) \models \phi (\overline{d} )\).
\end{proof}

\begin{lemma}\label{lma:QE-test}
	Let \(T\) be an \hyperref[def:theory]{\(\mathcal{L} \)-theory}. Suppose that for every \hyperref[not:quantifier-free]{quantifier free} \hyperref[def:formula]{formula} \(\gamma (\overline{x} , y)\), there is a \hyperref[not:quantifier-free]{quantifier free} \hyperref[def:formula]{formula} \(\psi (\overline{x} )\) such that
	\[
		T \models \forall \overline{x} \ (\exists y\ \gamma (\overline{x} , y) \leftrightarrow \psi (\overline{x} )),
	\]
	then \(T\) has \hyperref[def:quantifier-elimination]{quantifier elimination}.
\end{lemma}
\begin{proof}
	By induction on \hyperref[def:formula]{formulas}, we use the hypotheses of \autoref{thm:QE-test} for the \(\exists \) quantifier case.
\end{proof}

\begin{corollary}\label{col:QE-test}
	Let \(T\) be an \hyperref[def:theory]{\(\mathcal{L} \)-theory} with at least one constant. Suppose that for all \hyperref[not:quantifier-free]{quantifier free} \(\psi (\overline{x} , y)\). If \(\mathcal{M} , \mathcal{N} \models T\), and \(\mathcal{A} \subseteq \mathcal{M} \) and \(\mathcal{A} \subseteq \mathcal{N} \), \(\overline{a} \in A\) and \(b\in M\)\footnote{This is \autoref{thm:QE-test-b} for \(\exists y\ \psi (\overline{x} , y)\).} is such that \(\mathcal{M} \models \psi (\overline{a} , b)\), then there is \(c\in N\) such that \(\mathcal{N} \models \psi (\overline{a} , c)\). Then \(T\) admits \hyperref[def:quantifier-elimination]{quantifier elimination}.
	% https://q.uiver.app/?q=WzAsMyxbMCwwLCJcXG1hdGhjYWx7TX1cXG5pIGIiXSxbMiwwLCJcXG1hdGhjYWx7Tn1cXG5pIGMiXSxbMSwxLCJcXG1hdGhjYWx7QX1cXG5pIFxcb3ZlcmxpbmV7YX0iXSxbMiwxLCJcXHN1YnNldGVxIl0sWzIsMCwiXFxzdWJzZXRlcSIsMl0sWzAsMSwiIiwyLHsiY3VydmUiOi0yLCJzdHlsZSI6eyJib2R5Ijp7Im5hbWUiOiJkYXNoZWQifX19XV0=
	\[\begin{tikzcd}
			{\mathcal{M}\ni b} && {\mathcal{N}\ni c} \\
			& {\mathcal{A}\ni \overline{a}}
			\arrow["\subseteq", from=2-2, to=1-3]
			\arrow["\subseteq"', from=2-2, to=1-1]
			\arrow[curve={height=-12pt}, dashed, from=1-1, to=1-3]
		\end{tikzcd}\]
\end{corollary}
\begin{proof}
	By combining \autoref{lma:QE-test} and \autoref{thm:QE-test}.
\end{proof}

\subsection{Quantifier Elimination for Algebraically Closed Fields}

\begin{theorem}
	\(\ACF\) admits \hyperref[def:quantifier-elimination]{quantifier elimination}.
\end{theorem}
\begin{proof}
	We use \autoref{col:QE-test}. Suppose that \(\mathcal{M} , \mathcal{N} \models \ACF\), and \(\mathcal{A} \subseteq \mathcal{M} , \mathcal{N} \) an \hyperref[def:integral-domain]{integral domain}. Let \(\psi (\overline{x} , y)\) be a \hyperref[not:quantifier-free]{quantifier free} \hyperref[def:formula]{formula}, \(\overline{a} \in \mathcal{A} \), \(b\in \mathcal{M} \). Suppose \(\mathcal{M} \models \psi (\overline{a} , b)\).

	Firstly, we replace \(\mathcal{N} \) by the \hyperref[def:algebraic-closure]{algebraic closure} of \(\mathcal{A} \), then we can also assume that \(\mathcal{N} \subseteq \mathcal{M} \) since any \hyperref[def:algebraic-closure]{algebraic closure} of \(\mathcal{A} \) embeds in any \hyperref[def:algebraically-closed]{algebraically closed} field containing \(\mathcal{A} \).
	% https://q.uiver.app/?q=WzAsMyxbMCwwLCJcXG1hdGhjYWx7TX0gXFxuaSBiIl0sWzIsMCwiXFxtYXRoY2Fse059Il0sWzEsMSwiXFxtYXRoY2Fse0F9XFxuaSBcXG92ZXJsaW5le2F9Il0sWzIsMV0sWzIsMF0sWzEsMCwiIiwwLHsic3R5bGUiOnsidGFpbCI6eyJuYW1lIjoiaG9vayIsInNpZGUiOiJib3R0b20ifSwiYm9keSI6eyJuYW1lIjoiZGFzaGVkIn19fV1d
	\[
		\begin{tikzcd}
			{\mathcal{M} \ni b} && {\mathcal{N}} \\
			& {\mathcal{A}\ni \overline{a}}
			\arrow[from=2-2, to=1-3]
			\arrow[from=2-2, to=1-1]
			\arrow[dashed, hook', from=1-3, to=1-1]
		\end{tikzcd}
	\]

	What does \(\psi \) look like? We may assume that it is \(\theta _1(\overline{x} , y) \lor \theta _2(\overline{x} , y) \lor \ldots \), where each \(\theta _i\) is a conjunction of \hyperref[not:atomic]{atomic} and negated \hyperref[not:atomic]{atomic} \hyperref[def:formula]{formula}.\footnote{They look like \(s = t\) for \hyperref[def:term]{terms} \(s, t\), which are essentially polynomials. So an \hyperref[not:atomic]{atomic} \hyperref[def:formula]{formula} with variables \(\overline{x} , y\) is essentially \(p(\overline{x} , y) = 0\) where \(p\in \mathbb{Z} [\overline{x} , y]\).} We can replace \(\psi \) by whichever \(\theta _i\) is satisfied by \(b\), so \(\psi (\overline{x} , y)\) is equal to
	\[
		p_1(\overline{x} , y) g 0 \land \ldots \land p_k(\overline{x} , y) = 0 \land q_1(\overline{x} , y) \neq 0 \land q_{\ell } (\overline{x} , y) \neq 0
	\]
	for \(p_i, q_i\in \mathbb{Z} [\overline{x} , y]\). Then \(\psi (\overline{a} , y)\) says \(p_1(\overline{a} , y) = 0 \land \ldots\) where \(p_i(\overline{a} , y)\) are now in \(\mathcal{A} [y]\), i.e., polynomial in \(y\) with coefficient in \(\mathcal{A} \). If any \(p_i(\overline{a} , y)\) is non-trivial, then \(b\) is a solution, so \(b\in \mathcal{N} \), the \hyperref[def:algebraic-closure]{algebraic closure} of \(\mathcal{A}\). Otherwise, \(\psi (\overline{a} ,y)\) is just
	\[
		q_i(\overline{a} , y) \neq 0 \land \ldots \land q_{\ell } (\overline{a} , y) \neq 0.
	\]
	Since \(b\) satisfies this, each \(q_i(\overline{a} , y)\) is non-trivial, so \(q_i(\overline{a} , y) = 0\) has only finitely many solutions. But \(\mathcal{N} \) is infinite, so there is a \(c\in \mathcal{N} \) that is not a solution to any \(q_i(\overline{a} , y) = 0\), so \(\mathcal{N} \models \psi (\overline{a} , c)\).
\end{proof}