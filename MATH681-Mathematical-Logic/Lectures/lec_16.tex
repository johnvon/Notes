\lecture{16}{7 Mar. 11:30}{Quantifier Elimination}
\begin{remark}
	\(T\) is actually \hyperref[def:countably-categorical]{countably categorical}.
\end{remark}
\begin{explanation}
	We will show this on homework. But this is basically \autoref{thm:DLO-countably-categorical}.
\end{explanation}

Now we show that it is also \hyperref[def:satisfiable]{satisfiable}. Fix \(N\), \(p\in (0, 1)\), and vertices \(\left\{ 1, \dots , N \right\} \). Generate a \hyperref[def:random-graph]{random graph} \(G_N\) on vertices \(\left\{ 1, \dots , N \right\} \) by, for each \(i, j \leq N\), \(i \neq j\), putting an edge between \(i\) and \(j\) with probability \(p\) and no edge with probability \(1 - p\). Let \(\mathcal{G} _N\) be the class of all such graphs, and let \(p_N(\varphi )\) be the probability that a \hyperref[def:random-graph]{randomly generated graph} on \(N\) vertices satisfies \(\varphi \).

\begin{eg}
	If \(p = 1 / 2\), all graphs in \(\mathcal{G} _N\) are equally likely.
\end{eg}

Observe the following.

\begin{proposition}\label{prop:random-graph-almost-surely-satisfiable}
	For every \(n\), then \(\lim_{N \to \infty} p_N(\psi _n) = 1\).
\end{proposition}
\begin{proof}
	Fix \(n\), and let \(G\) be a \hyperref[def:random-graph]{random graph} in \(\mathcal{G} _N\), \(N > 2n\). Fix \(x_1, \dots , x_n\), and \(y_1, \dots , y_n\), and \(z\) in \(G\), all distinct. Consider the probability \(q\) that
	\[
		\lnot \left( \bigwedge_{i=1}^{n} E(x_i, z) \land \lnot E(y_i, z)\right),
	\]
	which is just \(1 - p^n (1-p)^n \coloneqq q\) as we already know. The probability that
	\[
		G \models \underbrace{\lnot \exists z}_{\forall z\ \lnot }\ \left( \bigwedge_{i=1}^{n} E(x_i, z) \land \lnot E(y_i, z) \right)
	\]
	is \(q^{N-2n}\), where \(N-2n\) is the number of possible \(z\)'s.\footnote{Notice that we already assume that \( x_i \neq z \land y_i \neq z\).} Let \(M\) be the number of different choices of \(x_i\)'s and \(y_i\)'s, then we have
	\[
		p_N(\lnot \psi _n)
		\leq M\cdot q^{N - 2n}
		\leq N^{2n} q^{N-2n}
	\]
	by the union bound.\footnote{This bound is not allowing the \(x\)'s and \(y\)'s to have repetitions. But this doesn't change anything.} Since \(0 < q < 1\), so \(p_N (\lnot \psi _n) \to 0\), i.e., \(p_N(\psi _n) \to 1\) as \(N \to \infty \).
\end{proof}

Then, we have the following.

\begin{theorem}
	\(T\) is \hyperref[def:satisfiable]{satisfiable}.
\end{theorem}
\begin{proof}
	From \autoref{prop:random-graph-almost-surely-satisfiable}, \(\lim_{N \to \infty} p_N(\psi _n) = 1\). In particular, for each \(n\), there is \(N\) such that \(p_N(\psi _n) > 0\), i.e., there is at least one \(G\in \mathcal{G} _N\) such that \(G \models \psi _n\), hence \(G\models \psi _m\) for \(m \leq n\).\footnote{Remember that for \(n > m\), \(\models \psi _n \to \psi _m\).} This means that for every finite \(T^{\ast} \subseteq T\) is \hyperref[def:satisfiable]{satisfiable}, so \(T\) is \hyperref[def:satisfiable]{satisfiable} by the \hyperref[thm:compactness]{compactness theorem}.
\end{proof}

Moreover, we have the following.

\begin{theorem}[Zero-one law for graphs]\label{thm:0-1-law-for-graph}
	For any \hyperref[def:sentence]{\(\mathcal{L} \)-sentence} \(\varphi \), we either have \(\lim\limits_{N \to \infty} p_N(\varphi ) = 0 \) or \(\lim\limits_{N \to \infty} p_N(\varphi )=1\). Moreover, \(T\) axiomatizes \(\left\{ \varphi \colon \lim_{N \to \infty} p_N(\varphi ) = 1 \right\} \), the ``almost sure theory for graphs'', which is \hyperref[def:decidable]{decidable} and \hyperref[def:theory-complete]{complete}.
\end{theorem}
\begin{proof}
	Since \(T\) is \hyperref[def:countably-categorical]{countably categorical}, and has only infinite models, so it is \hyperref[def:theory-complete]{complete} by the \hyperref[thm:Vaught-test]{Vaught's test}. If \(T \models \varphi \), there is some \(n\) such that \(\left\{ \psi _n, \hyperref[def:random-graph-theory-a]{(a)}, \hyperref[def:random-graph-theory-a]{(b)} \right\} \models \varphi \). Since \(p_N(\psi _n) \leq p_N(\varphi )\), so \(\lim_{N \to \infty} p_N(\varphi ) = 1\) and the left-hand side goes to \(1\) as well. On the other hand, if \(T \not \models \varphi \), \(T \models \lnot \varphi \), so \(\lim_{N \to \infty} p_N(\lnot \varphi ) = 1\), i.e., \(\lim_{N \to \infty} p_N(\varphi ) = 0\) from the same argument.
\end{proof}

Before ending this section, we note that there are other things we may explore.

\begin{remark}
	There are also Fraïssé constructions, and Ryll-Nardzewski theorem, etc.
\end{remark}

\chapter{Quantifier Elimination and Algebraic Applications}
\section{Quantifier Elimination}
Let's start with a definition.

\begin{definition}[Quantifier elimination]\label{def:quantifier-elimination}
	A \hyperref[def:theory]{theory} \(T\) has (admits) \emph{quantifier elimination} if for every \hyperref[def:formula]{formula} \(\varphi (\overline{x} )\) (with \(\overline{x} \) containing at least one variable), there is a \hyperref[not:quantifier-free]{quantifier-free} \(\psi (\overline{x} )\) such that
	\[
		T \models \forall \overline{x} \ (\varphi (\overline{x} ) \leftrightarrow \psi (\overline{x} )).
	\]
\end{definition}

The easiest example will be that \(\DLO\) admits \hyperref[def:quantifier-elimination]{quantifier elimination}. To prove this, we need some preliminaries.

\begin{note}
	\(\DLO\) has no constant symbols, so it has no \hyperref[not:quantifier-free]{quantifier-free} \hyperref[def:sentence]{sentences}. For example,
	\[
		\forall x\exists y\ y < x \leftrightarrow x = x,
	\]
	where the right-hand side has a \hyperref[def:free-variable]{free variable}. Another solution would be to allow \(\top\) and \(\perp \) for the \hyperref[def:truth]{true} and false \hyperref[def:sentence]{sentences}.
\end{note}

\begin{lemma}\label{lma:DLO-isomorphism}
	Let \((A, \leq )\) and \((B, \leq )\) be countable \(\DLO\)s. Let \(a_1, \dots , a_n\in A\) have \(a_1 < a_2 < \dots < a_n\) and \(b_1, \dots , b_n \in B\) have \(b_1 < \dots < b_n\). Then there is an \hyperref[def:isomorphism]{isomorphism} \(f\colon A \to B\) which maps \(a_i \mapsto b_i\). Hence,
	\[
		A \models \varphi (\overline{a} ) \iff B \models \varphi (\overline{b} ).
	\]
\end{lemma}
\begin{proof}
	This is the same as the \hyperref[not:back-and-forth]{back-and-forth} argument, but now we start with \(\overline{a} \mapsto \overline{b} \).
\end{proof}

\begin{theorem}\label{thm:DLO-QE}
	\(\DLO\) admits \hyperref[def:quantifier-elimination]{quantifier elimination}.
\end{theorem}
\begin{proof}
	Fix \(\varphi (\overline{x} )\). If \(\varphi \) is actually a \hyperref[def:sentence]{sentence} then we're done since either
	\begin{itemize}
		\item \(\DLO \models \varphi \), so \(\DLO \models \forall x\ (\varphi \leftrightarrow x = x)\), or
		\item \(\DLO \models \lnot \varphi \), so \(\DLO \models \forall x\ (\varphi \leftrightarrow x \neq x)\).
	\end{itemize}
	Now suppose \(\varphi (\overline{x} )\) has at least one \hyperref[def:free-variable]{free variable}, \(\overline{x} = (x_1, \dots , x_n)\). Since \(\DLO\) is \hyperref[def:theory-complete]{complete}, it's enough to find \(\psi (\overline{x} )\) with \(\mathbb{Q} \models \forall \overline{x} \ (\varphi (\overline{x} ) \leftrightarrow \psi (\overline{x} ))\). Let \(\sigma \) be a map from pairs \(i, j\) to \(\left\{ 1, 2, 3 \right\} \). Define\footnote{It's clear that some \(\theta _\sigma \) might be \hyperref[def:inconsistent]{inconsistent}.}
	\[
		\theta _\sigma (x_1, \dots , x_n) \coloneqq \bigwedge_{\sigma (i, j) = 1} x_i < x_j \land \bigwedge_{\sigma (i, j)=2} x_i = x_j \land \bigwedge_{\sigma (i, j)=3} x_j < x_i.
	\]
	If \(\mathbb{Q} \models \theta _\sigma (\overline{a} ) \land \theta _\sigma (\overline{b} )\), then \(\overline{a} \) and \(\overline{b} \) satisfy the same \hyperref[def:formula]{formulas} by \autoref{lma:DLO-isomorphism}, so\footnote{The second equality follows from the fact that if one such \(\overline{x} \) exists, all such \(\overline{x} \) work.}
	\[
		\Sigma
		\coloneqq \left\{ \sigma \mid \mathbb{Q} \models \exists \overline{x}\ (\theta _\sigma (\overline{x} ) \land \varphi (\overline{x} )) \right\}
		= \left\{ \sigma \mid \mathbb{Q} \models \forall \overline{x}\ (\theta _\sigma (\overline{x} ) \to \varphi (\overline{x} )) \right\}.
	\]
	If \(\Sigma = \varnothing \), then \(\varphi (x) \leftrightarrow x \neq x\); if \(\Sigma \neq \varnothing \), let \(\psi (\overline{x} ) = \bigvee_{\sigma \in \Sigma } \theta _\sigma (\overline{x} ) \), then \(\mathbb{Q} \models \forall \overline{x} \ (\varphi (\overline{x} ) \leftrightarrow \psi (\overline{x} ))\).
\end{proof}