\lecture{39}{15 Apr. 11:00}{Introduction to Fourier Analysis}
\begin{prev}
	\hyperref[note:Planchenel-identity]{Planchenel identity}  says that if \(e_{i} \) is an \hyperref[def:orthonormal-basis]{orthonormal basis},
	then
	\[
		\left< x, y \right> = \sum_{i=1}^{\infty} \alpha _{n} \overline{\beta} _n
	\]
	where \(\alpha _n = \left< x, e_{n}  \right> \), \(\beta _n = \left< y, e_{n}  \right> \).
\end{prev}

\chapter{Introduction to Fourier Analysis}\label{ch:Introduction-to-Fourier-Analysis}
\section{Fourier Series}
We'll be considering the \hyperref[def:Hilbert-space]{Hilbert space} \(L^2([-\pi , \pi ])\), which by scaling is equivalent to
any finite interval. Let's start with a lemma.
\begin{lemma}\label{lma:lec-39}
	The set
	\[
		\left\{e_{n} (x) = \frac{1}{\sqrt{2\pi } }e^{inx} = \frac{1}{\sqrt{2\pi }}\left(\cos (nx) + i\sin (nx)\right)\right\}
	\]
	is an \hyperref[def:orthonormal-set]{orthonormal set} in \(L^2([-\pi , \pi ])\).
\end{lemma}
\begin{proof}
	We see that
	\[
		\left< e_{n} , e_{m}  \right> = \frac{1}{2\pi } \int_{-\pi}^{\pi } e^{i(m-n)x} \,\mathrm{d}x = \begin{dcases}
			1, & \text{ if } m = n ;    \\
			0, & \text{ if } m \neq n ,
		\end{dcases}
	\]
	hence, the result follows directly.
\end{proof}

Now, a natural problem arises:
\begin{problem}
Is \(\{e_{n} \}\) defined in \autoref{lma:lec-39} an \hyperref[def:orthonormal-basis]{orthonormal basis}?
\end{problem}
\begin{answer}
	By \hyperref[thm:Holder-inequality]{Hölder's inequality} on \(L^{2} ([-\pi , \pi ])\), we have
	\[
		\left\lVert f\right\rVert _1 \leq \left\lVert 1\right\rVert _2 \left\lVert f\right\rVert _2 = \sqrt{2\pi } \cdot \left\lVert f\right\rVert _2 < \infty .
	\]
	Similarly, we have
	\[
		\left\lVert f\right\rVert _2 = \sqrt{\int_{-\pi}^{\pi} \left\vert f(t) \right\vert ^{2} \,\mathrm{d}t} \leq \sqrt{\left\lVert f\right\rVert ^{2} _\infty 2\pi },
	\]
	which implies
	\[
		\left\lVert f\right\rVert _1 \leq \sqrt{2\pi } \left\lVert f\right\rVert _2 \leq 2\pi \left\lVert f\right\rVert _\infty .
	\]
\end{answer}

\begin{definition}[Foruier coefficient]\label{def:Fourier-coefficient}
	For \(f\in L^{1}([-\pi , \pi ]) \), its \emph{\(n^{th}\) Fourier coefficients} is
	\[
		\hat{f} _n \coloneqq \left< f, e_{n}  \right> = \frac{1}{\sqrt{2\pi } }\int_{-\pi }^{\pi } f(y) e^{-iny}   \,\mathrm{d}y.
	\]
\end{definition}

We can show that \autoref{def:Fourier-coefficient} form an \hyperref[def:orthonormal-basis]{orthonormal basis}, we show that
\[
	\sum_{n=-M}^{N} \hat{f} _{n} e_{n} (x)
\]
\hyperref[def:strong-convergence]{converge strongly} to \(f(x)\) as \(M, N\to \infty \) Explicitly, this is true since
\[
	\sum_{n=-M}^{N} \hat{f} _{n} e_{n} (x) = \frac{1}{\sqrt{2\pi } }\sum_{n=-M}^{N} \left[\int_{-\pi}^{\pi} f(y) e^{-iny}  \,\mathrm{d}y \right]e^{inx} = \frac{1}{\sqrt{2\pi } } \int_{-\pi}^{\pi} f(y) \left(\sum_{n=-M}^{N} e^{in(x-y)} \right)\,\mathrm{d}y.
\]

\begin{definition}[Poisson kernel]\label{def:Poisson-kernel}
	For \(0 \leq r <1\), the \emph{Poisson kernel} is defined as
	\[
		P_r(t) = \frac{1}{2\pi } \sum_{n=-\infty }^{\infty} e^{int} r^{\left\vert n \right\vert }.
	\]
	Explicitly, summing as a geometric series gives us
	\[
		P_r(t) = \frac{1 - r^{2} }{2\pi (1 - 2r \cos t + r^{2} )}.
	\]
\end{definition}
\begin{lemma}
	For \(f\in L^1([-\pi , \pi ])\) and \(0 \leq r < 1\), for \(x\in [-\pi , \pi ]\),
	\[
		\sum_{n=-\infty }^{\infty} \hat{f} _{n} e_{n} (x) r^{\left\vert n \right\vert }
	\]
	converge absolutely and uniformly to
	\[
		\int_{-\pi}^{\pi} P_r(x - y) f(y) \,\mathrm{d}y.
	\]
\end{lemma}
\begin{proof}
	We have
	\[
		\sum_{n=-\infty }^{\infty} \left\vert \hat{f} _{n} e_{n} (x) r ^{\left\vert n \right\vert } \right\vert \leq \frac{1}{2\pi }\sum_{n=-\infty }^{\infty} \left(\int_{-\pi}^{\pi} \left\vert f (y) e^{-iny}  \right\vert  \,\mathrm{d}y \right) \left\vert e_{n} (x) \right\vert r^{\left\vert n \right\vert } = \frac{\left\lVert f\right\rVert _1}{2\pi }\sum_{n=-\infty }^{\infty} r^{\left\vert n \right\vert }< \infty .
	\]
	Therefore, \autoref{col:Tonelli-theorem-for-series-and-integrals} applies, and we further have
	\[
		\begin{split}
			\frac{1}{2\pi }\sum_{n=-\infty }^{\infty} \left(\int_{-\pi }^{\pi }f(y) e^{-iny} \,\mathrm{d}y \right)e^{inx} r^{\left\vert n \right\vert }
			&= \frac{1}{2\pi }\int_{-\pi }^{\pi} \left(f(y) \sum_{n=-\infty }^{\infty} e^{in(x-y)}r^{\left\vert n \right\vert }  \right) \,\mathrm{d}y\\
			&= \int _{-\pi }^\pi P_r(x-y) f(y)\,\mathrm{d} y.
		\end{split}
	\]
	Uniform convergence takes a small bit more work.\todo{DIY}
\end{proof}

\begin{note}
	Note that
	\[
		P_r(0)= \frac{1-r^{2} }{2\pi (1 - r)^{2} } = \frac{1 + r}{2\pi (1 - r)} \to 0 \text{ as }r \to 1.
	\]
	For any \(t\neq 0\) we have
	\[
		\frac{1-r^{2} }{1\pi (1 - 2r \cos t + r^{2} )}\to 0
	\]
	as the bottom is always nonzero and finite.
\end{note}

\begin{definition}[Good kernel]\label{def:good-kernel}
	A kernel \(K_r\) is said to be in a \emph{family\footnote{Respect to \(r\).} of good kernels} if
	\begin{enumerate}[(1)]
		\item \(K_r(t) \geq 0\).
		\item \(\int _{-\pi }^\pi K_r(t)\,\mathrm{d} t = 1\).
		\item For every \(\delta >0\),
		      \[
			      \int _{[-\pi , \pi ]\setminus [-\delta , \delta ]}K_{r} (t) \,\mathrm{d} t \to 0
		      \]
		      as \(r \to 1\).\footnote{This definition is not universal. Some text uses other convention.}
	\end{enumerate}
\end{definition}

\begin{lemma}
	\(P_r(t)\) form a \hyperref[def:good-kernel]{family of good kernels}.
\end{lemma}
\begin{proof}
	For \((2)\), we use the first formula with \hyperref[thm:Fubini-theorem]{Fubini's theorem}. For \((1)\) and \((3)\), we use the second formula, namely
	\[
		\int _{[-\pi , \pi ]\setminus [-\delta , \delta ]}P_r(t)\,\mathrm{d} t \leq \frac{1 - r^{2} }{2\pi (1 - 2r \cos \delta  + r^{2} )}2\pi \to 0
	\]
	as \(r \to 1\).
\end{proof}

\begin{lemma}
	For \(f\in C([-\pi , \pi ])\)  satisfying \(f(-\pi) = f(\pi )\), we have
	\[
		\lim\limits_{r \nearrow 1} \int _{-\pi }^\pi P_{r} (x - y)f(y)\,\mathrm{d} y = f(x)
	\]
	uniformly for \(x\in [-\pi , \pi ]\).
\end{lemma}
\begin{proof}
	We extend \(f\) to \(f\colon \mathbb{\MakeUppercase{r}} \to \mathbb{\MakeUppercase{c}} \) by setting \(f(x + 2\pi ) = f(x)\), then \(f\) is
	uniformly continuous and bounded. Also, we have
	\[
		\begin{split}
			&\int _{-\pi }^\pi P_{r} (x - y)f(y)\,\mathrm{d} y - f(x)\\
			= &\int _{-\pi }^\pi P_{r} f(x-y)\,\mathrm{d} y - f(x)\\
			= &\int _{-\pi }^\pi P_{r} (y) f(x-y) \,\mathrm{d} y - f(x)\int _{-\pi }^\pi P_{r} (y)\,\mathrm{d} y\\
			= &\int _{-\delta }^\delta P_{r} (y)(f(x-y) - f(x))\,\mathrm{d} y + \int _{[-\pi , \pi ]\setminus [-\delta , \delta ]}P_{r} (y)(f(x-y) - f(x))\,\mathrm{d} y.
		\end{split}
	\]
	Now, fix \(\epsilon >0\), then \(f\) is uniformly continuous, so we can choose a \(\delta >0\) so that \(\left\vert f(x-y)-f(x) \right\vert -\epsilon \) for
	any choice of \(x\). Then, the left-hand side is bounded by \(\epsilon \). As for the right-hand side, we note that \(f\) is bounded, so for some \(M\),
	we have
	\[
		\left\vert \int _{[-\pi , \pi ]\setminus [-\delta , \delta ]}P_{r} (y)(f(x-y)-f(x)) \,\mathrm{d} y\right\vert \leq M \int _{[-\pi , \pi ]\setminus [-\delta , \delta ]}P_{r} (y)\,\mathrm{d} y.
	\]
	Therefore, sending \(r \nearrow 1\) will give
	\[
		\int _{-\pi }^\pi P_{r} (x-y)f(y)\,\mathrm{d} y - f(x) \to 0,
	\]
	as we desired.
\end{proof}