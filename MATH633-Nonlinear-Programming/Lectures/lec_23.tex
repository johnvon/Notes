\chapter{Normed Vector Space}\label{ch:Normed-Vector-Space}
\lecture{23}{09 Mar. 11:00}{Metric, normed and \(L^p\) Spaces}
\section{Metric Spaces and Normed Spaces}
We have seen the definition of a \hyperref[def:norm]{norm} before, now we formally introduce the concept of \emph{metric}.
\begin{definition}[Metric]\label{def:metric}
	Let \(Y\) be a set, a function \(\rho \colon Y\times Y\to [0, \infty )\) is a \emph{metric} on \(Y\) if
	\begin{itemize}
		\item \(\rho (x, y) = \rho (y, x)\) for all \(x, y\in Y\).
		\item \(\rho (x, z) \leq \rho (x, y) + \rho (y, z)\) for all \(x, y, z\in Y\).
		\item \(\rho (x, y) = 0\) if and only if \(x = y\).
	\end{itemize}
\end{definition}

\begin{note}
	The following make sense in a \hyperref[def:metric]{metric} space.
	\begin{enumerate}[(1)]
		\item Open/closed balls.
		\item Open/closed sets.
		\item Convergence sequences (\(x_{n} \to x\) with respect to \(\rho \) if and only if \(\lim\limits_{n \to \infty} \rho (x_{n} , x) = 0\)).
		\item Continuous functions.
	\end{enumerate}
\end{note}

\begin{eg}
	We have the following \hyperref[def:metric]{metric} spaces.
	\begin{enumerate}[(1)]
		\item \(\mathbb{\MakeUppercase{q}}\) with \(\rho (x, y) = \left\vert x - y \right\vert \).
		\item \(\mathbb{\MakeUppercase{r}}\) with \(\rho (x, y) = \left\vert x - y \right\vert \).
		\item \(\mathbb{\MakeUppercase{r}}_+\) with \(\rho (x, y) = \left\vert \ln (y / x)\right\vert \).
		\item \(\mathbb{\MakeUppercase{r}}^d\) with
		      \[
			      \rho _p(x, y) = \left(\sum\limits_{i=1}^{d} \left\vert x_{i} - y_{i} \right\vert^p \right)^{1 / p}
		      \]
		      and
		      \[
			      \rho _\infty (x, y ) = \mathop{\max} _{1\leq i\leq d}\left\vert x_{i} - y_{i} \right\vert.
		      \]
		      These all give the same open sets, hence they are topologically equivalent.
		\item \(C([0, 1])\) with
		      \[
			      \rho _p(f, g) = \left(\int _0^1 \left\vert f-g \right\vert^p \right)^{1 / p}
		      \]
		      and
		      \[
			      \rho _\infty (f, g) = \mathop{\max} _{x\in[0, 1]}\left\vert f(x) - g(x) \right\vert.
		      \]
		\item Let \((X, \mathcal{\MakeUppercase{a}} , \mu )\) be a \hyperref[def:measure-space]{measure space} with \(\mu (X)< \infty \). Let \(Y\) be the set of
		      \hyperref[def:measurable-function]{measurable functions} on \(X\), then
		      \[
			      \rho (f, g) = \int \mathop{\min} \left\{\left\vert f(x) - g(x) \right\vert, 1 \right\}\,\mathrm{d} \mu (x)
		      \]
		      is a \hyperref[def:metric]{metric} and \(f_{n} \to f\) in \(\rho\) if and only if \hyperref[def:converge-in-measure]{\(f_n \to f\) in measure}.
	\end{enumerate}
\end{eg}

Let \(V\) be a vector space over scalar field \(K = \mathbb{\MakeUppercase{r}} \) or \(K = \mathbb{\MakeUppercase{c}} \).
\begin{prev}[Metric induced by a norm]\label{induced-metric}
	Recall the definition of \hyperref[def:seminorm]{seminorm} and \hyperref[def:norm]{norm}. We see that a \hyperref[def:norm]{norm} induces a \hyperref[def:metric]{metric}
	\[
		\rho (v, w) \coloneqq \left\lVert v - w\right\rVert,
	\]
	and we have
	\[
		v_{n} \to v \iff \lim\limits_{n \to \infty} \left\lVert v_{n} - v\right\rVert = 0.
	\]
\end{prev}

\begin{eg}
	We first see some common examples of \hyperref[def:norm]{normed} vector space.
	\begin{enumerate}[(1)]
		\item \(L^1(X, \mathcal{\MakeUppercase{a}}  , \mu )\) with \(\left\lVert f\right\rVert _1 \coloneqq \int_{}^{} \left\vert f \right\vert  \,\mathrm{d}\mu  \).
		\item \(C([0, 1])\) with \(\left\lVert f\right\rVert _1 \coloneqq \int_{0}^{1} \left\vert f(x) \right\vert  \,\mathrm{d}x \), \(\left\lVert f\right\rVert _\infty \coloneqq \mathop{\max}\limits_{0\leq x\leq 1}\left\vert f(x) \right\vert \).
		\item For \(\mathbb{\MakeUppercase{r}} ^d\) and \(0 < p < \infty \), we have
		      \[
			      \left\lVert x\right\rVert _p \coloneqq \left(\sum\limits_{i=1}^{d} \left\vert x_{i}  \right\vert^p \right)^{1/p},\qquad \left\lVert x\right\rVert _\infty \coloneqq \mathop{\max} _{1\leq i\leq d}\left\vert x_{i}  \right\vert.
		      \]
	\end{enumerate}
\end{eg}

\section{\(L^{p} \) Space}
It turns out that we can generalize \hyperref[def:L1-space]{\(L^1\)} into \(L^p\).
\begin{definition}[\(L^p\) space]\label{def:L-p-space}
	Given a \hyperref[def:measure-space]{measure space} \((X, \mathcal{\MakeUppercase{a}} , \mu )\) and a \hyperref[def:measurable-function]{measurable function} \(f\) and \(p\) such that
	\(0 < p < \infty \), we define a \hyperref[def:seminorm]{seminorm} \(\left\lVert \cdot\right\rVert _p\) such that
	\[
		\left\lVert f\right\rVert _p \coloneqq \left(\int _X \left\vert f \right\vert ^p \,\mathrm{d} \mu \right)^{1/p},
	\]
	which induces the so-called \emph{\(L^p\) space}, denoted as \(L^p(X, \mathcal{\MakeUppercase{a}} , \mu )\) where
	\[
		L^p(X, \mathcal{\MakeUppercase{a}} , \mu ) \coloneqq \left\{f\mid \left\lVert f\right\rVert _p < \infty \right\}.
	\]
\end{definition}
\begin{remark}
	Note that \(\left\lVert \cdot\right\rVert _p\) is only a \hyperref[def:seminorm]{seminorm}. But if we identity functions which are equal \hyperref[def:mu-almost-everywhere]{almost everywhere},
	then it's indeed a \hyperref[def:norm]{norm}.
\end{remark}

\begin{eg}
	\((\mathbb{\MakeUppercase{r}} , \mathcal{\MakeUppercase{l}} , m)\) has \(f(x) = x^{-\alpha } \mathbbm{1}_{(1, \infty )}(x) \in L^p\) if and only if \(\alpha p > 1\).
	In contrast, \(g(X) = x^{-\beta } \mathbbm{1}_{(0, 1)}(x) \in L^p\) if and only if \(\beta p<1\).
\end{eg}

Similar to \autoref{def:L-p-space}, we have the following.
\begin{definition}[\(\ell ^p\) space]\label{def:l-p-space}
	If \((X, \mathcal{\MakeUppercase{p}} (X), \nu )\) is equipped with the \hyperref[eg:counting-measure]{counting measure}, then we say it's an \emph{\(\ell ^p\) space} such that
	\[
		\ell ^p(X) \coloneqq L^p(X, \mathcal{\MakeUppercase{p}} (X), \nu ).
	\]
\end{definition}
\begin{remark}
	We are interested in \(\ell ^p(\mathbb{\MakeUppercase{n}} )\) in particular. We have
	\[
		\ell ^p\coloneqq \ell ^p(\mathbb{\MakeUppercase{n}} ) = \left\{a = (a_1, a_2, \ldots  )\mid \left\lVert a\right\rVert _p = \left(\sum\limits_{i=1}^{\infty} \left\vert a_{i} \right\vert^p \right)^{1/p}< \infty \right\}.
	\]
\end{remark}

\begin{lemma}
	\(L^p(X, \mathcal{\MakeUppercase{a}} , \nu )\) is a vector space for all \(p\in (0, \infty )\).
\end{lemma}
\begin{proof}
	We verify the following.
	\begin{itemize}
		\item \(c\cdot f \in L^p(X, \mathcal{\MakeUppercase{a}} , \mu )\) for \(c\in \mathbb{\MakeUppercase{r}} \). Indeed, since
		      \[
			      \left\lVert cf\right\rVert _p = \left(\int \left\vert cf \right\vert ^p\,\mathrm{d} \mu \right)^{1/p} = \left\vert c \right\vert \left\lVert f\right\rVert _p < \infty \iff \left\lVert f\right\rVert _p <\infty,
		      \]
		      which implies \(c\cdot f\in L^p(X, \mathcal{\MakeUppercase{a}} , \mu )\).
		\item \(f + g\in L^p(X, \mathcal{\MakeUppercase{a}} , \mu )\). Indeed, since for any real numbers \(\alpha , \beta \), we have
		      \[
			      (\alpha +\beta )^p \leq (2\cdot \mathop{\max} \left\{\left\vert \alpha  \right\vert, \left\vert \beta  \right\vert  \right\})^p = 2^p \cdot \mathop{\max} \{\left\vert \alpha  \right\vert^p, \left\vert \beta  \right\vert ^p \}\leq 2^p \left(\left\vert \alpha  \right\vert^p + \left\vert \beta  \right\vert^p  \right),
		      \]
		      which implies that for \(f, g\in L^p (X, \mathcal{\MakeUppercase{a}} , \mu )\), we have
		      \[
			      \left\lVert f+g\right\rVert _p < \infty \iff \left\lVert f + g\right\rVert _p^p = \int \left\vert f + g \right\vert ^p \,\mathrm{d} \mu  \leq 2^p \int (\left\vert f \right\vert ^p + \left\vert g \right\vert ^p)< \infty.
		      \]
		      This further implies
		      \[
			      \left\lVert f + g\right\rVert _p < \infty  \iff \left\lVert f\right\rVert _p, \left\lVert g\right\rVert _p < \infty ,
		      \]
		      which is what we want.
	\end{itemize}
\end{proof}

We see that in the above derivation, it doesn't give us the triangle inequality, namely
\[
	\left\lVert f + g\right\rVert _p \leq  \left\lVert f\right\rVert _p + \left\lVert g\right\rVert _p,
\]
hence we need some new results.

\begin{theorem}[Hölder's inequality]\label{thm:Holder-inequality}
	Let \(1 < p < \infty \), and let \(q \coloneqq p / (p - 1)\) so that \(1 / p + 1 / p = 1\). Then we have
	\[
		\left\lVert f\cdot g\right\rVert _1 \leq  \left\lVert f\right\rVert _p \left\lVert g\right\rVert _q.
	\]
\end{theorem}
\begin{proof}
	We prove this in steps.
	\begin{claim}
		We have
		\[
			t\leq \frac{t^p}{p}+1-\frac{1}{p} = \frac{t^p}{p}+ \frac{1}{q}
		\]
		for all \(t \geq 0\).
	\end{claim}
	\begin{explanation}
		By taking \(F(t) \coloneqq t - t^p / p\) and \(t \geq 0\), we see that the maximum of \(F\) implies the above inequality.
	\end{explanation}
	\begin{claim}[Young's Inequality]
		We have
		\[
			\alpha \beta \leq \frac{\alpha ^p}{p} + \frac{\beta ^q}{q}
		\]
		for \(\alpha , \beta > 0\).\footnote{\url{https://en.wikipedia.org/wiki/Young's_inequality_for_products}}
	\end{claim}
	\begin{explanation}
		This follows by taking \(t \coloneqq \alpha /\beta ^{q - 1}\) in the first inequality we obtained.
	\end{explanation}

	Then, without loss of generality, we can assume that \(0 <\left\lVert f\right\rVert _p, \left\lVert g\right\rVert _q < \infty \). Now, consider \(F(x) = f(x) / \left\lVert f\right\rVert _p\),
	\(G(x) = g(x) / \left\lVert g\right\rVert _q\). We know that \(\left\lVert F\right\rVert _p = 1 = \left\lVert G\right\rVert _q\). Then by Young's Inequality, we have
	\[
		\int \left\vert F(x)G(x) \right\vert \,\mathrm{d} \mu \leq \int \frac{\left\vert F(x) \right\vert ^p}{p} + \frac{\left\vert G(x) \right\vert ^q}{q} \implies \frac{\left\lVert fg\right\rVert _1}{\left\lVert f\right\rVert _p \left\lVert g\right\rVert _q}\leq \frac{1}{p} + \frac{1}{q} = 1,
	\]
	which implies our desired result.
\end{proof}
\begin{eg}
	For \(p = q = 2\), \(X = \{1, \ldots , d \}\) with \(\mu \) being the \hyperref[eg:counting-measure]{counting measure}, then for any \(x, y\in \mathbb{\MakeUppercase{r}} ^d\), we have
	\[
		\sum\limits_{i=1}^{d} \left\vert x_{i} y_{i}  \right\vert \leq \sqrt{\sum\limits_{i=1}^{d} x_{i} ^{2} } \sqrt{\sum\limits_{i=1}^{d} y_{i} ^{2} }
	\]
\end{eg}

We now see how we can obtain the desired triangle inequality.
\begin{theorem}[Minkowski's Inequality]\label{thm:Minkowski-inequality}
	Let \(1\leq p < \infty \), then for \(f, g\in L^p\),
	\[
		\left\lVert f + g\right\rVert _p \leq  \left\lVert f\right\rVert _p + \left\lVert g\right\rVert _p.
	\]
\end{theorem}
\begin{proof}
	For \(p = 1\), it's easy since it's just triangle inequality. Now, we assume that \(1 < p < \infty \), and we may assume also that \(\left\lVert f + g\right\rVert \neq 0\) without loss of generality. Then
	\[
		\begin{split}
			\int \left\vert f(x) + g(x) \right\vert ^p &\leq \int \left\vert f(x) + g(x) \right\vert ^{p - 1}\left(\left\vert f(x) \right\vert + \left\vert g(x) \right\vert \right)\\
			&\leq  \left(\int \left\vert f+g \right\vert ^{(p-1)q}\right)^{1 / q}\left[\left(\int \left\vert f \right\vert^p \right)^{1 / p} + \left(\int \left\vert g \right\vert^p \right)^{1 / p}\right]\\
			&\leq  \left(\int \left\vert f + g \right\vert^p \right)^{1 / q}\left(\left\lVert f\right\rVert _p + \left\lVert g\right\rVert _p\right).
		\end{split}
	\]
	We then see that
	\[
		\underbrace{\left(\left\vert f(x) + g(x) \right\vert ^p\right)^{1 - 1 / q}}_{\left(\left\vert f(x) + g(x) \right\vert ^p\right)^{1 / p}} \leq \left\lVert f\right\rVert _p + \left\lVert g\right\rVert_p,
	\]
	which is just \(\left\lVert f + g\right\rVert _{p} \leq \left\lVert f\right\rVert _p + \left\lVert g\right\rVert_p\).
\end{proof}