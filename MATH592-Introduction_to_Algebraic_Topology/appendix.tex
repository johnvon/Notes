\section{Additional Proofs}
\subsection{Seifert-Van Kampen Theorem on Groupoid}\label{thm:Seifert-Van-Kampen-Theorem-on-groupoid}
\begin{theorem}[Seifert-Van Kampen Theorem on groupoid]
	Given \(X_0, X_1, X\) as topological spaces with \(X_0 \cup X_1 = X\). Then the functor \(\Pi \colon \underline{\mathrm{Top}}\to \underline{\mathrm{Gpd}}\) maps the
	\hyperref[def:cocartesian]{cocartesian} diagram in \(\underline{\mathrm{Top} _\ast}\) to a \hyperref[def:cocartesian]{cocartesian} diagram in \(\underline{\mathrm{Gp} }\)
	as follows.
	\[\begin{tikzcd}
			{(X_0\cap X_1, x_0)} & {(X_0, x_0)} \\
			{(X_1, x_0)} & {(X, x_0)}
			\arrow["{j_0}", from=1-1, to=1-2]
			\arrow["{i_0}", from=1-2, to=2-2]
			\arrow["{i_1}"', from=2-1, to=2-2]
			\arrow["{j_1}"', from=1-1, to=2-1]
		\end{tikzcd}\overset{\Pi}{\longmapsto}
		\begin{tikzcd}
			{\Pi(X_0\cap X_1)} & {\Pi(X_0)} \\
			{\Pi(X_1)} & {\Pi(X)}
			\arrow["{\Pi (j_0)}", from=1-1, to=1-2]
			\arrow["{\Pi (i_0)}", from=1-2, to=2-2]
			\arrow["{\Pi (i_1)}"', from=2-1, to=2-2]
			\arrow["{\Pi (j_1)}"', from=1-1, to=2-1]
		\end{tikzcd} \]
\end{theorem}
\begin{note}
	Notice that \(X_0, X_1, X\) don't need to be \hyperref[def:path]{path}-connected in particular.
\end{note}

Surprisingly, the proof of \autoref{thm:Seifert-Van-Kampen-Theorem-on-groupoid} is much elegant with the elementary proof of \autoref{thm:Seifert-Van-Kampen-Theorem}, hence we give
the proof here.
\begin{proof}
	Let \(\mathscr{G} \in \Object (\underline{\mathrm{Gpd} })\) a \hyperref[def:groupoid]{groupoid}, and given \hyperref[def:functor]{functors}
	\[
		F\colon \Pi (X_0)\to \mathscr{G} ,\quad G\colon \Pi (X_1)\to \mathscr{G}
	\]
	such that
	\[
		F\circ \Pi (j_0) = G\circ \Pi (j_1).
	\]
	\[\begin{tikzcd}
			{\Pi(X_0\cap X_1)} & {\Pi_1(X_0)} \\
			{\Pi_1(X_1)} & {\Pi_1(X)} \\
			&& {\mathscr{G}}
			\arrow["{\Pi(j_0)}", from=1-1, to=1-2]
			\arrow["{\Pi(i_0)}", from=1-2, to=2-2]
			\arrow["{\Pi(i_1)}"', from=2-1, to=2-2]
			\arrow["{\Pi(j_1)}"', from=1-1, to=2-1]
			\arrow["{\exists!K}"{description}, dashed, from=2-2, to=3-3]
			\arrow["F", curve={height=-12pt}, from=1-2, to=3-3]
			\arrow["G"', curve={height=12pt}, from=2-1, to=3-3]
		\end{tikzcd}\]
	We now only need to prove that there exists a unique \hyperref[def:functor]{functor} \(K\colon \Pi (X)\to \mathscr{G} \)  such that the above diagram commutes.

	We can define \(K\) as
	\begin{itemize}
		\item on \hyperref[def:object]{objects}: For all \(x\in \Object (\Pi (X)) = X\),
		      \[
			      K(x) = \begin{dcases}
				      F(x), & \text{ if } x\in X_0 ; \\
				      G(x), & \text{ if } x\in X_1 .
			      \end{dcases}
		      \]
		      This is well-defined since the diagram (without \(K\)) commutes.
		\item on \hyperref[def:morphism]{morphisms}: For every \(p, q\in X\), \(\left< \gamma \right> \colon p\to q\) in \(\Homomorphism _{\Pi (X)}(p, q)\), we need to define
		      \(K(\left< \gamma  \right> )\in \Homomorphism _{\mathscr{G} }(K(p), K(q))\). Our strategy is for every path \(\gamma \) from \(p\) to \(q\), we define
		      \(\widetilde{K} (\gamma )\in \Homomorphism_{\mathscr{G} } (K(p), K(q))\).
		      Then if we also have \(\widetilde{K} (\gamma ) = \widetilde{K} (\gamma ^\prime )\) for \(\gamma \simeq \gamma ^\prime \ \mathrm{rel} \{0, 1\}\), then we can just let
		      \[
			      K(\left< \gamma  \right> ) \coloneqq \widetilde{K} (\gamma ).
		      \]
		      Now we start to construct \(\widetilde{K} \).

		      Given a path \(\gamma \colon [0, 1]\to X\), \(\gamma (0) = p, \gamma (1) = q\). Since \(\mathrm{int}(X_0) \cup \mathrm{int}(X_1) = X\), we see that
		      \[
			      \gamma ^{-1} (\mathrm{int}(X_0)) \cup \gamma ^{-1} (\mathrm{int}(X_1)) = [0, 1].
		      \]
		      From Lebesgue Lemma\footnote{\url{https://en.wikipedia.org/wiki/Lebesgue\%27s_number_lemma}}, there exists a finite partition
		      \[
			      0 = t_0 < t_1 < \ldots <t_{m-1} < t_{m} = 1
		      \]
		      such that for every \(i\),
		      \[
			      \gamma ([t_{i-1}, t_{i} ])\subset \mathrm{int}(X_0) \text{ or } \mathrm{int}(X_1) .
		      \]
		      \begin{figure}[H]
			      \centering
			      \incfig{pf:thm:Seifert-Van-Kampen-Theorem-on-groupoid}
			      \label{fig:pf:thm:Seifert-Van-Kampen-Theorem-on-groupoid}
		      \end{figure}
		      Now, let \(\gamma _{i} \colon [0, 1]\to X, t\mapsto \gamma ((1-t)t_{i-1}+t\cdot t_{i} )\), we see that \(\gamma _{i} \) is either a \hyperref[def:path]{path} in \(X_0\) or \(X_1\).
		      We then define \(\widetilde{K} (\gamma )\coloneqq \widetilde{K} (\gamma _{m} )\circ \widetilde{K} (\gamma _{m-1}) \circ \ldots \circ \widetilde{K} (\gamma _1)\in \Homomorphism _{\mathscr{G} }(K(P), K(q)) \)
		      such that
		      \[
			      \widetilde{K} (\gamma _{i} ) = \begin{dcases}
				      F(\left< \gamma _{i} \right> ), & \text{ if } \gamma _{i} \subset X_0 ; \\
				      G(\left< \gamma _{i} \right> ), & \text{ if } \gamma _{i} \subset X_1 .
			      \end{dcases}
		      \]
		      We need to prove that \(\widetilde{K} (\gamma )\) does not depend on the partition. It's sufficient to prove that for any partition
		      \[
			      0 = t_0 < t_1 < \ldots <t_{m-1} < t_{m} = 1,
		      \]
		      we consider any \textbf{finer} partition
		      \[
			      0 = t_0= t_{10}< t_{11} <\ldots < t_{1K_1}= t_1 = t_{20} <t_{21}<\ldots < t_{mK_{m} } = t_{m} = 1.
		      \]
		      As before, we denote \(\gamma _{ij}\colon [0, 1]\to X, t\mapsto \gamma ((1-t)t_{i j-1} + t\cdot t_{ij} )\). It's clear that as long as
		      \[
			      \widetilde{K} (\gamma _{i} ) = \widetilde{K} (\gamma _{i K_{i} })\circ \widetilde{K} (\gamma _{i K_{i}-1 })\circ \ldots \circ \widetilde{K} (\gamma _{i 0}),
		      \]
		      then our claim is proved. But this is immediate since \(F\) and \(G\) are \hyperref[def:functor]{functor} and for any \(i\), we only use either \(F\) or \(G\) all the time.

		      Now we prove \(\gamma \underset{H}{\simeq }\gamma ^\prime \  \mathrm{rel} \{0, 1\}\), then \(\widetilde{K} (\gamma ) = \widetilde{K} (\gamma ^\prime )\).
		      This is best shown by some diagram.
		      \begin{figure}[H]
			      \centering
			      \incfig{pf:thm:Seifert-Van-Kampen-Theorem-on-groupoid-2}
			      \label{fig:pf:thm:Seifert-Van-Kampen-Theorem-on-groupoid-2}
		      \end{figure}
		      The left-hand side represents a partition \(\mathcal{\MakeUppercase{p}} \) of \([0, 1]\times [0, 1]\) such that every small square's image in \(X\) under \(H\) is either entirely in \(X_0\)
		      or in \(x_1\). Consider all paths from \((0, 0)\) to \((1, 1)\) such that it only goes right or up. We see that for any such path \(L\), consider
		      \[
			      \gamma _{L} \colon [0, 1]\to L, \quad t\mapsto \gamma _{L} (t).
		      \]
		      We let \(\Gamma _{L} \colon \at{H}{L}{} \circ \gamma _{L} \colon [0, 1]\to X\), we see that \(\Gamma _{L} \) is a \hyperref[def:path]{path} from \(p\) to \(q\). Now, if
		      for two paths \(L_1\) and \(L_2\) such that they only differ from \underline{a square}.
		      \begin{figure}[H]
			      \centering
			      \incfig{pf:thm:Seifert-Van-Kampen-Theorem-on-groupoid-3}
			      \label{fig:pf:thm:Seifert-Van-Kampen-Theorem-on-groupoid-3}
		      \end{figure}
		      We claim that \(\gamma _{L_1}, \Gamma _{L_2}\) are two \hyperref[def:path]{paths} from \(p\) to \(q\), and \(\widetilde{K} (\Gamma _{L_1}) = \widetilde{K} (\Gamma _{L_2})\).
		      Now, we denote \(\Gamma _0\) and \(\Gamma _1\) as follows.
		      \begin{figure}[H]
			      \centering
			      \incfig{pf:thm:Seifert-Van-Kampen-Theorem-on-groupoid-4}
			      \caption{The definition of \(\Gamma _0\) and \(\Gamma _1\).}
			      \label{fig:pf:thm:Seifert-Van-Kampen-Theorem-on-groupoid-4}
		      \end{figure}
		      It's clearly that by only finitely many steps, we can transform \(\Gamma _0\) to \(\Gamma _1\), hence
		      \[
			      \widetilde{K} (\Gamma _0) = \widetilde{K} (\Gamma _1).
		      \]
		      Finally, we observe that
		      \[
			      \widetilde{K} (\gamma _0) = \widetilde{K} (\Gamma _0) = \widetilde{K} (\Gamma _1) = \widetilde{K} (\gamma _1).
		      \]

		      If we now define \(K(\left< \gamma  \right> ) = \widetilde{K} (\gamma )\), then \(K\colon \Morphism (\Pi (X))\to \Morphism (\mathscr{G} )\), then it's well-defined.
	\end{itemize}

	We now prove \(K\colon \Pi (X)\to \mathscr{G}\) is indeed a \hyperref[def:functor]{functor}. But this is immediate from the definition of \(K\), namely it'll send identity to identity and
	the composition associates.

	Also, we need to prove that the following diagram commutes.
	\[\begin{tikzcd}
			{\Pi(X_0\cap X_1)} & {\Pi_1(X_0)} \\
			{\Pi_1(X_1)} & {\Pi_1(X)} \\
			&& {\mathscr{G}}
			\arrow["{\Pi(j_0)}", from=1-1, to=1-2]
			\arrow["{\Pi(i_0)}", from=1-2, to=2-2]
			\arrow["{\Pi(i_1)}"', from=2-1, to=2-2]
			\arrow["{\Pi(j_1)}"', from=1-1, to=2-1]
			\arrow["{K}"{description}, from=2-2, to=3-3]
			\arrow["F", curve={height=-12pt}, from=1-2, to=3-3]
			\arrow["G"', curve={height=12pt}, from=2-1, to=3-3]
		\end{tikzcd}\]
	But this is again trivial.

	Finally, we need to show that such \(K\) is unique. This is the same as the proof of \autoref{lma:lec7}, hence the proof is done.
\end{proof}
\subsection{An alternative proof of \hyperref[thm:Seifert-Van-Kampen-Theorem]{Seifert Van-Kampen Theorem}}\label{pf:an-alternative-proof-of-Seifert-Van-Kampen-thm}
\begin{theorem}
	We claim that the diagram
	\[
		\begin{tikzcd}
			{\pi_1(X_0\cap X_1, x_0)} & {\pi_1(X_0, x_0)} \\
			{\pi_1(X_1, x_0)} & {\pi_1(X, x_0)}
			\arrow["{(j_0)_\ast}", from=1-1, to=1-2]
			\arrow["{(i_0)_\ast}", from=1-2, to=2-2]
			\arrow["{(i_1)_\ast}"', from=2-1, to=2-2]
			\arrow["{(j_1)_\ast}"', from=1-1, to=2-1]
		\end{tikzcd}
	\]
	is \hyperref[def:cocartesian]{cocartesian}.
\end{theorem}
\begin{proof}
	The basic idea is that, for this diagram,
	\[\begin{tikzcd}
			{\Pi(X_0\cap X_1)} & {\Pi(X_0)} \\
			{\Pi(X_1)} & {\Pi(X)} \\
			\arrow[from=1-1, to=1-2]
			\arrow[from=1-2, to=2-2]
			\arrow[from=2-1, to=2-2]
			\arrow[from=1-1, to=2-1]
		\end{tikzcd}\]
	we want to construct a \hyperref[def:morphism]{morphism} \(r\colon \Pi (Z)\to \pi _1(Z, p)\) in \(\underline{\mathrm{Gpd}}\) such that
	\(Z = X_0 \cap X_1, X_0, X_1, X\). For every \(x\in Z\), we fix a \hyperref[def:path]{path} \(\gamma _{x} \) such that it connects \(p\) and \(x\) and satisfies
	\begin{enumerate}
		\item If \(x\in X_0 \cap X_1\), then \(\mathrm{Im} (\gamma _{x} )\subset X_0 \cap X_1\)
		\item If \(x\in X_0\), then \(\mathrm{Im} (\gamma _{x} )\subset X_0\)
		\item If \(x\in X_1\), then \(\mathrm{Im} (\gamma _{x} )\subset X_1\)
		\item \(\gamma _p = c_p\)
	\end{enumerate}

	The proof is given in \url{https://www.bilibili.com/video/BV1P7411N7fW?p=38&spm_id_from=pageDriver}.
	\todo{If have time.}
\end{proof}

\section{Abelian Group}
This section aims to give some reference about \hyperref[def:Abelian-group]{Abelian groups}, specifically for \hyperref[def:free-Abelian-group]{free Abelian group}, which is used
heavily when discussing \hyperref[def:homology]{homology}.
\subsection{Abelian Group}
\begin{definition}[Abelian group]\label{def:Abelian-group}
	A group \((G, \cdot)\) is an \emph{Abelian group} if for every \(a, b\in G\), we have
	\[
		a\cdot b = b\cdot a.
	\]
	We often denote \(\cdot\) as \(+\) if \((G, \cdot)\) is a \hyperref[def:Abelian-group]{Abelian group}.
\end{definition}

\begin{definition}[Product of groups]\label{def:product-of-groups}
	Given two groups \((G, \cdot), (H, \cdot)\), the \emph{product of \(G\) and \(H\)}, denoted by \(G\times H\) is defined as
	\[
		G\times H = \left\{(g, h)\mid g\in G, h\in H\right\}
	\]
	and
	\[
		(g_1, h_1)\cdot (g_2, h_2)\coloneqq (g_1\cdot g_2, h_1\cdot h_2).
	\]
\end{definition}

\begin{notation}
	For simplicity, given an index set \(I\), we'll denote the order pair \((g_{\alpha _1}, g_{\alpha _2}, \ldots)\) as \((g_\alpha )_{\alpha \in I}\). Note that
	the latter notation can handle the case that \(I\) is either countable or uncountable, while the former can only handle the countable case.
\end{notation}

\begin{definition}[Direct product]\label{def:direct-product}
	Given \((G_\alpha , +)\), \(\alpha \in I\) as a collection of \hyperref[def:Abelian-group]{Abelian group}, we define their \emph{direct product} as
	\[
		\left(\prod\limits_{\alpha \in I}G_\alpha , + \right),
	\]
	where
	\[
		\prod\limits_{\alpha \in I} G_\alpha = \left\{(g_\alpha )_{\alpha \in I}\mid g_\alpha \in G_\alpha \right\}
	\]
	and \(\forall (g_\alpha ), (h_\alpha )\in \prod\limits_{\alpha \in I} G_\alpha \)
	\[
		(g_\alpha )+(h_\alpha ) \coloneqq g_\alpha + h_\alpha
	\]
	for all \(\alpha \in I\).

	Specifically, if \(I\) is finite, namely there are only finely many \hyperref[def:Abelian-group]{Abelian groups} \((G_1, +),\ldots , (G_n, +)\), and
	\(\left(\prod\limits_{i=1}^{n} G_{i} , +\right)\) can be denoted as
	\[
		\left(G_1 \times \ldots \times G_n, + \right).
	\]
\end{definition}

\begin{definition}[External direct sum]\label{def:external-direct-sum}
	Given a collection of \hyperref[def:Abelian-group]{Abelian groups} \(\{G_\alpha \}_{\alpha \in I}\), the \emph{external direct sum} of them, denoted as \(\left(\bigoplus_{\alpha \in I} G_\alpha , +\right)\)
	as
	\[
		\bigoplus_{\alpha \in I}G_\alpha \coloneqq \left\{(g_\alpha )_{\alpha \in I}\mid \underset{\alpha \in I}{\forall }\ g_\alpha \in G_\alpha, \text{\# non-zero elements in \(g_{\alpha} < \infty\)}\right\}.
	\]
	And for every \((g_\alpha ), (h_\alpha )\in \bigoplus_{\alpha \in I}G_\alpha \),
	\[
		(g_\alpha )+(h_\alpha )\coloneqq g_\alpha +h_\alpha
	\]
	for all \(\alpha \in I\).\footnote{This may not be the best notation: What we're really trying to say is \((g_\alpha )_{\alpha \in I}+ (h_\alpha )_{\alpha \in I} \coloneqq g_i + h_i \) for all \(i \in I\).}
\end{definition}
\begin{note}
	We see that
	\[
		\bigoplus_{\alpha \in I}G_\alpha \subset \prod\limits_{\alpha \in I}G_\alpha.
	\]
	Additionally, we also have
	\[
		\left(\bigoplus_{\alpha \in I}G_\alpha , +\right)< \left(\prod\limits_{\alpha \in I}G_\alpha , +\right).
	\]
\end{note}
\begin{remark}
	We see that the operation \(+\) is indeed closed since the sum of \(g, g^\prime \in \bigoplus_{\alpha \in I} G_\alpha \) will have only finitely non-zero elements if
	\(g, g^\prime \) both have only finitely many non-zero elements.
\end{remark}

We see that if \(I\) is a finite index set, given a collection of \hyperref[def:Abelian-group]{Abelian group} \(\{G_\alpha \}_{\alpha \in I}\), then
\[
	G_1 \times \ldots \times G_n = G_{1}  \oplus  \ldots \oplus G_n.
\]

\begin{definition}[Internal direct sum]\label{def:internal-direct-sum}
	Given an \hyperref[def:Abelian-group]{Abelian group} \(G\), and a collection of the subgroups \(\{G_\alpha \}_{\alpha \in I}\) of \(G\), we say \(G\) is an \emph{internal direct sum}
	of \(\{G_\alpha \}_{\alpha \in I}\) if for any \(g\in G\), we can write
	\[
		g = \sum\limits_{\alpha \in I} g_\alpha
	\]
	\textbf{uniquely}, where \(g_\alpha \in G_\alpha \) has only finitely many non-zero elements. In this case, we denote
	\[
		G = \bigoplus_{\alpha \in I}G_\alpha .
	\]
\end{definition}
Intuitively, the \hyperref[def:external-direct-sum]{external direct sum} is to build a new group based on the given collection of groups \(\{G_\alpha \}_{\alpha \in I}\), while the internal direct sum is
to express an \textbf{already known} group \(G\) with an \textbf{already known} collection of groups \(\{G_\alpha \}_{\alpha \in I}\).

\begin{remark}[Relation between Internal and External direct sum]\label{rmk:relation-between-internal-and-externam-direct-sum}
	Given an \hyperref[def:Abelian-group]{Abelian group} \(G\) and its \hyperref[def:internal-direct-sum]{internal direct sum} decomposition \(\bigoplus_{\alpha \in I}G_\alpha \),
	\(G\) is isomorphic to the \hyperref[def:external-direct-sum]{external direct sum} of \(\{G_\alpha \}_{\alpha \in I}\). We see this from the following group homomorphism:
	\[
		\underset{g\in G}{\forall }\ g = \sum\limits_{\alpha \in I}g_\alpha \mapsto (g_\alpha )_{\alpha \in I}.
	\]

	Conversely, given a collection of \hyperref[def:Abelian-group]{Abelian group} \(\{G_\alpha \}_{\alpha \in I}\), and let \(G = \bigoplus_{\alpha \in I}G_\alpha \) as the
	\hyperref[def:external-direct-sum]{external direct sum} of \(\{G_\alpha \}\), denote
	\(i_{\alpha_0 }\colon G_{\alpha_0} \to \bigoplus_{\alpha \in I}G_\alpha\)
	as a canonical embedding
	\[
		g_{ \alpha_0 }\mapsto i_{ \alpha_0 } (g_{ \alpha_0 }) = (h_\alpha)_{\alpha\in I},
	\]
	where
	\[
		h_\alpha  = \begin{dcases}
			g_{ \alpha_0 }, & \text{ if } \alpha_0 = \alpha  ; \\
			0,              & \text{ if } \alpha_0 \neq \alpha
		\end{dcases}
	\]
	given \(\alpha_0\). Then
	\[
		G_{\alpha_0} ^\prime \coloneqq i_{\alpha_0} (G_{\alpha_0}) < \bigoplus _{\alpha \in I}G_\alpha
	\]
	and \(G\) is the \hyperref[def:internal-direct-sum]{internal direct sum} of \(G^\prime _{\alpha_0} \), \(\alpha_0\in I\). This is because
	\(\forall g= (g_\alpha )_{\alpha \in I}\in G(= \bigoplus_{\alpha \in I}G_\alpha )\), we have
	\[
		g = \sum\limits_{\alpha \in I} i_\alpha (g_\alpha ).
	\]
	Note that the above sum is well-defined since there are only finitely many non-zero elements for each \(g_\alpha \). And additionally, we can
	see the uniqueness of this decomposition by defining \(\pi _{\alpha_0} \) such that
	\[
		\pi _{\alpha_0} \colon \bigoplus_{\alpha \in I}G_\alpha \to G_{\alpha_0} , \quad (g_\alpha )_{\alpha \in I}\mapsto g_{\alpha_0} ,
	\]
	then \(\pi _\alpha \circ i_\alpha = \identity_{G_\alpha } \), \(\pi _\alpha \circ i_\beta = 0\) for all \(\beta \neq \alpha \) and
	\[
		\pi _\beta (g) = \pi _\beta \left(\sum\limits_{\alpha \in I}^{}i_\alpha (g_\alpha ) \right) = \sum\limits_{\alpha \in I}\pi _\beta \circ i_\alpha (g_\alpha ) = \pi _\beta \circ i_\beta (g_\beta ) = g_\beta
	\]
	for all \(\beta \in I\), where the second equality is because this summation is finite. Hence, we have
	\[
		g = \sum\limits_{\alpha \in I}i_\alpha (\pi _\alpha (g)).
	\]
\end{remark}

\begin{definition}
	Given two \hyperref[def:Abelian-group]{Abelian groups} \(G, H\), we define \(\Homomorphism (G, H)\) as
	\[
		\Homomorphism (G, H) \coloneqq \left\{f\colon G\to H \mid \text{\(f\) is a group homomorphism} \right\},
	\]
	then we can define
	\[
		\begin{split}
			+\colon \Homomorphism (G, H)\times \Homomorphism (G, H) &\to \Homomorphism (G, H)\\
			(\varphi , \psi )&\mapsto \varphi + \psi,
		\end{split}
	\]
	where
	\[
		(\varphi + \psi )(g)\coloneqq \varphi (g) + \psi (g).
	\]
\end{definition}

\begin{remark}[Relation between direct sum and direct product]
	Given a collection of \hyperref[def:Abelian-group]{Abelian groups} \(\{G_\alpha \}_{\alpha \in I}\), and another \hyperref[def:Abelian-group]{Abelian group} \(H\), there exists a
	\(\varphi \) such that
	\[
		\begin{split}
			\varphi \colon \Homomorphism \left(\bigoplus_{\alpha \in I}G_\alpha , H\right)&\to \prod\limits_{\alpha \in I}\Homomorphism (G_\alpha , H)\\
			f&\mapsto \varphi (f)\coloneqq (f_\alpha )_{\alpha \in I}
		\end{split}
	\]
	where \(f_\alpha = f\circ i_\alpha \), where \(i_\alpha \) is the canonical embedding from \(G_\alpha \) to \(\bigoplus_{\alpha \in I}G _\alpha \). We claim
	that \(\varphi \) is an isomorphism.

	\begin{itemize}
		\item \(\varphi \) is injective. This is obvious since \(\mathrm{ker} (\varphi ) = 0\) from the fact that if \(\varphi (f) = 0\), then \(f_\alpha  = 0\) for all \(\alpha\), hence
		      \(f\) is \(0\).
		\item \(\varphi \) is surjective. For every \((f_\alpha )_{\alpha \in I}\in \prod\limits_{\alpha \in I}\Homomorphism (G_\alpha , H) \), we define
		      \[
			      \begin{split}
				      f\colon \bigoplus_{\alpha \in I}G_\alpha &\to H\\
				      \sum\limits_{\alpha \in I}^{}g_\alpha &\mapsto \sum\limits_{\alpha \in I}f_\alpha (g_\alpha ).
			      \end{split}
		      \]
		      We see that \(f\in \Homomorphism \left(\bigoplus_{\alpha \in I}G_\alpha , H\right)\) and \(\varphi (f) = (f_\alpha )_{\alpha \in I}\).
	\end{itemize}

	This shows that
	\[
		\Homomorphism \left(\bigoplus_{\alpha \in I}G_\alpha , H\right)\cong \prod\limits_{\alpha \in I}\Homomorphism (G_\alpha , H).
	\]
\end{remark}

\begin{exercise}
	We can show that
	\[
		\Homomorphism \left(\prod\limits_{\alpha \in I}G_\alpha, H\right)= \prod\limits_{\alpha \in I}\Homomorphism (G_\alpha, H).
	\]
\end{exercise}

\subsection{Free Abelian Group}
\begin{definition}[Free Abelian group]\label{def:free-Abelian-group}
	Given an  \hyperref[def:Abelian-group]{Abelian group} \((G, +)\), we say \(G\) is a \emph{free Abelian group} if there exists a collection of elements \(\{g_\alpha \}_{\alpha \in J}\) in \(G\) such that
	\(\{g_\alpha \}_{\alpha \in J}\) forms a \textbf{basis} of \(G\), i.e., for all \(g\in G\), \(\exists ! n _\alpha \in \mathbb{\MakeUppercase{z}} \) for all \(\alpha \in J\) such that
	\[
		g = \sum\limits_{\alpha \in J} n_\alpha g_\alpha,
	\]
	where there are only finitely many non-zero \(n_\alpha\).
\end{definition}
\begin{remark}
	If \(G\) is a \hyperref[def:free-Abelian-group]{free Abelian group}, and \(\{g_\alpha \}_{\alpha \in J}\) is a basis, then for every \(\alpha \in J\), \(\left< g_\alpha \right> \) is an
	infinite cyclic group since
	\[
		n\cdot g_\alpha = 0 = 0\cdot g_\alpha \implies n = 0.
	\]
	And from \autoref{def:free-Abelian-group}, we have
	\[
		G = \bigoplus_{\alpha \in J}\left< g_\alpha  \right>.
	\]

	Conversely, assume there are a collection of infinite cyclic group \(\left< g_\alpha  \right> \) for \(\alpha \in I\) in \(G\) such that
	\[
		G = \bigoplus_{\alpha \in I}\left< g_\alpha  \right>,
	\]
	then \(\{g_\alpha \}_{\alpha \in I}\) is a basis of \(G\), hence \(G\) is a \hyperref[def:free-Abelian-group]{free Abelian group}.
\end{remark}

\begin{proposition}
	If \(G\) is an  \hyperref[def:Abelian-group]{Abelian group}, then the followings are equivalent:
	\begin{enumerate}
		\item \(G\) is a \hyperref[def:free-Abelian-group]{free Abelian group}.
		\item \(G\) is an \hyperref[def:internal-direct-sum]{internal direct sum} of some infinite cyclic groups.
		\item \(G\) is isomorphic to the \hyperref[def:external-direct-sum]{external direct sum} of some additive groups of integers \(\mathbb{\MakeUppercase{z}} \).
	\end{enumerate}
\end{proposition}
\begin{proof}
	We see that \(1. \iff 2.\) is already proved. And for \(2. \iff 3.\), this follows directly from the \hyperref[rmk:relation-between-internal-and-externam-direct-sum]{relation between internal and external direct sum}.
\end{proof}

Now, consider \(G\) as a \hyperref[def:free-Abelian-group]{free Abelian group}, then
\[
	u\colon \begin{tikzcd}
		G & {\bigoplus\limits_{\alpha\in I} \mathbb{Z}}
		\arrow["\cong", from=1-1, to=1-2]
	\end{tikzcd}
\]
Denote \(e_\alpha \coloneqq i_\alpha (1)\in \bigoplus_{\alpha \in I}\mathbb{\MakeUppercase{z}} \), where \(i_\alpha \colon \mathbb{\MakeUppercase{z}} \to \bigoplus_{\alpha \in I}\mathbb{\MakeUppercase{z}} \) is the
canonical embedding, i.e., \(e_\alpha = (g_\alpha )_{\alpha \in I}\in \bigoplus_{\alpha \in I}\mathbb{\MakeUppercase{z}} \), where
\[
	g_\beta = \begin{dcases}
		1, & \text{ if } \beta = \alpha;    \\
		0, & \text{ if } \beta \neq \alpha.
	\end{dcases}
\]
Moreover, denote \(\epsilon _\alpha \) as the image of \(e_\alpha \) under the isomorphism \(u\), namely \(\epsilon _\alpha = u^{-1} (e_\alpha )\), then \(\{\epsilon _\alpha \}_{\alpha \in I}\) is a basis of \(G\).

Now, for every \hyperref[def:Abelian-group]{Abelian group} \(H\), we have
\[\begin{tikzcd}
		{\Homomorphism(G, H)} && {\Homomorphism\left(\bigoplus\limits_{\alpha\in I}\mathbb{Z}, H\right)} & f && {f\circ u^{-1}} \\
		&& {\prod\limits_{\alpha\in I}\Homomorphism\left(\mathbb{Z}, H\right)} &&& {(f\circ u^{-1}\circ i_{\alpha})_{\alpha\in I}} \\
		&& {\prod\limits_{\alpha\in I}H} &&& {\left(f\circ u^{-1}\circ i_\alpha(1)\right)_{\alpha\in I}}
		\arrow[maps to, from=1-6, to=2-6]
		\arrow[maps to, from=2-6, to=3-6]
		\arrow["\cong"', maps to, from=1-4, to=3-6]
		\arrow[maps to, from=1-4, to=1-6]
		\arrow["\varphi", from=1-3, to=2-3]
		\arrow["\cong", from=2-3, to=3-3]
		\arrow["{\circ u}"', from=1-3, to=1-1]
		\arrow["\cong"', from=1-1, to=3-3]
		\arrow["\cong"', draw=none, from=1-3, to=2-3]
		\arrow["\cong"', draw=none, from=1-1, to=1-3]
	\end{tikzcd}\]
% file:///Users/pbb/Developer/quiver/src/index.html?q=WzAsOCxbMCwxLCJmIl0sWzMsMSwiZlxcY2lyYyB1XnstMX0iXSxbMywyLCIoZlxcY2lyYyB1XnstMX1cXGNpcmMgaV97XFxhbHBoYX0pX3tcXGFscGhhXFxpbiBJfSJdLFszLDMsIlxcbGVmdChmXFxjaXJjIHVeey0xfVxcY2lyYyBpX1xcYWxwaGEoMSlcXHJpZ2h0KV97XFxhbHBoYVxcaW4gSX0iXSxbMCwwLCJcXEhvbW9tb3JwaGlzbShHLCBIKSJdLFsyLDAsIlxcSG9tb21vcnBoaXNtXFxsZWZ0KFxcYmlnb3BsdXNcXGxpbWl0c197XFxhbHBoYVxcaW4gSX1cXG1hdGhiYntafSwgSFxccmlnaHQpIl0sWzIsMiwiXFxwcm9kXFxsaW1pdHNfe1xcYWxwaGFcXGluIEl9SCJdLFsyLDEsIlxccHJvZFxcbGltaXRzX3tcXGFscGhhXFxpbiBJfVxcSG9tb21vcnBoaXNtXFxsZWZ0KFxcbWF0aGJie1p9LCBIXFxyaWdodCkiXSxbMSwyLCIiLDAseyJzdHlsZSI6eyJ0YWlsIjp7Im5hbWUiOiJtYXBzIHRvIn19fV0sWzIsMywiIiwyLHsic3R5bGUiOnsidGFpbCI6eyJuYW1lIjoibWFwcyB0byJ9fX1dLFswLDMsIlxcY29uZyIsMix7InN0eWxlIjp7InRhaWwiOnsibmFtZSI6Im1hcHMgdG8ifX19XSxbMCwxLCIiLDIseyJzdHlsZSI6eyJ0YWlsIjp7Im5hbWUiOiJtYXBzIHRvIn19fV0sWzUsNywiXFx2YXJwaGkiXSxbNyw2LCJcXGNvbmciXSxbNSw0LCJcXGNpcmMgdSIsMl0sWzQsNiwiXFxjb25nIiwyXSxbNSw3LCJcXGNvbmciLDIseyJzdHlsZSI6eyJib2R5Ijp7Im5hbWUiOiJub25lIn0sImhlYWQiOnsibmFtZSI6Im5vbmUifX19XSxbNCw1LCJcXGNvbmciLDIseyJzdHlsZSI6eyJib2R5Ijp7Im5hbWUiOiJub25lIn0sImhlYWQiOnsibmFtZSI6Im5vbmUifX19XV0=
\[\begin{tikzcd}
		{\Homomorphism(G, H)} && {\Homomorphism\left(\bigoplus\limits_{\alpha\in I}\mathbb{Z}, H\right)} \\
		f && {\prod\limits_{\alpha\in I}\Homomorphism\left(\mathbb{Z}, H\right)} & {f\circ u^{-1}} \\
		&& {\prod\limits_{\alpha\in I}H} & {(f\circ u^{-1}\circ i_{\alpha})_{\alpha\in I}} \\
		&&& {\left(f\circ u^{-1}\circ i_\alpha(1)\right)_{\alpha\in I}}
		\arrow[maps to, from=2-4, to=3-4]
		\arrow[maps to, from=3-4, to=4-4]
		\arrow["\cong"', maps to, from=2-1, to=4-4]
		\arrow[maps to, from=2-1, to=2-4]
		\arrow["\varphi", from=1-3, to=2-3]
		\arrow["\cong", from=2-3, to=3-3]
		\arrow["{\circ u}"', from=1-3, to=1-1]
		\arrow["\cong"', from=1-1, to=3-3]
		\arrow["\cong"', draw=none, from=1-3, to=2-3]
		\arrow["\cong"', draw=none, from=1-1, to=1-3]
	\end{tikzcd}\]
We see that
\[
	f\circ u^{-1} i_\alpha (1) = f\circ u^{-1} (e_\alpha )= f(\epsilon _\alpha ).
\]
In other words, for all \hyperref[def:Abelian-group]{Abelian group} \(H\), the morphism from the set \(\{\epsilon _\alpha \}_{\alpha \in I}\) to \(H\) can be uniquely extended to the group
a homomorphism from \(G\) to \(H\).

\hr

We now want to generate \hyperref[def:free-Abelian-group]{free Abelian group} by a set. Roughly speaking, given a set \(S\), we can generate a \hyperref[def:free-Abelian-group]{free Abelian group} \(Z\)
by defining
\[
	Z\coloneqq \left\{\sum\limits_{x\in S}^{} n_{x} x\mid n_{x} \in \mathbb{\MakeUppercase{z}} , \text{\# non-zero elements in \(n_x < \infty\)}\right\}
\]
with the trivial definition of \(+\). Formally, we have the following.

\begin{definition}[Free Abelian group generated by sets]\label{def:free-Abelian-group-generated-by-sets}
	Given a set \(S\), the \emph{\hyperref[def:free-Abelian-group]{free Abelian group} generated by \(S\)} \((Z, +)\) is defined as
	\[
		Z\coloneqq \left\{f\colon S\to \mathbb{\MakeUppercase{z}} \mid \text{only finitely many \(x\in S\) such that \(f(x)\neq 0\)}\right\},
	\]
	with
	\[
		\begin{split}
			+\colon Z\times Z&\to Z\\
			(f, g)&\mapsto f + g.
		\end{split}
	\]
\end{definition}

\begin{remark}
	\(\{\phi _{x} \mid x\in S\}\) forms a basis of \(Z\), where \(\phi _{x} \colon S\to \mathbb{\MakeUppercase{z}} \) such that
	\[
		y\mapsto \phi _{x} (y) = \begin{dcases}
			1, & \text{ if } y=x ;    \\
			0, & \text{ if } y \neq x
		\end{dcases}
	\]
	is the characteristic function at \(x\). We see this by for all \(f\in S\), \(f = \sum\limits_{x\in S}^{} f(x)\phi _{x} \), which is uniquely defined. Hence, \((Z, +)\) is a \hyperref[def:free-Abelian-group]{free Abelian group}.
\end{remark}

\begin{note}
	Note that
	\[
		\begin{split}
			S &\overset{1:1}{\longleftrightarrow} \{\phi _{x} \mid x\in S\}\\
			x &\mapsto \phi _{x}.
		\end{split}
	\]
	Hence, we often denote the element \(\sum\limits_{x\in S}^{} n_{x} \phi _{x} \) in \(Z\) as
	\[
		\sum\limits_{x\in S}n_{x} \cdot x.
	\]
\end{note}
\begin{theorem}[The universal property of Abelian group]\label{thm:universal-property-of-Abelian-group}
	Denote a canonical embedding \(i\colon S\to Z\), \(x\mapsto \phi _{x} \). Then for all \hyperref[def:Abelian-group]{Abelian group} \(H\) and \(f\colon S\to H\), there exists a unique group homomorphism
	\[
		\widetilde{f} \colon Z\to H
	\]
	such that \(\widetilde{f} \circ i = f\).
\end{theorem}
\begin{proof}
	We define
	\[
		\widetilde{f} \left(\sum\limits_{x\in S}^{} n_{x} \cdot x\right)\coloneqq \sum\limits_{x\in S}n_{x} f(x),
	\]
	and the uniqueness is obvious.
\end{proof}

Note that we can use the \hyperref[thm:universal-property-of-Abelian-group]{universal property of Abelian group} to describe a \hyperref[def:free-Abelian-group]{free Abelian group} as follows.
\begin{proposition}
	Given \(Z^\prime \) as another \hyperref[def:Abelian-group]{Abelian group} and \(i^\prime \colon S\to Z^\prime \) as another canonical embedding such that for all \hyperref[def:Abelian-group]{Abelian group} \(H\)
	and \(f\colon S\to H\), there exists a unique group homomorphism \(\widetilde{f} \colon Z^\prime \to H\) such that \(\widetilde{f} \circ i^\prime = f\), then
	\[
		Z^\prime \cong Z.
	\]
\end{proposition}

\begin{theorem}
	Assume \(G\) is a \hyperref[def:free-Abelian-group]{free Abelian group}. Assume there exists a finite basis \(\{g_1, \ldots , g_{n}  \}\) of \(G\), and also assume that there exists another basis
	\(\{h_\alpha \}_{\alpha \in I}\). Then we have
	\[
		\mathrm{card}(I)< \infty,
	\]
	specifically, we have
	\[
		\mathrm{card} (I)\leq n.
	\]
\end{theorem}
\begin{proof}
	Suppose \(I\) is an infinite set, then we can find \(h_{\alpha _1}, \ldots , h_{\alpha _m} \) such that \(m>n\) and \(h_{\alpha _i} \neq h_{\alpha _j}\) for \(i \neq j\). Then we have
	\[
		h_{\alpha _{i} } = \sum\limits_{j=1}^{n} k_{i} ^j g_{j} , \forall i = 1, \ldots , m.
	\]
	\[
		\begin{pmatrix}
			h_{\alpha _1} \\
			\vdots        \\
			h_{\alpha _m} \\
		\end{pmatrix} = \underbrace{\begin{pmatrix}
				k_1^1  & k_1^2 & \ldots & k_1^{n} \\
				\vdots &       & \ddots & \vdots  \\
				k_m^1  & k_m^2 & \ldots & k_m^{n} \\
			\end{pmatrix}}_{K\in M_{m\times n}(\mathbb{\MakeUppercase{z}})\subset M_{m\times n}(\mathbb{\MakeUppercase{q}} )}\begin{pmatrix}
			g_1    \\
			\vdots \\
			g_{n}  \\
		\end{pmatrix},
	\]
	where \(k_{i} ^j\in \mathbb{\MakeUppercase{z}} \). From linear algebra, we know that there exists \((r_1, \ldots, \mathrm{m} )\in \mathbb{\MakeUppercase{q}} ^m\setminus \{0\}\) such that
	\[
		(r_1, \ldots , v_{m}) K = (0, \ldots , 0 ).
	\]
	Multiplying both sides with the common multiple of the denominator of \(r_{i}\), we see that there exists \((\ell _1, \ldots \ell _{m})\in \mathbb{\MakeUppercase{z}} ^m\setminus \{0\}\) such that
	\[
		\begin{split}
			&(\ell _1, \ldots \ell _{m}  )K = (0, \ldots , 0 )\\
			\implies &(\ell _1, \ldots , \ell _{m}  )\begin{pmatrix}
				h_{\alpha _1} \\
				\vdots        \\
				h_{\alpha _m} \\
			\end{pmatrix} = (\ell _1, \ldots , \ell _m )K \begin{pmatrix}
				g_1    \\
				\vdots \\
				g_n    \\
			\end{pmatrix} = (0, \ldots , 0)\\
			\implies &\sum\limits_{i=1}^{m} \ell _{i} h_{\alpha _{i} }= 0 \implies \mathrm{card} (I) < \infty.
		\end{split}
	\]
	From the same argument, we see that \(\mathrm{card} (I) \leq n\implies \mathrm{card} (I) = n\).
\end{proof}

\begin{remark}
	Furthermore, one can prove that if \(G\) is a \hyperref[def:free-Abelian-group]{free Abelian group}, then we can prove that any two bases of \(G\) are equinumerous.
\end{remark}
This induces the following definition.

\begin{definition}[Rank]\label{def:rank}
	Let \(G\) ba a \hyperref[def:free-Abelian-group]{free Abelian group}, the \emph{rank} of \(G\) is the cardinality of any basis of \(G\).
\end{definition}


\subsection{Finitely Generated Abelian Group}
Since we're going to encounter some group as
\[
	\mathbb{\MakeUppercase{z}} \oplus \quotient{\mathbb{\MakeUppercase{z}}}{2\mathbb{\MakeUppercase{z}} },
\]
hence it's useful to look into those finitely generated \hyperref[def:Abelian-group]{Abelian group}.