\lecture{24}{09 Mar. 10:00}{Chain Homotopy}
To prove \autoref{thm:functoriality-is-homotopy-invariant}, we introduce some \hyperref[sec:homological-algebra]{homological algebra}.

\begin{definition}[Chain homotopy]\label{def:chain-homotopy}
	Given \hyperref[def:chain-complex]{chain complexes} $(A_\ast, d^A_\ast)$ and $(B_\ast, d^B_\ast)$ and \hyperref[def:chain-map]{chain maps}
	$f_\#, g_\# \colon  A_\ast \to B_\ast$. A \emph{chain homotopy} from $f_\#$ to $g_\#$ is a sequence of group homomorphisms $\psi_n\colon A_n \to B_{n + 1}$ such that
	\[
		f_n - g_n = \partial ^B_{n + 1} \circ \psi_n + \psi_{n - 1} \circ \partial _n^A.
	\]
	In diagram, letting $h_n = f_n - g_n$, we have the following.
	\[\begin{tikzcd}
			\ldots && {A_{n+1}} && {A_{n}} && {A_{n-1}} && \ldots \\
			\\
			\ldots && {B_{n+1}} && {B_{n}} && {B_{n-1}} && \ldots
			\arrow["{\partial^A_{n-1}}", from=1-7, to=1-9]
			\arrow["{\partial^B_{n-1}}", from=3-7, to=3-9]
			\arrow["{\partial^A_{n}}", color={rgb,255:red,92;green,214;blue,92}, from=1-5, to=1-7]
			\arrow["{\partial^B_{n}}", from=3-5, to=3-7]
			\arrow["{\partial^B_{n+1}}", color={rgb,255:red,92;green,214;blue,92}, from=3-3, to=3-5]
			\arrow["{\partial^A_{n+2}}", from=1-1, to=1-3]
			\arrow["{\partial^B_{n+2}}", from=3-1, to=3-3]
			\arrow["{h_{n+1}}", from=1-3, to=3-3]
			\arrow["{h_{n}}", color={rgb,255:red,214;green,92;blue,92}, from=1-5, to=3-5]
			\arrow["{h_{n-1}}", from=1-7, to=3-7]
			\arrow["{\partial^A_{n+1}}", from=1-3, to=1-5]
			\arrow["{\psi_n}"{description}, color={rgb,255:red,92;green,214;blue,92}, from=1-5, to=3-3]
			\arrow["{\psi_{n-1}}"{description}, color={rgb,255:red,92;green,214;blue,92}, from=1-7, to=3-5]
		\end{tikzcd}\]
	This diagram does \textbf{not} commute, however, the \textbf{\textcolor{red}{red}} map is the sum of the
	\textbf{\textcolor{green}{green}} maps composed up, so it shows everything that is going on.
\end{definition}

\begin{theorem}\label{thm:chain-homotopies-on-homology}
	If there is a \hyperref[def:chain-homotopy]{chain homotopy} $\psi$ from $f_\#$ to $g_\#$, then the induced maps \(f_\ast, g_\ast\) on \hyperref[def:homology-group]{homology} are equal.
\end{theorem}
\begin{proof}
	Let $\sigma \in A_n$ be an $n$-cycle, i.e. $\partial_n^A \sigma = 0$. Then we compute that:
	\begin{align*}
		(f_n - g_n)(\sigma) = \partial_{n + 1}^B(\psi_n(\sigma)) + \psi_{n - 1}(\partial_n^A(\sigma)) = \partial_{n + 1}^B(\psi_n(\sigma)) \in \im (\partial^B_{n + 1})
	\end{align*}
	This tells us that $(f_n - g_n)(\sigma)$ is a \hyperref[def:boundary]{boundary}, and so $(f_n - g_n)(\sigma) = 0$ when considered as an element of
	the \hyperref[def:homology-group]{homology group} (with degree \(n\)).
	Thus, $f_n(\sigma) = g_n(\sigma)$ in the homology group, and so $f, g$ induce the same map as desired.
\end{proof}
We now sketch the proof of \autoref{thm:functoriality-is-homotopy-invariant} given in Hatcher\cite{hatcher2002algebraic}. From this point in the course many
of the theorems require much more algebraic work than we are interested in. We instead want to learn how to use the computational tools.

Now we give the proof idea for \autoref{thm:functoriality-is-homotopy-invariant}.
\begin{proof}
	Suppose we have some \hyperref[def:homotopy]{homotopy} $F \colon I \times X \to Y$ from $f$ to $g$. The most difficulty in this proof is the
	combinatorial difficulty involved in the fact that the product of a \hyperref[def:standard-simplex]{simplex} in $X$ and $I$ is not a \hyperref[def:standard-simplex]{simplex}.

	We now consider
	\begin{enumerate}
		\item Subdivide $\Delta^n \times I$ into $(n + 1)$-dimensional \hyperref[def:subsimplex]{subsimplices}.\footnote{We want to do this since the product between two \hyperref[def:standard-simplex]{simplices} is not a \hyperref[def:standard-simplex]{simplex}, as we just note.}
		      \begin{figure}[H]
			      \centering
			      \incfig{pf:functoriality-is-homotopy-invariant}
			      \label{fig:pf:functoriality-is-homotopy-invariant}
		      \end{figure}
		\item We define the \underline{prism} operator:
		      \begin{align*}
			      P_n \colon C_n(X)         & \to C_{n + 1}(Y)                                                                                              \\
			      [\sigma : \Delta^n \to X] & \mapsto \begin{bmatrix} \text{ alternating sums of restrictions }                                             \\
				                                          \Delta^n \times I \xrightarrow{\sigma \times \identity } X \times I \xrightarrow{F} Y \\
				                                          \text{ to each \hyperref[def:standard-simplex]{simplex} in our subdivision }
			                                          \end{bmatrix}
		      \end{align*}
		\item We now need to check that
		      \[
			      \partial_{n + 1}^YP_n = \fcolorbox{blue}{white}{$g_\#$} - \fcolorbox{red}{white}{$f_\#$} - \fcolorbox{green}{white}{$P_{n - 1}\partial_n^X$}.
		      \]

		      We have the following diagram.
		      \begin{figure}[H]
			      \centering
			      \incfig{pf:functoriality-is-homotopy-invariant-2}
			      \label{fig:pf:functoriality-is-homotopy-invariant-2}
		      \end{figure}
		      Thus $P$ is a \hyperref[def:chain-homotopy]{chain homotopy}, and we're done.
	\end{enumerate}
\end{proof}