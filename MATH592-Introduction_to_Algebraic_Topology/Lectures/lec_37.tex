\lecture{37}{8 Apr. 10:00}{Simplicial Approximation}
\begin{eg}
	Let \(f \colon S^n \to S^n\) be a \hyperref[def:degree]{degree} \(d\) map. Then \(\tau(f)\) is
	\[
		(-1)^0\trace(f_\ast : H_0(S^n) \to H_0(S^n)) + (-1)^n\trace(f_\ast : H_n(S^n) \to H_n(S^n)).
	\]

	Since \(f_\ast \colon H_0(S^n) \to H_0(S^n)\) is the identity, and \(f_\ast \colon H_n(S^n) \to H_n(S^n)\) is given by the
	\(1 \times 1\) matrix with entry \(d\). And then we have
	\[
		\tau(f) = 1 + (-1)^n d.
	\]
\end{eg}

\begin{corollary}
	\(f\colon S^n \to S^n\) has a fixed point whenever \(1 + (-1)^n \neq 0\). Namely, whenever \(d \neq (-1)^{n + 1}\). That is \(f\) has a fixed point if its
	\hyperref[def:degree]{degree} is not equal to the \hyperref[def:degree]{degree} of the antipodal map.
\end{corollary}

\begin{exercise}
	If \(f \colon X \to X\), then \(\trace(f_\ast \colon H_0(X) \to H_0(X))\) is equal to the number of \hyperref[def:path]{path}-components of \(X\) mapped
	to themselves.
\end{exercise}

\begin{exercise}
	If \(X\) is \hyperref[def:contractible]{contractible}, then for \(f\colon X\to X\), \(\tau(f) = 1\).
\end{exercise}
\begin{answer}
	Since the \hyperref[def:homology-group]{homology} of \(f\) is concentrated in \hyperref[def:degree]{degree} zero, the result follows.
\end{answer}

\begin{exercise}
	If \(X\) is a \hyperref[def:contractible]{contractible} compact manifold or finite \hyperref[def:CW-Complex]{CW complex}, every \(f\colon X\to X\)
	has a fixed point.
\end{exercise}
\begin{answer}
	From the last exercise, we have \(\tau (f) = 1\). And since we have compactness, by \autoref{thm:Lefschetz-fixed-point},
	the result follows.
	\begin{remark}
		In particular, this recovers \hyperref[thm:Brouwer-fixed-point]{Brouwer's Fixed Point Theorem}.
	\end{remark}
\end{answer}
\begin{eg}
	If we consider the map \(f \colon \mathbb{R} \to \mathbb{R}\) given by translation by \(x \neq 0\), then \(\tau(f) = 1\), but \(f\) does not have a fixed point.
\end{eg}
\begin{explanation}
	The key here is that \(\mathbb{R}\) is not compact.
\end{explanation}

\begin{eg}[QR May 2016]
	Let \(X\) be a finite, connected \hyperref[def:CW-Complex]{CW complex}. \(\widetilde{X}\) is its \hyperref[def:universal-covering]{universal cover},
	and \(\widetilde{X}\) is compact. Show that \(\widetilde{X}\) cannot be \hyperref[def:contractible]{contractible} unless \(X\) is \hyperref[def:contractible]{contractible}.
\end{eg}
\begin{explanation}
	We actually have two different approaches.
	\begin{enumerate}
		\item By homework, we then know that, since \(\widetilde{X}\) is \hyperref[def:contractible]{contractible} and \(\widetilde{X}\)
		      has finitely many sheets \(d\) over \(X\),
		      \[
			      1 = \chi(\widetilde{X}) = d \cdot \chi(X).
		      \]
		      Therefore, \(\chi(X) = d = 1\), and so \(p \colon \widetilde{X} \to X\) is a \(1\)-sheeted \hyperref[def:covering-map]{cover}, so
		      it is a homeomorphism. Therefore, \(X\) is \hyperref[def:contractible]{contractible}.
		\item Since \(\widetilde{X}\) is \hyperref[def:contractible]{contractible}, \(\tau(f) = 1\) for all \(f \colon \widetilde{X} \to \widetilde{X}\).
		      Furthermore, because \(\widetilde{X}\) is compact and covers a finite \hyperref[def:CW-Complex]{CW complex}, it is a finite \hyperref[def:CW-Complex]{CW complex}.
		      Therefore, the \hyperref[thm:Lefschetz-fixed-point]{Lefschetz fixed point theorem} applies, so any such map has a fixed point. If \(f\) is a
		      \hyperref[def:deck-transformation]{deck map}, then that means that \(f = \identity_{\widetilde{X}}\) from \autoref{col:lec-17}.

		      We have proved then that \(X \cong \quotient{\widetilde{X}}{G(\widetilde{X})}\) because \(p \colon \widetilde{X} \to X\) is \hyperref[def:normal-cover]{normal},
		      but then the \hyperref[def:deck-transformation]{deck group} \(G(\widetilde{X})\) is trivial, so \(X \cong \widetilde{X}\), and we are done.
	\end{enumerate}
\end{explanation}
\begin{exercise}
	A \(1\)-sheeted \hyperref[def:covering-map]{cover} is always injective and surjective. Furthermore, it's a local homeomorphism.
	This suffices to show that a \(1\)-sheeted cover is a homeomorphism.
\end{exercise}

\begin{theorem}\label{thm-}
	If \(X\) is a finite \hyperref[def:CW-Complex]{CW complex}, with \hyperref[def:cellular-chain-group]{cellular chain groups} \(H_n(X^n, X^{n - 1})\).
	If we have a cellular map \(f \colon X \to X\), so \(f\) induces maps \(f_\ast \colon H_n(X^n, X^{n - 1}) \to H_n(X^n, X^{n - 1})\). Then
	\[
		\tau(f) = \sum_n (-1)^n \trace(f_\ast : H_n(X^n, X^{n - 1}) \to H_n(X^n, X^{n - 1})).
	\]
\end{theorem}
\begin{proof}
	Do some algebra! This is a purely algebraic fact
	\begin{exercise}
		Given a commutative diagram with \hyperref[def:exact]{exact} rows
		\[
			\begin{tikzcd}
				0 & A & B & C & 0 \\
				0 & {A^\prime} & {B^\prime} & {C^\prime} & 0
				\arrow["\gamma", from=1-4, to=2-4]
				\arrow["\beta", from=1-3, to=2-3]
				\arrow["\alpha", from=1-2, to=2-2]
				\arrow[from=2-1, to=2-2]
				\arrow[from=2-2, to=2-3]
				\arrow[from=2-4, to=2-5]
				\arrow[from=2-3, to=2-4]
				\arrow[from=1-4, to=1-5]
				\arrow[from=1-3, to=1-4]
				\arrow[from=1-2, to=1-3]
				\arrow[from=1-1, to=1-2]
			\end{tikzcd}
		\]
		then \(\trace(\beta) = \trace(\alpha) + \trace(\gamma)\).
	\end{exercise}

	Using the above result, the theorem follows by an argument analogous to the argument for Euler Characteristic in Homework.
\end{proof}

\section{Simplicial Approximation Theorem}
\begin{prev}
	Recall the definition of \hyperref[def:simplicial-complex]{simplicial complex}.
\end{prev}

\begin{definition}[Simplicial map]\label{def:simplicial-map}
	A \emph{simplicial map} \(f \colon K \to L\) is a continuous map that sends each \hyperref[def:standard-simplex]{simplex} of \(K\) to a (possibly smaller dimensional)
	\hyperref[def:standard-simplex]{simplex} of \(L\) by a linear map in the form of
	\[
		\sum t_iv_i \mapsto \sum t_if(v_i).
	\]
\end{definition}
\begin{remark}
	A \hyperref[def:simplicial-map]{simplicial map} is completely determined by its restriction to the vertex set.
\end{remark}
