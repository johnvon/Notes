\lecture{9}{26 Jan. 10:00}{Calculate Fundamental Group}
\begin{figure}[H]
	\centering
	\incfig{fundamental-group-hole-detector}
	\caption{\hyperref[def:fundamental-group]{Fundamental Group} is basically a \emph{hole detector}!}
	\label{fig:fundamental-group-hole-detector}
\end{figure}

\subsection{Calculations with \(\pi _1(S^n)\)}
Let's start with a simple theorem.
\begin{theorem}
	\(\pi _1(S^1)\cong \mathbb{\MakeUppercase{z}} \), and this identification is given by the paths
	\[
		n \leftrightarrow [\omega_{n}(t) = (\cos (2\pi nt), \sin (2\pi nt))].
	\]
\end{theorem}
\begin{remark}
	Intuitively, this winds around \(S^1\) \(n\) times. The key to this proof was to understand \(S^1\) via the covering space \(\mathbb{\MakeUppercase{r}} \to S^1\).
	We will talk about covering spaces more later.
\end{remark}
\begin{proof}
	\todo{HW}
\end{proof}

\begin{theorem}
	Given \((X, x_0)\) and \((Y, y_0)\), then
	\[
		\pi (X\times Y, (x_0, y_0)) \cong \pi_1(X, x_0)\times \pi _1(Y, y_0)
	\]
	such that
	\[
		\left[\begin{alignedat}{3}
				r\colon I&\to X\times Y\\
				r(t) &= \left(r_X(t), r_Y(t)\right)
			\end{alignedat}\right] \mapsto (r_{X}, r_{Y}),
	\]
	where \(\gamma\) is continuous \(\iff f_{X}, f_{Y}\) are continuous.
\end{theorem}
\begin{proof}
	Let \(Z\overset{f}{\to} X\times Y\) with \(z\overset{f}{\mapsto} \left(f_{X}(z), f_{Y}(z)\right)\). Then we have
	\[
		\text{continuous}\iff f_{X}, f_{Y}  \text{ are continuous}.
	\]
	Now, apply above to
	\begin{itemize}
		\item \hyperref[def:path]{Paths} \(I\to X\times Y\).
		\item \hyperref[def:homotopy-path]{Homotopies of paths} \(I\times I\to X\times Y\).
	\end{itemize}
\end{proof}

\begin{corollary}
	The torus \(T\cong S^{1}\times S^1\) has \hyperref[def:fundamental-group]{fundamental group} \(\pi _1(T)\cong \mathbb{\MakeUppercase{z}} ^2\). Additionally,
	for a \(k\)-torus \(\underbrace{S^{1}\times S^1 \times \ldots \times S^1}_{k\text{ times}} = (S^1)^k\), the \hyperref[def:fundamental-group]{fundamental group}
	is then \(\mathbb{\MakeUppercase{z}} ^k\), i.e.
	\[
		\pi _1\left((S^1)^k\right) \cong \mathbb{\MakeUppercase{z}} ^k.
	\]
\end{corollary}
\begin{proof}
	Since
	\[
		\pi _1 \cong \mathbb{\MakeUppercase{z}}^2 \cong \mathbb{\MakeUppercase{z}} _a \oplus \mathbb{\MakeUppercase{z}} _b.
	\]
	\begin{figure}[H]
		\centering
		\incfig{pf:torus-fundamental-group}
		\label{fig:pf:torus-fundamental-group}
	\end{figure}
\end{proof}
\begin{remark}
	One way to think of the \(k\)-torus is as a \(k\)-dimensional cube with opposite \((k-1)\)-dimensional faces identified by translation.
	\begin{figure}[H]
		\centering
		\incfig{3-torus}
		\caption{\(3\)-torus with cube identified with parallel sides.}
		\label{fig:3-torus}
	\end{figure}
\end{remark}

\begin{eg}
	We now see some examples.
	\begin{enumerate}
		\item \(\pi _1(S^{\infty }\times S^1) \cong \mathbb{\MakeUppercase{z}} \)
		\item \(\pi _1(\mathbb{\MakeUppercase{r}} ^2\setminus \{0\}) \cong 0\times \mathbb{\MakeUppercase{z}} = \mathbb{\MakeUppercase{z}}\) since
		      \[
			      \mathbb{\MakeUppercase{r}} ^2\setminus \{0\}\cong S^{1}\times \mathbb{\MakeUppercase{r}},
		      \]
		      which means that the generators are just loops around the hold intuitively.
	\end{enumerate}
\end{eg}

\begin{theorem}
	\(\pi _1\) is a \hyperref[def:functor]{functor} such that
	\[
		\begin{split}
			\pi _1\colon \underline{\mathrm{Top}_*} &\to \underline{\mathrm{Gp}}\\
			(X, x_0)&\mapsto \pi _1(X, x_0).
		\end{split}
	\]

	A map \(f\colon X\to Y\) taking base point \(x_0\) to \(y_0\) induces a map
	\[
		\begin{split}
			f_*\colon \pi _1(X, x_0)&\to \pi _1(Y, y_0)\\
			[\gamma]&\mapsto [f\circ \gamma]
		\end{split}
	\]
	i.e.,
	\[
		\left[f\colon X\to Y\right] \mapsto \left[f_*\colon \pi _1(X, x_0)\to \pi _1(Y, f(x_0))\right].
	\]
\end{theorem}
\begin{notation}
	We usually write \(f_*\) if it's a \hyperref[def:functor]{covarant functor}, while writing \(f^*\)
	if it's a \hyperref[def:contravariant-functor]{contravariant functor}.
\end{notation}
\begin{proof}
	We need to check
	\begin{itemize}
		\item well-defined on \hyperref[def:homotopy-path]{path homotopy} classes.
		\item \(f_*\) is a group homomorphism.
		      \[
			      f_*(\alpha \cdot \beta ) = f_*(\alpha )\cdot f_*(\beta ) = \begin{dcases}
				      f(\alpha (2s)),  & \text{ if }  s\in \left[0, \frac{1}{2}\right]  \\
				      f(\beta (1-2s)), & \text{ if }  s\in \left[\frac{1}{2}, 1\right].
			      \end{dcases}
		      \]
		\item \(\left(\identity_{(X, x_0)} \right)_* = \identity_{\pi _1(X, x_0)} \)
		\item \((f_*\circ g_*) = (f\circ g)_*\)
		      \[
			      (f\circ g)_*[\gamma] = [f\circ g\circ \gamma] = [f\circ (g\circ \gamma)]\implies f_*(\gamma_*(\gamma)).
		      \]
	\end{itemize}
	\todo{DIY}
\end{proof}