\lecture{10}{26 Jan. 10:00}{Seifert-Van Kampen Theorem}
The goal is to compute \(\pi _1(X)\) where \(X = A\cup B\) using the data
\[
	\pi _1(A), \pi _1(B), \pi _1(A\cap B).
\]

\subsection{Free Product}
\subsubsection{Free Product}
We first introduce a definition.
\begin{definition}[Free product]\label{def:free-product}
	Given some collections of groups \(\{G_{\alpha }\}_{\alpha }\), the \emph{free product},
	denoted by \(\underset{\alpha}{\ast}G_{\alpha }\) is a group such that
	\begin{itemize}
		\item Elements: \hyperref[def:word]{Words} in \(\{g\colon g\in G_{\alpha } \text{ for any \(\alpha\)}\}\) modulo by the equivalence relation
		      generated by
		      \[
			      wg_{i}g_{j}v\sim w(g_{i}g_{j})v
		      \]
		      when both \(g_{i}, g_{j}\in G_{\alpha }\). Also, for the identity element \(\identity_{}= e_{\alpha }\in G_{\alpha }\)
		      for any \(\alpha \) such that
		      \[
			      w e_{\alpha }v\sim wv.
		      \]
		      Specifically,
		      \[
			      \ast_\alpha G_\alpha \coloneqq \quotient{\left\{\text{\hyperref[def:word]{words} in \(\{G_\alpha \}_{\alpha}\)} \right\}}{\sim}.
		      \]
		\item Operation: Concatenation of \hyperref[def:word]{words}.
	\end{itemize}
\end{definition}
\begin{remark}
	In particular, we have the following universal property of \(\ast_\alpha G_\alpha \). For every \(\alpha \), there is a \(\iota _\alpha \) such that
	\[
		\iota _\alpha \colon G_\alpha \to \ast_\alpha G_\alpha ,\qquad g\mapsto \overline{g},
	\]
	where \(\iota _\alpha \) is a group homomorphism obviously. Further, \((\ast_\alpha G_\alpha ,\iota _\alpha )\) satisfies the following property: For every
	group \(H\) and a group homomorphism \(\varphi _\alpha \colon G_\alpha \to G\) for all \(\alpha \), there exists an unique group homomorphism \(\varphi \colon \ast_\alpha G_\alpha \to H\)
	such that \(\varphi \circ \iota _\alpha = \varphi _\alpha \), i.e., the following diagram commutes.
	\[\begin{tikzcd}
			{G_{\alpha_1}} \\
			{G_{\alpha_2}} && {\ast_\alpha G_\alpha} \\
			\vdots \\
			&&& H
			\arrow["{\iota_{\alpha_1}}", from=1-1, to=2-3]
			\arrow["{\iota_{\alpha_2}}"', from=2-1, to=2-3]
			\arrow["{\exists ! \varphi}", dashed, from=2-3, to=4-4]
			\arrow["{\varphi_{\alpha_1}}"{description}, from=1-1, to=4-4]
			\arrow["{\varphi_{\alpha_2}}"{description}, from=2-1, to=4-4]
		\end{tikzcd}\]
	\begin{proof}
		The proof is straightforward. Firstly, we define \(w = \overline{g_1 g_2\ldots g_n}\in \ast_\alpha G_\alpha\), \(g_{i} \in G_{\alpha_{i} } \),
		\[
			\varphi (w)\coloneqq \varphi _{\alpha _1}(g_1)\ldots \varphi _{\alpha _{n} }(g_{n} ).
		\]
		Now, we just need to check
		\begin{itemize}
			\item It's well-defined, since \(\varphi_\alpha \) is a group homomorphism.
			\item \(\varphi \) is a group homomorphism.
			\item \(\varphi \circ \iota _\alpha =\varphi _\alpha \).
			\item Such \(\varphi \) is unique. Suppose there exists another \(\psi \colon \ast_\alpha G_\alpha \to H\), then
			      \[
				      \psi \circ \iota _\alpha = \varphi _\alpha \implies \underset{g\in G_\alpha }{\forall }\ \psi (\overline{g} )= \psi _\alpha (g),
			      \]
			      But then for every \(w = \overline{g_1 g_2\ldots g_n}\in \ast_\alpha G_\alpha\), \(g_{i} \in G_{\alpha_{i} } \), we have
			      \[
				      \psi (w) = \psi (\overline{g_1}\ldots \overline{g_n}) = \psi (\overline{g_1}) \ldots \psi (\overline{g_n})  = \psi _{\alpha _1}(\overline{g_1}) \ldots \psi _{\alpha_n}(\overline{g_n}),
			      \]
			      which is just \(\varphi \).
		\end{itemize}
	\end{proof}
\end{remark}
\begin{remark}
	We further claim that this universal property determines such \hyperref[def:free-product]{free product} uniquely. i.e., assume there are another group \(\widetilde{G} \) and
	\(\widetilde{\iota} _\alpha \colon G_\alpha \to \widetilde{G}\). Assume \((\widetilde{G} , \widetilde{\iota} _\alpha  )\) also satisfies the following property: For every
	group \(H\) and group homomorphism \(\varphi _\alpha \colon G_\alpha \to H\), then there exists a unique group homomorphism \(\varphi \colon \widetilde{G} \to H\) such that
	the following diagram commutes.
	\[\begin{tikzcd}
			{G_{\alpha_1}} \\
			{G_{\alpha_2}} && {\ast_\alpha \widetilde{G} _\alpha} \\
			\vdots \\
			&&& H
			\arrow["{\widetilde{\iota}_{\alpha_1}}", from=1-1, to=2-3]
			\arrow["{\widetilde{\iota}_{\alpha_2}}"', from=2-1, to=2-3]
			\arrow["{\exists ! \varphi}", dashed, from=2-3, to=4-4]
			\arrow["{\varphi_{\alpha_1}}"{description}, from=1-1, to=4-4]
			\arrow["{\varphi_{\alpha_2}}"{description}, from=2-1, to=4-4]
		\end{tikzcd}\]
	Then, \(\widetilde{G} \cong \ast_\alpha G_\alpha \).
	\begin{proof}
		Assume \((\widetilde{G} , \widetilde{\iota} _\alpha  )\) satisfies the universal property mentioned above. Then from the universal property and viewing \(\widetilde{G} \) and
		\(\ast_\alpha G_\alpha \) as \(H\) separately, we obtain the following diagram.
		\[\begin{tikzcd}
				{\widetilde{G}} &&&&&& {\ast_\alpha G_\alpha} \\
				&&& {G_{\alpha_1}} \\
				&&& {G_{\alpha_2}} \\
				&&& \vdots
				\arrow["{\iota_{\alpha_1}}", from=2-4, to=1-7]
				\arrow["{\iota_{\alpha_2}}", from=3-4, to=1-7]
				\arrow["{\exists ! f}"', shift right=1, dashed, from=1-7, to=1-1]
				\arrow["{\widetilde{\iota}_{\alpha_1}}"', from=2-4, to=1-1]
				\arrow["{\widetilde{\iota}_{\alpha_2}}"', from=3-4, to=1-1]
				\arrow[from=4-4, to=1-1]
				\arrow[from=4-4, to=1-7]
				\arrow["{\exists! g}"', shift right=1, dashed, from=1-1, to=1-7]
			\end{tikzcd}\]
		We claim that
		\[
			g\circ f = \identity_{},\quad f\circ g = \identity_{}.
		\]
		To see this, we simply apply the same observation, for example,
		\[\begin{tikzcd}
				{G_{\alpha_1}} \\
				{G_{\alpha_2}} && {\widetilde{G}} \\
				\vdots \\
				&&& {\widetilde{G}}
				\arrow["{\widetilde{\iota}_{\alpha_1}}", from=1-1, to=2-3]
				\arrow["{\widetilde{\iota}_{\alpha_2}}"', from=2-1, to=2-3]
				\arrow["{\exists ! g\circ f}", dashed, from=2-3, to=4-4]
				\arrow["{\widetilde{\iota}_{\alpha_1}}"{description}, from=1-1, to=4-4]
				\arrow["{\widetilde{\iota}_{\alpha_2}}"{description}, from=2-1, to=4-4]
			\end{tikzcd}\]
		where \(g\circ f\) comes from the previous diagram. But notice that \(\identity_{} \) let the diagram commutes also, and since it's unique, hence \(g\circ f = \identity_{} \). Similarily,
		we have \(f\circ g = \identity_{} \).
	\end{proof}
\end{remark}

If you're careful enough, you may find out that all we're doing is just writing out a specific example of \autoref{lma:lec7}! Indeed, this is exactly the construction of a
\hyperref[def:free-group]{free group}.

\begin{definition}[Fibered coproduct]\label{def:fibered-coproduct}
	Given a \hyperref[def:category]{category} \(\mathscr{C}\), let \(f\colon Z\to X\), \(g\colon Z\to Y\). The \emph{fibered coproduct between \(f\) and \(g\)} is the
	data \((W, p_1, p_2)\), where \(W\in \Object (\mathscr{C} )\), \(;_1\colon X\to W\), \(p_2\colon Y\to W\) satisfy the following.
	\begin{itemize}
		\item The diagram commutes.
		      \[\begin{tikzcd}
				      Z & X \\
				      Y & W
				      \arrow["g"', from=1-1, to=2-1]
				      \arrow["{p_2}"', from=2-1, to=2-2]
				      \arrow["{p_1}", from=1-2, to=2-2]
				      \arrow["f", from=1-1, to=1-2]
			      \end{tikzcd}\]
		\item For every \(u\colon X\to U\), \(v\colon Y\to U\) such that the following diagram commutes
		      \[\begin{tikzcd}
				      Z & X \\
				      Y & W \\
				      && U
				      \arrow["g"', from=1-1, to=2-1]
				      \arrow["{p_2}"', from=2-1, to=2-2]
				      \arrow["{p_1}", from=1-2, to=2-2]
				      \arrow["{\exists! h}"{description}, dashed, from=2-2, to=3-3]
				      \arrow["f", from=1-1, to=1-2]
				      \arrow["u", curve={height=-6pt}, from=1-2, to=3-3]
				      \arrow["v"', curve={height=6pt}, from=2-1, to=3-3]
			      \end{tikzcd}\]
		      there exists a unique \(h\colon W\to U\) such that \(h\circ p= u\), \(h\circ p_2 = v\).
	\end{itemize}

	We say
	\[\begin{tikzcd}
			Z & X \\
			Y & W
			\arrow[from=1-1, to=2-1]
			\arrow[from=2-1, to=2-2]
			\arrow[from=1-2, to=2-2]
			\arrow[from=1-1, to=1-2]
		\end{tikzcd}\]
	is a \emph{Cocartesian} diagram. \label{def:cocartesian}
\end{definition}
\begin{exercise}
	Prove that in a \hyperref[def:category]{category} \(\mathscr{C} \), if the \hyperref[def:fibered-coproduct]{fibered coproduct} of \(f\) and \(g\) exists
	\[\begin{tikzcd}
			Z & X \\
			Y &
			\arrow["g"', from=1-1, to=2-1]
			\arrow["f", from=1-1, to=1-2]
		\end{tikzcd}\]
	then such \hyperref[def:fibered-coproduct]{fibered coproduct} is unique up to isomorphism.
\end{exercise}

\begin{remark}
	If we reverse all the directions of \hyperref[def:morphism]{morphism}, then we have so-called \emph{fibered product}.
\end{remark}

\begin{eg}
	Let's see some example.
	\begin{enumerate}
		\item Let \(\mathscr{C} = \underline{\mathrm{Top}}\), and let \(X\in \Object (\underline{\mathrm{Top} })\). Given \(X_{0}, X_1 \in X\), and \(\mathrm{int}(X_0)\cup \mathrm{int}(X_1) = X\),
		      if we have
		      \[
			      \begin{alignedat}{3}
				      i_0&\colon X_0\hookrightarrow X, \quad &&i_1\colon X_1\hookrightarrow X\\
				      j_0&\colon X_{0}\cap X_1\hookrightarrow X_0, \quad &&j_1\colon X_{0}\cap X_1\hookrightarrow X_1,
			      \end{alignedat}
		      \]
		      then
		      \[\begin{tikzcd}
				      {X_0\cap X_1} & {X_0} \\
				      {X_1} & X
				      \arrow["{j_1}"', hook', from=1-1, to=2-1]
				      \arrow["{i_1}"', hook', from=2-1, to=2-2]
				      \arrow["{i_0}", hook, from=1-2, to=2-2]
				      \arrow["{j_0}", hook, from=1-1, to=1-2]
			      \end{tikzcd}\]
		      is a \hyperref[def:cocartesian]{cocartesian} diagram.
		      \begin{proof}
			      All we need to show is that given a topological space \(Y\in \underline{\mathrm{Top}}\) and \(f\colon X_{0}\to Y \), \(g\colon X_1 \to Y\) in \(\underline{\mathrm{Top}}\),
			      we have
			      \[
				      f\circ j_0 = g\circ j_1.
			      \]
			      \[\begin{tikzcd}
					      {X_0\cap X_1} & {X_0} \\
					      {X_1} & X \\
					      && Y
					      \arrow["{j_1}"', hook', from=1-1, to=2-1]
					      \arrow["{i_1}"', hook', from=2-1, to=2-2]
					      \arrow["{i_0}", hook, from=1-2, to=2-2]
					      \arrow["{j_0}", hook, from=1-1, to=1-2]
					      \arrow["f", curve={height=-6pt}, from=1-2, to=3-3]
					      \arrow["g"', curve={height=6pt}, from=2-1, to=3-3]
					      \arrow["{\exists ! h}"{description}, dashed, from=2-2, to=3-3]
				      \end{tikzcd}\]
			      We simply define \(h\colon X\to Y\), \(x\mapsto h(x)\) such that
			      \[
				      h(x) = \begin{dcases}
					      f(x), & \text{ if } x\in X_0; \\
					      g(x), & \text{ if } x\in X_1.
				      \end{dcases}
			      \]
			      \(h\) is clearly well-defined since the diagram commutes, so if \(x\in X_0 \cap X_1\), then \(f(x) = g(x)\).
			      The only thing we need to show is that \(h\) is continuous. But this is obvious too since \(X = \mathrm{int}(X_0) \cup \mathrm{int}(X_1)\), and
			      \[
				      \at{h}{\mathrm{int}(X_0)}{} = \at{f}{\mathrm{int}(X_0)}{},\quad \at{h}{\mathrm{int}(X_1)}{} = \at{g}{\mathrm{int}(X_1)}{}.
			      \]
			      The uniqueness is trivial, hence this is indeed a \hyperref[def:cocartesian]{cocartesian} diagram.
		      \end{proof}
		\item Let \(\mathscr{C} = \underline{\mathrm{Top}_\ast}\). Given \(p\in X_0\cap X_1\), where all other data are the same with the above example, we see that
		      \[\begin{tikzcd}
				      {(X_0\cap X_1, p)} & {(X_0, p)} \\
				      {(X_1, p)} & {(X, p)}
				      \arrow["{j_1}"', hook', from=1-1, to=2-1]
				      \arrow["{i_1}"', hook', from=2-1, to=2-2]
				      \arrow["{i_0}", hook, from=1-2, to=2-2]
				      \arrow["{j_0}", hook, from=1-1, to=1-2]
			      \end{tikzcd}\]
		      is a \hyperref[def:cocartesian]{cocartesian} diagram.
		\item Let \(\mathscr{C} = \underline{\mathrm{Gp}}\). Given \(P, G, H\in \Object (\underline{\mathrm{Gp}})\), we claim that the \hyperref[def:fibered-coproduct]{fibered coproduct}
		      of \(i\) and \(j\) exists.
		      \[\begin{tikzcd}
				      P & G \\
				      H
				      \arrow["j"', from=1-1, to=2-1]
				      \arrow["i", from=1-1, to=1-2]
			      \end{tikzcd}\]

		      Consider \(G\ast H\) be the \hyperref[def:free-product]{free product} between \(G\) and \(H\), with two inclusions
		      \[
			      \iota _1\colon G\hookrightarrow G\ast H,\quad \iota _2\colon H\hookrightarrow G\ast H.
		      \]
		      \[\begin{tikzcd}
				      P & G \\
				      H & {G\ast H}
				      \arrow["j"', from=1-1, to=2-1]
				      \arrow["i", from=1-1, to=1-2]
				      \arrow["{\iota_1}", hook, from=1-2, to=2-2]
				      \arrow["{\iota_2}"', hook', from=2-1, to=2-2]
			      \end{tikzcd}\]
		      Let
		      \[
			      N\coloneqq \left< \left\{\iota _1\circ i(x)\cdot (\iota _2\circ j(x))^{-1} \mid x\in P\right\} \right> ,
		      \]
		      we define
		      \[
			      G\ast_p H = \quotient{G\ast H}{N}.
		      \]
		      \[\begin{tikzcd}
				      P & G \\
				      H & {G\ast H} \\
				      && {G\ast_p H}
				      \arrow["j"', from=1-1, to=2-1]
				      \arrow["i", from=1-1, to=1-2]
				      \arrow["{\iota_1}", hook, from=1-2, to=2-2]
				      \arrow["{\iota_2}"', hook', from=2-1, to=2-2]
				      \arrow["\tau", curve={height=-6pt}, from=1-2, to=3-3]
				      \arrow["\nu"', curve={height=6pt}, from=2-1, to=3-3]
				      \arrow["\pi"{description}, from=2-2, to=3-3]
			      \end{tikzcd}\]
		      We claim that
		      \[\begin{tikzcd}
				      P & G \\
				      H & {G\ast_p H}
				      \arrow["j"', from=1-1, to=2-1]
				      \arrow["i", from=1-1, to=1-2]
				      \arrow["{\tau}", from=1-2, to=2-2]
				      \arrow["{\nu}"', from=2-1, to=2-2]
			      \end{tikzcd}\]
		      is a \hyperref[def:cocartesian]{cocartesian} diagram in \(\underline{\mathrm{Gp}}\).
		      \begin{proof}
			      Firstly, since it's just an outer diagram from above, hence it commutes. So we only need to prove this diagram satisfies the second diagram. Given any group \(K\), for every
			      \(f\colon G\to K\), \(g\colon H\to K\) such that the following diagram commutes.
			      \[\begin{tikzcd}
					      P & G \\
					      H & {G\ast_{p}H} \\
					      && {K}
					      \arrow["j"', from=1-1, to=2-1]
					      \arrow["i", from=1-1, to=1-2]
					      \arrow["{\tau}", from=1-2, to=2-2]
					      \arrow["{\nu}"', from=2-1, to=2-2]
					      \arrow["f", curve={height=-6pt}, from=1-2, to=3-3]
					      \arrow["g"', curve={height=6pt}, from=2-1, to=3-3]
					      \arrow["h"{description}, dashed, from=2-2, to=3-3]
				      \end{tikzcd}\]
			      We want to prove that there exists a unique \(h\colon G\ast_{p} H\to K\) such that this diagram still commutes. The idea is simple, from the universal property of
			      \(G\ast H\), we see that there exists a unique \(\widetilde{h} \colon G\ast H\to K\) such that
			      \[
				      \widetilde{h} \circ \iota _1= f,\quad \widetilde{h} \circ \iota _2 = g.
			      \]
			      \[\begin{tikzcd}
					      P && G \\
					      & {G\ast H} \\
					      H && {G\ast H} \\
					      \\
					      &&&& {G\ast_p H}
					      \arrow["i", from=1-1, to=1-3]
					      \arrow["\tau", from=1-3, to=3-3]
					      \arrow["\nu"', from=3-1, to=3-3]
					      \arrow["f", curve={height=-6pt}, from=1-3, to=5-5]
					      \arrow["g"', curve={height=6pt}, from=3-1, to=5-5]
					      \arrow["h"{description}, dashed, from=3-3, to=5-5]
					      \arrow["j"', from=1-1, to=3-1]
					      \arrow["{\iota_2}"{description}, from=3-1, to=2-2]
					      \arrow["{\iota_1}"{description}, from=1-3, to=2-2]
					      \arrow["{\exists ! \widetilde{h}}"{description}, curve={height=-30pt}, from=2-2, to=5-5]
					      \arrow["\pi"{description}, from=2-2, to=3-3]
				      \end{tikzcd}\]
			      We see that we can actually factor \(\widetilde{h} \) through \(\pi \), as long as \(\ker (\widetilde{h} )\supset \ker  (\pi ) \). Now, since
			      \[
				      \ker  (\pi ) = \left< \left\{\iota _1\circ i(x)\cdot (\iota _2\circ j(x))^{-1} \mid x\in p\right\} \right> ,
			      \]
			      we see that the kernel of \(\pi\) is indeed in the kernel of \(\widetilde{h} \) since for every \(x\in P\),
			      \[
				      \widetilde{h}  \left(\iota _1\circ i(x)\cdot (\iota _2\circ j(x))^{-1} \right) = \underbrace{\widetilde{h} \circ \tau _1}_{f}\circ i(x)\cdot \underbrace{\widetilde{h} \circ \iota _2}_{g}\circ j(x^{-1} ) = 1,
			      \]
			      which implies \(\ker (\widetilde{h} )\supset \ker  (\pi ) \).
			      \[\begin{tikzcd}
					      {G\ast H} & K \\
					      {G\ast_{p} H} &
					      \arrow["\widetilde{h} "', from=1-1, to=2-1]
					      \arrow["\pi ", from=1-1, to=1-2]
				      \end{tikzcd}\]
			      We then see that there exists a unique \(h\colon G\ast_{p} H\to K\) such that the above diagram commutes.
		      \end{proof}
	\end{enumerate}
\end{eg}

\subsubsection{Free Product with Amalgamation}
After seeing the above examples, the following definition should make sense.
\begin{definition}[Free product with amalgamation]\label{def:free-product-with-amalgamation}
	If two groups \(G_{\alpha }\) and \(G_{\beta }\) have a common subgroup \(S_{\{\alpha , \beta \}}\)\footnote{In general, we don't need \(S_{\{\alpha , \beta\}}\) to be a subgroup.},
	given two inclusion maps\footnote{We don't actually need \(i_{\alpha \beta } , i_{\beta \alpha } \) to be inclusive as well.} \(i_{\alpha \beta }\colon S_{\{\alpha , \beta \}}\to G_{\alpha }\) and
	\(i_{\beta \alpha }\colon S_{\{\alpha , \beta \}}\to G_{\beta }\), the \emph{free product with amalgamation} \({}_\alpha\hspace{-0.15em}\ast_S G_{\alpha }\) is defined as
	\(\underset{\alpha }{\ast} G_{\alpha }\) modulo the normal subgroup generated by
	\[
		\left\{i_{\alpha \beta }(s_{\{\alpha , \beta \}})i_{\beta \alpha }(s_{\{\alpha, \beta\}})^{-1}  \mid s_{\{\alpha, \beta\} }\in S_{\{\alpha , \beta \}} \right\},
	\]
	Namely\footnote{i.e., \(i_{\alpha \beta }(s)\) and \(i_{\beta \alpha }(s)\) will be identified in the quotient.},
	\[
		{}_\alpha\hspace{-0.3em}\ast_S G_{\alpha } = \quotient{\underset{\alpha }{\ast}G_\alpha }{\left< i_{\alpha \beta }(s_{\{\alpha , \beta \}})i_{\beta \alpha }(s_{\{\alpha , \beta \}})^{-1}  \right> }
	\]
	and satisfies the universal property
	\[\begin{tikzcd}
			S & {G_\alpha} \\
			{G_\beta} & {G_\alpha \ast_S G_\beta} \\
			&& X
			\arrow["{i_{\alpha \beta} }", from=1-1, to=1-2]
			\arrow["{i_{\beta \alpha} }"', from=1-1, to=2-1]
			\arrow[from=1-2, to=2-2]
			\arrow[from=2-1, to=2-2]
			\arrow["{\exists !}", dashed, from=2-2, to=3-3]
			\arrow[curve={height=-12pt}, from=1-2, to=3-3]
			\arrow[curve={height=12pt}, from=2-1, to=3-3]
		\end{tikzcd}\]
\end{definition}
\begin{remark}
	We see that
	\begin{itemize}
		\item We can then write out \hyperref[def:word]{words} such as \(g_\alpha\cdot s\cdot g_\beta \) for \(s\in S\), and view \(s\) as an element of
		      \(G_{\alpha }\) or \(G_{\beta }\).
		      In fact, we can do this construction even when \(i_{\alpha }\) and \(i_{\beta }\) are not injective, though this means we are not working
		      with a subgroup.

		\item Aside, in \(\underline{\mathrm{Top} } \), the same universal property defines union
		      \[\begin{tikzcd}
				      {A\cap B} & A \\
				      B & {A\cup B} \\
				      && X
				      \arrow["{i_\alpha}", hook, from=1-1, to=1-2]
				      \arrow["{i_\beta}"', hook', from=1-1, to=2-1]
				      \arrow[from=1-2, to=2-2]
				      \arrow[from=2-1, to=2-2]
				      \arrow["{\exists !}", dashed, from=2-2, to=3-3]
				      \arrow[curve={height=-12pt}, from=1-2, to=3-3]
				      \arrow[curve={height=12pt}, from=2-1, to=3-3]
			      \end{tikzcd}\]
		      for \(A, B\) are open subsets and the inclusion of intersection.
	\end{itemize}
\end{remark}

\subsection{Seifert-Van Kampen Theorem}
With \autoref{def:free-product-with-amalgamation}, we can now see the important theorem.
\begin{theorem}[Seifert-Van Kampen Theorem]\label{thm:Seifert-Van-Kampen-Theorem}
	Given \((X, x_0)\) such that \(X = \bigcup\limits_{\alpha } A_{\alpha }\) with
	\begin{itemize}
		\item \(A_{\alpha }\) are open and \hyperref[def:path]{path}-connected and \(\forall \alpha \ x_0\in A_{\alpha }\)
		\item \(A_{\alpha }\cap A_{\beta }\) is \hyperref[def:path]{path}-connected for all \(\alpha, \beta  \).
	\end{itemize}
	Then there exists a surjective group homomorphism
	\[
		\underset{\alpha }{\ast}\colon \pi _1(A_{\alpha }, x_0)\to \pi _1(X, x_0).
	\]

	\par If we additionally have \(A_{\alpha }\cap A_{\beta}\cap A_{\gamma}\) where they are all \hyperref[def:path]{path}-connected for every \(\alpha , \beta , \gamma\), then
	\[
		\pi _1(X, x_0)\cong _{\alpha }\hspace{-0.3em}\ast_{\pi _1(A_{\alpha }\cap A_{\beta }, x_0)}{}\pi _1(A_{\alpha }, x_0)
	\]
	associated to all maps \(\pi _a(A_{\alpha }\cap A_{\beta }) \to \pi _1 (A_{\alpha }), \pi _1(A_{\beta })\) induced by inclusions of spaces. i.e., \(\pi _1(X, x_0)\)
	is a quotient of the \hyperref[def:free-product-with-amalgamation]{free product} \(\ast_{\alpha}\pi _1(A_{\alpha })\) where we have
	\[
		(i_{\alpha \beta })_\ast\colon \pi _1(A_{\alpha }\cap A_{\beta })\to \pi _1(A_\alpha )
	\]
	which is induced by the inclusion \(i_{\alpha \beta }\colon A_{\alpha}\cap A_{\beta }\to A_{\alpha }\). We then take the quotient by the normal subgroup generated by
	\[
		\left\{(i_{\alpha \beta })_{\ast}(\gamma)(i_{\beta \alpha })_{\ast} \mid \gamma\in \pi _1(A_{\alpha }\cap A_{\beta })\right\}.
	\]
\end{theorem}
We'll defer the \hyperref[pf:thm:Seifert-Van-Kampen-Theorem]{proof} of \autoref{thm:Seifert-Van-Kampen-Theorem} until we get familiar with this theorem.
\begin{eg}
	We first see a great visualization of the \autoref{thm:Seifert-Van-Kampen-Theorem}.
	\begin{figure}[H]
		\centering
		\incfig{eg:Seifert-Van-Kampen-Theorem}
		\label{fig:eg:Seifert-Van-Kampen-Theorem}
	\end{figure}
	Intuitively we see the \hyperref[def:fundamental-group]{fundamental group} of \(X\), which is built by gluing \(A\) and \(B\) along their intersection.
	As the \hyperref[def:fundamental-group]{fundamental group} of \(A\) and \(B\) glued along the \hyperref[def:fundamental-group]{fundamental group} of their
	intersection. In essence, \(\pi _1 (X, x_0)\) is the quotient of \(\pi _1(A)\ast \pi _1(B)\) by relations to impose the condition that loops like \(\gamma \)
	lying in \(A\cap B\) can be viewed as elements of either \(\pi _1(A)\) or \(\pi _1(B)\).
\end{eg}

\begin{remark}
	We can use a more abstract way to describe \autoref{thm:Seifert-Van-Kampen-Theorem}. Specifically, in the case that \(n = 2\), i.e., \(X = \bigcup\limits_{i=1}^{2} A_{i} \),
	we let \(A_{i} \eqqcolon X_{i} \), then we have the following. The \hyperref[def:functor]{functor} \(\pi _1\colon \underline{\mathrm{Top}_{\ast}}\to \underline{\mathrm{Gp}}\)
	maps the \hyperref[def:cocartesian]{cocartesian} diagram in \(\underline{\mathrm{Top} _\ast}\) to a \hyperref[def:cocartesian]{cocartesian} diagram in \(\underline{\mathrm{Gp} }\)
	as follows.
	\[\begin{tikzcd}
			{(X_0\cap X_1, x_0)} & {(X_0, x_0)} \\
			{(X_1, x_0)} & {(X, x_0)}
			\arrow["{j_0}", from=1-1, to=1-2]
			\arrow["{i_0}", from=1-2, to=2-2]
			\arrow["{i_1}"', from=2-1, to=2-2]
			\arrow["{j_1}"', from=1-1, to=2-1]
		\end{tikzcd}\overset{\pi _1}{\longmapsto}
		\begin{tikzcd}
			{\pi_1(X_0\cap X_1, x_0)} & {\pi_1(X_0, x_0)} \\
			{\pi_1(X_1, x_0)} & {\pi_1(X, x_0)}
			\arrow["{(j_0)_\ast}", from=1-1, to=1-2]
			\arrow["{(i_0)_\ast}", from=1-2, to=2-2]
			\arrow["{(i_1)_\ast}"', from=2-1, to=2-2]
			\arrow["{(j_1)_\ast}"', from=1-1, to=2-1]
		\end{tikzcd} \]
	Then, simply from the property of \hyperref[def:cocartesian]{cocartesian} diagram, we see that
	\[
		\pi _1(X, x_0)\cong \pi _1(X_0, x_0)\ast_{\pi _1(X_0 \cap X_1, x_0)}\pi _1(X_1, x_0).
	\]
\end{remark}

\hr
Additionally, there is a more general version of \autoref{thm:Seifert-Van-Kampen-Theorem}, which is defined on \hyperref[def:groupoid]{groupoid}. The theorem is
stated in \autoref{thm:Seifert-Van-Kampen-Theorem-on-groupoid} with the proof.

With this more general version and the proof of which, we can apply it to \autoref{thm:Seifert-Van-Kampen-Theorem}. But one question is that, the above proof
works in \(\underline{\mathrm{Gpd}}\) rather than in \(\underline{\mathrm{Gp}}\). We now see how to generalize a group to a \hyperref[def:groupoid]{groupoid}.

For any group \(G\), we can define a \hyperref[def:groupoid]{groupoid}, denoted as \(G\) also, as follows.
\begin{itemize}
	\item \(\Object (G) = \{\mathrm{pt} \}\), a one point set.
	\item \(\Homomorphism (G) = \{g\in G\}\).
	\item Composition: We define
	      \[
		      g\circ h \coloneqq h\cdot g.
	      \]
\end{itemize}
We see that the associativity of group elements implies the associativity of composition defined above, and since there is an identity element in \(G\), hence we also have
an identity \hyperref[def:morphism]{morphism}, these two facts ensure that \(G\) is an \hyperref[def:category]{category}.

Furthermore, since for every \(g\in G\), there is a \(g^{-1} \in G\), hence every \hyperref[def:morphism]{morphism} is an isomorphism, which implies \(G\) is a \hyperref[def:groupoid]{groupoid}.

With this, we see that we can view the following diagram in the \hyperref[def:category]{category} of \hyperref[def:groupoid]{groupoid} \(\underline{\mathrm{Gpd}}\).
\[
	\begin{tikzcd}
		{\pi_1(X_0\cap X_1, x_0)} & {\pi_1(X_0, x_0)} \\
		{\pi_1(X_1, x_0)} & {\pi_1(X, x_0)}
		\arrow["{(j_0)_\ast}", from=1-1, to=1-2]
		\arrow["{(i_0)_\ast}", from=1-2, to=2-2]
		\arrow["{(i_1)_\ast}"', from=2-1, to=2-2]
		\arrow["{(j_1)_\ast}"', from=1-1, to=2-1]
	\end{tikzcd}
\]

And to prove \autoref{thm:Seifert-Van-Kampen-Theorem}, we only need to show this diagram is \hyperref[def:cocartesian]{cocartesian}.
This version of proof is given in \autoref{pf:an-alternative-proof-of-Seifert-Van-Kampen-thm}.