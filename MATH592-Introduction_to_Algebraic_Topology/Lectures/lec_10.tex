\lecture{10}{26 Jan. 10:00}{Seifert-Van Kampen Theorem}
The goal is to compute \(\pi _1(X)\) where \(X = A\cup B\) using the data
\[
	\pi _1(A), \pi _1(B), \pi _1(A\cap B).
\]

We first introduce a definition.
\begin{definition}[Free product with amalgamation]\label{def:free-product-with-amalgamation}
	Given some collections of groups \(\{G_{\alpha }\}_{\alpha }\), the \emph{free product},
	denoted by \(\underset{\alpha}{\ast}G_{\alpha }\) is a group such that
	\begin{itemize}
		\item Elements: Words in \(\{g\colon g\in G_{\alpha } \text{ for any \(\alpha\)}\}\) modulo by the equivalence relation
		      generated by
		      \[
			      wg_{i}g_{j}v\sim w(g_{i}g_{j})v
		      \]
		      when both \(g_{i}, g_{j}\in G_{\alpha }\). Also, for the identity element \(\identity_{}= e_{\alpha }\in G_{\alpha }\)
		      for any \(\alpha \) such that
		      \[
			      w e_{\alpha }v\sim wv.
		      \]
		\item Operation: Concatenation of words.
	\end{itemize}

	\par Furthermore, if two groups \(G_{\alpha }\) and \(G_{\beta }\) have a common subgroup \(S_{\{\alpha , \beta \}}\)\footnote{In general, we don't need \(S_{\{\alpha , \beta\}}\) to be a subgroup.},
	given two inclusion maps\footnote{We don't actually need \(i_{\alpha \beta } , i_{\beta \alpha } \) to be inclusive as well.} \(i_{\alpha \beta }\colon S_{\{\alpha , \beta \}}\to G_{\alpha }\) and
	\(i_{\beta \alpha }\colon S_{\{\alpha , \beta \}}\to G_{\beta }\), the \emph{free product with amalgamation} \({}_\alpha\hspace{-0.15em}\ast_S G_{\alpha }\) is defined as
	\(\underset{\alpha }{\ast} G_{\alpha }\) modulo the normal subgroup generated by
	\[
		\left\{i_{\alpha \beta }(s_{\{\alpha , \beta \}})i_{\beta \alpha }(s_{\{\alpha, \beta\}})^{-1}  \mid s_{\{\alpha, \beta\} }\in S_{\{\alpha , \beta \}} \right\},
	\]
	Namely\footnote{i.e., \(i_{\alpha \beta }(s)\) and \(i_{\beta \alpha }(s)\) will be identified in the quotient.},
	\[
		{}_\alpha\hspace{-0.3em}\ast_S G_{\alpha } = \quotient{\underset{\alpha }{\ast}G_\alpha }{\left< i_{\alpha \beta }(s_{\{\alpha , \beta \}})i_{\beta \alpha }(s_{\{\alpha , \beta \}})^{-1}  \right> }
	\]
	and satisfies the universal property
	\[\begin{tikzcd}
			S & {G_\alpha} \\
			{G_\beta} & {G_\alpha \ast_S G_\beta} \\
			&& X
			\arrow["{i_{\alpha \beta} }", from=1-1, to=1-2]
			\arrow["{i_{\beta \alpha} }"', from=1-1, to=2-1]
			\arrow[from=1-2, to=2-2]
			\arrow[from=2-1, to=2-2]
			\arrow["{\exists !}", dashed, from=2-2, to=3-3]
			\arrow[curve={height=-12pt}, from=1-2, to=3-3]
			\arrow[curve={height=12pt}, from=2-1, to=3-3]
		\end{tikzcd}\]
\end{definition}

\begin{remark}
	We see that
	\begin{itemize}
		\item We can then write out words such as \(g_1 g_2 s g_3\) for \(s\in S\), and view \(s\) as an element of \(G_{\alpha }\) or \(G_{\beta }\).
		      In fact, we can do this construction even when \(i_{\alpha }\) and \(i_{\beta }\) are not injective, though this means we are not working
		      with a subgroup.

		\item Aside, in \(\underline{\mathrm{Top} } \), the same universal property defines union
		      \[\begin{tikzcd}
				      {A\cap B} & A \\
				      B & {A\cup B} \\
				      && X
				      \arrow["{i_\alpha}", from=1-1, to=1-2]
				      \arrow["{i_\beta}"', from=1-1, to=2-1]
				      \arrow[from=1-2, to=2-2]
				      \arrow[from=2-1, to=2-2]
				      \arrow["{\exists !}", dashed, from=2-2, to=3-3]
				      \arrow[curve={height=-12pt}, from=1-2, to=3-3]
				      \arrow[curve={height=12pt}, from=2-1, to=3-3]
			      \end{tikzcd}\]
		      for \(A, B\) are open subsets and the inclusion of intersection.
	\end{itemize}
\end{remark}

\begin{theorem}[Seifert-Van Kampen Theorem]\label{thm:Seifert-Van-Kampen-Theorem}
	Given \((X, x_0)\) such that \(X = \bigcup\limits_{\alpha } A_{\alpha }\) with
	\begin{itemize}
		\item \(A_{\alpha }\) are open and path-connected and \(\forall \alpha \ x_0\in A_{\alpha }\)
		\item \(A_{\alpha }\cap A_{\beta }\) is path-connected for all \(\alpha, \beta  \).
	\end{itemize}
	Then there exists a surjective group homomorphism
	\[
		\underset{\alpha }{\ast}\colon \pi _1(A_{\alpha }, x_0)\to \pi _1(X, x_0).
	\]

	\par If we additionally have \(A_{\alpha }\cap A_{\beta}\cap A_{\gamma}\) where they are all path-connected for every \(\alpha , \beta , \gamma\), then
	\[
		\pi _1(X, x_0)\cong _{\alpha }\hspace{-0.3em}\ast_{\pi _1(A_{\alpha }\cap A_{\beta }, x_0)}{}\pi _1(A_{\alpha }, x_0)
	\]
	associated to all maps \(\pi _a(A_{\alpha }\cap A_{\beta }) \to \pi _1 (A_{\alpha }), \pi _1(A_{\beta })\) induced by inclusions of spaces. i.e., \(\pi _1(X, x_0)\)
	is a quotient of the free product \(\ast_{\alpha}\pi _1(A_{\alpha })\) where we have
	\[
		(i_{\alpha \beta })_\ast\colon \pi _1(A_{\alpha }\cap A_{\beta })\to \pi _1(A+\alpha )
	\]
	which is induced by the inclusion \(i_{\alpha \beta }\colon A_{\alpha}\cap A_{\beta }\to A_{\alpha }\). We then take the quotient by the normal subgroup generated by
	\[
		\left\{(i_{\alpha \beta })_{\ast}(\gamma)(i_{\beta \alpha })_{\ast} \mid \gamma\in \pi _1(A_{\alpha }\cap A_{\beta })\right\}.
	\]
\end{theorem}

\begin{eg}
	We first see a great visualization of the \autoref{thm:Seifert-Van-Kampen-Theorem}.
	\begin{figure}[H]
		\centering
		\incfig{eg:Seifert-Van-Kampen-Theorem}
		\label{fig:eg:Seifert-Van-Kampen-Theorem}
	\end{figure}
	Intuitively we see the fundamental group of \(X\), which is built by gluing \(A\) and \(B\) along their intersection. As the fundamental group
	of \(A\) and \(B\) glued along the fundamental group of their intersection. In essence, \(\pi _1 (X, x_0)\) is the quotient of \(\pi _1(A)\ast \pi _1(B)\)
	by relations to impose the condition that loops like \(\gamma \) lying in \(A\cap B\) can be viewed as elements of either \(\pi _1(A)\) or \(\pi _1(B)\).
\end{eg}