\lecture{10}{26 Jan. 10:00}{Seifert-Van Kampen Theorem}
The goal is to compute \(\pi _1(X)\) where \(X = A\cup B\) using the data
\[
	\pi _1(A), \pi _1(B), \pi _1(A\cap B).
\]

\subsection{Seifert-Van Kampen Theorem}
\subsubsection{Free Product with Amalgamation}
We first introduce a definition.
\begin{definition}[Free product]\label{def:free-product}
	Given some collections of groups \(\{G_{\alpha }\}_{\alpha }\), the \emph{free product},
	denoted by \(\underset{\alpha}{\ast}G_{\alpha }\) is a group such that
	\begin{itemize}
		\item Elements: \hyperref[def:word]{Words} in \(\{g\colon g\in G_{\alpha } \text{ for any \(\alpha\)}\}\) modulo by the equivalence relation
		      generated by
		      \[
			      wg_{i}g_{j}v\sim w(g_{i}g_{j})v
		      \]
		      when both \(g_{i}, g_{j}\in G_{\alpha }\). Also, for the identity element \(\identity_{}= e_{\alpha }\in G_{\alpha }\)
		      for any \(\alpha \) such that
		      \[
			      w e_{\alpha }v\sim wv.
		      \]
		      Specifically,
		      \[
			      \ast_\alpha G_\alpha \coloneqq \quotient{\left\{\text{\hyperref[def:word]{words} in \(\{G_\alpha \}_{\alpha}\)} \right\}}{\sim}.
		      \]
		\item Operation: Concatenation of \hyperref[def:word]{words}.
	\end{itemize}
\end{definition}
\begin{remark}
	In particular, we have the following universal property of \(\ast_\alpha G_\alpha \). For every \(\alpha \), there is a \(\iota _\alpha \) such that
	\[
		\iota _\alpha \colon G_\alpha \to \ast_\alpha G_\alpha ,\qquad g\mapsto \overline{g},
	\]
	where \(\iota _\alpha \) is a group homomorphism obviously. Further, \((\ast_\alpha G_\alpha ,\iota _\alpha )\) satisfies the following property: For every
	group \(H\) and a group homomorphism \(\varphi _\alpha \colon G_\alpha \to G\) for all \(\alpha \), there exists an unique group homomorphism \(\varphi \colon \ast_\alpha G_\alpha \to H\)
	such that \(\varphi \circ \iota _\alpha = \varphi _\alpha \), i.e., the following diagram commutes.
	\[\begin{tikzcd}
			{G_{\alpha_1}} \\
			{G_{\alpha_2}} && {\ast_\alpha G_\alpha} \\
			\vdots \\
			&&& H
			\arrow["{\iota_{\alpha_1}}", from=1-1, to=2-3]
			\arrow["{\iota_{\alpha_2}}"', from=2-1, to=2-3]
			\arrow["{\exists ! \varphi}", dashed, from=2-3, to=4-4]
			\arrow["{\varphi_{\alpha_1}}"{description}, from=1-1, to=4-4]
			\arrow["{\varphi_{\alpha_2}}"{description}, from=2-1, to=4-4]
		\end{tikzcd}\]
	\begin{proof}
		The proof is straightforward. Firstly, we define \(w = \overline{g_1 g_2\ldots g_n}\in \ast_\alpha G_\alpha\), \(g_{i} \in G_{\alpha_{i} } \),
		\[
			\varphi (w)\coloneqq \varphi _{\alpha _1}(g_1)\ldots \varphi _{\alpha _{n} }(g_{n} ).
		\]
		Now, we just need to check
		\begin{itemize}
			\item It's well-defined, since \(\varphi_\alpha \) is a group homomorphism.
			\item \(\varphi \) is a group homomorphism.
			\item \(\varphi \circ \iota _\alpha =\varphi _\alpha \).
			\item Such \(\varphi \) is unique. Suppose there exists another \(\psi \colon \ast_\alpha G_\alpha \to H\), then
			      \[
				      \psi \circ \iota _\alpha = \varphi _\alpha \implies \underset{g\in G_\alpha }{\forall }\ \psi (\overline{g} )= \psi _\alpha (g),
			      \]
			      But then for every \(w = \overline{g_1 g_2\ldots g_n}\in \ast_\alpha G_\alpha\), \(g_{i} \in G_{\alpha_{i} } \), we have
			      \[
				      \psi (w) = \psi (\overline{g_1}\ldots \overline{g_n}) = \psi (\overline{g_1}) \ldots \psi (\overline{g_n})  = \psi _{\alpha _1}(\overline{g_1}) \ldots \psi _{\alpha_n}(\overline{g_n}),
			      \]
			      which is just \(\varphi \).
		\end{itemize}
	\end{proof}
\end{remark}
\begin{remark}
	We further claim that this universal property determines such \hyperref[def:free-product]{free product} uniquely. i.e., assume there are another group \(\widetilde{G} \) and
	\(\widetilde{\iota} _\alpha \colon G_\alpha \to \widetilde{G}\). Assume \((\widetilde{G} , \widetilde{\iota} _\alpha  )\) also satisfies the following property: For every
	group \(H\) and group homomorphism \(\varphi _\alpha \colon G_\alpha \to H\), then there exists a unique group homomorphism \(\varphi \colon \widetilde{G} \to H\) such that
	the following diagram commutes.
	\[\begin{tikzcd}
			{G_{\alpha_1}} \\
			{G_{\alpha_2}} && {\ast_\alpha \widetilde{G} _\alpha} \\
			\vdots \\
			&&& H
			\arrow["{\widetilde{\iota}_{\alpha_1}}", from=1-1, to=2-3]
			\arrow["{\widetilde{\iota}_{\alpha_2}}"', from=2-1, to=2-3]
			\arrow["{\exists ! \varphi}", dashed, from=2-3, to=4-4]
			\arrow["{\varphi_{\alpha_1}}"{description}, from=1-1, to=4-4]
			\arrow["{\varphi_{\alpha_2}}"{description}, from=2-1, to=4-4]
		\end{tikzcd}\]
	Then, \(\widetilde{G} \cong \ast_\alpha G_\alpha \).
	\begin{proof}
		Assume \((\widetilde{G} , \widetilde{\iota} _\alpha  )\) satisfies the universal property mentioned above. Then from the universal property and viewing \(\widetilde{G} \) and
		\(\ast_\alpha G_\alpha \) as \(H\) separately, we obtain the following diagram.
		\[\begin{tikzcd}
				{\widetilde{G}} &&&&&& {\ast_\alpha G_\alpha} \\
				&&& {G_{\alpha_1}} \\
				&&& {G_{\alpha_2}} \\
				&&& \vdots
				\arrow["{\iota_{\alpha_1}}", from=2-4, to=1-7]
				\arrow["{\iota_{\alpha_2}}", from=3-4, to=1-7]
				\arrow["{\exists ! f}"', shift right=1, dashed, from=1-7, to=1-1]
				\arrow["{\widetilde{\iota}_{\alpha_1}}"', from=2-4, to=1-1]
				\arrow["{\widetilde{\iota}_{\alpha_2}}"', from=3-4, to=1-1]
				\arrow[from=4-4, to=1-1]
				\arrow[from=4-4, to=1-7]
				\arrow["{\exists! g}"', shift right=1, dashed, from=1-1, to=1-7]
			\end{tikzcd}\]
		We claim that
		\[
			g\circ f = \identity_{},\quad f\circ g = \identity_{}.
		\]
		To see this, we simply apply the same observation, for example,
		\[\begin{tikzcd}
				{G_{\alpha_1}} \\
				{G_{\alpha_2}} && {\widetilde{G}} \\
				\vdots \\
				&&& {\widetilde{G}}
				\arrow["{\widetilde{\iota}_{\alpha_1}}", from=1-1, to=2-3]
				\arrow["{\widetilde{\iota}_{\alpha_2}}"', from=2-1, to=2-3]
				\arrow["{\exists ! g\circ f}", dashed, from=2-3, to=4-4]
				\arrow["{\widetilde{\iota}_{\alpha_1}}"{description}, from=1-1, to=4-4]
				\arrow["{\widetilde{\iota}_{\alpha_2}}"{description}, from=2-1, to=4-4]
			\end{tikzcd}\]
		where \(g\circ f\) comes from the previous diagram. But notice that \(\identity_{} \) let the diagram commutes also, and since it's unique, hence \(g\circ f = \identity_{} \). Similarily,
		we have \(f\circ g = \identity_{} \).
	\end{proof}
\end{remark}

If you're careful enough, you may find out that all we're doing is just writing out a specific example of \autoref{lma:lec7}! Indeed, this is exactly the construction of a
\hyperref[def:free-group]{free group}.

\begin{definition}[Free product with amalgamation]\label{def:free-product-with-amalgamation}
	If two groups \(G_{\alpha }\) and \(G_{\beta }\) have a common subgroup \(S_{\{\alpha , \beta \}}\)\footnote{In general, we don't need \(S_{\{\alpha , \beta\}}\) to be a subgroup.},
	given two inclusion maps\footnote{We don't actually need \(i_{\alpha \beta } , i_{\beta \alpha } \) to be inclusive as well.} \(i_{\alpha \beta }\colon S_{\{\alpha , \beta \}}\to G_{\alpha }\) and
	\(i_{\beta \alpha }\colon S_{\{\alpha , \beta \}}\to G_{\beta }\), the \emph{free product with amalgamation} \({}_\alpha\hspace{-0.15em}\ast_S G_{\alpha }\) is defined as
	\(\underset{\alpha }{\ast} G_{\alpha }\) modulo the normal subgroup generated by
	\[
		\left\{i_{\alpha \beta }(s_{\{\alpha , \beta \}})i_{\beta \alpha }(s_{\{\alpha, \beta\}})^{-1}  \mid s_{\{\alpha, \beta\} }\in S_{\{\alpha , \beta \}} \right\},
	\]
	Namely\footnote{i.e., \(i_{\alpha \beta }(s)\) and \(i_{\beta \alpha }(s)\) will be identified in the quotient.},
	\[
		{}_\alpha\hspace{-0.3em}\ast_S G_{\alpha } = \quotient{\underset{\alpha }{\ast}G_\alpha }{\left< i_{\alpha \beta }(s_{\{\alpha , \beta \}})i_{\beta \alpha }(s_{\{\alpha , \beta \}})^{-1}  \right> }
	\]
	and satisfies the universal property
	\[\begin{tikzcd}
			S & {G_\alpha} \\
			{G_\beta} & {G_\alpha \ast_S G_\beta} \\
			&& X
			\arrow["{i_{\alpha \beta} }", from=1-1, to=1-2]
			\arrow["{i_{\beta \alpha} }"', from=1-1, to=2-1]
			\arrow[from=1-2, to=2-2]
			\arrow[from=2-1, to=2-2]
			\arrow["{\exists !}", dashed, from=2-2, to=3-3]
			\arrow[curve={height=-12pt}, from=1-2, to=3-3]
			\arrow[curve={height=12pt}, from=2-1, to=3-3]
		\end{tikzcd}\]
\end{definition}
\begin{remark}
	We see that
	\begin{itemize}
		\item We can then write out \hyperref[def:word]{words} such as \(g_\alpha\cdot s\cdot g_\beta \) for \(s\in S\), and view \(s\) as an element of
		      \(G_{\alpha }\) or \(G_{\beta }\).
		      In fact, we can do this construction even when \(i_{\alpha }\) and \(i_{\beta }\) are not injective, though this means we are not working
		      with a subgroup.

		\item Aside, in \(\underline{\mathrm{Top} } \), the same universal property defines union
		      \[\begin{tikzcd}
				      {A\cap B} & A \\
				      B & {A\cup B} \\
				      && X
				      \arrow["{i_\alpha}", from=1-1, to=1-2]
				      \arrow["{i_\beta}"', from=1-1, to=2-1]
				      \arrow[from=1-2, to=2-2]
				      \arrow[from=2-1, to=2-2]
				      \arrow["{\exists !}", dashed, from=2-2, to=3-3]
				      \arrow[curve={height=-12pt}, from=1-2, to=3-3]
				      \arrow[curve={height=12pt}, from=2-1, to=3-3]
			      \end{tikzcd}\]
		      for \(A, B\) are open subsets and the inclusion of intersection.
	\end{itemize}
\end{remark}

\subsubsection{Seifert-Van Kampen Theorem}
With \autoref{def:free-product-with-amalgamation}, we can now see the important theorem.
\begin{theorem}[Seifert-Van Kampen Theorem]\label{thm:Seifert-Van-Kampen-Theorem}
	Given \((X, x_0)\) such that \(X = \bigcup\limits_{\alpha } A_{\alpha }\) with
	\begin{itemize}
		\item \(A_{\alpha }\) are open and \hyperref[def:path]{path}-connected and \(\forall \alpha \ x_0\in A_{\alpha }\)
		\item \(A_{\alpha }\cap A_{\beta }\) is \hyperref[def:path]{path}-connected for all \(\alpha, \beta  \).
	\end{itemize}
	Then there exists a surjective group homomorphism
	\[
		\underset{\alpha }{\ast}\colon \pi _1(A_{\alpha }, x_0)\to \pi _1(X, x_0).
	\]

	\par If we additionally have \(A_{\alpha }\cap A_{\beta}\cap A_{\gamma}\) where they are all \hyperref[def:path]{path}-connected for every \(\alpha , \beta , \gamma\), then
	\[
		\pi _1(X, x_0)\cong _{\alpha }\hspace{-0.3em}\ast_{\pi _1(A_{\alpha }\cap A_{\beta }, x_0)}{}\pi _1(A_{\alpha }, x_0)
	\]
	associated to all maps \(\pi _a(A_{\alpha }\cap A_{\beta }) \to \pi _1 (A_{\alpha }), \pi _1(A_{\beta })\) induced by inclusions of spaces. i.e., \(\pi _1(X, x_0)\)
	is a quotient of the \hyperref[def:free-product-with-amalgamation]{free product} \(\ast_{\alpha}\pi _1(A_{\alpha })\) where we have
	\[
		(i_{\alpha \beta })_\ast\colon \pi _1(A_{\alpha }\cap A_{\beta })\to \pi _1(A_\alpha )
	\]
	which is induced by the inclusion \(i_{\alpha \beta }\colon A_{\alpha}\cap A_{\beta }\to A_{\alpha }\). We then take the quotient by the normal subgroup generated by
	\[
		\left\{(i_{\alpha \beta })_{\ast}(\gamma)(i_{\beta \alpha })_{\ast} \mid \gamma\in \pi _1(A_{\alpha }\cap A_{\beta })\right\}.
	\]
\end{theorem}
We'll defer the proof of \autoref{thm:Seifert-Van-Kampen-Theorem} until we get familiar with this theorem.\footnote{The proof can be found in \autoref{pf:thm:Seifert-Van-Kampen-Theorem}.}
\begin{eg}
	We first see a great visualization of the \autoref{thm:Seifert-Van-Kampen-Theorem}.
	\begin{figure}[H]
		\centering
		\incfig{eg:Seifert-Van-Kampen-Theorem}
		\label{fig:eg:Seifert-Van-Kampen-Theorem}
	\end{figure}
	Intuitively we see the \hyperref[def:fundamental-group]{fundamental group} of \(X\), which is built by gluing \(A\) and \(B\) along their intersection.
	As the \hyperref[def:fundamental-group]{fundamental group} of \(A\) and \(B\) glued along the \hyperref[def:fundamental-group]{fundamental group} of their
	intersection. In essence, \(\pi _1 (X, x_0)\) is the quotient of \(\pi _1(A)\ast \pi _1(B)\) by relations to impose the condition that loops like \(\gamma \)
	lying in \(A\cap B\) can be viewed as elements of either \(\pi _1(A)\) or \(\pi _1(B)\).
\end{eg}