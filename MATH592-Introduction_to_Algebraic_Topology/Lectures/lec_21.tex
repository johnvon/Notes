\lecture{21}{23 Feb. 10:00}{Simplicial Homology}
To demonstrate how the definition of homology arise, we first see the idea behind it. Fix a space \(X\) which equips with the \hyperref[def:delta-complex]{\(\Delta\)-complex}
structure. Then, we define \(C_{n} (X)\) to be the \hyperref[def:free-Abelian-group]{free Abelian group} on the \hyperref[def:standard-simplex]{\(n\)-simplices} of \(X\). That is,
\[
	C_{n} (X) = \left\{\text{finite sums } \sum\limits_{}^{} m_\alpha \sigma _\alpha \mid m_\alpha \in \mathbb{\MakeUppercase{z}} , \sigma _\alpha \colon \Delta ^n\to X\right\}.
\]
\begin{figure}[H]
	\centering
	\incfig{C2-lec21}
	\caption{\(C_2(X) = \mathbb{\MakeUppercase{z}} \sigma _1 \oplus \mathbb{\MakeUppercase{z}} \sigma _2\oplus \mathbb{\MakeUppercase{z}} \sigma _3\).}
	\label{fig:C2-lec21}
\end{figure}
Then, the \(n\)-th homology group will be a subquotient of \(C_{n} (X)\), where the heuristic/imprecise idea is
\begin{itemize}
	\item Take subgroup of \(C_{n}\) of \emph{cycles}. These are sums of \hyperref[def:standard-simplex]{simplices} satisfying a combinatorial condition
	      on the boundary gluing maps to ensure that they \emph{close up}, i.e., they have no \hyperref[def:boundary]{boundary}.
	      \begin{figure}[H]
		      \centering
		      \incfig{lec21-demo-of-homology-group}
		      \label{fig:lec21-demo-of-homology-group}
	      \end{figure}
	\item To take the quotient, we consider two cycles to be equivalent if their difference is a \hyperref[def:boundary]{boundary}. For example, in
	      the case of torus, \(a\) is homologous to \(b\) since \(a - b\) is the \hyperref[def:boundary]{boundary} of the shaded subsurface \(S\) on
	      of the torus below.
	      \begin{figure}[H]
		      \centering
		      \incfig{lec21-torus}
		      \label{fig:lec21-torus}
	      \end{figure}
	      In fact, \(a\) and \(b\) are \hyperref[def:homotopic]{homotopic} (which will imply they're homologous essentially), but two loops do not
	      need to be \hyperref[def:homotopic]{homotopic} to be homologous. For example, in the figure below, \(a+b\) is homologous to \(c\), since
	      \(a+b-c\) is the \hyperref[def:boundary]{boundary} of \(S\) (\(a+b\)\footnote{Which isn't even a loop} and \(c\) are \underline{not} \hyperref[def:homotopic]{homotopic}).
	      \begin{figure}[H]
		      \centering
		      \incfig{lec21-3-torus}
		      \label{fig:lec21-3-torus}
	      \end{figure}
\end{itemize}

Let's now see the formal definition.
\begin{definition}[Simplicial chain group]\label{def:simplicial-chain-group}
	We define the \emph{simplicial chain group} \(C_{n} (X)\) of order \(n\) to be the \hyperref[def:free-Abelian-group]{free Abelian group} on the \hyperref[def:standard-simplex]{\(n\)-simplices} of \(X\) such
	that
	\[
		C_{n} (X) \coloneqq \left\{\text{finite sums } \sum\limits_{}^{} m_\alpha \sigma _\alpha \mid m_\alpha \in \mathbb{\MakeUppercase{z}} , \sigma _\alpha \colon \Delta ^n \to X\right\}.
	\]
\end{definition}

\begin{definition}[Cycles]\label{def:cycle}
	Given any chain group \(C_{n} (X)\), a \emph{cycle} of \(C_{n} (X)\) is those chains \(\sum m_\alpha \sigma _\alpha\)
	with no \hyperref[def:boundary]{boundaries}.
\end{definition}

\begin{definition}[Boundary homomorphism]\label{def:boundary-homomorphism}
	A map \(\partial_{n} \colon C_{n} (X)\to C_{n-1}(X) \) is called a \emph{boundary homomorphism} such that
	\[
		\begin{split}
			\partial_{n} \colon C_{n} (X)&\to C_{n-1}(X)\\
			[\sigma _\alpha ]&\mapsto \sum\limits_{i=1}^{n} (-1)^i \at{\sigma _\alpha}{[v_0, \ldots , \hat{v}_{i}, \ldots , v_{n}]}{},
		\end{split}
	\]
	which defines the map on the basis, and we extend it linearly.
\end{definition}
\begin{remark}
	We see that the definition of \hyperref[def:boundary-homomorphism]{boundary homomorphism} indeed coincides with the definition of
	\hyperref[def:boundary]{boundary} when considering either \hyperref[def:delta-complex]{\(\Delta\)-complex} or
	\hyperref[def:simplicial-complex]{simplicial complex} structure.
\end{remark}

\begin{eg}
	We give some lower dimensions examples of \autoref{def:boundary-homomorphism} to motivate the general definition.
	\begin{itemize}
		\item For \(n=1\), \(\partial_1\colon C_1(X)\to C_0(X)\) such that
		      \[
			      [\sigma _\alpha \colon [v_0, v_1]\to X]\mapsto \at{\sigma _\alpha }{[v_1]}{} - \at{\sigma _\alpha }{[v_0]}{}.
		      \]
		\item For \(n=2\), \(\partial_2\colon C_2(X)\to C_1(X)\) such that
		      \[
			      [\sigma _\alpha \colon [v_0, v_1, v_2]\to X]\mapsto \at{\sigma _\alpha }{[v_1,v_2]}{} - \at{\sigma _\alpha }{[v_0, v_2]}{} + \at{\sigma _\alpha }{[v_0, v_1]}{}.
		      \]
	\end{itemize}
\end{eg}

\begin{lemma}\label{lma:lec21}
	For any \(n\geq 2\), we have
	\[
		\begin{tikzcd}
			{C_n(X)} & {C_{n-1}(X)} & {C_{n-2}(X)}
			\arrow["{\partial_{n-1}}", from=1-2, to=1-3]
			\arrow["{\partial_{n}}", from=1-1, to=1-2]
			\arrow["{\partial_{n-1}\circ \partial_{n}=0}"', curve={height=12pt}, from=1-1, to=1-3]
		\end{tikzcd}
	\]
\end{lemma}
\begin{proof}
	Since all \(C_{i} \) are \hyperref[def:free-Abelian-group]{free Abelian group}, hence we only need to consider \(\partial _{n-1}\circ \partial _{n} (\sigma ) = 0\) for a generator \(\sigma \).
	Given a generator \(\sigma \), the result follows from directly applying the \hyperref[def:boundary-homomorphism]{definition} and with some calculation.
\end{proof}

\begin{definition}[Chain complex]\label{def:chain-complex}
	A \emph{chain complex} \((C_\ast, d_\ast)\) is a collection of maps such that
	\[
		\begin{tikzcd}
			\ldots & {C_{n+1}} & {C_{n}} & {C_{n-1}} & \ldots
			\arrow[from=1-1, to=1-2]
			\arrow["{d_{n+1}}", from=1-2, to=1-3]
			\arrow["{d_{n}}", from=1-3, to=1-4]
			\arrow["{d_{n-1}}", from=1-4, to=1-5]
		\end{tikzcd}
	\]
	of \hyperref[def:Abelian-group]{Abelian groups} and group homomorphism such that
	\[
		d_{n-1}\circ d_n = 0.
	\]
	We call \(C_{n} \) the \emph{\(n\)-th chain group} and \(d_{n} \) the \emph{\(n\)-th differential}.
\end{definition}
\begin{note}
	Note that \autoref{def:chain-complex} is purely \emph{abstract}, namely we can put different chain group
	structure on \(C_{n} \). We'll see what this means later.\footnote{Spoiler: It just means we can give different definition about the map \(\sigma\).}
	But for now, \(C_{n} \) can be equipped with the definition we gave for	\hyperref[def:simplicial-chain-group]{simplicial chain group}.
\end{note}

\begin{remark}
	We see that
	\begin{itemize}
		\item \autoref{lma:lec21} guarantees that our \hyperref[def:simplicial-chain-group]{simplicial chain groups} form a \hyperref[def:chain-complex]{chain complex}.
		\item \autoref{def:chain-complex} means that \(\ker (d_{n} ) \) contains \(\im  (d_{n+1})\), since \(d_{n} \circ d_{n+1} = 0\).
	\end{itemize}
\end{remark}

\begin{definition}[Exact]\label{def:exact}
	We say that the sequence is \emph{exact at \(C_{n} \)} provided that \(\ker (d_{n}) = \im (d_{n+1})\). A \hyperref[def:chain-complex]{chain complex} is \emph{exact}
	if it is \emph{exact at each point}.
\end{definition}

\begin{definition}[Homology group]\label{def:homology-group}
	The \emph{\(n^{th} \) homology group of a chain complex \((C_\ast, d_\ast)\)}, denoted as \(H_{n} \) or \(H_{n} (C_\ast)\), is the quotient
	\[
		H_{n} \coloneqq \quotient{\ker  (d_{n} )}{\im  (d_{n+1})}.
	\]
\end{definition}
\begin{remark}
	The \hyperref[def:homology-group]{homology group} measures how far the \hyperref[def:chain-complex]{chain complex} is from being \hyperref[def:exact]{exact} at \(C_{n} \).
\end{remark}

With what we have just defined, it's natural to define \hyperref[def:homology-group]{homology groups} of space \(X\) with a \hyperref[def:delta-complex]{\(\Delta \)-complex}
structure.
\begin{definition}[Homology class]\label{def:homology-class}
	We say \(\ker  (\partial _{n} )\) is the subgroup of \hyperref[def:cycle]{cycles} is \(C_{n} (X)\), and \(\im (\partial_{n+1}) \) is the
	subgroup of \hyperref[def:boundary]{boundaries} in \(C_{n} (X)\). We then set
	\[
		H_{n} (X)\coloneqq \quotient{\ker  (\partial _{n} )}{\im  (\partial _{n+1})} = \quotient{\{\text{\hyperref[def:cycle]{cycles}}\}}{\{\text{\hyperref[def:boundary]{boundaries}}\}}.
	\]
	In other words, it's the \hyperref[def:homology-group]{homology} of the \hyperref[def:chain-complex]{chain complex}
	\[
		\begin{tikzcd}
			\ldots & {C_{n+1}} & {C_{n}} & {C_{n-1}} & \ldots
			\arrow[from=1-1, to=1-2]
			\arrow["{\partial_{n+1}}", from=1-2, to=1-3]
			\arrow["{\partial_{n}}", from=1-3, to=1-4]
			\arrow["{\partial_{n-1}}", from=1-4, to=1-5]
		\end{tikzcd}
	\]
	where we take it to be \(0\) in all negative indices, namely
	\[
		\begin{tikzcd}
			\ldots & {C_{n+1}} & {C_{n}} & {C_{n-1}} & 0
			\arrow[from=1-1, to=1-2]
			\arrow["{\partial_{3}}", from=1-1, to=1-2]
			\arrow["{\partial_{2}}", from=1-2, to=1-3]
			\arrow["{\partial_{1}}", from=1-3, to=1-4]
			\arrow["{\partial_{0}}", from=1-4, to=1-5]
		\end{tikzcd}
	\]
	We then call the elements of \(H_{n} (X)\) as \emph{homology classes}.
\end{definition}

\begin{definition}[Simplicial homology group]\label{def:simplicial-homology-group}
	By considering the \hyperref[def:chain-complex]{chain complex} with \hyperref[def:simplicial-chain-group]{simplicial chain group},
	we have so-called \emph{simplicial homology group} induced by \autoref{def:homology-group}.
\end{definition}