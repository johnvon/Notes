\chapter{Foundation of Algebraic Topology}
\lecture{1}{05 Jan. 10:00}{Homotopies of Maps}
\section{Homotopy}
We start with the most important and fundamental concept, \hyperref[def:homotopy]{homotopy}.
\begin{definition*}
	Let \(X\), \(Y\) be topological spaces, and \(f\), \(g\colon X\to Y\) being two continuous maps.
	\begin{definition}[Homotopy]\label{def:homotopy}
		A \emph{homotopy} from \(f\) to \(g\) is a \(1\)-parameter family of maps that continuously deforms \(f\) to \(g\), i.e., it's a continuous function
		\(F\colon X\times I\to Y\), where \(I=\left[0,1\right]\), such that
		\[
			F(x, 0) = f(x),\quad F(x, 1) = g(x).
		\]
		We often write \(F_{t}(x)\) for \(F(x, t)\).
		\begin{figure}[H]
			\centering
			\incfig{def:homotopy}
			\caption{The continuous deforming from \(f\) to \(g\) described by \(F_t\)}
			\label{fig:def:homotopy}
		\end{figure}
	\end{definition}
	\begin{definition}[Homotopic]\label{def:homotopic}
		If a \hyperref[def:homotopy]{homotopy} exists between \(f\) and \(g\), we say they are \emph{homotopic} and write
		\[
			f\simeq g.
		\]
	\end{definition}
	\begin{definition}[Nullhomotopic]\label{def:nullhomotopic}
		If \(f\) is \hyperref[def:homotopic]{homotopic} to a constant map, we call it \emph{nullhomotopic}.
	\end{definition}

\end{definition*}

\begin{remark}
	Later, we'll not state that a map is continuous explicitly since we almost always assume this in this context.
\end{remark}

\begin{eg}[Straight line homotopy]\label{eg:lec1:straight-line-homotopy}
	Any two (continuous) maps with specification
	\[
		f,\ g\colon X\to \mathbb{\MakeUppercase{R}}^n
	\]
	are \hyperref[def:homotopic]{homotopic} by considering
	\[
		F_{t}(x) = (1 - t)f(x) + t g(x).
	\]
	We call it the \emph{straight line homotopy}.
\end{eg}
\begin{eg}
	Let \(S^1\) denotes the unit circle in \(\mathbb{\MakeUppercase{R}} ^2\), and
	\(D^2\)  denotes the unit disk in \(\mathbb{\MakeUppercase{R}} ^2\). Then the inclusion
	\(f\colon S^1\hookrightarrow D^2\) is \hyperref[def:nullhomotopic]{nullhomotopic}\dots
\end{eg}
\begin{explanation}
	We see this by considering
	\[
		F_t(x) = (1 - t)f(x) + (t\cdot 0).
	\]
	\begin{figure}[H]
		\centering
		\incfig{eg:homotopy}
		\caption{The illustration of \(F_{t}(x)\)}
		\label{fig:eg:homotopy}
	\end{figure}
	We see that there is a \hyperref[def:homotopy]{homotopy} from \(f(x)\) to \(0\) (the zero
	map which maps everything to \(0\)), and since \(0\) is a constant map, hence it's actually
	a \hyperref[def:nullhomotopic]{nullhomotopy}.
\end{explanation}
\begin{eg}
	The maps
	\[
		\begin{array}{ccc}
			S^1    & \to     & S^1 \\
			\Theta & \mapsto & S^1 \\
		\end{array}\quad \text{ and } \quad
		\begin{array}{ccc}
			S^1    & \to     & S^1     \\
			\Theta & \mapsto & -\Theta \\
		\end{array}
	\]
	are \textbf{not} \hyperref[def:homotopy]{homotopy}.
	\begin{remark}
		It will essentially \textbf{flip} the orientation, hence we can't deform one to another continuously.
	\end{remark}
\end{eg}

\begin{exercise}
	A subset \(S\subseteq \mathbb{\MakeUppercase{R}} ^n\) is star-shaped if \(\exists x_0\in S \text{ s.t. }\forall x\in S\),
	the line from \(x_0\) to \(x\) lies in \(S\).

	Show that \(\identity\colon S\to S\) is \hyperref[def:nullhomotopic]{nullhomotopic}.

	\begin{figure}[H]
		\centering
		\incfig{eg:star-shaped}
		\caption{Star-shaped illustration}
		\label{fig:eg:star-shaped}
	\end{figure}
\end{exercise}
\begin{answer}
	Consider
	\[
		F_{t}(x) \coloneqq (1 - t)x+tx_0,
	\]
	which essentially just concentrates all points \(x\) to \(x_0\).
\end{answer}
\begin{exercise}
	Suppose
	\[
		\begin{tikzcd}
			X & Y & Z
			\arrow["{f_0}"', shift right=1, from=1-1, to=1-2]
			\arrow["{g_0}"', shift right=1, from=1-2, to=1-3]
			\arrow["{f_1}", shift left=1, from=1-1, to=1-2]
			\arrow["{g_1}", shift left=1, from=1-2, to=1-3]
		\end{tikzcd}
	\]
	where
	\[
		f_0 \underset{F_{t}}{\simeq} f_1,\quad g_0\underset{G_{t}}{\simeq} g_1.
	\]
	Show
	\[
		g_0\circ f_0\simeq g_{1}\circ f_1.
	\]
\end{exercise}
\begin{answer}
	Consider \(I\times X\to Z\), where
	\[
		\begin{array}{ccccc}
			X \times I & \to     & Y \times I  & \to     & Z                \\
			(x, t)     & \mapsto & (F_t(x), t) & \mapsto & G_{t}(F_{t}(x)).
		\end{array}
	\]
	\begin{remark}
		Noting that if one wants to be precise, you need to check the continuity of this construction.
	\end{remark}
\end{answer}

\begin{exercise}
	How could you show \(2\) maps are \textbf{not} \hyperref[def:homotopic]{homotopic}?
\end{exercise}
\begin{answer}
	We'll see!
\end{answer}