\lecture{1}{05 Jan. 10:00}{Homotopies of Maps}
\section{Foundation of Algebraic Topology}
\subsection{Homotopy}
\begin{definition}[Homotopy, homotopic, nullhomotopic]\label{def:homotopy}\label{def:homotopic}\label{def:nullhomotopic}
	Let \(X\), \(Y\) be topological spaces. Let \(f\), \(g\colon X\to Y\) continuous maps. Then a \emph{homotopy} from \(f\)
	to \(g\) is a \(1-\)parameter family of maps that continuously deforms \(f\) to \(g\), i.e., it's a continuous function
	\(F\colon X\times I\to Y\), where \(I=\left[0,1\right]\), such that
	\[
		F(x, 0) = f(x),\quad F(x, 1) = g(x).
	\]
	We often write \(F_{t}(x)\) for \(F(x, t)\).

	If a homotopy exists between \(f\) and \(g\), we say they are \emph{homotopic} and write
	\[
		f\simeq g.
	\]
	If \(f\) is homotopic to a constant map, we call it \emph{nullhomotopic}.
\end{definition}
\begin{figure}[H]
	\centering
	\incfig{def:homotopy}
	\caption{The continuous deforming from \(f\) to \(g\) described by \(F_t\)}
	\label{fig:def:homotopy}
\end{figure}

\begin{remark}
	Later, we'll not state that a map is continuous explicitly since we almost always assume this in this context.
\end{remark}

\begin{eg}
	We first see some examples.
	\begin{enumerate}
		\item \label{eg:lec1:straight-line-homotopy} Any two maps (continuous) with specification
		      \[
			      f,\ g\colon X\to \mathbb{\MakeUppercase{R}}^n
		      \]
		      are \hyperref[def:homotopic]{homotopic} by considering
		      \[
			      F_{t}(x) = (1 - t)f(x) + t g(x).
		      \]
		      We call it \emph{the straight line homotopy}.
		\item Let \(S^1\) denotes the unit circle in \(\mathbb{\MakeUppercase{R}} ^2\), and
		      \(D^2\)  denotes the unit disk in \(\mathbb{\MakeUppercase{R}} ^2\). Then the inclusion
		      \(f\colon S^1\hookrightarrow D^2\) is \hyperref[def:nullhomotopic]{nullhomotopic} by considering
		      \[
			      F_t(x) = (1 - t)f(x) ( + t\cdot 0).
		      \]
		      \begin{figure}[H]
			      \centering
			      \incfig{eg:homotopy}
			      \caption{The illustration of \(F_{t}(x)\)}
			      \label{fig:eg:homotopy}
		      \end{figure}
		      We see that there is a \hyperref[def:homotopy]{homotopy} from \(f(x)\) to \(0\) (the zero
		      map which maps everything to \(0\)), and since \(0\) is a constant map, hence it's actually
		      a \hyperref[def:nullhomotopic]{nullhomotopy}.
		\item The maps
		      \[
			      \begin{array}{ccc}
				      S^1    & \to     & S^1 \\
				      \Theta & \mapsto & S^1 \\
			      \end{array}\quad \text{ and } \quad
			      \begin{array}{ccc}
				      S^1    & \to     & S^1     \\
				      \Theta & \mapsto & -\Theta \\
			      \end{array}
		      \]
		      are \textbf{not} \hyperref[def:homotopy]{homotopy}.
		      \begin{remark}
			      It will essentially \textbf{flip} the orientation, hence we can't deform one to another continuously.
		      \end{remark}
	\end{enumerate}
\end{eg}

\begin{exercise}
	We first see some exercises.
	\begin{enumerate}
		\item A subset \(S\subseteq \mathbb{\MakeUppercase{R}} ^n\) is star-shaped if
		      \[
			      \exists x_0\in S \text{ s.t. }\forall x\in S,
		      \]
		      the line from \(x_0\) to \(x\) lies in \(S\).
		      \begin{figure}[H]
			      \centering
			      \incfig{eg:star-shaped}
			      \caption{Star-shaped illustration}
			      \label{fig:eg:star-shaped}
		      \end{figure}
		      Show that \(\identity\colon S\to S\) is \hyperref[def:nullhomotopic]{nullhomotopic}.

		      \begin{answer}
			      Consider
			      \[
				      F_{t}(x) \coloneqq (1 - t)x+tx_0,
			      \]
			      which essentially just concentrates all points \(x\) to \(x_0\).
		      \end{answer}
		\item Suppose
		      \[
			      \begin{tikzcd}
				      X \ar[r,"f_1", "f_0"'] & Y \ar[r,"g_1", "g_0"'] & Z
			      \end{tikzcd}
		      \]
		      where
		      \[
			      f_0 \underset{F_{t}}{\simeq} f_1,\quad g_0\underset{G_{t}}{\simeq} g_1.
		      \]
		      Show
		      \[
			      g_0\circ f_0\simeq g_{1}\circ f_1.
		      \]
		      \begin{answer}
			      Consider \(I\times X\to Z\). Then
			      \[
				      \begin{array}{ccccc}
					      X \times I & \to     & Y \times I  & \to     & Z                \\
					      (x, t)     & \mapsto & (F_t(x), t) & \mapsto & G_{t}(F_{t}(x)).
				      \end{array}
			      \]
		      \end{answer}
		      \begin{remark}
			      Noting that if one wants to be precise, you need to check the continuity of this construction.
		      \end{remark}
		\item How could you show \(2\) maps are \textbf{not} \hyperref[def:homotopic]{homotopic}?
		      \begin{answer}

		      \end{answer}
	\end{enumerate}
\end{exercise}