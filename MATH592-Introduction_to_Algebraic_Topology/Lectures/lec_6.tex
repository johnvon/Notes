\lecture{6}{19 Jan. 10:00}{A Foray into Category Theory}
\subsection{Category Theory}
We start with a definition.
\begin{definition}[Object, Morphism]
	A category \(\mathscr{C} \) is \(3\) pieces of data
	\begin{itemize}
		\item A class of objects \(\mathrm{Ob}(\mathscr{C})\)
		\item \(\forall X, Y\in\mathrm{Ob} (\mathscr{C})\) a class of \underline{morphisms} or \underline{arrows},
		      \(\Homomorphism_{\mathscr{C}}(X, Y)\).
		\item \(\forall X, Y, Z\in \Object (\mathscr{C})\), there exists a composition law
		      \[
			      \begin{split}
				      \Homomorphism (X, Y)\times \Homomorphism (Y, Z)&\to \Homomorphism (X, Z)\\
				      (f, g)&\mapsto g\circ f
			      \end{split}
		      \]
	\end{itemize}
	and \(2\) axioms
	\begin{itemize}
		\item Associativity. \((f\circ g)\circ h = f\circ (g\circ h)\) for all morphisms \(f, g, h\) where composites are defined.
		\item Identity. \(\forall X\in\Object (\mathscr{C})\ \exists \identity_{X}\in\Homomorphism _{\mathscr{C}} (X, X)\) such that
		      \[
			      f\circ \identity_{X} = f,\quad \identity_{X} \circ g = g
		      \]
		      for all \(f, g\) where this makes sense.
	\end{itemize}
\end{definition}

Let's see some examples.
\begin{eg}
	We introduce some common category.
	\begin{table}[H]
		\centering
		\begin{tabular}{c|c|c}
			\toprule
			\(\mathcal{C} \)                & \(\Object(\mathcal{C} )\)               & \(\Morphism (\mathcal{C} )\)        \\
			\midrule
			\(\underline{\mathrm{set}}\)    & Sets \(X\)                              & All maps of sets                    \\
			\(\underline{\mathrm{fset}}\)   & Finite sets                             & All maps                            \\
			\(\underline{\mathrm{Gp}}\)     & Groups                                  & Group Homomorphisms                 \\
			\(\underline{\mathrm{Ab}}\)     & Abelian groups                          & Group Homomorphisms                 \\
			\(\underline{k\mathrm{-vect}}\) & Vector spaces over \(k\)                & \(k\)-linear maps                   \\
			\(\underline{\mathrm{Rng}}\)    & Rings                                   & Ring Homomorphisms                  \\
			\(\underline{\mathrm{Top}}\)    & Topological spaces                      & Continuous maps                     \\
			\(\underline{\mathrm{Haus}}\)   & Hausdorff Spaces                        & Continuous maps                     \\
			\(\underline{\mathrm{hTop}}\)   & Topological spaces                      & Homotopy classes of continuous maps \\
			\(\underline{\mathrm{Top}^*}\)  & Based topological spaces\footnotemark{} & Based maps\footnotemark{}           \\
			\bottomrule
		\end{tabular}
	\end{table}
	\addtocounter{footnote}{-2}
	\stepcounter{footnote}\footnotetext{Topological spaces with a distinguished base point \(x_0\in X\) }
	\stepcounter{footnote}\footnotetext{Continuous maps that presence base point \(f\colon (x, x_0)\to (y, y_0)\) such that \[f\colon X\to Y,\quad f(x_0) = y_0\] is continuous.}
\end{eg}

\begin{remark}
	Any \textbf{diagram} plus composition law.
	\[
		\begin{tikzcd}[cells={nodes={}}]
			\arrow[loop left, distance=1em, start anchor={[yshift=-1ex]west}, end anchor={[yshift=1ex]west}]{}{\identity_{A} } \arrow{r} A
			& B \arrow[loop right, distance=1em, start anchor={[yshift=1ex]east}, end anchor={[yshift=-1ex]east}]{}{\identity_{B} }
		\end{tikzcd}.
	\]
\end{remark}

\begin{definition}[monic, epic]
	A morphism \(f\colon M\to N\) is \emph{monic} if
	\[
		\forall g_1, g_2\ f\circ g_1 = f\circ g_2 \implies g_1 = g_2.
	\]
	\[
		\begin{tikzcd}
			A\ar[r, bend left, "g_1"]\ar[r, bend right, "g_2"']& M \ar[r, "f"]& N
		\end{tikzcd}
	\]
	Dually, \(f\) is \emph{epic} if
	\[
		\forall g_1, g_2\ g_{1} \circ f = g_2 \circ f \implies g_1 = g_2.
	\]
	\[
		\begin{tikzcd}
			M\ar[r, "f"]& N\ar[r, bend left, "g_1"]\ar[r, bend right, "g_2"']& B
		\end{tikzcd}
	\]
\end{definition}

\begin{lemma}
	In \(\underline{\mathrm{set}}, \underline{\mathrm{Ab}}, \underline{\mathrm{Top}}, \underline{\mathrm{Gp}}\), a map is monic if and only
	if \(f\) is injective, and epic if and only if \(f\) is surjective.
\end{lemma}
\begin{proof}
	In \(\underline{\mathrm{set}}\), we prove that \(f\) is monic if and only if \(f\) is injective. Suppose
	\(f\circ g_1 = f\circ g_2\) and \(f\) is injective, then for any \(a\),
	\[
		f(g_1(a)) = f(g_2(a))\implies g_1(a) = g_2(a),
	\]
	hence \(g_1 = g_2\).

	\par Now we prove another direction, with contrapositive. Namely, we assume that \(f\) is \underline{not} injective and show that
	\(f\) is not monic. Suppose \(f(a) = f(b)\) and \(a\neq b\), we want to show such \(g_{i}\) exists. This is easy by considering
	\[
		g_1\colon *\mapsto a,\quad g_2\colon *\mapsto b.
	\]
\end{proof}

\subsubsection{Functor}
After introducing the category, we then see the most important concept we'll use, a \emph{functor}. Again, we start with the definition.
\begin{definition}[Functor]
	Given \(\mathscr{C} , \mathscr{D} \) be two categories. A (\underline{covariant}) \emph{functor}
	\[
		F\colon \mathscr{C} \to \mathscr{D}
	\]
	is
	\begin{enumerate}
		\item a map on objects
		      \[
			      \begin{split}
				      F\colon \Object(\mathscr{C} )&\to \Object(\mathscr{D} )\\
				      X&\mapsto F(X).
			      \end{split}
		      \]
		\item maps of morphisms
		      \[
			      \begin{split}
				      \Homomorphism_{\mathscr{C} }(X, Y)&\to \Homomorphism_{\mathscr{D} }(F(X), F(Y))\\
				      \left[f\colon X\to Y\right] &\mapsto \left[F(f)\colon F(X)\to F(Y)\right]
			      \end{split}
		      \]
		      such that
		      \begin{itemize}
			      \item \(F(\identity_{X} ) = \identity_{F(x)} \)
			      \item \(F(f\circ g) = F(f)\circ F(g)\)
		      \end{itemize}
	\end{enumerate}
\end{definition}