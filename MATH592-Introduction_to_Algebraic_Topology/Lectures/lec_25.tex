\lecture{25}{11 Mar. 10:00}{Relative Homology}
We are now interested in the relationship between \(H_{n}(X), H_{n} (A), H_{n} \left(\quotient{X}{A}\right)\).

\subsection{Relative Homology}
\begin{definition}[Reduced homology group]\label{def:reduced-homology-group}
	The \emph{reduced homology groups} \(\widetilde{H}_n(X) = H_n(X)\) when \(n > 0\). When \(n = 0\) we have that:
	\[
		\widetilde{H}_0(X) \oplus \mathbb{\MakeUppercase{z}} = H_0(X).
	\]
\end{definition}
\begin{remark}
	The usefulness of this is that for \hyperref[def:path]{path}-connected space \(X\) we have \(\widetilde{H}_0(X) = 0\), and for
	\hyperref[def:contractible]{contractible} spaces \(X\) we have \(\widetilde{H}_n(X) = 0\).
\end{remark}

\begin{definition}[Good pair]\label{def:good-pair}
	Let \(X\) be a space, and \(A \subseteq X\). Then \((X, A)\) is a \emph{good pair} if \(A\) is closed and nonempty, and also it is a
	\hyperref[def:deformation-retraction]{deformation retract} of a neighborhood in \(X\).
\end{definition}

\begin{eg}
	Let's see some examples.
	\begin{enumerate}
		\item If \(X\) is a \hyperref[def:CW-Complex]{CW complex} and \(A\) is a nonempty \hyperref[def:CW-subcomplex]{subcomplex}, then \((X, A)\)
		      is a \hyperref[def:good-pair]{good pair}. The proof is given in the Appendix of Hatcher\cite{hatcher2002algebraic} and requires some
		      point-set topology.
		\item If \(M\) is a smooth manifold, and \(N\subseteq M\) is a smooth submanifold which is nonempty, then \((M, N)\) is a \hyperref[def:good-pair]{good pair}.
		\item \((\text{Hawaiian earring}, \text{bad point})\) is \underline{not} a \hyperref[def:good-pair]{good pair}.
		\item \((\mathbb{\MakeUppercase{r}} ^n, \text{ proper open set})\) is \underline{not} a \hyperref[def:good-pair]{good pair}.
	\end{enumerate}
\end{eg}

\begin{theorem}[Long exact sequence of a good pair]\label{thm:les-of-a-good-pair}
	If \((X, A)\) is a \hyperref[def:good-pair]{good pair}, then there exists a long \hyperref[def:exact]{exact} sequence
	(\hyperref[def:exact]{exact} at every \(n\)) on \hyperref[def:reduced-homology-group]{reduced homology groups} given by the
	following commutative diagram.
	\[
		\begin{tikzcd}
			\ldots & {\widetilde{H}_n(A)} & {\widetilde{H}_n(X)} & {\widetilde{H}_n(\quotient{X}{A})} \\
			& {\widetilde{H}_{n-1}(A)} & {\widetilde{H}_{n-1}(X)} & {\widetilde{H}_{n-1}(\quotient{X}{A})} \\
			& \ldots & {\widetilde{H}_{0}(X)} & {\widetilde{H}_{0}(\quotient{X}{A})} & 0
			\arrow["{j_\ast}"{description}, from=1-3, to=1-4]
			\arrow["{j_\ast}"{description}, from=2-3, to=2-4]
			\arrow[from=3-4, to=3-5]
			\arrow["{i_\ast}"{description}, from=2-2, to=2-3]
			\arrow["{i_\ast}"{description}, from=1-2, to=1-3]
			\arrow[from=1-1, to=1-2]
			\arrow["\delta"{description}, from=1-4, to=2-2]
			\arrow["\delta"{description}, from=2-4, to=3-2]
			\arrow["{j_\ast}"{description}, from=3-3, to=3-4]
			\arrow["{i_\ast}"{description}, from=3-2, to=3-3]
		\end{tikzcd}
	\]
	where \(i \colon A \hookrightarrow X\) is the inclusion and \(j \colon X \to \quotient{X}{A}\) is the quotient map.
\end{theorem}
We see that both \(i_\ast\) and \(j_\ast\) is naturally induced, but not for \(\delta \). In fact, we'll construct \(\delta\) in the proof!
Specifically, we'll see that \autoref{thm:les-of-a-good-pair} is just a special case of \autoref{thm:long-exact-sequence-of-a-pair}, hence
rather than proof \autoref{thm:les-of-a-good-pair} directly, we will prove \autoref{thm:long-exact-sequence-of-a-pair} instead later.

\begin{remark}
	The fact that this sequence is \hyperref[def:exact]{exact} often means that if we know the \hyperref[def:homology-group]{homology groups} of two of the
	spaces we can compute the \hyperref[def:homology-group]{homology} of the remaining space.
\end{remark}

Before we see the proof of \autoref{thm:les-of-a-good-pair}, we see one application.

\begin{proposition}\label{prop:homology-of-spheres}
	We have that:
	\[
		\widetilde{H}_i(S^n) = \begin{dcases}
			\mathbb{\MakeUppercase{z}} , & \text{ if } i = n ;    \\
			0,                           & \text{ if } i \neq n .
		\end{dcases}
	\]
\end{proposition}
\begin{exercise}
	Verify \autoref{prop:homology-of-spheres} in the case \(n = 0\), so \(S^0 \) is just \(2\) points.
\end{exercise}
\begin{proof}
	Some facts we need:
	\begin{itemize}
		\item \((D^n, \partial D^n)\) is a \hyperref[def:good-pair]{good pair} (since it is a \hyperref[def:CW-Complex]{CW complex}
		      and a \hyperref[def:CW-subcomplex]{subcomplex})
		\item \(D^n /\partial D^n \cong S^n\).
		\item \(\widetilde{H}_n(D^n) = 0\) for all \(n\) since \(D^n\) is \hyperref[def:contractible]{contractible}.
		\item \(\partial D^n \cong S^{n - 1}\).
	\end{itemize}
	We then proceed by induction on \(n\). Using the long \hyperref[def:exact]{exact} sequence, we have
	\[
		\begin{tikzcd}
			\ldots & {\widetilde{H}_n(\partial D^n)} & {\widetilde{H}_n(D^n)} & {\widetilde{H}_n(S^n)} \\
			& {\widetilde{H}_{n-1}(\partial D^n)} & {\widetilde{H}_{n-1}(D^n)} & {\widetilde{H}_{n-1}(S^n)} \\
			& \ldots & {\widetilde{H}_{0}(D^n)} & {\widetilde{H}_{0}(S^n)} & 0
			\arrow["{j_\ast}"{description}, from=1-3, to=1-4]
			\arrow["{j_\ast}"{description}, from=2-3, to=2-4]
			\arrow[from=3-4, to=3-5]
			\arrow["{i_\ast}"{description}, from=2-2, to=2-3]
			\arrow["{i_\ast}"{description}, from=1-2, to=1-3]
			\arrow[from=1-1, to=1-2]
			\arrow["\delta"{description}, from=1-4, to=2-2]
			\arrow["\delta"{description}, from=2-4, to=3-2]
			\arrow["{j_\ast}"{description}, from=3-3, to=3-4]
			\arrow["{i_\ast}"{description}, from=3-2, to=3-3]
		\end{tikzcd}
	\]
	By induction, we have \(\widetilde{H} _{n-1}(\partial D^n) = \widetilde{H} _{n-1}(S^{n-1}) = \mathbb{\MakeUppercase{z}} \), hence we can fill in some of these groups as follows.
	\[
		\begin{tikzcd}
			\ldots & {0} & {0} & {\widetilde{H}_n(S^n)} \\
			& {\mathbb{\MakeUppercase{z}} } & {0} & {\widetilde{H}_{n-1}(S^n)} \\
			& \ldots & {0} & {\widetilde{H}_{0}(S^n)} & 0
			\arrow["{j_\ast}"{description}, from=1-3, to=1-4]
			\arrow["{j_\ast}"{description}, from=2-3, to=2-4]
			\arrow[from=3-4, to=3-5]
			\arrow["{i_\ast}"{description}, from=2-2, to=2-3]
			\arrow["{i_\ast}"{description}, from=1-2, to=1-3]
			\arrow[from=1-1, to=1-2]
			\arrow["\delta"{description}, from=1-4, to=2-2]
			\arrow["\delta"{description}, from=2-4, to=3-2]
			\arrow["{j_\ast}"{description}, from=3-3, to=3-4]
			\arrow["{i_\ast}"{description}, from=3-2, to=3-3]
		\end{tikzcd}
	\]
	In all, we have an \hyperref[def:exact]{exact} sequence:
	\[
		\begin{tikzcd}
			0 & {\widetilde{H}_n(S^n)} & {\mathbb{Z}} & 0
			\arrow["\delta", from=1-2, to=1-3]
			\arrow[from=1-3, to=1-4]
			\arrow[from=1-1, to=1-2]
		\end{tikzcd}
	\]
	By \hyperref[def:exact]{exactness}, \(\delta\) is an isomorphism, thus \(\widetilde{H}_n(S^n) \cong \mathbb{\MakeUppercase{z}}\). Now we must verify \(\widetilde{H}_i(S^n) = 0\)
	when \(i\neq n\). In that case the \hyperref[def:exact]{exact} sequence looks like:
	\[
		\begin{tikzcd}
			{} & {\widetilde{H}_i(D^n)} & {\widetilde{H}_i(S^n)} & {\widetilde{H}_{i-1}(\partial D^n)} \\
			{} & \color{red}0 & \color{red}{\widetilde{H}_i(S^n)} & \color{red}0
			\arrow[from=1-2, to=1-3]
			\arrow[from=1-3, to=1-4]
			\arrow[from=1-1, to=1-2]
			\arrow[color={rgb,255:red,214;green,92;blue,92}, from=2-1, to=2-2]
			\arrow[color={rgb,255:red,214;green,92;blue,92}, from=2-2, to=2-3]
			\arrow[color={rgb,255:red,214;green,92;blue,92}, from=2-3, to=2-4]
		\end{tikzcd}
	\]
	\hyperref[def:exact]{Exactness} then tells us that \(\widetilde{H}_i(S^n) = 0\).
\end{proof}

\begin{theorem}[Brouwer's fixed point theorem]\label{thm:Brouwer-fixed-point}
	\(\partial D^n\) is not a \hyperref[def:retraction]{retract} of \(D^n\). Hence, every continuous map \(f \colon D^n \to D^n\) has a fixed point.
\end{theorem}

\begin{proof}
	If \(r \colon D^n \to \partial D^n\) were a \hyperref[def:retraction]{retraction}, then by definition this would give us
	\[
		\begin{tikzcd}
			{\partial D^n} & {D^n} & {\partial D^n}
			\arrow["i", from=1-1, to=1-2]
			\arrow["r", from=1-2, to=1-3]
			\arrow["{\identity_{\partial D^n}}"', curve={height=12pt}, from=1-1, to=1-3]
		\end{tikzcd}
	\]
	\hyperref[thm:functoriality-is-homotopy-invariant]{Functoriality} of \hyperref[def:homology-group]{homology} implies
	\[
		\begin{tikzcd}
			{\widetilde{H}_{n-1}(\partial D^n)} & {\widetilde{H}_{n-1}(D^n)} & {\widetilde{H}_{n-1}(\partial D^n)}
			\arrow["{i_\ast}", from=1-1, to=1-2]
			\arrow["{r_\ast}", from=1-2, to=1-3]
			\arrow["\identity"', curve={height=12pt}, from=1-1, to=1-3]
		\end{tikzcd}
	\]
	So then:
	\[
		\begin{tikzcd}
			{\mathbb{Z}} & 0 & {\mathbb{\MakeUppercase{z}} }
			\arrow["{i_{\ast}}", from=1-1, to=1-2]
			\arrow["{r_\ast}", from=1-2, to=1-3]
			\arrow["\identity"', curve={height=12pt}, from=1-1, to=1-3]
		\end{tikzcd}
	\]
	which is impossible since the map \(\identity_{Z} \) can't be factored through \(0\).
\end{proof}
\begin{exercise}
	As with \(D^2\), if \(f \colon D^n \to D^n\) had no fixed point, we could build a \hyperref[def:retraction]{retraction}.
\end{exercise}

In order to proof \autoref{thm:les-of-a-good-pair}, we introduce the concept of \emph{diagram chase}.

\begin{lemma}[The short five lemma]\label{lma:the-short-five-lemma}
	Suppose we have a commutative diagram
	\[
		\begin{tikzcd}
			0 & A & B & C & 0 \\
			0 & {A^\prime} & B & {C^\prime} & 0
			\arrow[from=1-1, to=1-2]
			\arrow["\psi", from=1-2, to=1-3]
			\arrow["\varphi", from=1-3, to=1-4]
			\arrow[from=1-4, to=1-5]
			\arrow["\gamma", from=1-4, to=2-4]
			\arrow[from=2-4, to=2-5]
			\arrow["{\varphi^\prime}", from=2-3, to=2-4]
			\arrow["\beta", from=1-3, to=2-3]
			\arrow["{\psi^\prime}", from=2-2, to=2-3]
			\arrow["\alpha", from=1-2, to=2-2]
			\arrow[from=2-1, to=2-2]
		\end{tikzcd}
	\]
	so that the rows are \hyperref[def:exact]{exact}. Then:
	\begin{enumerate}
		\item If \(\alpha, \gamma\) are injective then \(\beta\) is injective.
		\item If \(\alpha, \gamma\) are surjective then \(\beta\) is surjective.
		\item If \(\alpha, \gamma\) are isomorphisms then \(\beta\) is an isomorphism
	\end{enumerate}
\end{lemma}

\begin{proof}
	1. and 2. imply 3. We leave 2. as an exercise. We fix \(b \in B\) such that \(\beta(b) = 0\). We want to show that \(\beta = 0\). Well, we draw a diagram chase as
	\[
		\begin{tikzcd}
			0 & {\bullet } & b & {\varphi(b)} & 0 \\
			0 & {\bullet } & 0 & 0 & 0
			\arrow[maps to, from=1-1, to=1-2]
			\arrow["\psi", maps to, from=1-2, to=1-3]
			\arrow["\varphi", maps to, from=1-3, to=1-4]
			\arrow[maps to, from=1-4, to=1-5]
			\arrow["\gamma", maps to, from=1-4, to=2-4]
			\arrow[maps to, from=2-4, to=2-5]
			\arrow["{\varphi^\prime}", maps to, from=2-3, to=2-4]
			\arrow["\beta", maps to, from=1-3, to=2-3]
			\arrow["{\psi^\prime}", maps to, from=2-2, to=2-3]
			\arrow["\alpha", maps to, from=1-2, to=2-2]
			\arrow[maps to, from=2-1, to=2-2]
		\end{tikzcd}
	\]
	And thus by injectivity of \(\gamma\) we know \(\varphi(b) = 0\). By \hyperref[def:exact]{exactness}, \(b \in \im \psi\). We then
	may write for some \(a \in A\) such that the following diagram commutes.
	\[
		\begin{tikzcd}
			0 & a & b & 0 & 0 \\
			0 & {\alpha(a)} & 0 & 0 & 0
			\arrow[maps to, from=1-1, to=1-2]
			\arrow["\varphi", maps to, from=1-3, to=1-4]
			\arrow[maps to, from=1-4, to=1-5]
			\arrow["\gamma", maps to, from=1-4, to=2-4]
			\arrow[maps to, from=2-4, to=2-5]
			\arrow["{\varphi^\prime}", maps to, from=2-3, to=2-4]
			\arrow["\beta", maps to, from=1-3, to=2-3]
			\arrow["{\psi^\prime}", maps to, from=2-2, to=2-3]
			\arrow["\alpha", maps to, from=1-2, to=2-2]
			\arrow[maps to, from=2-1, to=2-2]
			\arrow["\psi", maps to, from=1-2, to=1-3]
		\end{tikzcd}
	\]
	Therefore \(\psi^\prime (\alpha(a)) = \beta(\psi(a)) = \beta(b) = 0\) by commutativity. By \hyperref[def:exact]{exactness} of the
	bottom row we know that \(\psi^\prime\) is an injection.

	Thus, \(\alpha(a) = 0\), so since \(\alpha\) is injective, \(a = 0\). With this \(b = \psi(a) = \psi(0) = 0\). Great! With this
	\(\ker(\beta) = 0\), and \(\beta\) injects.
\end{proof}

