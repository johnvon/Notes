\lecture{28}{18 Mar. 10:00}{Singular Homology v.s. Simplicial Homology}
\begin{remark}
	If \(M\) is a smooth manifold and \(N\) is an embedded smooth closed submanifold, then \((M, N)\) is a \hyperref[def:good-pair]{good pair}.
	Why? Well this follows from the tubular neighborhood theorem, which should be proven in a course like 591. We will only use the result
	in obvious cases, and simply assert that certain pairs are \hyperref[def:good-pair]{good pairs}.
\end{remark}

With pairs like \((\mathbb{\MakeUppercase{r}}^{n + 1}, S^n)\), you can just assert that this is a \hyperref[def:good-pair]{good pair} (and do
not need to prove that \(S^n\) is a smooth submanifold of \(\mathbb{\MakeUppercase{r}}^{n + 1}\)). Another good example is manifolds and their
boundary always form a \hyperref[def:good-pair]{good pair}.

\begin{theorem}[Singular homology agrees with simplicial homology]\label{thm:singular-homology-agrees-with-simplicial-homology}
	Let \(X\) be a \hyperref[def:delta-complex]{\(\Delta\)-complex}. We use \(\Delta_n(X)\) to represent the
	\hyperref[def:simplicial-complex]{simplicial chain groups} on \(X\), and \(C_n(X)\) to denote the
	\hyperref[def:singular-chain]{singular chain groups}. Likewise, we denote
	\[
		\Delta_n(X, A) = \quotient{\Delta_n(X)}{\Delta_n(A)}
	\]
	and
	\[
		C_n(X, A) = \quotient{C_n(X)}{C_n(A)}.
	\]
	The inclusion \(\Delta_\ast(X, A) \hookrightarrow C_\ast(X, A)\) given by
	\[
		[\sigma \colon \Delta^n \to X] \mapsto [\sigma : \Delta^n \to X]
	\]
	induces an isomorphism on \hyperref[def:homology-group]{homology} such that
	\[
		H_n^\Delta(X, A) \cong H_n(X, A).
	\]

	If we consider the case that \(A = \varnothing\), we recover the case of \hyperref[def:homology-group]{absolute homology}
	\[
		H_n^\Delta(X) \cong H_n(X).
	\]
\end{theorem}
The proof of \autoref{thm:singular-homology-agrees-with-simplicial-homology} uses the following lemma.
\begin{lemma}[The five lemma]\label{lma:the-five-lemma}
	If we have a commutative diagram with \hyperref[def:exact]{exact} rows as following,
	\[
		\begin{tikzcd}
			A & B & C & D & E \\
			{A^\prime} & {B^\prime} & {C^\prime} & {D^\prime} & {E^\prime}
			\arrow["\alpha", from=1-1, to=2-1]
			\arrow["\gamma", from=1-3, to=2-3]
			\arrow["\delta", from=1-4, to=2-4]
			\arrow["\epsilon", from=1-5, to=2-5]
			\arrow["{i^\prime}"', from=2-1, to=2-2]
			\arrow["\beta", from=1-2, to=2-2]
			\arrow["{j^\prime}"', from=2-2, to=2-3]
			\arrow["{k^\prime}"', from=2-3, to=2-4]
			\arrow["{\ell^\prime}"', from=2-4, to=2-5]
			\arrow["\ell", from=1-4, to=1-5]
			\arrow["k", from=1-3, to=1-4]
			\arrow["j", from=1-2, to=1-3]
			\arrow["i", from=1-1, to=1-2]
		\end{tikzcd}
	\]
	If \(\alpha, \beta, \delta, \epsilon\) are isomorphisms, then so is \(\gamma\).
\end{lemma}
\begin{proof}
	Diagram chase!
\end{proof}