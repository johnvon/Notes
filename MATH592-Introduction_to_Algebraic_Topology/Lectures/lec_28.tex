\lecture{28}{18 Mar. 10:00}{Singular Homology Agrees with Simplicial Homology}
\begin{remark}
	If $M$ is a smooth manifold and $N$ is an embedded smooth closed submanifold, then $(M, N)$ is a \hyperref[def:good-pair]{good pair}. Why? Well this
	follows from the tubular neighborhood theorem, which should be proven in a course like 591. We will only use the result in obvious cases, and simply
	assert that certain pairs are \hyperref[def:good-pair]{good pairs}.
\end{remark}

With pairs like $(\mathbb{\MakeUppercase{r}}^{n + 1}, S^n)$, you can just assert that this is a \hyperref[def:good-pair]{good pair} (and do not need to
prove that $S^n$ is a smooth submanifold of $\mathbb{\MakeUppercase{r}}^{n + 1}$). Another good example is manifolds and their boundary always form a \hyperref[def:good-pair]{good pair}.

\begin{theorem}[Singular homology agrees with simplicial homology]\label{thm:singular-homology-agrees-with-simplicial-homology}
	Let $X$ be a \hyperref[def:delta-complex]{$\Delta$-complex}. We use $\Delta_n(X)$ to represent the \hyperref[def:simplicial-complex]{simplicial chain groups} on $X$, and
	$C_n(X)$ to denote the \hyperref[def:singular-chain]{singular chain groups}. Likewise, we denote
	\[
		\Delta_n(X, A) = \quotient{\Delta_n(X)}{\Delta_n(A)}
	\]
	and
	\[
		C_n(X, A) = \quotient{C_n(X)}{C_n(A)}.
	\]
	The inclusion $\Delta_\ast(X, A) \hookrightarrow C_\ast(X, A)$ given by
	\[
		[\sigma : \Delta^n \to X] \mapsto [\sigma : \Delta^n \to X]
	\]
	induces an isomorphism on \hyperref[def:homology-group]{homology} such that
	\[
		H_n^\Delta(X, A) \cong H_n(X, A).
	\]

	If we consider the case that \(A = \varnothing\), we recover the case of \hyperref[def:homology-group]{absolute homology}
	\[
		H_n^\Delta(X) \cong H_n(X).
	\]
\end{theorem}

\begin{proof}[Proof Sketch]
	The Proof uses the following lemma.
	\begin{lemma}[The five lemma]\label{lma:the-five-lemma}
		If we have a commutative diagram with \hyperref[def:exact]{exact} rows as following,
		\[
			\begin{tikzcd}
				A & B & C & D & E \\
				{A^\prime} & {B^\prime} & {C^\prime} & {D^\prime} & {E^\prime}
				\arrow["\alpha", from=1-1, to=2-1]
				\arrow["\gamma", from=1-3, to=2-3]
				\arrow["\delta", from=1-4, to=2-4]
				\arrow["\epsilon", from=1-5, to=2-5]
				\arrow["{i^\prime}"', from=2-1, to=2-2]
				\arrow["\beta", from=1-2, to=2-2]
				\arrow["{j^\prime}"', from=2-2, to=2-3]
				\arrow["{k^\prime}"', from=2-3, to=2-4]
				\arrow["{\ell^\prime}"', from=2-4, to=2-5]
				\arrow["\ell", from=1-4, to=1-5]
				\arrow["k", from=1-3, to=1-4]
				\arrow["j", from=1-2, to=1-3]
				\arrow["i", from=1-1, to=1-2]
			\end{tikzcd}
		\]
		If $\alpha, \beta, \delta, \epsilon$ are isomorphisms, then so is $\gamma$.
	\end{lemma}
	\begin{proof}
		Diagram chase!
	\end{proof}

	The idea is as follows.
	\begin{itemize}
		\item We can use the long \hyperref[def:exact-sequence]{exact sequence} of a pair and the \autoref{lma:the-five-lemma} to reduce to proving the result for
		      \hyperref[def:homology-group]{absolute homology groups} (and we will recover the general result).
		\item Because the image $\Delta^n \to X$ is compact, it is contained in some finite skeleton $X^k$. Use this to reduce the proof to the finite skeleton $X^k$ of $X$
	\end{itemize}
	From the \hyperref[def:exact-sequence]{long exact sequence} of a \hyperref[def:good-pair]{pair} we get
	\par
	\adjustbox{scale=0.9,center}{%
		\begin{tikzcd}[column sep=small]
			{H^\Delta_{n+1}(X^k, X^{k-1})} & {H^\Delta_{n}(X^{k-1})} & {H^\Delta_{n}(X^{k})} & {H^\Delta_{n}(X^k, X^{k-1})} & {H^\Delta_{n-1}(X^{k-1})} \\
			{H_{n+1}(X^k, X^{k-1})} & {H_{n}(X^{k-1})} & {H_{n}(X^{k})} & {H_{n}(X^k, X^{k-1})} & {H_{n-1}(X^{k-1})}
			\arrow["\alpha", color={rgb,255:red,214;green,92;blue,92}, from=1-1, to=2-1]
			\arrow["\beta", color={rgb,255:red,92;green,92;blue,214}, from=1-2, to=2-2]
			\arrow["\gamma", color={rgb,255:red,124;green,194;blue,112}, from=1-3, to=2-3]
			\arrow["\delta", color={rgb,255:red,214;green,92;blue,92}, from=1-4, to=2-4]
			\arrow["\epsilon", color={rgb,255:red,92;green,92;blue,214}, from=1-5, to=2-5]
			\arrow[from=1-1, to=1-2]
			\arrow[from=1-2, to=1-3]
			\arrow[from=1-3, to=1-4]
			\arrow[from=1-4, to=1-5]
			\arrow[from=2-1, to=2-2]
			\arrow[from=2-2, to=2-3]
			\arrow[from=2-3, to=2-4]
			\arrow[from=2-4, to=2-5]
		\end{tikzcd}
	}

	The Goal is to prove $\gamma$ is an isomorphism using the \autoref{lma:the-five-lemma}.

	We assume that \textcolor{blue}{$\beta, \epsilon $} are isomorphisms by induction, checking the case manually for $X^0$ (which will be a discrete set of points).
	It remains to show that \textcolor{red}{$\alpha, \delta $} are isomorphisms.

	We know then that:
	\[
		\begin{split}
			\Delta_n(X^k, X^{k - 1}) & = \begin{dcases}
				\mathbb{\MakeUppercase{z}} [\text{\hyperref[def:singular-simplex]{\(k\)-simplices}}], & \text{ if } k=n ; \\
				0,                                                                                    & \text{ otherwise}
			\end{dcases} \\
			& = H_n^\Delta(X^k, X^{k - 1}).
		\end{split}
	\]

	We claim that $H_n(X^k, X^{k - 1})$ are also \hyperref[def:free-Abelian-group]{free Abelian} on the \hyperref[def:singular-simplex]{singular $k$-simplices} defined by the
	characteristic maps $\Delta^k \to X^k$ when $n = k$, and $0$ otherwise. Consider the map
	\[
		\Phi \colon \coprod_\alpha (\Delta^k_\alpha, \partial \Delta^k_\alpha) \to (X^k, X^{k - 1})
	\]
	defined by the characteristic map. This induces an isomorphism on \hyperref[def:homology-group]{homology} since
	\[
		\quotient{\coprod_\alpha \Delta_\alpha^k}{\coprod \partial \Delta_\alpha^k} \overset{\cong}{\longrightarrow}  \quotient{X^k}{X^{k - 1}}.
	\]

	This reduces to check that
	\[
		H_n(\Delta^k, \partial \Delta^k) = \begin{dcases}
			0,                           & \text{ if } n \neq k ; \\
			\mathbb{\MakeUppercase{z}} , & \text{ if } n=k
		\end{dcases}
	\]
	generated by the identity map $\Delta^k \to \Delta^k$.
\end{proof}

\begin{corollary}
	If \(X\) has a \hyperref[def:delta-complex]{\(\Delta \)-complex} structure (or is \hyperref[def:homotopy-equivalence]{homotopy equivalent} to one) such that
	\begin{enumerate}
		\item the dimension is \(\leq d\), then \(H_n(X) = 0\) for all \(n>d\).
		\item \(\overline{w} \) no cells of dimension \(p\), then \(H_p(X) = 0\).
		\item \(\overline{w} \) no cells of dimension \(p\), then \(H_{p-1}(X)\) is \hyperref[def:free-Abelian-group]{free Abelian}.
	\end{enumerate}
\end{corollary}

\begin{corollary}
	Given a \hyperref[def:singular-homology-group]{singular homology} class on \(X\), without loss of generality we can choose a
	\hyperref[def:delta-complex]{\(\Delta \)-complex} structure on \(X\), and we then we can assume the class is represented by a
	\hyperref[def:simplicial-complex]{simplicial $n$-cycle}.
\end{corollary}
\subsection{Degree}
\begin{definition}[Degree]\label{def:degree}
	Let $f \colon S^n \to S^n$. Then $f_\ast \colon \mathbb{\MakeUppercase{z}} \cong H_n(S^n) \to H_n(S^n) \cong \mathbb{\MakeUppercase{z}} $.
	From group theory, this map must be multiplication by some integer $d \in \mathbb{\MakeUppercase{z}} $, which we call the \emph{degree} $\deg(f)$ of $f$
\end{definition}
