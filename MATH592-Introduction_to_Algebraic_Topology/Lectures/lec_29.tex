\lecture{29}{21 Mar. 10:00}{Proof of \autoref{thm:singular-homology-agrees-with-simplicial-homology}}
\begin{proof}[Proof Sketch]
	The idea is as follows.
	\begin{itemize}
		\item We can use the long \hyperref[def:exact-sequence]{exact sequence} of a pair and the \autoref{lma:the-five-lemma} to
		      reduce to proving the result for \hyperref[def:homology-group]{absolute homology groups} (and we will recover the general result).
		\item Because the image \(\Delta^n \to X\) is \emph{compact}, it is contained in some finite skeleton \(X^k\). Use this
		      to reduce the proof to the finite skeleton \(X^k\) of \(X\), namely we can use induction.
	\end{itemize}
	From the long \hyperref[def:exact-sequence]{exact sequence} of a \hyperref[def:good-pair]{pair} we get
	\par
	\adjustbox{scale=0.9,center}{%
		\begin{tikzcd}[column sep=small]
			{H^\Delta_{n+1}(X^k, X^{k-1})} & {H^\Delta_{n}(X^{k-1})} & {H^\Delta_{n}(X^{k})} & {H^\Delta_{n}(X^k, X^{k-1})} & {H^\Delta_{n-1}(X^{k-1})} \\
			{H_{n+1}(X^k, X^{k-1})} & {H_{n}(X^{k-1})} & {H_{n}(X^{k})} & {H_{n}(X^k, X^{k-1})} & {H_{n-1}(X^{k-1})}
			\arrow["\alpha", color={rgb,255:red,214;green,92;blue,92}, from=1-1, to=2-1]
			\arrow["\beta", color={rgb,255:red,92;green,92;blue,214}, from=1-2, to=2-2]
			\arrow["\gamma", color={rgb,255:red,124;green,194;blue,112}, from=1-3, to=2-3]
			\arrow["\delta", color={rgb,255:red,214;green,92;blue,92}, from=1-4, to=2-4]
			\arrow["\epsilon", color={rgb,255:red,92;green,92;blue,214}, from=1-5, to=2-5]
			\arrow[from=1-1, to=1-2]
			\arrow[from=1-2, to=1-3]
			\arrow[from=1-3, to=1-4]
			\arrow[from=1-4, to=1-5]
			\arrow[from=2-1, to=2-2]
			\arrow[from=2-2, to=2-3]
			\arrow[from=2-3, to=2-4]
			\arrow[from=2-4, to=2-5]
		\end{tikzcd}
	}

	The Goal is to prove \(\gamma\) is an isomorphism using the \autoref{lma:the-five-lemma}.

	We assume that \textcolor{blue}{\(\beta, \epsilon \)} are isomorphisms by induction, checking the case manually for \(X^0\)
	(which will be a discrete set of points). It remains to show that \textcolor{red}{\(\alpha, \delta\)} are isomorphisms.

	We know then that
	\[
		\Delta_n(X^k, X^{k - 1})  = \begin{dcases}
			\mathbb{\MakeUppercase{z}} [\text{\hyperref[def:singular-simplex]{\(k\)-simplices}}], & \text{ if } k=n ; \\
			0,                                                                                    & \text{ otherwise}
		\end{dcases}
		\cong H_n^\Delta(X^k, X^{k - 1}).
	\]

	We claim that \(H_n(X^k, X^{k - 1})\) are also \hyperref[def:free-Abelian-group]{free Abelian} on the
	\hyperref[def:singular-simplex]{singular \(k\)-simplices} defined by the characteristic maps \(\Delta^k \to X^k\) when \(n = k\),
	and \(0\) otherwise. Consider the map
	\[
		\Phi \colon \coprod_\alpha (\Delta^k_\alpha, \partial \Delta^k_\alpha) \to (X^k, X^{k - 1})
	\]
	defined by the characteristic map. This induces an isomorphism on \hyperref[def:homology-group]{homology} since
	\[
		\quotient{\coprod_\alpha \Delta_\alpha^k}{\coprod \partial \Delta_\alpha^k} \overset{\cong}{\longrightarrow}  \quotient{X^k}{X^{k - 1}}.
	\]

	This reduces to check that
	\[
		H_n(\Delta^k, \partial \Delta^k) = \begin{dcases}
			0,                           & \text{ if } n \neq k ; \\
			\mathbb{\MakeUppercase{z}} , & \text{ if } n=k
		\end{dcases}
	\]
	generated by the identity map \(\Delta^k \to \Delta^k\).
\end{proof}

\begin{corollary}
	If \(X\) has a \hyperref[def:delta-complex]{\(\Delta \)-complex} structure (or is
	\hyperref[def:homotopy-equivalence]{homotopy equivalent} to one), then we have the followings.
	\begin{enumerate}
		\item If the dimension is \(\leq d\), then \(H_n(X) = 0\) for all \(n>d\).
		\item If \(\overline{X} \) has no cells of dimension \(p\), then \(H_p(X) = 0\).
		\item If \(\overline{X} \) has no cells of dimension \(p\), then \(H_{p-1}(X)\) is \hyperref[def:free-Abelian-group]{free Abelian}.
	\end{enumerate}
\end{corollary}

\begin{corollary}
	Given a \hyperref[def:singular-homology-group]{singular} \hyperref[def:homology-class]{homology class} on \(X\), without loss of generality we can choose a
	\hyperref[def:delta-complex]{\(\Delta \)-complex} structure on \(X\), and we then we can assume the \hyperref[def:homology-class]{class} is represented by a
	\hyperref[def:simplicial-complex]{simplicial} \hyperref[def:cycle]{\(n\)-cycle}.
\end{corollary}

\subsection{Degree}
\begin{definition}[Degree]\label{def:degree}
	Let \(f \colon S^n \to S^n\), then
	\[
		f_\ast \colon \mathbb{\MakeUppercase{z}} \cong H_n(S^n) \to H_n(S^n) \cong \mathbb{\MakeUppercase{z}}.
	\]

	From group theory, this map must be multiplication by some integer \(d \in \mathbb{\MakeUppercase{z}}\), which we call it as the
	\emph{degree}, denotes as \(\deg(f)\) of \(f\).
\end{definition}

\begin{remark}[Properties of Degree]\label{rmk:property-of-degree}
	We first see some properties of \hyperref[def:degree]{degree}.
	\begin{enumerate}
		\item \(\deg(\identity_{S^n}) = 1\) since \((\identity_{S_n})_\ast = \identity_{\mathbb{Z}}\).
		\item If \(f \colon S^n \to S^n\), \(n n\geq 0\) is not surjective, then \(\deg(f) = 0\). To see this, we know that \(f_\ast\) factors as
		      \[
			      \begin{tikzcd}
				      {H_n(S^n)} & {H_n(S^n-\{\ast\})=0} & {H_n(S^n)}
				      \arrow[from=1-1, to=1-2]
				      \arrow[from=1-2, to=1-3]
				      \arrow["f_\ast", curve={height=18pt}, from=1-1, to=1-3]
			      \end{tikzcd}
		      \]
		      And since the middle group is zero, \(f_\ast = 0\).
		\item If \(f \simeq g\), then \(f_\ast = g_\ast\), so \(\deg(f) = \deg(g)\).
		      \begin{note}
			      The converse is true! We'll see this later.
		      \end{note}
		\item \((f \circ g)_\ast = f_\ast \circ g_\ast\), and so \(\deg(f \circ g) = \deg(f)\deg(g)\).

		      \par Consequently, if \(f\) is a \hyperref[def:homotopy-equivalence]{homotopy equivalence} then \(\deg f = \pm 1\).

		      \begin{exercise}
			      It is possible to put a \hyperref[def:delta-complex]{\(\Delta\)-complex} structure with \(2\) \(n\)-cells, \(\Delta_1\) and \(\Delta_2\) glued
			      together along their \hyperref[def:boundary]{boundary} \((\cong S^{n-1})\), and
			      \[
				      H_n(S^n) = \langle \Delta_1 - \Delta_2 \rangle .
			      \]
		      \end{exercise}
		      If \(f\) is a reflection fixing the equator, and swapping the \(2\)-cells, then \(\deg f = -1\).
		      \begin{figure}[H]
			      \centering
			      \incfig{reflection-about-equator}
			      \label{fig:reflection-about-equator}
		      \end{figure}
		\item We now have the following linear algebra exercise.
		      \begin{exercise}
			      The map \(S^{n + 1} \to S^{n + 1}\) given by \(x \mapsto -x\) is the composite of \((n + 1)\) reflections.
		      \end{exercise}
		      So the antipodal map \(S^n \to S^n\) given by \(x \mapsto -x\) has degree which is the product of \(n + 1\) copies of \((-1)\), and so it has
		      \hyperref[def:degree]{degree} \((-1)^{n + 1}\). (i.e., since the \((n+1)\times (n+1)\) scalar matrix \((-1)\) is composition of \((n+1)\) reflections.)
		\item We see the following.
		      \begin{exercise}
			      If \(f\colon S^n \to S^n\) has no fixed points, then we can homotope \(f\) to the antipodal map via
			      \[
				      f_t(x) = \frac{(1 - t)f(x) - tx}{\left\lVert (1 - t)f(x) - tx\right\rVert}.
			      \]
		      \end{exercise}

		      Therefore, \(\deg f = (-1)^{n + 1}\).
	\end{enumerate}
\end{remark}