\lecture{16}{11 Feb. 10:00}{Proving \autoref{prop:lifting-criterion}}
We now continue our proof of \autoref{prop:lifting-criterion}.
\begin{proof}[\unskip\nopunct]
	\begin{enumerate}
		\item[2.] \(\widetilde{f} \) is continuous. Choose \(x\in X\) and a neighborhood \(\widetilde{U} \) of
			\(\widetilde{f} (x)\) in \(\widetilde{Y} \). Note that we can choose \(\widetilde{U} \) small enough to \(\at{p}{\widetilde{U} }{} \) is homeomorphism
			to \(U\) in \(Y\). Now, there exists a neighborhood \(V\) of \(x\) in \(X\) with \(f(V)\subseteq U\).
			\begin{figure}[H]
				\centering
				\incfig{pf:prop:lifting-criterion-2}
				\label{fig:pf:prop:lifting-criterion-2}
			\end{figure}
			The goal is \(\widetilde{f} (V)\subseteq \widetilde{U}\). Without loss of generality, we can assume that
			\(V\) is \hyperref[def:path]{path}-connected. Then,
			\[
				\widetilde{f \gamma } \cdot \widetilde{f \alpha } = \widetilde{\left[f \gamma \cdot f \alpha \right]}.
			\]
			Hence,
			\[
				\widetilde{f \alpha } = (\at{p}{\widetilde{U} }{})^{-1} \circ f\circ \alpha,
			\]
			where \((\at{p}{\widetilde{U} }{})^{-1}\)'s image is in \(\widetilde{U} \), so
			\[
				\widetilde{f} (x ^\prime ) = f \gamma \cdot f \alpha (1)\in \widetilde{U},
			\]
			which implies
			\[
				\widetilde{f} (V)\subseteq \widetilde{U}.
			\]
	\end{enumerate}
\end{proof}

\begin{proposition}[Uniqueness of lifts]
	Let \(p\colon \widetilde{Y} \to Y\) be a \hyperref[def:covering-map]{covering map} with \(X\) is a connected space. If two \hyperref[prop:homotopy-lifting-property]{lifts}
	\(\widetilde{f} _1, \widetilde{f} _2\) of the same map \(f\) agree at a single point, then they agree everywhere.
	\[
		\begin{tikzcd}
			&& {\widetilde{Y}} \\
			\\
			X && Y
			\arrow["p", from=1-3, to=3-3]
			\arrow["{\widetilde{f}_2}"', curve={height=9pt}, from=3-1, to=1-3]
			\arrow["{\widetilde{f}_1}", curve={height=-9pt}, from=3-1, to=1-3]
			\arrow["f"', from=3-1, to=3-3]
		\end{tikzcd}
	\]
\end{proposition}
\begin{proof}
	Let \(S\) being
	\[
		S \coloneqq \left\{x\in X  \mid \widetilde{f}_1(x) = \widetilde{f}_2(x) \right\}.
	\]
	We want to show that \(S\) is both closed and open, so if \(S\) is nonempty, \(S = X\).
	\begin{figure}[H]
		\centering
		\incfig{pf:lec16:prop:1}
		\label{fig:pf:lec16:prop:1}
	\end{figure}
	We see that \(\widetilde{U} _1\) and \(\widetilde{U} _2\) are slices of \(p^{-1} (U)\), where \(U\) is an \hyperref[def:evenly-covered]{evenly covered}
	neighborhood of \(f(x)\).
	\begin{enumerate}
		\item If \(\widetilde{f} _1(x)\neq \widetilde{f} _2(x)\). Then \(\widetilde{U} _1, \widetilde{U} _2\) are disjoint. Since \(\widetilde{f} _1, \widetilde{f} _2\)
		      are continuous, there exists a neighborhood \(N\) of \(x\) with
		      \[
			      \widetilde{f} _1(N)\subseteq \widetilde{U} _1,\quad \widetilde{f} _2(N)\subseteq \widetilde{U} _2,
		      \]
		      with the fact that they're disjoint, so \(x\) is an interior point of \(S^c\).
		\item If \(\widetilde{f} _1(x) = \widetilde{f} _2(x)\). Then \(\widetilde{U} _1 = \widetilde{U} _2\). Choose \(N\) as before, then we have
		      \[
			      \widetilde{f} _1(n) = (\at{p}{\widetilde{u} _1}{} )^{-1} \left(f(n)\right) = \widetilde{f} _2(n),
		      \]
		      hence \(x\in \mathrm{int}(S) \).
	\end{enumerate}
\end{proof}

\section{Deck Transformation}
We now want to introduce a special kind of transformation.

\begin{definition}[Isomorphism of covers]\label{def:isomorphism-of-covers}
	Given \hyperref[def:covering-map]{covering maps}
	\[
		p_1\colon \widetilde{X} _1\to X,\qquad p_2\colon \widetilde{X} _2\to X,
	\]
	an \emph{isomorphism of covers} is a homeomorphism \(f\colon \widetilde{X} _1\to \widetilde{X} _2\) such that \(p_1 = p_2\circ f\).
	\[
		\begin{tikzcd}
			{\widetilde{X}_1} && {\widetilde{X}_2} \\
			& X
			\arrow["{p_1}"', from=1-1, to=2-2]
			\arrow["{p_2}", from=1-3, to=2-2]
			\arrow["f", from=1-1, to=1-3]
		\end{tikzcd}
	\]
\end{definition}

\begin{exercise}
	This defines equivalent relation on \hyperref[def:isomorphism-of-covers]{covers} of \(X\).
\end{exercise}

\begin{definition}[Deck transformation]\label{def:deck-transformation}
	Given a \hyperref[def:covering-map]{covering map} \(p\colon \widetilde{X} \to X\), the  \hyperref[def:isomorphism-of-covers]{isomorphisms of covers}
	\(\widetilde{X} \to \widetilde{X} \) are called \emph{deck transformation}.
\end{definition}

\begin{definition}[Set of deck transformation]\label{def:set-of-deck-transformation}
	We let \(G(\widetilde{X} )\) denotes the \emph{set of \hyperref[def:deck-transformation]{deck transformations}}.
\end{definition}
\begin{note}
	Note that we've suppressed the data of \(p\) in the notation, but this data is essential to what a \hyperref[def:deck-transformation]{deck transformation}
	is, when this is unclear we write \(G(\widetilde{X} , p)\).
\end{note}