\lecture{16}{11 Feb. 10:00}{Proving \autoref{prop:lifting-criterion}}
\begin{proof}[\unskip\nopunct]
	\begin{enumerate}
		\item[2.] \(\widetilde{f} \) is continuous. Choose \(x\in X\) and a neighborhood \(\widetilde{U} \) of
			\(\widetilde{f} (x)\). Note that we can choose \(\widetilde{U} \) small enough to \(\at{p}{\widetilde{U} }{} \) is homeomorphism
			to \(U\) in \(Y\). Now, there exists a neighborhood \(V\) of \(x\) in \(X\) with \(f(V)\subseteq U\).
			\begin{figure}[H]
				\centering
				\incfig{pf:prop:lifting-criterion-2}
				\label{fig:pf:prop:lifting-criterion-2}
			\end{figure}
			The goal is \(\widetilde{f} (V)\subseteq \widetilde{U}\). Without loss of generality, we can assume that
			\(V\) is \hyperref[def:path]{path}-connected. Then,
			\[
				\widetilde{f \gamma } \cdot \widetilde{f \alpha } = \widetilde{\left[f \gamma \cdot f \alpha \right]}.
			\]
			Hence,
			\[
				\widetilde{f \alpha } = (\at{p}{\widetilde{U} }{})^{-1} \circ f\circ \alpha,
			\]
			where \((\at{p}{\widetilde{U} }{})^{-1}\)'s image is in \(\widetilde{U} \), so
			\[
				\widetilde{f} (x ^\prime ) = f \gamma \cdot f \alpha (1)\in \widetilde{u}.
			\]
	\end{enumerate}
\end{proof}

\begin{proposition}
	Given
	\begin{itemize}
		\item \(p\colon \widetilde{Y} \to Y\)
		\item \(f\colon X\to Y\)
		\item \(X\) is connected.
	\end{itemize}
	Given \hyperref[prop:homotopy-lifting-property]{lifts} \(\widetilde{f} _1, \widetilde{f} _2\colon X\to \widetilde{Y} \),
	if they agree at \(1\) point \(x\in X\), they are equal.
\end{proposition}
\begin{proof}
	Let \(S\) being
	\[
		S \coloneqq \left\{x\in X  \mid \widetilde{f}_1(x) = \widetilde{f}_2(x) \right\}.
	\]
	We want to show that \(S\) is both closed and open, so if \(S\) is nonempty, \(S = X\).
	\begin{figure}[H]
		\centering
		\incfig{pf:lec16:prop:1}
		\label{fig:pf:lec16:prop:1}
	\end{figure}
	We see that \(\widetilde{U} _1\) and \(\widetilde{U} _2\) are slices of \(p^{-1} (U)\), where \(U\) is evenly covered neighborhood of \(f(x)\).
	\begin{enumerate}
		\item If \(\widetilde{f} _1(x)\neq \widetilde{f} _2(x)\). Then \(\widetilde{U} _1, \widetilde{U} _2\) are disjoint. Since \(\widetilde{f} _1, \widetilde{f} _2\)
		      are continuous, there exists a neighborhood \(N\) of \(x\) with
		      \[
			      \widetilde{f} _1(N)\subseteq \widetilde{U} _1,\quad \widetilde{f} _2(N)\subseteq \widetilde{U} _2,
		      \]
		      with the fact that they're disjoint, so \(x\) is an interior point of \(S^c\).
		\item If \(\widetilde{f} _1(x) = \widetilde{f} _2(x)\). Then \(\widetilde{U} _1 = \widetilde{U} _2\). Choose \(N\) as before, then we have
		      \[
			      \widetilde{f} _1(n) = (\at{p}{\widetilde{u} _1}{} )^{-1} \left(f(n)\right) = \widetilde{f} _2(n),
		      \]
		      hence \(x\in \mathrm{int}(S) \).
	\end{enumerate}
\end{proof}

\subsection{Morphisms of Covers}
\begin{definition}[Covers]\label{def:covers}
	An \emph{isomorphism of covers} is a homeomorphism
	\[
		f\colon \widetilde{X} _1\to \widetilde{X} _2
	\]
	so \(p_1 = p_2\circ f\).
	\[\begin{tikzcd}
			{\widetilde{X}_1} && {\widetilde{X}_2} \\
			& X
			\arrow["{p_1}"', from=1-1, to=2-2]
			\arrow["{p_2}", from=1-3, to=2-2]
			\arrow["f", from=1-1, to=1-3]
		\end{tikzcd}\]
\end{definition}

\begin{exercise}
	Defines equivalent relation on \hyperref[def:covers]{covers} of \(X\).
\end{exercise}

\begin{definition}
	Given a cover \(p\colon \widetilde{X} \to X\), the isomorphism of covers
	\(\widetilde{X} \to \widetilde{X} \) are called Deck transformation.
\end{definition}