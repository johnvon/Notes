\lecture{31}{25 Mar. 10:00}{Local Degree and Local Homology}
\begin{eg}
	Consider the composition of the quotient maps below
	\[
		\begin{tikzcd}
			{S^n} & {\mathbb{R}P^n} & {\quotient{\mathbb{R}P^n}{\mathbb{R}P^{n-1}}\cong S^n}
			\arrow[from=1-1, to=1-2]
			\arrow[from=1-2, to=1-3]
			\arrow["f", curve={height=20pt}, from=1-1, to=1-3]
		\end{tikzcd}
	\]
	We want to compute the \hyperref[def:degree]{degree} of this map.
	\begin{figure}[H]
		\centering
		\incfig{eg:real-projective-space-degree}
		\label{fig:eg:real-projective-space-degree}
	\end{figure}
	Note that this restricts to a homeomorphism on each component of \(S^n \setminus \text{equator}\) as a map to
	\(\quotient{\mathbb{\MakeUppercase{r}}P^n}{\mathbb{\MakeUppercase{r}} P^{n - 1}}\).
	Suppose we've oriented our copies of \(S^n\) in such a way that the homeomorphism on the top hemisphere is orientation-preserving. The homeomorphism on the bottom hemisphere
	is given by taking the antipodal map and composing with the homeomorphism of the top hemisphere
	\[
		\deg = \deg(\identity) = \deg(\text{antipodal}) = 1 + (-1)^{n + 1} = \begin{dcases}
			0, & \text{ if }  n \text{ even}; \\
			2, & \text{ if }  n \text{ odd}.
		\end{dcases}
	\]
\end{eg}

We can now prove \autoref{thm:calculation-with-local-degree}.
\begin{proof}
	If \(f\colon S^n \to S^n\) and we have some \(y\in S^n \) with \(f^{-1} (\{y\})= \{x_1, \ldots , x_m \}\), then
	we have a nice commutative diagram as follows.
	\[
		\begin{tikzcd}
			{H_n(S^n)} & {H_n(Srn)} \\
			{\bigoplus\limits_{i=1}^m\mathbb{Z}=H_n(S^n, S^n-\{x_1, \ldots, x_m\})} \\
			{H_n\left(\coprod\limits_{i=1}^m U_i, \coprod\limits_{i=1}^m(U_i - \{x_i\})\right)} & {H_n(S^n, S^n-\{y\})} \\
			{\bigoplus\limits_i H_n\left(U_i, U_i - \{x_i\}\right)} & {H_n(V, V-\{y\})} & \bullet \\
			1 & {\deg(f_\ast)} \\
			\\
			\\
			{(1, 1, \ldots, 1)} & {\deg f = \sum \deg \at{f}{x_i}{}}
			\arrow["{f_\ast}", from=1-1, to=1-2]
			\arrow["\cong"', from=1-2, to=3-2]
			\arrow["\cong", from=4-2, to=3-2]
			\arrow["{\text{\hyperref[thm:long-exact-sequence-of-a-pair]{LES of pair}}}"', from=1-1, to=2-1]
			\arrow["\cong"', from=3-1, to=2-1]
			\arrow["\cong"', from=4-1, to=3-1]
			\arrow[from=4-1, to=4-2]
			\arrow[maps to, from=5-1, to=8-1]
			\arrow[maps to, from=8-1, to=8-2]
			\arrow[maps to, from=5-1, to=5-2]
			\arrow["{\text{\hyperref[thm:long-exact-sequence-of-a-pair]{LES of a pair}}}", draw=none, from=1-2, to=3-2]
			\arrow["{\text{\hyperref[thm:excision]{excision}}}"', draw=none, from=4-2, to=3-2]
			\arrow["{\text{\hyperref[def:homology-group]{homology} of disjoint union}}", draw=none, from=4-1, to=3-1]
			\arrow["{\text{\hyperref[thm:excision]{excision}}}", draw=none, from=3-1, to=2-1]
			\arrow[maps to, from=5-2, to=8-2]
		\end{tikzcd}
	\]
	where we trace around the outside of the diagram at the bottom, which just proves the result.
\end{proof}

\hr

Let's grab some intuition. \emph{What really is local homology}?

By \hyperref[thm:excision]{excision}, there is an isomorphism \(H_n(S^n, S^n \setminus \{x_i\}) \cong H_n(U, U \setminus \{x_i\})\) for any open
neighborhood \(U\) of \(x_i\).

The long \hyperref[def:exact-sequence]{exact sequence} of a \hyperref[def:good-pair]{pair} also gives us
\par
\adjustbox{scale=0.9,center}{%
	\begin{tikzcd}[column sep=small]
		\ldots & {H_k(S^n\setminus\{x_i\})} & {H_k(S^n)} & {H_k(S^n, S^n\setminus\{x_i\})} & {H_{k-1}(S^n\setminus\{x_i\})} & \ldots
		\arrow[from=1-1, to=1-2]
		\arrow[from=1-2, to=1-3]
		\arrow[from=1-3, to=1-4]
		\arrow[from=1-4, to=1-5]
		\arrow[from=1-5, to=1-6]
	\end{tikzcd}
}

\par Since \(S^n \setminus \{x_i\}\) is homeomorphic to an open \(n\)-ball, we see that \(H_k(S^n \setminus \{x_i\}) = H_{k - 1}(S^n \setminus \{x_i\}) = 0\).
With this in mind, \(j_\ast\) is an isomorphism.

We want to think about what \(j_\ast\) does when \(k = n\), i.e., when this is an isomorphism \(\mathbb{\MakeUppercase{z}}\cong H_n(S^n) \to H_n(S^n, S^n \setminus \{x_i\}) \cong \mathbb{Z}\).

We see that \(\Delta_1 - \Delta_2\) generate \(H_n(S^n)\), where \(\Delta_1, \Delta_2\) are the top and bottom hemisphere indicated below.
\begin{figure}[H]
	\centering
	\incfig{les-on-relative-spheres}
	\label{fig:les-on-relative-spheres}
\end{figure}
We then understand that \(j_\ast(\Delta_1 - \Delta_2) = \Delta_1 - \Delta_2 = \Delta_1\) since \(\Delta_2 = 0\) in \(C_n(S^n)/C_n(S^n \setminus \{x_i\})\).

The upshot is that \(H_n(S^n, S^n \setminus \{x\})\) is generated by an \hyperref[def:standard-simplex]{\(n\)-simplex} with \(x\) in its interior.

Suppose \(M\) is an \(n\)-manifold. Then \(H_n(M, M \setminus \{x\}) \cong H_n(U, U \setminus \{x\})\), where \(U\) is a small ball around \(x\).
Because \(U\) is a ball homeomrphic to \(\mathbb{\MakeUppercase{r}} ^n\), we see that
\[
	H_n(M, M \setminus \{x\}) \cong H_n(U, U \setminus \{x\}) \cong H_n(\mathbb{\MakeUppercase{r}} ^n, \mathbb{\MakeUppercase{r}} ^n \setminus \{x\}).
\]
By the long \hyperref[def:exact-sequence]{exact sequence} of a \hyperref[def:good-pair]{pair}
\[
	\begin{tikzcd}[column sep=small]
		{0=H_n(\mathbb{R}^n)} & {H_n(\mathbb{R}^n, \mathbb{R}^n\setminus\{x\})} & {H_{n-1}(\mathbb{R}^n\setminus\{x\})} & {H_{n-1}(\mathbb{R}^n)=0}
		\arrow[from=1-1, to=1-2]
		\arrow[from=1-2, to=1-3]
		\arrow[from=1-3, to=1-4]
	\end{tikzcd}
\]
And since \(\mathbb{\MakeUppercase{r}}^n \setminus \{x\}\) is \hyperref[def:homotopy-equivalence]{homotopy equivalent} to an \(n - 1\) sphere, this means that
\(H_n(\mathbb{\MakeUppercase{r}} ^n, \mathbb{\MakeUppercase{r}} ^n \setminus \{x\}) \cong \mathbb{\MakeUppercase{z}} \). By homework, this
connecting homomorphism is given by taking the \hyperref[def:boundary]{boundary} of a \hyperref[def:relative-cycle]{relative cycle} as below.
\begin{figure}[H]
	\centering
	\incfig{connecting-homomorphism-relative-homology-rn}
	\label{fig:connecting-homomorphism-relative-homology-rn}
\end{figure}

We intuitively want to use this idea to compute \hyperref[def:degree]{degree} using this idea. We use naturality of the long \hyperref[def:exact-sequence]{exact sequence},
namely the fact that where \(f \colon (U_i, U_i \setminus \{x_i\}) \to (V, y)\) is a map of \hyperref[def:good-pair]{pairs}, then the following diagram commutes.
\[
	\begin{tikzcd}
		\ldots & {H_n(U_i, U_i\setminus\{x_i\})} & {H_{n-1}(U_i, U_i\setminus\{x_i\})} & \ldots \\
		\ldots & {H_n(V, V\setminus\{y\})} & {H_{n-1}(V, V\setminus\{y\})} & \ldots
		\arrow[from=1-1, to=1-2]
		\arrow[from=1-2, to=1-3]
		\arrow[from=1-3, to=1-4]
		\arrow["{f_\ast}", from=1-3, to=2-3]
		\arrow[from=2-3, to=2-4]
		\arrow[from=2-2, to=2-3]
		\arrow["{f_\ast}", from=1-2, to=2-2]
		\arrow[from=2-1, to=2-2]
	\end{tikzcd}
\]
By naturality of the long \hyperref[def:exact-sequence]{exact sequence} and the isomorphism discussed above, we can compute the \hyperref[def:local-degree]{local degree}
of a map \(S^n \to S^n\) at a point \(x\) by computing the \hyperref[def:degree]{degree} of the map
\[
	\begin{tikzcd}
		{H_{n-1}(U\setminus\{x\})} & {H_{n-1}(V-\{y\})}
		\arrow[from=1-1, to=1-2]
	\end{tikzcd}
\]

In fact the \hyperref[def:local-degree]{local degree} will be the \hyperref[def:degree]{degree} restricted to a small \(S^{n - 1}\) n the neighborhood \(U\).
\begin{figure}[H]
	\centering
	\incfig{computing-local-homology-idea}
	\label{fig:computing-local-homology-idea}
\end{figure}

\begin{eg}
	Consider
	\[
		\begin{split}
			\hat{\mathbb{\MakeUppercase{c}}} & \to \hat{\mathbb{\MakeUppercase{c}}}\\
			z&\mapsto z^n.
		\end{split}
	\]

	\begin{figure}[H]
		\centering
		\incfig{eg:lec31-degree-n}
		\label{fig:eg:lec31-degree-n}
	\end{figure}

	We see that
	\[
		\deg \at{f}{0}{} = n.
	\]
\end{eg}