\lecture{31}{25 Mar. 10:00}{Cellular Homology}
\begin{eg}
	Consider the composition of the quotient maps below \(S^n \to \mathbb{\MakeUppercase{r}} P^n \to \quotient{\mathbb{\MakeUppercase{r}} P^n}{\mathbb{\MakeUppercase{r}} P^{n - 1}} \cong S^n\).
	We want to compute the \hyperref[def:degree]{degree} of this map.

	Note that this restricts to a homeomorphism on each component of \(S^n \setminus \text{equator}\) as a map to \(\mathbb{\MakeUppercase{r}}P^n \setminus \mathbb{\MakeUppercase{r}} P^{n - 1}\).
	Suppose we've oriented our copies of \(S^n\) in such a way that the homeomorphism on the top hemisphere is orientation-preserving. The homeomorphism on the bottom hemisphere
	is given by taking the antipodal map and composing with the homeomorphism of the top hemisphere
	\[
		\deg = \deg(\identity) = \deg(\text{antipodal}) = 1 + (-1)^{n + 1} = \begin{dcases}
			0, & \text{ if }  n \text{ even}; \\
			2, & \text{ if }  n \text{ odd}.
		\end{dcases}
	\]
\end{eg}

\subsection{Cellular Homology}
Suppose that \(X\) is a \hyperref[def:CW-Complex]{CW complex}. Then \((X^n, X^{n - 1})\) is a \hyperref[def:good-pair]{good pair} for all \(n > 1\), and
\(\quotient{X^n}{X^{n - 1}}\) is a \hyperref[sssec:Wedge-sum]{wedge of \(n\)-spheres}, one for each \(n\)-cell \(e^n_\alpha\). Hence,
\[
	H_k(X^n, X^{n - 1}) \cong \begin{dcases}
		0,                                                                    & \text{ if } k\neq n ; \\
		\langle e_\alpha^n \mid  e_\alpha^n \text{ is an \(n\)-cell} \rangle, & \text{ if } k = n .
	\end{dcases}
\]

\begin{definition}[Cellular chain complex]\label{def:cellular-chain-complex}
	The \emph{cellular chain complex} of \(X\) has chain groups \(H_n(X^n, X^{n - 1})\) with \(X^{-1} = \varnothing\), and the boundary maps at \(n=0\) is given as
	\[
		\begin{split}
			d_1 : H_1(X^1, X^0)            & \to H_0(X^0)                       \\
			\langle \text{1-cells} \rangle & \to \langle \text{0-cells} \rangle,
		\end{split}
	\]
	which is the usual \hyperref[def:boundary-homomorphism]{simplicial boundary map}. For \(n > 1\), the boundary map \(d_{n}\) are defined as
	\[
		d_n(e_\alpha^n) = \sum_\beta d_{\alpha\beta} e_\beta^{n - 1}
	\]
	where \(d_{\alpha\beta}\) is the \hyperref[def:degree]{degree} of the map
	\[
		\begin{tikzcd}
			{\partial e^n_\alpha=S^{n-1}_{\alpha}} &&&& {X^{n-1}} &&&& {S^{n-1}_\beta}
			\arrow["{\text{attaching map}}", from=1-1, to=1-5]
			\arrow["{\text{quotient by \(\quotient{X^{n-1}}{e^{n-1}_\beta}\)}}", from=1-5, to=1-9]
		\end{tikzcd}
	\]
	In pictures, this is given as the following.
	\begin{figure}[H]
		\centering
		\incfig{cellular-boundary-map}
		\label{fig:cellular-boundary-map}
	\end{figure}
\end{definition}

\begin{definition}[Cellular homology group]\label{def:cellular-homology-group}
	We define the so-called \emph{cellular homology group} by \hyperref[def:cellular-chain-complex]{cellular chain complex} in our usual way of defining
	\hyperref[def:homology-group]{homology group}.
\end{definition}

\begin{theorem}\label{thm:cellular-homology-coincide}
	The \hyperref[def:cellular-homology-group]{cellular homology groups} coincide with the \hyperref[def:singular-homology-group]{singular homology groups}.
\end{theorem}

\autoref{thm:cellular-homology-coincide} implies the following.
\begin{corollary}
	We have the followings.
	\begin{itemize}
		\item \(H_n(X) = 0\) if \(X\) has a \hyperref[def:CW-Complex]{CW complex} structure with no \(n\)-cells.
		\item If \(X\) has a \hyperref[def:CW-Complex]{CW complex} with \(k\) \(n\)-cells, then \(H_n(X)\) is generated by at most \(k\) elements.
		\item If \(H_n(X)\) is a group with a minimum of \(k\) generators, then any \hyperref[def:CW-Complex]{CW complex} structure on \(X\) must have at least \(k\) \(n\)-cells.
		\item If \(X\) has a \hyperref[def:CW-Complex]{CW complex} with no cells in consecutive dimensions, then its homology is
		      \hyperref[def:free-Abelian-group]{free Abelian} on its \(n\)-cells. For example \(S^n, n \geq 2\) or \(\mathbb{\MakeUppercase{c}} P^n\).
	\end{itemize}
\end{corollary}

\begin{eg}
	\(S^n\) with \(n \geq 2\), using the \hyperref[def:CW-Complex]{CW complex} structure of \(e^n\) attached to a single point \(x_0\). The \hyperref[def:cellular-chain-complex]{cellular chain complex} is given as
	\[
		\begin{tikzcd}
			0 & 0 & {\left<e^n\right>} & 0 & \ldots & 0 & {\left<x_0\right>}
			\arrow[from=1-1, to=1-2]
			\arrow[from=1-2, to=1-3]
			\arrow[from=1-3, to=1-4]
			\arrow[from=1-4, to=1-5]
			\arrow[from=1-5, to=1-6]
			\arrow[from=1-6, to=1-7]
		\end{tikzcd}
	\]
	So then all the boundary maps are zero, and we see that
	\[
		H_k(S^n) = \begin{dcases}
			\mathbb{\MakeUppercase{z}}, & \text{ if } k=0, n ; \\
			0,                          & \text{ otherwise}.
		\end{dcases}
	\]
\end{eg}

\begin{exercise}
	Redo this calculation with other \hyperref[def:CW-Complex]{CW complex} structure on \(S^n\), e.g. glue \(2\) \(n\)-cells onto \(S^{n - 1}\) and proceed inductively.
\end{exercise}

\begin{eg}
	Let's do this with the torus
	\begin{figure}[H]
		\centering
		\incfig{eg:lec31:CW-complex-torus}
		\label{fig:eg:lec31:CW-complex-torus}
	\end{figure}

	The \hyperref[def:cellular-chain-complex]{chain complex} looks like
	\[
		\begin{tikzcd}
			0 & {\left<D\right>} & {\left<a, b\right>} & {\left<x\right>} & 0
			\arrow[from=1-1, to=1-2]
			\arrow[from=1-2, to=1-3]
			\arrow[from=1-3, to=1-4]
			\arrow[from=1-4, to=1-5]
		\end{tikzcd}
	\]
	Note that \(a \mapsto x - x = 0\) and \(b \mapsto x - x = 0\) and so \(\partial_1 = 0\). Now \(D\) is glued along \(aba^{-1}b^{-1}\), so we look at the composed up map

	\begin{figure}[H]
		\centering
		\incfig{eg:cellular-homology-calc-torus}
		\label{fig:eg:cellular-homology-calc-torus}
	\end{figure}
	We wind forwards then backwards around $a$, so the \hyperref[def:degree]{degree} is zero. The same thing happens for $b$ so
	\[
		\partial_2 D = 0 \cdot a + 0 \cdot b = 0.
	\]

	This gives a nice \textbf{principle}, namely if a \(2\)-cell \(D\) is glued down via some \hyperref[def:word]{words} \(w\) (this only makes sense for \(2\)-cells), then the
	coefficient to a letter \(b\) in \(\partial_2 D\) is the sum of the exponents of \(b\) in \(w\).

	Now we just have that the \hyperref[def:cellular-homology-group]{homology groups} are equal to the \hyperref[def:cellular-chain-complex]{chain groups} because the boundary maps are all zero.
\end{eg}

\begin{eg}
	A genus \(g\) surface \(\Sigma_g\) has the \hyperref[def:CW-Complex]{CW complex} structure as
	\begin{itemize}
		\item \(1\) \(0\)-cell \(x\).
		\item \(2g\) \(1\)-cells \(a_1, b_1, a_2, b_2, \ldots\).
		\item \(1\) \(2\)-cell \(D\) glued along \([a_1, b_2][a_2, b_2]\cdots[a_g, b_g]\) (a product of commutators)
	\end{itemize}
	We obtain the result
	\[
		\partial_1(a_i) = \partial_1(b_i) = x - x = 0.
	\]

	Furthermore, by the principle discussed above, we know that every \(1\)-cell appears once in the \hyperref[def:word]{word}, and its inverse appears once,
	so all the coefficients of \(1\)-cells in \(\partial_2(D)\) are zero, so \(\partial_2(D) = 0\). This means we have a \hyperref[def:cellular-chain-complex]{chain complex}
	\[
		\begin{tikzcd}
			0 & {\mathbb{Z}} & {\mathbb{Z}^{2g}} & {\mathbb{Z}} & 0
			\arrow[from=1-1, to=1-2]
			\arrow["0", from=1-2, to=1-3]
			\arrow["0", from=1-3, to=1-4]
			\arrow[from=1-4, to=1-5]
		\end{tikzcd}
	\]
	And so then we have that
	\[
		H_k(\Sigma _{g} ) = \begin{dcases}
			\mathbb{\MakeUppercase{z}} ,      & \text{ if } k = 0, 2 ; \\
			\mathbb{\MakeUppercase{z}} ^{2g}, & \text{ if } k = 1;     \\
			0,                                & \text{ otherwise}.
		\end{dcases}
	\]
\end{eg}

\begin{eg}[Torus example: \(\partial_2\) in more detail]
	We're going to work through this example a bit more carefully.
	\begin{figure}[H]
		\centering
		\incfig{eg:more-careful-torus-cellular}
		\label{fig:eg:more-careful-torus-cellular}
	\end{figure}
	Let's zoom in on these two preimage points and use \emph{local homology} to compute this:
	\begin{center}
		%\includegraphics[scale=0.5]{zoom-torus-cellular-preimage}
	\end{center}
\end{eg}