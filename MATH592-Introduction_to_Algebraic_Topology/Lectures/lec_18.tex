\lecture{18}{16 Feb. 10:00}{Proving \autoref{prop:lec17}}
Now let's prove \autoref{prop:lec17}.
\begin{proof}[Proof of \autoref{prop:lec17}]
	Let \(X, x_0\) be the base space and \(\widetilde{x} _0, \widetilde{x} _1\in p^{-1} (\{x_0\})\) where \(p\colon \widetilde{X} \to X\) is a
	\hyperref[def:covering-map]{covering map}. Further, let \(H\coloneqq p_\ast(\pi _1(\widetilde{X} , \widetilde{x} _0))\).

	In homework, given \((X, x_0), \widetilde{x} _0, \widetilde{x} _1\in p^{-1} (\{x_0\})\) if we change the basepoint from \(\pi _1(\widetilde{X} , \widetilde{x} _0)\)
	to \(\pi _1(\widetilde{X} , \widetilde{x} _1)\), then we have the induced subgroups of the base spaces \hyperref[def:fundamental-group]{fundamental group}
	are conjugate by some loop \([\gamma ]\in \pi _1(X, x_0)\), i.e.,
	\[
		p_\ast (\pi _1(\widetilde{X} , \widetilde{x} _1)) = [\gamma ]\cdot p_\ast (\pi _1(\widetilde{X} , \widetilde{x} _0))\cdot [\gamma ]^{-1}
	\]
	where \(\gamma\) is \hyperref[prop:homotopy-lifting-property]{lifted} to a \hyperref[def:path]{path} from \(\widetilde{x} _0\) to \(\widetilde{x} _1\).

	\par Therefore, \([\gamma ]\in N(H)\) if and only if \(p_\ast(\pi _1(\widetilde{X} , \widetilde{x} _1)) = p_\ast(\pi _1(\widetilde{X} , \widetilde{x} _0))\),
	and this holds if and only if there is a \hyperref[def:deck-transformation]{deck transformation} taking \(\widetilde{x} _0\) to \(\widetilde{x} _1\)
	by the classification of based \hyperref[def:covering-space]{covering spaces} in the homework.
	\begin{note}
		Alternatively, we can use the \hyperref[prop:lifting-criterion]{lifting criterion}.
	\end{note}
	This shows that \(p\) is a \hyperref[def:normal-cover]{normal cover} if and only if \(H\) is \hyperref[def:normal-subgroup]{normal}, which proves the first claim.

	\par We then define a map \(\Phi \) such that
	\[
		\Phi \colon N(H)\to G(\widetilde{X} )[\gamma ],\quad \cdot \mapsto \tau
	\]
	where \(\tau \) \hyperref[prop:homotopy-lifting-property]{lifts} to a \hyperref[def:path]{path} from \(\widetilde{x} _0\) to \(\widetilde{x} _1\) and
	\(\tau \) is a \hyperref[def:deck-transformation]{deck transformation} mapping \(\widetilde{x} _0\) to \(\widetilde{x} _1\), which will be
	uniquely defined by the uniqueness of \hyperref[prop:homotopy-lifting-property]{lifts} with specified base points. We now need to check
	\begin{enumerate}
		\item \(\Phi \) is surjective.
		\item \(\ker  (\Phi )= H\).
		\item \(\Phi \) is a group homomorphism.
	\end{enumerate}
	If we can prove all the above, then the result follows directly from the first isomorphism theorem.

	\begin{enumerate}
		\item We've proved that \(\Phi \) is surjective before in our work above.
		\item \(\Phi ([\gamma ])\) is the identity if and only if \(\tau \) sends \(\widetilde{x} _0\) to \(\widetilde{x} _0\), meaning that \([\gamma ]\)
		      \hyperref[prop:homotopy-lifting-property]{lifts} to a loop. Then by our characterization of the \hyperref[def:fundamental-group]{fundamental group} downstairs:
		      \[
			      \ker (\Phi ) = \left\{[\gamma ] \mid \text{\([\gamma ]\) \hyperref[prop:homotopy-lifting-property]{lifts} to a loop} \right\} = H.
		      \]
		\item Suppose we have loops \([\gamma _1]\overset{\Phi }{\mapsto } \tau _1\) and \([\gamma _2]\overset{\Phi }{\mapsto }\tau _2\). We claim that \(\gamma _1\cdot \gamma _2\)
		      \hyperref[prop:homotopy-lifting-property]{lifts} to \(\widetilde{\gamma} _1\cdot \tau (\widetilde{\gamma} _2)\).
		      \begin{figure}[H]
			      \centering
			      \incfig{pf:prop:lec17-1}
			      \label{fig:pf:prop:lec17-1}
		      \end{figure}
		      \begin{exercise}
			      Check that the \hyperref[def:lift]{lift} of \(\gamma _2\) starting at \(\widetilde{x} _1\) is exactly
			      \(\Phi _1(\widetilde{\gamma} _2)\), where \(\widetilde{\gamma} _2\) is a \hyperref[def:lift]{lift} starting at \(\widetilde{x} _0\).
			      \begin{figure}[H]
				      \centering
				      \incfig{pf:prop:lec17-2}
				      \caption{Must be \hyperref[def:lift]{lift} of \(\gamma ^\prime \) starting at \(\widetilde{x} _2\)}
				      \label{fig:pf:prop:lec17-2}
			      \end{figure}
		      \end{exercise}
		      \begin{answer}
			      The key is the uniqueness of \hyperref[prop:homotopy-lifting-property]{lifts}.
		      \end{answer}

		      We then just observe that this \hyperref[def:path]{path} \(\widetilde{\gamma} _1\cdot \tau _1(\widetilde{\gamma} _2)\) is a \hyperref[def:path]{path}
		      from \(\widetilde{x} _0\) to \(\gamma_1(\widetilde{\gamma} _2(1)) = \tau_1(\tau _2(\widetilde{x} _0))\), so the image must be a
		      \hyperref[def:deck-transformation]{deck transformation} sending \(\widetilde{x} _0\) to \(\tau _1(\tau _2(\widetilde{x} _0))\). But then \(\tau _1\circ \tau _2\)
		      maps \(\widetilde{x} _0\) to this same point, and from \hyperref[ex:lec17]{this exercise}, we know that the \hyperref[def:deck-transformation]{deck transformations}
		      are determined by where they send a single point, hence we're done.
	\end{enumerate}
\end{proof}

\begin{corollary}
	If \(p\) is a \hyperref[def:normal-cover]{normal covering}, then \(G(\widetilde{X} ) \cong \quotient{\pi _1(X, x_0)}{H} \).
\end{corollary}

\begin{definition}[Universal covering]\label{def:universal-covering}
	A \hyperref[def:covering-map]{cover} \(p\colon \widetilde{X} \to X\) is called a \emph{universal covering} if \(\widetilde{X} \) is simply connected.
\end{definition}
\begin{corollary}
	If \(\widetilde{X} \) is the \hyperref[def:universal-covering]{universal cover}, then \(G(\widetilde{X} )\cong \pi _1(X, x_0)\).
\end{corollary}

\begin{exercise}
	Whether \(\im (p_\ast)\) is \hyperref[def:normal-subgroup]{normal} is independent of the basepoint in \(\widetilde{X} \) and \(X\).
\end{exercise}

So, \(p\) is \hyperref[def:normal-cover]{normal} if and only if \(G(\widetilde{X} )\) is transitive on \(p^{-1} (x_0)\) for at least one
\(x_0\in X\).
\begin{exercise}
	Let \(\Sigma g\) be the genus \(g\) surface. Prove that \(\Sigma g\) has a \hyperref[def:normal-cover]{normal} \(n\)-sheeted \hyperref[def:path]{path}-connected
	\hyperref[def:covering-map]{cover} for every \(n\).
\end{exercise}