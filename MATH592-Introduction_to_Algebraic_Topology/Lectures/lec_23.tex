\lecture{23}{07 Mar. 10:00}{Singular Homology}
\subsection{Singular Homology}
\begin{definition}[Singular simplex]\label{def:singular-simplex}
	A \emph{singular \(n\)-simplex} in a space \(X\) is a continuous map
	\[
		\sigma \colon \Delta^n \to X.
	\]
\end{definition}

\begin{definition}[Singular chain]\label{def:singular-chain}
	Let \(C_n(X)\) be the \hyperref[def:free-group]{free group} on \hyperref[def:singular-simplex]{singular \(n\)-simplices} in \(X\).
	The \emph{singular \(n\)-chains} with \hyperref[def:boundary-homomorphism]{boundary maps}:
	\[
		\begin{split}
			\partial_n \colon C_n(X) & \to C_{n - 1}(X)                                                                       \\
			\sigma              & \mapsto \sum_{i = 1}^n (-1)^i \at{\sigma}{[v_0, \ldots, \widehat{v}_i, \ldots, v_n]}{}.
		\end{split}
	\]
	This gives us a \emph{singular chain complex}
\end{definition}

\begin{definition}[singular homology group]\label{def:singular-homology-group}
	The \emph{singular homology groups} are the \hyperref[def:homology-group]{homology groups} of this \hyperref[def:singular-chain]{singular chain complex} given as
	\[
		H_n(X) = \quotient{\ker \partial_n}{\im \partial_{n + 1}}.
	\]
\end{definition}

Since the generating sets for \(C_n(X)\) are almost always hugely uncountable, it's almost impossible to compute with these. However, it does give us a
definition that does not depend on any other structure than the topology of $X$, making it useful for \underline{developing theory}.

\begin{note}
	The heuristic is that, we interpret a \hyperref[def:chain-complex]{chain} $\sigma_1 \pm \sigma_2 \pm \cdots \pm \sigma_k$ as a map from a \hyperref[def:delta-complex]{\(\Delta\)-complex} to \(X\).
	For example with \(\sigma_1 + \sigma_2\).
\end{note}

%fig 
Where we've glued $[v_1, v_2]$ of $\sigma_1$ to $[v_0, v_2]$ of $\sigma_2$ if $\at{\sigma_1}{[v_1, v_2]}{}$ and $\sigma_{[v_0, v_2]}$ are the same singular \hyperref[def:chain-group]{\(n\)-chain}
with opposite signs.

\emph{Goals}:
\begin{itemize}
	\item \hyperref[def:singular-homology-group]{Singular homology} is a \hyperref[def:homotopy]{homotopy} invariant
	\item \hyperref[def:singular-homology-group]{Singular} and simplicial homology groups are isomorphic.
\end{itemize}

\begin{exercise}
	Check that if \(X\) has \hyperref[def:path]{path} components \(\{X_\alpha\}\) then
	\[
		H_n(X) \cong \bigoplus_\alpha H_n(X_\alpha).
	\]
\end{exercise}

\begin{exercise}
	If \(X = \{\ast\}\), then
	\[
		H_n(X) = \begin{dcases}
			\mathbb{\MakeUppercase{z}}, & \text{ if } n=0 ;     \\
			0,                          & \text{ if } n\geq 1 .
		\end{dcases}
	\]
\end{exercise}

\begin{exercise}
	If \(X\) is \hyperref[def:path]{path}-connected, then $H_0(X) \cong \mathbb{\MakeUppercase{z}} $
\end{exercise}

\subsection{Functoriality and Homotopy Invariance}
\begin{definition}[Induced map on chains]\label{def:induced-map-on-chains}
	For a given continuous map $f \colon X \to Y$ we can consider the \emph{map \(f_{\#}\) induced by chains} as
	\begin{align*}
		f_{\#} \colon C_n(X)           & \to C_n(Y)                                      \\
		[\sigma \colon \Delta^n \to X] & \mapsto [f \circ \sigma \colon \Delta^n \to Y].
	\end{align*}
\end{definition}

\begin{definition}[Chain map]\label{def:chain-map}
	Given two \hyperref[def:chain-complex]{chain complexes} $(C_\ast, \partial_\ast)$ and $(D_\ast, \delta_\ast)$, a \emph{chain map} between them is a collection of
	group homomorphisms $g_n \colon C_n \to D_n$ such that the following diagram commutes.
	\[\begin{tikzcd}
			\ldots & {C_{n+1}} & {C_n} & {C_{n-1}} & \ldots \\
			\ldots & {D_{n+1}} & {D_n} & {D_{n-1}} & \ldots
			\arrow["{\partial_{n+2}}", from=1-1, to=1-2]
			\arrow["{\delta_{n+2}}", from=2-1, to=2-2]
			\arrow["{\delta_{n+1}}", from=2-2, to=2-3]
			\arrow["{\delta_{n}}", from=2-3, to=2-4]
			\arrow["{\partial_{n-1}}", from=1-4, to=1-5]
			\arrow["{\partial_{n}}", from=1-3, to=1-4]
			\arrow["{\partial_{n+1}}", from=1-2, to=1-3]
			\arrow["{\delta_{n-1}}", from=2-4, to=2-5]
			\arrow["{g_{n-1}}", from=1-4, to=2-4]
			\arrow["{g_{n}}", from=1-3, to=2-3]
			\arrow["{g_{n+1}}", from=1-2, to=2-2]
		\end{tikzcd}\]
	i.e. we have that $\delta_n \circ f_n = f_{n - 1} \circ \partial_n$.
\end{definition}

\begin{exercise}
	We have that $f_{\#} \partial = \partial f_{\#}$. In other words, $f_{\#}$ is a \hyperref[def:chain-map]{chain map}. Thus, by the homework $f_{\#}$ induces a group
	homomorphism on the \hyperref[def:homology-group]{homology groups}. We write this as $f_\ast \colon H_n(X) \to H_n(Y)$ for all $n$.
\end{exercise}

\begin{exercise}
	We have \underline{functoriality}, i.e. $(f \circ g)_\ast = f_\ast \circ g_\ast$. Also, we have that $(\identity_X)_\ast = \identity_{H_n(X)}$.
\end{exercise}

\begin{theorem}[Homology group defines a functor]\label{thm:homology-group-defines-a-functor}
	The \hyperref[def:homology-group]{\(n\)-th homology group} $H_n \colon X \mapsto H_n(X)$ gives a \hyperref[def:functor]{functor} from $\underline{\mathrm{Top}}$
	to $\underline{\mathrm{Ab} }$. This follows from the two exercises above.
\end{theorem}

\begin{theorem}[Functoriality is homotopy invariant]\label{thm:functoriality-is-homotopy-invariant}
	If $f, g\colon X \to Y$ are \hyperref[def:homotopic]{homotopic}, then they will induce the same map on \hyperref[def:homology-group]{homology}
	\[
		f_\ast = g_\ast : H_n(X) \to H_n(Y).
	\]
\end{theorem}

\begin{exercise}
	\autoref{thm:homology-group-defines-a-functor} and \autoref{thm:functoriality-is-homotopy-invariant} imply that \(H_{n} \) is a \hyperref[def:homotopy]{homotopy} invariant.
\end{exercise}

To prove the second theorem, we introduce some \hyperref[sec:homological-algebra]{homological algebra}.

\begin{definition}[Chain homotopy]\label{def:chain-homotopy}
	Given \hyperref[def:chain-complex]{chain complexes} $(A_\ast, d^A_\ast)$ and $(B_\ast, d^B_\ast)$ and \hyperref[def:chain-map]{chain maps}
	$f_\ast, g_\ast \colon  A_\ast \to B_\ast$. A \emph{chain homotopy} from $f$ to $g$ is a sequence of group homomorphisms $\psi_n\colon A_n \to B_{n + 1}$ such that
	\[
		f_n - g_n = d^B_{n + 1} \circ \psi_n + \psi_{n - 1} d_n^A
	\]
	In a diagram, letting $h_n = f_n - g_n$, we have the following.
	\[\begin{tikzcd}
			\ldots && {A_{n+1}} && {A_{n}} && {A_{n-1}} && \ldots \\
			\\
			\ldots && {B_{n+1}} && {B_{n}} && {B_{n-1}} && \ldots
			\arrow["{d^A_{n-1}}", from=1-7, to=1-9]
			\arrow["{d^B_{n-1}}", from=3-7, to=3-9]
			\arrow["{d^A_{n}}", color={rgb,255:red,92;green,214;blue,92}, from=1-5, to=1-7]
			\arrow["{d^B_{n}}", from=3-5, to=3-7]
			\arrow["{d^B_{n+1}}", color={rgb,255:red,92;green,214;blue,92}, from=3-3, to=3-5]
			\arrow["{d^A_{n+2}}", from=1-1, to=1-3]
			\arrow["{d^B_{n+2}}", from=3-1, to=3-3]
			\arrow["{h_{n+1}}", from=1-3, to=3-3]
			\arrow["{h_{n}}", color={rgb,255:red,214;green,92;blue,92}, from=1-5, to=3-5]
			\arrow["{h_{n-1}}", from=1-7, to=3-7]
			\arrow["{d^A_{n+1}}", from=1-3, to=1-5]
			\arrow["{\psi_n}"{description}, color={rgb,255:red,92;green,214;blue,92}, from=1-5, to=3-3]
			\arrow["{\psi_{n-1}}"{description}, color={rgb,255:red,92;green,214;blue,92}, from=1-7, to=3-5]
		\end{tikzcd}\]
	This diagram does not commute, but it shows everything that is going on. However, the \textbf{\textcolor{red}{red}} map is the sum of the
	\textbf{\textcolor{green}{green}} maps composed up.
\end{definition}

