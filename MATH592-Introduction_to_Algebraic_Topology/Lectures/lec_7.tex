\lecture{7}{21 Jan. 10:00}{Functors}
\begin{eg}[Applying a covariant functor]
	Assume that we initially have a commutative diagram in \(\mathscr{C} \) as
	\[
		\begin{tikzcd}
			X \ar[rd, "g\circ f"']\ar[r, "f"] & Y\ar[d, "g"]\\
			& Z
		\end{tikzcd}
	\]
	and a \hyperref[def:functor]{functor} \(F\colon \mathscr{C} \to \mathscr{D}\).  After applying \(F\), we'll have
	\[
		\begin{tikzcd}
			F(X) \ar[rd, "F(g\circ f)=F(g)\circ F(f)"']\ar[r, "F(f)"] & F(Y)\ar[d, "F(g)"]\\
			& F(Z)
		\end{tikzcd}
	\]
	which is a commutative diagram in \(\mathscr{D}\).
\end{eg}

We can also have a so-called \underline{contravariant \hyperref[def:functor]{functor}}.
\begin{definition}[Contravariant functor]\label{def:contravariant-functor}
	Given \(\mathscr{C} , \mathscr{D} \) be two \hyperref[def:category]{categories}. A \emph{contravariant functor}
	\(F\colon \mathscr{C} \to \mathscr{D}\) is
	\begin{enumerate}[(1)]
		\item a map on \hyperref[def:object]{objects}
		      \[
			      \begin{split}
				      F\colon \Object(\mathscr{C} )&\to \Object(\mathscr{D} )\\
				      X&\mapsto F(X).
			      \end{split}
		      \]
		\item maps of \hyperref[def:morphism]{morphisms}
		      \[
			      \begin{split}
				      \Homomorphism_{\mathscr{C} }(X, Y)&\to \Homomorphism_{\mathscr{D} }(F(Y), F(X))\\
				      \left[f\colon X\to Y\right] &\mapsto \left[F(f)\colon F(Y)\to F(X)\right]
			      \end{split}
		      \]
		      such that
		      \begin{itemize}
			      \item \(F(\identity_{X} ) = \identity_{F(x)} \)
			      \item \(F(f\circ g) = F(g)\circ F(f)\)
		      \end{itemize}
	\end{enumerate}
\end{definition}

\begin{eg}[Applying a contravariant functor]
	Then, we see that in this case, when we apply a \hyperref[def:contravariant-functor]{contravariant functor} \(F\), the diagram becomes
	\[
		\begin{tikzcd}
			F(X)  & F(Y)\ar[l, "F(f)"']\\
			& F(Z)\ar[lu, "F(g\circ f)=F(f)\circ F(g)"]\ar[u, "F(g)"']
		\end{tikzcd}
	\]
	which is a commutative diagram in \(\mathscr{D}\).
\end{eg}

We now see some common \hyperref[def:functor]{functors} as examples.
\begin{eg}[Identity functor]
	Define \(I\) as \(I\colon \mathscr{C} \to \mathscr{C}\) such that it just send an object \(C\in \mathscr{C}\) to itself.
\end{eg}
\begin{eg}[Forgetful functor]\label{eg:forgetful-functor}
	We see two examples.
	\begin{itemize}
		\item Define \(F\) as \(F\colon \underline{\mathrm{Gp}}\to \underline{\mathrm{set}}\) such that \(G\mapsto G\).\footnote{\(G\) is now just the underlying set of the group \(G\).}
		      Specifically,
		      \[
			      \left[f\colon G\to H\right]\mapsto \left[f\colon G\to H\right].
		      \]
		\item Define \(F\) as \(F\colon \underline{\mathrm{Top}}\to \underline{\mathrm{set}}\) such that \(X\mapsto X\).\footnote{\(X\) is now just the underlying set of the topological space \(X\).}
		      Specifically,
		      \[
			      \left[f\colon X\to Y\right]\mapsto \left[f\colon X\to Y\right].
		      \]
	\end{itemize}
\end{eg}
\begin{eg}[Free functor]\label{eg:free-functor}
	Define a \hyperref[def:functor]{functor} as
	\[
		\begin{split}
			\underline{\mathrm{set}}&\to \underline{k\mathrm{-vect}}  \\
			s&\mapsto \text{"free" \(k\)-vector space on \(s\)}
		\end{split}
	\]
	i.e., vector space with basis \(s\) such that
	\[
		\left[f\colon A\to B\right]\mapsto \left[\text{unique \(k\)-linear map extending \(f\)}\right]
	\]
\end{eg}
\begin{eg}
	Let \(T\) be defined as
	\[
		\begin{split}
			T\colon \underline{k\mathrm{-vect}}&\to \underline{k\mathrm{-vect}}  \\
			V&\mapsto V^{\ast} = \Homomorphism _k(V, k).
		\end{split}
	\]
	If we are working on a basis, we can then represent \(T\) as a matrix \(A\), and we further have
	\[
		A\mapsto A^{\top}.
	\]
\end{eg}

\begin{remark}
	Specifically, we care about two \hyperref[def:functor]{functors}.
	\begin{enumerate}[(1)]
		\item \(\pi _1\colon \underline{\mathrm{Top}^{\ast}} \to \underline{\mathrm{Gp}}\) such that
		      \[
			      \begin{split}
				      \pi _1\colon \underline{\mathrm{Top}^{\ast}} &\to \underline{\mathrm{Gp}}\\
				      (X, x_0)&\mapsto \pi_1(X, x_0),
			      \end{split}
		      \]
		      where \(\pi _1\) is the so-called \hyperref[def:fundamental-group]{fundamental group}.
		\item \(H_p\colon \underline{\mathrm{Top}} \to \underline{\mathrm{Ab}}\) such that
		      \[
			      \begin{split}
				      H_p\colon \underline{\mathrm{Top}} &\to \underline{\mathrm{Ab}} \\
				      X&\mapsto H_{p} (X),
			      \end{split}
		      \]
		      where \(H_{p}\) is the so-called \hyperref[def:homology-group]{\(p^{th}\) homology group}.
	\end{enumerate}
	We are not building toward \hyperref[def:fundamental-group]{fundamental groups}.
\end{remark}

Let's see some building blocks we need.

\section{Free Groups}
\begin{definition}[Free group]\label{def:free-group}
	Given a set \(S\), the \emph{free group} is a group \(F_S\) on \(S\) with a map \(S\to F_S\) satisfying the following
	\underline{universal property}: If \(G\) is any group, \(f\colon S\to G\) is any map of sets, \(f\) extends uniquely
	to group homomorphism \(\overline{f} \colon F_S \to G\).
	\[
		\begin{tikzcd}
			S \ar[rd, "f"']\ar[r, ] & F_S\ar[d, dashed, "\exists ! \overline{f} \colon \text{group homorphism} "]\\
			& G
		\end{tikzcd}
	\]
\end{definition}
\begin{note}
	This defines a \emph{natural bijection}
	\[
		\Homomorphism_{\underline{\mathrm{set}}}(S, \mathscr{U}(G))\cong \Homomorphism_{\underline{\mathrm{Gp}}}(F_S, G),
	\]
	where \(\mathscr{U} (G)\) is the \hyperref[eg:forgetful-functor]{forgetful functor} from the
	\hyperref[def:category]{category} of groups to the \hyperref[def:category]{category} of sets. This is the
	statement that the \hyperref[eg:free-functor]{free functor} and the \hyperref[eg:forgetful-functor]{forgetful functor} are \hyperref[def:adjoint-functor]{adjoint};
	specifically that the \hyperref[eg:free-functor]{free functor} is the left \hyperref[def:adjoint-functor]{adjoint} (appears on the left in the \(\Homomorphism\) above).
\end{note}

\begin{definition}[Adjoints functor]\label{def:adjoint-functor}
	A \hyperref[eg:free-functor]{free} and \hyperref[eg:forgetful-functor]{forgetful} \hyperref[def:functor]{functor} is \emph{adjoints}.
\end{definition}
\begin{remark}
	Whenever we state a universal property for an \hyperref[def:object]{object} (plus a map), an
	\hyperref[def:object]{object} (plus a map) \underline{may or may not} exist.
	If such \hyperref[def:object]{object} exists, then it defines the
	\hyperref[def:object]{object} \textbf{uniquely up to unique isomorphism}, so we can use the universal
	property as the \emph{definition} of the \hyperref[def:object]{object} (plus a map).
\end{remark}

\begin{lemma}\label{lma:lec7}
	Universal property defines \(F_S\) (plus a map \(S\to F(S)\) ) uniquely up to unique isomorphism.
\end{lemma}
\begin{proof}
	Fix \(S\). Suppose
	\[
		S\to F_S,\quad S\to \widetilde{F} _S
	\]
	both satisfy the unique property. By universal property, there exist maps such that
	\[
		\begin{tikzcd}
			S \ar[rd, "f"']\ar[r, ] & \widetilde{F}_S\ar[d, dashed, "\exists ! \varphi"]\\
			& F_S
		\end{tikzcd}\qquad
		\begin{tikzcd}
			S \ar[rd, "f"']\ar[r, ] & F_S\ar[d, dashed, "\exists ! \psi"]\\
			& \widetilde{F} _S
		\end{tikzcd}
	\]

	We'll show \(\varphi\) and \(\psi \) are inverses (and the unique isomorphism making above commute). Since
	we must have the following two commutative graphs.
	\[
		\begin{tikzcd}
			& F_S\ar[dd, "\identity_{F_S}"]\\
			S \ar[rd, "f"']\ar[ru, "f"]&\\
			& F_S
		\end{tikzcd}\quad
		\begin{tikzcd}
			& \widetilde{F}_S\ar[dd, "\identity_{\widetilde{F}_S} "]\\
			S \ar[rd, "f"']\ar[ru, "f"]&\\
			& \widetilde{F}_S
		\end{tikzcd}
	\]
	Hence, we see that
	\[
		\begin{tikzcd}
			& F_S\ar[d, "\psi"]\ar[dd, bend left, "\varphi \circ \psi = \identity_{F_S}"]\\
			S \ar[rd, "f"']\ar[ru, "f"]\ar[rd]\ar[r] & \widetilde{F}_S\ar[d, "\varphi"]\\
			& F_S
		\end{tikzcd}\qquad
		\begin{tikzcd}
			& \widetilde{F} _S\ar[d, "\varphi"]\ar[dd, bend left, "\psi \circ \varphi = \identity_{\widetilde{F}_S}"]\\
			S \ar[rd, "f"']\ar[ru, "f"]\ar[rd]\ar[r] & F_S\ar[d, "\psi "]\\
			& \widetilde{F} _S
		\end{tikzcd}
	\]
	where the identity makes these outer triangles commute, then by the uniqueness in universal property, we must have
	\[
		\varphi\circ \psi  = \identity_{F_S},\qquad \psi \circ \varphi = \identity_{\widetilde{F} _S},
	\]
	so \(\varphi\) and \(\psi \) are inverses (thus group isomorphism).
\end{proof}