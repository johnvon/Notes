\lecture{2}{07 Jan. 10:00}{Homotopy Equivalence}
\begin{prev}
	Two maps \(f, g\colon X\to Y\) is \hyperref[def:homotopy]{homotopy} if there exists a map
	\[
		F_t(x)\colon X\times I\to Y
	\]
	with the properties
	\begin{enumerate}
		\item Continuous
		\item \(F_0(x) = f(x)\)
		\item \(F_1(x) = g(x)\)
	\end{enumerate}
	\begin{remark}
		The continuity of \(F_t\) is an even stronger condition for the continuity of \(F_t\) for a fixed \(t\).
	\end{remark}
\end{prev}

We now introduce another concept.
\begin{definition}[Homotopy relative]\label{def:homotopy-relative}
	Given two spaces \(X, Y\), and let \(B\subseteq X\). Then a \hyperref[def:homotopy]{homotopy} \(F_t(x)\colon X\to Y\) is called
	\emph{homotopy relative \(B\)} (denotes \(\mathrm{rel} B\)) if \(F_t(b)\) is independent of \(t\) for all \(b\in B\).
\end{definition}

\begin{eg}
	Given \(X\) and \(B = \{0, 1\}\). Then the \hyperref[def:homotopy]{homotopy} of paths from \([0, 1]\to X\) is
	\(\mathrm{rel} \{0, 1\}\).
	\begin{figure}[H]
		\centering
		\incfig{eg:rel-homotopy}
		\label{fig:eg:rel-homotopy}
	\end{figure}
\end{eg}

\section{Homotopy Equivalence}
With this, we can introduce the concept of \emph{homotopy equivalence}.
\begin{definition}[Homotopy equivalence, homotopy inverse]\label{def:homotopy-equivalence}\label{def:homotopy-inverse}
	A map \(f\colon X\to Y\) is a \emph{homotopy equivalence} if \(\exists g\colon Y\to X\) such that
	\[
		f\circ g \simeq \identity_Y, \quad g\circ f \simeq \identity_X.
	\]
	We say that \(X\), \(Y\) are \emph{homotopy equivalent}, and \(g\) is called \emph{homotopy inverse} of \(f\).

	\begin{figure}[H]
		\centering
		\incfig{def:homotopy-equivalence}
		\caption{\hyperref[def:homotopy-equivalence]{Homotopy Equivalence}}
		\label{fig:def:homotopy-equivalence}
	\end{figure}
\end{definition}

\begin{definition}[Homotopy type]\label{def:homotopy-type}
	If \(X, Y\) are \hyperref[def:homotopy-equivalence]{homotopy equivalent}, then we say that they have the
	same \emph{homotopy type}.
\end{definition}

\begin{notation}
	We denote a closed \(n\)-disk as \(D^n\).
\end{notation}
\begin{eg}
	\(D^n\) is \hyperref[def:homotopy-equivalence]{homotopy equivalent} to a point.
	\begin{figure}[H]
		\centering
		\incfig{eg:closed-disk-eq-point}
		\label{fig:eg:closed-disk-eq-point}
	\end{figure}
\end{eg}
\begin{explanation}
	We see that \(f\circ g = \identity_\ast\) and
	\[
		g\circ f = \text{constant map at }\underbrace{0}_{g(\ast)},
	\]
	which is \hyperref[def:homotopic]{homotopic} to \(\identity_{D^n}\) by
	\hyperref[eg:lec1:straight-line-homotopy]{straight line homotopy} \(F_t(x) = tx\). Specifically, we see that this holds for any convex set.
\end{explanation}

\begin{definition}[Contractible]\label{def:contractible}
	We say that a space \(X\) is \emph{contractible} if \(X\) is \hyperref[def:homotopy-equivalence]{homotopy equivalent}
	to a point.
\end{definition}

The following proposition is added much after, which may uses some concepts not yet covered.
\begin{proposition}
	The followings are equivalent.
	\begin{enumerate}
		\item \(X\) is \hyperref[def:contractible]{contractible}.
		\item \(\forall x\in X, \identity_{X}\simeq c _{x} \).
		\item \(\exists x\in X, \identity_{X}\simeq c _{x} \).
	\end{enumerate}
\end{proposition}
\begin{remark}
	Note that the above notation \(c _{x} \) is introduced \hyperref[not:constant-loop]{here}.
\end{remark}
\begin{proof}
	We see that \(2. \implies 3.\) is obvious. We consider \(3.\implies 2.\) This follows from the following general lemma.
	\begin{lemma}\label{lma:lec2}
		Given a topological space \(X\) such that \(\exists x\in X, \identity_{X}\simeq c _{p} \), with \(f, g\colon Y\to X\), then \(f\simeq g\).
	\end{lemma}
	\begin{proof}
		Let \(x\in X\) such that \(\identity_{X} \simeq c _{x} \). Then
		\[
			f = \identity_{X} \circ f\simeq c_{x} \circ f= c_{x} \circ g\simeq \identity_{X} \circ g = g.
		\]
	\end{proof}
	Then, from this \autoref{lma:lec2}, we see that assuming \(x_0\in X\) such that \(\identity_{X} \simeq c_{x_0}\), then consider \(c_{x} \) for all \(x\in X\), then
	from \autoref{lma:lec2}, we see that \(c_{x} \simeq \identity_{X} \).

	To show \(3. \implies 1.\), we let \(x_0\in X\) such that \(\identity_{X}\simeq c_{x_0} \).
	\[
		\begin{tikzcd}
			X & {\{\ast\}}
			\arrow["f", curve={height=-6pt}, from=1-1, to=1-2]
			\arrow["g", curve={height=-6pt}, from=1-2, to=1-1]
		\end{tikzcd}
	\]
	Since \(g(\ast) = x_0\), and
	\[
		\begin{split}
			g\circ f\colon X&\to X\\
			x&\mapsto x_0,
		\end{split}
	\]
	which is just \(c_{x_0}\), from the assumption we're done.

	Now, we show \(1. \implies 3.\) Let
	\[
		\begin{tikzcd}
			X & {\{\ast\}}
			\arrow["f", curve={height=-6pt}, from=1-1, to=1-2]
			\arrow["g", curve={height=-6pt}, from=1-2, to=1-1]
		\end{tikzcd}
	\]
	be a \hyperref[def:homotopy-equivalence]{homotopy equivalent}, let \(g(\ast) = x_0\). We see that \(c_{x_0}\simeq \identity_{X} \) since
	\[
		g\circ f = c_{x_0} \simeq \identity_{X}.
	\]
\end{proof}

Before doing exercises, we introduce two new concepts.
\begin{definition}[Retraction, retract]\label{def:retraction}
	Given \(B\subseteq X\), a \emph{retraction} from \(X\) to \(B\) is a map \(f\colon X\to X\) (or \(X\to B\))
	such that \(\forall b\in B\ f(b) = b\), namely \(\at{r}{B}{} = \identity_B\). Or one can see this from
	\[
		\begin{tikzcd}
			B \ar[rr, swap, bend right=30, "r \circ i"] \ar[r,hook,"i"] & X \ar[r, "r"] & B
		\end{tikzcd}
	\]
	where \(r\) is a retraction if and only if \(r\circ i = \identity_B\), where \(i\) is an inclusion identity.

	\par If \(r\) exists, \(B\) is a \emph{retract} of \(X\).
\end{definition}
\begin{definition}[Deformation retraction]\label{def:deformation-retraction}
	Given \(X\) and \(B\subseteq X\), a \emph{(strong) deformation retraction} \(F_t\colon X\to X\) onto \(B\) is
	a \hyperref[def:homotopy]{homotopy} \(\mathrm{rel} B\) from the \(\identity_X\) to a \hyperref[def:retraction]{\emph{retraction}}
	from \(X\) to \(B\). i.e.,
	\[
		\begin{alignedat}{4}
			&F_0(x) &&= &&x\quad &&\forall x\in X\\
			&F_1(x) &&\in &&B\quad &&\forall x\in X\\
			&F_t(b) &&= &&b\quad &&\forall t\ \forall b\in B.
		\end{alignedat}
	\]
\end{definition}

\begin{exercise}
	Let \(X\simeq Y\). Show \(X\) is path-connected if and only if \(Y\) is.
\end{exercise}
\begin{answer}
	Suppose \(X\) is path-connected. Then we see that given two points \(x_1\) and \(x_2\) in \(X\), there exists a path \(\gamma(t)\) with
	\[
		\gamma\colon [0, 1]\to X,\quad \gamma(0) = x_1,\quad \gamma(1) = x_2.
	\]
	Since \(X\simeq Y\), then there exists a pair of \(f\) and \(g\) such that \(f\colon X\to Y\) and \(g\colon Y\to X\) with
	\[
		f\circ g\underset{F}{\simeq} \identity_Y,\quad g\circ f\underset{G}{\simeq} \identity_X.
	\]
	(Notice the abuse of notation)

	For any two \(y_1\) and \(y_2\in Y\), we want to construct a path \(\gamma^\prime (t)\) such that
	\[
		\gamma^\prime \colon [0, 1]\to Y,\quad \gamma^\prime (0) = y_1,\quad \gamma^\prime (1) = y_2.
	\]

	Firstly, we let \(g(y_1) \eqqcolon x_1\) and \(g(y_2) \eqqcolon x_2\). From the argument above, we know there exists such
	a \(\gamma\) starting at \(x_1 = g(y_1)\) ending at \(x_2 = g(y_2)\). Now, consider \(f(\gamma(t)) = (f\circ \gamma) (t)\)
	such that
	\[
		f\circ \gamma\colon I\to Y,\quad f\circ \gamma(0) = y_1^\prime,\quad f\circ \gamma(1) = y_2^\prime,
	\]
	we immediately see that \(y_{1}^\prime\) and \(y_2^\prime \) is path connected. Now, we
	claim that \(y_1\) and \(y_1^\prime\) are path connected in \(Y\), hence so are \(y_2\) and \(y_2^\prime \).
	To see this, note that
	\[
		f\circ g\underset{F}{\simeq} \identity_Y,
	\]
	which means that there exists \(F\colon Y\times I\to Y\) such that
	\[
		\begin{dcases}
			 & F(y_1, 0) = f\circ g(y_1) = f(x_1) = f(\gamma(0)) = (f\circ \gamma)(0) = y_1^\prime \\
			 & F(y_1, 1) = \identity_Y(y_1) = y_1.
		\end{dcases}
	\]
	Since \(F\) is continuous in \(I\), we see that there must exist a path connects \(y_1\) and \(y_1^\prime \). The same argument applies to
	\(y_2\) and \(y_2^\prime \). Now, we see that the path
	\[
		y_1 \to y_1^\prime \to y_2^\prime \to y_2
	\]
	is a path in \(Y\) for any two \(y_1\) and \(y_2\), which shows \(Y\) is path-connected.

	\begin{figure}[H]
		\centering
		\incfig{eg:path-connected}
		\caption{Demonstration of the proof.}
		\label{fig:eg:path-connected}
	\end{figure}
\end{answer}

\textbf{Challenge}: One can further show that the connectedness is also preserved by any \hyperref[def:homotopy-equivalence]{homotopy equivalence}.
\begin{corollary}
	A \hyperref[def:contractible]{contractible} space is \hyperref[def:path]{path}-connected.
\end{corollary}

\begin{exercise}
	Show that if there exists \hyperref[def:deformation-retraction]{deformation retraction} from \(X\) to \(B\subseteq X\), then \(X\simeq B\).
\end{exercise}
