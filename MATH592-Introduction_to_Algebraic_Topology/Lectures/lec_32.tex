\lecture{32}{28 Mar. 10:00}{Cellular Homology}
\section{Cellular Homology}
Suppose that \(X\) is a \hyperref[def:CW-Complex]{CW complex}, then \((X^n, X^{n - 1})\) is a \hyperref[def:good-pair]{good pair} for all \(n > 1\), and
\(\quotient{X^n}{X^{n - 1}}\) is a \hyperref[CW-complex-wedge-sum]{wedge of \(n\)-spheres}, one for each \(n\)-cell \(e^n_\alpha\). Hence,
\[
	H_k(X^n, X^{n - 1}) \cong \begin{dcases}
		0,                                                                    & \text{ if } k\neq n ; \\
		\langle e_\alpha^n \mid  e_\alpha^n \text{ is an \(n\)-cell} \rangle, & \text{ if } k = n .
	\end{dcases}
\]

\begin{definition}[Cellular chain complex]\label{def:cellular-chain-complex}
	The \emph{cellular chain complex} on \(X\), denoted as \(\overline{\omega}\), has
	\paragraph{(Chain groups).} The chain groups \(C_{n} (X)\) are defined as
	\[
		C_{n} (X) \coloneqq \mathbb{\MakeUppercase{z}} \left< e^n_\alpha \mid e^n_\alpha \text{ an \(n\)-cell of \(X\)}\right> (\cong H_n(X^n, X^{n-1}))
	\]
	with \(X^{-1} = \varnothing\).

	\paragraph{(Boundary maps).} For \(n=0\), we have
	\[
		\begin{split}
			\partial _1 \colon C_1(X)            & \to C_0(X)                       \\
			\langle \text{1-cells} \rangle & \to \langle \text{0-cells} \rangle,
		\end{split}
	\]
	which is the usual \hyperref[def:boundary-homomorphism]{simplicial boundary map}.\footnote{i.e.,
		\(\partial _1 \colon C_1(X)=H_1(X^1,X^0) \to C_0(X)=H_0(X^0)\) is just \(\Delta _1(X)\to \Delta _0(X)\).}
	For \(n > 1\), the \hyperref[def:boundary-homomorphism]{boundary map} \(\partial _{n}\) are defined as
	\[
		\partial _n(e_\alpha^n) = \sum_\beta \partial _{\alpha\beta} e_\beta^{n - 1}
	\]
	where \(\partial _{\alpha\beta}\) is the \hyperref[def:degree]{degree} of the map
	\[
		\begin{tikzcd}
			{\partial e^n_\alpha=S^{n-1}_{\alpha}} &&& {X^{n-1}} &&& {S^{n-1}_\beta}
			\arrow["{\text{attaching}}", "{\text{map}}"', from=1-1, to=1-4]
			\arrow["{\text{quotient by}}", "{\text{\(X^{n-1} \setminus e^{n-1}_\beta\)}}"', from=1-4, to=1-7]
		\end{tikzcd}
	\]
	In pictures, this is given as the following.
	\begin{figure}[H]
		\centering
		\incfig{cellular-boundary-map}
		\label{fig:cellular-boundary-map}
	\end{figure}
\end{definition}
\begin{remark}
	We see that
	\[
		C_{n} (X) \cong H_{n} (X^n, X^{n-1})
	\]
	since \((X^n, X^{n-1})\) is a \hyperref[def:good-pair]{good pair}, so \(H_{n} (X^n, X^{n-1}) \cong H_n(\quotient{X^n}{X^{n-1}})\), which is
	just the \hyperref[CW-complex-wedge-sum]{wedge} of \(1\) \(n\)-sphere for each \(n\)-cell of \(X\).

	Furthermore, the orientations on spheres are defined by identifying the domains of characteristic maps 	\(D^n_\alpha \to X\) with an (oriented) disk in
	\(\mathbb{\MakeUppercase{r}} ^n\). i.e., we need to choose a generator of
	\[
		H_{n-1}(\partial D^n_\alpha) \cong H_{n-1}(S^{n-1})\cong \mathbb{\MakeUppercase{z}}.
	\]
\end{remark}
\begin{note}
	In Hatcher\cite{hatcher2002algebraic}, the approach of the definition of \hyperref[def:cellular-chain-complex]{cellular chain complex} is
	a bit different, especially for how we define the boundary maps. Here we simply define \(\partial _n(e^n_\alpha ) \coloneqq \sum_\beta \partial _{\alpha \beta }e^{n-1}_\beta\),
	where this is so-called \emph{cellular boundary formula} in Hatcher\cite{hatcher2002algebraic}. Here, we just defined \(\partial _n\) in this way instead, but
	we should still check that this is well-defined of this definition. The proof is given in \autoref{pf:cellular-boundary-formula-is-well-defined}.
\end{note}

\begin{definition}[Cellular homology group]\label{def:cellular-homology-group}
	We define the so-called \emph{cellular homology group} by \hyperref[def:cellular-chain-complex]{cellular chain complex} in our usual way of defining
	\hyperref[def:homology-group]{homology group}.
\end{definition}
\begin{remark}
	We sometimes denote the \hyperref[def:cellular-homology-group]{cellular homology group} as \(H_n^{\mathrm{CW} }(X) \) if it causes confusion.
\end{remark}
\begin{theorem}\label{thm:lec-32}
	\autoref{def:cellular-chain-complex} indeed forms a \hyperref[def:chain-complex]{chain complex}.
\end{theorem}
\begin{proof}
	We need to check two things, namely the chain group \(H_n(X^n, X^{n-1} )\) defined in \autoref{def:cellular-chain-complex}
	is indeed \hyperref[def:free-Abelian-group]{free Abelian} with basis in each \(n\)-cell. But this is trivial since we have
	an one-to-one correspondence with the \(n\)-cells of \(X\) as we have shown, and we can think of elements of \(H_n(X^n, X^{n-1} )\)
	as linear combinations of \(n\)-cells of \(X\).

	The fact that the \hyperref[def:boundary-homomorphism]{boundary map} defined in \autoref{def:cellular-chain-complex} has the property \(\partial ^{2} = 0\)
	will be proved in \autoref{thm:cellular-homology-agrees-with-singular-homology}.
\end{proof}
\begin{theorem}[Cellular homology agrees with singular homology]\label{thm:cellular-homology-agrees-with-singular-homology}
	The \hyperref[def:cellular-homology-group]{cellular homology groups} coincide with the \hyperref[def:singular-homology-group]{singular homology groups}, i.e.,
	\[
		H_n^{\mathrm{CW} }(X) \cong H_n(X).
	\]
\end{theorem}
\begin{note}
	i.e., the isomorphism commutes \(\overline{\omega} f_\ast\) for all continuous \(f\colon X\to Y\).
\end{note}

\autoref{thm:cellular-homology-agrees-with-singular-homology} implies the following.
\begin{corollary}\label{col:lec-32}
	We have the followings.
	\begin{itemize}
		\item \(H_n(X) = 0\) if \(X\) has a \hyperref[def:CW-Complex]{CW complex} structure with no \(n\)-cells.
		\item If \(X\) has a \hyperref[def:CW-Complex]{CW complex} with \(k\) \(n\)-cells, then \(H_n(X)\) is generated by at most \(k\) elements.
		\item If \(H_n(X)\) is a group with a minimum of \(k\) generators, then any \hyperref[def:CW-Complex]{CW complex} structure on \(X\) must have at least \(k\) \(n\)-cells.
		\item If \(X\) has a \hyperref[def:CW-Complex]{CW complex} with no \(n\)-cells, then
		      \[
			      H_{n-1}(X) = \ker (\partial _{n-1}),
		      \]
		      which is \hyperref[def:free-Abelian-group]{free Abelian}.
		\item If \(X\) has a \hyperref[def:CW-Complex]{CW complex} with no cells in consecutive dimensions, then all \(\partial _n=0\).
		      Its \hyperref[def:cellular-homology-group]{homology} are \hyperref[def:free-Abelian-group]{free Abelian} on its \(n\)-cells,
		      namely the \hyperref[def:cellular-chain-complex]{cellular chain groups}.
	\end{itemize}
\end{corollary}
\begin{eg}
	The last point in \autoref{col:lec-32} is quite useful, as the following examples will show.
	\begin{enumerate}
		\item \(S^n, n \geq 2\). Since if we have \(S^n\) with \(n \geq 2\), using the \hyperref[def:CW-Complex]{CW complex} structure of
		      \(e^n\) attached to a single point \(x_0\). The \hyperref[def:cellular-chain-complex]{cellular chain complex} is given as
		      \[
			      \begin{tikzcd}
				      0 & 0 & {\left<e^n\right>} & 0 & \ldots & 0 & {\left<x_0\right>} & 0
				      \arrow[from=1-1, to=1-2]
				      \arrow[from=1-2, to=1-3]
				      \arrow[from=1-3, to=1-4]
				      \arrow[from=1-4, to=1-5]
				      \arrow[from=1-5, to=1-6]
				      \arrow[from=1-6, to=1-7]
				      \arrow[from=1-7, to=1-8]
			      \end{tikzcd}
		      \]
		      So then all the boundary maps are zero, and we see that
		      \[
			      H_k(S^n) = \begin{dcases}
				      \mathbb{\MakeUppercase{z}}, & \text{ if } k=0, n ; \\
				      0,                          & \text{ otherwise}.
			      \end{dcases}
		      \]
		\item \(\mathbb{\MakeUppercase{c}} P^n, \forall n\). In this case, we can let \(\mathbb{\MakeUppercase{c}} P^n\) equipped with a
		      \hyperref[def:CW-Complex]{CW complex} structure with one cell of each even dimension \(2k \leq 2n\), thus
		      \[
			      H_{k}(\mathbb{\MakeUppercase{c}} P^n) \cong \begin{dcases}
				      \mathbb{\MakeUppercase{z}}, & \text{ if } k = 0, 2, \ldots , 2n; \\
				      0,                          & \text{ otherwise}.
			      \end{dcases}
		      \]
		\item \(S^n \times S^n, n > 1\). We let \(S^n \times S^n\) has the \hyperref[CW-complex-product]{product CW structure} consisting
		      of a \(0\)-cell, two \(n\)-cells, and a \(2n\)-cell.
	\end{enumerate}
\end{eg}

\begin{exercise}
	Redo this calculation with other \hyperref[def:CW-Complex]{CW complex} structure on \(S^n\), e.g. glue \(2\) \(n\)-cells onto \(S^{n - 1}\) and proceed inductively.
\end{exercise}