\lecture{12}{2 Feb. 10:00}{Presentations for \(\pi_1\) of CW Complexes}
Let's first see an exercise.
\begin{exercise}
	Consider \(G_1 = \left< S_1 \mid R_1 \right> \) and \(G_2 = \left< S_2 \mid R_2 \right> \). Then we have
	\begin{itemize}
		\item \(G_1\ast G_2 = \left< S_{1}\cup S_2  \mid R_1 \cup R_2 \right> \)
		\item \(G_1\oplus G_2 = \left< S_1 \cup S_2  \mid R_1 \cup R_2\cup \left\{[g_1, g_2] \mid g_1\in G_1, g_2\in G_2\right\} \right> \)
		\item \(G_1 \ast_H G_2 \) where \(f_1\colon H\to G_1\) and \(f_2\colon H\to G_2\). Then we have
		      \[
			      G_1\ast_H G_2 = \left< S_1 \cup S_2  \mid R_1 \cup  R_2\cup \left\{f_1(h)f_2(h)^{-1}  \mid h\in H\right\} \right>.
		      \]
	\end{itemize}
\end{exercise}

\subsubsection{Presentations for \(\pi _1\) of CW Complexes}
For \(X\) a \hyperref[def:CW-Complex]{CW complex}, we have
\begin{enumerate}
	\item A \(1\)-dimensional \hyperref[def:CW-Complex]{CW complex} has free \(\pi _1\) (call its generators as \(a_1, \ldots , a_n \)).
	\item Gluing a \(2\)-disk by its boundary along a word \(w\) in the generators \emph{kills} \(w\) in \(\pi _1\). We then get a \hyperref[def:group-presentation]{presentation} for \(\pi _1(X^2)\)
	      given by
	      \[
		      \left< a_1, \ldots , a_n \mid w \text{ for each \(2\)-cell in \(X_2\)}\right>.
	      \]
	\item Gluing in any higher dimensional cells along their boundary will not change \(\pi _1\). That is, in a \hyperref[def:CW-Complex]{CW complex},
	      we have \(\pi _1(X) = \pi _1(X^2)\).
\end{enumerate}

\begin{remark}
	We can write the above more precise.
	\begin{enumerate}
		\item Find free generators \(\{a _i\}_{i\in I}\) for \(\pi _1(X^1)\).
		\item For each \(2\)-disk \(D^2_\alpha \), write attaching map as word \(w_\alpha \) in \(a_{i}\). i.e.,
		      \[
			      \pi _1(X^2) = \left< a_{i} \mid w_\alpha  \right>.
		      \]
		\item \(\pi_1(X) = \pi _1(X^2)\).
	\end{enumerate}
\end{remark}

\begin{eg}
	Given \(G = \quotient{\mathbb{\MakeUppercase{z}} }{n\mathbb{\MakeUppercase{z}} } = \left< a, a^n \right>  \), then we take a loop and then wind a \(2\)-disk
	around the loop \(a\) for \(n\) times.
	\begin{figure}[H]
		\centering
		\incfig{lec12-eg}
		\caption{For \(G = \quotient{\mathbb{\MakeUppercase{z}}}{n\mathbb{\MakeUppercase{z}} } = \left< a \mid a^n \right> \), we wind the boundary around \(a\) for \(n\) times.}
		\label{fig:lec12-eg}
	\end{figure}
\end{eg}

We then see that given a group \(G\) with \hyperref[def:group-presentation]{presentation} \(\left< S \mid R \right> \), one can construct a \(2\)-dimensional \hyperref[def:CW-Complex]{CW complex}
with \(\pi _1 = G\) by
\begin{itemize}
	\item Set \(X^1 = \bigvee_{s\in S} S^1\)
	\item For each relation \(r\in R\), glue in a \(2\)-disk along loops specified by the \hyperref[def:word]{word} \(r\).
\end{itemize}
Every group is then \(\pi _1\) of some space.

\begin{theorem}
	If \(X\) is a \hyperref[def:CW-Complex]{CW complex} and \(\iota _1\colon X^1\hookrightarrow X\) and \(\iota \colon X^2\hookrightarrow X\),
	then \((\iota _1)_{\ast}\) surjects onto \(\pi _1\) and \((\iota _2)_{\ast}\) is an isomorphism on \(\pi _1\).
\end{theorem}
\begin{proof}
	\todo{HW}
\end{proof}

\begin{definition}(Graph, subgraph, tree, maximal tree)\label{def:graph}\label{def:subgraph}\label{def:tree}\label{def:maximal-tree}
	We import some topological definitions of graph theoretic concepts.
	\begin{itemize}
		\item A \emph{graph} is a \(1\)-dimensional \hyperref[def:CW-Complex]{CW complex}.
		\item A \emph{subgraph} is a \hyperref[def:CW-subcomplex]{subcomplex}.
		\item A \emph{tree} is a contractible \hyperref[def:graph]{graph}.
		\item A \hyperref[def:tree]{tree} in \hyperref[def:graph]{graph} \(X\) (necessarily a \hyperref[def:subgraph]{subgraph}) is
		      \emph{maximal} or \emph{spanning} if it contains all the vertices.
	\end{itemize}
\end{definition}

\begin{theorem}
	Every connected \hyperref[def:graph]{graph} has a \hyperref[def:maximal-tree]{maximal tree}.
	Every \hyperref[def:tree]{tree} is contained in a \hyperref[def:maximal-tree]{maximal tree}.
\end{theorem}

\begin{corollary}
	Suppose \(X\) is a connected \hyperref[def:graph]{graph} with basepoint \(x_0\). Then \(\pi _1(X, x_0)\) is a
	\hyperref[def:free-group]{free group}.

	\par Furthermore, we can give a \hyperref[def:group-presentation]{presentation} for \(\pi _1(X, x_0)\) by finding a
	\hyperref[def:maximal-tree]{spanning tree} \(T\) in \(X\). The generators of
	\(\pi _1\) will be indexed by cells \(e_\alpha \in X-T\), and \(e_\alpha \) will correspond to a loop that passes through \(T\),
	traverses \(e_\alpha \) once, then returns to the basepoint \(x_0\) through \(T\).
\end{corollary}
\begin{proof}
	The idea is simple. \(X\) is \hyperref[def:homotopy-equivalence]{homotopy equivalent} to \(\quotient{X}{T}\) via previous work on the homework,
	\(T\) contains all the vertices, so the quotient has a single vertex. Thus, it is a \hyperref[sssec:Wedge-sum]{wedge} of circles, and each
	\(e_\alpha \) projects to a loop in \(\quotient{X}{T} \).
	\begin{figure}[H]
		\centering
		\incfig{lec12-pf}
		\label{fig:lec12-pf}
	\end{figure}

	\par The current plan is to calculate the  \hyperref[def:fundamental-group]{fundamental group} of \hyperref[def:CW-Complex]{CW complexes}.
	For now, we need to see that the \hyperref[def:fundamental-group]{fundamental group} of a \(1\)-skeleton (a graph) can be found by taking
	a \hyperref[def:maximal-tree]{maximal tree}, and then quotienting out the space by the \hyperref[def:tree]{tree} to get a \hyperref[sssec:Wedge-sum]{wedge} of circles.
	\begin{figure}[H]
		\centering
		\incfig{lec12:pf}
		\label{fig:lec12:pf}
	\end{figure}

	\par We now prove that the \hyperref[def:maximal-tree]{maximal trees} exist. Recall that \(X\) is a quotient of
	\[
		X^0\coprod_\alpha I_\alpha.
	\]
	Each subset \(U\) is open if and only if it intersects each edge \(\overline{e} _\alpha \) in an open subset. A map \(X\to Y\) if and only if
	its restriction to each edge \(\overline{e} _\alpha \) is continuous. Now, take \(X_0\) to be a \hyperref[def:subgraph]{subgraph}.
	Our goal is to construct a \hyperref[def:subgraph]{subgraph} \(Y\) with
	\begin{itemize}
		\item \(X_0 \subset Y\subset X\)
		\item \(Y\) \hyperref[def:deformation-retraction]{deformation retracts} to \(X_0\)
		\item \(Y\) contains all vertices of \(X\).
	\end{itemize}

	So if we take \(X_0\) to be a vertex, then \(Y\) is out \hyperref[def:tree]{tree} and we're done!

	\par Our strategy now is to build a sequence \(X_0\subset X_1\subset \ldots  \) and correspondingly, \(Y_0\subset Y_1\subset \ldots\). We start with
	\(X_0\) and inductively define
	\[
		X_i \coloneqq X_{i-1}\bigcup \text{ all edges \(\overline{e} _\alpha \) with one or both vertices in \(X_{i-1}\) }.
	\]
	We then see that \(X = \bigcup\limits_{i} X_{i} \).\todo{Check.}\footnote{\cite{hatcher2002algebraic} do this by arguing the union on the right is both open and closed.}
	Now, let \(Y_0 = X_0\). By induction, we'll assume that \(Y_{i}\) is a \hyperref[def:subgraph]{subgraph} of \(X_{i}\) such that
	\begin{itemize}
		\item \(Y_{i}\) contains all vertices of \(X_{i}\).
		\item \(Y_{i}\) \hyperref[def:deformation-retraction]{deformation retracts} to \(Y_{i-1}\).
	\end{itemize}
	We can then construct \(Y_{i+1}\) by taking \(Y_{i}\) and adding to it one edge to adjoin every vertex of \(X_{i+1}\), namely
	\[
		Y_{i+1} \coloneqq  Y_{i}\bigcup \text{ one edge to adjoint every vertex of \(X_{i}\)}\footnotemark
		\footnotetext{This is possible if we assume Axiom of Choice.}
	\]
	We then see that \(Y_{i+1}\) \hyperref[def:deformation-retraction]{deformation retracts} to \(Y_{i}\) by just smashing down each edge. Now, we can show that
	\(Y\) \hyperref[def:deformation-retraction]{deformation retracts} to \(Y_0 = X_0\) by performing the \hyperref[def:deformation-retraction]{deformation retraction}
	from \(Y_{i}\) to \(Y_{i-1}\) during the time interval \([1/2^i, 1/2^{i-1}]\).
\end{proof}

\begin{eg}
	Let
	\begin{itemize}
		\item \(S^n\): decompose into \(2\) open disks
		\item \(A_1\): neighborhood of top hemisphere
		\item \(A_2\): neighborhood of lower hemisphere
	\end{itemize}
	We see that \(A_1 \cap A_2\simeq S^{n-1}\), where we need \(n\geq 2\) to let \(S^{n-1}\) be connected. We then have
	\[
		\pi _1(S^n)\cong 0\underset{\pi _1(A_1 \cap A_2)}{\ast}0 = 0.
	\]

	On the other hand, if \(n\geq 3\), then we see that
	\[
		S^n = \quotient{D^{n} \cup \ast}{\sim} .
	\]
	Since \(2\)-skeleton is a point, thus \(\pi _1(S^n) = 0\).
\end{eg}