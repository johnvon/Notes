\lecture{12}{2 Feb. 10:00}{Presentations for \(\pi_1\) of CW Complexes}
Let's first see an exercise.
\begin{exercise}
	Consider \(G_1 = \left< S_1 \mid R_1 \right> \) and \(G_2 = \left< S_2 \mid R_2 \right> \). Then we have
	\begin{itemize}
		\item \(G_1\ast G_2 = \left< S_{1}\cup S_2  \mid R_1 \cup R_2 \right> \)
		\item \(G_1\oplus G_2 = \left< S_1 \cup S_2  \mid R_1 \cup R_2\cup \left\{[g_1, g_2] \mid g_1\in G_1, g_2\in G_2\right\} \right> \)
		\item \(G_1 \ast_H G_2 \) where \(f_1\colon H\to G_1\) and \(f_2\colon H\to G_2\). Then we have
		      \[
			      G_1\ast_H G_2 = \left< S_1 \cup S_2  \mid R_1 \cup  R_2\cup \left\{f_1(h)f_2(h)^{-1}  \mid h\in H\right\} \right>.
		      \]
	\end{itemize}
\end{exercise}

\subsection{Presentations for \(\pi _1\) of CW Complexes}
For \(X\) a \hyperref[def:CW-Complex]{CW complex}, we have
\begin{enumerate}
	\item A \(1\)-dimensional \hyperref[def:CW-Complex]{CW complex} has free \(\pi _1\) (call its generators as \(a_1, \ldots , a_n \)).
	\item Gluing a \(2\)-disk by its boundary along a word \(w\) in the generators \emph{kills} \(w\) in \(\pi _1\). We then get a presentation for \(\pi _1(X^2)\)
	      given by
	      \[
		      \left< a_1, \ldots , a_n \mid w \text{ for each \(2\)-cell in \(X_2\)}\right>.
	      \]
	\item Gluing in any higher dimensional cells along their boundary will not change \(\pi _1\). That is, in a \hyperref[def:CW-Complex]{Cw complex},
	      we have \(\pi _1(X) = \pi _1(X^2)\).
\end{enumerate}

\begin{remark}
	We can write the above more precise.
	\begin{enumerate}
		\item Find free generators \(\{a _i\}_{i\in I}\) for \(\pi _1(X^1)\).
		\item For each \(2\)-disk \(D^2_\alpha \), write attaching map as word \(w_\alpha \) in \(a_{i}\). i.e.,
		      \[
			      \pi _1(X^2) = \left< a_{i} \mid w_\alpha  \right>.
		      \]
		\item \(\pi_1(X) = \pi _1(X^2)\).
	\end{enumerate}
\end{remark}

\begin{eg}
	Given \(G = \quotient{\mathbb{\MakeUppercase{z}} }{n\mathbb{\MakeUppercase{z}} } = \left< a, a^n \right>  \), then we take a loop and then wind a \(2\)-disk
	around the loop \(a\) for \(n\) times.
	\begin{figure}[H]
		\centering
		\incfig{lec12-eg}
		\caption{For \(G = \quotient{\mathbb{\MakeUppercase{z}}}{n\mathbb{\MakeUppercase{z}} } = \left< a \mid a^n \right> \), we wind the boundary around \(a\) for \(n\) times.}
		\label{fig:lec12-eg}
	\end{figure}
\end{eg}

We then see that given a group \(G\) with presentation \(\left< S \mid R \right> \), one can construct a \(2\)-dimensional \hyperref[def:CW-Complex]{CW complex}
with \(\pi _1 = G\) by
\begin{itemize}
	\item Set \(X^1 = \bigvee_{s\in S} S^1\)
	\item For each relation \(r\in R\), glue in a \(2\)-disk along loops specified by the word \(r\).
\end{itemize}
Every group is then \(\pi _1\) of some space.

\begin{theorem}
	If \(X\) is a \hyperref[def:CW-Complex]{Cw complex} and \(\iota _1\colon X^1\hookrightarrow X\) and \(\iota \colon X^2\hookrightarrow X\),
	then \((\iota _1)_{\ast}\) surjects onto \(\pi _1\) and \((\iota _2)_{\ast}\) is an isomorphism on \(\pi _1\).
\end{theorem}
\begin{proof}
	\todo{HW}
\end{proof}

\begin{definition}
	We import some topological definitions of graph theoretic concepts.
	\begin{itemize}
		\item A \emph{graph} is a \(1\)-dimensional \hyperref[def:CW-Complex]{CW complex}.
		\item A \emph{subgraph} is a subcomplex.
		\item A \emph{tree} is a contractible graph.
		\item A tree in graph \(X\) (necessarily a subgraph) is \emph{maximal} or \emph{spanning} if it contains all the vertices.
	\end{itemize}
\end{definition}

\begin{theorem}
	Every connected graph has a maximal tree. Every tree is contained in a maximal tree.
\end{theorem}

\begin{corollary}
	Suppose \(X\) is a connected graph with basepoint \(x_0\). Then \(\pi _1(X, x_0)\) is a \hyperref[def:free-group]{free group}.

	\par Furthermore, we can give a presentation for \(\pi _1(X, x_0)\) by finding a spanning tree \(T\) in \(X\). The generators of
	\(\pi _1\) will be indexed by cells \(e_\alpha \in X-T\), and \(e_\alpha \) will correspond to a loop that passes through \(T\),
	traverses \(e_\alpha \) once, then returns to the basepoint \(x_0\) through \(T\).
\end{corollary}
\begin{proof}
	The idea is simple. \(X\) is homotopy equivalent to \(\quotient{X}{T}\) via previous work on the homework, \(T\) contains all the vertices,
	so the quotient has a single vertex. Thus, it is a wedge of circles, and each \(e_\alpha \) projects to a loop in \(\quotient{X}{T} \).
	\begin{figure}[H]
		\centering
		\incfig{lec12-pf}
		\label{fig:lec12-pf}
	\end{figure}
\end{proof}

\begin{eg}
	Let
	\begin{itemize}
		\item \(S^n\): decompose into \(2\) open disks
		\item \(A_1\): neighborhood of top hemisphere
		\item \(A_2\): neighborhood of lower hemisphere
	\end{itemize}
	We see that \(A_1 \cap A_2\simeq S^{n-1}\), where we need \(n\geq 2\) to let \(S^{n-1}\) be connected. We then have
	\[
		\pi _1(S^n)\cong 0\underset{\pi _1(A_1 \cap A_2)}{\ast}0 = 0.
	\]

	On the other hand, if \(n\geq 3\), then we see that
	\[
		S^n = \quotient{D^{n} \cup \ast}{\sim} .
	\]
	Since \(2\)-skeleton is a point, thus \(\pi _1(S^n) = 0\).
\end{eg}