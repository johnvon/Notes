\lecture{15}{9 Feb. 10:00}{Lifting}
Before proving \autoref{prop:lifting-criterion}, we first see an application.
\begin{eg}
	Prove that every continuous map \(f\colon \mathbb{\MakeUppercase{r}} P^{2}\to S^1\) is \hyperref[def:nullhomotopic]{nullhomotopic}.
	\begin{proof}
		If we can show that there is a \hyperref[prop:homotopy-lifting-property]{lift} \(\widetilde{f} \colon \mathbb{\MakeUppercase{r}} P^{2}\to \mathbb{\MakeUppercase{r}}\) of \(f\),
		then we're done since we can apply the \hyperref[eg:lec1:straight-line-homotopy]{straight line} \hyperref[def:nullhomotopic]{nullhomotopy}
		on \(\mathbb{\MakeUppercase{r}} \) since
		\[\begin{tikzcd}
				& {\mathbb{R}} \\
				{\mathbb{R}P^2} & {S^1}
				\arrow["f"', from=2-1, to=2-2]
				\arrow["{\widetilde{f}}", from=2-1, to=1-2]
				\arrow["p", from=1-2, to=2-2]
			\end{tikzcd}\]
		and consider \(f = p \circ \widetilde{f} \) compose \hyperref[def:nullhomotopic]{nullhomotopy} with \(p\), so \(f\simeq \text{constant map} \).
		Specifically, since \(\pi _1(\mathbb{\MakeUppercase{r}} P^{2}) = \quotient{\mathbb{\MakeUppercase{z}}}{2 \mathbb{\MakeUppercase{z}}}\) and
		\(\pi _1(S^1) = \mathbb{\MakeUppercase{z}} \), hence
		\[
			f_\ast (\pi _1(\mathbb{\MakeUppercase{r}} P^{2} )) = 0
		\]
		since \(\mathbb{\MakeUppercase{z}} \) has no (nonzero) torsion. So it \hyperref[prop:homotopy-lifting-property]{lifts} by
		\autoref{prop:lifting-criterion}.
	\end{proof}
\end{eg}

Now we can proof \autoref{prop:lifting-criterion}.
\begin{proof}\let\qed\relax
	We prove two directions as follows.
	\paragraph{Necessary.} We see that we can \hyperref[def:factorization]{factorize} \(f_\ast\) as
	\[
		f_\ast = p_\ast \circ \widetilde{f} _\ast
	\]
	follows from the \hyperref[def:functor]{functoriality} of \(\pi _1\).
	\paragraph{Sufficient.} Let \(x\in X\). Choose a \hyperref[def:path]{path} \(\gamma\) from \(x_0\) to \(x\) by the assumption that \(X\) is \hyperref[def:path]{path}-connected.
	Then, \(f \gamma \) has a unique \hyperref[prop:homotopy-lifting-property]{lift} starting at \(\widetilde{y} _0\), denote by \(\widetilde{f\gamma}\).
	Now, define
	\[
		\widetilde{f} (x) = \widetilde{f \gamma } (1).
	\]
	Then, we need to check
	\begin{enumerate}
		\item \(\widetilde{f} \) is well-defined. Suppose \(\gamma , \gamma ^\prime \) are \hyperref[def:path]{paths} in \(X\) from \(x_0\)
		      to \(x\). We want to show
		      \[
			      \widetilde{f \gamma^\prime} (1) = \widetilde{f \gamma } (1).
		      \]
		      Since \(\gamma \cdot \overline{\gamma^\prime}\) is a loop in \(X\) at \(x_{0}\), we know that \([(f \gamma)\cdot (\overline{f \gamma ^\prime}) ]\) is a class of
		      loops in \(Y\) in \(\mathrm{Im} (f_\ast)\). By hypothesis, this class of loops is in \(\mathrm{Im} (p_*)\).
		      It \hyperref[prop:homotopy-lifting-property]{lifts} to a loop which is based at \(\widetilde{y} _0\). By uniqueness of
		      \hyperref[prop:homotopy-lifting-property]{lifts}, this loop lifting \((f \gamma )\cdot \overline{(f \gamma ^\prime )}\) to \(\widetilde{Y} \)
		      must be equal to the \hyperref[prop:homotopy-lifting-property]{lifts} \(\widetilde{f \gamma }\cdot \widetilde{\overline{f \gamma ^\prime }}  \)
		      with a common value at \(t = 1 / 2\). Hence, \(\widetilde{f \gamma }(1) = \widetilde{f \gamma ^\prime }(1)\) as desired, namely the endpoints agree.
		      \begin{figure}[H]
			      \centering
			      \incfig{pf:prop:lifting-criterion}
			      \label{fig:pf:prop:lifting-criterion}
		      \end{figure}
	\end{enumerate}
\end{proof}