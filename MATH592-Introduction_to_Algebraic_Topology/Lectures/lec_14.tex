\lecture{14}{7 Feb. 10:00}{Covering Spaces}
\section{Covering Spaces}
\subsection{Lifting Properties}
As always, we start with a definition.
\begin{definition}[Covering space]\label{def:covering-space}\label{def:covering-map}
	A \emph{covering space} \(\widetilde{X} \) of \(X\) is a space \(\widetilde{X} \) and a map \(p\colon \widetilde{X} \to X\)
	such that \(\forall x\in X\ \exists \text{ neighborhood } u_{x}\) with \(p^{-1} (u_{x})\) the disjoint union of open sets
	\[
		\coprod_\alpha u_\alpha
	\]
	such that
	\[
		\at{p}{u_\alpha }{} \colon u_{x}\to u_{x}
	\]
	is a homeomorphism for every \(\alpha \).
	\begin{figure}[H]
		\centering
		\incfig{def:covering-space}
		\label{fig:def:covering-space}
	\end{figure}
	We sometimes call \(p\) as \emph{covering map}.
\end{definition}

Although we already investigate into \hyperref[def:covering-space]{covering spaces} quite a lot in homework, but a terminology is still
worth mentioning.
\begin{definition}[Evenly covered]\label{def:evenly-covered}
	Let \(p\colon \widetilde{X} \to X\) be a continuous map of spaces. Then an open subset \(U\subseteq X\) is called \emph{evenly covered by \(p\) }
	if
	\[
		\at{p}{V_{i}}{} \colon V_{i}\to U
	\]
	is a homeomorphism.

	\par We call the parts \(V_{i}\) of the partition \(\coprod_i V_{i}\) of \(p^{-1} (U)\) \emph{slices}.
\end{definition}
\begin{remark}
	We see that \(p\) is a \hyperref[def:covering-map]{covering map} if and only if every point \(x\in X\) has a neighborhood which is
	\hyperref[def:evenly-covered]{evenly covered}.
\end{remark}

We immediately have the following proposition.
\begin{proposition}[Homotopy lifting property]\label{prop:homotopy-lifting-property}
	The \hyperref[def:covering-space]{covering spaces} satisfy the \emph{\hyperref[def:homotopy]{homotopy} lifting property} such that
	the following diagram commutes.
	\[
		\begin{tikzcd}
			{X\times \{0\}} && {\widetilde{Y} } \\
			\\
			{X\times I} && Y
			\arrow["{F_t}"', from=3-1, to=3-3]
			\arrow[hook, from=1-1, to=3-1]
			\arrow["{\widetilde{F}_0 }", from=1-1, to=1-3]
			\arrow["p", from=1-3, to=3-3]
			\arrow["{\exists ! \widetilde{F}_t }"{description}, from=3-1, to=1-3]
		\end{tikzcd}
	\]
\end{proposition}
\begin{proof}
	We already proved this in homework!
\end{proof}

\begin{corollary}
	For each \hyperref[def:path]{path} \(\gamma \colon I\to X\) in \(X\), \(\widetilde{x} _0\in p^{-1} (\gamma (0)) \) such that there exists a
	unique \hyperref[prop:homotopy-lifting-property]{lift} \(\widetilde{\gamma} \) starting at \(\widetilde{x} _0\).

	\par And for each \hyperref[def:homotopy-path]{path homotopy} \(I\times I\to X\), there exists a unique \hyperref[def:homotopy-path]{path homotopy}
	\(\widetilde{\gamma} \colon I\times I\to \widetilde{X} \) starting at \(\widetilde{x}_0\).
	\begin{figure}[H]
		\centering
		\incfig{col:lec14:1}
		\label{fig:col:lec14:1}
	\end{figure}
\end{corollary}

\begin{eg}
	Let see some examples.
	\begin{enumerate}
		\item Covers of \(S^1\vee S^1\).
		      \begin{figure}[H]
			      \centering
			      \incfig{eg:lec14:1}
			      \label{fig:eg:lec14:1}
		      \end{figure}
	\end{enumerate}
	Note that in each cover (those three on the top), the black dot is the preimage of \(\{x_0\}\), namely \(p_{i}^{-1} (\{x_0\})\).
	\begin{remark}
		We see that for each \(p_{i}^{-1} (\{x_0\})\), there are exactly
		\begin{itemize}
			\item one \(a\) edge goes out
			\item one \(b\) edge goes out
			\item one \(a\) edge goes in
			\item one \(b\) edge goes in
		\end{itemize}

		It turns out that there are much more covers of \(S^1\vee S^1\), as long as this main property is satisfied.
	\end{remark}
\end{eg}

\begin{proposition}
	Let
	\[
		p\colon (\widetilde{X} , \widetilde{x} _0)\to (X, x_0)
	\]
	be a \hyperref[def:covering-map]{covering map}. Then
	\begin{enumerate}
		\item \(p_*\colon \pi _1(\widetilde{X} , \widetilde{x} _0)\to \pi _1(X, x_0)\) is injective.
		\item \(p_*(\pi _1(\widetilde{X} , \widetilde{x} _0))\subseteq \pi _1(X, x_0) =
		      \left\{[\gamma ] \mid \text{\hyperref[prop:homotopy-lifting-property]{Lift} \(\widetilde{\gamma} \) starting at \(\widetilde{x} _0\) is a loop.} \right\}\).
	\end{enumerate}
\end{proposition}
\begin{proof}
	We prove this one by one.
	\begin{enumerate}
		\item Suppose \(\widetilde{\gamma} \in \pi _1(\widetilde{X} , \widetilde{x} _0)\) is in \(\mathrm{ker}(p_*) \). Then
		      \[
			      [\gamma] =p_*([\widetilde{\gamma} ]) = \left[p \circ \widetilde{\gamma} \right].
		      \]
		      Let \(\gamma _{t}\) be a \hyperref[def:nullhomotopic]{nullhomotopy} from \(\gamma \) to the constant loop \(c_{x_0}\ \mathrm{rel} \{0, 1\}\).
		      We can then \hyperref[prop:homotopy-lifting-property]{lift} \(\gamma _{t}\) to \(\widetilde{\gamma} _{t}\) where \(\widetilde{\gamma} _0 = \widetilde{\gamma} \).
		      Now, we claim that
		      \begin{itemize}
			      \item \(\widetilde{\gamma} \) is a \hyperref[def:homotopy-relative]{homotopy \(\mathrm{rel} \{0, 1\}\)}.
			      \item \(\widetilde{\gamma} _1\) is the constant loop \(c_{\widetilde{x} _0}\).
		      \end{itemize}
		      \[
			      \begin{tikzcd}
				      & {\widetilde{X}} \\
				      I & X
				      \arrow["{\widetilde{\gamma}}", from=2-1, to=1-2]
				      \arrow["p", from=1-2, to=2-2]
				      \arrow["\gamma"', from=2-1, to=2-2]
			      \end{tikzcd}\qquad
			      \begin{tikzcd}
				      & {\widetilde{X}} \\
				      {I\times I} & X
				      \arrow["{\widetilde{\gamma}_t}", from=2-1, to=1-2]
				      \arrow["p", from=1-2, to=2-2]
				      \arrow["{\gamma_t}"', from=2-1, to=2-2]
			      \end{tikzcd}
		      \]
		      We see that the above diagrams prove the first claim, since we know that the left and right edge of \(I\times I\) maps to \(x_0\) under
		      \(\gamma _{t}\), and \(c_{\widetilde{x} _0}\) \hyperref[prop:homotopy-lifting-property]{lifts} this, so by uniqueness \(t\mapsto \widetilde{\gamma} _t(0)\) and
		      \(t\mapsto \widetilde{\gamma} _{t}(1)\) must be constant \hyperref[def:path]{paths} at \(\widetilde{x} _0\) as desired.

		      \par Then the \hyperref[prop:homotopy-lifting-property]{lift} \(\widetilde{\gamma} _{t}\) is a \hyperref[def:homotopy-path]{homotopy of paths}
		      to the constant loop, so \([\widetilde{\gamma} ] = 1\).
		\item Let see an example to show the idea of the proof.
		      \begin{eg}
			      Given
			      \begin{figure}[H]
				      \centering
				      \incfig{eg:lec14:2}
				      \label{fig:eg:lec14:2}
			      \end{figure}
		      \end{eg}
		      Then
		      \[
			      p_*\pi _1 = \left< b, a^{2} , ab \overline{a}  \right> \subseteq \pi _1(X) = \left< a, b \mid \ \right>.
		      \]
	\end{enumerate}
\end{proof}
\begin{proposition}[Lifting criterion]\label{prop:lifting-criterion}
	Let
	\[
		p\colon (\widetilde{Y} , \widetilde{y} _0)\to (Y, y_0 )
	\]
	be \hyperref[def:covering-map]{covering map}. Given
	\begin{itemize}
		\item \(f\colon (X, x_0) \to (Y, y_0)\);
		\item \(X\) is \hyperref[def:path]{path}-connected, locally \hyperref[def:path]{path}-connected,
	\end{itemize}
	then a \hyperref[prop:homotopy-lifting-property]{lift}
	\[
		\widetilde{f} \colon (X, x_0)\to (\widetilde{Y} , y_0)
	\]
	exists if and only if
	\[
		f_*\left(\pi _1(X, x_0)\right)\subseteq p_*\left(\pi _1(\widetilde{Y} , \widetilde{y} _\alpha )\right).
	\]

	\[\begin{tikzcd}
			& {(\widetilde{Y}, \widetilde{y}_0)} \\
			{(X, x_0)} & {(Y, y_0)}
			\arrow["p", from=1-2, to=2-2]
			\arrow["f"', from=2-1, to=2-2]
			\arrow["{\exists \widetilde{f}}", dashed, from=2-1, to=1-2]
		\end{tikzcd}\qquad
		\begin{tikzcd}
			& {\pi_1(\widetilde{Y}, \widetilde{y}_0)} \\
			{\pi_1(X, x_0)} & {\pi_1(Y, y_0)}
			\arrow["{f_\ast}"', from=2-1, to=2-2]
			\arrow["{p_\ast}", from=1-2, to=2-2]
			\arrow["{\widetilde{f}_\ast}", from=2-1, to=1-2]
		\end{tikzcd}\]
\end{proposition}