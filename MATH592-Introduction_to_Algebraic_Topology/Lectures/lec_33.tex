\lecture{33}{30 Mar. 10:00}{Cellular Homology Examples}
\begin{eg}
	Let's do this with the torus
	\begin{figure}[H]
		\centering
		\incfig{eg:lec31:CW-complex-torus}
		\label{fig:eg:lec31:CW-complex-torus}
	\end{figure}

	The \hyperref[def:cellular-chain-complex]{chain complex} looks like
	\[
		\begin{tikzcd}
			0 & {\left<D\right>} & {\left<a, b\right>} & {\left<x\right>} & 0
			\arrow[from=1-1, to=1-2]
			\arrow[from=1-2, to=1-3]
			\arrow[from=1-3, to=1-4]
			\arrow[from=1-4, to=1-5]
		\end{tikzcd}
	\]
	Note that \(a \mapsto x - x = 0\) and \(b \mapsto x - x = 0\) and so \(\partial_1 = 0\). Now \(D\) is glued along \(aba^{-1}b^{-1}\), so we look at the composed up map

	\begin{figure}[H]
		\centering
		\incfig{eg:cellular-homology-calc-torus}
		\label{fig:eg:cellular-homology-calc-torus}
	\end{figure}
	We wind forwards then backwards around $a$, so the \hyperref[def:degree]{degree} is zero. The same thing happens for $b$ so
	\[
		\partial_2 D = 0 \cdot a + 0 \cdot b = 0.
	\]

	This gives a nice \textbf{principle}, namely if a \(2\)-cell \(D\) is glued down via some \hyperref[def:word]{words} \(w\) (this only makes sense for \(2\)-cells), then the
	coefficient to a letter \(b\) in \(\partial_2 D\) is the sum of the exponents of \(b\) in \(w\).

	Now we just have that the \hyperref[def:cellular-homology-group]{homology groups} are equal to the \hyperref[def:cellular-chain-complex]{chain groups} because the boundary maps are all zero.
\end{eg}

\begin{eg}
	A genus \(g\) surface \(\Sigma_g\) has the \hyperref[def:CW-Complex]{CW complex} structure as
	\begin{itemize}
		\item \(1\) \(0\)-cell \(x\).
		\item \(2g\) \(1\)-cells \(a_1, b_1, a_2, b_2, \ldots\).
		\item \(1\) \(2\)-cell \(D\) glued along \([a_1, b_2][a_2, b_2]\cdots[a_g, b_g]\) (a product of commutators)
	\end{itemize}
	We obtain the result
	\[
		\partial_1(a_i) = \partial_1(b_i) = x - x = 0.
	\]

	Furthermore, by the principle discussed above, we know that every \(1\)-cell appears once in the \hyperref[def:word]{word}, and its inverse appears once,
	so all the coefficients of \(1\)-cells in \(\partial_2(D)\) are zero, so \(\partial_2(D) = 0\). This means we have a \hyperref[def:cellular-chain-complex]{chain complex}
	\[
		\begin{tikzcd}
			0 & {\mathbb{Z}} & {\mathbb{Z}^{2g}} & {\mathbb{Z}} & 0
			\arrow[from=1-1, to=1-2]
			\arrow["0", from=1-2, to=1-3]
			\arrow["0", from=1-3, to=1-4]
			\arrow[from=1-4, to=1-5]
		\end{tikzcd}
	\]
	And so then we have that
	\[
		H_k(\Sigma _{g} ) = \begin{dcases}
			\mathbb{\MakeUppercase{z}} ,      & \text{ if } k = 0, 2 ; \\
			\mathbb{\MakeUppercase{z}} ^{2g}, & \text{ if } k = 1;     \\
			0,                                & \text{ otherwise}.
		\end{dcases}
	\]
\end{eg}

\begin{eg}[Torus example: \(\partial_2\) in more detail]
	We're going to work through this example a bit more carefully.
	\begin{figure}[H]
		\centering
		\incfig{eg:more-careful-torus-cellular}
		\label{fig:eg:more-careful-torus-cellular}
	\end{figure}
	Let's zoom in on these two preimage points and use \emph{local homology} to compute this:
	\begin{center}
		%\includegraphics[scale=0.5]{zoom-torus-cellular-preimage}
	\end{center}
\end{eg}