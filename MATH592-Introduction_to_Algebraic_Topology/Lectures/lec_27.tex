\lecture{27}{16 Mar. 10:00}{Excision}
Let's start with a theorem.
\begin{theorem}[Excision]\label{thm:excision}
	Suppose we have subspace \(Z \subseteq A \subseteq X\) such that \(\overline{Z} \subseteq \Int  (A)\). Then the inclusion
	\[
		(X - Z, A - Z) \hookrightarrow (X, A)
	\]
	induces isomorphisms
	\[
		H_n(X - Z, A - Z) \xrightarrow{\cong} H_n(X, A).
	\]
\end{theorem}
\begin{proof}[Proof Sketch]
	We first see an equivalent formulation of \autoref{thm:excision}.
	\begin{remark}
		Equivalently, for subspaces \(A, B \subseteq X\) whose interiors cover \(X\), the inclusion
		\[
			(B, A \cap B) \hookrightarrow (X, A)
		\]
		induces an isomorphism
		\[
			H_n(B, A \cap B) \xrightarrow{\cong} H_n(X, A)
		\]
	\end{remark}
	\begin{explanation}
		We see that this follows from
		\[
			B \coloneqq X \setminus Z,\quad Z = X \setminus B,
		\]
		then we see that \(A \cap B = A - Z\) and the condition requires from \autoref{thm:excision},
		\(\overline{Z} \subseteq \Int (A)\) is then equivalent to
		\[
			X = \Int (A) \cup \Int (B)
		\]
		since \(X \setminus \Int (B) = \overline{Z} \).
		\begin{figure}[H]
			\centering
			\incfig{eg:excision-1}
			\label{fig:eg:excision-1}
		\end{figure}
	\end{explanation}

	We now sketch the proof of the above equivalent form of \autoref{thm:excision}, which is notorious for being hairy.
	\begin{itemize}
		\item Given a \hyperref[def:relative-cycle]{relative cycle} \(x\) in \((X, A)\), subdivide the \hyperref[def:standard-simplex]{simplices} to make \(x\) a
		      linear combination of chains on \emph{smaller \hyperref[def:standard-simplex]{simplices}}, each contained in \(\Int (A)\) or \(X \setminus Z\).
		      \begin{figure}[H]
			      \centering
			      \incfig{pf:excision}
			      \caption{\(\Delta ^n\to X\) subdivide into \hyperref[def:subsimplex]{subsimplices} with images in. }
			      \label{fig:pf:excision}
		      \end{figure}
		      This means \(x\) is homologous to sum of \hyperref[def:subsimplex]{subsimplices} with images in \(\Int (A)\) or \(X \setminus Z\). One of the things we
		      use is that \hyperref[def:standard-simplex]{simplices} are compact, so this process takes finite time.

		      The key is that the \underline{Subdivision operator} is chain \hyperref[def:homotopic]{homotopic} to the identity.
		\item Since we are working \hyperref[def:relative-homology-group]{relative} to \(A\), the \hyperref[def:relative-chain-group]{chains} with image
		      in \(A\) are zero, thus we have a \hyperref[def:relative-cycle]{relative cycle} homologous to \(x\)
		      with all \hyperref[def:standard-simplex]{simplices} contained in \(X \setminus Z\).
	\end{itemize}
\end{proof}

\begin{exercise}
	Show that \(H_\ast(Y, y_0) \cong \widetilde{H}_\ast(Y)\).
\end{exercise}

\begin{theorem}\label{thm:good-pairs-relative-homology}
	For \hyperref[def:good-pair]{good pairs} \((X, A)\), the quotient map \(q \colon (X, A) \to \left(\quotient{X}{A} , \quotient{A}{A} \right)\)
	induces isomorphisms
	\[
		\begin{tikzcd}
			{q_\ast\colon H_n(X, A)} & {H_n(\quotient{X}{A}, \quotient{A}{A})\cong\widetilde{H}_n(\quotient{X}{A})}
			\arrow["\cong", from=1-1, to=1-2]
		\end{tikzcd}
	\]
	for all  \(n\).
\end{theorem}
\begin{proof}[Proof Sketch]
	Let \(A \subseteq V \subseteq X\) where \(V\) is a neighborhood of \(A\) that \hyperref[def:deformation-retraction]{deformation retracts} onto \(A\).
	Using \hyperref[thm:excision]{excision}, we obtain a commutative diagram
	\[
		\begin{tikzcd}
			{H_n(X, A)} & {H_n(X, V)} & {H_n(X-A, V-A)} \\
			{H_n(\quotient{X}{A}, \quotient{A}{A})} & {H_n(\quotient{X}{A}, \quotient{V}{A})} & {H_n(\quotient{X}{A}-\quotient{A}{A}, \quotient{V}{A}-\quotient{A}{A})}
			\arrow["{q_\ast}", from=1-1, to=2-1]
			\arrow["{\color{green}{\cong}}", from=2-1, to=2-2]
			\arrow["{\color{green}{\cong}}", from=1-1, to=1-2]
			\arrow["{\color{red}{\cong}}"', from=1-3, to=1-2]
			\arrow["{\color{red}{\cong}}"', from=2-3, to=2-2]
			\arrow["{q_\ast}", from=1-2, to=2-2]
			\arrow["{q_\ast}", from=1-3, to=2-3]
			\arrow["{\color{blue}{\cong}}"', draw=none, from=1-3, to=2-3]
		\end{tikzcd}
	\]

	\par Done if we can prove all the colored isomorphisms.
	\begin{itemize}
		\item \(\color{red}{\cong}\) is an isomorphism by \hyperref[thm:excision]{excision}.
		\item \(\color{blue}{\cong}\) is an isomorphism by direct calculation (since \(q\) is a homeomorphism on the complement of \(A\)).
		\item \(\color{green}{\cong}\) on Homework, since \(V\) \hyperref[def:deformation-retraction]{deformation retracts} to \(A\).
	\end{itemize}
	\begin{remark}
		The last equality is from the above exercise with \(\quotient{A}{A} = \{\ast\}\).
	\end{remark}
\end{proof}