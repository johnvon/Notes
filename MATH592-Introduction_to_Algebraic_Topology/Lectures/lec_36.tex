\lecture{36}{6 Apr. 10:00}{Lefschetz Fixed Point Theorem}
\begin{definition}[Lefschetz number]\label{def:Lefschetz-number}
	Let \(X\) be a space with the assumption that \(\bigoplus_k H_k(X)\) is finitely generated.\footnote{That is, each \hyperref[def:homology-group]{homology group}
		is finitely	generated, and there are finitely many nonzero \hyperref[def:homology-group]{homology groups}.
		For example \(X\) could be a finite \hyperref[def:CW-Complex]{CW complex}.}
	Then the \emph{Lefschetz number} \(\tau(f)\) of a map \(f \colon X \to X\) is
	\[
		\tau(f) \coloneqq \sum_k (-1)^k \trace(f_\ast \colon H_k(X) \to H_k(X)).
	\]
\end{definition}
\begin{remark}
	In particular, we can also write
	\[
		\trace(f_{\ast} \circlearrowright H_{k} (X)).
	\]
\end{remark}

\begin{eg}[Euler characteristic]
	When \(f \simeq \identity _X\), we have \(f_\ast = \identity_{H_k(X)}\) for all \(k\).
	Then
	\[
		\trace(f_\ast \colon H_k(X) \to H_k(X)) = \rank(H_k(X)).
	\]
	Hence, we further have
	\[
		\tau(f) = \sum_k (-1)^k\rank(H_k(X)) \eqqcolon \chi(X),
	\]
	where \(\chi(X)\) is the \emph{Euler characteristic}.
\end{eg}

\begin{theorem}[Lefschetz Fixed Point Theorem]\label{thm:Lefschetz-fixed-point}
	Suppose \(X\) admits a finite triangulation,\footnote{i.e. a finite \hyperref[def:simplicial-complex]{simplicial complex} structure}
	or more generally, \(X\) is a \hyperref[def:retraction]{retract} of a finite \hyperref[def:simplicial-complex]{simplicial complex}.
	If \(f \colon X \to X\) is a map with \(\tau(f) \neq 0\), then \(f\) has a fixed point.
\end{theorem}
\begin{note}
	Note that the converse does not hold. And in particular, we ahve
	\[
		\tau (f) = \sum\limits_{k} \trace(f_\# \circlearrowright C^{\mathrm{CW}}_k(X) ).
	\]
\end{note}
\begin{theorem}\label{thm:retract-simplicial-complex}
	If \(X\) is a compact, locally \hyperref[def:contractible]{contractible} space that can be embedded in \(\mathbb{R}^n\) for some \(n\), then \(X\) is a
	\hyperref[def:retraction]{retract} of a finite \hyperref[def:simplicial-complex]{simplicial complex}.
\end{theorem}
\begin{remark}
	This includes
	\begin{itemize}
		\item Compact Manifolds.
		\item Finite \hyperref[def:CW-Complex]{CW complexes}.
	\end{itemize}
\end{remark}
\begin{definition}\label{def:Lefschetz-number-better}
	Let \(\mathbb{F}\) be a field, and let \(H_k(X; \mathbb{F})\) be the \(k\)-th homology of \(X\) with coefficients in \(\mathbb{F}\).
	Then \(H_k(X; \mathbb{F})\) is always a vector space over \(\mathbb{F}\). Define \(\tau^{\mathbb{F}}(X)\) be
	\[
		\sum_k (-1)^k \trace(f_\ast \colon H_k(X; \mathbb{F}) \to H_k(X; \mathbb{F})).
	\]
\end{definition}
\begin{remark}
	The \hyperref[thm:Lefschetz-fixed-point]{Lefschetz fixed point theorem} still holds if we replace \(\tau(x) \neq 0\) with \(\tau^{\mathbb{F}} \neq 0\).
\end{remark}
