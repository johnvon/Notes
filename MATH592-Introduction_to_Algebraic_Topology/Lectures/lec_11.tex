\lecture{11}{31 Jan. 10:00}{Group Presentations}
\begin{eg}
	We now see some applications of \autoref{thm:Seifert-Van-Kampen-Theorem}.
	\begin{enumerate}
		\item We can use \hyperref[thm:Seifert-Van-Kampen-Theorem]{Seifert Van Kampen Theorem} to compute the \hyperref[def:fundamental-group]{fundamental group}
		      of \(S^{2}\). We see that
		      \begin{figure}[H]
			      \centering
			      \incfig{lec11-eg:2-sphere}
			      \label{fig:lec11-eg:2-sphere}
		      \end{figure}
		      We see that \(\pi _1(S^{2} )\) must be a quotient of \(\pi _1(A)\ast \pi _1(B)\), but since \(A, B\simeq D^{2} \), we know that
		      \(\pi _1(A)\) and \(\pi _1(B)\) are both zero groups, thus \(\pi _1(A)\ast \pi _1(B)\) is the zero group, and \(\pi _1(S^{2} )\) is
		      also the zero group.
		      \begin{remark}
			      Note that the inclusion of \(A\cap B\to A\) induces the zero map \(\pi _1(A\cap B)\to \pi _1(A)\), which cannot be an injection.
			      In fact, we know that \(\pi _1(A\cap B)\cong \mathbb{\MakeUppercase{z}} \) since \(A\cap B\simeq S^1\).
		      \end{remark}
		\item \label{lec11:eg:torus} In the case of torus, consider the following.
		      \begin{figure}[H]
			      \centering
			      \incfig{lec11-eg2:torus}
			      \caption{\(A\) is the interior, while \(B\) is the neighborhood of the boundary.}
			      \label{fig:lec11-eg2:torus}
		      \end{figure}
		      Now note that \(A\simeq D^{2} \) and \(B\simeq S^1\vee S^1\), and since it's a thickening of the two loops around the torus in
		      both ways, this suggests the question of how do we find \(\pi _1(B)\)? We grab a bit of knowledge from \hyperref[thm:Seifert-Van-Kampen-Theorem]{Seifert Van Kampen Theorem}
		      before we continue.

		      \hr

		      \begin{exercise}
			      Suppose we have \hyperref[def:path]{path}-connected spaces \((X_\alpha , x_\alpha )\), and we take their \hyperref[sssec:Wedge-sum]{wedge sum} \(\bigvee_\alpha X_\alpha \) by
			      identifying the points \(x_\alpha \) to a single point \(x\). We also suppose a mild condition for all \(\alpha\), the point
			      \(x_\alpha \) is a \hyperref[def:deformation-retraction]{deformation retract} of some neighborhood of \(x_\alpha \).

			      \par For example, this doesn't work if we choose the \emph{bad point} on the Hawaiian earring. Then we can use \hyperref[thm:Seifert-Van-Kampen-Theorem]{Seifert Van Kampen Theorem}
			      to show that
			      \[
				      \pi _1\left(\bigvee_\alpha X_\alpha , x\right) \cong \underset{\alpha }{\ast}\pi _1\left(X_\alpha , x_\alpha \right).
			      \]
			      \begin{proof}
				      If we denote
				      \begin{figure}[H]
					      \centering
					      \incfig{eg-2:Seifert-Van-Kampen-Theorem}
					      \label{fig:eg-2:Seifert-Van-Kampen-Theorem}
				      \end{figure}
				      as \(C_n\), then \(\pi _1(C_n)\cong F_n\). Then we apply \autoref{thm:Seifert-Van-Kampen-Theorem} to \(A_\alpha = X_\alpha \cup_{\beta }U_\beta \)
				      Specifically, take \(A_\alpha = X_\alpha \cup_\beta U_\beta \simeq X_\alpha \), where \(U_\beta \) is a neighborhood of \(x_\beta \) which
				      \hyperref[def:deformation-retraction]{deformation retracts} to \(x_\beta \). This makes \(A_\alpha \) open as desired.
			      \end{proof}
		      \end{exercise}

		      \begin{corollary}
			      The \hyperref[sssec:Wedge-sum]{wedge sum} of circles \(\pi _1(\bigvee_{\alpha \in A}S^1) = \ast_\alpha \mathbb{\MakeUppercase{z}} \) is a \hyperref[def:free-group]{free group} on \(A\).
			      In particular, when \(A\) is finite, the \hyperref[def:fundamental-group]{fundamental group} of a bouquet of circles is the \hyperref[def:free-group]{free group}
			      on \(\left\vert A \right\vert \).
		      \end{corollary}

		      \hr
		      Returning to the \hyperref[lec11:eg:torus]{example of torus}, we see that
		      \begin{itemize}
			      \item \(\pi _1(A) = 0\)
			      \item \(\pi _1(B) = \pi _1(S^1\vee S^1) = \mathbb{\MakeUppercase{z}} \ast \mathbb{\MakeUppercase{z}}  = F_2\)
			      \item \(\pi _1(A\cap B) = \pi _1(S^1) = \mathbb{\MakeUppercase{z}} \)
		      \end{itemize}

		      Further, we know that \(\pi _1(A\cap B)\to \pi _1(A)\) is the zero map. We need to understand \(\pi_1(A\cap B)\to \pi _1(B)\). To do so we
		      need to understand how we're able to identify \(\pi _1(S^1\vee S_1)\) with \(F_2\) and how we identify \(\pi _1(S^1)\) with \(\mathbb{\MakeUppercase{z}} \).
		      We update our \autoref{fig:lec11-eg2:torus} to talk about this.
		      \begin{figure}[H]
			      \centering
			      \incfig{lec11-eg2:torus-ver2}
			      \label{fig:lec11-eg2:torus-ver2}
		      \end{figure}
		      From this, we have
		      \[
			      \pi _1(A\cap B) \to \pi _1(B)\cong F_{a, b},\quad \gamma \mapsto aba^{-1} b^{-1}.
		      \]
		      By \hyperref[thm:Seifert-Van-Kampen-Theorem]{Seifert Van Kampen Theorem}, we identify the image of \(\gamma \) in \(\pi _1(B)[aba^{-1} b^{-1} ]\) with
		      its image in \(\pi _1(A)\), which is just trivial. Therefore, we have
		      \[
			      \pi _1(T^2) = \quotient{F_{a, b}}{\left< aba^{-1} b^{-1}  \right> } \cong \mathbb{\MakeUppercase{z}} ^2.
		      \]
		\item Let's see the last example which illustrate the power of \hyperref[thm:Seifert-Van-Kampen-Theorem]{Seifert Van Kampen Theorem}. Start with a torus, and
		      we glue in two disks into the hollow inside.
		      \begin{figure}[H]
			      \centering
			      \incfig{lec11:eg-3:1}
			      \label{fig:lec11:eg-3:1}
		      \end{figure}
		      We'll call this space \(X\), and out goal is to find \(\pi_1(X)\). We can place a \hyperref[def:CW-Complex]{CW complex} structure on this
		      space so that each disk is a \hyperref[def:CW-subcomplex]{subcomplex}. Then, we take quotient of each disk to a point without changing the \hyperref[def:homotopy-type]{homotopy type},
		      hence \(X\) is \hyperref[def:homotopy]{homotopy} to
		      \begin{figure}[H]
			      \centering
			      \incfig{lec11:eg-3:2}
			      \label{fig:lec11:eg-3:2}
		      \end{figure}
		      By the same property, we can expand one of those points into an interval, and then contract the red \hyperref[def:path]{path} as follows.
		      \begin{figure}[H]
			      \centering
			      \incfig{lec11:eg-3:3}
			      \label{fig:lec11:eg-3:3}
		      \end{figure}
		      This is exactly \(S^{2} \vee S^{2} \vee S^1\). With \hyperref[thm:Seifert-Van-Kampen-Theorem]{Seifert Van Kampen Theorem}, we have
		      \[
			      \pi _1(X) = \pi _1(S^{2} \vee S^{2} \vee S^1) = 0\ast 0\ast \mathbb{\MakeUppercase{z}} \cong \mathbb{\MakeUppercase{z}}.
		      \]
	\end{enumerate}
\end{eg}

\begin{exercise}
	Consider \(\mathbb{\MakeUppercase{r}} ^2 \setminus \{x_1, \ldots , x_n \}\), that is the plane punctured at \(n\) points. Then \(X \simeq \bigvee_n S^1\), so then
	\[
		\pi _1(X)\simeq F_n.
	\]
	One way to do this is to convince yourself that you can do a \hyperref[def:deformation-retraction]{deformation retract} the plane onto the following \hyperref[sssec:Wedge-sum]{wedge}.
	\begin{figure}[H]
		\centering
		\incfig{lec11:ex}
		\caption{\hyperref[def:deformation-retraction]{Deformation retract} \(X\) onto \hyperref[sssec:Wedge-sum]{wedge}.}
		\label{fig:lec11:ex}
	\end{figure}
\end{exercise}

\subsection{Group Presentation}
In order to go further, we introduce the concept of \emph{group presentation}.
\begin{definition}[Group presentation]\label{def:group-presentation}
	A \emph{presentation} \(\left< S \mid R \right> \) of a group \(G\) is
	\begin{itemize}
		\item \(S\): set of \emph{generators}
		\item \(R\): set of \emph{relaters} (\hyperref[def:word]{words} in a generator and inverses)
	\end{itemize}
	such that
	\[
		G\cong \quotient{F_S}{\left< R \right>},
	\]
	where \(\left< R \right> \) is a subgroup normally generated by the elements of \(R\).
\end{definition}

\begin{definition}[Finite presentation]\label{def:finite-presentation}
	If \(S\) and \(R\) are both finite, then \(G = \left< S \mid R \right> \) is a \emph{finite presentation} if \(S, R\) are, and we say that \(G\) is \emph{finitely presented}.
\end{definition}
\begin{note}
	One way to think about whether \(G\) is \hyperref[def:finite-presentation]{\underline{finitely presented}} is that if \(r\) is a
	\hyperref[def:word]{word} in \(R\) then \(r = 1\), where \(1\) is the identity of \(G\).
\end{note}

\begin{eg}
	We see that
	\begin{enumerate}
		\item \(F_2 = \left< a, b \mid\ \right> \)
		\item \(\mathbb{\MakeUppercase{z}} ^2 = \left< a, b \mid aba^{-1} b^{-1}  \right> = \quotient{\left< a, b \right> }{\overline{\{aba^{-1} b^{-1} \}}} \)
		\item \(\quotient{\mathbb{\MakeUppercase{z}}}{3\mathbb{\MakeUppercase{z}}} = \left< a \mid a^3 \right> \)
		\item \(S_3 = \left< a, b \mid a^2, b^2, (ab)^3 \right> \)
	\end{enumerate}
\end{eg}

\begin{theorem}
	Any group \(G\) has a \hyperref[def:group-presentation]{presentation}.
\end{theorem}
\begin{proof}
	We first choose a generating set \(S\) for \(G\). Notice that we can even choose \(S = G\) directly. From the universal property of \hyperref[def:free-group]{free group},
	we see that there exists a surjective map \(\varphi \colon F_S \to G, s \mapsto s\). Now, let \(R\) be the generating set for \(\mathrm{ker}(\varphi) \),
	by the first isomorphism theorem\footnote{\url{https://en.wikipedia.org/wiki/Isomorphism_theorems}}, \(G\cong \quotient{F_S}{\mathrm{ker} \varphi}\).
	In fact, we have \(G = \left< S \mid R \right> \).

	Specifically, \(i\colon S\to G\) with \(\iota \colon S\to F_S\), we have \(\varphi \circ \iota = i\).
	\[
		\begin{tikzcd}
			S \ar[rd, "i"']\ar[r, "\iota "] & F_S\ar[d, dashed, "\exists !\varphi"]\\
			& G
		\end{tikzcd}
	\]
\end{proof}

\begin{remark}
	The advantages of using \hyperref[def:group-presentation]{group presentation} are that given \(G = \left< S \mid R \right> \), it's now easy to define
	a homomorphism \(\psi \colon G\to H\) given a map \(\varphi \colon S\to H\), \(\psi \) extends to a group homomorphism \(G\to H\) if and only if \(\psi \)
	vanishes on \(R\), i.e., \(\psi (r) = 0\) for all \(r\in R\). We see an example to illustrate this.

	\begin{eg}
		If we have \(G = \left< a, b \mid aba \right> \), a map \(\varphi \colon \{a, b\}\to H\) gives a group homomorphism if and only if
		\[
			\varphi (aba) = \varphi (a)\varphi (b)\varphi (a) = 1_H.
		\]
		This essentially uses the universal property of quotients.
	\end{eg}
\end{remark}

\hr

\begin{remark}
	It's sometimes easy to calculate \(G^{\mathrm{Ab}}\)
	\[
		G^{\mathrm{Ab}} = \left< S \mid R, \text{commutators in \(S\)}\right>.
	\]
	\begin{eg}
		Suppose all relations in \(R\) are commutators, so \(R\subseteq[G, G]\). Then,
		\[
			G^{\mathrm{Ab}} = (F_S)^{\mathrm{Ab}} = \bigoplus_S \mathbb{\MakeUppercase{z}}.
		\]
	\end{eg}
\end{remark}
\begin{remark}
	The disadvantages are that this is computationally \textbf{very difficult}.
\end{remark}
\hr

\begin{eg}
	Given \(\mathbb{\MakeUppercase{z}} ^2 = \left< a, b \mid aba^{-1} b^{-1}  \right> \), let
	\[
		\psi \colon \{a, b\}\to H
	\]
	extends to a homomorphism if and only if
	\[
		\psi (a)\psi (b)\psi (a)^{-1} \psi (b)^{-1} = 1_H\in H.
	\]
	Namely, this is a \hyperref[def:group-presentation]{presentation} of the trivial group, but this is entirely unclear.
\end{eg}