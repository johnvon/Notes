\lecture{11}{31 Jan. 10:00}{Group Presentation}
\begin{eg}
	We now see some applications. Given spaces \(\{X_\alpha \}\) \(\overline{w} \) basepoints \(x_\alpha \). Now, consider
	the wedge sum \(\bigvee\limits_\alpha X_\alpha \). Suppose \(\forall \alpha\), \(x_\alpha \) is a deformation retract of
	some neighborhood \(u_\alpha \) of \(x_\alpha \). Then,
	\[
		\pi _1\left(\bigvee_\alpha X_\alpha , x_\alpha \right) \cong \underset{\alpha }{\ast}\pi _1\left(X_\alpha , x_\alpha \right).
	\]
	In particular, if we denote
	\begin{figure}[H]
		\centering
		\incfig{eg-2:Seifert-Van-Kampen-Theorem}
		\label{fig:eg-2:Seifert-Van-Kampen-Theorem}
	\end{figure}

	as \(C_n\), then \(\pi _1(C_n)\cong F_n\). Then we apply \autoref{thm:Seifert-Van-Kampen-Theorem} to \(A_\alpha = X_\alpha \cup_{\beta }u_\beta \)
\end{eg}

\subsection{Group Presentation}
In order to go further, we introduce the concept of \emph{group presentation}.
\begin{definition}[Group presentation]\label{def:group-presentation}
	A \emph{presentation} \(\left< S \mid R \right> \) of a group \(G\) is
	\begin{itemize}
		\item \(S\): set of \emph{generators}
		\item \(R\): set of \emph{relaters} (words in a generator and inverses)
	\end{itemize}
	such that
	\[
		G\cong \quotient{F_S}{\left< R \right>},
	\]
	where \(\left< R \right> \) is a subgroup normally generated by the elements of \(R\).

	\par Notice that \(\left< S \mid R \right> \) is \underline{finite} if \(S, R\) are, and \(G\) is \emph{finitely presented} if
	there exists a finite presentation.
\end{definition}

\begin{note}
	One way to think about whether \(G\) is \underline{finitely presented} is that if \(r\) is a word in \(R\) then \(r = 1\), where
	\(1\) is the identity of \(G\).
\end{note}

\begin{eg}
	We see that
	\begin{enumerate}
		\item \(F_2 = \left< a, b \mid\ \right> \)
		\item \(\mathbb{\MakeUppercase{z}} ^2 = \left< a, b \mid aba^{-1} b^{-1}  \right> \)
		\item \(\quotient{\mathbb{\MakeUppercase{z}}}{3\mathbb{\MakeUppercase{z}}} = \left< a \mid a^3 \right> \)
		\item \(S_3 = \left< a, b \mid a^2, b^2, (ab)^3 \right> \)
	\end{enumerate}
\end{eg}

\begin{theorem}
	Any group \(G\) has a presentation.
\end{theorem}
\begin{proof}
	We first choose a generating set \(S\) for \(G\). Notice that we can even choose \(S = G\) directly. From the universal property of free group,
	we see that there exists a surjective map \(\varphi \colon F_S \to G, s \mapsto s\). Now, let \(R\) be the generating set for \(\mathrm{ker}(\varphi) \),
	by the first isomorphism theorem\footnote{\url{https://en.wikipedia.org/wiki/Isomorphism_theorems}}, \(G\cong \quotient{F_S}{\mathrm{ker} \varphi}\), hence \(G = \left< S \mid R \right> \).
\end{proof}

\hr
\begin{remark}
	The advantages are that given \(G = \left< S \mid R \right> \), it's now easy to define a homomorphism \(\psi \colon G\to H\) given a map
	\(\varphi \colon S\to H\), \(\psi \) extends to a group homomorphism \(G\to H\) if and only if \(\psi \) vanishes on \(R\), i.e., \(\phi (r) = 0\)
	for all \(r\in R\). We see an example to illustrate this.

	\begin{eg}
		If we have \(G = \left< a, b \mid aba \right> \), a map \(\varphi \colon \{a, b\}\to H\) gives a group homomorphism if and only if
		\[
			\varphi (aba) = \varphi (a)\varphi (b)\varphi (a) = 1_H.
		\]
		This essentially uses the universal property of quotients.
	\end{eg}
\end{remark}

\hr

\begin{remark}
	It's sometimes easy to calculate \(G^{\mathrm{Ab}}\)
	\[
		G^{\mathrm{Ab}} = \left< S \mid R, \text{commutators in \(S\)}\right>.
	\]
	\begin{eg}
		Suppose all relations in \(R\) are commutators, so \(R\subseteq[G, G]\). Then,
		\[
			G^{\mathrm{Ab}} = (F_S)^{\mathrm{Ab}} = \bigoplus_S \mathbb{\MakeUppercase{z}}.
		\]
	\end{eg}
\end{remark}
\begin{remark}
	The disadvantages are that, the computationally \textbf{very difficult}.
\end{remark}
\hr

\begin{eg}
	Given \(\mathbb{\MakeUppercase{z}} ^2 = \left< a, b \mid aba^{-1} b^{-1}  \right> \), let
	\[
		\psi \colon \{a, b\}\to H
	\]
	extends to a homomorphism if and only if
	\[
		\psi (a)\psi (b)\psi (a)^{-1} \psi (b)^{-1} = 1_H\in H.
	\]
	Namely, this is a presentation of the trivial group, but this is entirely unclear.
\end{eg}