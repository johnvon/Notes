\lecture{11}{31 Jan. 10:00}{}
\begin{eg}
	We now see some applications. Given spaces \(\{X_\alpha \}\) \(\overline{w} \) basepoints \(x_\alpha \). Now, consider
	the wedge sum \(\bigvee\limits_\alpha X_\alpha \). Suppose \(\forall \alpha\), \(x_\alpha \) is a deformation retract of
	some neighborhood \(u_\alpha \) of \(x_\alpha \). Then,
	\[
		\pi _1\left(\bigvee_\alpha X_\alpha , x_\alpha \right) \cong \underset{\alpha }{\ast}\pi _1\left(X_\alpha , x_\alpha \right).
	\]
	In particular, if we denote
	\begin{figure}[H]
		\centering
		\incfig{eg:Seifert-Van-Kampen-Thm}
		\label{fig:eg:Seifert-Van-Kampen-Thm}
	\end{figure}

	as \(C_n\), then \(\pi _1(C_n)\cong F_n\). Then we apply \autoref{thm:Seifert-Van-Kampen-Theorem} to \(A_\alpha = X_\alpha \cup_{\beta }u_\beta \)
\end{eg}

\subsection{Group Presentation}
In order to go further, we introduce the concept of \emph{group presentation}.
\begin{definition}[Group presentation]\label{def:group-presentation}
	A \emph{presentation} \(\left< S \mid R \right> \) of a group \(G\) is
	\begin{itemize}
		\item \(S\): set of generators
		\item \(R\): set of relaters (words in a generator and inverses)
	\end{itemize}
	such that
	\[
		G\cong \quotient{F_S}{\left< R \right>},
	\]
	where \(\left< R \right> \) is a subgroup normally generated.

	\par Notice that \(\left< S \mid R \right> \) is finite if \(S, R\) are, and \(G\) is \emph{finitely presented} if
	there exists a finite presentation.
\end{definition}
\begin{eg}
	We see that
	\begin{enumerate}
		\item \(F_2 = \left< a, b \mid\ \right> \)
		\item \(\mathbb{\MakeUppercase{z}} ^2 = \left< a, b \mid aba^{-1} b^{-1}  \right> \)
		\item \(\quotient{\mathbb{\MakeUppercase{z}}}{3\mathbb{\MakeUppercase{z}}} = \left< a \mid a^3 \right> \)
		\item \(S_3 = \left< a, b \mid a^2, b^2, (ab)^3 \right> \)
	\end{enumerate}
\end{eg}

\begin{theorem}
	Any group \(G\) has a presentation.
\end{theorem}
\begin{proof}
	We first choose a generating set \(S\) for \(G\). From the universal property of free group, we see that there exists a surjective map
	\(\varphi \colon F_S \to G, s \mapsto s\). Now, let \(R\) be the generating set for \(\mathrm{ker}(\varphi) \), \(G = \left< S \mid R \right> \).
\end{proof}

\begin{remark}
	The advantages are that given \(\left< S \mid R \right> \), it's now easy to define a homomorphism \(\psi \colon G\to H\) given a map
	\(\psi \colon S\to H\), \(\psi \) extends to a group homomorphism \(G\to H\) if and only if \(\psi \) vanishes on \(R\).
	\begin{eg}
		Given \(\mathbb{\MakeUppercase{z}} ^2 = \left< a, b \mid aba^{-1} b^{-1}  \right> \), let
		\[
			\psi \colon \{a, b\}\to H
		\]
		extends to a homomorphism if and only if
		\[
			\psi (a)\psi (b)\psi (a)^{-1} \psi (b)^{-1} = 1\in H.
		\]
	\end{eg}

	\par It's sometimes easy to calculate \(G^{\mathrm{Ab}}\)
	\[
		G^{\mathrm{Ab}} = \left< S \mid R, \text{commutators in \(S\)}\right>.
	\]

	\par The disadvantages are that, the computationally \textbf{very difficult}.
\end{remark}

\subsection{Presentations for \(\pi _1\)}
