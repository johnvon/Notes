\lecture{22}{25 Feb. 10:00}{Calculation of Homology}
\subsection{Calculation of Homology}
We start from some calculation about \hyperref[def:homology-group]{homology group} of some spaces.
\begin{eg}
	Let \(X = \mathbb{\MakeUppercase{r}} P^{2} \).
	\begin{figure}[H]
		\centering
		\incfig{eg:homology-RP2}
		\label{fig:eg:homology-RP2}
	\end{figure}
	We see that we have
	\begin{itemize}
		\item \(C_0 = \mathbb{\MakeUppercase{z}} \left< v, w \right> \)
		\item \(C_1 = \mathbb{\MakeUppercase{z}} \left< a, b, c \right> \)
		\item \(C_2 = \mathbb{\MakeUppercase{z}} \left< A, B \right> = \mathbb{\MakeUppercase{z}} A \oplus \mathbb{\MakeUppercase{z}} B\)
	\end{itemize}
	The \hyperref[def:chain-complex]{chain complex} is then
	\[
		\begin{tikzcd}
			0 & {C_2} & {C_1} & {C_0} & 0
			\arrow["{\partial_3}", from=1-1, to=1-2]
			\arrow["{\partial_2}", from=1-2, to=1-3]
			\arrow["{\partial_1}", from=1-3, to=1-4]
			\arrow["{\partial_0}", from=1-4, to=1-5]
		\end{tikzcd}
	\]
	Where
	\[
		\partial_2: \begin{dcases}
			A & \mapsto b - c + a \\
			B & \mapsto -a-c-b    \\
		\end{dcases},\qquad \partial_1: \begin{dcases}
			a & \mapsto w - v     \\
			b & \mapsto v - w     \\
			c & \mapsto v - v = 0 \\
		\end{dcases}
	\]
	We can also calculate the image and the kernel of \(C_{i} \), i.e.,
	\[
		\begin{alignedat}{3}
			C_2&: \mathrm{Im} = 0, &\mathrm{ker} &= 0, \\
			C_1&: \mathrm{Im} = \left< 2c, b - c + a \right> , \qquad&\mathrm{ker} &= \left< b +a, c \right>,\\
			C_0&: \mathrm{Im} = \left< v, w\right> , &\mathrm{ker} &= \left< v - w\right>.
		\end{alignedat}
	\]
	Hence,
	\[
		\begin{split}
			H_0 &\cong \quotient{\mathbb{\MakeUppercase{z}} \left< v, w \right> }{\mathbb{\MakeUppercase{z}} \left< v - w \right> } \cong \mathbb{\MakeUppercase{z}}\\
			H_1 &\cong \quotient{\mathbb{\MakeUppercase{z}} \left< b + a, c \right> }{\mathbb{\MakeUppercase{z}} \left< 2c, b+a-c \right> } \cong \quotient{\mathbb{\MakeUppercase{z}} \left< b + a - c, c \right> }{\mathbb{\MakeUppercase{z}} \left< 2c, b+a-c \right> } \cong \quotient{\mathbb{\MakeUppercase{z}} }{2\mathbb{\MakeUppercase{z}}}\\
			H_2 &= 0\\
		\end{split}
	\]
\end{eg}

\begin{remark}
	Warning! Care is needed when doing \emph{change of bases} over \(\mathbb{\MakeUppercase{z}} \). For example,
	\[
		\mathbb{\MakeUppercase{z}} \left< v, w \right> \quad \begin{dcases}
			v - w, & \text{ if }  ; \\
			v + w, & \text{ if }  .
		\end{dcases},\quad
		\begin{bmatrix}
			-1 & 1 \\
			1  & 1 \\
		\end{bmatrix}
	\]
\end{remark}