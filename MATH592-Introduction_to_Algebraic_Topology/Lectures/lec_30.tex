\lecture{30}{23 Mar. 10:00}{Degree}
With the definition of \hyperref[def:degree]{degree} and some of its \hyperref[rmk:property-of-degree]{properties}, we have the following theorems.
\begin{theorem}[Hairy ball theorem]\label{thm:hairy-ball-theorem}
	The sphere \(S^n\) admits a nonvanishing continuous tangent vector field if and only if \(n\) is odd.
\end{theorem}
\begin{proof}
	Recall that a \underline{tangent vector field} to the unit sphere \(S^n\subseteq \mathbb{\MakeUppercase{r}} ^{n+1}\) is a continuous map
	\[
		v \colon S^n \to \mathbb{\MakeUppercase{r}} ^{n+1}
	\]
	such that \(v(x)\) is tangent to \(S^n\) at \(x\), i.e., \(v(x)\) is perpendicular to the vector \(x\) for each \(x\). Let \(v(x)\) be a
	nonvanishing tangent vector field on the sphere \(S^n\), then we define
	\[
		f_t(x) \coloneqq \cos (\pi t) + \sin (\pi t)\left(\frac{v(x)}{\left\lVert v(x)\right\rVert }\right),
	\]
	which is a \hyperref[def:homotopy]{homotopy} from the identity map \(\identity_{S^n}\colon S^n \to S^n \) to the antipodal map
	\(-\identity_{S^n}\colon S^{n}\to S^n\). This simply follows from varying \(t\) from \(0\) to \(1\), where we have
	\[
		f_0(x) = \cos (0) x + \sin (0) \left(\frac{v(x)}{\left\lVert v(x)\right\rVert }\right) = x \implies f_0 = \identity_{S^n} ,
	\]
	while
	\[
		f_1(x) = \cos (\pi )x + \sin (\pi )\left(\frac{v(x)}{\left\lVert v(x)\right\rVert }\right) = - x \implies f_{1} = - \identity_{S^n}.
	\]
	The last thing needs to be verified is that \(f_t(x)\) is continuous, but this is trivial.


	From the \hyperref[rmk:property-of-degree]{property of degree}, we know that it's a \hyperref[def:homotopy]{homotopy} invariant, hence
	\[
		\deg (-\identity_{S^n}) = \deg (\identity_{S^n} ),
	\]
	which implies
	\[
		(-1)^{n+1} = 1,
	\]
	so \(n\) must be odd.

	Conversely, if \(n\) is odd, say \(n=2k-1\), we can define
	\[
		v(x_1, x_2, \ldots , x_{2k-1}, x_{2k})= (-x_2, x_1, \ldots , -x_{2k}, x_{2k-1}).
	\]
	Then \(v(x)\) is orthogonal to \(x\), so \(v\) is a tangent vector field on \(S^n\), and \(\left\vert v(x) \right\vert = 1\) for all \(x\in S^n\).
\end{proof}

\begin{theorem}[Groups acting on \(S^{2n}\)]\label{thm:actions-on-spheres}
	If \(G\) acts on \(S^{2n}\) \hyperref[def:free-group]{freely}, then
	\[
		G = \quotient{\mathbb{Z}}{2\mathbb{Z}} \text{ or } 1.
	\]
\end{theorem}
\begin{proof}
	There exists a homomorphism given by
	\[
		\begin{split}
			G & \to \{\pm 1\}        \\
			g & \mapsto \deg(\tau_g)
		\end{split}
	\]
	Where \(\tau_g\) is the action of \(g \in G\) on \(S^{2n}\) as a map \(S^{2n} \to S^{2n}\). We know this map is well-defined since \(\tau_g\) is invertible
	(simply take \(\tau_{g^{-1}}\)) for each \(g \in G\). Our note on composites shows this is a homomorphism.

	We want to show that the kernel is trivial, since then by the first isomorphism theorem \(G \cong \im\), and the image is either trivial or
	\(\quotient{\mathbb{Z}}{2\mathbb{Z}}\). Suppose that \(g\) is a nontrivial element of \(G\), then since \(G\) acts \hyperref[def:free-group]{freely} we know that
	\(\tau_g\) has no fixed points. With this in mind we have
	\[
		\deg \tau_g = (-1)^{2n + 1} = - 1.
	\]
	Thus, \(g \not\in \ker\), hence the kernel is trivial as desired.
\end{proof}

\begin{corollary}
	\(S^{2n}\) has only the trivial \hyperref[def:covering-space]{cover} \(S^{2n} \to S^{2n}\) or degree
	\(2\) \hyperref[def:covering-space]{cover} (for example, \(S^{2n} \to \mathbb{R}P^{2n}\)).
\end{corollary}
\begin{proof}
	This follows since any \hyperref[def:covering-space]{covering space} action acts \hyperref[def:free-group]{freely}.
\end{proof}

\begin{definition}[Local degree]\label{def:local-degree}
	Let \(f \colon S^n \to S^n\) (\(n > 0\)). Suppose there exists \(y \in S^n\) such that \(f^{-1}(y)\) is finite, say, \(\{x_1, \ldots, x_m\}\).
	Then let \(U_1, \ldots, U_m\) be disjoint neighborhoods of \(x_1, \ldots, x_m\) that are mapped by \(f\) to some neighborhood \(V\) of \(y\).
	\begin{figure}[H]
		\centering
		\incfig{def:local-degree}
		\label{fig:def:local-degree}
	\end{figure}

	The \emph{local degree} of \(f\) at \(x_i\), denote as \(\at{\deg f}{x_i}{}\), is the \hyperref[def:degree]{degree} of the map
	\[
		f_\ast \colon \mathbb{Z} \cong H_n(U_i, U_i - \{x_i\}) \to H_n(V, V - \{y\}) \cong \mathbb{Z}.
	\]
\end{definition}
\begin{remark}
	The homomorphism \(f_\ast\) is a multiplication by an integer, which is the \hyperref[def:local-degree]{local degree} as we just defined,
	arises from the following natural diagram.
	\[
		\begin{tikzcd}
			& {H_n(U_i, U_i-\{x_i\})} & {H_n(V, V-\{y\})} \\
			{H_n(S^n,S^n-\{x_i\})} & {H_n(S^n,S^n-f^{-1}(y))} & {H_n(S^n, S^n-\{y\})} \\
			& {H_n(S^n)} & {H_n(S^n)}
			\arrow["\cong"', from=1-2, to=2-1]
			\arrow["\cong", from=3-2, to=2-1]
			\arrow["j"', from=3-2, to=2-2]
			\arrow["{k_i}", hook, from=1-2, to=2-2]
			\arrow["{p_i}"', hook', from=2-2, to=2-1]
			\arrow["{f_\ast}", from=1-2, to=1-3]
			\arrow["{f_\ast}", from=2-2, to=2-3]
			\arrow["{f_\ast}", from=3-2, to=3-3]
			\arrow["\cong", from=1-3, to=2-3]
			\arrow["\cong"', from=3-3, to=2-3]
		\end{tikzcd}
	\]
	The two isomorphisms in the upper half come from \hyperref[thm:excision]{excision}, and the lower two isomorphisms come from
	\hyperref[thm:long-exact-sequence-of-a-pair]{exact sequences of pairs}.
\end{remark}

\begin{theorem}\label{thm:calculation-with-local-degree}
	Let \(f \colon  S^n \to S^n\) with \(f^{-1}(y) = \{x_1, \ldots, x_m\}\) as in \autoref{def:local-degree}, then we have
	\[
		\deg f = \sum_{i = 1}^m \at{\deg f}{x_i}{}.
	\]
\end{theorem}
\begin{remark}
	Thus, we can compute the \hyperref[def:degree]{degree} of \(f\) by computing these \hyperref[def:local-degree]{local degrees}.
\end{remark}

Let's work with some examples for our edification.
\begin{eg}
	Consider \(S^n\) and choose \(m\) disks in \(S^n\). Namely, we first collapse the complement of the \(m\) disks to a point, and then we identify each of
	the \hyperref[CW-complex-wedge-sum]{wedged \(n\)-spheres} with the \(n\)-sphere itself.
	\begin{figure}[H]
		\centering
		\incfig{eg:degree-m-map-disks}
		\label{fig:eg:degree-m-map-disks}
	\end{figure}
	The result will be a map of \hyperref[def:degree]{degree} \(m\). We can see this by computing \hyperref[def:local-degree]{local degree}.
	\begin{figure}[H]
		\centering
		\incfig{eg:degree-m-map-disks-computation}
		\label{fig:eg:degree-m-map-disks-computation}
	\end{figure}
	By choosing a good point in the codomain, we get one point for each disk in the preimage, and the map is a local homeomorphism around these
	points which is orientation preserving. We could likewise compose the maps to \(S^n\) from the \hyperref[CW-complex-wedge-sum]{wedge} with a reflection to
	construct a map of \hyperref[def:degree]{degree} \(-m\).
\end{eg}
\begin{remark}
	We see that from the above construction, we can produce a map \(S^n \to S^n\) in any \hyperref[def:degree]{degree}.
\end{remark}