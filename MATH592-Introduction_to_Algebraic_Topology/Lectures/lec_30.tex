\lecture{30}{23 Mar. 10:00}{Degree}

\begin{theorem}[Hairy ball theorem]\label{thm:hairy-ball-theorem}
	See the homework. This essentially says that there is no nonvanishing continuous tangent vector field on even-dimensional spheres.
\end{theorem}

\begin{theorem}[Groups acting on \(S^{2n}\)]\label{thm:actions-on-spheres}
	If \(G\) acts on \(S^{2n}\) \hyperref[def:free-group]{freely}, then
	\[
		G = \quotient{\mathbb{Z}}{2\mathbb{Z}} \text{ or } 1.
	\]
\end{theorem}
\begin{proof}
	There exists a homomorphism given by
	\[
		\begin{split}
			G & \to \{\pm 1\}        \\
			g & \mapsto \deg(\tau_g)
		\end{split}
	\]
	Where \(\tau_g\) is the action of \(g \in G\) on \(S^{2n}\) as a map \(S^{2n} \to S^{2n}\). We know this map is well-defined since \(\tau_g\) is invertible
	(simply take \(\tau_{g^{-1}}\)) for each \(g \in G\). Our note on composites shows this is a homomorphism.

	We want to show that the kernel is trivial, since then by the first isomorphism theorem \(G \cong \im\), and the image is either trivial or
	\(\quotient{\mathbb{Z}}{2\mathbb{Z}}\). Suppose that \(g\) is a nontrivial element of \(g\), then since \(G\) acts \hyperref[def:free-group]{freely} we know that
	\(\tau_g\) has no fixed points. With this in mind we have
	\[
		\deg \tau_g = (-1)^{2n + 1} = - 1.
	\]
	Thus, \(g \not\in \ker\), hence the kernel is trivial as desired.
\end{proof}

\begin{corollary}
	\(S^{2n}\) is only the trivial cover \(S^{2n} \to S^{2n}\) or \hyperref[def:degree]{degree} \(2\) cover (for example, \(S^{2n} \to \mathbb{R}P^{2n}\)).
\end{corollary}
\begin{proof}
	This follows since any covering space action acts \hyperref[def:free-group]{freely}.
\end{proof}

\hr

Let's grab some intuition. \emph{What really is local homology}?

By \hyperref[thm:excision]{excision}, there is an isomorphism \(H_n(S^n, S^n \setminus \{x_i\}) \cong H_n(U, U \setminus \{x_i\})\) for any open
neighborhood \(U\) of \(x_i\).

The long \hyperref[def:exact-sequence]{exact sequence} of a \hyperref[def:good-pair]{pair} also gives us
\par
\adjustbox{scale=0.9,center}{%
	\begin{tikzcd}[column sep=small]
		\ldots & {H_k(S^n\setminus\{x_i\})} & {H_k(S^n)} & {H_k(S^n, S^n\setminus\{x_i\})} & {H_{k-1}(S^n\setminus\{x_i\})} & \ldots
		\arrow[from=1-1, to=1-2]
		\arrow[from=1-2, to=1-3]
		\arrow[from=1-3, to=1-4]
		\arrow[from=1-4, to=1-5]
		\arrow[from=1-5, to=1-6]
	\end{tikzcd}
}

\par Since \(S^n \setminus \{x_i\}\) is homeomorphic to an open \(n\)-ball, we see that \(H_k(S^n \setminus \{x_i\}) = H_{k - 1}(S^n \setminus \{x_i\}) = 0\).
With this in mind, \(j_\ast\) is an isomorphism.

We want to think about what \(j_\ast\) does when \(k = n\), i.e., when this is an isomorphism \(\mathbb{\MakeUppercase{z}}\cong H_n(S^n) \to H_n(S^n, S^n \setminus \{x_i\}) \cong \mathbb{Z}\).

We see that \(\Delta_1 - \Delta_2\) generate \(H_n(S^n)\), where \(\Delta_1, \Delta_2\) are the top and bottom hemisphere indicated below.
\begin{figure}[H]
	\centering
	\incfig{les-on-relative-spheres}
	\label{fig:les-on-relative-spheres}
\end{figure}
We then understand that \(j_\ast(\Delta_1 - \Delta_2) = \Delta_1 - \Delta_2 = \Delta_1\) since \(\Delta_2 = 0\) in \(C_n(S^n)/C_n(S^n \setminus \{x_i\})\).

The upshot is that \(H_n(S^n, S^n \setminus \{x\})\) is generated by an \hyperref[def:standard-simplex]{\(n\)-simplex} with \(x\) in its interior.

Suppose \(M\) is an \(n\)-manifold. Then \(H_n(M, M \setminus \{x\}) \cong H_n(U, U \setminus \{x\})\), where \(U\) is a small ball around \(x\).
Because \(U\) is a ball homeomrphic to \(\mathbb{\MakeUppercase{r}} ^n\), we see that
\[
	H_n(M, M \setminus \{x\}) \cong H_n(U, U \setminus \{x\}) \cong H_n(\mathbb{\MakeUppercase{r}} ^n, \mathbb{\MakeUppercase{r}} ^n \setminus \{x\}).
\]
By the long \hyperref[def:exact-sequence]{exact sequence} of a \hyperref[def:good-pair]{pair}
\[
	\begin{tikzcd}[column sep=small]
		{0=H_n(\mathbb{R}^n)} & {H_n(\mathbb{R}^n, \mathbb{R}^n\setminus\{x\})} & {H_{n-1}(\mathbb{R}^n\setminus\{x\})} & {H_{n-1}(\mathbb{R}^n)=0}
		\arrow[from=1-1, to=1-2]
		\arrow[from=1-2, to=1-3]
		\arrow[from=1-3, to=1-4]
	\end{tikzcd}
\]
And since \(\mathbb{\MakeUppercase{r}}^n \setminus \{x\}\) is \hyperref[def:homotopy-equivalence]{homotopy equivalent} to an \(n - 1\) sphere, this means that
\(H_n(\mathbb{\MakeUppercase{r}} ^n, \mathbb{\MakeUppercase{r}} ^n \setminus \{x\}) \cong \mathbb{\MakeUppercase{z}} \). By homework, this
connecting homomorphism is given by taking the \hyperref[def:boundary]{boundary} of a \hyperref[def:relative-cycle]{relative cycle} as below.
\begin{figure}[H]
	\centering
	\incfig{connecting-homomorphism-relative-homology-rn}
	\label{fig:connecting-homomorphism-relative-homology-rn}
\end{figure}

We intuitively want to use this idea to compute \hyperref[def:degree]{degree} using this idea. We use naturality of the long \hyperref[def:exact-sequence]{exact sequence},
namely the fact that where \(f \colon (U_i, U_i \setminus \{x_i\}) \to (V, y)\) is a map of \hyperref[def:good-pair]{pairs}, then the following diagram commutes.
\[
	\begin{tikzcd}
		\ldots & {H_n(U_i, U_i\setminus\{x_i\})} & {H_{n-1}(U_i, U_i\setminus\{x_i\})} & \ldots \\
		\ldots & {H_n(V, V\setminus\{y\})} & {H_{n-1}(V, V\setminus\{y\})} & \ldots
		\arrow[from=1-1, to=1-2]
		\arrow[from=1-2, to=1-3]
		\arrow[from=1-3, to=1-4]
		\arrow["{f_\ast}", from=1-3, to=2-3]
		\arrow[from=2-3, to=2-4]
		\arrow[from=2-2, to=2-3]
		\arrow["{f_\ast}", from=1-2, to=2-2]
		\arrow[from=2-1, to=2-2]
	\end{tikzcd}
\]
By naturality of the long \hyperref[def:exact-sequence]{exact sequence} and the isomorphism discussed above, we can compute the \hyperref[def:local-degree]{local degree}
of a map \(S^n \to S^n\) at a point \(x\) by computing the \hyperref[def:degree]{degree} of the map
\[
	\begin{tikzcd}
		{H_{n-1}(U\setminus\{x\})} & {H_{n-1}(V-\{y\})}
		\arrow[from=1-1, to=1-2]
	\end{tikzcd}
\]

In fact the \hyperref[def:local-degree]{local degree} will be the \hyperref[def:degree]{degree} restricted to a small \(S^{n - 1}\) n the neighborhood \(U\).
\begin{figure}[H]
	\centering
	\incfig{computing-local-homology-idea}
	\label{fig:computing-local-homology-idea}
\end{figure}

\begin{definition}[Local degree]\label{def:local-degree}
	Let \(f \colon S^n \to S^n\) (\(n > 0\)). Suppose there exists \(y \in S^n\) such that \(f^{-1}(y)\) is finite, say, \(\{x_1, \ldots, x_m\}\).
	Then let \(U_1, \ldots, U_m\) be disjoint neighborhoods of \(x_1, \ldots, x_m\) that are mapped by \(f\) to some neighborhood \(V\) of \(y\).
	\begin{figure}[H]
		\centering
		\incfig{def:local-degree}
		\label{fig:def:local-degree}
	\end{figure}

	The \emph{local degree} of \(f\) at \(x_i\), denote as \(\at{\deg f}{x_i}{}\), is the \hyperref[def:degree]{degree} of the map
	\[
		f_\ast \colon \mathbb{Z} \cong H_n(U_i, U_i - \{x_i\}) \to H_n(V, V - \{y\}) \cong \mathbb{Z}.
	\]
\end{definition}

\begin{theorem}\label{thm:calculation-with-local-degree}
	Let \(f \colon  S^n \to S^n\) with \(f^{-1}(y) = \{x_1, \ldots, x_m\}\) as in \autoref{def:local-degree}, then we have
	\[
		\deg f = \sum_{i = 1}^m \at{\deg f}{x_i}{}.
	\]
\end{theorem}
\begin{remark}
	Thus, we can compute the \hyperref[def:degree]{degree} of \(f\) by computing these \hyperref[def:local-degree]{local degrees}.
\end{remark}

Let's work with some examples for our edification.
\begin{eg}
	Consider \(S^n\) and choose \(m\) disks in \(S^n\). Namely, we first collapse the complement of the \(m\) disks to a point, and then we identify each of
	the \hyperref[sssec:Wedge-sum]{wedged \(n\)-spheres} with the \(n\)-sphere itself.
	\begin{figure}[H]
		\centering
		\incfig{eg:degree-m-map-disks}
		\label{fig:eg:degree-m-map-disks}
	\end{figure}
	The result will be a map of \hyperref[def:degree]{degree} \(m\). We can see this by computing \hyperref[def:local-degree]{local degree}.
	\begin{figure}[H]
		\centering
		\incfig{eg:degree-m-map-disks-computation}
		\label{fig:eg:degree-m-map-disks-computation}
	\end{figure}
	By choosing a good point in the codomain, we get one point for each disk in the preimage, and the map is a local homeomorphism around these
	points which is orientation preserving. We could likewise compose the maps to \(S^n\) from the \hyperref[sssec:Wedge-sum]{wedge} with a reflection to
	construct a map of \hyperref[def:degree]{degree} \(-m\).
\end{eg}