\lecture{30}{23 Mar. 10:00}{Degree}

\begin{theorem}[Hairy ball theorem]\label{thm:hairy-ball-theorem}
	See the homework. This essentially says that there is no nonvanishing continuous tangent vector field on even-dimensional spheres.
\end{theorem}

\begin{theorem}[Groups acting on \(S^{2n}\)]\label{thm:actions-on-spheres}
	If \(G\) acts on \(S^{2n}\) \hyperref[def:free-group]{freely}, then
	\[
		G = \quotient{\mathbb{Z}}{2\mathbb{Z}} \text{ or } 1.
	\]
\end{theorem}
\begin{proof}
	There exists a homomorphism given by
	\[
		\begin{split}
			G & \to \{\pm 1\}        \\
			g & \mapsto \deg(\tau_g)
		\end{split}
	\]
	Where \(\tau_g\) is the action of \(g \in G\) on \(S^{2n}\) as a map \(S^{2n} \to S^{2n}\). We know this map is well-defined since \(\tau_g\) is invertible
	(simply take \(\tau_{g^{-1}}\)) for each \(g \in G\). Our note on composites shows this is a homomorphism.

	We want to show that the kernel is trivial, since then by the first isomorphism theorem \(G \cong \im\), and the image is either trivial or
	\(\quotient{\mathbb{Z}}{2\mathbb{Z}}\). Suppose that \(g\) is a nontrivial element of \(g\), then since \(G\) acts \hyperref[def:free-group]{freely} we know that
	\(\tau_g\) has no fixed points. With this in mind we have
	\[
		\deg \tau_g = (-1)^{2n + 1} = - 1.
	\]
	Thus, \(g \not\in \ker\), hence the kernel is trivial as desired.
\end{proof}

\begin{corollary}
	\(S^{2n}\) is only the trivial cover \(S^{2n} \to S^{2n}\) or \hyperref[def:degree]{degree} \(2\) cover (for example, \(S^{2n} \to \mathbb{R}P^{2n}\)).
\end{corollary}
\begin{proof}
	This follows since any \hyperref[def:covering-space]{covering space} action acts \hyperref[def:free-group]{freely}.
\end{proof}

\begin{definition}[Local degree]\label{def:local-degree}
	Let \(f \colon S^n \to S^n\) (\(n > 0\)). Suppose there exists \(y \in S^n\) such that \(f^{-1}(y)\) is finite, say, \(\{x_1, \ldots, x_m\}\).
	Then let \(U_1, \ldots, U_m\) be disjoint neighborhoods of \(x_1, \ldots, x_m\) that are mapped by \(f\) to some neighborhood \(V\) of \(y\).
	\begin{figure}[H]
		\centering
		\incfig{def:local-degree}
		\label{fig:def:local-degree}
	\end{figure}

	The \emph{local degree} of \(f\) at \(x_i\), denote as \(\at{\deg f}{x_i}{}\), is the \hyperref[def:degree]{degree} of the map
	\[
		f_\ast \colon \mathbb{Z} \cong H_n(U_i, U_i - \{x_i\}) \to H_n(V, V - \{y\}) \cong \mathbb{Z}.
	\]
\end{definition}

\begin{theorem}\label{thm:calculation-with-local-degree}
	Let \(f \colon  S^n \to S^n\) with \(f^{-1}(y) = \{x_1, \ldots, x_m\}\) as in \autoref{def:local-degree}, then we have
	\[
		\deg f = \sum_{i = 1}^m \at{\deg f}{x_i}{}.
	\]
\end{theorem}
\begin{remark}
	Thus, we can compute the \hyperref[def:degree]{degree} of \(f\) by computing these \hyperref[def:local-degree]{local degrees}.
\end{remark}

Let's work with some examples for our edification.
\begin{eg}
	Consider \(S^n\) and choose \(m\) disks in \(S^n\). Namely, we first collapse the complement of the \(m\) disks to a point, and then we identify each of
	the \hyperref[sssec:Wedge-sum]{wedged \(n\)-spheres} with the \(n\)-sphere itself.
	\begin{figure}[H]
		\centering
		\incfig{eg:degree-m-map-disks}
		\label{fig:eg:degree-m-map-disks}
	\end{figure}
	The result will be a map of \hyperref[def:degree]{degree} \(m\). We can see this by computing \hyperref[def:local-degree]{local degree}.
	\begin{figure}[H]
		\centering
		\incfig{eg:degree-m-map-disks-computation}
		\label{fig:eg:degree-m-map-disks-computation}
	\end{figure}
	By choosing a good point in the codomain, we get one point for each disk in the preimage, and the map is a local homeomorphism around these
	points which is orientation preserving. We could likewise compose the maps to \(S^n\) from the \hyperref[sssec:Wedge-sum]{wedge} with a reflection to
	construct a map of \hyperref[def:degree]{degree} \(-m\).
\end{eg}