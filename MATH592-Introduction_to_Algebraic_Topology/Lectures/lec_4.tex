\lecture{4}{12 Jan. 10:00}{Cell Complex (CW Complex)}
\begin{prev}
	We saw that
	\begin{itemize}
		\item homotopy equivalence
		\item homotopy invariants
		      \begin{itemize}
			      \item path-connectedness
		      \end{itemize}
		\item not invariant
		      \begin{itemize}
			      \item dimension
			      \item orientability
			      \item compactness
		      \end{itemize}
	\end{itemize}
\end{prev}

\subsection{CW Complexes}
\begin{eg}
	Let's start with a few examples.
	\begin{enumerate}
		\item Constructing spheres:
		      \begin{itemize}
			      \item \(S^1\) (up to homeomorphism)
			            \begin{figure}[H]
				            \centering
				            \incfig{eg:constructing-sphere-1}
				            \label{fig:eg:constructing-sphere-1}
			            \end{figure}
			      \item \(S^2\)
			            \begin{itemize}
				            \item glue boundary of 2-disk to a point
				            \item glue 2 disks onto a circle
			            \end{itemize}
			            \begin{figure}[H]
				            \centering
				            \incfig{eg:constructing-sphere-2}
				            \caption{\textbf{Left}: Glue a \(2\)-disk to a point along its boundary. \textbf{Right}: Glue \(2\) disks to \(S^1\).}
				            \label{fig:eg:constructing-sphere-2}
			            \end{figure}
			            The gluing instruction to construct \(S^2\) in the right-hand side can be demonstrated as follows.
			            \begin{figure}[H]
				            \centering
				            \incfig{eg:constructing-sphere-2-gluing-instr}
				            \label{fig:eg:constructing-sphere-2-gluing-instr}
			            \end{figure}
			      \item \(T = S^1 \times S^1\)
			            \begin{figure}[H]
				            \centering
				            \incfig{eg:constructing-torus}
				            \label{fig:eg:constructing-torus}
			            \end{figure}
			            view as gluing instructions
			            \[
				            \text{vertex }+ 2 \text{ edges }+2\text{-disks}.
			            \]
			            Specifically, we have
			            \begin{figure}[H]
				            \centering
				            \incfig{eg:constructing-torus-gluing-instr}
				            \label{fig:eg:constructing-torus-gluing-instr}
			            \end{figure}
		      \end{itemize}
	\end{enumerate}
\end{eg}

\hr

Formally, we have the following definition.
\begin{notation}
	Let \(D^n\) denotes a closed n-disk (or n-ball)
	\[
		D^n\simeq \left\{x\in\mathbb{\MakeUppercase{R}} ^n\colon \left\lVert x\right\rVert \leq 1\right\}.
	\]
	And let \(S^n\) denotes an \(n\)-sphere
	\[
		S^n\simeq \left\{x\in \mathbb{\MakeUppercase{R}}^{n+1}\colon \left\lVert x\right\rVert = 1\right\}.
	\]
	Lastly, we call a point as a \emph{0-cell}, and the interior of \(D^n\) \(\mathrm{int}(D^n)\) for \(n\geq 1\) as a \emph{\(n\)-cell}.
\end{notation}
\begin{definition}[CW Complex]
	A \emph{CW Complex} is a topological space constructed inductively as
	\begin{enumerate}
		\item \(X^0\) (the \underline{0-skeleton}) is a set of discrete points.
		\item We inductively construct the \underline{n-skeleton} \(X^n\) from \(X^{n-1}\) by attaching \(n\)-cells \(e^n_{\alpha}\), where
		      \(\alpha\) is the index.
		      \par The gluing instructions glued by an \underline{attaching map} is that \(\forall \alpha\), \(\exists \) continuous map \(\varphi_{\alpha}\)
		      \[
			      \varphi_{\alpha}\colon \partial D^n_{\alpha}\to X^{n-1},
		      \]
		      then
		      \[
			      X^n = \quotient{\left(X^{n-1}\coprod\limits_{\alpha}D^n_\alpha\right)}{x\sim \varphi_{\alpha}(x)}
		      \]
		      with identification \(x\sim \varphi_{\alpha}(x)\) for all \(x\in \partial D^n_{\alpha}\) with quotient topology.
		\item
		      \[
			      X = \bigcup\limits_{n=0} X^n,
		      \]
		      and let \(\overline{w} \) denotes \underline{weak topology}. Then
		      \[
			      u\subseteq X \text{ is open }\iff \forall n\ u\cap X^n \text{ is open }.
		      \]
		      If all cells have dimension less than \(N\) and a \(\exists N\)-cell, then \(X = X^N\) and we call it \(N\)-dim CW complex.
	\end{enumerate}
\end{definition}

\begin{remark}
	We write \(X^{(n)}\) for \(n\)-skeleton if we need to distinguish from the Cartesian product.
\end{remark}

\begin{eg}
	Let's look at some examples.
	\begin{enumerate}
		\item \(0\)-dim CW complex is a discrete space.
		\item \(1\)-dim CW complex is a graph.
		\item A CW complex \(X\) is \underline{finite} if it has finitely many cells.
	\end{enumerate}
\end{eg}

\begin{definition}[CW subcomplex]
	A \emph{CW subcomplex} \(A\subseteq X\) is a closed subset equal to a union of cells
	\[
		e^n_{\alpha} = \mathrm{int}\left(D^n_{\alpha}\right).
	\]
\end{definition}
\begin{remark}
	This inherits a CW complex structure.\todo{Check the images of attaching maps.}
\end{remark}

\begin{exercise}
	Given the following gluing instruction:
	\begin{figure}[H]
		\centering
		\incfig{ex:CW-complex-gluing}
		\label{fig:ex:CW-complex-gluing}
	\end{figure}
	identify Torus, Klein bottle, Cylinder, Möbius band, 2-sphere, \(\mathbb{\MakeUppercase{R}} P\).

	\begin{answer}
		We see that
		\begin{table}[H]
			\centering
			\begin{tabular}{lll}
				1. Torus        & 2. Cylinder    & 3. 2-sphere                                 \\\\
				4. Klein bottle & 5. Möbius band & 6.         \(\mathbb{\MakeUppercase{R}} P\) \\
			\end{tabular}
		\end{table}
	\end{answer}
	\begin{notation}
		We call the real projection space as \(\mathbb{\MakeUppercase{R}} P\), and we also have so-called
		complex projection space, denote as \(\mathbb{\MakeUppercase{C}} P\).
	\end{notation}
\end{exercise}
