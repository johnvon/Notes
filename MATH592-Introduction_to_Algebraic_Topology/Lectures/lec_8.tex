\lecture{8}{24 Jan. 10:00}{The Fundamental Group \(\pi _1\)}
\begin{eg}
	In \hyperref[def:category]{category} \(\underline{\mathrm{Ab}}\) \hyperref[def:free-Abelian-group]{free Abelian group} on a set \(S\) is
	\[
		\bigoplus_S \mathbb{\MakeUppercase{z}}.
	\]
	In \hyperref[def:category]{category} of fields, no such thing as \textbf{\hyperref[def:free-group]{free} field on \(\bm{S} \)}.
\end{eg}

\subsection{Constructing the Free Groups \(F_S\)}
\begin{proposition}
	The \hyperref[def:free-group]{free group} defined by the universal property exists.
\end{proposition}
\begin{proof}
	We'll just give a construction below. First, we see the definition.
	\begin{definition}[Word]\label{def:word}
		Fix a set \(S\), and we define a \emph{word} as a finite sequence (possibly \(\varnothing \))
		in the formal symbols
		\[
			\left\{s, s ^{-1} \mid s\in S\right\}.
		\]
	\end{definition}

	Then we see that elements in \(F_S\) are equivalence classes of \hyperref[def:word]{words} with the equivalence relation being
	\begin{itemize}
		\item deleted \(s s ^{-1} \) or \(s ^{-1} s\). i.e.,
		      \[
			      \begin{split}
				      vs ^{-1} s w&\sim vw\\
				      v s s ^{-1}  w &\sim vw
			      \end{split}
		      \]
		      for every \hyperref[def:word]{word} \(v, w, s\in S\),
	\end{itemize}
	with the group operation being concatenation.
	\begin{eg}
		Given \hyperref[def:word]{words} \(ab^{-1} , bba\), their product is
		\[
			ab^{-1} \cdot bba = ab^{-1}  bb a = aba.
		\]
	\end{eg}
	\begin{exercise}
		There are something left to check.
		\begin{enumerate}[(1)]
			\item This product is well-defined on equivalence classes.
			\item Every equivalence class of \hyperref[def:word]{words} has a unique \emph{reduced form}, namely the representation.
			\item Check that \(F_S\) satisfies the universal property with respect to the map
			      \[
				      S\to F_S,\quad s\mapsto s.
			      \]
		\end{enumerate}
	\end{exercise}
\end{proof}

\chapter{The Fundamental Group}
\section{Path}
We start with the definition.
\begin{definition}[Path]\label{def:path}
	A \emph{path} in a space \(X\) is a continuous map
	\[
		\gamma\colon I\to X
	\]
	where \(I = [0, 1]\).
\end{definition}

\begin{definition}[Homotopy path]\label{def:homotopy-path}
	A \emph{homotopy of paths} \(\gamma_0\), \(\gamma_1\) is a \hyperref[def:homotopy]{homotopy} from \(\gamma_0\) to \(\gamma_1\) \(\mathrm{rel} \{0, 1\}\).
	\begin{figure}[H]
		\centering
		\incfig{def:homotopy-of-paths}
		\label{fig:def:homotopy-of-paths}
	\end{figure}
\end{definition}

\begin{eg}
	Fix \(x_1, x_0\in X\), then \underline{\(\exists\) \hyperref[def:homotopy]{homotopy} of \hyperref[def:path]{paths}} is an
	equivalence relation on \hyperref[def:path]{paths}
	from \(x_0\) to \(x_1\) (i.e., \(\gamma\) with \(\gamma(0)=x_0, \gamma(1)=x_1\)).
\end{eg}

\begin{definition}[Path composition]\label{def:path-composition}
	For \hyperref[def:path]{paths} \(\alpha , \beta \) in \(X\) with \(\alpha (1) = \beta (0)\), the \emph{composition}\footnote{Also named \emph{product}, \emph{concatenation}.}
	\(\alpha \cdot \beta \) is
	\[
		(\alpha \cdot \beta )(t) \coloneqq \begin{dcases}
			\alpha (2t),  & \text{ if } t\in \left[0, \frac{1}{2}\right]  \\
			\beta (2t-1), & \text{ if } t\in \left[\frac{1}{2}, 1\right].
		\end{dcases}
	\]
	\begin{figure}[H]
		\centering
		\incfig{def:path-composition}
		\label{fig:def:path-composition}
	\end{figure}
\end{definition}

\begin{remark}
	By the pasting lemma, this is continuous, hence \(\alpha \cdot \beta \) is actually a \hyperref[def:path]{path} from \(\alpha (0)\) to \(\beta (1)\).
\end{remark}

\begin{definition}[Reparameterization]\label{def:reparameterization}
	Let \(\gamma\colon I\to X\) be a \hyperref[def:path]{path}, then a \emph{reparameterization} of \(\gamma\) is a \hyperref[def:path]{path}
	\[
		\gamma ^\prime \colon I\overset{\varphi }{\longrightarrow} I\overset{\gamma}{\longrightarrow} X
	\]
	where \(\varphi \) is \underline{continuous} and
	\[
		\varphi (0) = 0,\quad \varphi (1) = 1.
	\]
\end{definition}

\begin{exercise}
	A \hyperref[def:path]{path} \(\gamma\) is \hyperref[def:homotopy-relative]{homotopic \(\mathrm{rel} \{0, 1\}\)} to all of its
	\hyperref[def:reparameterization]{reparameterizations}.
\end{exercise}
\begin{answer}
	\par We show that \(\gamma\) and \(\gamma\circ \phi \) are \hyperref[def:homotopic]{homotopic} \(\mathrm{rel} \{0, 1\}\) by showing that
	there exists a continuous \(F_t\) such that
	\[
		F_0 = \gamma,\quad F_1 = \gamma\circ \phi.
	\]
	Notice that since \(\phi \) is continuous, so we define
	\[
		F_t(x) = (1 - t) \gamma(x) + t\cdot \gamma\circ \phi (x).
	\]
	We see that
	\[
		F_0(x) = \gamma(x),\quad F_1(x) = \gamma\circ \phi (x),
	\]
	and also, we have
	\[
		F_t(x)\in X
	\]
	for all \(x, t\in I\).

	\par Now, we check that \(F_t\) really gives us a \hyperref[def:homotopy-relative]{homotopic \(\mathrm{rel} \{0, 1\}\)}. We have
	\[
		\begin{split}
			F_t(0) &= (1 - t)\gamma(0) + t\cdot \gamma\circ \phi (0) = (1 - t)\gamma(0) + t\cdot \gamma(\underbrace{\phi (0)}_{0}) = \gamma(0),\\
			F_t(1) &= (1 - t)\gamma(1) + t\cdot \gamma\circ \phi (1) = (1 - t)\gamma(1) + t\cdot \gamma(\underbrace{\phi (1)}_{1}) = \gamma(1),
		\end{split}
	\]
	which shows that \(0\) and \(1\) are independent of \(t\), hence \(\gamma\) and \(\gamma\circ \phi \) are \hyperref[def:homotopy-relative]{homotopic \(\mathrm{rel} \{0, 1\}\)}.
\end{answer}

\begin{exercise}
	Fix \(x_1, x_1\in X\). Then \hyperref[def:homotopy-path]{\underline{homotopy of paths}} (\hyperref[def:homotopy-relative]{relative \(\{0, 1\}\)}) is an
	equivalence relation on \hyperref[def:path]{paths} from \(x_0\) to \(x_1\).
\end{exercise}

\section{Fundamental Group and Groupoid}
In class, we only discuss \hyperref[def:fundamental-group]{fundamental groups}. But there is an also important concept called
\hyperref[def:fundamental-groupoid]{fundamental groupoid}, which is kind of similar but more general and power to use. I include
this just for completeness, feel free to skip the additional content. Now, let's start with \hyperref[def:fundamental-group]{fundamental group}.

\subsection{Fundamental Group}
\begin{definition}[Fundamental group]\label{def:fundamental-group}
	Let \(X\) denotes the space and let \(x_0\in X\) be the base point. The \emph{fundamental group of \(X\) based at \(x_0\)},
	denoted by \(\pi_1(X, x_0)\), is a group such that
	\begin{itemize}
		\item Elements: \hyperref[def:homotopy]{Homotopy} classes \(\mathrm{rel} \{0, 1\}\) of \hyperref[def:path]{paths} \([\gamma]\) where \(\gamma\) is a \textbf{loop}
		      with \(\gamma(0) = \gamma(1) = x_0\)\footnote{We say \(\gamma\) is \textbf{based} at \(x_0\).}
		      \begin{figure}[H]
			      \centering
			      \incfig{def:fundamental-group-elements}
			      \label{fig:def:fundamental-group-elements}
		      \end{figure}
		\item Operation: \hyperref[def:path-composition]{Composition of paths}.
		\item Identity: Constant loop \(\gamma\) based at \(x_0\) such that
		      \[
			      \gamma\colon I\to X,\quad t\mapsto x_0
		      \]
		\item Inverses: The inverse \([\gamma]^{-1}\) of \([\gamma]\) is represented by the loop \(\overline{\gamma}\) such that
		      \[
			      \overline{\gamma} (t) = \gamma(1-t).
		      \]
		      \begin{figure}[H]
			      \centering
			      \incfig{def:fundamental-group-inverses}
			      \label{fig:def:fundamental-group-inverses}
		      \end{figure}
	\end{itemize}
\end{definition}
\begin{proof}
	We actually need to prove that the defined \(\pi _1\) actually is a group, hence, we prove that
	\paragraph{Associativity.} \([\gamma_{1}\cdot (\gamma_2\cdot \gamma_3)] = [(\gamma_1\cdot \gamma_2)\cdot \gamma_3]\). We break this down into
	\[
		\gamma_1\cdot(\gamma_2\cdot \gamma_3)(t) = \begin{dcases}
			\gamma_{1}(2t ),                 & t\in\left[0, \frac{1}{2}\right]; \\
			(\gamma_2\cdot \gamma_3)(2t-1) , & t\in\left[\frac{1}{2}, 1\right]
		\end{dcases} = \begin{dcases}
			\gamma_{1}(2t ), & t\in\left[0, \frac{1}{2}\right];           \\
			\gamma_2(4t-2) , & t\in\left[\frac{1}{2}, \frac{3}{4}\right]; \\
			\gamma_3(4t-3) , & t\in\left[\frac{3}{4}, 1\right],
		\end{dcases}
	\]
	and
	\[
		(\gamma_1\cdot\gamma_2)\cdot \gamma_3(t) = \begin{dcases}
			(\gamma_{1}\cdot \gamma_2) (2t ), & t\in\left[0, \frac{1}{2}\right]; \\
			\gamma_3(2t-1) ,                  & t\in\left[\frac{1}{2}, 1\right]
		\end{dcases} = \begin{dcases}
			\gamma_{1}(4t),  & t\in\left[0, \frac{1}{4}\right];           \\
			\gamma_2(4t-1) , & t\in\left[\frac{1}{4}, \frac{1}{2}\right]; \\
			\gamma_3(2t-1) , & t\in\left[\frac{1}{2}, 1\right].
		\end{dcases}
	\]

	\par Then, we define \(\phi \colon I\to I\) such that
	\[
		\phi (t) = \begin{dcases}
			2t\in\left[0, \frac{1}{2}\right],                      & t\in\left[0, \frac{1}{4}\right];           \\
			t+\frac{1}{4}\in\left[\frac{1}{2}, \frac{3}{4}\right], & t\in\left[\frac{1}{4}, \frac{1}{2}\right]; \\
			\frac{t+1}{2}\in\left[\frac{3}{4}, 1\right],           & t\in\left[\frac{1}{2}, 1\right].
		\end{dcases}
	\]
	We easily see that
	\[
		\gamma_1\cdot(\gamma_2\cdot \gamma_3)(t) = (\gamma_1\cdot\gamma_2)\cdot \gamma_{3}\circ \phi (t)
	\]
	and \(\phi (t)\) is continuous and satisfied \(\phi (0) = 0\) and \(\phi (1) = 1\), which implies that the associativity holds.

	\paragraph{Identity.} We want to show that \([\gamma\cdot c] = [\gamma]\). Again, we consider
	\[
		(\gamma\cdot c) (t) = \begin{dcases}
			\gamma(2t),                    & t\in\left[0, \frac{1}{2}\right] ; \\
			c(2t-1) = c = x_0 = \gamma(0), & t\in\left[\frac{1}{2}, 1\right].
		\end{dcases}
	\]
	Now, consider \(\phi \colon I\to I\) such that
	\[
		\phi (t) = \begin{dcases}
			2t, & t\in\left[0, \frac{1}{2}\right] ; \\
			1,  & t\in\left[\frac{1}{2}, 1\right].
		\end{dcases}
	\]
	We easily see that
	\[
		(\gamma\cdot c)(t) = (\gamma\circ \phi) (t)
	\]
	and \(\phi (t)\) is continuous and satisfied \(\phi (0) = 0\) and \(\phi (1) = 1\).

	\paragraph{Inverses.} We want to show that \(\gamma\cdot \overline{\gamma} \simeq c\), where \(\overline{\gamma} (t) = \gamma(1-t)\).
	Firstly, we have
	\[
		(\gamma\cdot \overline{\gamma})(t) = \begin{dcases}
			\gamma(2t),               & t\in\left[0, \frac{1}{2}\right]; \\
			\overline{\gamma} (1-2t), & t\in\left[\frac{1}{2}, 1\right].
		\end{dcases}
	\]

	\par We consider \(F_t\) given by
	\[
		F_{t}(x) = \begin{dcases}
			\gamma(2xt),              & x\in\left[0, \frac{1}{2}\right]; \\
			\overline{\gamma}(1-2xt), & x\in\left[\frac{1}{2}, 1\right].
		\end{dcases}
	\]
	If \(t = 0\), we have
	\[
		F_{0}(x) = \begin{dcases}
			\gamma(0),            & x\in\left[0, \frac{1}{2}\right]; \\
			\overline{\gamma}(1), & x\in\left[\frac{1}{2}, 1\right]
		\end{dcases} = x_0
	\]
	for all \(x\in I\), namely \(F_0 = c\), while when \(t = 1\), we have
	\[
		F_{1}(x) = \begin{dcases}
			\gamma(2x),              & x\in\left[0, \frac{1}{2}\right]; \\
			\overline{\gamma}(1-2x), & x\in\left[\frac{1}{2}, 1\right]
		\end{dcases} = (\gamma\cdot \overline{\gamma})(x),
	\]
	and we see that \(F_t\) is continuous since at \(x = \frac{1}{2}\), we have
	\[
		\gamma(2x) = \gamma(1) = \overline{\gamma} (0) = \overline{\gamma} (1 - 2x),
	\]
	hence we see that \(F_{t}\) is the \hyperref[def:homotopy]{homotopy} between \(\gamma\cdot \overline{\gamma} \) and \(c\).
	\begin{figure}[H]
		\centering
		\incfig{def:fundamental-group}
		\caption{Illustration of \(F_t\). Intuitively, the \hyperref[def:path]{path} \(\gamma\cdot \overline{\gamma}\) is \(x_0 \overset{\gamma}{\to} x_0 \overset{\overline{\gamma} }{\to} x_0\).
			But now, \(F_t\) is \(x_0 \overset{\gamma}{\to} t \overset{\overline{\gamma} }{\to} x_0\). We can think of this \hyperref[def:homotopy]{homotopy} is \emph{pulling back}
			the turning point along the original \hyperref[def:path]{path}.}
		\label{fig:def:fundamental-group}
	\end{figure}
\end{proof}

\begin{theorem}\label{thm:lec8}
	If \(X\) is \hyperref[def:path]{path}-connected, then
	\[
		\forall x_0, x_1\in X\quad \pi_1(X, x_0)\cong \pi _1(X, x_1).
	\]
\end{theorem}
\begin{proof}
	\par To show that the \emph{change-of-basepoint map} is isomorphism, we show that it's one-to-one and onto.
	\begin{itemize}
		\item one-to-one. Consider that if \([h\cdot \gamma\cdot \overline{h} ] = [h\cdot \gamma ^\prime \cdot \overline{h} ]\), then since we know that \(h^{-1}  = \overline{h} \), hence
		      in the \hyperref[def:fundamental-group]{fundamental group} \(\pi _1(X, x_0)\), we see that
		      \[
			      \overline{h} \cdot h\cdot \gamma\cdot \overline{h} \cdot h = \overline{h} \cdot h\cdot \gamma ^\prime \cdot \overline{h} \cdot h. \implies \gamma = \gamma ^\prime
		      \]
		      as we desired.
		\item onto. We see that for every \(\alpha \in \pi_1(X, x_0)\), there exists a \(\gamma\in \pi_1(X, x_{0})\) such that
		      \[
			      \gamma = \overline{h} \cdot \alpha \cdot h\in \pi _1(X, x_1)
		      \]
		      since \(h\cdot \gamma\cdot \overline{h} =\alpha \).\footnote{Notice that this is indeed the case, one can verify this by the fact that \(h\colon x_{0}\to x_1\) and \(\overline{h} \colon x_1 \to x_0\).}
	\end{itemize}

	\par We then see that the \hyperref[def:fundamental-group]{fundamental group} of \(X\) does not depend on the choice of basepoint, only on the choice of the \hyperref[def:path]{path} component of the basepoint.
	If \(X\) is \hyperref[def:path]{path}-connected, it now makes sense to refer to \emph{the} \hyperref[def:fundamental-group]{fundamental group} of \(X\) and write \(\pi _1(X)\) for the abstract group (up to isomorphism).
\end{proof}
\begin{remark}
	We see that we can write \(\pi _1(X)\) up to isomorphism if \(X\) is \hyperref[def:path]{path}-connected from \autoref{thm:lec8}.
\end{remark}

\begin{exercise}
	Composition of \hyperref[def:path]{paths} is well-defined on \hyperref[def:homotopy]{homotopy} classes \hyperref[def:homotopy-relative]{\(\mathrm{rel} \{0, 1\}\)}.
\end{exercise}

\begin{exercise}
	If \(X\) is a \hyperref[def:contractible]{contractible} space, then \(X\) is \hyperref[def:path]{path}-connected and \(\pi _1(X)\)  is trivial.
\end{exercise}

The followings are the properties about \hyperref[def:homotopy-path]{homotopy path}. They are useful when we introduce \hyperref[def:fundamental-groupoid]{fundamental groupoid}.
\begin{lemma}\label{lma:lec8-1}
	Given \(x_0, x_1, x_2\in X\), \(\alpha , \alpha ^\prime\) are two \hyperref[def:path]{paths} from \(x_0\) to \(x_1\), and \(\beta , \beta ^\prime \) are two \hyperref[def:path]{paths} from \(x_1\) to \(x_2\). If
	\(\left< \alpha  \right> = \left< \alpha ^\prime  \right> \), \(\left< \beta  \right> = \left< \beta ^\prime  \right> \), then \(\left< \alpha \cdot \beta  \right> = \left< \alpha ^\prime \cdot \beta ^\prime  \right>\).
\end{lemma}
\begin{proof}
	Given \(\alpha \underset{F}{\simeq }\alpha ^\prime \ \mathrm{rel} \{0, 1\}\), \(\beta \underset{G}{\simeq }\beta ^\prime \ \mathrm{rel} \{0, 1\}\), then we want to prove
	\[
		\alpha \cdot \beta \simeq \alpha ^\prime \cdot \beta ^\prime\ \mathrm{rel} \{0, 1\}.
	\]
	This is done by using \hyperref[def:homotopy]{homotopy} \(H\colon I\times I\to X\) such that it combines \(F(2s, t)\) and \(G(2s-1, t)\).
	\[
		\begin{tikzcd}
			{x_0} & {x_1} & {x_2}
			\arrow["{\alpha^\prime}"', curve={height=6pt}, from=1-1, to=1-2]
			\arrow["\alpha", curve={height=-6pt}, from=1-1, to=1-2]
			\arrow["\beta", curve={height=-6pt}, from=1-2, to=1-3]
			\arrow["{\beta^\prime}"', curve={height=6pt}, from=1-2, to=1-3]
		\end{tikzcd}
	\]
	\begin{figure}[H]
		\centering
		\incfig{pf:lma:lec8-1}
		\label{fig:pf:lma:lec8-1}
	\end{figure}

\end{proof}

\begin{lemma}\label{lma:lec8-2}
	Let \(x_0, x_1, x_2, x_3\in X\), \(\alpha\) is a \hyperref[def:path]{path} from \(x_0\) to \(x_{1}\), \(\beta\) is a \hyperref[def:path]{path} from \(x_1\) to \(x_2\), \(\gamma\) is a \hyperref[def:path]{path} from \(x_2\) to \(x_3\). Then
	\[
		\left< (\alpha \cdot \beta ) \cdot \gamma \right>  = \left< \alpha \cdot (\beta \cdot \gamma ) \right>.
	\]
\end{lemma}
\begin{proof}
	We can write out the \hyperref[def:homotopy]{homotopy} by the following diagram.
	\begin{figure}[H]
		\centering
		\incfig{pf:lma:lec8-2}
		\label{fig:pf:lma:lec8-2}
	\end{figure}
\end{proof}

\begin{lemma}\label{lma:lec8-3}
	Let \(X\) be a topological space, and \(x_0\in X\). Then for every \hyperref[def:homotopy-path]{path homotopy} \(\left< \alpha  \right> \) from
	\(x_1\) to \(x_2\), we have
	\[
		\left< c_{x_1}\cdot \alpha  \right> = \left< \alpha  \right> = \left< \alpha \cdot c_{x_2} \right>.
	\]
\end{lemma}
\begin{proof}
	We only need to prove \(c_{x_1}\cdot \alpha \simeq \alpha\ \mathrm{rel} \{0,1\} \). The \hyperref[def:homotopy]{homotopy} can be written out explicitly by the following diagram.
	\begin{figure}[H]
		\centering
		\incfig{pf:lma:lec8-3}
		\label{fig:pf:lma:lec8-3}
	\end{figure}
\end{proof}

\begin{lemma}\label{lma:lec8-4}
	For every \hyperref[def:homotopy-path]{path homotopy} \(\left< \alpha  \right> \) from \(x_1\) to \(x_2\), then
	\[
		\left< \alpha \cdot \alpha ^{-1}  \right> = \left< c_{x_1} \right>, \qquad \left< \alpha ^{-1} \cdot \alpha  \right> = \left< c_{x_2} \right>.
	\]
\end{lemma}
\begin{proof}
	For the first case, we have the following diagram.
	\begin{figure}[H]
		\centering
		\incfig{pf:lma:lec8-4}
		\label{fig:pf:lma:lec8-4}
	\end{figure}
	The second case follows similarly.
\end{proof}

\subsection{Fundamental Groupoid}
This section is not covered in class, but it's a useful concept. The idea is that after giving \autoref{def:fundamental-group}, we see that we actually create a \hyperref[def:fundamental-group]{fundamental group}
at \textbf{every} point in \(X\), furthermore, when we use \autoref{thm:lec8} if \(X\) is \hyperref[def:path]{path}-connected, we actually \textbf{lose} some
information about this space. Here is how we can store all the information.

\begin{notation}[Constant loop]\label{not:constant-loop}
	We denote \(c_x\), where \(x\in X\) such that
	\[
		\begin{split}
			c_{x}\colon [0, 1]&\to X\\
			t&\mapsto x
		\end{split}
	\]
	as a \emph{constant loop}.
\end{notation}

\begin{definition}[Groupoid]\label{def:groupoid}
	A \hyperref[def:category]{category} \(\mathscr{C}\) is a \emph{groupoid} if any \hyperref[def:morphism]{morphisms} in \(\mathscr{C}\)
	is and isomorphism.
\end{definition}
\begin{remark}
	We'll soon see that for any topological space \(x\), \autoref{def:fundamental-group} defines a \hyperref[def:groupoid]{groupoid}, denoted by \(\Pi (X)\).
\end{remark}

\begin{definition}[Fundamental groupoid]\label{def:fundamental-groupoid}
	Let \(X\) denotes the space, then the \hyperref[def:category]{category} \(\Pi (X)\) is a \emph{fundamental groupoid of \(X\)} such that
	\begin{itemize}
		\item \(\Object (\Pi (X)) \coloneqq X \)
		\item \(\Homomorphism (\Pi (X))\colon \forall p, q\in \Object (\Pi (X)) = X\),
		      \[
			      \Homomorphism _{\Pi (X)}(p, q) \coloneqq \quotient{\left\{\text{\hyperref[def:path]{Paths} from \(p\) to \(q\)} \right\}}{\sim }.
		      \]
		\item Composition: For every \(p, q, r\in \Object (\Pi (X)) = X\),
		      \[
			      \begin{split}
				      \circ \colon \Homomorphism _{\Pi (X)}(p, q)\times \Homomorphism _{\Pi (X)}(q, r)&\to \Homomorphism _{\Pi (X)}(p, r)\\
				      (\left< \alpha  \right> , \left< \beta  \right> )&\mapsto \left< \beta  \right> \circ \left< \alpha  \right> \coloneqq \left< \alpha \cdot \beta  \right>.
			      \end{split}
		      \]
		\item Identity: For every \(p\in \Object (\Pi (X)) = X\), we define \(1_p\coloneqq \left< c_p \right> \in \Homomorphism _{\Pi (X)}(p, p)\) be the constant loop
		      based at \(p\) such that for every \(\left< \alpha \right> \in \Homomorphism _{\Pi (X)}(p, q)\),
		      \[
			      \left< \alpha  \right> \circ \identity_{p} = \identity_{q}\circ \left< \alpha  \right> = \left< \alpha  \right>.
		      \]
		\item Associativity: Given \(p, q, r, s\in \Object (\Pi (X)) = X\), with the \hyperref[def:path]{paths}
		      \[
			      \begin{tikzcd}
				      p & q & r & s
				      \arrow["{\left<\alpha\right>}", curve={height=-6pt}, from=1-1, to=1-2]
				      \arrow["{\left<\beta\right>}", curve={height=6pt}, from=1-2, to=1-3]
				      \arrow["{\left<\gamma\right>}", curve={height=-6pt}, from=1-3, to=1-4]
			      \end{tikzcd}
		      \]
		      Then
		      \[
			      \left< \gamma  \right> \circ \left(\left< \beta  \right> \circ \left< \alpha  \right> \right) = \left(\left< \gamma  \right> \circ \left< \beta  \right> \right)\circ \left< \alpha  \right>.
		      \]
	\end{itemize}
\end{definition}
\begin{proof}
	Note that in \autoref{def:fundamental-groupoid}, we need to show some of the definitions is indeed well-defined, and we also need to show that \(\Pi (X)\) is actually a \hyperref[def:groupoid]{groupoid}.
	\begin{itemize}
		\item Composition: Since if \(\alpha \simeq \alpha ^\prime , \beta \simeq \beta ^\prime \), we
		      have
		      \[
			      \alpha \cdot \beta \simeq \alpha ^\prime \cdot \beta ^\prime
		      \]
		      from \autoref{lma:lec8-1}.
		\item Identity: It follows that
		      \[
			      \left< \alpha  \right> \circ \identity_{p} = \left< c_p\cdot \alpha  \right> = \left< \alpha  \right>
		      \]
		      from \autoref{lma:lec8-3}. The left identity can be shown similarly.
		\item Associativity: It's trivial in the sense that all the \hyperref[def:homotopy]{homotopy} can be easily derived from
		      \autoref{lma:lec8-2}.
	\end{itemize}

	Additionally, from \autoref{lma:lec8-4}, we see that given \(\alpha \) is a \hyperref[def:path]{path} from \(p\) to \(q\), then
	\[
		\begin{dcases}
			\left< \alpha ^{-1} \cdot \alpha  \right> & = \left< c_q \right> \eqqcolon \identity_{q}  \\
			\left< \alpha \cdot \alpha^{-1}  \right>  & = \left< c_p \right> \eqqcolon \identity_{p}.
		\end{dcases}
	\]
	Furthermore, since \(\left< \alpha ^{-1} \cdot \alpha  \right> = \left< \alpha  \right> \circ \left< \alpha ^{-1}  \right> \) and
	\(\left< \alpha \cdot \alpha^{-1} \right> = \left< \alpha ^{-1}\right> \circ \left< \alpha\right> \), hence this means
	\(\Pi (X)\) is indeed a \hyperref[def:groupoid]{groupoid}.
\end{proof}

\begin{remark}
	Assume \(\mathscr{C}\) is a \hyperref[def:groupoid]{groupoid}, then for every \(x\in \Object (\mathscr{C})\), we can define
	\[
		\cdot \colon \Homomorphism _{\mathscr{C}}(x, x) \times \Homomorphism _{\mathscr{C}}(x, x)\to \Homomorphism _{\mathscr{C}}(x, x)
	\]
	such that
	\[
		(f, g)\mapsto f\cdot g \coloneqq g\circ f.
	\]
	We can prove that
	\[
		\left(\Homomorphism _{\mathscr{C}}(x, x), \cdot \right)
	\]
	defines a group \(\mathrm{Aut}_{\mathscr{C}}(x) \) called the \emph{isotropy group} of \(\mathscr{C}\) at \(x\).
\end{remark}
\begin{exercise}
	For every \(x, y\in \Object (\mathscr{C})\), if there exists \(f\in \Homomorphism _{\mathscr{C}}(x, y)\), then \(f\) induces
	\[
		f_\ast \colon \mathrm{Aut}_{\mathscr{C}}(x) \overset{\simeq}{\to } \mathrm{Aut}_{\mathscr{C}}(y),
	\]
	where \(f_\ast \) is a group homomorphism.
\end{exercise}

\begin{remark}
	For every \(p\in X=\Object (\Pi (X))\), we have
	\[
		\mathrm{Aut} _{\Pi (X)}(p) = \pi _1(X, p).
	\]

	Firstly, since they're the same in the sense of \textbf{set}:
	\[
		\mathrm{Aut}_{\Pi (X)} (p) = \Homomorphism _{\Pi (X)}(p, p) = \quotient{\left\{\text{Loops in \(X\) based at \(p\)} \right\}}{\sim} = \pi _1(X, p).
	\]
	Hence, we only need to verify their group composition agrees. But this is trivial, since for every two \(\left< \alpha  \right> , \left< \beta  \right> \in \mathrm{Aut}_{\Pi (X)}(p) \),
	\[
		\underbrace{\left< \alpha  \right> \cdot \left< \beta  \right> }_{\text{Composition from \(\mathrm{Aut}_{\Pi(X)}\)}} = \left< \beta  \right> \circ \left< \alpha  \right> =\underbrace{ \left< \alpha \cdot \beta\right>}_{\text{Composition from \(\pi_1\) } }.
	\]
	This implies that \autoref{thm:lec8} is just a particular example as a \hyperref[def:groupoid]{groupoid}.
\end{remark}