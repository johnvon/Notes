\lecture{8}{24 Jan. 10:00}{The Fundamental Group \(\pi _1\)}
\begin{eg}
	In category \(\underline{\mathrm{Ab}}\) free Abelian group on a set \(S\) is
	\[
		\bigoplus_S \mathbb{\MakeUppercase{z}}.
	\]
	In category of fields, no such thing as \textbf{free field on \(\bm{S} \) }.
\end{eg}

\subsubsection{Constructing the Free Groups \(F_S\)}
\begin{proposition}
	The free group defined by the universal property exists.
\end{proposition}
\begin{proof}
	We'll just give a construction below. First, we see the definition.
	\begin{definition}
		Fix a set \(S\), and we define a \underline{word} as a finite sequence (possibly \(\varnothing \))
		in the formal symbols
		\[
			\left\{s, s ^{-1} \mid s\in S\right\}.
		\]
	\end{definition}

	Then we see that elements in \(F_S\) are equivalence classes of words with the equivalence relation being
	\begin{itemize}
		\item delete \(s s ^{-1} \) or \(s ^{-1} s\). i.e.,
		      \[
			      \begin{split}
				      vs ^{-1} s w&\sim vw\\
				      v s s ^{-1}  w &\sim vw
			      \end{split}
		      \]
		      for every word \(v, w, s\in S\),
	\end{itemize}
	with the group operation being concatenation.
\end{proof}

\begin{eg}
	Given words \(ab^{-1} , bba\), their product is
	\[
		ab^{-1} \cdot bba = ab^{-1}  bb a = aba.
	\]
\end{eg}

\begin{exercise}
	There are something we can check.
	\begin{enumerate}
		\item This product is well-defined on equivalence classes.
		\item Every equivalence class of words has a unique \emph{reduced form}, namely the representation.
		\item Check that \(F_S\) satisfies the universal property with respect to the map
		      \[
			      S\to F_S,\quad s\mapsto s.
		      \]
	\end{enumerate}
\end{exercise}

\section{The Fundamental Group \(\pi_1\) }
We start with the definition.
\begin{definition}[Path]\label{def:path}
	A \emph{path} in a space \(X\) is a continuous map
	\[
		\gamma\colon I\to X
	\]
	where \(I = [0, 1]\).
\end{definition}

\begin{definition}[Homotopy path]\label{def:homotopy-path}
	A \emph{homotopy of paths} \(\gamma_0\), \(\gamma_1\) is a homotopy from \(\gamma_0\) to \(\gamma_1\) \(\mathrm{rel} \{0, 1\}\).
	\begin{figure}[H]
		\centering
		\incfig{def:homotopy-of-paths}
		\label{fig:def:homotopy-of-paths}
	\end{figure}
\end{definition}

\begin{eg}
	Fix \(x_1, x_0\in X\), then \underline{\(\exists\) homotopy of paths} is an equivalence relation on paths
	from \(x_0\) to \(x_1\) (i.e., \(\gamma\) with \(\gamma(0)=x_0, \gamma(1)=x_1\)).
\end{eg}

\begin{definition}[Path composition]\label{def:path-composition}
	For paths \(\alpha , \beta \) in \(X\) with \(\alpha (1) = \beta (0)\), the \emph{composition}\footnote{Also named \emph{product}, \emph{concatenation}.}
	\(\alpha \cdot \beta \) is
	\[
		(\alpha \cdot \beta )(t) \coloneqq \begin{dcases}
			\alpha (2t),  & \text{ if } t\in \left[0, \frac{1}{2}\right]  \\
			\beta (2t-1), & \text{ if } t\in \left[\frac{1}{2}, 1\right].
		\end{dcases}
	\]
	\begin{figure}[H]
		\centering
		\incfig{def:path-composition}
		\label{fig:def:path-composition}
	\end{figure}
\end{definition}

\begin{remark}
	By the pasting lemma, this is continuous, hence \(\alpha \cdot \beta \) is actually a path from \(\alpha (0)\) to \(\beta (1)\).
\end{remark}

\begin{definition}[Reparameterization]\label{def:reparameterization}
	Let \(\gamma\colon I\to X\) be a path, then a \emph{reparameterization} of \(\gamma\) is a path
	\[
		\gamma ^\prime \colon I\overset{\varphi }{\longrightarrow} I\overset{\gamma}{\longrightarrow} X
	\]
	where \(\varphi \) is \underline{continuous} and
	\[
		\varphi (0) = 0,\quad \varphi (1) = 1.
	\]
\end{definition}

\begin{exercise}
	A path \(\gamma\) is homotopic \(\mathrm{rel} \{0, 1\}\) to all of its reparameterizations.\todo{HW}
\end{exercise}

\begin{exercise}
	Fix \(x_1, x_1\in X\). Then \underline{Homotopy of paths} (relative \(\{0, 1\}\)) is an equivalence relation on paths from \(x_0\) to \(x_1\).
\end{exercise}

\begin{definition}[Fundamental Group]\label{def:fundamental-group}
	Let \(X\) denotes the space and let \(x_0\in X\) be the base point. The \emph{fundamental group of \(X\) based at \(x_0\)},
	denoted by \(\pi_1(X, x_0)\), is a group such that
	\begin{itemize}
		\item Elements: Homotopy classes \(\mathrm{rel} \{0, 1\}\) of paths \([\gamma]\) where \(\gamma\) is a \textbf{loop}
		      with \(\gamma(0) = \gamma(1) = x_0\)\footnote{We say \(\gamma\) is \textbf{based} at \(x_0\).}
		      \begin{figure}[H]
			      \centering
			      \incfig{def:fundamental-group-elements}
			      \label{fig:def:fundamental-group-elements}
		      \end{figure}
		\item Operation: \hyperref[def:path-composition]{Composition of paths}.
		\item Identity: Constant loop \(\gamma\) based at \(x_0\) such that
		      \[
			      \gamma\colon I\to X,\quad t\mapsto x_0
		      \]
		\item Inverses: The inverse \([\gamma]^{-1}\) of \([\gamma]\) is represented by the loop \(\overline{\gamma}\) such that
		      \[
			      \overline{\gamma} (t) = \gamma(1-t).
		      \]
		      \begin{figure}[H]
			      \centering
			      \incfig{def:fundamental-group-inverses}
			      \label{fig:def:fundamental-group-inverses}
		      \end{figure}
	\end{itemize}
\end{definition}
\begin{proof}
	We need to prove that the above define a group. \todo{HW.}
\end{proof}

\begin{theorem}
	If \(X\) is path-connected, then
	\[
		\forall x_0, x_1\in X\quad \pi_1(X, x_0)\cong \pi _1(X, x_1).
	\]
\end{theorem}
\begin{remark}
	We often write \(\pi _1(X)\) up to isomorphism.
\end{remark}
\begin{proof}
	\todo{HW.}
\end{proof}

\begin{exercise}
	Composition of paths is well-defined on homotopy classes \(\mathrm{rel} \{0, 1\}\).
\end{exercise}

\begin{exercise}
	If \(X\) is a contractible space, then \(X\) is path connected and \(\pi _1(X)\)  is trivial.
\end{exercise}