\lecture{2}{1 Sep. 12:30}{Graph Theory}
\begin{definition}[Path]\label{def:path}
	Given a list of vertices indexed in order as \(v_1, \ldots , v_n\) where \(v_i\in \mathcal{\MakeUppercase{v}} \),
	\begin{enumerate}[(a)]
		\item A \emph{path} \(P\) connecting \(v_i\) in an \hyperref[def:undirected-graph]{undirected graph} is a \hyperref[def:subgraph]{subgraph} \(P = (V, E)\), where \(V=\{v_i\}_{i=1}^n\), and
		      \(E = \{\{v_i, v_{i+1}\}\}_{i = 1}^{n-1}\).
		\item A \emph{path} \(P\) connecting \(v_i\) in a \hyperref[def:directed-graph]{directed graph} is a \hyperref[def:subgraph]{subgraph} \(P = (V, E)\), where \(V=\{v_i\}_{i=1}^n\), and
		      \(E = \{(v_i, v_{i+1})\}_{i = 1}^{n-1}\).
	\end{enumerate}
\end{definition}

With the definitions we introduced, we can now try to characterize a \hyperref[def:graph]{graph}.

\begin{definition*}
	We can then define connectivity of a \hyperref[def:graph]{graph} as follows.
	\begin{definition}[Connected]\label{def:connected}
		An \hyperref[def:undirected-graph]{undirected graph} is \emph{connected} if for every two nodes, there exists at least one \hyperref[def:path]{path} connect them together.
	\end{definition}

	\begin{definition}[Strongly connected]\label{def:strongly-connected}
		A \hyperref[def:directed-graph]{directed graph} is \emph{strongly connected} if for every two nodes, there exists at least one \hyperref[def:path]{path} connect them together.
	\end{definition}
\end{definition*}

\begin{notation}[Connected component]
	The above definitions can be generalized to a \hyperref[def:subgraph]{subgraph} as well. In this case, we say that this \hyperref[def:subgraph]{subgraph} is
	\hyperref[def:connected]{connected}, or more often we'll say this \hyperref[def:subgraph]{subgraph} is a \emph{\hyperref[def:connected]{connected} \hyperref[def:subgraph]{component}}.
	Same for \hyperref[def:strongly-connected]{strongly connected}.
\end{notation}

\begin{definition}[Giant component]\label{def:giant-component}
	A \hyperref[def:subgraph]{subgraph} is called a \emph{giant component} of a \hyperref[def:graph]{graph} if it is a \hyperref[def:connected]{connected} component and with a
	significant number of nodes of the original \hyperref[def:graph]{graph}.
\end{definition}

Finally, we introduce the first matrix associated with a \hyperref[def:graph]{graph}.
\begin{definition}[Adjacency matrix]\label{def:adjacency-matrix}
	For an \hyperref[def:undirected-graph]{undirected graph} \(\mathcal{\MakeUppercase{g}} = (\mathcal{\MakeUppercase{v}} , \mathcal{\MakeUppercase{e}} )\),
	the \emph{adjacency matrix of an \hyperref[def:undirected-graph]{undirected graph}} is a matrix \(A\) such that
	\[
		A_{ij} =\begin{dcases}
			1, & \text{ if }\{i, j\}\in\mathcal{E} \\
			0, & \text{ otherwise.}
		\end{dcases}
	\]

	For a \hyperref[def:directed-graph]{directed graph} \(\mathcal{\MakeUppercase{g}}^\prime = (\mathcal{\MakeUppercase{v}}^\prime , \mathcal{\MakeUppercase{e}}^\prime )\),
	the \emph{adjacency matrix of a \hyperref[def:directed-graph]{directed graph}} is a matrix \(A^\prime\) such that
	\[
		A^\prime _{ij} = \begin{dcases}
			1, & \text{ if }(i, j)\in\mathcal{E}^\prime \\
			0, & \text{ otherwise.}
		\end{dcases}
	\]
\end{definition}

\begin{remark}
	The \hyperref[def:adjacency-matrix]{adjacency matrix} of an \hyperref[def:undirected-graph]{undirected graph} must be \hyperref[def:symmetric-matrix]{symmetric} by definition.
\end{remark}
\begin{explanation}
	Since if \(\{i, j\}\in \mathcal{\MakeUppercase{e}} \), we know that \(\{j, i\}\in \mathcal{\MakeUppercase{e}} \) as well since \(\{j, i\} = \{i, j\}\), so
	\(A_{ij} = A_{ji} = 1\).

	On the other hand, if \(\{i, j\}\notin \mathcal{\MakeUppercase{e}} \), then \(\{j, i\}\notin \mathcal{\MakeUppercase{e}} \), hence in this case
	\(A_{ij} = A_{ji} = 0\) clearly.
\end{explanation}