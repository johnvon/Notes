\chapter{Page Rank}
\lecture{10}{4 Oct. 12:30}{PageRank}
\begin{prev}
	Voting procedure - hubs voted authorities, while authorities voted for hubs.
\end{prev}

But this is not always the case. Rankings that can be collaborated.
\section{Basic Page Rank Algorithm}
\begin{enumerate}
	\item INITIALIZE: For all nodes in \(\mathcal{V}\)(\(\left\vert \mathcal{V} \right\vert = V\)),
	      set page rank to be \(\frac{1}{V}\).
	\item SEND: Basic Page Rank updates: Each node divides of its current page rank equally across its outgoing
	      links and sends the value to the node at the end. If there are no outgoing links, then the node retains its page rank.
	\item RECEIVE: Each node updates its page rank to be the sum of the values it receives on its incoming links. If
	      a node has retained its page rank, then it adds it to this value.
	\item LOOP: Run step 2 for \(L\) steps.
\end{enumerate}
\begin{note}
	Page rank is never lost over gained in the network, it is just passed around. This means no normalization is required,
	the page ranks always sum to \(1\), with all terms non-negative.
\end{note}

\begin{problem}
Will it converge? To what if it does? Is the final state always interesting?
\end{problem}
\begin{answer}
	Fact: \begin{enumerate}
		\item It will converge in most cases.
		\item Eigenvector of some eigenvalue of some matrix associated with the graph.
		\item It may not always be interesting.
	\end{enumerate}
\end{answer}

\begin{eg}
	Consider a cycle graph, with different initial page rank. It will not converge! All values just cycling around.
\end{eg}

We want
\begin{enumerate}
	\item convergence for all starting vectors
	\item the final vector to be relevant to the graph
\end{enumerate}

\section{Scaled Page Rank}
\begin{enumerate}
	\item INITIALIZE: Denote \(r^{old}\) as the column vector of ranks.
	\item SEND: if \(d^{out}_i > 0\), then
	\item \[
		      \frac{r_{i}^{old}}{d_{i}^{out}}
	      \]send to neighbor at the end of the outgoing links. If \(d_{i}^{out} = 0\), then \(r_{i}^{old}\) sent to itself.
	\item RECEIVE: Collect all the page rank values. Therefore,
	      \[
		      r_{i}^{new} = \sum\limits_{j:j\to i}\frac{r_{j}^{old}}{d_{j}^{out}}+\begin{cases}
			      r_{i}^{old} & , \text{ if }d_{i}^{out} = 0 \\
			      0           & , \text{ if }d_{i}^{out} > 0 \\
		      \end{cases} .
	      \]
\end{enumerate}

We have \(r^{new} = N^{\top} r^{old}\), where
\[
	\begin{alignedat}{3}
		&\text{if }i\neq  j,\ (N^{\top})_{ij} = \begin{dcases}
			\frac{1}{d_{j}^{out}} & , \text{ if } A_{ji} = 1 \\
			0                     & , \text{ if } A_{ji} = 0 \\
		\end{dcases} &&= \begin{dcases}
			\frac{A_{ji}}{d_{j}^{out}} & , \text{ if }d_{j}^{out}>0 \\
			A_{ji}                     & , \text{ otherwise}        \\
		\end{dcases}\\
		&\text{if }i = j,\ (N^{\top})_{ij} = \begin{dcases}
			1 & , \text{ if } d_{i}^{out} = 0 \\
			0 & , \text{ if } d_{i}^{out} > 0 \\
		\end{dcases} &&= \begin{dcases}
			1      & , \text{ if } d_{i}^{out} = 0 \\
			A_{ii} & , \text{ if } d_{i}^{out} > 0 \\
		\end{dcases}
	\end{alignedat}
\]
Hence,
\[
	N = \begin{dcases}
		\frac{A_{ij}}{d_{i}^{out}} & , \text{ if } d_{i}^{out}>0                \\
		A_{ij}                     & , \text{ if }d_{i}^{out} = 0 \land i\neq j \\
		1                          & , \text{ if }d_{i}^{out} = 0 \land i= j    \\
	\end{dcases}.
\]

After \(L\) steps, we have
\[
	r(L) = (N^{\top})^L r(0).
\]

To study what happens as \(L\to \infty\), we suppose there is a convergence vector \(r\) such that
\[
	r = (N^{\top})r.
\]
We see that \(r\) must be an eigenvector for \(N^{\top}\) for eigenvalue \(1\). Notice that \(u\) should be
\emph{non-negative} and sums to \(1\) over entries. Now we have
\[
	(N^{\top})r = r \implies r^{\top}N = r^{\top},
\]
\(r\) is a left eigenvector of \(N\) with eigenvalue \(1\).
\begin{note}
	\(N\) has an eigenvalue \(1\)(we will show this)
\end{note}
To study \(N\),
\[
	N_{ij} = \begin{dcases}
		\frac{A_{ij}}{d_{i}^{out}}, & \text{ if }d_{i}^{out}>0               \\
		A_{ij},                     & \text{ if }d_{i}^{out}=0\land i\neq  j \\
		1,                          & \text{ if }d_{i}^{out}=0 \land i = j   \\
	\end{dcases}.
\]
Then
\[
	\sum\limits_{j\in\mathcal{V}}N_{ij} = \begin{dcases}
		\sum\limits_{j\in\mathcal{V}} \frac{A_{ij}}{d_{i}^{out}} = \frac{\sum\limits_{j\in\mathcal{V}} A_{ij} }{d_{i}^{out}} = \frac{d_{i}^{out}}{d_{i}^{out}}= 1 & , \text{ if }d_{i}^{out} > 0 \\\\
		\underbrace{1}_{j = i} + \underbrace{\sum\limits_{j\neq i} A_{ij}}_{ = d_{i}^{out} = 0} = 1 + 0 = 1                                                       & , \text{ if }d_{i}^{out} = 0 \\
	\end{dcases}
\]
\(N\) is a matrix such that the row sums are always \(1\),
\[
	N \vec{1} = \vec{1}.
\]
This is an eigenvalue! \(1\) is an eigenvalue of \(N\), with \(\vec{1}\) being a right eigenvector!

We can ask about the left eigenvector — it exists. Corresponding to \(\vec{1}\).
\begin{problem}
What do we know about the entries?
\end{problem}
\begin{answer}
	We have
	\begin{enumerate}
		\item Row sums on \(1\)
		\item Non-negative entries
		\item Row stochastic matrix(associated with Markov chains)
	\end{enumerate}
\end{answer}

\begin{remark}
	Each row vector is a probability mass function(PMF) on
	\[
		\{1, 2, \ldots , V\}.
	\]
\end{remark}


\section{Non-negative Matrices}
In the following lectures, we will study so-called Perron-Frobenius theorem: about non-negative matrices. Before this, we need to study
\emph{irreducible} matrix.

\begin{definition}[Irreducible]
	A non-navigate matrix \(A_{n\times n}\) is called \emph{irreducible} if
	\[
		\forall i, j\ \exists \underbrace{k(i, j)}_{\in\mathbb{\MakeUppercase{Z}}}>0
	\]
	such that
	\[
		(A^k)_{ij}>0.
	\]
\end{definition}

We have
\[
	\begin{split}
		(A^2)_{ij} &= \sum\limits_{k} A_{ik}A_{kj}\\
		(A^3)_{ij} &= (A A^2)_{ij} = \sum\limits_{k} A_{ik}(A^2)_{kj} = \sum\limits_{k}\sum\limits_{l} A_{ik} A_{kl}A_{lj}\\
		&\vdots\\
		(A^n)_{ij} &= \sum\limits_{k_1}\sum\limits_{k_2}\ldots \sum\limits_{k_n}A_{ik_1}A_{k_1 k_2}A_{k_2 k_3}\ldots A_{k_n j}
	\end{split}
\]

\(A\) is non-negative so \(A^{n+1}_{ij} > 0\implies \) at least one of the sums \(>0\).

"Adjacency" matrix associated with \(A\):
\[
	\widetilde{A}_{ij} = \begin{dcases}
		1, & \text{ if }A_{ij}>0 \\
		0, & \text{ if }A_{ij}=0 \\
	\end{dcases}.
\]

Find graph from \(\widetilde{A}, g\) if \((A^{n+1})_{ij}\) is positive \(\iff\) there is a path in \(g\) from
\(i\) to \(j\).