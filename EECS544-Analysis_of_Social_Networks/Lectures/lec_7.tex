\chapter{Diagonalization}
\lecture{7}{22 Sep. 12:30}{Spectral Theorem}
We now want to study a very useful theorem called \hyperref[thm:spectral-theorem]{spectral theorem}. Before this, we need to study \emph{eigenvalue}.
\section{Matrices with Real Eigenvalues}
Need to rely on some structural properties.

\subsection{Symmetric matrices}
\begin{definition}[Symmetric matrix]\label{def:symmetric-matrix}
	A \emph{symmetric matrix} is a square matrix such that
	\[
		A = A^{\top}.
	\]
\end{definition}

\begin{theorem}[Spectral Theorem]\label{thm:spectral-theorem}
	If \(A\) is a \hyperref[def:symmetric-matrix]{symmetric matrix}, then all eigenvalues are real. Furthermore, the left and right
	eigenvectors are the same, and they construct an orthonormal basis.
\end{theorem}
\begin{proof}
	We verify the second part of the theorem.  If we have \(A\) is \hyperref[def:symmetric-matrix]{symmetric}, then suppose \(\vec{x} \) is a right eigenvector,
	we have
	\[
		A \vec{x} = \lambda \vec{x}\Rightarrow
		(A \vec{x})^{\top} = \lambda \vec{x}^{\top}\Rightarrow
		\vec{x}^{\top} A^{\top} = \lambda\vec{x}^{\top}\Rightarrow
		\vec{x}^{\top} A = \lambda\vec{x}^{\top},
	\]
	where we see that \(\vec{x}\) is just left eigenvector, hence the left and right eigenvectors are the same.
\end{proof}