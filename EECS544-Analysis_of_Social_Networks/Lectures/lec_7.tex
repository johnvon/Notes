\lecture{7}{22 Sep. 12:30}{Spectral Theorem}
\section{Diagonalization}
We now want to study a very useful theorem called \textbf{Spectral Theorem}. Before this, we need to study \emph{eigenvalue}.
\subsection{Matrices with real eigenvalues}
Need to rely on some structural properties.

\subsubsection{Symmetric matrices}
\begin{definition}
	A symmetric matrix is a square matrix such that
	\[
		A = A^{T}.
	\]
\end{definition}

\begin{theorem}
	\label{Spectral Theorem}
	Spectral Theorem: If $A$ is a symmetric matrix, then \emph{all} eigenvalues are real. Furthermore, the left and right
	eigenvectors are the same, and they construct an orthonormal basis.
\end{theorem}

\begin{definition}
	Left eigenvector: If we have $A$ is symmetric, then we will have
	\[
		A \vec{x} = \lambda \vec{x}\Rightarrow
		(A \vec{x})^T = \lambda \vec{x}^T\Rightarrow
		\vec{x}^T A^T = \lambda\vec{x}^T\Rightarrow
		\vec{x}^T A = \lambda\vec{x}^T
	\]
	where $\vec{x}$ is so-called left eigenvector.
\end{definition}

