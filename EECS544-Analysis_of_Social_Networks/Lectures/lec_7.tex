\chapter{Diagonalization}
\lecture{7}{22 Sep. 12:30}{Spectral Theorem}
We now want to study a very useful theorem called \textbf{Spectral Theorem}. Before this, we need to study \emph{eigenvalue}.
\section{Matrices with Real Eigenvalues}
Need to rely on some structural properties.

\subsection{Symmetric matrices}
\begin{definition}[symmetric matrix]
	A \emph{symmetric matrix} is a square matrix such that
	\[
		A = A^{\top}.
	\]
\end{definition}

\begin{theorem}[Spectral Theorem]\label{thm:spectral-theorem}
	If $A$ is a symmetric matrix, then \emph{all} eigenvalues are real. Furthermore, the left and right
	eigenvectors are the same, and they construct an orthonormal basis.
\end{theorem}

\begin{definition}[Left eigenvector]
	If we have $A$ is symmetric, then we will have
	\[
		A \vec{x} = \lambda \vec{x}\Rightarrow
		(A \vec{x})^{\top} = \lambda \vec{x}^{\top}\Rightarrow
		\vec{x}^{\top} A^{\top} = \lambda\vec{x}^{\top}\Rightarrow
		\vec{x}^{\top} A = \lambda\vec{x}^{\top}
	\]
	where $\vec{x}$ is so-called \emph{left eigenvector}.
\end{definition}

