\lecture{23}{01 Dec. 12:30}{Bayesian Nash Equilibrium}
\begin{prev}
	The \(1^{st}\) price auction.
\end{prev}

Now we look at the expected utility of agent \(i\) over other agents' valuation and bids. From the monotonically increasing property
of \(\sigma(\cdot)\), the expected utility of \(i\) is
\[
	\begin{split}
		&\expectation{-i}{u_{i}(v_{i}, \sigma_{i}(v_{i}), v_{-i}, \sigma_{-i}(v_{-i}))}\\
		= &\expectation{-i}{(v_{i} - \sigma_{i}(v_{i})) \prod\limits_{j\in -i}\mathbbm{1}_{\{ \sigma_{i}(v_{i})\geq \sigma_{j}(v_{j}) \}}}\\
		= &(v_{i} - \sigma_{i}(v_{i}))\prod\limits_{j\in -i}\probability{}{\sigma_{j}(v_{j})\leq \sigma_{i}(v_{i})}\\
		= &(v_{i} - \sigma_{i}(v_{i}))\prod\limits_{j\in -i}\probability{}{v_{j}\leq v_{i}}\\
		= &(v_{i} - \sigma_{i}(v_{i}))F^{I-1}(v_{i})\\
		= &(v_{i} - \sigma_{i}(v_{i}))v_{i}^{I-1}.
	\end{split}
\]

\hr

Now the question is how should an agent bid? Namely, how to find \(\sigma_{i}(\cdot)\)?
\subsection{Reason of Response}
To reason about the bidding strategy, we first see an important principle.
\begin{theorem}
	\label{Revelation-Principle}
	Revelation Principle. Let \(\sigma\) be the equilibrium bidding function. Then any deviation can be decomposed into two steps:
	\begin{enumerate}
		\item \(v_{i}\to v\): Map value to same \emph{false} value.
		\item Apply \(\sigma\) to the \emph{false} value, so that the bid is \(\sigma(v)\).
	\end{enumerate}
\end{theorem}

\begin{intuition}
	This principle simply tells us that we don't need to perturb \(\sigma\) itself to analyze incentive of deviation, we can simply perturb \(v\) to achieve
	the same result.
\end{intuition}

\begin{remark}
	The \emph{Equilibrium bidding} is
	\begin{itemize}
		\item Step 1: \(v_{i}\to v_{i}\): identical mapping.
		\item Step 2: The same, so bid is \(\sigma(v_{i})\).
	\end{itemize}
\end{remark}

Assume a deviation of \(v_{i}\to v\) and the bid is \(\sigma(v)\). Then the utility becomes
\[
	\begin{split}
		u_{i} &= (v_{i} - \sigma(v))\probability{}{ \text{Winning with bid } \sigma(v)}\\
		&= (v_{i} - \sigma(v))\prod\limits_{j\in -i}\probability{}{\sigma(v_{j})\leq \sigma(v)}\\
		&= (v_{i} - \sigma(v))F^{I-1}(v)\\
		&= (v_{i} - \sigma(v))v^{I-1} \eqqcolon f_{\sigma}(v; v_{i}).
	\end{split}
\]

Then the best response calculation for \(i\) is as follows. Given a \(\sigma\), maximize utility by properly choosing \(v\) for each \(v_{i}\).
We simply differentiate the function and set the value be zero
\[
	\frac{\mathrm{d}}{\mathrm{d}v}f_{\sigma}(v;v_{i}) = v_{i}(I - 1)v^{I-2} - (I - 1)v^{I-2}\sigma(v) - \sigma^1(v)v^{I-1} \coloneqq 0
\]
for the \textbf{first order condition}. It's equivalent to solve the following O.D.E.
\[
	v_{i}(I - 1)v^{I-2} = \frac{\mathrm{d}}{\mathrm{d}v} (\sigma(v)v^{I-1}).
\]

We see that \(\sigma(v)\leq v_{i}\) is needed as you never bid higher than your valuation in a first-price auction.

\hr

For the equilibrium bidding \(v_{i}\to v_{i}\), the maximizer should be \(v_{i}\) itself. For the equilibrium bidding function \(\sigma^{\ast}\),
\[
	\at{\frac{\mathrm{d}}{\mathrm{d}v} f_{\sigma^{\ast}}(v;v_{i})}{v = v_{i}}{} = 0.
\]
Expand it out, we have
\[
	\frac{\mathrm{d}}{\mathrm{d}v} f_{\sigma^{\ast}}(v;v_{i}) = v_{i}(I - 1)v^{I-2} - \left((I - 1)v^{I-2}\sigma^{\ast}(v) + \frac{\mathrm{d}\sigma^{\ast}(v)}{\mathrm{d}v} v^{I-1}\right).
\]
By setting \(v = v_{i}\), we should get zero!

\[
	(I - 1)v_{i}^{I-1} = (I - 1)v_{i}^{I-2}\sigma^{\ast}(v_{i})+\at{\frac{\mathrm{d}\sigma^{\ast}(v)}{\mathrm{d}v}}{v = v_{i}}{}{v_{i}}^{I-1}.
\]
This equation has to hold for every \(v_{i}\) in \([0, 1]\).

\par Now, for a \(\overline{v}\), \(\sigma^{\ast}(\overline{v})\) should be
\[
	(I - 1)v_{i}^{I-1} = \frac{\mathrm{d}}{\mathrm{d}v_{i}} \left(v_{i}^{I-1}\sigma^{\ast}(v_{i})\right).
\]
Integrate on both sides, we have
\[
	\int_0^{\overline{v}}(I - 1)v_{i}^{I-1}\,\mathrm{d} v_{i} = \at{v_{i}^{I-1}\sigma^{\ast}(v_{i})}{0}{\overline{v}}.
\]
Solve for the left-hand side, we have
\[
	\frac{I - 1}{I}\overline{v}^I = \overline{v}^{I-1}\sigma^{\ast}(\overline{v}).
\]

Hence,
\[
	\sigma^{\ast}(\overline{v}) = \frac{I-1}{I}\overline{v} = (1 - \frac{1}{I})\overline{v},
\]
which means that agent bids \(\frac{I - 1}{I}\) times the value who estimate!

\begin{note}
	We see that
	\begin{itemize}
		\item Agents will shade their bid, hence it's not truthful.
		\item The good goes to the agent with the highest valuation, hence it's a social-welfare maximizer.
	\end{itemize}
\end{note}

\begin{remark}
	If an agent whose valuation is zero, then from the function, we see that he would bid zero, hence the utility is zero as well.
\end{remark}

\subsection{Revenue of the Auctioneer}
The expected revenue for the auctioneer is equal to
\[
	\frac{I - 1}{I}\expectation{}{\widetilde{v}_I} = \frac{I - 1}{I}\frac{I}{I+1} = \frac{I - 1}{I+1}.
\]

\begin{remark}
	Recall that the expected revenue for \(2^{nd}\) price auction for the auctioneer is \(\expectation{}{\widetilde{v}_{I-2}} = \frac{I-1}{I+1}\).
	We see that both auctions give the same expected revenue.
\end{remark}

\begin{note}
	If two auctions implement the same outcome in Bayesian Nash Equilibrium, and agent with value \(0\) bid \(0\), then the expected
	revenue will be the same. This is so-called \emph{Revenue Equivalence}. And this equivalence relation holds for more general kinds of
	auctions, namely under some conditions of auctions, the revenue for auctioneer is always the same. We'll come back to this
	relation soon.
\end{note}

\section{Mechanism Design of Auctions}
After seeing two kinds of auctions, we now try to give a general framework of auctions and try to design a mechanism of such games.
Assume that the auctioneer doesn't know values of agent. How the auction will be conducted?
\begin{itemize}
	\item For every \(b_1, \ldots , b_I\):
	      \begin{enumerate}
		      \item Who wins the auction
		      \item What that person pays
	      \end{enumerate}
\end{itemize}

\begin{eg}
	We first see an example.
	\begin{itemize}
		\item \textbf{All-pay auction}. For bids \(b_1, \ldots , b_I\),
		      \begin{itemize}
			      \item The highest bidder wins.
			      \item Everyone pays their bid.
		      \end{itemize}
	\end{itemize}
\end{eg}

For a general auction, the bids are just messages. Auctioneer is specifying two things.
\begin{itemize}
	\item Allocation rule: \((x_1, \ldots , x_I)\), where \(x_{i}\) is the amount of good goes to agent \(i\).
	\item Payment rule: \((p_1, \ldots , p_I)\), where \(p_{i}\) to be paid by agent \(i\).
\end{itemize}

\begin{definition}
	In the specification, if the messages are in the same space as values, then we call such a mechanism as a \emph{direct mechanism}.
\end{definition}

\begin{prev}
	From \hyperref[Revelation-Principle]{Revelation Principle}, \emph{direct mechanisms} are sufficient. Since bids are messages, which can be anything, hence
	it suffices to consider messages to be in the same space as the values.
\end{prev}

\hr

We assume that the message space is the same as value space, namely we are designing a direct mechanism. For every message vector, specifies the
allocation rule and the payment rule. Furthermore, we let
\[
	\text{Utility} = \text{Value} - \text{Payment},
\]
which turns out to be so-called Quasilinear utility\footnote{\url{https://en.wikipedia.org/wiki/Quasilinear_utility}}.

\begin{note}
	There are things that a mechanism designer may want.
	\begin{itemize}
		\item Incentive Compatible(IC)
		\item Individual rationality(IR)
	\end{itemize}
	Also, a mechanism designer may have a goal in mind.
	\begin{itemize}
		\item Social welfare maximization(Efficiency)
		\item Revenue maximization
		\item Budget-Balance: Payments should sum to \(0\).
	\end{itemize}

	\begin{remark}
		All can't be met simultaneously.
	\end{remark}
\end{note}

Assume every agent \(i\) values \(v_1, \ldots , v_{I}\), and bids \(b_1, \ldots , b_{I}\) respectively. With the specification being
\[
	\begin{split}
		\text{Allocations}&: x_1, \ldots , x_I\\
		\text{Payments}&:p_1, \ldots , p_I,
	\end{split}
\]
where all are functions of \(b_{i}\). Then the utility is
\[
	u_{i}(v_{i}, n_{i}(\vec{v})) - p_{i}(\vec{v}).
\]
With interim set-up, we average across all other agents.

Now, we assume that \(x_{i}(\vec{v})\in\{0, 1\}\) and will allow multiple agents to get
\[
	x_{i}(\vec{v}) = 1.
\]

If \(x_{i}(\vec{v})=1\), then utility is \(v_{i}\), else \(0\). We then see that
\[
	u_{i}(v_{i}, x_{i}(\vec{v})) = v_{i}x_{i}(\vec{v}) - p_{i}(\vec{v}).
\]
Then the expected value is
\[
	\begin{split}
		\expectation{-i}{u_{i}(v_{i}, x_{i}(\vec{v}))} &= v_{i} \expectation{-i}{x_{i}(\vec{v})} - \expectation{-i}{p_{i}(\vec{v})}\\
		&= v_{i} \expectation{-i}{x_{i}(v_{i}, v_{-i})} - \expectation{-i}{p_{i}(v_{i}, v_{-i})}\\
		&= v_{i} \overline{x}_{i}(v_{i})-\overline{p}_{i}(v_{i}).
	\end{split}
\]

Now we see a theorem.
\begin{theorem}
	Bayesian Nash Equilibrium characterization. When the types are drawn from a continuous joint distribution \(\vec{F}\) with independent types,
	strategies
	\[
		\vec{\sigma}^{\ast} = (\sigma^{\ast}_1, \sigma^{\ast}_2, \ldots , \sigma^{\ast}_I)\qquad (b_{i} = \sigma^{\ast}_{i}(v_{i}))
	\]
	are in a Bayesian Nash Equilibrium if and only if
	\begin{itemize}
		\item Monotonicity. \(\overline{x}_{i}(v_{i})\) is monotonically non-decreasing.
		\item Payment identity.
		      \[
			      \overline{p}_{i}(v_{i}) = v_{i}\overline{x}_{i}(v_{i}) - \int_0^{v_{i}} \overline{x}_{i}(z)\,\mathrm{d}z + \overline{p}_{i}(0),
		      \]
		      where often \(\overline{p}_{i}(0) = 0\).
	\end{itemize}
	If the strategy profile is onto, the converse is true as well.
\end{theorem}