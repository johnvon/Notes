\lecture{4}{13 Sep. 12:30}{Network Properties}
\begin{eg}
	The number of triangles can be used to quantify the connectedness of a network. For example, the \emph{triadic closure}.
\end{eg}

\begin{definition}
	For each node \(i\), define
	\[
		c_{i} = \dfrac{\#\text{ of triangles in the graph that include node }i}{\frac{d_{i}(d_{i}-1)}{2}},
	\]
	where \(c_{i}\) is called \emph{clustering coefficient}.
\end{definition}
\begin{remark}
	We have, for any node \(i\), \(c_{i}\in \left[ 0, 1 \right].\) If \(\forall i\ c_{i} = 1\), then this graph is complete. If \(\forall i\ c_{i} = 0\), then
	this graph is a tree.
\end{remark}

\subsubsection{Bridges and Local Bridges}
\begin{definition}
	A \emph{bridge} is the only edge that connects together two subgraphs that are themselves connected.
\end{definition}

\begin{remark}
	Deleting a bridge may change the distance between two nodes from some finite value to \(\infty\).
\end{remark}

\begin{definition}
	An edge \((i, j)\) is said to be a \emph{local bridge} if \(i, j\) have no neighbors in common. Equivalently, a local bridge is an edge that is not included in any
	triangles.
\end{definition}

\begin{remark}
	If \((i, j)\) is a local bridge, then we have
	\[
		d_{\mathcal{G}}(i, j) \geq  3
	\]
	after removing this local bridge. Furthermore, a bridge must also be a local bridge by definition.
\end{remark}

\begin{definition}
	The \emph{span of a local bridge} \((i, j)\) is defined as
	\[
		d_{\mathcal{G}\mid_{(i, j)\text{ is removed from the graph}}}.
	\]
\end{definition}

\subsection{Triadic closure and Overlapping Triangles}
\begin{intuition}
	(David Easley, Jon Kleinberg, 2010 pp.62:) If two people in a social network have a friend in common, then there is an increased likelihood
	that they will become friends themselves at some point in the future.
\end{intuition}

\begin{definition}
	If \(a\) is a node and there are two nodes \(b, c\) such that \((a, b)\) and \((a, c)\) are strong ties, then \((b, c)\) will form an edge.
	If for any nodes \(a, b, c\) in this graph \(\mathcal{G}\) satisfy this property, then we say this graph has \emph{strong triadic closure property}, or STC property.
\end{definition}

\begin{definition}
	The \emph{embeddness} of an edge is defined as the number of common neighbors shared by the end points.
\end{definition}


