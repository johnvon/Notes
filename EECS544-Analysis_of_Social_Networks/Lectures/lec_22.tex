
\chapter{Auctions}\label{ch:auctions}
\lecture{22}{29 Nov. 12:30}{Auctions}
We now introduce a specific kind of \hyperref[def:mathematical-Bayesian-game]{game}, auction. Rather than the simple set up we have seen,
the auction is more complex and interesting. We'll see how can we extend what we have discussed in \hyperref[ch:game-theory]{game theory}
into this set up. Firstly, we see some examples.

\begin{eg}[Auctions]
	There are mainly four types of auction:
	\begin{enumerate}
		\item\label{eg:first-price-auction} First-price auction (Sealed bid): Each \hyperref[def:player]{player} bids an amount for the item (how much the \hyperref[def:player]{player} is willing to pay),
		and the sellers selects the highest bidder and charges him the price that was bid by him and give the item. Others pay nothing, receive nothing.
		\item\label{eg:Dutch-auction} Dutch auction (Descending price auction): Seller announces a price (high price). If no \hyperref[def:player]{players} raise hands, seller drops price until
		the first time some \hyperref[def:player]{players} have their hand up. Item sold to the \hyperref[def:player]{player} at that price.
		\item\label{eg:second-price-auction} Second-price auction (Sealed bid):
		\begin{enumerate}
			\item Sort the bids from the highest
			\item Find the highest bidder
			\item Give the good to this bidder but charge him/her the second-highest bid.
		\end{enumerate}
		\item\label{eg:English-auction} English auction (Ascending price auction):
		\begin{enumerate}
			\item Start from price \(0\)
			\item Keeping increasing price until only one left
			\item Sell the good to the person remaining and at the price at which everyone else drops out.
		\end{enumerate}
	\end{enumerate}
\end{eg}
\begin{remark}
	The first two auctions basically implement the same result.
\end{remark}

\begin{note}[Tie breaking]
	Some form of auctions may encounter tie issues. Some common way to do ties breaking are
	\begin{itemize}
		\item Random order.
		\item Choose some fixed order (specified earlier).
	\end{itemize}
\end{note}

\begin{eg}[Second-price auction]
	For example, we have
	\begin{table}[H]
		\centering
		\begin{tabular}{c|c|c|c|c}
			\toprule
			          & A  & B & C & D \\
			\midrule
			\(b_{i}\) & 10 & 2 & 3 & 7 \\
			\bottomrule
		\end{tabular}
	\end{table}
	After sorted:
	\begin{table}[H]
		\centering
		\begin{tabular}{c|c|c|c|c}
			\toprule
			          & A  & D & C & B \\
			\midrule
			\(b_{i}\) & 10 & 7 & 3 & 2 \\
			\bottomrule
		\end{tabular}
	\end{table}
	We then sell the good to \(A\) with price \(7\).
\end{eg}

\begin{eg}[English auction]
	For example, we start with \(0\), and everyone wants the good. Then we slightly increase the price, until:
	\begin{enumerate}
		\item Price is \(2\) or \((2+\epsilon)\):
		      A, D and C want the good, B drops out.
		\item Price is \(3\) or \((3+\epsilon)\):
		      A and D want the good, C drops out too.
		\item Price is \(7\) or \((7+\epsilon)\):
		      A want the good, D drops out too.
	\end{enumerate}
	Then A gets the good with price being \(7\).
\end{eg}

We relate auctions with \hyperref[def:mathematical-Bayesian-game]{game} as we discussed before. Given the \hyperref[def:type]{types} for \hyperref[def:player]{player}
\(i\in \mathcal{I} \) being \hyperref[def:i.i.d.]{i.i.d.} such that
\[
	t_1, t_2, \ldots , t_I,
\]
and also for the valuations such that
\[
	v_1, v_2, \ldots , v_I.
\]

\begin{note}
	We see that
	\begin{itemize}
		\item For simplicity, without specifying, we assume they are distributed in
		      \[
			      \mathrm{uniform} ([0, 1]).
		      \]
		\item The valuation \(v_{i}\) is taken as the satisfaction in dollar amount from buying the good.
	\end{itemize}
\end{note}

Then, the \hyperref[def:reward]{utility} function is determined based on specific forms of auction.

\section{Order Statistics}
For all kinds of auctions we mentioned, we see that to analyze auctions, it'll involve some kind of order of bidding. In the
content of auctions, we often refer this as \emph{order statistics}.

Let \(X_1, X_2, \ldots , X_n\) be \hyperref[def:i.i.d.]{i.i.d.} and non-negative, continuous valued \hyperref[def:random-variable]{random variables}. Let
\(F_{X}(\cdot)\) be the CDF, and \(f_{X}(\cdot)\) be the probability density function.
\begin{eg}
	Consider
	\begin{enumerate}
		\item \(X = \mathrm{uniform}([0, 1])\):
		      \[
			      F_X(x) = \begin{dcases}
				      x,         & \text{ if } x\in[0, 1]; \\
				      \min(x, 1) & \text{ if }x>1,         \\
			      \end{dcases}
		      \]
		      with
		      \[
			      f_X(x) = \begin{dcases}
				      1, & \text{ if } x\in[0, 1]; \\
				      0, & \text{ if }x>1.
			      \end{dcases}
		      \]
		\item \(\mathrm{Exp}(\lambda)\):
		      \[
			      F_X(x) = 1 - e^{-\lambda x}, \quad x\geq 0,
		      \]
		      with
		      \[
			      f_X(x) = \lambda e^{-\lambda x},\quad x\geq 0.
		      \]
	\end{enumerate}
\end{eg}

The idea is simple, we first sort \(x_1, x_2, \ldots , x_n\) in decreasing order and denotes
\[
	y_1, y_2, \ldots , y_{n}
\]
such that
\[
	\begin{split}
		y_1 &= \max\{x_1, \ldots , x_n\}\\
		y_2 &= \max\{x_1, \ldots , x_n\}\setminus \{y_1\}\\
		y_3 &= \max\{x_1, \ldots , x_n\}\setminus \{y_2, y_3\}\\
		&\vdots
	\end{split}
\]

Then we have
\[
	\max\{x_1, \ldots , x_n\} = y_1\geq y_2\geq \ldots \geq y_n = \min\{x_1, \ldots , x_{n}\}.
\]
Now, we want to find out \(\expectation{}{Y_1} \), which is just
\[
	\expectation{}{Y_1} = \int_0^{\infty } y f_{Y_1}(y)\,\mathrm{d}y
\]
where \(f_{Y_1}\) is the probability density function of \(Y_1\). Since \(f_{Y_1}(y) = \frac{\mathrm{d}}{\mathrm{d}y} \left(F_{Y_1}(y)\right)\),
where
\[
	\begin{split}
		F_{Y_1}(y) &= \probability{}{Y_1\leq y}
		= \probability{}{\max\{X_1, \ldots , X_{n}\}\leq y}\\
		&= \probability{}{X_1\leq y,x_2\leq y, \ldots , X_n\leq y}\\
		&= \prod\limits_{i=1}^{n} \probability{}{X_{i}\leq y}
		= \prod\limits_{i=1}^{n} F_{X_{i}}(y)
		= \left(F_X(y)\right)^n,
	\end{split}
\]
where we have used the properties of the \hyperref[def:i.i.d.]{i.i.d. assumption}.

Further, since we now know
\[
	f_{Y_1}(y) = \frac{\mathrm{d}}{\mathrm{d}y}\left(\left(F_X(y)\right)^n\right) = n \left(F_X(y)\right)^{n-1} \frac{\mathrm{d}}{\mathrm{d}y} F_X(y) = n\left(F_X (y)\right)^{n-1}f_X(y),
\]
we have
\[
	\expectation{}{Y_1} = \int_0^{\infty }n y \left(F_{X}(y)\right)^{n-1} f_X(y)\,\mathrm{d}y.
\]

\begin{eg}
	Let \(F_{X}(\cdot)\sim \mathrm{uniform}([0,1]) \). Then
	\[
		\expectation{}{Y_1} = \int_0^1 n y y^{n-1}\,\mathrm{d}y = n\int_0^1 y^n \,\mathrm{d}y = \at{\frac{ny^{n+1}}{n+1}}{0}{1} = \frac{n}{n+1} .
	\]
	Hence, \(\expectation{}{X_1}= \frac{1}{2} \).
	\begin{remark}
		The expected value of the \(k^{\mathrm{th}}\) element is exactly
		\[
			\frac{k}{n+1}
		\]
		as one can check.
	\end{remark}
\end{eg}

\section{Second-Price Auction}
We first analyze \hyperref[eg:second-price-auction]{second-price auction}, since it's easier to analyze. Recall the \hyperref[def:interim]{interim} setting,
namely each \hyperref[def:player]{agent} know its valuation but only the distribution of the valuation of others. Now, let
\[
	\sigma_{i}\colon [0, 1]\to [0, 1]\qquad v_{i}\mapsto  \sigma_{i}(v_{i})
\]
where \(\sigma_{i}\) is just the \hyperref[def:strategy]{strategy} of bidding. Then the \hyperref[def:reward]{utility} is
\[
	\expectation{-i}{u_{i}(\underbrace{v_{i}, \sigma_{i}(v_{i})}_{\text{known for \(i\)}}, \underbrace{v_{-i}, \sigma_{-i}(v_{-i})}_{\text{random}})}.
\]

The goal is to choose \(\sigma_{i}(\cdot)\) such that expected \hyperref[def:reward]{utility} is maximized for each \(v_{i}\)
assuming \(\sigma_{-i}\) is fixed.

Then,
\[
	\begin{split}
		u_{i}(v_{i}, \sigma_{i}(v_{i}), v_{-i}, \sigma_{-i}(v_{-i}))
		=&\left(v_{i} - \max_{j\in I\setminus\{i\}}\sigma_{j}(v_{j})\right)\mathbbm{1}_{\{\text{\(i\) wins the auction}\}}\\
		=&\left(v_{i} - \max_{j\in -i}\sigma_{j}(v_{j})\right)\mathbbm{1}_{\{ \sigma_{i}(v_{i})\geq \sigma_{j}(v_{j}) \forall j\in -i \}}.
	\end{split}
\]

We now aim to maximize
\[
	\expectation{-i}{u_{i}(v_{i}, \sigma_{i}(v_{i}), v_{-i}, \sigma_{-i}(v_{-i}))}.
\]

\subsection{Reason of Response}
We need to reason how \hyperref[def:player]{agents} will bid strategically.
\begin{itemize}
	\item If \(b_{i} = \sigma_{i}(v_{i})>v_{i}\).
	      \begin{enumerate}[(a)]
		      \item \(b_{i}< \widetilde{P}_i = \max\limits_{j\in -i}\sigma_{j}(v_{j}) = b_{j}\), then we see that
		            \(\max(b_{i}, \widetilde{P}_{i}) = \widetilde{P}_{i}\), so \(i\) never wins the auction.
		            \par Continue to do so even if he reduces bid to \(v_{i}\).
		      \item \(b_{i} = \widetilde{P}_{i}\), then \(i\) can win the auction.
		            \par Since \(v_{i} - \widetilde{P}_{i}<0\),  and the second-highest bid is \(\widetilde{P}_{i}\). So He better to bid \(v_{i}\) and
		            lost the auction.
		      \item \(b_{i} > \widetilde{P}_{i}\), then \(i\) definitely wins the auction.
		            \begin{itemize}
			            \item \(v_{i}<\widetilde{P}_{i}\): \hyperref[def:reward]{utility}  is \(v_{i} - \widetilde{P}_{i}<0\), not a viable option.
			            \item \(v_{i}\geq \widetilde{P}_{i}\): \hyperref[def:reward]{utility}  is \(v_{i} - \widetilde{P}_{i}\geq 0\), but
			                  same result would also hold if \(b_{i}\) is exactly \(v_{i}\).
		            \end{itemize}
	      \end{enumerate}
	\item  If \(b_{i} = \sigma_{i}(v_{i})<v_{i}\).
	      \begin{enumerate}
		      \item[(a)] \(\widetilde{P}_{i}\leq b_{i}<v_{i}\). \hyperref[def:player]{Agent} \(i\) wins with \hyperref[def:reward]{utility}  being \(v_{i} - \widetilde{P}_{i}\geq 0\) and
			      no lose in increasing bid to \(v_{i}\).
		      \item[(b)] \(b_{i}< v_{i}\leq \widetilde{P}_{i}\). Don't win the auction. No lose in increasing bid to \(v_{i}\).
		      \item[(c)] \(b_{i}\leq \widetilde{P}_{i}\leq v_{i}\). Lose the auction. However, bidding \(v_{i}\) the \hyperref[def:player]{agent} could have won the
			      auction and gotten non-negative \hyperref[def:reward]{utility} .
	      \end{enumerate}
	      \begin{remark}[Incentive compatible]\label{rmk:incentive-compatible}
		      Bidding \(v_i\) is a (\hyperref[def:weakly-dominant-strategy]{weakly}) \hyperref[def:dominant-strategy]{dominant strategy} for
		      \hyperref[def:player]{agent} \(i\). This further implies that for \hyperref[eg:second-price-auction]{second-price auction}, \hyperref[def:player]{agents} bid their true value.
		      This is essentially \emph{incentive compatibility}, or \emph{truth-telling desirable}.
	      \end{remark}
\end{itemize}

\begin{remark}
	We see that \hyperref[eg:second-price-auction]{second-price auction} is truth-telling in \hyperref[def:dominant-strategy]{dominant strategies}.
	Hence, irrespective of how the other players bid, truthfully bidding is optimal, namely
	\[
		\sigma_{i}(v_{i}) = v_{i}.
	\]
\end{remark}

\subsection{Revenue of the Auctioneer}
Given \(v_1, v_2, \ldots , v_I\), defined \(\widetilde{v}_i\) as before, namely
\[
	\max\{v_1, \ldots , v_I\} = \widetilde{v}_{I}\geq \widetilde{v}_{I-1}\geq \ldots \geq \widetilde{v}_1 = \min\{v_1, \ldots , v_I\}.
\]

And since it's a \hyperref[eg:second-price-auction]{second-price auction}, so the revenue is \(\widetilde{v}_{I - 1}\).

\begin{note}
	The expected revenue of \(\widetilde{v}_{I - 1}\) is
	\[
		\expectation{}{\widetilde{v}_{I-1}} = \frac{I-1}{I+1}
	\]
	if we assume uniform valuation.
\end{note}

Hence, we see that the winner gets \(\widetilde{v}_{I} - \widetilde{v}_{I-1}\), and the expected \hyperref[def:reward]{utility}  is
\[
	\expectation{}{\widetilde{v}_{I}} - \expectation{}{\widetilde{v}_{I - 1}}.
\]
In the uniform case, the above further equals to
\[
	\frac{I}{I+1} - \frac{I - 1}{I + 1} = \frac{1}{I+1}.
\]

With winner and auctioneer together, their \hyperref[def:reward]{utility}  is
\[
	\widetilde{v}_{I} - \widetilde{v}_{I - 1}+\widetilde{v}_{I - 1} = \widetilde{v}_{I}.
\]

\begin{remark}
	We see that the good always goes to the highest valuation.
\end{remark}

Sums of \hyperref[def:reward]{utilities} of all players in auctioneer is \hyperref[def:social-welfare]{social welfare} - max value possible of \(\widetilde{v}_{I}\).

\begin{remark}
	\hyperref[eg:second-price-auction]{second-price auction} is \hyperref[def:social-welfare]{social welfare} maximizing.
\end{remark}

Now, if \hyperref[def:player]{agent} \(i\) bids \(v_{i}\) and wins, then he gets
\[
	v_{i} = \max_{j\in -i} v_{j}\geq 0,
\]
and if lose then get \(0\). Hence, the expected \hyperref[def:reward]{utility} y is always non-negative if you attend the auction, while staying
out will get you \(0\), hence we see that it's better to participate in the auction.

\begin{remark}[Voluntary participation]\label{rmk:voluntary-participation}
	This phenomenon is called \emph{voluntary participation}, or \emph{ndividual rationality}.
\end{remark}

And since the probability of winning is \(\frac{1}{I}\), hence the expected \hyperref[def:reward]{utility} is
\[
	\frac{1}{I}\times \frac{1}{I+1} = \frac{1}{I(I + 1)}
\]
for every \hyperref[def:player]{agent}.

\begin{note}
	Even if we are using \hyperref[def:ex-post]{ex-post} setting, it's still optimal to bid truthfully in a
	\hyperref[eg:second-price-auction]{second-price auction}.
\end{note}

\section{First-Price Auction}
We'll see that the \hyperref[eg:first-price-auction]{first price auction} is not as \textbf{simple}. There are no \hyperref[def:dominant-strategy]{dominant strategy}
\hyperref[def:Nash-equilibrium]{equilibrium}, only so-called \hyperref[def:mathematical-Bayesian-game]{Bayes}-\hyperref[def:Nash-equilibrium]{Nash equilibrium} in
\hyperref[def:interim]{interim}.

Consider \(I = 2\), and \(v_{i}\) choose \(b_{i} = \sigma_{i}(v_{i})\). The \hyperref[def:reward]{utility}  is
\[
	u_{i}(v_{i}, b_{i}, v_{-i}, b_{-i}) = \left(v_{i} - \max_{j\in \mathcal{I} }b_{i}\right)\mathbbm{1}_{\{ i \text{ is the winner} \}} = (v_{i} - b_{i})\mathbbm{1}_{\{ b_{i}\geq b_{-i} \}}
\]

If \(I>2\), then the \hyperref[def:reward]{utility}  is now
\[
	(v_{i} - b_{i})\mathbbm{1}_{\{ \sigma_{i}(v_{i})\geq \sigma_{j}(v_{j})\forall j\in -i \}} = (v_{i} - b_{i})\prod\limits_{j\in -i} \mathbbm{1}_{\{ \sigma_{i}(v_{i})\geq \sigma_{j}(v_{j}) \}}
\]
Then, since we're in the \hyperref[def:interim]{interim} setup, the expected value of the \hyperref[def:reward]{utility}  is
\[
	\begin{split}
		\expectation{-i}{u_{i}(v_{i}, \sigma_{i}(v_{i}), v_{-i}, \sigma_{-i}(v_{-i}))}
		= &\expectation{-i}{(v_{i} - \sigma_{i}(v_{i})) \prod\limits_{j\in -i}\mathbbm{1}_{\{ \sigma_{i}(v_{i})\geq \sigma_{j}(v_{j}) \}}}\\
		= &(v_{i} - \sigma_{i}(v_{i}))\prod\limits_{j\in -i}\probability{}{\sigma_{j}(v_{j})\leq \sigma_{i}(v_{i})},
	\end{split}
\]
where the last equality comes from \hyperref[def:independent]{independence}.

\begin{note}
	There are two facts to note.
	\begin{itemize}
		\item Identical valuation implies \(\sigma_{i} \equiv \sigma\) for all \(i\in \mathcal{I} \). Same bidding function so it's symmetric.
		\item \(\sigma(\cdot)\) is monotonically increasing. Then
		      \[
			      \probability{}{\sigma_{j}(v_{j})\leq \sigma_{i}(v_{i})} = \probability{}{\sigma(v_{i})\leq \sigma(v_{i})} = \probability{}{v_{j}\leq v_{i}} = F(v_{i}) = v_{i},
		      \]
		      where \(F\) is the CDF, and the last equality comes from the fact that we assume the valuation is \(\mathrm{uniform}([0, 1])\).

		      This is a fairly reasonable assumption, since it'll rule out the situations like if two buyers' true value are different, but they somehow decide to bid
		      the same value.
	\end{itemize}
\end{note}

With the second fact of the note above, the expected \hyperref[def:reward]{utility}  is
\[
	(v_{i} - \sigma(v_{i})) F^{I-1}(v_{i}).
\]
If we assume that this is uniform valuation, then above further equals to
\[
	(v_{i} - \sigma(v_{i}))v^{I-1}_{i}.
\]