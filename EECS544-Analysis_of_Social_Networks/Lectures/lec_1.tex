\chapter{Graph and Network}
\lecture{1}{30 Aug. 12:30}{Introduction to Graph Theory}
To analyze a social network, we should first formalize what counts as a network mathematically. In such a way, we're able to
develop a mathematically complete and rigorous framework to analyze a given social network and infer some useful properties.

\section{Graph}
Let's start with some of the most fundamental definitions.

\begin{definition}[Vertex set]\label{def:vertex-set}
	A set is called a \emph{vertex set} if we view its elements as vertices.\footnote{Sometimes \emph{nodes}.}
\end{definition}

\begin{definition}[Graph]\label{def:graph}
	There are two kinds of \emph{graph} we're interested in, \hyperref[def:undirected-graph]{undirected graph} and \hyperref[def:directed-graph]{directed graph}.
	\begin{definition}[Undirected graph]\label{def:undirected-graph}
		An \emph{undirected graph} is a pair \(\mathcal{G} = (\mathcal{\MakeUppercase{v}} , \mathcal{\MakeUppercase{e}} )\), where \(\mathcal{\MakeUppercase{v}} \) is
		the \hyperref[def:vertex-set]{vertex set} and \(\mathcal{\MakeUppercase{e}} \) is the \hyperref[def:edge-set]{edge set}.

		\begin{definition}[Edge set]\label{def:edge-set}
			An \emph{edge set} \(\mathcal{\MakeUppercase{e}} \) of an \hyperref[def:undirected-graph]{undirected graph} \(\mathcal{\MakeUppercase{g}} \) is a set of paired vertices, which we call edges.\footnote{Sometimes \emph{links}.}
		\end{definition}
	\end{definition}

	\begin{definition}[Directed grpah]\label{def:directed-graph}
		A \emph{directed graph} is a pair \(\mathcal{\MakeUppercase{g}} = (\mathcal{\MakeUppercase{v}}, \mathcal{\MakeUppercase{e}} )\), where \(\mathcal{\MakeUppercase{v}} \)
		is the \hyperref[def:vertex-set]{vertex set} and \(\mathcal{\MakeUppercase{e}} \) is the \hyperref[def:directed-edge-set]{directed edge set}.

		\begin{definition}[Directed edge set]\label{def:directed-edge-set}
			A \emph{directed edge set} \(\mathcal{\MakeUppercase{e}} \) of a \hyperref[def:directed-graph]{directed graph} \(\mathcal{\MakeUppercase{g}} \) is a set of ordered-paired vertices, which we call directed edges.\footnote{Sometimes \emph{directed links}.}
		\end{definition}
	\end{definition}
\end{definition}

\begin{eg}[Undirected and directed graph]
	Let \(A = \{a, b, c, d, e\}\) be a \hyperref[def:vertex-set]{vertex set}, then an example of \hyperref[def:edge-set]{edge sets} can be
	\[
		B \coloneqq \{\{a, b\}, \{b, e\}, \{c, d\}\},
	\]
	and
	\[
		B^\prime \coloneqq \{(a, b), (b, e), (c, d)\}
	\]
	is an example of \hyperref[def:directed-edge-set]{directed edge set}. Together, \((A, B)\) is a valid \hyperref[def:undirected-graph]{undirected graph}, while
	\((A, B^\prime )\) is a \hyperref[def:directed-graph]{directed graph}.
\end{eg}

\begin{notation}
	We some time will write an edge between \(a\) and \(b\) as \((a, b)\) instead of \(\{a, b\}\) and a directed edge from \(a\) to \(b\) as \(a\to b\) if the content is clear.
\end{notation}

\begin{note}
	While the definition of a \hyperref[def:vertex-set]{vertex set} is purely abstract, we often just deal with concrete graph where the structure of the graph is
	clear (e.g., in social network, every people are vertices, and relationships between people are edges).
\end{note}

\begin{definition*}
	Given a \hyperref[def:directed-graph]{directed graph} \(\mathcal{\MakeUppercase{g}} \), we can define the followings.
	\begin{definition}[In-degree]\label{def:in-degree}
		For a node \(v\in\mathcal{V}\), the \emph{in-degree} of \(v\) in a \hyperref[def:directed-graph]{directed graph} \(\mathcal{G}\) is defined as the number of
		directed edges pointing in to \(v\).
	\end{definition}

	\begin{definition}[Out-degree]\label{def:out-degree}
		For a node \(v\in\mathcal{V}\), the \emph{out-degree} of \(v\) in a \hyperref[def:directed-graph]{directed graph} \(\mathcal{G}\) is defined as the number of
		directed edges pointing out from \(v\).
	\end{definition}
\end{definition*}

Moreover, if we generalize above to an \hyperref[def:undirected-graph]{undirected graph} we have the following.
\begin{definition}[Degree]\label{def:degree}
	Given an \hyperref[def:undirected-graph]{undirected graph}, for a node \(v\in \mathcal{\MakeUppercase{v}} \), the \emph{degree} of \(v\) in an
	\hyperref[def:undirected-graph]{undirected graph} is defined as the number of edges link with \(v\).
\end{definition}

\begin{definition}[Subgraph]\label{def:subgraph}
	A \emph{subgraph}\footnote{Sometimes \emph{component}.} \(\mathcal{G}^\prime\) is defined by a pair \((\mathcal{V}^\prime, \mathcal{E}^\prime)\) such that \(\mathcal{V}^\prime \subseteq \mathcal{V}\),
	\(\mathcal{E}^\prime\subseteq\mathcal{E}\).
\end{definition}

\begin{note}
	Notice that the edges in \(\mathcal{E}^\prime\) need to be well-defined. Namely, for every
	\(e = (a,b) \in\mathcal{E}^\prime\), \(a, b\) need to also be in \(\mathcal{V}^\prime\).
\end{note}

Finally, we introduce the following definition.
\begin{definition}[Simple graph]\label{def:simple-graph}
	A \emph{simple graph} is a \hyperref[def:graph]{graph} without loops and multiple edges.\footnote{i.e., no edges like \(\{a, a\}\) and there is only one edge between any two nodes.}
\end{definition}