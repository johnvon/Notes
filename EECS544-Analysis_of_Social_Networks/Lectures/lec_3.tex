\lecture{3}{8 Sep. 12:30}{Network Property}
\section{Network}
A network is essentially a graph, with some interesting properties. We start with connectivity.

\subsection{Measuring Connectivity}
\begin{intuition}
	If there are more paths in the graph between different parts of the graph, then this graph is \emph{more connected}.
\end{intuition}
\subsubsection{Measuring the Difference between Two Paths}
\begin{definition}
	Two paths are said to be \emph{edge independent(EI)} if they do not share a common edge.
\end{definition}

\begin{definition}
	To paths are said to be \emph{vertex independent(VI)} if they do not share a common vertex except the starting vertex and the ending vertex.
\end{definition}

\subsubsection{Cut Set}
\begin{definition}
	Given nodes \(i, j\), a set of edges is called an \emph{edge cut set} if \(i, j\) are in different connected components after removing this set of edges from the original
	graph.
\end{definition}
Similarly, one can define \emph{vertex cut set} for a given pair of nodes.

\begin{remark}
	This is something to do with graph subtraction, which is essentially just the subtraction between the edge set and the vertex set. Notice that this subtraction needs to be
	well-defined as well.
\end{remark}

\begin{theorem}
	Mengur's Theorem: If for any pair of nodes \((i, j)\),
	\[
		\left\vert \mathrm{ECS}(i, j) \right\vert > n,
	\]
	then for any pair of nodes \((i, j)\),
	\[
		\# \text{ edge independent paths between }i \text{ and }j > n.
	\]
\end{theorem}

