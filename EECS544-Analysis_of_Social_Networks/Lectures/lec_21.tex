\lecture{21}{17 Nov. 12:30}{Bayesian Game}
\begin{prev}
	We look back some games and introduce another measure of Nash Equilibrium.
	\begin{itemize}
		\item Coordination game. The payoff matrix is defined as
		      \begin{table}[H]
			      \centering
			      \setlength{\extrarowheight}{2pt}
			      \begin{tabular}{cc|c|c|}
				                                & \multicolumn{1}{c}{} & \multicolumn{2}{c}{Player $2$}                           \\
				                                & \multicolumn{1}{c}{} & \multicolumn{1}{c}{$S$}        & \multicolumn{1}{c}{$H$} \\\cline{3-4}
				      \multirow{2}*{Player $1$} & $S$                  & $(5, 5)$                       & $(0, 3)$                \\\cline{3-4}
				                                & $H$                  & $(3, 0)$                       & $(3, 3)$                \\\cline{3-4}
			      \end{tabular}
		      \end{table}
		      As we can see, there are two Nash Equilibrium \((S, S)\) and \((H, H)\) with payoffs \((5, 5)\) and \((3, 3)\) respectively. We can
		      further consider so-called \textbf{Social welfare}, then in order to maximize the sum of
		      payoffs, one will choose to play \((S, S)\) with social welfare as \(10\) rather than \(6\).
		\item \textbf{Prisoner's Dilemma}.
		      \begin{table}[H]
			      \centering
			      \setlength{\extrarowheight}{2pt}
			      \begin{tabular}{cc|c|c|}
				                                & \multicolumn{1}{c}{} & \multicolumn{2}{c}{Player $2$}                                       \\
				                                & \multicolumn{1}{c}{} & \multicolumn{1}{c}{$\mathrm{NC}$} & \multicolumn{1}{c}{$\mathrm{C}$} \\\cline{3-4}
				      \multirow{2}*{Player $1$} & $\mathrm{NC}$        & $(-1, -1)$                        & $(-10, 0)$                       \\\cline{3-4}
				                                & $\mathrm{C}$         & $(0, -10)$                        & $(-4, -4)$                       \\\cline{3-4}
			      \end{tabular}
		      \end{table}
		      We see that \((\mathrm{C} , \mathrm{C} )\) is the only Nash Equilibrium, but \((\mathrm{NC} , \mathrm{NC} )\) is the social welfare maximization, which is not in
		      the Nash Equilibrium.
	\end{itemize}
\end{prev}

\begin{definition}
	A Nash Equilibrium reaches \emph{Social Welfare} if it's the maximizer of the \textbf{sum of the payoffs} among all other states.
\end{definition}


\subsection{Operating Points}
Let \(i\in \mathcal{I} \) be the set of players, and \(P_{i}\) be the corresponding distribution of strategy on player \(i\). Then we see the
so-called \emph{operating points}.

\begin{figure}[H]
	\centering
	\incfig{NE-social-welfare}
	\label{fig:NE-social-welfare}
\end{figure}

Let \(P\) denotes the set of operating points. We see that we cannot increase the payoff of one agent without decreasing the payoff of at least
one other agent. From here, we have the following criteria for \emph{Pareto Optimal}.

\begin{definition}
	A solution is called \emph{Pareto Optimal} if no individual or preference criterion can be better off without making at least one
	individual or preference criterion worse off or without any loss thereof.
\end{definition}

\begin{figure}[H]
	\centering
	\incfig{Pareto-Optimal}
	\caption{Parteo Optimality}
	\label{fig:Pareto-Optimal}
\end{figure}

\begin{note}
	For either social welfare maximizing or Pareto operating point,
	\begin{itemize}
		\item Nash Equilibrium need not achieve either goal.
		\item Social welfare maximizing points are always Pareto optimal.
	\end{itemize}
\end{note}

\begin{remark}
	We can (re)designing games such that some nice objective is additive, then social welfare maximization can be achieved.
	\begin{itemize}
		\item The goal is that any Nash Equilibrium should be social welfare maximizing.
	\end{itemize}
\end{remark}

\hr

Until this point, we only study the games are complete and symmetric information - Every agent knows actions, utility function of everyone(common knowledge assumption).
Now, we can look at some more complicated case such that we have an Incomplete and/or asymmetric information setting.

\section{Bayesian Game}
This is a framework for asymmetric/incomplete games such as auctions. And since we need to use \emph{Bayesian Theorem} to analysis such games, so
it gets its name.

An important difference is that in Bayesian games, nation is also a player. Further, randomness decides the state of everyone and people may
know their state but not of the others.

\begin{note}
	Randomness is different from mixed strategies. This happens \textbf{at the beginning}.
\end{note}

\subsection{Basic setup}
Let \(\mathcal{I} \) denotes the set of players, and \(I = \left\vert \mathcal{I}  \right\vert \). Let each player \(i\) has a type
\(\tau_{i}\in T_{i}\) chosen with distribution \(\pi_i\) such that
\[
	\probability{}{\tau_{i} = t_{i}} = \pi_i(t_{i}) .
\]
And also, we have the set of actions \(\mathcal{S}_{i}\), and the set of feasible action
\[
	c_{i}(t_{i})\in \mathcal{S}_{i}.
\]

Furthermore, the utility of player \(i\) is
\[
	u_{i}(t_1, s_1, t_2, s_2, \ldots , t_I, s_I),
\]
where we need types and action of all players to determine this. Also, we have
\[
	\vec{t}\coloneqq (t_1, \ldots , t_I),\quad \vec{s} \coloneqq (s_1, \ldots , s_I),
\]
with the common notation for \(-i\), namely
\[
	\vec{t}_{-i} \coloneqq (t_1, \ldots , t_{i-1}, t_{i+1}, \ldots , t_{I}),\quad \vec{s}_{-i} \coloneqq (s_1, \ldots , s_{i-1}, s_{i+1}, \ldots , s_{I}).
\]

We see that player \(i\) has to choose a strategy \(\sigma_{i}\colon T_{i}\to \mathcal{S}_{i}\) such that
\[
	\forall t_{i}\in T_{i}, \text{ choose }\sigma_{i}(t_{i})\in c_{i}(t_{i})\subseteq \mathcal{S}_{i}.
\]
\begin{intuition}
	We are simply picking functions as strategies.
\end{intuition}
\begin{eg}
	Let \(T_{i} = \mathbb{\MakeUppercase{R}}_+ = [0, +\infty )\) be non-negative real number. Let
	\[
		\sigma_{i}(t_{i}) = \min(100, t_{i}),
	\]
	and \(\mathcal{S}_{i} = \mathbb{\MakeUppercase{R}}_+\).

	\begin{figure}[H]
		\centering
		\incfig{action-function-eg}
		\label{fig:action-function-eg}
	\end{figure}
\end{eg}

\hr

Then we can have so-called \textbf{Independent types model}. Let \(\tau_{i}\), \(i = 1, \ldots , I\) be the set of mutually independent
variables, then
\[
	\probability{}{\tau_{1} = t_1, \ldots , \tau_{I} = t_{I}}=\prod\limits_{i = 1}^{I} \pi_{i}(t_{i}).
\]

\hr

We see that we can express all games we have previously seen like follows. Let \(T_{i} = \left\{i\right\}\), and
\(c_{i}(i) = \mathcal{S}_{i}\) with type being \(\pi_i(i) = 1\) such that
\[
	u_{i}(1, s_1, 2, s_2, \ldots , I, s_I) = u_{i}(s_1, s_2, \ldots , s_I).
\]

\hr

Now we define the notions of incomplete information.
\begin{itemize}
	\item Ex-Ante: No player has information about anyone's types but only the distribution information where strategy is chosen.
	      Since we don't know the exact utility, hence we simply take the expected value for the payoff based on the strategy functions:
	      \[
		      u_{i}(\sigma_1, \ldots , \sigma_I)=\expectation{\tau_1, \ldots , \tau_I}{u_{i}(\tau_1, \sigma_1(\tau_1), \ldots ,\tau_I, \sigma_I(\tau_I) )}.
	      \]
	      Then the Nash Equilibrium \((\sigma_1^{*}, \ldots , \sigma_{I}^{*})\) such that
	      \[
		      \underset{i\in\mathcal{I}}{\forall} \ \underset{\sigma_{i}}{\forall}\ u_{i}(\sigma_{i}^{*}, \sigma^{*}_{-i})\geq u_{i}(\sigma_{i}, \sigma^{*}_{-i}).
	      \]
	      This type of information mechanism can be summarized as follows.
	      \begin{enumerate}
		      \item Given player \(i, \ldots , I\).
		      \item \(\sigma_1, \ldots , \sigma_{I}\) are chosen: Each player picks strategy \(\sigma_{i}\).
		      \item \(\tau_1, \ldots \tau_{I}\) are realized: Nation chooses types for each player.
		      \item \(\underset{i}{\forall }\ s_{i} = \sigma_{i}(\tau_{i})\) is chosen: Actions produced, and the game is realized.
		      \item \(u_{i}(\tau_1, s_1, \ldots , \tau_{I}, s_{I})\), namely the payoff is received.
	      \end{enumerate}
	\item Interim: Players get to know their types since Nation play first, and they can reason based on that. So we see that player \(i\) knows \(\tau_{i}\)
	      but not \(\tau_{-i}\). Then \(u_{i}(t_{i}, \sigma_{i}(t_{i}), \sigma_{-i})\) can be described as
	      \[
		      \begin{split}
			      \expectation{\tau_{-i}}{u_{i}(\underline{t_{i}, \sigma_{i}(t_{i})}, \tau_1, \sigma_1(\tau_1), \ldots ,\right.&\underline{\tau_{i-1}, \sigma_{i-1}(\tau_{i-1}),}\\
				      &\left.\underline{\tau_{i+1}, \sigma_{i+1}(\tau_{i+1})}, \ldots ,\tau_I, \sigma_I(\tau_I) )}.
		      \end{split}
	      \]
	      This type of information mechanism can be summarized as follows.
	      \begin{enumerate}
		      \item Given player \(i, \ldots , I\).
		      \item \(\tau_1, \ldots \tau_{I}\) are realized: Nation chooses types for each player.
		      \item \(\sigma_1, \ldots , \sigma_{I}\) are chosen: Each player picks strategy \(\sigma_{i}\).
		      \item \(\underset{i}{\forall }\ s_{i} = \sigma_{i}(\tau_{i})\) is chosen: Actions produced, and the game is realized.
		      \item \(u_{i}(\tau_1, s_1, \ldots , \tau_{I}, s_{I})\), namely the payoff is received.
	      \end{enumerate}
	\item Ex-Post: Every player gets to know everyone's type.
	      \[
		      u_{i}(t_1, \sigma_1(t_1), \ldots , t_{I}, \sigma_{I}(t_{I})).
	      \]
\end{itemize}
We will focus on \textbf{Interim} setting. We first see the example of auctions, which model this well.

Think about the best response. Suppose \(\sigma_{-i}\) is fixed. Then for all \(s_{i}\in \mathcal{S}_{i} \), \(u_{i}(t_{i}, s_{i}, \sigma_{-i})\) is defined.
\[
	\arg\max_{s_{i}\in\mathcal{S}_{i}} u_{i}(t_{i}, s_{i}, \sigma_{-i})
\]
for every \(t_{i}\), \(\mathrm{BR}_{i}(\sigma_{-i}) = \overline{\sigma}_{i}(\sigma_{-i})\). Then we call that
\[
	(\sigma^{*}_1, \ldots , \sigma^{*}_I)
\]
is a (Bayes interim) Nash Equilibrium if
\[
	\underset{i\in \mathcal{I}}{\forall}\ \sigma_{i}^{*}\in \overline{\sigma}_{i}(\sigma_{-i}).
\]

We now use \textbf{auctions} to see how it applies.

\chapter{Auctions}
We now introduce a specific kind of game, auction. Rather than the simple set up we have seen, the auction is more complex and interesting. We'll see how can we
extend what we have discussed in Game Theory into this set up. Firstly, we see some examples.

\begin{eg}
	There are mainly four types of auction:
	\begin{enumerate}
		\item[1.] First-price auction(Sealed bid): Each player bids an amount for the item (how much the player is willing to pay), and the sellers selects
			the highest bidder and charges him/her the price that was bid by him/her and give the item. Others pay nothing, receive nothing.
		\item[2.] Dutch auction(Descending price auction): Seller announces a price(high price). If no players raise hands, seller drops price until
			the first time some player ha hand up. Item sold to the player at that price.
			\par Ties breaking:
			\begin{itemize}
				\item Random order.
				\item Choose some fixed order(specified earlier).
			\end{itemize}
	\end{enumerate}
	\begin{remark}
		The first two auctions basically implement the same result.
	\end{remark}
\end{eg}