\lecture{17}{3 Nov. 12:30}{Random Graph}
\section{Real-World Graphs}
In this section, we compare the random graph models with the real-world graphs. In practice, a real-world graph will have the following
properties:
\begin{enumerate}
	\item The number of edges is linear in \(V\).
	\item Giant component usually entice graph(connectivity)
	\item Lots of triangles
\end{enumerate}

Compare to the real-world graphs, the Erdős-Rényi random graphs family usually have the following properties, for each different settings:
\begin{enumerate}
	\item Very sparse
	      \begin{itemize}
		      \item edges is linear in \(V\)
		      \item giant component
		      \item isolated nodes
		      \item no triangles as \(V\to \infty \)
	      \end{itemize}
	\item Sparse
	      \begin{itemize}
		      \item edges is \(\sim \Theta(V\log V)\)
		      \item most nodes in the giant component
		      \item can have connectivity
		      \item Many triangles
	      \end{itemize}
	\item Dense
	      \begin{itemize}
		      \item edges is \(\sim \Omega(V^{1+\epsilon})\) with \(\epsilon>0\)
		      \item fully connected
		      \item lots of triangles
	      \end{itemize}
\end{enumerate}

\section{R-MAT Graphs}
Besides Erdős-Rényi random graph family, we can use different approaches to generate a random graph. We now introduce so-called \emph{R-MAT Graphs}.
\subsection{Recursion Adjacency Matrix Method}
We assume that the number of nodes is
\[
	V = 2^k
\]
for some \(k\). Now, given four numbers \(a, b, c, d \geq 0\) with \(a+b+c+d = 1\)
\[
	\begin{bmatrix}
		a & b \\
		c & d \\
	\end{bmatrix},
\]
which is a distribution on \(\{1, 2, 3, 4\}\). Further, we denote
\begin{itemize}
	\item \(E\): Target number of edges to be placed on a directed graph. Furthermore, we add entries in the adjacency matrix one edge of a time.
	\item \(A\): \(2k \times 2k\) matrix.
\end{itemize}
\[
	\substack{2^{k - 1}\\ \\ \\ \\ 2^{k-1}} \overset{\ \ \ 2^{k - 1}\ \ \ 2^{k-1}}{\begin{bmatrix}
			 & 1 &  & 2 & \\
			 &   &  &   & \\
			 & 3 &  & 4 & \\
		\end{bmatrix}}_{2^k\times 2^k}
\]

Then the algorithm to generate a R-MAT graph just works like the following.
\begin{enumerate}
	\item[1.] For \(e = 1 : E\)
		\begin{enumerate}
			\item \(l = k: -1 : 2\)
			      \begin{itemize}
				      \item Choose a region of \(A_{2^l \times 2^l}\) with size \(2^{l-1} \times  2^{l-1}\):
				            \[
					            \substack{2^{l - 1}\\ \\ \\ \\ 2^{l-1}} \overset{\ \ \ 2^{l - 1}\ \ \ 2^{l-1}}{\begin{bmatrix}
							             & 1 &  & 2 & \\
							             &   &  &   & \\
							             & 3 &  & 4 & \\
						            \end{bmatrix}}_{2^l\times 2^l}
				            \]
				            with the probability \(1\sim a\), \(2\sim b\), \(3\sim c\), \(4\sim c\).
				      \item Determine the region for the next step. If \(i\), then we simply pick the region \(i\) and \textbf{GOTO (a).}
			      \end{itemize}
			\item Place edge in the chosen region if entry is \(0\), else we ignore. \textbf{GOTO 1.}
		\end{enumerate}
	\item[2.] Clean of. Delete any entries on the diagonal and produce the adjacency matrix \(A\) to make it a simple directed graph.
\end{enumerate}

\begin{remark}
	\begin{itemize}
		\item We implement \(1\sim a, 2\sim b, 3\sim c, 4\sim d\) by letting \(r = \texttt{rand(1)}\)(uniform in \([0, 1]\)) such that if
		      \begin{itemize}
			      \item \(r\leq a\), then \(1\)
			      \item \(a< r\leq a+b\), then \(2\)
			      \item \(a+b<r\leq a+b+c\), then \(3\)
			      \item \(a+b+c < r\leq a+b+c+d = 1\), then \(4\)
		      \end{itemize}
		\item \begin{eg}
			      \[
				      M = \left(
				      \begin{array}{c|c}
						      \begin{array}{cc}
							       & \\
							       &
						      \end{array} & \begin{array}{c|c}
							                    \begin{array}{c|c}
								            & \\
								      \hline
								      \cdot &
							      \end{array} &    \\
							                    \hline
							                                       &
						                    \end{array} \\
						      \hline
						      \begin{array}{cc}
							       & \\
							       &
						      \end{array} & \begin{array}{cc}
							                     & \\
							                     &
						                    \end{array}
					      \end{array}
				      \right),
			      \]
			      where we choose \(k = 3\) with the sequence
			      \[
				      2, 1, 3.
			      \]
		      \end{eg}
		\item Clean-up stage. In this step, we delete any entries on the diagonal and produce the adjacency matrix \(A\) for a simple directed graph.
		      \begin{eg}
			      Let \(A\) be
			      \[
				      A = \begin{pmatrix}
					      0 & 0 & 0 & 1 \\
					      1 & 0 & 0 & 0 \\
					      0 & 0 & 1 & 0 \\
					      0 & 1 & 0 & 1 \\
				      \end{pmatrix}.
			      \]
			      After cleaning up, we have
			      \[
				      A^\prime = \begin{pmatrix}
					      0 & 0 & 0 & 1 \\
					      1 & 0 & 0 & 0 \\
					      0 & 0 & 0 & 0 \\
					      0 & 1 & 0 & 0 \\
				      \end{pmatrix}.
			      \]
		      \end{eg}
	\end{itemize}
\end{remark}

\section{Preferential Attachment Graph(Directed)}
Yet, we have another algorithm to produce a random graph. Let the (undirected) degree distribution on \(V\) being
\[
	d_1, d_2, \ldots , d_V
\]
with \(\sum\limits_{i=1}^{V} d_i\) being even. Then by considering a histogram, for \(k = \{0, 1, \ldots , V - 1\}\), we have
\[
	P_k^{\alpha} = \frac{1}{V}\sum\limits_{i=1}^{V} \mathbbm{1}_{\{d_i = k\}} = \frac{\#\text{ of nodes with degree }k}{V}.
\]

\begin{remark}
	We can do some comparison between what we have already seen.
	\begin{enumerate}
		\item Erdős-Rényi Random Graph: Poisson degree distribution with \(p = \frac{c}{V}\). Then the tail probability(CCDF) is
		      \[
			      \probability{}{\deg\geq k}\cong c e^{-\alpha k},
		      \]
		      which decreases exponentially.
		\item Real-World graphs. Tail probability is
		      \[
			      \probability{}{\deg\geq k} = \frac{c}{k^{\alpha - 1}},
		      \]
		      which is polynomial decrease if \(\alpha\in(2, 3)\). More specifically,
		      \begin{itemize}
			      \item \(\alpha>2\):
			            \begin{itemize}
				            \item mean degree exists, such that
				                  \[
					                  \text{mean degree }\cong \sum\limits_{k=1}^{\infty} k \frac{\widetilde{c}}{k^{\alpha}} = \sum\limits_{k=1}^{\infty} \frac{\widetilde{c}}{k^{\alpha - 1}}.
				                  \]
				                  We see that \(\alpha - 1 > 1\) is needed for above to converge to a finite number.
				            \item mean squared of degree exists, such that
				                  \[
					                  \text{mean squared of degree }\cong \sum\limits_{k=1}^{\infty} k^2 \frac{\widetilde{c}}{k^{\alpha}} = \sum\limits_{k=1}^{\infty} \frac{\widetilde{c}}{k^{\alpha-2}}.
				                  \]
			            \end{itemize}
			      \item \(\alpha<3 \implies \alpha - 2 < 1\): Above will just diverge to \(+\infty\).
		      \end{itemize}
		\item In practice, the variance of degrees is exponentially high.
	\end{enumerate}
\end{remark}

\hr

We make the following assumptions for constructing a preferential attachment graph.
\begin{itemize}
	\item Nodes arrive in sequence \(1, 2, \ldots , V\)
	\item Each node has an out-degree. Out going edges that need to be paired with nodes earlier in sequence such that
	      \[
		      d_{i}^{out} \leq i - 1.
	      \]
\end{itemize}

Now, with \(p\in(0, 1)\), we do the following to generate a preferential attachment graph.
\begin{enumerate}
	\item[1.] for each edge in \(d_{i}^{out}\)(\(d_{i}^{out}\) time)
		\begin{itemize}
			\item Toss a biased coin with probability \(p\) for Heads(\(q\coloneqq 1 - p\) for Tails).
			      \begin{itemize}
				      \item If Heads, then choose one of unpaired nodes at random and pair.
				      \item If Tails, then choose one of the unpaired nodes in \textbf{proportion} to their incoming edges.
			      \end{itemize}
		\end{itemize}
	\item[3.] \textbf{GOTO 1.}
\end{enumerate}

\begin{remark}
	Node \(1 \) always has out-degree \(0\).
\end{remark}

Now, lets analysis the situation such that if \(d_{i}^{out} = 1\) for all \(i = 2, 3, \ldots \) with \(t = 2, 3, \ldots , V\). Further, we define
\[
	X_i(t)\coloneqq \#\text{ of incoming edges for node \(i\) after node \(t\) is added to the graph}.
\]
We then see
\[
	\sum\limits_{i=1}^{V} X_i(t) = t - 1 = \sum\limits_{i=1}^{t - 1} X_{i}(t)
\]
since \(X_i(t) = 0\) for all \(i\) such that \(t\leq i\leq V\). Look at the change of \(X_{i}(t)\) with respect to \(X_i(t - 1)\), we further have
\[
	X_{j}(t) = \begin{dcases}
		X_{j}(t - 1) = 0, & \text{ if }t\leq j\leq V                                                 \\
		X_{j}(t - 1),     & \text{ w.p. } p \frac{t - 2}{t - 1}+q \frac{t - 1 - X_{j}(t - 1)}{t - 1} \\
		X_{j}(t - 1)+1    & \text{ w.p. } p \frac{1}{t - 1} + q \frac{X_{j}(t - 1)}{t - 1}.          \\
	\end{dcases}
\]

\begin{remark}
	This is a vector valued random process.
\end{remark}

In order to simply the analysis, we take a look at the expectation. We have
\[
	x_{j}(t) = \expectation{X_{j}(t)} \\
	x_{j}(t)=\begin{dcases}
		x_{j}(t - 1)= 0, \text{ if }t\leq j\leq V \\
		x_{j}(t - 1) + p \frac{1}{t - 1} + q \frac{x_{j}(t - 1)}{t - 1}, \text{ otherwise}.
	\end{dcases}
\]
Then
\[
	\underbrace{x_{j}(t) - x_{j}(t - 1)}_{\text{like a derivative}} = \begin{dcases}
		0,                                              & \text{ if }t\leq j\leq V \\
		\frac{p}{t - 1} + q \frac{x_{j}(t - 1)}{t - 1}, & \text{ otherwise}.
	\end{dcases}
\]

Let \(V\to \infty \) and \(\widetilde{x}_i(t) = \frac{x_{i}(t)}{V}\cong \frac{X_{i}(t)}{V}\) will be well-approximated by a differential equation
\[
	\frac{\mathrm{d}\widetilde{x}_j(t)}{\mathrm{d}t} = \frac{p}{t} + q \frac{\widetilde{x}_j(t)}{t} = \frac{p + q \widetilde{x}_j(t)}{t}.
\]
Rearranging, we have
\[
	\frac{1}{p+q \widetilde{x}_j(t)}\frac{\mathrm{d}\widetilde{x}_j(t)}{\mathrm{d}t} = \frac{1}{t}.
\]
Then since
\[
	\frac{\mathrm{d}}{\mathrm{d}t}\left(\ln(p + q \widetilde{x}_j(t))\right) = \frac{1}{p + q \widetilde{x}_j(t)}\cdot q \frac{\mathrm{d}\widetilde{x}_j(t)}{\mathrm{d}t},
\]
hence we have
\[
	\frac{\mathrm{d}}{\mathrm{d}t}\left(\ln(p + q \widetilde{x}_j(t))\right) = \frac{q}{t},
\]

In all, we have
\[
	\ln(p + q \widetilde{x}_j(t)) = q \ln(t) + qc.
\]
Take exponential on both sides,
\[
	p + q \widetilde{x}_j(t) = t^q e^{cq}.
\]
Solving for \(\widetilde{x}_j\), we see that
\[
	\widetilde{x}_j(t) = \frac{t^q e^{cq} - p}{q}.
\]

We now need the initial value. In this case, we have
\[
	x_{j}(t) = 0, \forall\ t\leq j\leq V\implies \widetilde{x}_j(j) = 0.
\]
This further implies
\[
	\widetilde{x}_j(j) = 0 \iff j^q e^{cq} = p \implies e^{cq} = \frac{p}{j^q} \implies \widetilde{x}_j(t) = \frac{p}{q}\left(\left(\frac{t}{j}\right)^q - 1\right)\forall\ t\geq j.
\]

From this, we see that the fraction of nodes with degree greater than \(k\) is
\[
	\frac{1}{\left(1 + \frac{q}{p}\frac{k}{V}\right)^{\frac{1}{q}}} \cong \frac{c}{k^{\alpha}}
\]
where \(\alpha = \frac{1}{q} = \frac{1}{1 - p}\).

\begin{note}
	This is what we called \textbf{power law tail}.
\end{note}

\begin{remark}
	This process is biasing towards incoming degrees, which is caused from the second step.
\end{remark}

\begin{remark}
	A more formal setup is the following. Let
	\[
		\left[X_j(t)\right]_{j = 1, \ldots , V}
	\]
	be a Markov Chain. Considering two aggregations:
	\begin{enumerate}
		\item For \(i = 0, 1, \ldots \), we determine the fraction of nodes with in-degree at least \(i\) at time \(t\). Then
		      \[
			      y_{i}(t) = \frac{1}{V}\sum\limits_{k=1}^{V} \sum\limits_{j=1}^{V} \mathbbm{1}_{\{ X_{j}(t) = k \}} = \frac{1}{V}\sum\limits_{j=1}^{V} \mathbbm{1}_{\{ X_{j}(t)\geq i \}}.
		      \]
		\item The for \(s\in[0, 1]\), define \(t = \left\lfloor sV\right\rfloor \). Now, for \(i = 0, 1, \ldots \), set \(\widetilde{y}^V_t(s) = y_{i}(\left\lfloor sV\right\rfloor )\).
		      Then the formal comparison of \(\widetilde{y}^V_i(s)\) can be made with an appropriate O.D.E. solution
		      \[
			      \left\{y_{i}(s)\right\}_{i = 0, 1, \ldots }
		      \]
		      for \(s\in [0, 1]\). This is the formal mean-field analysis that is not in the scope of the course. Hence, we give a flavour of the
		      results using a hand-wavy argument.
	\end{enumerate}
\end{remark}
