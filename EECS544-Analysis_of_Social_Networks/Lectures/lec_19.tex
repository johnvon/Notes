\lecture{19}{10 Nov. 12:30}{Nash Equilibrium}
\begin{prev}
	We assume that all players have
	\begin{itemize}
		\item Common knowledge
		\item Rationality
	\end{itemize}
\end{prev}

\begin{note}
	We also have something called \emph{bounded rationality}.
	\begin{itemize}
		\item Some finite levels
		\item Not fully rational
	\end{itemize}
\end{note}

\begin{prev}
	Nash equilibrium.
	\begin{itemize}
		\item \(\mathcal{I} \): the set of players.
		\item \(I\): \(\left\vert \mathcal{I}  \right\vert \), the number of players.
		\item \(\mathcal{S}_{i}\): player \(i\) choose strategy/action from \(\mathcal{S}_{i}\) such that
		      \[
			      s_{i}\in \mathcal{S}_i
		      \]
		      with
		      \[
			      S_{i} = \left\vert \mathcal{S}_i \right\vert.
		      \]
		\item \(s\): \((s_1, s_2, \ldots , s_I)\), strategy chosen by all players
		\item Player \(i\in \mathcal{I}\), then \(-i\coloneqq \mathcal{I}\setminus\{i\}\): Everyone else but \(i\).
		      \begin{eg}
			      \[
				      \begin{split}
					      i = 1\colon\quad& -1 = \{2, 3, \ldots , I\},\\
					      i = 2\colon\quad& -2 = \{1, 3, \ldots , I\},\\
					      &\vdots
				      \end{split}
			      \]
			      since
			      \[
				      s_{-i} = (s_1, \ldots , s_{i - 1}, s_{i + 1}, \ldots , s_{I}).
			      \]
		      \end{eg}
		\item Best-response correspondence(set): For player \(i\), give the strategies of \(-i\) (\(s_{-i} \) is given). Find the
		      subset of strategies in \(\mathcal{S}_i \) that give player \(i\) the highest payoff/utility.
		      \begin{intuition}
			      \(\mathrm{BR}_i(s_{-i})\) can be though as follows: For player \(i\), given every other players' strategies they will play, find the best response. And since there may be
			      several strategies can achieve the optimal result, so it's should be a set.
		      \end{intuition}
		\item Utilities/Payoffs: \(\forall i\in \mathcal{I} \),
		      \[
			      u_{i}\colon \prod\limits_{j = 1}^{I} \mathcal{S}_{j}\to \mathbb{\MakeUppercase{R}}.
		      \]
		      For every vector \(s = (s_1, s_2, \ldots , s_I)\in \prod\limits_{j\in \mathcal{I}} \mathcal{S}_j\), \(u_{i}(s)\)
		      is a real number.
	\end{itemize}
	\begin{remark}
		We see that \(s_{i}, s_{-i}\) form the entire vector of strategies every player plays, so \(u_{i}(s_{i}, s_{-i})\) is \textbf{well-defined}.
		Moreover,
		\[
			\max_{s_{i}\in\mathcal{S}_i}u_{i}(s_{i}, s_{-i})
		\]
		finds all strategies that attain the maximum rewards.
	\end{remark}
\end{prev}

\begin{prev}
	We now view the \textbf{Golden Balls} game with what we just defined. The payoff matrix is
	\begin{table}[H]
		\centering
		\setlength{\extrarowheight}{2pt}
		\begin{tabular}{cc|c|c|}
			                          & \multicolumn{1}{c}{} & \multicolumn{2}{c}{Player $2$}                                              \\
			                          & \multicolumn{1}{c}{} & \multicolumn{1}{c}{$\mathrm{split}$} & \multicolumn{1}{c}{$\mathrm{steal}$} \\\cline{3-4}
			\multirow{2}*{Player $1$} & $\mathrm{split}$     & $(x/2, x/2)$                         & $(0, x)$                             \\\cline{3-4}
			                          & $\mathrm{steal}$     & $(x, 0)$                             & $(0, 0)$                             \\\cline{3-4}
		\end{tabular}
	\end{table}

	For player \(1\), given
	\begin{enumerate}
		\item \(s_{-1} = \mathrm{steal}\):
		      \begin{itemize}
			      \item \(u_1(\mathrm{split} , \underbrace{\mathrm{steal}}_{s_{-1}} ) = 0\)
			      \item \(u_1(\mathrm{steal} , \underbrace{\mathrm{steal}}_{s_{-1}} ) = 0\)
		      \end{itemize}
		      So
		      \[
			      \arg\max_{s_1\in \mathcal{S}_1} u_1(s_1, \mathrm{steal} ) = \{\mathrm{split}, \mathrm{steal}  \},
		      \]
		      which is our best response correspondence.
		\item \(s_{-1} = \mathrm{split}\):
		      \begin{itemize}
			      \item \(u_1(\mathrm{split} , \underbrace{\mathrm{split}}_{s_{-1}} ) = \frac{x}{2}\)
			      \item \(u_1(\mathrm{steal} , \underbrace{\mathrm{split}}_{s_{-1}} ) = x\)
		      \end{itemize}
		      So
		      \[
			      \arg\max_{s_1\in \mathcal{S}_1} u_1(s_1, \mathrm{split} ) = \{\mathrm{steal}\},
		      \]
		      which is our best response correspondence, and moreover, since it's unique and strictly greater than any other choices, so we call it \emph{strictly best response}.
	\end{enumerate}
\end{prev}

\begin{definition}
	\(\overline{s}_{i}\) is called a \emph{strict best response} to \(s_{-i}\) if
	\[
		u_{i}(\overline{s}_i, s_{-i})>u_{i}(s_{i}, s_{-i})
	\]
	for every \(s_{i}\in \mathcal{S}_{i}\setminus\{\overline{s}_i\} \).
\end{definition}

\begin{remark}
	Obviously, a strict best response will be unique.
\end{remark}

\begin{definition}
	\(s_{i}\) is \emph{strictly dominant} if it is the strict best response for all \(s_{-i}\).
\end{definition}
\begin{definition}
	\(s_{i}\) is \emph{weakly dominant} if it is one of the best response for all \(s_{-i}\).
\end{definition}

\begin{remark}
	We see that
	\begin{itemize}
		\item In the Golden balls example, where steal is a weakly dominant strategy.
		\item Rationality plus common knowledge implies that if a player with dominant strategies, then the player will play them.
	\end{itemize}
\end{remark}

\hr

\begin{prev}
	Nash equilibrium. Let \(s^{\ast} = (s_1^{\ast}, s_2^{\ast}, \ldots , s_I^{\ast})\) be the \textbf{strategy profile}. Then we call it as a
	\emph{pure strategy} Nash equilibrium(NE) if
	\[
		s_{i}^{\ast}\in\mathrm{BR}(s_{-i}^{\ast})
	\]
	for all \(i\in \mathcal{I} \).
\end{prev}

\begin{note}
	We see that
	\begin{itemize}
		\item Every agent is playing a \(\mathrm{BR}\) to the others.
		\item Utility(Return/Payoff/Reward) is the incentive for agents. Setting everyone else at \(s^{\ast}_{-i}\), there is no incentive for player \(i\) to deviate(not necessary to change action). This implies there are no \emph{unilateral dominants} are possible.
		      \[
			      \underset{i\in \mathcal{I}}{\forall}\ \underset{s_{i}\in \mathcal{S}_i}{\forall}\ u_{i}(s^{\ast}_i, s^{\ast}_{-i}) \geq u_{i}(s_{i}, s^{\ast}_{-i}).
		      \]
	\end{itemize}
	\begin{prev}
		There are two different types of strategies:
		\begin{itemize}
			\item Pure strategy: Fixing an action for every player.
			\item Mixed strategy: A randomized action is played by every player.
		\end{itemize}
	\end{prev}
\end{note}

\begin{remark}
	We see that
	\begin{enumerate}
		\item Pure strategy Nash Equilibrium needs not exist for every game.
		\item Even if they exist, they need not be unique.
		\item If there are multiple equilibrium, then which one gets played is a tough question and usually involves external quantities(outside information).
	\end{enumerate}
\end{remark}

\subsection{Pure Strategy Nash Equilibrium}
We first introduce the concept of \emph{Coordination Games}. Coordination games are games such that players get a higher payoff by working together(taking the same action).

\begin{eg}
	We first see some coordination game examples.
	\begin{enumerate}
		\item \textbf{Shaking hands}. There are two men want to shake their hands. They can either position their hand up(\(U\)) or down(\(D\)).
		      \begin{table}[H]
			      \centering
			      \setlength{\extrarowheight}{2pt}
			      \begin{tabular}{cc|c|c|}
				                                & \multicolumn{1}{c}{} & \multicolumn{2}{c}{player $2$}                           \\
				                                & \multicolumn{1}{c}{} & \multicolumn{1}{c}{$U$}        & \multicolumn{1}{c}{$D$} \\\cline{3-4}
				      \multirow{2}*{player $1$} & $U$                  & $(1, 1)$                       & $(0, 0)$                \\\cline{3-4}
				                                & $D$                  & $(0, 0)$                       & $(1, 1)$                \\\cline{3-4}
			      \end{tabular}
		      \end{table}
		      We see that \((U, U)\) and \((D, D)\) are pure-strategy Nash Equilibrium.
		\item \textbf{Battle of the sexes}. A man and a woman are going to a date. The man prefer to see football(\(F\)) while the woman prefer to go to theater(\(T\)). If
		      their opinion are not equal, then they can't go to anywhere, so their fulfillment(payoff) will both be \(0\).
		      \begin{table}[H]
			      \centering
			      \setlength{\extrarowheight}{2pt}
			      \begin{tabular}{cc|c|c|}
				                         & \multicolumn{1}{c}{} & \multicolumn{2}{c}{woman}                           \\
				                         & \multicolumn{1}{c}{} & \multicolumn{1}{c}{$T$}   & \multicolumn{1}{c}{$F$} \\\cline{3-4}
				      \multirow{2}*{man} & $T$                  & $(1, 2)$                  & $(0, 0)$                \\\cline{3-4}
				                         & $F$                  & $(0, 0)$                  & $(2, 1)$                \\\cline{3-4}
			      \end{tabular}
		      \end{table}
		      We see that \((T, T)\) and \((F, F)\) are pure-strategy Nash Equilibrium.
		\item \textbf{Stay Hunt}.
		      \begin{table}[H]
			      \centering
			      \setlength{\extrarowheight}{2pt}
			      \begin{tabular}{cc|c|c|}
				                                & \multicolumn{1}{c}{} & \multicolumn{2}{c}{player $1$}                             \\
				                                & \multicolumn{1}{c}{} & \multicolumn{1}{c}{$S$}        & \multicolumn{1}{c}{$H/R$} \\\cline{3-4}
				      \multirow{2}*{player $2$} & $S$                  & $(4, 4)$                       & $(0, 3)$                  \\\cline{3-4}
				                                & $H/R$                & $(3, 0)$                       & $(3, 3)$                  \\\cline{3-4}
			      \end{tabular}
		      \end{table}
		      We see that \((S, S)\) and \((H, H)\) are pure-strategy Nash Equilibrium.
	\end{enumerate}
\end{eg}

\hr

We also have so-called \emph{Anti-Coordination Games}. Such games mean that one player prefers to coordinate, and the other does not.

\begin{eg}
	Again, we first see some examples.
	\begin{enumerate}
		\item \textbf{Rock, paper, scissors}. This is an example that a game without pure Nash Equilibrium. The payoff matrix is
		      \begin{table}[H]
			      \centering
			      \setlength{\extrarowheight}{2pt}
			      \begin{tabular}{cc|c|c|c|}
				                                & \multicolumn{1}{c}{} & \multicolumn{3}{c}{player $2$}                                                                                       \\
				                                & \multicolumn{1}{c}{} & \multicolumn{1}{c}{$\mathrm{Rock}$} & \multicolumn{1}{c}{$\mathrm{Paper}$} & \multicolumn{1}{c}{$\mathrm{Scissors}$} \\\cline{3-5}
				      \multirow{3}*{player $1$} & $\mathrm{Rock}$      & $(0, 0)$                            & $(-1, \underline{+1})$               & $(\underline{+1}, -1)$                  \\\cline{3-5}
				                                & $\mathrm{Paper}$     & $(\underline{+1}, -1)$              & $(0, 0)$                             & $(-1, \underline{+1})$                  \\\cline{3-5}
				                                & $\mathrm{Scissors}$  & $(-1, \underline{+1})$              & $(\underline{+1}, -1)$               & $(0, 0)$                                \\\cline{3-5}
			      \end{tabular}
		      \end{table}
		      And it's clear that no matter what we fixed a strategy for one player first, another player will have incentive to deviate right after another player changes its strategy.
		\item \textbf{Hawk-Dove} game. Two neighboring countries are likely going to fight. If one choose Hawk, then that country will fight; otherwise if choosing Dove, then that country will not
		      going to fight and will not going to fight back either.
		      \begin{table}[H]
			      \centering
			      \setlength{\extrarowheight}{2pt}
			      \begin{tabular}{cc|c|c|}
				                                 & \multicolumn{1}{c}{} & \multicolumn{2}{c}{Country $2$}                           \\
				                                 & \multicolumn{1}{c}{} & \multicolumn{1}{c}{$H$}         & \multicolumn{1}{c}{$D$} \\\cline{3-4}
				      \multirow{2}*{Country $1$} & $H$                  & $(0, 0)$                        & $(5, 1)$                \\\cline{3-4}
				                                 & $D$                  & $(1, 5)$                        & $(3, 3)$                \\\cline{3-4}
			      \end{tabular}
		      \end{table}
		      \begin{problem}
		      Is \((D, D)\) a Nash Equilibrium?
		      \end{problem}
		      \begin{answer}
			      False. Since if player \(2\) is playing \(D\). We see that \(u_1(H, D) = 5\), \(u_1(D, D) = 3\). Then player \(1\) has a deviation to get a higher payoff.
		      \end{answer}
		      \begin{problem}
		      Is \((H, D)\) a Nash Equilibrium?
		      \end{problem}
		      \begin{answer}
			      True. Since
			      \begin{itemize}
				      \item Player \(1\): \(H\) is the best response
				      \item Player \(2\): Player \(1\) is \(H\). Then \(u_2(H, H) = 0\), and \(u_2(H, D) = 1\).
			      \end{itemize}
			      We see that neither player \(1\) nor player \(2\) has an incentive to deviate. So \((D, H)\)(also \((H, D)\)) is a Nash
			      Equilibrium.
		      \end{answer}
		\item \textbf{Matching Pennies}\label{matching-pennies}.
		      This is a simple example for \emph{attack-defense} game. Suppose there are two people each hold a penny and simultaneously choose whether to show heads(\(H\)) or tails(\(T\)) on
		      their penny. Player \(1\) loses his penny to Player \(2\) if they match; Player \(1\) wins Player \(2\)'s penny if they don't match.
		      \begin{table}[H]
			      \centering
			      \setlength{\extrarowheight}{2pt}
			      \begin{tabular}{cc|c|c|}
				                                & \multicolumn{1}{c}{} & \multicolumn{2}{c}{Player $2$}                           \\
				                                & \multicolumn{1}{c}{} & \multicolumn{1}{c}{$H$}        & \multicolumn{1}{c}{$T$} \\\cline{3-4}
				      \multirow{2}*{Player $1$} & $H$                  & $(-1, 1)$                      & $(1, -1)$               \\\cline{3-4}
				                                & $T$                  & $(1, -1)$                      & $(-1, 1)$               \\\cline{3-4}
			      \end{tabular}
		      \end{table}
		      We see that there are no pure strategy Nash Equilibrium. So we have no idea what strategy each player will play.
		      \begin{note}
			      An \emph{attack-defense} game can be simply described as follows. There are two players, one is attacker and another is defender. Attacker has two actions,
			      they are attack with strategy \(A\) or \(B\), respectively. Correspondingly, defender also has two actions to choose from, namely defend against \(A\)
			      or defend against \(B\). Clearly, if defender correctly defend attacker's attack in terms of strategy attack chooses, then defender gets a higher payoff,
			      otherwise the attacker gets a higher payoff.
		      \end{note}
		      \begin{remark}
			      This is so-called a \emph{Zero-Sum Game}. One player's gain is exactly another player's loss.
		      \end{remark}
	\end{enumerate}
\end{eg}

\begin{remark}
	From \hyperref[matching-pennies]{matching pennies} games example, we see that there are no Nash Equilibrium. This is due to the fact that all player will play
	their strategy in a deterministic way, if the best response exists. And if not, then we have no idea what they will play in this case, which is hard to analyze.
	Now if we allow some randomness, then by John Nash, he states that with such set up, the Nash Equilibrium always exists.
\end{remark}

\subsection{Mixed Strategy Nash Equilibrium}
Follows the intuition of \hyperref[matching-pennies]{matching pennies} game, we now develop the concept so-called \emph{mixed strategy}. Essentially, we are just
randomizing one's strategies such that the randomization is done independently. This will bring the game structure with much more interesting properties.

\begin{prev}
	The reason why we call it \emph{mixed strategies} is initially, we only have distinct strategy to choose from, but now, with probability, we have some kind of
	\textbf{mixed} strategy between choices. In \hyperref[matching-pennies]{matching pennies} game example, we have strategies which mixes between \(H\) and \(T\).
\end{prev}

\hr

For player \(i\), denote \(\overrightarrow{P}_{i}\) as a probability distribution on \(\mathcal{S}_i\)(PMF) such that
\[
	\left\{s_{i1}, s_{i2}, s_{i3}, s_{i4}\right\} = \mathcal{S}_i,
\]
and \(S_i = 4\). More specifically, \(\overrightarrow{P}_i\) looks like
\[
	(P_{i1}, P_{i2}, P_{i3}, P_{i4})
\]
with all entries non-negative and their sum is \(1\). There are lots of example for such \(P_{i}\). For example,
\begin{itemize}
	\item \((\frac{1}{4}, \frac{1}{4}, \frac{1}{4}, \frac{1}{4})\): uniformly mixed.
	\item \((1, 0, 0, 0)\): pure strategy.
\end{itemize}

We can then define \emph{probability simplex} \(\Delta(\mathcal{S}_i )\), which is equal to
\[
	\left\{\left(P_{i1}, P_{i2}, \ldots , P_{i S_{i}}\right)\colon P_{ij}\geq 0\land \sum\limits_{j=1}^{S_i} P_{ij} = 1\quad \forall j = 1, \ldots , S_{i}\right\}.
\]
\begin{itemize}
	\item \(2\) actions.
	      \begin{figure}[H]
		      \centering
		      \incfig{probability-simplex-2actions}
		      \caption{Probability Simplex with 2 Actions.}
		      \label{fig:probability-simplex-2actions}
	      \end{figure}
	\item \(3\) actions.
	      \begin{figure}[H]
		      \centering
		      \incfig{probability-simplex-3actions}
		      \caption{Probability Simplex with 3 Actions.}
		      \label{fig:probability-simplex-3actions}
	      \end{figure}
\end{itemize}
\begin{intuition}
	This is just a set of probability distributions over \(\mathcal{S}_i\), but with a good geometry representation.
\end{intuition}

\hr

Players' actions are enhanced to picking probability distribution(adding uncertainty to your choice).

\begin{problem}
How to evaluate payoff?
\end{problem}
\begin{answer}
	Let
	\[
		u_{i}(\overrightarrow{P}_i, \overrightarrow{P}_{-i}) = \expectation{\overrightarrow{P}_{1}\times \overrightarrow{P}_2 \times \ldots \times \overrightarrow{P}_I}{u_{i}\left(\overline{S}_1, \overline{S}_2, \ldots , \overline{S}_I\right)},
	\]
	such that \(\overline{S}_i\) is a random variable that takes values in \(\mathcal{S}_i\) with distribution \(\overrightarrow{P}_i\), and they are mutually independent. This is further equal to
	\[
		\sum\limits_{j_1\in \mathcal{S}_1 }\sum\limits_{j_2\in \mathcal{S}_2}\ldots \sum\limits_{j_I\in \mathcal{S}_I }P_{1j_1}\times P_{2j_2}\times \ldots \times P_{Ij_I}\times u_{i}\left(\overline{S}_{1j_1}, \overline{S}_{2j_2}, \ldots , \overline{S}_{Ij_I}\right).
	\]
\end{answer}
\begin{eg}
	Recall the last game we consider, namely the \hyperref[matching-pennies]{\textbf{Matching Pennies} game}.
	\begin{table}[H]
		\centering
		\setlength{\extrarowheight}{2pt}
		\begin{tabular}{cc|c|c|}
			                          & \multicolumn{1}{c}{} & \multicolumn{2}{c}{Player $2$}                           \\
			                          & \multicolumn{1}{c}{} & \multicolumn{1}{c}{$H$}        & \multicolumn{1}{c}{$T$} \\\cline{3-4}
			\multirow{2}*{Player $1$} & $H$                  & $(1, -1)$                      & $(-1, 1)$               \\\cline{3-4}
			                          & $T$                  & $(-1, 1)$                      & $(1, -1)$               \\\cline{3-4}
		\end{tabular}
	\end{table}
	with \(P_1 = (\frac{3}{4}, \frac{1}{4})\) and \(P_2 = (\frac{1}{3}, \frac{2}{3})\). Then
	\[
		\begin{split}
			u_1(P_1, P_2) &= \frac{3}{4}\times \frac{1}{3} u_1(H, H) + \frac{3}{4}\times \frac{2}{3}u_1(H, T) + \frac{1}{3}\times \frac{1}{3}u_1(T, H)+\frac{1}{4}\times \frac{2}{3}u_1(T, T)\\
			&= \frac{3}{12} - \frac{6}{12} - \frac{1}{12} + \frac{2}{12}\\
			&= -\frac{1}{6}.
		\end{split}
	\]
	Also, \(u_2(P_1, P_2)\) is \(\frac{1}{6}\) from Zero-Sum setting.
\end{eg}

We can now combine the probability set up with the definition of the best response, which gives us
\[
	\mathrm{BR}_{i}(\overrightarrow{P}_{-i}) = \left\{\overrightarrow{P}_i\in \Delta(\mathcal{S}_i)\colon u_{i}(\overrightarrow{P}_i, \overrightarrow{P}_{-i}) = \max_{\hat{P}_i\in \Delta(\mathcal{S}_i)}u_{i}(\hat{P}_i, \overrightarrow{P}_{-i})\right\}.
\]
Also, for Nash Equilibrium, let \(P^{\ast}\) be the same probability distribution for all \(i\in \mathcal{I} \), then
\[
	P^{\ast}_i\in\mathrm{BR}_i(P^{\ast}_{-i})\iff \underset{P_{i}\in \Delta(\mathcal{S}_i)}{\forall }\ u_{i}(P^{\ast}_i, P^{\ast}_{-i})\geq u_{i}(P_{i}, P^{\ast}_{-i}).
\]

\begin{theorem}
	Nash's Existence Theorem. Every finite game(finite action) has at least one mixed strategy Nash Equilibrium.
\end{theorem}
\begin{remark}
	This is a theorem given by Nash by using fixed-point theorem.
\end{remark}