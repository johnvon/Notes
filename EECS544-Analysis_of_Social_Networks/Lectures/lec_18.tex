\lecture{18}{8 Nov. 12:30}{Game Theory}
\section{Game Theory}
The goal is to analyze situations with rational agents who want to maximize their won rewards under a game with a particular structure.

\subsection{Game Structure}
We restrict a game with the following structure.
\begin{enumerate}
	\item Players/Agents: \textbf{finite} or infinite
	\item Strategies/Actions: Each player has a set of strategies/actions to choose from - \textbf{finite} or infinite, homogeneous or inhomogeneous
	\item Payoff/Utility/Rewards: Return for each agent based on the action or strategy of all agents(including self).
\end{enumerate}

Besides the game structure, we also have some assumption for the player in the game.
\begin{itemize}
	\item Common knowledge: All agents know all players' strategy set and payoffs. Basically means each player knows everything about the structure of the game.
	\item Rationality: All agents are fully rational and self utility maximizer. As a result of common knowledge, everyone knows that all
	      agents are rational.
\end{itemize}

\begin{note}
	By above two assumptions, players succeed in selecting optimal strategies, and our goal is to understand these optimal strategies.
\end{note}

\subsection{Normal Form, One-Shot Games}
For simplicity, we assume that the number of players and the size of the actions set per player and also the payoffs are all finite.

\begin{definition}
	We call a game a \emph{One-Shot Game} if players will simultaneously and independently choose their actions, and they do
	so only once in this game.
\end{definition}

We note that contrarily, there are games called \emph{dynamic game}.
\begin{definition}
	We call a game a \emph{Dynamic Game} if actions can be played sequentially over time for players in the game.
\end{definition}

\begin{eg}
	We first see some examples for one-shot games describing in \emph{normal form}\footnote{\url{https://en.wikipedia.org/wiki/Normal-form_game}}.
	\begin{enumerate}
		\item \textbf{Prisoner's Dilemma}. Assume that two people are taken for investigation of a crime. They're interrogated simultaneously and in different rooms so
		      they can't communicate. Each of them is offered the choice to confess or not.
		      \begin{itemize}
			      \item Two players.
			      \item Action Set: \(\left\{\text{confess}(\mathrm{C}), \text{not confess}(\mathrm{NC})\right\}\).
			      \item Payoffs: There are four possibilities:
			            \begin{itemize}
				            \item If you confess and partner doesn't, then you are released and partner gets \(10\) years in jail.
				            \item If you don't confess and partner confess, then you'll get \(10\) years and partner released.
				            \item If both confess, both get \(4\) years.
				            \item If both do not confess, both get charges on minor crime, which will let them get \(1\) year.
			            \end{itemize}
			            We can then define so-called \emph{payoff matrix},
			            \begin{table}[H]
				            \centering
				            \setlength{\extrarowheight}{2pt}
				            \begin{tabular}{cc|c|c|}
					                                      & \multicolumn{1}{c}{} & \multicolumn{2}{c}{Player $2$}                                       \\
					                                      & \multicolumn{1}{c}{} & \multicolumn{1}{c}{$\mathrm{NC}$} & \multicolumn{1}{c}{$\mathrm{C}$} \\\cline{3-4}
					            \multirow{2}*{Player $1$} & $\mathrm{NC}$        & $(-1, -1)$                        & $(-10, 0)$                       \\\cline{3-4}
					                                      & $\mathrm{C}$         & $(0, -10)$                        & $(-4, -4)$                       \\\cline{3-4}
				            \end{tabular}
			            \end{table}
		      \end{itemize}
		      \begin{problem}
		      How will you react?
		      \end{problem}

		      \begin{answer}
			      We first see what action should suspect \(1\) plays based on rational analysis.

			      \textbf{Suspect 1's reasoning}
			      \begin{itemize}
				      \item If suspect \(2\) confess \(\to \) since \(-4>-10 \implies\) best action is to confess.
				      \item If suspect \(2\) don't confess \(\to \) since \(0>-1 \implies\) best action is to confess.
			      \end{itemize}
			      Since this problem is symmetric, namely the strategy for both suspects are the same, so both of them will confess, leading to an equilibrium.
		      \end{answer}
		      \begin{remark}
			      We see that
			      \begin{itemize}
				      \item No hidden payoffs other than what we have in payoff matrix.
				      \item Both not confessing is better as a pair(if we sum the payoffs). We see that utilize sum is not maximized, which means this action is not \emph{efficient}.
				      \item Confessing is the best option of the other suspect's choice. In this case, confessing is a \emph{dominant strategy}. Where \emph{dominant strategy} is
				            the best response to every other strategy of the other player.
			      \end{itemize}
		      \end{remark}
		\item \textbf{Golden ball} example. Two players have two options, one is split, and another is steal. And there are \(x\) amounts of money in total. If both players choose split, then they
		      can both get half of the money; if one chooses split and another chooses steal, the player chooses steal can get all the money while another player get nothing. If both choose steal, then they both get
		      nothing.
		      \begin{itemize}
			      \item Two players.
			      \item Action Set: \(\left\{\text{split}, \text{ steal}\right\}\).
			      \item Payoff: Assume the total rewards is \(x\).
			            \begin{enumerate}
				            \item (split, split): \(\frac{x}{2}\) for both.
				            \item (split, steal): player who plays steal gets \(x\), steal gets \(0\)
				            \item (steal, steal): \(0\) for both.
			            \end{enumerate}
			            The payoff matrix for Golden Ball is
			            \begin{table}[H]
				            \centering
				            \setlength{\extrarowheight}{2pt}
				            \begin{tabular}{cc|c|c|}
					                                      & \multicolumn{1}{c}{} & \multicolumn{2}{c}{Player $2$}                                              \\
					                                      & \multicolumn{1}{c}{} & \multicolumn{1}{c}{$\mathrm{split}$} & \multicolumn{1}{c}{$\mathrm{steal}$} \\\cline{3-4}
					            \multirow{2}*{Player $1$} & $\mathrm{split}$     & $(x/2, x/2)$                         & $(0, x)$                             \\\cline{3-4}
					                                      & $\mathrm{steal}$     & $(x, 0)$                             & $(0, 0)$                             \\\cline{3-4}
				            \end{tabular}
			            \end{table}
		      \end{itemize}

		      \textbf{Player 1's reasoning}
		      \begin{itemize}
			      \item If player \(2\) chooses to split \(\to \) since \(x>\frac{x}{2}\), the best action is to steal.
			      \item If player \(2\) chooses to steal \(\to \) since \(0=0\), the best action is \emph{undetermined}.
		      \end{itemize}

		      We see that stealing is a \emph{weakly dominant strategy}.

		      \begin{remark}
			      Comparison between strictly and weakly dominant strategy:
			      \begin{itemize}
				      \item Strictly dominant strategy: payoff is strictly higher in all cases.
				      \item Weakly dominant strategy: Strategy is always in the best action set. In some cases, some other strategies will give you the same payoff.
			      \end{itemize}
		      \end{remark}
		\item Consider again the prisoner's dilemma, but with different payoff matrix.
		      \begin{table}[H]
			      \centering
			      \setlength{\extrarowheight}{2pt}
			      \begin{tabular}{cc|c|c|}
				                                & \multicolumn{1}{c}{} & \multicolumn{2}{c}{Player $2$}                                       \\
				                                & \multicolumn{1}{c}{} & \multicolumn{1}{c}{$\mathrm{NC}$} & \multicolumn{1}{c}{$\mathrm{C}$} \\\cline{3-4}
				      \multirow{2}*{Player $1$} & $\mathrm{NC}$        & $(-1, -1)$                        & $(-3, -2)$                       \\\cline{3-4}
				                                & $\mathrm{C}$         & $(-2, -3)$                        & $(-4, -4)$                       \\\cline{3-4}
			      \end{tabular}
		      \end{table}
		      With the same analysis, we see that the equilibrium is not confessing, namely \((\mathrm{NC}, \mathrm{NC})\) is a dominant strategy equibrilium, furthermore, it's a
		      strictly dominant strategy.
		\item \textbf{Two firms} game. There are two firms competing with each other. They both have options to produce lower-priced products or upscale product. And we simply assume that
		      their goal is to maximize the market share. We further assume that in the whole market, there are \(60\%\) of population will only buy low-priced products, and other \(40\%\)
		      of population will only buy upscale products.
		      \begin{itemize}
			      \item Two firms.
			      \item Action Set: \(\left\{\text{low-priced}, \text{upscales}\right\}\)
			      \item Payoff:
			            \begin{table}[H]
				            \centering
				            \setlength{\extrarowheight}{2pt}
				            \begin{tabular}{cc|c|c|}
					                                    & \multicolumn{1}{c}{} & \multicolumn{2}{c}{firm $2$}                                          \\
					                                    & \multicolumn{1}{c}{} & \multicolumn{1}{c}{$\mathrm{Lp}$} & \multicolumn{1}{c}{$\mathrm{Uc}$} \\\cline{3-4}
					            \multirow{2}*{firm $1$} & $\mathrm{Lp}$        & $(0.48, 0.12)$                    & $(0.6, 0.4)$                      \\\cline{3-4}
					                                    & $\mathrm{Uc}$        & $(0.4, 0.6)$                      & $(0.32, 0.08)$                    \\\cline{3-4}
				            \end{tabular}
			            \end{table}
		      \end{itemize}
		      \textbf{Firm 1's reasoning}
		      \begin{itemize}
			      \item \(\mathrm{Lp}\) is strictly dominant strategy
		      \end{itemize}

		      \textbf{Firm 2's reasoning}
		      \begin{itemize}
			      \item If firm \(1\) choose \(\mathrm{Lp} \to \mathrm{Uc}\)
			      \item If firm \(1\) choose \(\mathrm{Uc} \to \mathrm{Lp}\)
		      \end{itemize}

		      We see that with common knowledge and rationality assumptions, firm \(2\) will assume that firm \(1\) will play its dominant strategy.
		      As a result, firm \(2\) will play \(\mathrm{Uc}\). Hence, \((\mathrm{Lp}, \mathrm{Uc})\) is the equibrilium point, and it's also efficient.
		\item Two firms game, but with three actions. The payoff matrix now becomes
		      \begin{table}[H]
			      \centering
			      \setlength{\extrarowheight}{2pt}
			      \begin{tabular}{cc|c|c|c|}
				                              & \multicolumn{1}{c}{} & \multicolumn{3}{c}{firm $2$}                                                     \\
				                              & \multicolumn{1}{c}{} & \multicolumn{1}{c}{$A$}      & \multicolumn{1}{c}{$B$} & \multicolumn{1}{c}{$C$} \\\cline{3-5}
				      \multirow{3}*{firm $1$} & $A$                  & $(4, 4)$                     & $(0, 2)$                & $(0, 2)$                \\\cline{3-5}
				                              & $B$                  & $(0, 0)$                     & $(1, 1)$                & $(0, 2)$                \\\cline{3-5}
				                              & $C$                  & $(0, 0)$                     & $(0, 2)$                & $(1, 1)$                \\\cline{3-5}
			      \end{tabular}
		      \end{table}

		      \textbf{Best Response}
		      \begin{enumerate}
			      \item For firm \(1\):
			            \begin{itemize}
				            \item If firm \(2\) play \(A\to \) pick \(A\)
				            \item If firm \(2\) play \(B\to \) pick \(B\)
				            \item If firm \(2\) play \(C\to \) pick \(C\)
			            \end{itemize}
			      \item For firm \(2\):
			            \begin{itemize}
				            \item If firm \(1\) play \(A\to \) pick \(A\)
				            \item If firm \(1\) play \(B\to \) pick \(C\)
				            \item If firm \(1\) play \(C\to \) pick \(B\)
			            \end{itemize}
		      \end{enumerate}
		      We see that there are no dominant strategy. But \((A, A)\) is the equilibrium point, which we will call it as \emph{Nash equilibrium}.
	\end{enumerate}
\end{eg}

\begin{note}
	One may notice that we are using normal form to describe one-shot games. Contrarily, we will use so-called \emph{Extensive form} to describe a
	dynamic game.
\end{note}

\subsection{Nash Equilibrium}
\begin{definition}
	Nash Equilibrium.
	\begin{itemize}
		\item A set of action when each player doesn't have an incitement to \textbf{unilaterally deviate} is called a \emph{Nash Equilibrium}.
		\item Comparison between pure or mixed strategy:
		      \begin{itemize}
			      \item Pure Strategy: Pick exactly one strategy deterministically.
			      \item Mixed Strategy: Play strategy with some probability.
		      \end{itemize}
	\end{itemize}
\end{definition}

We then define a game structure mathematically in the following way.
\begin{definition}
	Denote the set of player \(\mathcal{I}\), with the number of the player being \(I\coloneqq \left\vert \mathcal{I} \right\vert \).
	Assume that each player has a finite set \(\mathcal{S}_i\) of actions to choose from with size \(\left\vert \mathcal{S}_i \right\vert \).
	Then, we can define a \emph{strategy vector}, which denotes the action chosen by all players such that
	\[
		s\coloneqq (s_1, \ldots , s_I)
	\]
	with dimension being \(I\) for every player \(i\). We also define the vector of opponents' strategy \(s_{-i}\) such that
	\[
		s_{-i}\coloneqq (s_1,\ldots,s_{i - 1},s_{i + 1} ,\ldots,s_I)
	\]
	with dimension being \(I - 1\).

	Finally, the utility function for player \(i\) is
	\[
		u_{i} \coloneqq \prod\limits_{j = 1}^{I} \mathcal{S}_j \to \mathbb{\MakeUppercase{R}},
	\]
	where \(u_{i}(s) = u_{i}(s_1, \ldots , s_I) = u_{i}(s_{i}, s_{-i})\).
\end{definition}

\begin{definition}
	Given the strategies of all other players, the subset of \(s_{i}\) that maximize the payoff of player \(i\) is called the \emph{Best response correspondence}.
	Let
	\[
		\mathrm{BR}_i(s_{-i})
	\]
	denotes the \emph{best response} for player \(i\) when the other player is \(s_{-i}\). Note that
	\[
		\mathrm{BR}_i(s_{-i}) \subseteq \mathcal{S}_i
	\]
	such that
	\[
		\mathrm{BR}_i(s_{-i})= \underset{s_{i}\in\mathcal{S}_i}{\arg\max}\ u_{i}(s_{i}, s_{-i}).
	\]
\end{definition}
