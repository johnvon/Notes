\lecture{6}{20 Sep. 12:30}{Homophily}
\section{Shortest Path Algorithm}
We want to develop some algorithms similar to BFS to find the distance between nodes. We first make the notion of distance between nodes rigorous.

\begin{definition}[Shortest path]\label{def:shortest-path}
	Given a \hyperref[def:graph]{graph}, a \emph{shortest path} between two nodes \(u, v\) is a \hyperref[def:path]{path} whose end nodes are
	\(u, u\) while the sum of the weights\footnote{We only discuss unweighted \hyperref[def:graph]{graph} so far, where we treat the weight as \(1\) for all edges.}
	of its constituent edges is minimized.
\end{definition}

\begin{definition}[Distance between nodes]\label{def:distance-between-nodes}
	We usually let the length of the \hyperref[def:shortest-path]{shortest path} between two nodes in a \hyperref[def:graph]{graph} \(\mathcal{\MakeUppercase{g}}\)
	be the \emph{distance between these two nodes}, denotes as
	\[
		d_{\mathcal{G}}(\cdot, \cdot),
	\]
	which takes two input for two corresponding nodes.
\end{definition}

Something like
\[
	d_{\mathcal{\MakeUppercase{g}} } (i, j) = 1 + \min_{k\in N_{\mathcal{\MakeUppercase{g}} } (j)} d_{\mathcal{\MakeUppercase{g}} } (i, k).
\]

\subsection{Dynamic programming}
There are many approaches to calculate the distance between nodes, one way is \emph{Djikstra's algorithm}.

\section{Homophily}
Links in a (social) network are formed between people that are \emph{similar}. This is an important feature that leads to
the formation of communities.

\subsection{Define \emph{Like} and \emph{Opposite}}
Consider a network with two types of individuals:
\begin{eg}[Middle school]
	Consider a school of boys and girls. Suppose the fraction of boy is $p$ and the fraction of girls is $q = 1 - p$. We have
	\[
		\frac{\binom{np}{2}}{\binom{n}{2}} \cong p^2
	\]
	and also
	\[
		1 - p^2 - q^2 = 2pq.
	\]

	If the fraction of cross connection $\ll 2pq$, homophily is considered likely. Otherwise, if the fraction $\gg 2pq$, inverse-homophily is likely to occur.
\end{eg}

We see that some social implications arise from this
\begin{enumerate}
	\item schelling model
	\item parameters
	      \begin{enumerate}
		      \item satisfaction threshold
		      \item population of each partition
		      \item vacancy
	      \end{enumerate}
	\item a grid of X and O (denoting partition of agents).
\end{enumerate}

\begin{itemize}
	\item Satisfied Agent: A satisfied agent is one that is surrounded by at least \(t\) percent of agents that are like itself.
	\item Dynamics: When an agent is not satisfied, it can be moved to any vacant location in the grid.
\end{itemize}

