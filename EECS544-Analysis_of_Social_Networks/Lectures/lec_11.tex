\lecture{11}{6 Oct. 12:30}{Page Rank - Scaling}
\chapter{Scaled Page Rank}
\begin{prev}
	Basic page Rank Algorithm: If the page rank converges, then we must have
	\[
		r^{*} = (N^{T})r^{*} \iff (r^{*})^{T} N = (r^{*})^{T},
	\]
	\(r^{*}\) is a left eigenvector of \(N\) for eigenvalue \(1\).
\end{prev}
\begin{remark}
	row stochastic property \(\implies\) \(1\) is eigenvalue.
\end{remark}

\section{Irreducible}
\par Now, we consider a very useful property of matrix, called \emph{irreducible}.
\begin{prev}
	\(A\) is a non-negative \(n\times n\) matrix. We say that \(A\) is irreducible if \(\forall i, j\),
	\[
		\exists k>0 \text{ such that }(A^k)_{ij} > 0.
	\]
\end{prev}
\begin{note}
	We can express the above as
	\[
		\underbrace{A_{im_1} A_{m_{1}m_2} \ldots A_{m_{k-1}j}}_{k\text{ terms}}>0.
	\]
\end{note}
Define \(\widetilde{A}\) with entries in \(\{0, 1\}\) such that
\[
	\widetilde{A}_{ij} = \begin{dcases}
		1 & , \text{ if }A_{ij}>0 \\
		0 & , \text{ otherwise}.
	\end{dcases}.
\]
\(\widetilde{A}\) is an adjacency matrix of same directed graph on \(n\) nodes.
\[
	A_{im_1} A_{m_1 m_2}\ldots A_{m_{k-1} j} > 0 \iff \widetilde{A}_{im_1} = \widetilde{A}_{m_1 m_2} = \ldots = \widetilde{A}_{m_{k-1}j} = 1.
\]
Then we see that the irreducibility of \(A \iff\) irreducibility of \(\widetilde{A}\iff\) \(\mathcal{G}\) corresponding to \(\widetilde{A}\) is strongly connected such that
\(\forall  i, j\) we can find a path on \(\mathcal{G}\).
\begin{remark}
	This graph \(\mathcal{G}\) can be non-simple graph.
\end{remark}

\section{Periodicity}
\begin{definition}[Period]
	Given \(A\) and \(i\in\{1, \ldots , n\}\), find all \(m>0\) such that
	\[
		(A^m)_{ii}>0.
	\]
	And the greatest common divisor of them is called the \emph{period of} \(i\).
\end{definition}

\begin{definition}[Aperiodic]
	Given \(A\) and \(i\in\{1, \ldots , n\}\), if the period is \(1\) for node \(i\), then\(i\) is said to be \emph{aperiodic}.
\end{definition}
\begin{remark}
	From this definition, we have
	\begin{itemize}
		\item A sufficient condition for aperiodic is \(A_{ii} > 0 \implies \) period is \(1\) for \(i\).
		\item And \(A\) is said to be aperiodic if the period of all nodes is \(1\).
		\item If \(A\) is a positive matrix(all entries are positive), then we know that
		      \[
			      \begin{alignedat}{3}
				      &\implies A_{ij}>0\ \forall i, j &&\implies \text{irreducible}\\
				      &\implies A_{ii}>0\ \forall i &&\implies \text{aperiodic}
			      \end{alignedat}
		      \]
		      Notice that this is only a simple sufficient condition for a matrix to be irreducible and aperiodic.
	\end{itemize}
\end{remark}

\section{Perron-Frobenius Theorem}
Come back to Perron-Frobenius Theorem:
\begin{theorem}[Perron-Frobenius theorem]\label{Perron-Frobenius Theorem}
	Let \(A\) be an irreducible non-negative matrix, then the eigenvalue with the \emph{largest magnitude is positive}
	and has \emph{multiplicity 1}. This eigenvalue is called the \textbf{spectral radius} or the \textbf{Perron-Frobenius eigenvalue},
	denoted as \(\rho(A)\).
	Both the right eigenvector(\(\vec{v}, A \vec{v} = \rho(A) \vec{v}\)) and the left eigenvector(\(\vec{w}, \vec{w}^{T}A = \rho(A) \vec{w}^{T}\))
	can be taken to be positive with
	\[
		\vec{w}^{T} \vec{v} = 1.
	\]
	Furthermore, we have
	\begin{enumerate}
		\item The following holds:
		      \[
			      \lim_{m \to \infty} \frac{1}{m} \sum\limits_{k=0}^{m-1} \frac{A^k}{(\rho(A))^k} = \vec{v} \vec{w}^{T}_{(n\times n)}.
		      \]
		\item The following holds:
		      \begin{itemize}
			      \item Row sum bound:
			            \[
				            \min_i \sum\limits_{j} A_{ij} \leq \rho(A).\leq \max_i \sum\limits_{j} A_{ij}.
			            \]
			      \item Column sum bound:
			            \[
				            \min_j \sum\limits_{i} A_{ij} \leq \rho(A).\leq \max_j \sum\limits_{i} A_{ij}.
			            \]
		      \end{itemize}
		\item If, in addition, \(A\) is also \emph{aperiodic}, then
		      \[
			      \lim_{m \to \infty} \frac{A^m}{(\rho(A))^m} = \vec{v} \vec{w}^{T}.
		      \]
		      This is a stronger statement!
	\end{enumerate}
\end{theorem}

\begin{problem}
We now see the problem for the basic page rank algorithm:
\begin{enumerate}
	\item \(N\) is non-negative and row stochastic \(\implies\) \(\rho(N) = 1\).
	\item \(N \to \widetilde{A}\): Adjacency matrix of origin
	\item Irreducibility is not a
\end{enumerate}
\end{problem}

\section{Scaled Page Rank}
Scaled Page Rank is a modification of Basic Page Rank that gives a matrix \(\widetilde{N}(s)\) that is always irreducible and aperiodic with
\[
	s\in \left( 0, 1 \right).
\]
\begin{enumerate}
	\item INITIALIZATION: same as Basic Page Rank
	\item SEND \& RECEIVE: same as Basic Page Rank
	\item SCALE down all page ranks by \(s\)(multiply by \(s\)) and add \(\frac{1-s}{V}\) to the page ranks of all nodes.
	\item REPEAT from step 2.
\end{enumerate}

\[
	r_{i}^{new} = \left(\sum\limits_{j:A_{ji} = 1} \frac{u_{j}^{old}}{d_{j}^{old}}+\begin{dcases}
		u_{i}^{old} & , \text{ if }d_{i}^{out} = 0, \\
		0           & , \text{ otherwise}
	\end{dcases}\right)+\frac{1-s}{V}
\]
Then we have
\[
	\begin{split}
		r^{new} &= s(N^{T}r^{old})+\frac{1-s}{V}\vec{1}\\
		&= s(N^{T}r^{old})+\frac{1-s}{V}\vec{1}\cdot 1\\
		&= s(N^{T}u^{old}) + \frac{1-s}{V}\vec{1} \underbrace{(\vec{1}^{T}r^{odd})}_{\text{add to }1}
		= \underbrace{(sN + \frac{1-s}{V}\vec{1}\vec{1}^{T})^{T}}_{\widetilde{N}(s)} r^{old}.
	\end{split}
\]
Finally, we have the scaled page rank being
\[
	\begin{split}
		&\implies r^{new} = (\widetilde{N}(s))^{T}r^{old}\\
		&\implies r(k) = (\widetilde{N}(s)^{T})^k r(0).
	\end{split}
\]

\begin{note}
	\(\widetilde{N}(s)\):
	\begin{itemize}
		\item non-negative
		\item row-stochastic
		\item strictly positive with all entries at least \(\frac{1-s}{V}\)
	\end{itemize}
\end{note}
\(\rho(\widetilde{N}(s)) = 1\) and irreducible and aperiodic! \(\implies (\widetilde{N}(s))^k \to \vec{v}\vec{w}^{T}\) where
\(\vec{v}\) is the right eigenvector of \(\widetilde{N}(s)\) and \(\vec{w}\) is the left eigenvector of \(\widetilde{N}(s)\) with
\(\vec{w}^{T} \vec{v} = 1\).

And since \(\widetilde{N}(s)\) is row stochastic, we have \(\vec{v} = \vec{1}\), hence,
\[
	\vec{w}^{T}\vec{1} = 1
\]
which means the sum of all entries is \(1\) and \(\vec{2}\) has all positive entries.

If \((\vec{v}, \vec{w})\) are right and left eigenvectors of \(\widetilde{N}(s) \iff (\vec{w}, \vec{v})\) are the right and left eigenvectors
of \(\widetilde{N}^{T}(s)\).

Scaled Page Rank converges to
\[
	r^{*} = (\widetilde{N}(s))^{T}r^{*}.
\]

We need the right eigenvector of \(\widetilde{N}^{T}(s) \iff \) same as left eigenvector of \(\widetilde{N}(s)\), which is \(\vec{w}(s)\).

\begin{remark}
	\(s \cong 0.8 ~ 0.85\)
\end{remark}

\begin{prev}
	HITS: We now start to investigate \(A A^{T}\), where \(A\) is adjacency matrix.
	\begin{itemize}
		\item Non-negative
		\item largest eigenvalue - Perron-Frobenius eigenvalues
	\end{itemize}
\end{prev}
