\lecture{11}{6 Oct. 12:30}{Scaled Page Rank}
\begin{prev}
	\hyperref[algo:basic-page-rank-algorithm]{Basic page rank algorithm}: If the page rank converges,
	then we must have
	\[
		r^{\ast} = (N^{\top})r^{\ast} \iff (r^{\ast})^{\top} N = (r^{\ast})^{\top},
	\]
	where \(r^{\ast}\) is a left \hyperref[def:eigenvector]{eigenvector} of \(N\) for \hyperref[def:eigenvalue]{eigenvalue} \(1\).
\end{prev}
\begin{remark}
	Row stochastic property \(\implies\) \(1\) is \hyperref[def:eigenvalue]{eigenvalue}.
\end{remark}

\subsection{Periodicity}
\begin{definition}[Period]\label{def:period}
	Given \(A_{n\times n}\) and \(i\in\{1, \ldots , n\}\), find all \(m>0\) such that
	\[
		(A^m)_{ii}>0.
	\]
	And the greatest common divisor of them is called the \emph{period of} \(i\).
\end{definition}

\begin{definition}[Aperiodic]\label{def:aperiodic}
	Given \(A_{n\times n}\) and \(i\in\{1, \ldots , n\}\), if the \hyperref[def:period]{period} is \(1\) for node \(i\), then \(i\)
	is said to be \emph{aperiodic}. And \(A\) is said to be \hyperref[def:aperiodic]{aperiodic} all \(i\) are \emph{aperiodic}.
\end{definition}

\begin{remark}\label{rmk:lec11-1}
	We see that
	\begin{itemize}
		\item A sufficient condition for \hyperref[def:aperiodic]{aperiodic} is \(A_{ii} > 0 \) since it implies the \hyperref[def:period]{period} is \(1\) for \(i\).
		\item If \(A\) is a positive matrix\footnote{All entries are positive}, then we know that
		      \[
			      \begin{alignedat}{3}
				      A_{ij}>0\ &\forall i, j &&\implies \text{\hyperref[def:irreducible]{irreducible}}\\
				      A_{ii}>0\ &\forall i &&\implies \text{\hyperref[def:aperiodic]{aperiodic}}.
			      \end{alignedat}
		      \]
		      Notice that this is only a simple sufficient condition for a matrix to be \hyperref[def:irreducible]{irreducible} and \hyperref[def:aperiodic]{aperiodic}.
	\end{itemize}
\end{remark}

\subsection{Perron-Frobenius Theorem}
We now start to study our main theorem, the \hyperref[thm:Perron-Frobenius-theorem]{Perron-Frobenius theorem}. Before that, we introduce a useful definition.

\begin{definition}[Spectral radius]\label{def:spectral-radius}
	The \hyperref[def:eigenvalue]{eigenvalue} of \(A\) with the largest absolute value is called the \emph{spectral radius} or the \emph{Perron-Frobenius \hyperref[def:eigenvalue]{eigenvalue}} of \(A\), denoted as \(\rho(A)\).
\end{definition}

\begin{theorem}[Perron-Frobenius theorem]\label{thm:Perron-Frobenius-theorem}
	Let \(A\) be an \hyperref[def:irreducible]{irreducible} non-negative matrix, then the \hyperref[def:spectral-radius]{spectral radius} is positive
	and has \hyperref[def:multiplicity]{multiplicity} \(1\).

	Both the right \hyperref[def:eigenvector]{eigenvector} (\(\vec{v}, A \vec{v} = \rho(A) \vec{v}\)) and the left \hyperref[def:eigenvector]{eigenvector} (\(\vec{w}, \vec{w}^{\top}A = \rho(A) \vec{w}^{\top}\))
	can be taken to be positive with
	\[
		\vec{w}^{\top} \vec{v} = 1.
	\]
	Furthermore, we have
	\begin{enumerate}[(a)]
		\item The following holds
		      \[
			      \lim_{m \to \infty} \frac{1}{m} \sum\limits_{k=0}^{m-1} \frac{A^k}{(\rho(A))^k} = \vec{v} \vec{w}^{\top}_{(n\times n)}.
		      \]
		\item The following holds
		      \begin{itemize}
			      \item Row sum bound:
			            \[
				            \min_i \sum\limits_{j} A_{ij} \leq \rho(A) \leq \max_i \sum\limits_{j} A_{ij}.
			            \]
			      \item Column sum bound:
			            \[
				            \min_j \sum\limits_{i} A_{ij} \leq \rho(A) \leq \max_j \sum\limits_{i} A_{ij}.
			            \]
		      \end{itemize}
		\item If \(A\) is also \hyperref[def:aperiodic]{aperiodic}, then
		      \[
			      \lim_{m \to \infty} \frac{A^m}{(\rho(A))^m} = \vec{v} \vec{w}^{\top}.
		      \]
	\end{enumerate}
\end{theorem}
\begin{proof}
	We omit the proof, but it's worth checking out. See \href{https://pi.math.cornell.edu/~web6720/Perron-Frobenius_Hannah%20Cairns.pdf}{here} for a short proof.
\end{proof}

\begin{prev}
	For the \hyperref[algo:basic-page-rank-algorithm]{basic page rank algorithm}:
	\begin{enumerate}
		\item \(N\) is non-negative and row stochastic, so we know \(\rho (N) \leq 1\). But since we already showed \(1\) is one of the \hyperref[def:eigenvalue]{eigenvalues} of \(N \implies \rho(N) = 1\).
		\item \(N \to \widetilde{A}\): \hyperref[def:adjacency-matrix]{Adjacency matrix} of original \hyperref[def:graph]{graph} plus self-loops for any nodes with no outgoing links.
	\end{enumerate}
\end{prev}

We can now say something about \(N\).

\begin{remark}
	We don't certainly have that \(N\) is \hyperref[def:irreducible]{irreducible}.
\end{remark}
\begin{explanation}
	Since \(\rho (N) = 1\) but the \hyperref[def:multiplicity]{multiplicity} can be \(> 1\) or \hyperref[def:eigenvector]{eigenvectors} may not all be positive.
	From \autoref{thm:Perron-Frobenius-theorem}, \(N\) cannot be \hyperref[def:irreducible]{irreducible} in both cases.
\end{explanation}

\begin{remark}
	We don't certainly have that \(N\) is \hyperref[def:aperiodic]{aperiodic}.
\end{remark}
\begin{explanation}
	Consider the following network:
	\begin{figure}[H]
		\centering
		\incfig{rmk:lec11-1}
		\label{fig:rmk:lec11-1}
	\end{figure}
	For this network, we have
	\[
		N=\begin{pmatrix}
			0 & 1 & 0 \\
			0 & 0 & 1 \\
			1 & 0 & 0 \\
		\end{pmatrix},\quad N^2 = \begin{pmatrix}
			0 & 0 & 1 \\
			1 & 0 & 0 \\
			0 & 1 & 0 \\
		\end{pmatrix}, \quad N^3 = \begin{pmatrix}
			1 & 0 & 0 \\
			0 & 1 & 0 \\
			0 & 0 & 1 \\
		\end{pmatrix}
	\]
	and for higher power it just repeats. We see that the \hyperref[def:period]{period} for all three nodes	are all \(3\),
	which means that \(N\) is not \hyperref[def:aperiodic]{aperiodic}.
\end{explanation}
This is problematic, since for \hyperref[algo:basic-page-rank-algorithm]{basic page rank algorithm}, we have
\[
	r(k) = (N^{\top} )^k r(0),
\]
hence we are interested in the case that \((N^{\top})^k\) converges.

\section{Scaled Page Rank Algorithm}
Scaled page rank is a modification of page rank that gives a matrix \(\widetilde{N}(s)\) that is always \hyperref[def:irreducible]{irreducible}
and \hyperref[def:aperiodic]{aperiodic} with \(s\in (0, 1)\). We first give the algorithm to calculate it.
\par
\begin{algorithm}[H]\label{algo:scaled-page-rank-algorithm}
	\DontPrintSemicolon
	\caption{Scaled Page Rank Algorithm}
	\SetKwData{pg}{SPageRank}
	\SetKwData{pgnew}{SPageRankNew}
	\SetKwData{n}{neighborNum}
	\KwData{A network \(\mathcal{\MakeUppercase{g}} = (\mathcal{\MakeUppercase{v}} , \mathcal{\MakeUppercase{e}} )\), scale factor \(s\in (0, 1)\), iteration \(L\)}
	\KwResult{\pg}

	\BlankLine

	\For(\Comment*[f]{\hyperref[algo:scaled-page-rank-algorithm:initialize]{Initialize}}){\(v\in \mathcal{\MakeUppercase{v}}\)}{
		\(\pg(v) \gets 1 / \left\vert \mathcal{\MakeUppercase{v}} \right\vert \)\;
	}

	\;

	\For{\(i = 1, \ldots , L\)}{
		\For(\Comment*[f]{\hyperref[algo:scaled-page-rank-algorithm:send]{Send}}){\(v\in \mathcal{\MakeUppercase{v}}\)}{
			\(\n \gets \left\vert \{v\to u \in \mathcal{\MakeUppercase{e}} \} \right\vert\)\;
			\uIf(\Comment*[f]{divides \(\pg(v)\) to its (out)neighbor}){\(\n \neq 0\)}{
				\For(){\(v \to u \in \mathcal{\MakeUppercase{e}}\)}{
					\(\pgnew(u) \gets \pgnew(u) + \pg(v) / \n\)\;
				}
			}\Else(\Comment*[f]{no outgoing links}){
				\(\pgnew(v)\gets \pgnew(v) + \pg(v)\)\;
			}
		}
		\;
		\For(\Comment*[f]{\hyperref[algo:scaled-page-rank-algorithm:receive]{Receive}}){\(v\in \mathcal{\MakeUppercase{v}}\)}{
			\(\pg(v)\gets \pgnew(v)\)\Comment*[r]{Update the page rank}
			\(\pgnew(v) \gets 0\)\Comment*[r]{Clean up for the next iteration's updating}
		}
		\For(\Comment*[f]{\hyperref[algo:scaled-page-rank-algorithm:scale]{Scale}}){\(v\in \mathcal{\MakeUppercase{v}}\)}{
			\(\pg(v)\gets s\times \pgnew(v)\)\Comment*[r]{Scaled down by \(s\)}
			\(\pg(v)\gets \frac{1-s}{\left\vert \mathcal{\MakeUppercase{v}}  \right\vert }\)\Comment*[r]{Re-distribute page rank}
		}
	}
	\;
	\Return{\pg}\;
\end{algorithm}

\begin{remark}[Algorithm explanation]
	The following is the detailed explanation of the \hyperref[algo:scaled-page-rank-algorithm]{scaled page rank algorithm}.
	\begin{enumerate}[(a)]
		\item \label{algo:scaled-page-rank-algorithm:initialize} Initialization: same as \hyperref[algo:basic-page-rank-algorithm]{basic page rank algorithm}.
		\item \label{algo:scaled-page-rank-algorithm:send} Send: same as \hyperref[algo:basic-page-rank-algorithm]{basic page rank algorithm}.
		\item \label{algo:scaled-page-rank-algorithm:receive} Receive: same as \hyperref[algo:basic-page-rank-algorithm]{basic page rank algorithm}.
		\item \label{algo:scaled-page-rank-algorithm:scale} Scale: Scale down all page ranks by \(s\) (multiply by \(s\)) and
		      add \((1-s) / \left\vert \mathcal{\MakeUppercase{v}} \right\vert \) to the page ranks of all nodes.
	\end{enumerate}
\end{remark}

\section{Analysis on Scaled Page Rank}
Again, let's denote the scaled page rank of node \(i\) as \(r_i\) throughout the following discussion.
\[
	r_{i}^{\text{new}} = s\left(\sum\limits_{j\colon A_{ji} = 1} \frac{u_{j}^{\text{old}}}{d_{j}^{\text{old}}}+\begin{dcases}
		u_{i}^{\text{old}} & , \text{ if }d_{i}^{\text{out}} = 0; \\
		0                  & , \text{ otherwise};
	\end{dcases}\right)+\frac{1-s}{\vert \mathcal{V}\vert}.
\]
Then we have
\[
	\begin{split}
		r^{\text{new}} = s(N^{\top}r^{\text{old}})+\frac{1-s}{\vert \mathcal{V}\vert}\vec{1}
		= s(N^{\top}u^{\text{old}}) + \frac{1-s}{\vert \mathcal{V}\vert}\vec{1} \underbrace{\vphantom{\left(sN + \frac{1-s}{\vert \mathcal{V}\vert}\vec{1}\vec{1}^{\top}\right)^{\top}}(\vec{1}^{\top}r^{\text{old}})}_{\text{add to }1}
		= \underbrace{\left(sN + \frac{1-s}{\vert \mathcal{V}\vert}\vec{1}\vec{1}^{\top}\right)^{\top}}_{\widetilde{N}(s)^{\top}} r^{\text{old}}.
	\end{split}
\]
Finally, we have the rule of updating scaled page rank being \(r^{\text{new}} = (\widetilde{N}(s))^{\top}r^{\text{old}}\), hence
\[
	r(k) = \left(\widetilde{N}(s)^{\top}\right)^k r(0).
\]

\begin{note}
	We see that \(\widetilde{N}(s)\) is
	\begin{itemize}
		\item non-negative
		\item row stochastic
		\item strictly positive with all entries at least \((1-s) / \left\vert \mathcal{\MakeUppercase{v}} \right\vert \)
	\end{itemize}
\end{note}
Firstly, since \(\widetilde{N} (s)\) is non-negative and is row-stochastic, we see that \(\widetilde{N} (s)\vec{1} = \vec{1} \), hence
\(\rho(\widetilde{N}(s)) = 1\).

Moreover, since \(\widetilde{N} (s)\) is strictly positive, from the \hyperref[rmk:lec11-1]{remark}, we have that \(\widetilde{N} (s)\) is
both \hyperref[def:irreducible]{irreducible} and \hyperref[def:aperiodic]{aperiodic}. With the non-negativity, we can then apply
\autoref{thm:Perron-Frobenius-theorem} (c) to get
\[
	\lim\limits_{m \to \infty} \frac{(\widetilde{N} (s))^m}{(\rho (\widetilde{N} (s)))^m} = \lim\limits_{m \to \infty} (\widetilde{N} (s))^m = \vec{v} \vec{w} ^{\top},
\]
where \(\vec{v}\) is the right \hyperref[def:eigenvector]{eigenvector} of \(\widetilde{N}(s)\) and \(\vec{w}\) is the left \hyperref[def:eigenvector]{eigenvector} of \(\widetilde{N}(s)\) with
\(\vec{w}^{\top} \vec{v} = 1\).

And since \(\widetilde{N}(s)\) is row stochastic, we have \(\vec{v} = \vec{1}\), hence \(\vec{w}^{\top}\vec{1} = 1\),
which means the sum of all entries is \(1\) and \(\vec{w}\) has all positive entries.

If \((\vec{v}, \vec{w})\) are right and left \hyperref[def:eigenvector]{eigenvectors} of \(\widetilde{N}(s) \iff (\vec{w}, \vec{v})\) are the right and left \hyperref[def:eigenvector]{eigenvectors}
of \(\widetilde{N}(s)^{\top}\).

\begin{remark}
	Scaled page rank converges to \(r^{\ast} \) where
	\[
		r^{\ast} = \widetilde{N}(s)^{\top}r^{\ast},
	\]
	which is the right \hyperref[def:eigenvector]{eigenvector} of \(\widetilde{N}(s)^{\top}\). But it's the same as the left \hyperref[def:eigenvector]{eigenvector}
	of \(\widetilde{N}(s)\), which is just
	\(\vec{w}(s)\).
\end{remark}

\begin{note}
	In practice, the scale factor is usually set as \(s \cong 0.8 \sim 0.85\).
\end{note}