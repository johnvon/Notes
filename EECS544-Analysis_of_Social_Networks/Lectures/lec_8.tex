
\lecture{8}{27 Sep. 12:30}{Diagonalization}
\newcommand*{\horzbar}{\rule[.5ex]{2.5ex}{0.5pt}}
\section{Eigen Decomposition}
From \hyperref[thm:spectral-theorem]{spectral theorem}, we can \textbf{diagonalize} a \hyperref[def:symmetric-matrix]{symmetric matrix} \(A\)
via so-called \hyperref[thm:eigen-decomposition]{eigen decomposition} as follows.

\begin{theorem}[Eigen decomposition]\label{thm:eigen-decomposition}
	For a \hyperref[def:symmetric-matrix]{symmetric matrix} \(A\), we can always write
	\[
		A = U D U^{\top},
	\]
	where \(U\) is the \hyperref[def:eigenvalue]{eigenvalues} matrix, and \(D\) is the diagonal matrix of \hyperref[def:eigenvalue]{eigenvalues}. W
\end{theorem}
\begin{proof}
	Firstly, from \autoref{thm:spectral-theorem}, denote \(n\) right \hyperref[def:eigenvector]{eigenvectors} of \(A\) by \(u_1, \ldots , u_n\) where \(u_{i} \)
	corresponds to eigenvalue \(\lambda _i\) such that
	\[
		\lambda _1 \leq \lambda _2 \leq \ldots  \leq \lambda _n.
	\]
	Corresponding to this order, we define \(U\) as
	\[
		U = \begin{pmatrix}
			\mid & \mid &        & \mid \\
			u_1  & u_2  & \cdots & u_n  \\
			\mid & \mid &        & \mid \\
		\end{pmatrix},
	\]
	then we have
	\begin{enumerate}[(a)]
		\item \(u_i\) are linear independent, and they form a basis in \(\mathbb{\MakeUppercase{r}} ^n\).
		\item \(u_{i} ^{\top} u_{j} ( = \left< u_{i} , u_{j}  \right> ) = 0\) for \(i\neq j\).
		\item \(u_{i} ^{\top} u_{i} = \sum_{j=1}^{n} u_{ij}^{2} = 1 \) (\emph{length} \(= 1\)).
	\end{enumerate}
	\begin{remark}[\(U\) is invertible]
		We see that \(U\) is invertible. Indeed, we have \(U^{-1} = U^{\top} \) since
		\[
			U^{\top}U = \begin{pmatrix}
				\horzbar & u^{\top}_1 & \horzbar \\
				\horzbar & u^{\top}_2 & \horzbar \\
				         & \vdots     &          \\
				\horzbar & v^{\top}_n & \horzbar \\
			\end{pmatrix}\begin{pmatrix}
				\mid & \mid &        & \mid \\
				u_1  & u_2  & \cdots & u_n  \\
				\mid & \mid &        & \mid \\
			\end{pmatrix} = I.
		\]
	\end{remark}
	Then, we claim that we can write \(A\) as \(A = U D U^{\top}\) where \[
		D  = \begin{pmatrix}
			\lambda_1 &           &        &           \\
			          & \lambda_2 &        &           \\
			          &           & \ddots &           \\
			          &           &        & \lambda_n \\
		\end{pmatrix}.
	\]
	To see this, from \(U^{\top} = U^{-1} \), we have \(A U = (U D U^{\top} ) U = U D\) since
	\[
		\begin{split}
			A U = A \begin{pmatrix}
				\mid &        & \mid \\
				u_1  & \cdots & u_n  \\
				\mid &        & \mid \\
			\end{pmatrix} &= \begin{pmatrix}
				\mid  &        & \mid  \\
				A u_1 & \cdots & A u_n \\
				\mid  &        & \mid  \\
			\end{pmatrix}\\ &= \begin{pmatrix}
				\mid           &        & \mid           \\
				\lambda _1 u_1 & \cdots & \lambda _n u_n \\
				\mid           &        & \mid           \\
			\end{pmatrix} = \begin{pmatrix}
				\mid &        & \mid \\
				u_1  & \cdots & u_n  \\
				\mid &        & \mid \\
			\end{pmatrix} \begin{pmatrix}
				\lambda_1 &        &           \\
				          & \ddots &           \\
				          &        & \lambda_n \\
			\end{pmatrix} = UD
		\end{split}
	\]
	Hence we're done.
\end{proof}

\begin{eg}
	Find the \hyperref[def:eigenvalue]{eigenvalue} of
	\[
		A = \begin{pmatrix}
			1 & 0 & 0 \\
			0 & 2 & 1 \\
			0 & 1 & 2 \\
		\end{pmatrix}.
	\]
\end{eg}
\begin{explanation}
	The \hyperref[note:characteristic-equation]{characteristic equation} of \(A\) is
	\[
		\begin{split}
			\det(\lambda I - A) &= \det \left(  \begin{pmatrix}
				\lambda - 1 & 0            & 0             \\
				0           & \lambda -  2 & -1            \\
				0           & -1           & \lambda -   2 \\
			\end{pmatrix}\right)\\
			&= (\lambda - 1)((\lambda - 2 )^2 - 1)\\
			&= (\lambda - 1)(\lambda - 2 - 1)(\lambda - 2 + 1)\\
			&= (\lambda - 1)(\lambda - 3)(\lambda - 1) = 0.
		\end{split}
	\]
	We see that the roots of the \hyperref[note:characteristic-equation]{characteristic equation} are \(1, 1, 3\) with
	\(1\) has \hyperref[def:multiplicity]{multiplicity} \(2\).
\end{explanation}

\begin{eg}
	Give an example of \(a\) where \(A\) defined as
	\[
		A = \begin{pmatrix}
			1 & 0 & 0 \\
			0 & 2 & 1 \\
			0 & a & 2 \\
		\end{pmatrix}
	\]
	has complex \hyperref[def:eigenvalue]{eigenvalues}.
\end{eg}
\begin{explanation}
	The \hyperref[note:characteristic-equation]{characteristic equation} of \(A\) is
	\[
		\begin{split}
			\det(\lambda I - A) &= \det \left(  \begin{pmatrix}
				\lambda - 1 & 0            & 0             \\
				0           & \lambda -  2 & -1            \\
				0           & -a           & \lambda -   2 \\
			\end{pmatrix}\right)\\
			&= (\lambda - 1)((\lambda - 2 )^2 - a)\\
			&= (\lambda - 1)(\lambda^2 - 4\lambda + 4 - a) = 0
		\end{split}
	\]
	Clearly, there is a root \(1\) for \(\lambda - 1 = 0\), as for the quadratic \(\lambda^2 - 4\lambda + (4 - a) = 0\),
	\(\vartriangle = 16 - 4(4-a) = 4a\).
	We see that for \(a = -1\), \(\vartriangle = -4 < 0 \implies\) the quadratic has complex roots. Hence,
	\[
		A = \begin{pmatrix}
			1 & 0  & 0 \\
			0 & 2  & 2 \\
			0 & -1 & 2 \\
		\end{pmatrix}
	\]
	has a complex \hyperref[def:eigenvalue]{eigenvalue}.
\end{explanation}

\section{Network Decomposition}\label{sec:network-decomposition}
Now comes to our main goal, i.e., to do the network decomposition. First, let's define a matrix which capture the \hyperref[def:graph]{graph}
structure.
\begin{definition}[Laplacian]\label{def:Laplacian}
	For an \hyperref[def:undirected-graph]{undirected graph}, the \emph{Laplacian} is defined as \(L \coloneqq D - A\), where \(A\) is the \hyperref[def:adjacency-matrix]{adjacent matrix},
	\(D\) is diagonal \hyperref[def:degree]{degree} matrix.
\end{definition}

\begin{remark}
	Since
	\[
		L^{\top} = (D - A)^{\top} = D^{\top} - A^{\top} = D - A = L,
	\]
	we see that \(L\) is \hyperref[def:symmetric-matrix]{symmetric}, hence its \hyperref[def:eigenvalue]{eigenvalues} are real by \autoref{thm:spectral-theorem}.
	Many incidence matrices \(B_{\left\vert \mathcal{\MakeUppercase{v}} \right\vert \times \left\vert \mathcal{\MakeUppercase{e}} \right\vert }\) which encode the
	relation between edges and nodes satisfies
	\[
		L = BB^{\top}.
	\]
	We'll show that \(L\) has non-negative \hyperref[def:eigenvalue]{eigenvalues} \hyperref[rmk:property-of-Laplacian]{later}.
\end{remark}
\subsection{Positive Definite and Positive Semi-Definite}
\begin{definition}[Positive definite]\label{def:positive-definite}
	\(A \) is \emph{positive definite} (PD) if and only if
	\[
		\forall \vec{x} \neq  \vec{0},\quad \vec{x}^{\top}A \vec{x} > 0.
	\]
\end{definition}

\begin{definition}[Positive semi-definite]\label{def:positive-semi-definite}
	\(A \) is \emph{positive semi-definite} (PSD) if and only if
	\[
		\forall \vec{x} \neq  \vec{0},\quad \vec{x}^{\top}A \vec{x} \geq  0.
	\]
\end{definition}

\begin{lemma}\label{lma:lec8-1}
	If \(\lambda\) is a real \hyperref[def:eigenvalue]{eigenvalue}, then \(\lambda \geq  0 \iff A\) is \hyperref[def:positive-semi-definite]{positive semi-definite};
	also, \(\lambda > 0 \iff A\) is \hyperref[def:positive-definite]{positive-definite}.
\end{lemma}
\begin{proof}
	Suppose \(\lambda\) is a real \hyperref[def:eigenvalue]{eigenvalue}, let \(\vec{u}\) be the \hyperref[def:eigenvector]{eigenvector} such that \(A \vec{u} = \lambda \vec{u}\), and
	\[
		\vec{u}^{\top} A \vec{u} = \vec{u}^{\top} (\lambda \vec{u}) = \lambda \underbrace{\vec{u}^{\top}\vec{u}}_{>0} \geq 0,
	\]
	which implies \(A\) is \hyperref[def:positive-semi-definite]{positive semi-definite}. Similarly, if \(\lambda > 0 \implies \vec{u}^{\top} \vec{u} > 0\),
	\(A\) is \hyperref[def:positive-definite]{positive definite}. Lastly, the backward direction is obvious.
\end{proof}

\begin{lemma}\label{lma:lec8-2}
	\(A\) is \hyperref[def:positive-semi-definite]{positive semi-definite} and \hyperref[def:symmetric-matrix]{symmetric} \(\iff \exists \widetilde{B}\) such
	that \(A = \widetilde{B} \widetilde{B} ^{\top}\).
\end{lemma}
\begin{proof}
	We see that if \(A = \widetilde{B} \widetilde{B} ^{\top} \), it's clearly \hyperref[def:symmetric-matrix]{symmetric} and
	\[
		x^{\top} A x = x^{\top} \widetilde{B} \widetilde{B} ^{\top} x = (\widetilde{B} x)^{\top} \widetilde{B} x = y^{\top} y \geq 0,
	\]
	hence \(A\) is \hyperref[def:positive-semi-definite]{positive semi-definite}.

	For another direction, from \autoref{thm:spectral-theorem}, since \(A\) is \hyperref[def:symmetric-matrix]{symmetric}, we know that \(A = U D U^{\top}\) where
	all entries of \(D\) is \(\geq 0\) by \autoref{lma:lec8-1}. We see that we can decompose \(D\) into \(D = C C^{\top} \) where
	\[
		D = \begin{pmatrix}
			\lambda_1 &           &        &           \\
			          & \lambda_2 &        &           \\
			          &           & \ddots &           \\
			          &           &        & \lambda_n \\
		\end{pmatrix} = \begin{pmatrix}
			\sqrt{\lambda_1} &                  &        &                  \\
			                 & \sqrt{\lambda_2} &        &                  \\
			                 &                  & \ddots &                  \\
			                 &                  &        & \sqrt{\lambda_n} \\
		\end{pmatrix} \begin{pmatrix}
			\sqrt{\lambda_1} &                  &        &                  \\
			                 & \sqrt{\lambda_2} &        &                  \\
			                 &                  & \ddots &                  \\
			                 &                  &        & \sqrt{\lambda_n} \\
		\end{pmatrix} \eqqcolon C C^{\top},
	\]
	then by defining \(\widetilde{B} \coloneqq UC\), we see that
	\[
		A = U D U^{\top} = U C C^{\top} U^{\top} = (UC) (UC)^{\top} = \widetilde{B} \widetilde{B} ^{\top}
	\]
	as desired.
\end{proof}

\begin{remark}[Properties of Laplacian]\label{rmk:property-of-Laplacian}
	Given a Laplacian \(L = B B^{\top}\), from \autoref{lma:lec8-2}, \(L\) is \hyperref[def:positive-semi-definite]{positive semi-definite},
	which further implies
	\[
		0 \leq \lambda_1 \leq \lambda_2 \leq \ldots \leq \lambda_n.
	\]
	But \(\lambda_1 = 0\) since we have \(L = D-A\) where \(\diag(d_1, \ldots , d_n)\) being the diagonal \hyperref[def:degree]{degree} matrix,
	and furthermore,
	\[
		L \vec{1} = (D - A)\vec{1} = \begin{pmatrix}
			d_1    \\
			d_2    \\
			\vdots \\
			d_n    \\
		\end{pmatrix} - A\vec{1} = \vec{0}
	\]
	since \(d_{i} = \sum_{j=1}^{n} A_{ij}\) from the \hyperref[def:adjacency-matrix]{definition} of \(A\), so
	\[
		L \vec{1} = 0 \vec{1}.
	\]
	Hence, \(0\) is an \hyperref[def:eigenvalue]{eigenvalue} with \hyperref[def:eigenvector]{eigenvector} \(\vec{1}\), so \(\lambda _1=0\).
\end{remark}

We now study the \hyperref[def:multiplicity]{multiplicity} of an \hyperref[def:eigenvalue]{eigenvalue} of \(L\). We first introduce the following definition.
\begin{definition}[Multiplicity]\label{def:multiplicity}
	We say the \hyperref[def:eigenvalue]{eigenvalue} \(\lambda ^{\ast} \) of \(A\) has \emph{multiplicity} \(k\) if
	\[
		\det(\lambda I - A) = (\lambda - \lambda ^{\ast} )^k g(\lambda )
	\]
	where \(g(\cdot)\) is a polynomial in \(\lambda\) with \(g(\lambda ^{\ast} )\neq 0\).
\end{definition}

If the network has two \hyperref[def:connected]{connected} components, we can then relabel the nodes and obtain a concatenation of two
\hyperref[def:adjacency-matrix]{adjacency matrices} that doesn't intersect
\[
	A = \begin{bmatrix}
		A_1 & 0   \\
		0   & A_2 \\
	\end{bmatrix}
\]
\begin{figure}[H]
	\centering
	\incfig{two-component-graph}
	\caption{Two \hyperref[def:connected]{connected} components \hyperref[def:graph]{graph}.}
	\label{fig:two-component-graph}
\end{figure}

Then we see that we can also represent the \hyperref[def:Laplacian]{Laplacian} in block matrix form as
\[
	L = \begin{bmatrix}
		D_{1} - A_{1} & 0         \\
		0             & D_2 - A_2 \\
	\end{bmatrix} \eqqcolon \begin{bmatrix}
		L_{1} & 0   \\
		0     & L_2 \\
	\end{bmatrix}.
\]
We then see that the \hyperref[note:characteristic-equation]{characteristic equation} of \(L\) is\footnote{Detailed derivation is omitted. It's essentially followed from the block structure.}
\[
	\det(\lambda I - L) = \det(\lambda I_1 - L_2)\det(\lambda  I_2 - L_2)
\]
Then, as we just \hyperref[rmk:property-of-Laplacian]{remarked}, both \(L_1\) and \(L_2\) has an \hyperref[def:eigenvalue]{eigenvalue} being \(0\), which implies that \(0\) is
an \hyperref[def:eigenvalue]{eigenvalue} of \(L\) with \hyperref[def:multiplicity]{multiplicity} \(2\).

We see that \hyperref[def:multiplicity]{multiplicity} of \hyperref[def:eigenvalue]{eigenvalue} of \hyperref[def:Laplacian]{Laplacian} can help us find communities.
This suggests the following definition.

\begin{definition}[Fiedler eigenvalue]\label{def:Filder-eigenvalue}
	The \emph{Fiedler \hyperref[def:eigenvalue]{eigenvalue}} of a \hyperref[def:graph]{graph} \(\mathcal{\MakeUppercase{g}}\) is the second-smallest eigenvalue
	(neglect \hyperref[def:multiplicity]{multiplicity}) of the \hyperref[def:Laplacian]{Laplacian} of \(\mathcal{\MakeUppercase{g}}\).
\end{definition}

We see that by first doing the \hyperref[thm:eigen-decomposition]{eigen-decomposition} on \hyperref[def:Laplacian]{Laplacian} and find all
\hyperref[def:eigenvalue]{eigenvalues} that are close to zero, which can be used to find communities, and nearly all community detection
algorithms rely on this fact.

\begin{prev}
	Turning back to \hyperref[algo:Girvan-Newman-algorithm]{Givran-Newman algorithm}, though it works at social network, but what about other tasks like
	internet?
	\begin{problem}
	Hyperlinks (directed): How important is the edge?
	\end{problem}
	\begin{answer}
		Internet search seems to rank nodes, not edges. (the website itself is the point?)
	\end{answer}

	An immediate question arises.

	\begin{problem}
	How does one rank nodes?
	\end{problem}
	\begin{answer}
		Network centrality.
	\end{answer}
\end{prev}
\begin{eg}[Network centrality]
	There are many examples of methods to quantify network centrality, e.g.,
	\begin{itemize}
		\item Degree. Ranks nodes by edges.
		\item Eigen-centrality. \hyperref[def:adjacency-matrix]{Adjacency matrix}, finds an \hyperref[def:eigenvector]{eigenvector} and use its entries.
	\end{itemize}
\end{eg}

We now study this in depth.

\chapter{HITS}
HITS is a hyperlinks-induced topic search algorithm proposed by \href{https://en.wikipedia.org/wiki/Jon_Kleinberg}{John Kleinberg}, whose backbones are the
idea of \textbf{hubs} and \textbf{authorities}.
\begin{itemize}
	\item Hubs: nodes that \emph{aggregate}.
	\item Authorities: nodes that are \emph{important}.
\end{itemize}

The key idea is that
\begin{enumerate}
	\item If many websites (nodes) point to it, then this website is important (authority property).
	\item If a node points to many authority nodes, then it is important as well.
\end{enumerate}
Then by iteratively identifying hubs and authorities, we can rank the nodes in the network. We now describe in detailed how this algorithm works.

\section{HITS Algorithm}
The pseudocode of the algorithm is as follows.

\par
\begin{algorithm}[H]\label{algo:HITS-algorithm}
	\DontPrintSemicolon
	\caption{HITS Algorithm}
	\SetKwData{h}{hub}
	\SetKwData{a}{auth}
	\KwData{A network \(\mathcal{\MakeUppercase{g}} = (\mathcal{\MakeUppercase{v}} , \mathcal{\MakeUppercase{e}} ) \)}
	\KwResult{\h, \a}

	\BlankLine

	\For(\Comment*[f]{Initialize}){\(v\in \mathcal{\MakeUppercase{v}} \)}{
		\(\a(v)\gets 1\)\;
		\(\h(v)\gets 1\)\;
	}
	\;
	\While{not converge}{
		\For(\Comment*[f]{\a update}){\(v\in \mathcal{\MakeUppercase{v}} \) }{
			\(\a(v) \gets \sum_{u\in \mathcal{\MakeUppercase{v}} \colon u\to v\in \mathcal{\MakeUppercase{e}}} \h(u)\)\;
		}
		\For(\Comment*[f]{\h update}){\(v\in \mathcal{\MakeUppercase{v}} \) }{
			\(\h(v) \gets \sum_{u\in \mathcal{\MakeUppercase{v}} \colon v\to u\in \mathcal{\MakeUppercase{e}}} \a(u)\)\;
		}
		\;
		\For(\Comment*[f]{\a normalize}){\(v\in \mathcal{\MakeUppercase{v}} \) }{
			\(\a(v) \gets \a(v) / \sum_{u\in \mathcal{\MakeUppercase{v}} } \a(u) \)\;
		}
		\For(\Comment*[f]{\h normalize}){\(v\in \mathcal{\MakeUppercase{v}} \) }{
			\(\h(v) \gets \h(v) / \sum_{u\in \mathcal{\MakeUppercase{v}}} \h(u) \)\;
		}
	}
	\Return{\h, \a}\;
\end{algorithm}

\begin{note}
	There are two facts are worth noting.
	\begin{enumerate}
		\item \(\mathtt{hub(\mathnormal{v})}\) and \(\mathtt{auth(\mathnormal{v})}\) will converge as long as initial score are positive.
		\item Final scores will be independent of the initial scores.
	\end{enumerate}
\end{note}
