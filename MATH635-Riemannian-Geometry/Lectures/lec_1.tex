\chapter{Manifolds}
\lecture{1}{5 Jan. 14:30}{A Foray to Smooth Manifolds}
\section{Differentiable Manifolds}
\subsection{Topological Manifolds}
Let's start with a common definition.

\begin{definition}[Topological manifold]\label{def:topological-manifold}
	A \emph{topological manifold} \(\mathcal{M} \) of dimension \(n\) is a (topological) Hausdorff space such that each point \(p\in \mathcal{M} \) has a neighborhood \(U\) homeomorphic via \(\varphi\colon U \to U^\prime \) to an open subset \(U^\prime \subseteq \mathbb{R} ^n\).
	\begin{definition}[Local coordinate map]\label{def:local-coordinate-map}
		For every \(p\in \mathcal{M} \), the corresponding homeomorphism \(\varphi \) is called the \emph{local coordinate map}.
	\end{definition}
	\begin{definition}[Local coordinate]\label{def:local-coordinate}
		The pull-back \((x^1, \ldots , x^n)\) of the \hyperref[def:local-coordinate-map]{local coordinate map} \(\varphi \) from \(\mathbb{R} ^n\) is called the \emph{local coordinates} on \(U\), given by
		\[
			\varphi (p) = (x^1(p), \ldots , x^n(p)).
		\]
	\end{definition}
	\begin{definition}[Coordinate chart]\label{def:coordinate-chart}
		The pair \((U, \varphi )\) is called a \emph{(coordinate) chart} on \(M\).
	\end{definition}
\end{definition}

In other words, a \hyperref[def:topological-manifold]{topological manifold} can be thought of as a space such that it looks like \(\mathbb{R} ^n\) locally.

\begin{center}
	\incfig{coordinate-chart}
\end{center}

\begin{definition}[Atlas]\label{def:atlas}
	An \emph{atlas} \(\mathcal{A} = \left\{ (U_\alpha , \varphi _\alpha) \right\}_\alpha \) for a \hyperref[def:topological-manifold]{manifold} \(\mathcal{M}\) is a collection of \hyperref[def:coordinate-chart]{charts} such that \(\left\{ U_\alpha\subseteq \mathcal{M} \mid U_\alpha \text{ open} \right\} _\alpha \) are an open covering of \(\mathcal{M} \), i.e., \(\mathcal{M} = \bigcup_{\alpha } U_\alpha \).
\end{definition}

In other words, for all \(p\in \mathcal{M} \), there exists a neighborhood \(U \subseteq \mathcal{M} \) and homeomorphism \(h\colon U \to U^\prime \subseteq \mathbb{R} ^n\) open.

\begin{definition}[Locally finite]\label{def:locally-finite}
	An \hyperref[def:atlas]{atlas} is said to be \emph{locally finite} if each point \(p\in \mathcal{M} \) is contained in only a finite collection of its open sets.
\end{definition}

Clearly, without any help of ambient space such as \(\mathbb{R} ^n\), there's no clear way to make sense of differentiability of a \hyperref[def:topological-manifold]{manifold}. But thankfully, we now have an explicit relation to the ambient space \(\mathbb{R} ^n\) via \(\varphi _\alpha \). To formalize, let \(\mathcal{A} \) be an \hyperref[def:atlas]{atlas} for a \hyperref[def:topological-manifold]{manifold} \(\mathcal{M} \), and assume that \((U_1, \varphi _1), (U_2, \varphi _2)\) are \(2\) elements of \(\mathcal{A} \). Then clearly, the map \(\varphi _2 \circ \varphi _1 ^{-1} \colon \varphi _1(U_1 \cap U_2) \to \varphi _2(U_1 \cap U_2)\) is a homeomorphism between \(2\) open sets of Euclidean spaces since both \(\varphi _1\) and \(\varphi _2\) are homeomorphism. Due to this map's importance, it has its own name.

\begin{definition}[Coordinate transition]\label{def:coordinate-transition}
	The map \(\varphi _2 \circ \varphi _1 ^{-1} \) is called the \emph{coordinate transition} of \(\mathcal{A} \) for the pair of \hyperref[def:coordinate-chart]{charts} \((U_1, \varphi _1), (U_2, \varphi _2)\).
\end{definition}

\begin{center}
	\incfig{coordinate-transition}
\end{center}

\subsection{Differentiable Structures}
Notice that the \hyperref[def:coordinate-transition]{coordinate transitions} are from \(\mathbb{R} ^n\) to \(\mathbb{R} ^n\); hence differentiability makes sense now, which induces the following.

\begin{definition}[Differentiable atlas]\label{def:differentiable-atlas}
	The \hyperref[def:atlas]{atlas} \(\mathcal{A} = \left\{ (U_\alpha , \varphi _\alpha) \right\} \) is \emph{differentiable} if all \hyperref[def:coordinate-transition]{transitions} are differentiable.
\end{definition}

\begin{remark}
	Here, the differentiability depends on the content. Sometimes, we may want it to be \(C^{\infty} \), and sometimes may be \(C^k\) for some finite \(k\). On the other hand, smooth always refers to \(C^{\infty} \). We'll use them interchangeably if it's clear which case we're referring to.
\end{remark}

\begin{definition}[Equivalence atlas]\label{def:equivalence-atlas}
	Two \hyperref[def:atlas]{atlases} \(\mathcal{U} , \mathcal{V} \) of a \hyperref[def:topological-manifold]{manifold} are equivalent if for every \((U, \varphi)\in \mathcal{U} \), \((V, \psi)\in \mathcal{V} \),
	\[
		\varphi \circ \psi ^{-1} \colon \psi(U \cap V) \to \varphi (U \cap V)
	\]
	and
	\[
		\psi \circ \varphi ^{-1} \colon \varphi (U \cap V) \to \psi(U \cap V)
	\]
	are diffeomorphisms between subsets of Euclidean spaces.
\end{definition}

Notably, we have the following notation.

\begin{notation}[Smoothly compatible]\label{not:smoothly-compatible}
	Two \hyperref[def:coordinate-chart]{charts} \((U, \varphi )\) and \((V, \psi)\) are \emph{smoothly compatible} if either \(U \cap V = \varnothing \) or \(\psi \circ \varphi ^{-1} \) is a diffeomorphism.
\end{notation}

This suggests the following.

\begin{definition}[Smooth structure]\label{def:smooth-structure}
	A \emph{smooth structure} on \(\mathcal{M} \) is an equivalence class \(\mathcal{U} \) of \hyperref[def:atlas]{coordinate atlas} with the property that all \hyperref[def:coordinate-transition]{transition functions} are diffeomorphisms.
\end{definition}

\begin{remark}
	In the literature, we can also use the \emph{maximal} \hyperref[def:differentiable-atlas]{differentiable atlas} to be our differentiable structure. Either way is fine.
\end{remark}

\begin{definition}[Smooth manifold]\label{def:smooth-manifold}
	A \emph{smooth manifold} is a \hyperref[def:topological-manifold]{manifold} \(\mathcal{M} \) with a \hyperref[def:smooth-structure]{smooth structure}.
\end{definition}

In this way, we can do calculus on smooth manifolds! Furthermore, it now makes sense to say that a function \(f\colon \mathcal{M} \to \mathbb{R} \) is differentiable (or \(C^{\infty} \)) by considering differentiability of \(f \circ \varphi ^{-1} \) around \(p\).

\begin{notation}
	The collection of smooth functions on \hyperref[def:smooth-manifold]{smooth manifold} \(\mathcal{M} \) is denoted by \(C^{\infty} (\mathcal{M} , \mathbb{R} )\), or \(C^k(\mathcal{M} , \mathbb{R} )\).
\end{notation}

\begin{remark}
	The class \(C^{\infty} (\mathcal{M} , \mathbb{R} )\) consists of functions with property is well-defined.
\end{remark}
\begin{explanation}
	Let \(\mathcal{A} \) be any given \hyperref[def:atlas]{atlas} from \hyperref[def:equivalence-atlas]{equivalence class} that defines the \hyperref[def:smooth-structure]{smooth structure}, and as we have shown, if \((U, \varphi)\in \mathcal{A} \), then \(f \circ \varphi ^{-1} \) is a smooth function on \(\mathbb{R} ^n\). This requirement defines the same set of smooth functions no matter the choice of representative \hyperref[def:atlas]{atlas} by the nature of \autoref{def:equivalence-atlas} requirement that defines the equivalent \hyperref[def:smooth-manifold]{manifolds}.
\end{explanation}

\subsection{Orientation}
Another essential property of a \hyperref[def:topological-manifold]{manifold} is its orientability.

\begin{definition*}
	Consider an \hyperref[def:atlas]{atlas} \(\mathcal{A} \) for a \hyperref[def:smooth-manifold]{differentiable manifold} \(\mathcal{M} \).
	\begin{definition}[Oriented]\label{def:oriented}
		\(\mathcal{A} \) is \emph{oriented} if all \hyperref[def:coordinate-transition]{transitions} have positive functional determinant.
	\end{definition}

	\begin{definition}[Orientable]\label{def:orientable}
		\(\mathcal{M} \) is \emph{orientable} if \(\mathcal{A} \) is an \hyperref[def:oriented]{oriented atlas}.
	\end{definition}
\end{definition*}

Motivated by the above definitions, we see that we can actually use an \hyperref[def:atlas]{atlas} to define an \hyperref[def:orientation]{orientation}.

\begin{definition}[Orientation]\label{def:orientation}
	Let \(\mathcal{M} \) be an \hyperref[def:orientable]{orientable manifold}. Then a \hyperref[def:oriented]{oriented} \hyperref[def:smooth-structure]{differentiable structure} is called an \emph{orientation} of \(\mathcal{M} \).
\end{definition}

If \(\mathcal{M} \) possesses an \hyperref[def:orientation]{orientation}, we can also say that it's \emph{oriented}. But we don't bother to make a new definition to confuse ourselves with \autoref{def:oriented}.

\begin{remark}
	Two \hyperref[def:smooth-structure]{differentiable structures} obeying \autoref{def:oriented} determine the same \hyperref[def:orientation]{orientation} if the union again satisfying \autoref{def:oriented}.
\end{remark}

\begin{remark}
	If \(\mathcal{M} \) is \hyperref[def:orientable]{orientable} and connected, then there exists exactly \(2\) distinct \hyperref[def:orientation]{orientations} on \(\mathcal{M} \).
\end{remark}

Now, we can see some examples of \hyperref[def:smooth-manifold]{smooth manifolds}.

\begin{eg}[Sphere]
	The sphere \(S^n \subseteq \mathbb{R} ^{n+1}\) given by
	\[
		S^n = \left\{ (x_1, \ldots , x_{n+1} )\in \mathbb{R} ^{n+1} \mid x_1^2 + \ldots + x_{n+1}^2 = 1 \right\}.
	\]
	Consider \(U_i^+ = \left\{ x\in S^n \mid x_i > 0 \right\} \), \(U_i^-=\left\{ x\in S^n \mid x_i < 0 \right\} \) for \(i = 1, \ldots , n+1\), and \(h_i^{\pm} \colon U_i^{\pm} \to \mathbb{R} ^n\) such that
	\[
		h_i^{\pm}(x_1, \ldots , x_{n+1}) = (x_1, \ldots , \hat{x} _i, .., x_{n+1}).
	\]
	Note that the minimum charts needed to cover \(S^n\) is \(2\).
\end{eg}

\begin{eg}
	Let \(\mathcal{M} = U \subseteq \mathbb{R} ^n\), then \(\left\{ (U, \varphi ) \right\} \) is a \hyperref[def:smooth-structure]{smooth structure} with \(\varphi = \mathbbm{1}\).
\end{eg}

\begin{eg}
	Open sets of \(C^{\infty} \)-\hyperref[def:smooth-manifold]{manifolds} are \(C^{\infty}\)-\hyperref[def:smooth-manifold]{manifolds}.
\end{eg}

\begin{eg}[General linear group]
	\(\mathrm{GL} (n) = \left\{ A\in M_n(\mathbb{R} ) \mid \det A \neq 0 \right\} \subseteq M_n(\mathbb{R} ) = \mathbb{R} ^{n^2}\), open.
\end{eg}

\begin{eg}[Real projective space]
	\(\mathbb{R} P^n = \quotient{S^n}{\sim } \) where \(x \sim -x\) with \(\pi \colon S^n \to \mathbb{R} P^n\), \(x \mapsto [x]\).
\end{eg}
\begin{explanation}
	\(\pi \) is a homeomorphism on each \(U_i^+\) for \(i=1, \ldots , n+1\), with
	\[
		\left\{ \big( \pi (U_i^+), \varphi _i^+ \circ \pi ^{-1}  \big), i=1, \ldots , n+1 \right\}
	\]
	is a \(C^{\infty} \)-\hyperref[def:differentiable-atlas]{atlas} for \(\mathbb{R} P^n\).
\end{explanation}

\begin{note}
	Observe that \(\mathbb{R} P^n = \quotient{\mathbb{R} ^{n+1} \setminus \left\{ 0 \right\} }{\sim } \).
\end{note}

\begin{eg}[Grassmannian manifold]
	Given \(m, n\), \(G(n, m)\) is the set of all \(n\)-dimensional subspaces of \(\mathbb{R} ^{n+m}\).
\end{eg}