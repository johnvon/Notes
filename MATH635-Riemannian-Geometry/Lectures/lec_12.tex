\lecture{12}{14 Feb. 14:30}{Tangent and Cotangent Bundles}
\section{More on Tangent and Cotangent Bundles}
\subsection{Preliminaries}
Let \(f\colon \mathcal{M} \to \mathcal{N} \) be a differentiable map between two \hyperref[def:smooth-manifold]{differentiable manifolds}, and let \((E, \pi , \mathcal{N} )\)  be a \hyperref[def:vector-bundle]{vector bundle} over \(\mathcal{N} \). We want to pull back the \hyperref[def:vector-bundle]{bundle} via \(f\), i.e., construct a \hyperref[def:vector-bundle]{bundle} \(f^{\ast} \in E\)for which the \hyperref[def:fiber]{fiber} over \(x\in \mathcal{M} \) is \(E_{f(x)}\):

\begin{definition}[Pullback]\label{def:pullback}
	The \emph{pullback} \hyperref[def:vector-bundle]{bundle} \(f^{\ast} E\) is the \hyperref[def:vector-bundle]{bundle} over \(\mathcal{M} \) with the \hyperref[def:bundle-chart]{bundle charts} \(\varphi \circ f, f^{-1} (U)\) with \((\varphi , U)\) being the \hyperref[def:bundle-chart]{bundle charts} of \(E\).
\end{definition}

\begin{definition}[Bundle homomorphism]\label{def:bundle-homomorphism}
	Consider \(2\) \hyperref[def:vector-bundle]{vector bundles} \((E_1, \pi _1, \mathcal{M} ), (E_2, \pi _2, \mathcal{M} )\) over \(\mathcal{M} \), and let the differentiable map \(f\colon E_1 \to E_2\) be \hyperref[def:fiber]{fiber} preserving, i.e., \(\pi _2 \circ f = \pi _1\). Let the \hyperref[def:fiber]{fiber} maps \(f_x \colon E_{1,x} \to E_{2,x}\) be linear.\footnote{I.e., vector homomorphisms.} Then \(f\) is called a \emph{bundle homomorphism}.
\end{definition}

\begin{definition}[Subbundle]\label{def:subbundle}
	Let \((E, \pi , \mathcal{M} )\) of rank \(n\) be a \hyperref[def:vector-bundle]{vector bundle}. Let \(E^1 \subseteq E\), and assume that for all \(x\in \mathcal{M} \), there exists a \hyperref[def:bundle-chart]{bundle chart} \((\varphi , U)\) for \(x\in U\) and \(\varphi (\pi ^{-1} (U) \cap E^1) = U \times \mathbb{R} ^m \subseteq U \times \mathbb{R} ^n\) for \(m \leq n\). The so constructed \hyperref[def:vector-bundle]{vector bundle} \((E^1, \at{\pi }{E^1}{}, \mathcal{M} )\) is called the \emph{subbundle} of \(E\) of rank \(m\).
\end{definition}

\subsection{Identifications}
Let \(\mathcal{M} \) be a \hyperref[def:smooth-manifold]{differentiable manifold}.

\begin{prev}
	The elements of \(T \mathcal{M} \) is \((x, V)\) with \(x\in \mathcal{M} \), \(V\) the \hyperref[def:tangent-vector]{tangent vector} to \(\mathcal{M} \) at \(x\). Also, the \hyperref[def:tangent-bundle]{tangent bundle} is the \hyperref[def:section]{section} of \(T \mathcal{M} \).

	Also, the elements of \(T*8\mathcal{M} \) are called \hyperref[def:cotangent-vector]{cotangent vector}, i.e., linear functionals \(\alpha \in T_x ^{\ast} \mathcal{M} \) such that \(\alpha \colon T_x \mathcal{M} \to \mathbb{R} \). Then, the \hyperref[def:section]{section} of \(T^{\ast} \mathcal{M} \) are \(1\)-forms, i.e., \(\alpha \colon T \mathcal{M} \to \mathbb{R} \), \(\alpha _x \colon \at{\alpha }{T_x \mathcal{M} }{} \to \mathbb{R} \).
\end{prev}

A \hyperref[def:Riemannian-metric]{Riemannian metric} induces bundle metrics on all tensor bundles over \(\mathcal{M} \). Metric of \(T^{\ast} \mathcal{M} \) given by
\[
	\left\langle \omega , \eta \right\rangle = g(\omega , \eta ) = g^{ij} \omega _i \eta _i
\]
where \(\omega = \omega _i \mathrm{d} x^i\), \(\eta = \eta _i \mathrm{d} x^i\). The identification between \(T\mathcal{M} \) such that \(T^{\ast} \mathcal{M} \) through \hyperref[def:Riemannian-metric]{Riemannian metric} is given by
\[
	V = V^i \frac{\partial }{\partial x^i} \in T \mathcal{M} \text{ corresponds to } \omega = \omega _j \mathrm{d} x^j \in T^{\ast} \mathcal{M}
\]
with \(\omega _j = g_{ij} V^i\) or \(V^i = g^{ij} \omega _j\) with
\begin{enumerate}[(a)]
	\item \(g(X, Y) = g_{ij}X^i Y^j \) for \(X, Y\in T \mathcal{M} \).
	\item \(g(\omega , \eta ) = g^{ij} \omega _i \eta _j\) for \(\omega , \eta \in T^{\ast} \mathcal{M} \).
\end{enumerate}

Then, for \(V\in T_x \mathcal{M} \), there corresponds a \(1\)-form \(\omega \in T_x ^{\ast} \mathcal{M} \) via the metric \(\omega (Y) \coloneqq g(V, Y)\) for all \(Y\), and \(\lVert \omega \rVert = \lVert V \rVert \). Let \((e_i)_{i=1, \ldots , d}\) be a basis of \(T_x \mathcal{M} \) and \((\omega ^j)_{j=1, \ldots , d}\) the dual basis of \(T_x ^{\ast} \mathcal{M} \), i.e., \(w^j(e_i)=\delta _i^j\). Let \(V = V^i e_i \in T_x \mathcal{M} \), \(\eta =\eta _j \omega ^j \in T_x^{\ast} \mathcal{M} \), then \(\eta (V) = \eta _i V^i\).

Consider basis \((e_i), (\omega ^j)\) in the \hyperref[def:coordinate-chart]{local coordinates}: \(e_i = \partial / \partial x^i\) and \(\omega ^j = \mathrm{d} x^j\). Let \(f\) be a \hyperref[def:coordinate-chart]{local coordinates} change. Then \(V\) transform as
\[
	f_{\ast }(V) \coloneqq V^i \frac{\partial f^\alpha }{\partial x^i} \frac{\partial }{\partial f^\alpha } ;
\]
and \(\eta \) transforms as
\[
	f^{\ast} (\eta ) \coloneqq \eta _j \frac{\partial x^j}{\partial f^\beta } \mathrm{d} f^\beta .
\]
Hence,
\[
	f^{\ast} (\eta )(f_{\ast } (V))
	= \eta _j \frac{\partial x^j}{\partial f^\alpha } V^i \frac{\partial f^\alpha }{\partial x^i}
	= \eta _i V^i = \eta (V).
\]

\begin{intuition}
	The above means that
	\begin{itemize}
		\item the \hyperref[def:tangent-vector]{tangent vectors} transform with the functional matrix of \hyperref[def:coordinate-chart]{coordinates} change;
		\item the \hyperref[def:cotangent-vector]{cotangent vectors} transform with the transposed inverse of the above matrix.
	\end{itemize}
\end{intuition}

To compute the \hyperref[def:coordinate-chart]{coordinates} change \(y \mapsto x(y)\) for \(\omega = \omega _i \mathrm{d} x^i\), \(\eta = \eta _i \mathrm{d} x^i\) with \(\left\langle \omega , \eta \right\rangle = g^{ij} \omega _i \eta _j\), we have
\[
	\omega _i \mathrm{d} x^i
	= \omega _i \frac{\partial x^i}{\partial y^\alpha } \mathrm{d} y^\alpha
	\eqqcolon \widetilde{w} _\alpha \mathrm{d} y^\alpha,
\]
and \(g^{ij} \) is transformed as
\[
	h^{\alpha \beta } = g^{ij} \frac{\partial y^\alpha }{\partial x^i} \frac{\partial y^\beta }{\partial x^j}
\]
and \(h^{\alpha \beta } \widetilde{w} _\alpha \widetilde{\eta} _\beta = g^{ij} \omega _i \eta _j \) and
\[
	\lVert \omega (x) \rVert = \sup \left\{ \omega (x) (V) \mid V\in T_x \mathcal{M} , \lVert v \rVert = 1\right\}.
\]

Consider \(T \mathcal{M} \otimes T\mathcal{M} \), the metric is
\[
	\left\langle V \otimes Y, \xi \otimes \eta \right\rangle = g_{ij} V^i Y^i g_{k \ell }\omega ^k \eta ^\ell .
\]

\begin{definition}[Lie derivative]\label{def:Lie-derivative}
	Consider a \hyperref[def:vector-field]{vector field} \(X\) with a \hyperref[def:local-1-parameter-group]{local \(1\)-parameter group} \((\psi _t)_{t\in I}\) of local \hyperref[def:diffeomorphism]{diffeomorphism}, and a tensor field \(S\) on \(\mathcal{M} \). The \emph{Lie derivative} of \(S\) in the direction of \(X\) is defined as
	\[
		L_x S \coloneqq \at{\frac{\mathrm{d}}{\mathrm{d}t} (\psi _t^{\ast} S)}{t=0}{} .
	\]
\end{definition}

\begin{definition}[Pushforward]\label{def:push-forward}
	Let \(\psi \colon \mathcal{M} \to \mathcal{N} \) be a \hyperref[def:diffeomorphism]{diffeomorphism} between two \hyperref[def:smooth-manifold]{differentiable manifolds}. Let \(X\) be a \hyperref[def:vector-field]{vector field} on \(\mathcal{M} \). Then define a \hyperref[def:vector-field]{vector field} \(Y = \psi _\ast X\) on \(\mathcal{N} \) be by \(Y(p) = \mathrm{d} \psi (X(\psi ^{-1} (p)))\).
\end{definition}

\begin{lemma}
	The following holds.
	\begin{enumerate}
		\item For every differentiable function \(f\colon \mathcal{N} \to \mathbb{R} \), then \((\psi _\ast X) (f)(p) = X(f\circ \psi ) (\psi ^{-1} p)\).
		\item Let \(X\) be a \hyperref[def:vector-field]{vector field} on \(\mathcal{M} \), \(\psi \colon \mathcal{M} \to \mathcal{N} \) be a \hyperref[def:diffeomorphism]{diffeomorphism}. If the \hyperref[def:local-1-parameter-group]{local \(1\)-parameter group} generated by \(X\) given by \(\varphi _t\), then the local group generated by \(\psi _\ast X\) is \(\psi \circ \varphi _t \circ \psi ^{-1} \).
	\end{enumerate}
\end{lemma}