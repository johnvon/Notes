\lecture{12}{14 Feb. 14:30}{Tangent and Cotangent Bundles}
\section{More on Tangent and Cotangent Bundles}
Let \(f\colon \mathcal{M} \to \mathcal{N} \) be a differentiable map between two \hyperref[def:smooth-manifold]{differentiable manifolds}, until now, we have only talked about how to transform \hyperref[def:tangent-vector]{tangent vectors} or \(1\)-form via \(f\). Implicitly, these are just \hyperref[def:pullback-one-form]{pullback} (\(f^{\ast} \)) and \hyperref[def:pushforward-tangent-vector]{pushforward} (\(f_{\ast}\)), as we now define formally.

\begin{definition*}
	Let \(f\colon \mathcal{M} \to \mathcal{N} \) be a smooth map between two \hyperref[def:smooth-manifold]{smooth manifolds} and \(p\in \mathcal{M} \).
	\begin{definition}[Pushforward]\label{def:pushforward-tangent-vector}
		The \emph{pushforward} is the linear map \(f_{\ast} \coloneqq \mathrm{d} f_p \colon T_p \mathcal{M} \to T_{f(p)} \mathcal{N}\).
	\end{definition}

	\begin{definition}[Pullback]\label{def:pullback-one-form}
		The \emph{pullback} is the linear map \(f^{\ast} \colon T^{\ast} _{f(p)} \mathcal{N} \to T^{\ast} _p \mathcal{M} \) where
		\[
			(f^{\ast} \omega )(X) = \omega (f_\ast X)
		\]
		for \(\omega \in T_{f(p)}^{\ast} \mathcal{N} \) and \(X\in T_p \mathcal{M} \).
	\end{definition}
\end{definition*}

In all, the following diagram commutes:
% https://q.uiver.app/?q=WzAsOCxbMywwLCJUX3BcXG1hdGhjYWx7TX0iXSxbNCwwLCJUX3BcXG1hdGhjYWx7Tn0iXSxbNCwxLCJcXG1hdGhjYWx7Tn0iXSxbMywxLCJcXG1hdGhjYWx7TX0iXSxbMSwwLCJUXipfcFxcbWF0aGNhbHtOfSJdLFswLDAsIlReKl9wXFxtYXRoY2Fse019Il0sWzEsMSwiXFxtYXRoY2Fse059Il0sWzAsMSwiXFxtYXRoY2Fse019Il0sWzAsMywiXFxwaSIsMl0sWzEsMiwiXFxwaSJdLFswLDEsImZfXFxhc3QiXSxbMywyLCJmIiwyXSxbNCw1LCJmXlxcYXN0IiwyXSxbNCw2LCJcXHBpIl0sWzUsNywiXFxwaSIsMl0sWzcsNiwiZiIsMl1d
\[
	\begin{tikzcd}
		{T^*_p\mathcal{M}} & {T^*_p\mathcal{N}} && {T_p\mathcal{M}} & {T_p\mathcal{N}} \\
		{\mathcal{M}} & {\mathcal{N}} && {\mathcal{M}} & {\mathcal{N}}
		\arrow["\pi"', from=1-4, to=2-4]
		\arrow["\pi", from=1-5, to=2-5]
		\arrow["{f_\ast}", from=1-4, to=1-5]
		\arrow["f"', from=2-4, to=2-5]
		\arrow["{f^\ast}"', from=1-2, to=1-1]
		\arrow["\pi", from=1-2, to=2-2]
		\arrow["\pi"', from=1-1, to=2-1]
		\arrow["f"', from=2-1, to=2-2]
	\end{tikzcd}
\]

\subsection{Pullbacks and Pushforwards on Bundles}
Now, consider a \hyperref[def:vector-bundle]{vector bundle} \((E, \pi , \mathcal{N} )\) over \(\mathcal{N} \), we want to use \(f\) to ``pull back'' the \hyperref[def:vector-bundle]{vector bundle}, i.e., construct a \hyperref[def:vector-bundle]{vector bundle}, denote as \(f^{\ast} E\), for which the \hyperref[def:fiber]{fiber} over \(x\in \mathcal{M} \) is \(E_{f(x)}\).

\begin{definition}[Pullback bnudle]\label{def:pullback-bundle}
	The \emph{pullback bundle} \(f^{\ast} E\) is the \hyperref[def:vector-bundle]{vector bundle} over \(\mathcal{M} \) with the \hyperref[def:bundle-chart]{bundle charts} \((\varphi \circ f, f^{-1} (U))\) if \((\varphi , U)\) is the \hyperref[def:bundle-chart]{bundle charts} of \(E\).
\end{definition}

Similarly, we can ``push forward'' a \hyperref[def:vector-bundle]{vector bundle} \((E, \pi , \mathcal{M} )\) over \(\mathcal{M} \) via \(f\) in the same fashion.

\begin{definition}[Pushforward bnudle]\label{def:pushforward-bundle}
	The \emph{pushforward bundle} \(f_{\ast} E\) is the \hyperref[def:vector-bundle]{vector bundle} over \(\mathcal{N} \) with the \hyperref[def:bundle-chart]{bundle charts} \((\varphi \circ f^{-1} , f (U))\) if \((\varphi , U)\) is the \hyperref[def:bundle-chart]{bundle charts} of \(E\).
\end{definition}

\begin{note}
	In \autoref{def:pushforward-bundle}, it only makes sense if \(\mathcal{M} \hookrightarrow \mathcal{N} \).
\end{note}

\begin{definition}[Bundle homomorphism]\label{def:bundle-homomorphism}
	Consider \(2\) \hyperref[def:vector-bundle]{vector bundles} \((E_1, \pi _1, \mathcal{M} ), (E_2, \pi _2, \mathcal{M} )\) over \(\mathcal{M} \), and let the differentiable map \(f\colon E_1 \to E_2\) be \hyperref[def:fiber]{fiber} preserving, i.e., \(\pi _2 \circ f = \pi _1\). If the \hyperref[def:fiber]{fiber} maps \(f_x \colon E_{1,x} \to E_{2,x}\) is linear,\footnote{I.e., vector homomorphisms.} then \(f\) is called a \emph{bundle homomorphism}.
\end{definition}

\begin{definition}[Subbundle]\label{def:subbundle}
	Let \((E, \pi , \mathcal{M} )\) of rank \(n\) be a \hyperref[def:vector-bundle]{vector bundle}. Let \(E^1 \subseteq E\), and assume that for all \(x\in \mathcal{M} \), there exists a \hyperref[def:bundle-chart]{bundle chart} \((\varphi , U)\) for \(x\in U\) and
	\[
		\varphi (\pi ^{-1} (U) \cap E^1) = U \times \mathbb{R} ^m \subseteq U \times \mathbb{R} ^n
	\]
	for \(m \leq n\). Then the \emph{subbundle} of \(E\) of rank \(m\) is the \hyperref[def:vector-bundle]{vector bundle} \((E^1, \at{\pi }{E^1}{}, \mathcal{M} )\).
\end{definition}

\begin{eg}
	Consider \(f\colon \mathcal{M} \hookrightarrow \mathcal{N} \) where \(g_{\mathcal{N} } \) is a \hyperref[def:pseudo-Riemannian-metric]{metric} on \(\mathcal{N} \). Then, \(g_{\mathcal{N} } \) induces a \hyperref[def:pseudo-Riemannian-metric]{metric} \(g_{\mathcal{M} } \) on \(\mathcal{M} \) by \(f\) since we can define
	\[
		g_{\mathcal{M} } (X, Y) \coloneqq g_{\mathcal{N} } (f_{\ast} (X), f_{\ast} (Y)).
	\]
\end{eg}

\subsection{Pullbacks and Pushforwards of Vector Fields}
Now, we consider to ``pull back'' or ``push forward'' a \hyperref[def:vector-field]{vector field}, i.e., a \hyperref[def:section]{section} of a \hyperref[def:bundle]{bundle}.

\begin{definition}[Pushforward]\label{def:pushforward}
	Let \(\psi \colon \mathcal{M} \to \mathcal{N} \) be a \hyperref[def:diffeomorphism]{diffeomorphism} between \hyperref[def:smooth-manifold]{smooth manifolds}, and let \(X\) be a \hyperref[def:vector-field]{vector field} on \(\mathcal{M} \). Then the \emph{pushforward} \hyperref[def:vector-field]{vector field} \(Y = \psi _\ast X = \mathrm{d} \psi X\) on \(\mathcal{N} \) is
	\[
		Y(p) = \mathrm{d} \psi (X(\psi ^{-1} (p))).
	\]
\end{definition}

\begin{definition}[Pullback]\label{def:pullback}
	Let \(\psi \colon \mathcal{M} \to \mathcal{N} \) be a \hyperref[def:diffeomorphism]{diffeomorphism} between \hyperref[def:smooth-manifold]{smooth manifolds}, and let \(Y\) be a \hyperref[def:vector-field]{vector field} on \(\mathcal{N} \). Then the \emph{pullback} \hyperref[def:vector-field]{vector field} \(X = \psi ^\ast Y\) on \(\mathcal{M} \) is just \(X(p) = Y_{\psi (p)}\).
\end{definition}

\begin{note}
	We let \(\psi \) be a \hyperref[def:diffeomorphism]{diffeomorphism} just for convenient: we can also consider a \hyperref[def:vector-field-along-curve]{vector fields along curve} when \(\psi \) injects/surjects.
\end{note}

\begin{lemma}
	For every differentiable function \(f\colon \mathcal{N} \to \mathbb{R} \), \((\psi _\ast X) (f)(p) = X(f\circ \psi ) (\psi ^{-1} p)\).
\end{lemma}

\begin{lemma}
	Let \(X\) be a \hyperref[def:vector-field]{vector field} on \(\mathcal{M} \) and \(\psi \colon \mathcal{M} \to \mathcal{N} \) be a \hyperref[def:diffeomorphism]{diffeomorphism}. If the \hyperref[def:local-1-parameter-group]{local \(1\)-parameter group} \((\varphi _t)_{t\in I}\) generated by \(X\), then the \hyperref[def:local-1-parameter-group]{local \(1\)-parameter group} generated by \(\psi _\ast X\) is \(\psi \circ \varphi _t \circ \psi ^{-1} \).
\end{lemma}

\subsection{Induced Bundle Metrics}
Let \((\mathcal{M} , g)\) be a \hyperref[def:Riemannian-manifold]{Riemannian manifold}, then \(g\) induces the \emph{bundle metrics} on all \hyperref[def:vector-bundle]{vector bundles} over \(\mathcal{M} \): for \(T^{\ast} \mathcal{M} \), it is given by
\[
	g(\omega , \eta ) \coloneqq g^{ij} \omega _i \eta _i
\]
for \(\omega = \omega _i \mathrm{d} x^i, \eta = \eta _i \mathrm{d} x^i\). Hence, we can talk about the identification between \(T\mathcal{M} \) and \(T^{\ast} \mathcal{M} \) through \(g\):
% https://q.uiver.app/?q=WzAsMixbMCwwLCJWID0gVl5pIFxcZnJhY3tcXHBhcnRpYWwgfXtcXHBhcnRpYWwgeF5pfSBcXGluIFQgXFxtYXRoY2Fse019Il0sWzAsMSwiXFxvbWVnYSA9IFxcb21lZ2EgX2ogXFxtYXRocm17ZH0geF5qIFxcaW4gVF57XFxhc3R9IFxcbWF0aGNhbHtNfSJdLFswLDEsIiIsMCx7InN0eWxlIjp7InRhaWwiOnsibmFtZSI6ImFycm93aGVhZCJ9fX1dXQ==
\[
	\begin{tikzcd}
		{V = V^i \frac{\partial }{\partial x^i} \in T \mathcal{M}} \\
		{\omega = \omega _j \mathrm{d} x^j \in T^{\ast} \mathcal{M}}
		\arrow[tail reversed, Leftrightarrow, from=1-1, to=2-1]
	\end{tikzcd}
\]
with \(\omega _j = g_{ij} V^i\) (or \(V^i = g^{ij} \omega _j\)) such that
\begin{enumerate}[(a)]
	\item \(g(X, Y) = g_{ij}X^i Y^j \) for \(X, Y\in T \mathcal{M} \);
	\item \(g(\omega , \eta ) = g^{ij} \omega _i \eta _j\) for \(\omega , \eta \in T^{\ast} \mathcal{M} \).
\end{enumerate}

Thus, for \(V\in T_x \mathcal{M} \), there corresponds a \(1\)-form \(\omega \in T_x ^{\ast} \mathcal{M} \) via the metric \(\omega (Y) \coloneqq g(V, Y)\) for all \(Y\), and we further have \(\lVert \omega \rVert = \lVert V \rVert \).

We can also consider the \hyperref[def:local-coordinate]{coordinate} transformation behavior. Let \((e_i)_{i=1, \ldots , d}\) be a basis of \(T_x \mathcal{M} \) and \((\omega ^j)_{j=1, \ldots , d}\) the dual basis of \(T_x ^{\ast} \mathcal{M} \), i.e., \(w^j(e_i)=\delta _i^j\). Given \(V = V^i e_i \in T_x \mathcal{M} \), \(\eta =\eta _j \omega ^j \in T_x^{\ast} \mathcal{M} \), we then have \(\eta (V) = \eta _i V^i\). Now, consider bases \((e_i), (\omega ^j)\) in the \hyperref[def:coordinate-chart]{local coordinates}, i.e., \(e_i = \partial / \partial x^i\) and \(\omega ^j = \mathrm{d} x^j\). Let \(f\) be a \hyperref[def:coordinate-chart]{local coordinates} change, then \(V\) and \(\eta \) transformed as
\[
	f_{\ast }(V) \coloneqq V^i \frac{\partial f^\alpha }{\partial x^i} \frac{\partial }{\partial f^\alpha },\qquad
	f^{\ast} (\eta ) \coloneqq \eta _j \frac{\partial x^j}{\partial f^\beta } \mathrm{d} f^\beta
\]
correspondingly, and we see that
\[
	f^{\ast} (\eta )(f_{\ast } (V))
	= \eta _j \frac{\partial x^j}{\partial f^\alpha } V^i \frac{\partial f^\alpha }{\partial x^i}
	= \eta _i V^i = \eta (V).
\]

\begin{intuition}
	The above means that
	\begin{itemize}
		\item the \hyperref[def:tangent-vector]{tangent vectors} transform with the functional matrix of \hyperref[def:coordinate-chart]{coordinates} change;
		\item the \hyperref[def:cotangent-vector]{cotangent vectors} transform with the transposed inverse of the above matrix.
	\end{itemize}
\end{intuition}

To compute the \hyperref[def:coordinate-chart]{coordinates} change \(y \mapsto x(y)\) for \(\omega = \omega _i \mathrm{d} x^i\), \(\eta = \eta _i \mathrm{d} x^i\) with \(\left\langle \omega , \eta \right\rangle = g^{ij} \omega _i \eta _j\), we have
\[
	\omega _i \mathrm{d} x^i
	= \omega _i \frac{\partial x^i}{\partial y^\alpha } \mathrm{d} y^\alpha
	\eqqcolon \widetilde{w} _\alpha \mathrm{d} y^\alpha.
\]

\begin{prev}
	\(g^{ij} \) is transformed as
	\[
		h^{\alpha \beta } = g^{ij} \frac{\partial y^\alpha }{\partial x^i} \frac{\partial y^\beta }{\partial x^j}.
	\]
\end{prev}

Then, we see that \(h^{\alpha \beta } \widetilde{w} _\alpha \widetilde{\eta} _\beta = g^{ij} \omega _i \eta _j \) and \(\lVert \omega (x) \rVert = \sup \left\{ \omega (x) (V) \mid V\in T_x \mathcal{M} , \lVert v \rVert = 1\right\}\).

\begin{remark}
	If we consider \(T \mathcal{M} \otimes T\mathcal{M} \), then metric is
	\[
		\left\langle V \otimes Y, \xi \otimes \eta \right\rangle = g_{ij} V^i Y^i g_{k \ell }\omega ^k \eta ^\ell .
	\]
\end{remark}

\begin{prev}[Lie derivative]
	Consider a \hyperref[def:vector-field]{vector field} \(X\) with a \hyperref[def:local-1-parameter-group]{local \(1\)-parameter group} \((\psi _t)_{t\in I}\) and a \hyperref[def:tensor-field]{tensor field} \(S\) on \(\mathcal{M} \). The \hyperref[def:Lie-derivative]{Lie derivative} of \(S\) in the direction of \(X\) is defined as
	\[
		\mathcal{L} _X S \coloneqq \at{\frac{\mathrm{d}}{\mathrm{d}t} (\psi _t^{\ast} S)}{t=0}{} .
	\]
\end{prev}