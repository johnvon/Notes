\lecture{2}{10 Jan.\ 13:00}{Maps Between Smooth Manifolds}
\subsection{Smooth Maps}
We can now consider the maps between \hyperref[def:topological-manifold]{manifolds}, specifically, the \hyperref[def:smooth-manifold]{smooth manifolds}.

\begin{definition}[Smooth function]\label{def:smooth-function}
	Let \(\mathcal{M} , \mathcal{N} \) be two \hyperref[def:smooth-manifold]{smooth manifolds}, and let \(\mathcal{U}\) be \hyperref[def:locally-finite]{locally finite} \hyperref[def:atlas]{atlas} from the \hyperref[def:equivalence-atlas]{equivalence class} that gives the \hyperref[def:smooth-structure]{smooth structure} on \(\mathcal{M} \), and let \(\mathcal{V}\) be the corresponding for \(\mathcal{N} \). A map \(h\colon \mathcal{M} \to \mathcal{N} \) is said to be \emph{smooth} if each map in the collection
	\[
		\left\{ \psi \circ h\circ \varphi ^{-1} \colon h(U) \cap V \neq \varnothing \right\}
	\]
	where \((U, \varphi )\in \mathcal{U} \), \((V, \psi )\in \mathcal{V} \) is \(C^{\infty} \)-differentiable as a map from one Euclidean space to another.
\end{definition}

\begin{center}
	\incfig{maps-between-manifolds}
\end{center}

\begin{remark}
	\hyperref[def:equivalence-atlas]{Equivalence relation} guarantees that \autoref{def:smooth-function} depends only on the \hyperref[def:smooth-structure]{smooth structure} of \(M, N\), but not on the chosen representative \hyperref[def:atlas]{coordinate atlas}.
\end{remark}

\begin{definition*}
	Consider two \hyperref[def:smooth-manifold]{smooth manifolds} \(\mathcal{M} , \mathcal{N} \) and a \hyperref[def:smooth-function]{smooth} homeomorphism \(h\colon \mathcal{M} \to \mathcal{N} \) with \hyperref[def:smooth-function]{smooth} inverse.
	\begin{definition}[Diffeomorphic]\label{def:diffeomorphic}
		The two \hyperref[def:smooth-manifold]{manifolds} \(\mathcal{M} , \mathcal{N} \) are said to be \emph{diffeomorphic}.
	\end{definition}
	\begin{definition}[Diffeomorphism]\label{def:diffeomorphism}
		The map \(h\) is said to be a \emph{diffeomorphism}.
	\end{definition}
\end{definition*}

Let \(\mathcal{M} _1, \mathcal{M} _2\) be two \hyperref[def:smooth-manifold]{smooth manifolds}, and let \(\varphi \colon \mathcal{M} _1 \to \mathcal{M} _2\) be a \hyperref[def:diffeomorphism]{diffeomorphism}. Then
\begin{enumerate}[(a)]
	\item \(\mathcal{M} _1\) is \hyperref[def:orientable]{orientable} if and only if \(\mathcal{M} _2\) is \hyperref[def:orientable]{orientable}.
	\item If in addition, \(\mathcal{M} _1\) and \(\mathcal{M} _2\) are both connected and \hyperref[def:oriented]{oriented}, then \(\varphi \) induces an \hyperref[def:orientation]{orientation} on \(\mathcal{M} _2\) that may or may not coincide with the initial \hyperref[def:orientation]{orientation} of \(\mathcal{M} _2\).
\end{enumerate}

If the induced \hyperref[def:orientation]{orientation} coincides, then we say \(\varphi \) preserves the \hyperref[def:orientation]{orientation}, otherwise \(\varphi \) reverses the \hyperref[def:orientation]{orientation}.

\subsection{Grassmannian Manifold}
Before proceeding, let's consider an interesting \hyperref[def:smooth-manifold]{smooth manifold}.

\begin{definition}[Grassmannian manifold]\label{def:Grassmannian-manifold}
	Given \(m, n\in \mathbb{N} \), the so-called \emph{Grassmannian manifold} \(G(n, m)\) is the set of all \(n\)-dimensional subspaces of \(\mathbb{R} ^{n+m}\).
\end{definition}

\begin{note}
	\(G(1, m)\) is just \(\mathbb{R} P^m\), and \(G(0, m)\), \(G(n, 0)\) are one-point sets.
\end{note}

As we will soon see, \(G(n, m)\) has the \hyperref[def:smooth-structure]{smooth structure} of an \(mn\)-dimensional \hyperref[def:smooth-manifold]{manifold}.

\begin{intuition}
	We obtain the \hyperref[def:smooth-structure]{structure} by exhibiting an \hyperref[def:atlas]{atlas} whose \hyperref[def:coordinate-transition]{transitions} are \hyperref[def:diffeomorphism]{diffeomorphisms}.
\end{intuition}

Firstly, we give \(G(n, m)\) a suitable topology, i.e., the metric topology. Let \(\Pi \in G(n, m)\), and let \(\mathcal{L} (\Pi , \Pi ^{\perp})\) denote the \(mn\)-dimensional space of linear maps from \(\Pi\) to \(\Pi ^{\perp} \). Define the map
\[
	\varphi _\Pi \colon \mathcal{L} (\Pi , \Pi ^{\perp} ) \to G(n, m),\qquad
	\varphi _\Pi (\alpha ) = \left( \mathbbm{1}_{\Pi } \oplus \alpha \right) (\Pi )
\]
where \(\mathbbm{1}_{\Pi }\oplus \alpha \) is regarded as a map \(\Pi \to \Pi \oplus \Pi ^{\perp} = \mathbb{R} ^{n+m}\).\footnote{In other words, \(\varphi _\Pi (\alpha )\) is the graph of \(\alpha \) in \(\Pi \oplus \Pi ^{\perp} = \mathbb{R} ^{n+m}\).} Clearly, \(\varphi _\Pi \) is injective, and thus, \(\big(\mathcal{L} (\Pi , \Pi ^{\perp} ), \varphi _\Pi \big)\) is an \(mn\)-dimensional \hyperref[def:coordinate-chart]{chart} of \(G(n, m)\).

\begin{remark}
	The images \(\varphi _\Pi \big(\mathcal{L} (\Pi , \Pi ^{\perp} )\big)\) cover \(G(n, m)\).
\end{remark}

\begin{eg}
	\(\Pi = \varphi _\Pi (0)\in \varphi _\Pi \big(\mathcal{L} (\Pi , \Pi ^{\perp} )\big)\).
\end{eg}

We can now prove that these \hyperref[def:coordinate-chart]{charts} are mutually \hyperref[not:smoothly-compatible]{compatible}. Let \(\Pi , \Pi ^\prime \in G(n, m)\), and let \(P, P^\prime \) be orthogonal projections from \(\mathbb{R} ^{n+m}\) onto \(\Pi , \Pi ^\prime \) respectively. Firstly,
\[
	F = \varphi _{\Pi ^\prime }^{-1} \varphi _\Pi \colon \varphi _\Pi ^{-1} \big(\varphi _{\Pi ^\prime }(\mathcal{L} (\Pi ^\prime , (\Pi ^\prime )^{\perp} ))\big) \to \varphi _{\Pi ^\prime }^{-1} \big(\varphi _\Pi (\mathcal{L} (\Pi , \Pi ^{\perp} ))\big)
\]
is smooth.

\begin{center}
	\incfig{Grassmannian-transition}
\end{center}

Consider \(\alpha \in \mathcal{L} (\Pi ,\Pi ^{\perp} )\), and \(\beta \in \mathcal{L} (\Pi ^\prime , (\Pi ^\prime )^{\perp} )\), then for \(\alpha , \beta \), the equality \(F(\alpha ) = \beta\) means that \(\varphi _\Pi (\alpha ) = \varphi _{\Pi ^\prime }(\beta )\). Let \(f_\alpha \colon \Pi \to \Pi ^\prime \) be defined by
\[
	f_\alpha = P^\prime \circ (\mathbbm{1}_{_\Pi }\oplus \alpha ).
\]

We need to check
\begin{enumerate}[(a)]
	\item \(f_\alpha \) is invertible, and
	\item \(\forall y\in \Pi \), \(y+\alpha (y) = f_\alpha (y) + \beta (f_\alpha (y))\).
\end{enumerate}

\begin{note}
	The condition that \(\det f_\alpha \neq 0\) gives an exact description of the subset \(\varphi _{\Pi ^{-1} }\big(\varphi _{\Pi ^\prime }(\mathcal{L} (\Pi ^\prime , (\Pi ^\prime )^{\perp} ))\big)\) of \(\mathcal{L} (\Pi , \Pi ^{\perp} )\), which is therefore open.
\end{note}

For \(\beta \), it is \(\big(\mathbbm{1}_{\Pi ^\prime }\oplus \beta \big) \circ f_\alpha = \mathbbm{1}_{\Pi }\oplus \alpha \), and hence
\[
	\beta = F(\alpha ) = (\mathbbm{1}_{\Pi }\oplus \alpha )\circ f_\alpha ^{-1} - \mathbbm{1}_{\Pi ^\prime } .
\]
It follows by the construction that the image of \(\beta \) is contained in \((\Pi ^\prime )^{\perp} \).

\begin{remark}
	We obtain an infinite \hyperref[def:atlas]{atlas} for \(G(n, m)\) with \hyperref[def:coordinate-chart]{charts} labeled by \(\Pi \in G(n, m)\). But it's suffices to consider only \(\binom{n+m}{n}\) \hyperref[def:coordinate-chart]{charts} corresponding to subspaces \(\Pi \) spanned with \(n\) coordinate axes.
\end{remark}

We now introduce two notions.

\begin{definition}[Closed manifold]\label{def:closed-manifold}
	A \hyperref[def:topological-manifold]{manifold} is \emph{closed} if it is compact and without boundary.
\end{definition}

\begin{definition}[Open manifold]\label{def:open-manifold}
	A \hyperref[def:topological-manifold]{manifold} is \emph{open} if it has only non-compact components without boundary.
\end{definition}

\begin{lemma}
	If \(M\) can be covered by two \hyperref[def:coordinate-chart]{coordinate} neighborhoods \(V_1, V_2\) such that \(V_1 \cap V_2\) is connected, then \(M\) is \hyperref[def:orientable]{orientable}.
\end{lemma}
\begin{proof}
	The determinant of the differential of the \hyperref[def:coordinate-transition]{coordinate change} \(\neq 0\), so it does not change sign in \(V_1 \cap V_2\). If it's negative at a single point, it's enough to change the sign of one of the coordinates to make it positive at that point, hence on \(V_1 \cap V_2\).
\end{proof}

\begin{eg}
	Let \(S^n = \left\{ (x_1, \dots , x_{n+1})\in \mathbb{R} ^{n+1} \mid \sum_{i=1}^{n+1} x_i^2 = 1 \right\} \subseteq \mathbb{R} ^{n+1}\) is \hyperref[def:orientable]{orientable}.
\end{eg}
\begin{explanation}
	Let \(N=(0, \dots , 0, 1)\) and \(S=(0, \dots , 0, -1)\), consider given \(p=(0, \dots , 0, x_i, 0, \dots , x_{n+1} )\), then \(\pi _1\colon S^n \setminus \left\{ N \right\} \to \mathbb{R} ^n\) given by
	\[
		\pi _1(p) = \left( 0, \dots , 0, \frac{x_i}{1-x_{n+1}}, 0, \dots , 0 \right)
	\]
	to be the stereographic projection from the north pole \(N\).

	\begin{center}
		\incfig{stereographic-projection}
	\end{center}

	More generally, it takes \(p(x_1, \dots , x_{n+1})\in S^{n}-\left\{ N \right\}\) into the intersection at the hyperplane \(x_{n+1}= 0\) with the line passing through \(p\) ad \(N\). In this way, we have
	\[
		\pi _1(x_1, \dots , x_n) = \left( \frac{x_1}{1-x_{n+1}}, \frac{x_2}{1-x_{n+1}}, \dots , \frac{x_n}{1-x_{n+1}} \right),
	\]
	hence \(\pi _1\colon S^n \setminus \left\{ N \right\} \to \mathbb{R} ^n\) is differentiable, and is injective. Similarly, \(\pi _2\colon S^n \setminus \left\{ S \right\} \to \mathbb{R} ^n\) for \(S\) can also be defined and everything holds similarly. We see that these two parametrizations \((\mathbb{R} ^n, \pi _1 ^{-1} ), (\mathbb{R} ^n, \pi _2 ^{-1} )\) cover \(S^n\). The change of coordinate is given by
	\[
		y_j = \frac{x_j}{1 - x_{n+1}} \leftrightarrow y_j^\prime = \frac{x_j}{1 + x_{n+1}},\ (y_1, \dots , y_n) \in \mathbb{R} ^n,\ j = 1, \dots , n,
	\]
	where
	\[
		y_j ^\prime = \frac{y_j }{\sum_{i=1}^n y_i ^2 }.
	\]
	This implies that \(\left\{ (\mathbb{R} ^n, \pi _1 ^{-1} ), (\mathbb{R} ^n, \pi _2 ^{-1} ) \right\} \) is a \hyperref[def:smooth-structure]{differentiable structure} for \(S^n\). Now, consider \(\pi _1 ^{-1} (\mathbb{R} ^n) \cap \pi _2 ^{-1} (\mathbb{R} ^n) = S^n \setminus \left\{ N \cup S \right\} \), which is connected, and hence \(S^n\) is \hyperref[def:orientable]{orientable}, and the above \hyperref[def:smooth-structure]{structure} gives an \hyperref[def:orientation]{orientation} of \(S^n\).
\end{explanation}