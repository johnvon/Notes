\lecture{26}{11 Apr. 13:00}{Toward Proving the Sphere Theorem}\todo{This lecture we use \(\dim \mathcal{M} = m\), change it back to \(n\).}
\begin{corollary}
	Let \(\mathcal{M} \) be a compact \(2\)-dimensional \hyperref[def:Riemannian-manifold]{Riemannian manifold}, and \(K\) be the \hyperref[rmk:Gauss-curvature]{Gauss curvature}.
	\begin{enumerate}[(a)]
		\item If \(\mathcal{M} \) is homeomorphic to the sphere or the projective plane, then \(K > 0\) somewhere.
		\item If \(\mathcal{M} \) is homeomorphic to torus or Klein bottle, then either \(K = 0\) or \(K\) takes on both positive and negative values.
		\item If \(\mathcal{M} \) is any other compact surfaces, then \(K < 0\) somewhere.
	\end{enumerate}
\end{corollary}

\begin{corollary}
	Let \(\mathcal{M} \) be a compact \(2\)-dimensional \hyperref[def:Riemannian-manifold]{Riemannian manifold}, and \(K\) be the \hyperref[rmk:Gauss-curvature]{Gauss curvature}.
	\begin{enumerate}[(a)]
		\item If \(K > 0\), then \(\mathcal{M} \) is homeomorphic to sphere or projective plane, and \(\pi _i(\mathcal{M} )\) is finite.
		\item If \(K \leq 0\), then \(\pi _1(\mathcal{M} )\) is infinite and \(\mathcal{M} \) has genus at least \(1\).
	\end{enumerate}
\end{corollary}

To go beyond \(2\)-dimension, some consider the so-called \hyperref[def:phaffian]{phaffian}.

\begin{definition}[Pfaffian]\label{def:phaffian}
	Let \(\mathcal{P} \) be the map from \hyperref[def:tensor-field]{\((0, 4)\)-tensors} to \(\mathbb{R} \) with the domain carries symmetries as \hyperref[def:Riemannian-curvature]{Riemannian curvature}.
\end{definition}

\begin{theorem}\label{thm:phaffian}
	On any oriented vector space, there exists a basis independent functions \(\mathcal{P} \) such that for all compact, even-dimension \hyperref[def:Riemannian-manifold]{Riemannian manifold} \(\mathcal{M} \),
	\[
		\int _\mathcal{M} \mathcal{P} (R)\,\mathrm{d} V = \frac{1}{2} \mathop{\mathrm{Vol}}(S^n) \chi (\mathcal{M} ) .
	\]
\end{theorem}

\begin{note}
	This is too much information swallowed...
\end{note}

\begin{notation}
	Let \(p\in \mathcal{M} \), then \(d_p \colon \mathcal{M} \to \mathbb{R} \) such that \(d_p(q) = \mathop{\mathrm{dist}}(p, q) \).
\end{notation}

We see that \(d_p\) is Lipschitz continuous and smooth on \(\mathcal{M} \setminus (\{ p \} \cup \mathop{\mathrm{Cut}}(p) ) \eqqcolon \mathcal{M} _p\). At nay \(q\in \mathcal{M} _p\), the gradient \(\nabla d_p\) is the \hyperref[def:tangent-vector]{tangent vector} at \(q\) of the unique normal minimizing \hyperref[def:geodesic]{geodesic} from \(p\) to \(q\). In particular, \(\vert \nabla d_p \vert = 1\) at most points of \(\mathcal{M} \).

Now, we want to compare distance functions on different \hyperref[def:smooth-manifold]{manifolds}. This requires comparing the Hessian.

\begin{prev}[Hessian]
	For all smooth function \(f\) on \(\mathcal{M} \), its Hessian \(\nabla ^2 f\) is defined as
	\[
		\nabla ^2 f(X, Y) = \langle \nabla _X \nabla f, Y \rangle.
	\]
	The Hessian of \(f\) is symmetric, and we can write \(\Delta f = \Tr (\nabla ^2 f)\).
\end{prev}

\begin{notation}
	Let \(K^+ \coloneqq \max _{\sigma \subseteq T_p \mathcal{M} } K(\sigma )\) and \(K^- \coloneqq \min _{\sigma \subseteq T_p \mathcal{M} } K(\sigma )\).
\end{notation}

\begin{theorem}[Hessian comparsion theorem]\label{thm:comparsion-Hessian}
	Let \((\mathcal{M} , g)\), \((\widetilde{\mathcal{M}} , \widetilde{g} )\) be \hyperref[def:geodesically-complete]{complete} \hyperref[def:Riemannian-manifold]{Riemannian manifolds}, and \(\gamma \colon [0, b] \to \mathcal{M} \) and \(\widetilde{\gamma} \colon [0, b] \to \widetilde{\mathcal{M}} \) be minimizing normal \hyperref[def:geodesic]{geodesics} in \(\mathcal{M} \) and \(\widetilde{\mathcal{M}} \), respectively, such that
	\[
		\widetilde{K} ^+ (t)\leq K^-(t)
	\]
	for all \(t\in [0, b]\). Denote \(q = \gamma (a)\), \(\widetilde{q} = \widetilde{\gamma} (a)\) for \(a \leq b\). Suppose \(X_q\in T_q \mathcal{M}\), \(\widetilde{X} \in T_{\widetilde{q} } \widetilde{\mathcal{M}} \) satisfy
	\[
		\langle X_q, \dot{\gamma } (a) \rangle = \langle X_{\widetilde{q} }, \dot{\widetilde{\gamma} } (a) \rangle
	\]
	and \(\vert X_q \vert = \vert \widetilde{X} _{\widetilde{q} } \vert \). Then,
	\[
		\nabla ^2 d_p(X_q, X_q) \leq \widetilde{\nabla} ^2 \widetilde{d} _{\widetilde{p} } (\widetilde{X} _{\widetilde{q} }, \widetilde{X} _{\widetilde{q} }).
	\]
\end{theorem}

\subsection{Toponogor Theorem}
We now state the main tools we need in order to prove the \hyperref[thm:sphere]{sphere theorem}.

\begin{definition*}
	Let \((\mathcal{M} , g)\) be a \hyperref[def:geodesically-complete]{complete} \hyperref[def:Riemannian-manifold]{Riemannian manifold}.
	\begin{definition}[Geodesic triangle]\label{def:geodesic-triangle}
		A \emph{geodesic triangle} \(\triangle ABC\) consists of \(3\) points \(A, B, C\in \mathcal{M} \) and \(3\) minimizing \hyperref[def:geodesic]{geodesics} (sides) \(\gamma _{AB}, \gamma _{BC}, \gamma _{CA}\) joining each \(2\) of them.
	\end{definition}

	\begin{definition}[Generalized geodesic triangle]\label{def:generalized-geodesic-triangle}
		A \emph{generalized geodesic triangle} \(\triangle ABC\) consists of \(3\) points \(A, B, C\in \mathcal{M} \) and \(2\) minimizing \hyperref[def:geodesic]{geodesics} \(\gamma _{AB}, \gamma _{AC}\) and \(1\) \hyperref[def:geodesic]{geodesic} \(\gamma _{BC}\) of \hyperref[def:length]{length} \(L(\gamma _{BC}) \leq L(\gamma _{AB}) + L(\gamma _{AC})\), joining each \(2\) of them.
	\end{definition}

	\begin{definition}[Geodesic hinge]\label{def:geodesic-hinge}
		A \emph{geodesic hinge} \(\angle BAC\) consists of a point \(A\in \mathcal{M} \) and \(2\) minimizing \hyperref[def:geodesic]{geodesics} \(\gamma _{AB}, \gamma _{AC}\) emanating from \(A\) with endpoints \(B, C\).
	\end{definition}
	\begin{definition}[Generalized geodesic hinge]\label{def:generalized-geodesic-hinge}
		A \emph{generalized geodesic hinge} \(\angle BAC\) consists of a point \(A\in \mathcal{M} \) and \(2\) \hyperref[def:geodesic]{geodesics} \(\gamma _{AB}, \gamma _{AC}\) emanating from \(A\) with endpoints \(B, C\), with only one is minimizing.
	\end{definition}
\end{definition*}

For all \(k\in \mathbb{R} \), denote \(\mathcal{M} _k^m\) the \(m\)-dimensional \hyperref[def:space-form]{space form} of constant \hyperref[def:sectional-curvature]{curvature} \(k\), i.e.,
\[
	\mathcal{M} _k^m = S^m(k) \text{ or } \mathbb{R} ^m \text{ or } \mathbb{H} ^m(k).
\]

\begin{lemma}
	Let \((\mathcal{M}^m , g)\) be a \hyperref[def:geodesically-complete]{complete} \hyperref[def:Riemannian-manifold]{Riemannian manifold} with \hyperref[def:sectional-curvature]{sectional curvature} \(K \geq k\).
	\begin{enumerate}[(a)]
		\item For all \hyperref[def:generalized-geodesic-hinge]{generalized geodesic hinge} \(\angle BAC\) in \(\mathcal{M} \), there exists a \hyperref[def:geodesic-hinge]{geodesic hinge} \(\angle \widetilde{B} \widetilde{A} \widetilde{C} \) in \(\mathcal{M} ^m_k\) with the same angle and corresponding sides are with the same \hyperref[def:length]{length} as \(\angle BAC\).
		\item For all \hyperref[def:generalized-geodesic-triangle]{generalized geodesic triangle} \(\triangle ABC\) in \(\mathcal{M} \), there exists a \hyperref[def:geodesic-triangle]{geodesic triangle} \(\triangle \widetilde{A} \widetilde{B} \widetilde{C} \) in \(\mathcal{M} ^m_k\) whose corresponding sides have the same \hyperref[def:length]{length} as \(\triangle ABC\).
	\end{enumerate}
\end{lemma}

\begin{theorem}[Toponogor theorem]\label{thm:Toponogor}
	Let \((\mathcal{M} , g)\) be a \hyperref[def:geodesically-complete]{complete} \hyperref[def:Riemannian-manifold]{Riemannian manifold} with \hyperref[def:sectional-curvature]{sectional curvature} \(K \geq k\).
	\begin{enumerate}[(a)]
		\item Let \(\angle BAC\) be a \hyperref[def:geodesic-hinge]{geodesic hinge} in \(\mathcal{M} \) and \(\angle \widetilde{B} \widetilde{A} \widetilde{C} \) in \(\mathcal{M} _k^m\). Then, \(\mathop{\mathrm{dist}}(B, C) = \mathop{\mathrm{dist}}(\widetilde{B} , \widetilde{C} ) \).
		\item Let \(\triangle ABC\) be a \hyperref[def:geodesic-triangle]{geodesic triangle} in \(\mathcal{M} \), \(\triangle \widetilde{A} \widetilde{B} \widetilde{C} \) in \(\mathcal{M} _k^m\). Then, the \(3\) angles in \(\triangle ABC\) are greater than the corresponding angles in \(\triangle \widetilde{A} \widetilde{B} \widetilde{C} \).
	\end{enumerate}
\end{theorem}

\begin{theorem}[Klingenberg]\label{thm:Klingenberg}
	Let \((\mathcal{M} , g)\) be a \hyperref[def:geodesically-complete]{complete}, simply-connected \hyperref[def:Riemannian-manifold]{Riemannian manifold} with \hyperref[def:sectional-curvature]{sectional curvature} \(1 / 4 < K \geq 1\). Then,
	\[
		\mathop{\mathrm{inj}}(\mathcal{M} , g) \geq \pi .
	\]
\end{theorem}

\subsection{Proof of the Sphere Theorem}
Now, we can prove the \hyperref[thm:sphere]{sphere theorem}. Let's first restate it (after \hyperref[rmk:sphere-theorem-scaling]{scaling}) for our reference.

\begin{theorem}[(Scaled) Sphere theorem]\label{thm:sphere*}
	Let \(\mathcal{M} ^m\) be a compact and simply-connected \hyperref[def:Riemannian-manifold]{Riemannian manifold} with \hyperref[def:sectional-curvature]{sectional curvature} \(K\) such that
	\[
		\frac{1}{4} < K \leq 1.
	\]
	Then \(\mathcal{M} \) is homeomorphic to a sphere \(S^m\).
\end{theorem}
\begin{proof}\let\qed\relax
	By \hyperref[thm:Bonnet-Mayers]{Bonnet-Mayers theorem}, we know that \(\mathcal{M} \) is compact, hence there exists \(k > 1 / 4\) such that \(k \leq K \leq 1\). By the \hyperref[thm:Klingenberg]{Klingenberg theorem},
	\[
		\ell
		= \mathop{\mathrm{diam}}(\mathcal{M} , g)
		\geq \mathop{\mathrm{inj}}(\mathcal{M} , g)
		\geq \pi
		> \frac{\pi}{2 \sqrt{k} }.
	\]
	Take \(p, q\in \mathcal{M} \) such that \(\mathop{\mathrm{dist}}(p, q) = \mathop{\mathrm{diam}}(\mathcal{M} , g)\). Let \(q_0 \in \mathcal{M} \) such that \(\ell _1 = \mathop{\mathrm{dist}}(p, q_0) > \pi / 2 \sqrt{k} \), and \(\gamma _1\) be a minimizing normal \hyperref[def:geodesic]{geodesic} connecting \(p = \gamma _1(0)\) and \(q_0 = \gamma _1(\ell _1)\). Then, consider the following.
	\begin{lemma}
		Let \((\mathcal{M} , g)\) be a compact \hyperref[def:Riemannian-manifold]{Riemannian manifold} where there exists \(p, q\in \mathcal{M} \) such that \(\mathop{\mathrm{dist}}(p, q) = \mathop{\mathrm{diam}}(\mathcal{M} , g) \). Then, for all \(X_p \in T_p \mathcal{M} \), there exists a minimizing \hyperref[def:geodesic]{geodesic} \(\gamma \) connecting \(p=\gamma (0)\) to \(q\) such that
		\[
			\langle \dot{\gamma }(0) , X_p \rangle \geq 0.
		\]
	\end{lemma}
	From this, there exists a minimizing normal \hyperref[def:geodesic]{geodesic} \(\gamma _2\) connecting \(p=\gamma _2(0)\) to \(q = \gamma _2(\ell )\) such that \(\langle \dot{\gamma }_1(0), \dot{\gamma }_2(0) \rangle \geq 0\), i.e., the angle \(\alpha \) between \(\dot{\gamma } _1 (0) \) and \(\dot{\gamma } _2 (0)\) is no more than \(\pi / 2\). According \hyperref[thm:Toponogor]{Toponogor theorem}, by looking at the \hyperref[def:geodesic-hinge]{geodesic hinge} \(\angle q_1 p q\), \(\mathop{\mathrm{dist}}(q_1, q) \leq \mathop{\mathrm{dist}}(\widetilde{q} _1, \widetilde{q} )\) for a comparison \hyperref[def:geodesic-hinge]{geodesic hinge} \(\angle \widetilde{q} _1 \widetilde{p} \widetilde{q} \) in \(\mathcal{M} _k^m= S^m(1 / \sqrt{k} )\). Now, due to the cosine law for \(S^m(1 / \sqrt{k} )\),
	\[
		\cos (\sqrt{k}\cdot \mathop{\mathrm{dist}}(q,_1, q) )
		\geq \cos (\sqrt{k}\cdot \mathop{\mathrm{dist}}(\widetilde{q} _1, \widetilde{q} ) )
		= \cos \sqrt{k} \ell \cdot \cos \sqrt{k} \ell _1 + \sin \sqrt{k} \ell \cdot \sin \sqrt{k} \ell _1\cos \alpha
		\geq \dots
		> 0.
	\]
\end{proof}