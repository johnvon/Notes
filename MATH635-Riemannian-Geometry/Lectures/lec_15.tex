\lecture{15}{23 Feb. 14:30}{}
Given a \(1\)-form \(\omega \), and \hyperref[def:vector-field]{vector fields} \(X, Y\), we have
\[
	X(\omega (Y)) = (\nabla _X \omega )(Y) + \omega (\nabla _X Y).
\]
For arbitrary \hyperref[def:tensor]{tensors} \(S, T\), we similarly have
\[
	\nabla _X (S \otimes T) = \nabla _X S \otimes T + S \otimes \nabla _X T.
\]
If \(S\) is a \(p\)-times covariant tensor, and \(Y_1, \ldots , Y_p\) \hyperref[def:vector-field]{vector fields},
\[
	(\nabla _X S) ( Y_1, \ldots , Y_p) = X(S(Y_1, \ldots , Y_p)) - \sum_{i=1}^{p} S(Y_1, \ldots , Y_{i-1}, \nabla _X Y_i, Y_{i+1}, \ldots , Y_p ).
\]
For \(T\) a \hyperref[def:tensor]{\((p, q)\)-tensor field},
\[
	\begin{split}
		(\nabla _Y T) ( \alpha _1, \ldots , \alpha _q, X_1, \ldots , X_p)
		&= Y(T(\alpha _1, \ldots , \alpha _q, X_1, \ldots , X_p))\\
		&- \sum_{i=1}^{q} T(\alpha _1, \ldots , \nabla _Y \alpha _i, \ldots , \alpha _q , X_1, \ldots , X_p)\\
		&- \sum_{i=1}^{q} T(\alpha _1, \ldots , \alpha _q, X_1, \ldots , \nabla _Y X_i, \ldots , X_p).
	\end{split}
\]

If \(S = g_{ij} \mathrm{d} x^i \otimes \mathrm{d} x^j\), then \(\nabla _X g = 0\) for all \hyperref[def:vector-field]{vector fields} \(X\).

Also,
\[
	\begin{split}
		(\mathcal{L} _X S) (Y_1, \ldots , Y_p)
		&=X(S(Y_1, \ldots , Y_p))
		- \sum_{i=1}^{p} S(Y_1, \ldots , [X, Y_i], \ldots , Y_p)\\
		&= (\nabla _X S)(Y_1, \ldots , Y_p) + \sum_{i=1}^{p} S(Y_i, \ldots , \nabla _{Y_i}X, \ldots , Y_p)
	\end{split}
\]
since \(\nabla \) is \hyperref[def:torsion-free]{torsion-free}, we have \(\nabla _X Y_i - \nabla _{Y_i}X = [X, Y_i]\).

\begin{definition}[Killing field]\label{def:killing-field}
	Consider a \hyperref[def:Riemannian-manifold]{Riemannian manifold} \((\mathcal{M} , g)\), and \(g = g_{ij} \mathrm{d} x^i \otimes \mathrm{d} x^j \). Then a \hyperref[def:vector-field]{vector field} \(X\) such that
	\[
		\mathcal{L} _X g = 0
	\]
	is called a \emph{killing field} (or \emph{infinitesimal isometry}).
\end{definition}

\begin{lemma}
	A \hyperref[def:vector-field]{vector field} \(X\) on \((\mathcal{M} , g)\) is a \hyperref[def:killing-field]{killing field} if and only if the \hyperref[def:local-1-parameter-group]{local \(1\)-parameter group} generated by \(X\) consisted of \hyperref[def:local-isometry]{local isometries}.
\end{lemma}

\begin{lemma}
	The \hyperref[def:killing-field]{killing fields} of a \hyperref[def:Riemannian-manifold]{Riemannian manifold} constitute a \hyperref[def:Lie-algebra]{Lie algebra}.
\end{lemma}

Let \(\dim \mathcal{N} = m+1\), \(\dim \mathcal{M} = m\), then for all \(x\in \mathcal{M} \), there are exactly \(2\) normal vectors \(\nu \in T_x \mathcal{M} ^{\perp} \) with \(\left\langle \nu, \nu \right\rangle \equiv 1\), i.e., \(\nabla _X^{\mathcal{N} } \nu \) always tangential to \(\mathcal{M} \).

\begin{remark}
	\(\nabla _X^{\mathcal{N} } \nu \) measures the ``tilting velocity'' with which \(\nu \) is tilted relative to a fixed \hyperref[def:parallel]{parallel} \hyperref[def:vector-field]{vector field} in \(\mathcal{N} \), when on \(\mathcal{M} \) in direction \(X\).
\end{remark}

\begin{theorem}\label{thm:lec15}
	Given \(\mathcal{M} \subseteq \widetilde{\mathcal{M}} \) such that \(\mathcal{M} \) is \hyperref[def:totally-geodesic]{totally geodesic} in \(\widetilde{\mathcal{M}} \) if and only if all \hyperref[def:2nd-fundamental-form]{\(2^{nd} \) fundamental form} of \(\mathcal{M} \) vanish identically.
\end{theorem}
\begin{proof}
	Let \(c\colon I \to \mathcal{M} \) be a \hyperref[def:geodesic]{geodesic} in \(\mathcal{M} \), i.e., \(\nabla _{\dot{c} }^{\mathcal{M}} \dot{c} = 0\). By \autoref{thm:lec14}, we have that
	\[
		\nabla _{\dot{c} }^{\mathcal{M}} \dot{c} = (\nabla _{\dot{c} }^{\widetilde{\mathcal{M}} } \dot{c} )^{\top} = 0 ,
	\]
	i.e., \(c\) is a \hyperref[def:geodesic]{geodesic} in \(\widetilde{\mathcal{M}} \) if and only if \((\nabla _{\dot{c} }^{\widetilde{\mathcal{M}} } \dot{c} )^{\top} = 0\), i.e.,
	\[
		\left\langle \nabla _{\dot{c} }^{\widetilde{\mathcal{M}} } \dot{c} , \nu  \right\rangle = 0
	\]
	for all \(\nu \in T \mathcal{M} ^{\perp} \). Notice that \(\left\langle \dot{c} , \nu \right\rangle = 0\) and \(\dot{c} \left\langle \dot{c} , \nu \right\rangle = \left\langle \nabla _{\dot{c} }^{\widetilde{\mathcal{M}} } \dot{c} , \nu \right\rangle + \left\langle \dot{c} , \nabla _{\dot{c} }^{\widetilde{\mathcal{M}} } \nu  \right\rangle = 0\), we have
	\[
		0 = \left\langle \nabla _{\dot{c}}^{\widetilde{\mathcal{M}} } \dot{c} , \nu  \right\rangle
		= \left\langle \dot{c} , \nabla _{\dot{c} }^{\widetilde{\mathcal{M}} } \nu \right\rangle
		= -\ell _{\nu } (\dot{c}, \dot{c}  ).
	\]
\end{proof}

\begin{note}
	\autoref{thm:lec15} holds for \hyperref[def:Lorentzian]{Lorentzian manifolds} \((\widetilde{\mathcal{M}} , \widetilde{g} )\).
\end{note}

\begin{eg}
	The initial value problem for Einstein equations. Given a \((\widetilde{\mathcal{M}} ^4, \widetilde{g} )\) a \hyperref[def:Lorentzian]{Lorentzian manifolds} satisfying Einstein equations. \((\mathcal{M} ^3, g)\) non-degenerate \hyperref[def:Riemannian-manifold]{Riemannian manifold}. If the \hyperref[def:2nd-fundamental-form]{\(2^{nd} \) fundamental form} of \(\mathcal{M} ^3\) in \(\widetilde{\mathcal{M}} ^4\) vanishes identically, then \(\mathcal{M} ^3\) is \hyperref[def:totally-geodesic]{totally geodesic}. This is a special case and not in general.
\end{eg}

\begin{notation}
	Greek indices \((\alpha , \beta , \ldots )\) occurring twice are summed over from \(1\) to \(k\) for \(X, Y, Z, W \in T_x \mathcal{M} \).
\end{notation}

\begin{theorem}[Gauss equations]\label{thm:Gauss-equations}
	Let \(\mathcal{N} \) be a \hyperref[def:Riemannian-manifold]{Riemannian manifold} with \(\dim \mathcal{N} = n\), and let \(\mathcal{M} \subseteq \mathcal{N} \) be a \hyperref[def:submanifold]{submanifold} with \(\dim \mathcal{M} = m\). Let \(k = n - m\), and \(x\in \mathcal{M} \), \(\nu _1, \ldots , \nu _k\) be an orthonormal basis of \((T_x \mathcal{M} )^{\perp} \), \(S_\alpha \coloneq^{2} _{\nu _\alpha }\), \(\ell _\alpha \coloneqq \ell _{\nu _\alpha }\), \(\alpha = 1, \ldots , k\). Then,
	\[
		R^{\mathcal{M}}(X, Y)Z - \left( R^{\mathcal{N} } (X, Y)Z \right) ^{\top}
		= \ell _\alpha (Y, Z)S_\alpha (X) - \ell _\alpha (X, Z)S_{\alpha }(Y).
	\]
	Thus, we also have
	\[
		\left\langle R^{\mathcal{M} }(X, Y)Z, W  \right\rangle - \left\langle R^{\mathcal{N} }(X, Y)Z, W \right\rangle
		= \ell _\alpha (Y, Z) \ell _\alpha (X, W) - \ell _\alpha (X, Z) \ell _\alpha (Y, W).
	\]
\end{theorem}
\begin{proof}
	We can extend \(X, Y, Z, W\), ad \(\nu , \ldots , \nu _k\) to \hyperref[def:vector-field]{vector fields} inn \(T_{\mathcal{M} } \) and \(T \mathcal{M} ^{\perp} \), respectively. Let \(\nu _\alpha \) be orthonormal, then
	\[
		\nabla _Y^{\mathcal{N} } Z
		= (\nabla _Y^{\mathcal{N} } Z)^{\top} = (\nabla _X^{\mathcal{N} } Z)^{\perp}
		= \nabla _Y ^\mathcal{M} Z + \left\langle \nu _\alpha , \nabla _Y^{\mathcal{N} } Z \right\rangle \nu _{\alpha }
	\]
	as \(\nu _\alpha \) form orthonormal basis of \(T \mathcal{M} ^{\perp} \). Hence,
	\[
		\nabla _X ^\mathcal{N} \nabla _Y ^\mathcal{N} Z
		= \nabla _X ^{\mathcal{N} }\nabla _Y^{\mathcal{M} }Z + X(\left\langle \nu _\alpha , \nabla _Y^{\mathcal{N} } Z \right\rangle ) \nu _\alpha + \left\langle \nu _\alpha , \nabla _Y^{\mathcal{N} } Z \right\rangle \nabla _X^{\mathcal{N} } \nu _\alpha.
	\]
	Then,
	\[
		(\nabla _X^{\mathcal{N} } \nabla _Y ^{\mathcal{N} }Z)^{\top}
		= \nabla _X ^\mathcal{M} \nabla _Y ^\mathcal{M} Z + \underbrace{\left\langle \nu _\alpha , \nabla _Y ^\mathcal{N} Z \right\rangle}_{-\ell _\alpha (Y, Z)} \underbrace{(\nabla _X ^\mathcal{N} \nu _\alpha )^{\top} }_{S_\alpha (X)}
		= \nabla _X ^\mathcal{M} \nabla _Y ^\mathcal{M} Z - \ell _\alpha (Y, Z) S_\alpha (X).
	\]
	Analogously, we have
	\[
		(\nabla ^\mathcal{N} _Y \nabla ^\mathcal{N} _X Z)^{\top}
		= \nabla _Y ^\mathcal{M} \nabla _X ^\mathcal{M} Z - \ell _\alpha (X, Z) S_\alpha (Y),
	\]
	and also, we have
	\[
		(\nabla ^\mathcal{N} _{[X, Y]} Z)^{\top} = \nabla ^\mathcal{M} _{[X, Y]} Z.
	\]
	By collecting terms, we have
	\[
		\begin{split}
			(\nabla ^\mathcal{N} _X \nabla ^\mathcal{N} _Y Z) ^{\top} &- (\nabla ^\mathcal{N} _Y \nabla ^\mathcal{N} _X Z) ^{\top} - (\nabla ^\mathcal{N} _{[X, Y]} Z) ^{\top}\\
			&= \nabla _X ^\mathcal{M} \nabla _Y ^\mathcal{M} Z - \nabla _Y ^{\mathcal{M} } \nabla _X ^\mathcal{M} Z - \nabla ^\mathcal{M} _{[X, Y]} Z - \ell _\alpha (Y, Z) S_\alpha (X) + \ell _\alpha (X, Z) S_\alpha (Y),
		\end{split}
	\]
	equivalently,
	\[
		R^{\mathcal{M}} (X, Y)Z - (R^{\mathcal{N} } (X, Y)Z)^{\top} = \ell _\alpha (Y, Z)S_\alpha (X) - \ell _\alpha (X, Z)S_{\alpha }(Y).
	\]
\end{proof}

\autoref{thm:Gauss-equations} tells us that for a surface \(\mathcal{M} \) in \(\mathbb{R} ^3\), the \hyperref[def:Gauss-Kronecker-curvature]{Gauss-Kronecker curvature} coincides with the \hyperref[def:Riemannian-curvature]{Riemannian curvature} of \(\mathcal{M} \), which is independent of the \hyperref[def:embedding]{embedding}. Therefore, \hyperref[def:Gauss-Kronecker-curvature]{Gauss-Kronecker curvature} does not depend on \hyperref[def:embedding]{embeddings} of \(\mathcal{M} \) into \(\mathbb{R} ^3\).

\begin{remark}[Codazzi equations]
	Let \(\mathcal{M} ^m \subseteq \mathcal{N} ^{m+1}\) where \(N\) is unit normal on \(\mathcal{M} \)
	\[
		\left\langle R(X, Y) e_j , N \right\rangle = (\nabla _X ^{\mathcal{M} } \ell ) (Y, e_j) - (\nabla _Y ^{\mathcal{M} } \ell )(X, e_j)
		= Xrk Y^i \nabla _k^{\mathcal{M} } \ell _{ij} - Y^k X^i \nabla _k^{\mathcal{M} } \ell _{ij},
	\]
	i.e., \(\left\langle R(X, Y) Z, N \right\rangle = (\nabla _X^{\mathcal{M} } \ell )(Y, Z) - (\nabla _Y^{\mathcal{M} } \ell )(X, Z)\).
\end{remark}