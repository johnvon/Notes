\lecture{17}{9 Mar. 13:00}{Jacobi Fields and General Relativity}
\begin{lemma}
	A \hyperref[def:vector-field-along-curve]{vector field along} a \hyperref[def:geodesic]{geodesic} \(c \colon [a, b] \to \mathcal{M} \) is a \hyperref[def:Jacobi-field]{Jacobi field} if and only if it is a critical point of \(I(X, X)\) w.r.t.\ all vanishes with fixed endpoints, i.e.,
	\[
		\at{\frac{\mathrm{d}}{\mathrm{d}s} I(X + s Y, X + sY)}{s=0}{} = 0
	\]
	for every \hyperref[def:vector-field-along-curve]{vector field along \(c\)} with \(Y(a) = 0 = Y(b)\).
\end{lemma}
\begin{proof}
	We just use the proof of \autoref{prop:Jacobi-field}.
\end{proof}

This tells us that the \hyperref[eq:Jacobi]{Jacobi equation} is the \href{https://en.wikipedia.org/wiki/Euler%E2%80%93Lagrange_equation}{Euler-Lagrange equations} for \(I(X) \coloneqq I(X, X)\).

\subsection{Existence and Uniqueness of Jacobi Fields}
Given the initial data, how can we characterize the \hyperref[eq:Jacobi]{Jacobi equation} on a \hyperref[def:Riemannian-manifold]{Riemannian manifold} \((\mathcal{M} , g)\) with \(\dim \mathcal{M} = d\)? Firstly, we know that the \hyperref[eq:Jacobi]{Jacobi equation} is a system of \(d\) linear secondorder ODE.

\begin{theorem}
	Let \(c\colon [a, b] \to \mathcal{M} \) be a \hyperref[def:geodesic]{geodesic}. For all \(v, w\in T_{c(a)}\mathcal{M} \), there exists a unique \hyperref[def:Jacobi-field]{Jacobi field} \(X\) along \(c\) with \(X(a) = v\), \(\dot{X} (a) = w\).
\end{theorem}
\begin{proof}
	Let \(\left\{ v_i \right\} _{i=1}^d\) be an orthonormal basis of \(T_{c(a)} \mathcal{M} \). Let \(\left\{ X_i \right\} _{i=1}^d\) be \hyperref[def:parallel]{parallel} \hyperref[def:vector-field-along-curve]{vector field along} with \(X_i(a) = v_i\) for \(i = 1, \dots , d\). Then for all \(t\in [a, b]\), \(X_1(t), \dots , X_d(t)\) is an orthonormal basis of \(T_{c(t)}\mathcal{M} \). Choose arbitrary \hyperref[def:vector-field-along-curve]{vector field \(X\) along \(c\)} as \(X = \xi ^i X_i\), i.e., \(\xi ^i(t) = \left\langle X(t), X_i(t) \right\rangle \). As \hyperref[def:vector-field-along-curve]{vector field} \(X_i\) are \hyperref[def:parallel]{parallel}, we have
	\[
		\nabla _{\frac{\mathrm{d}}{\mathrm{d}t} } X
		= \frac{\mathrm{d}\xi ^i}{\mathrm{d}t} X_i + \xi _i \underbrace{\nabla _{\frac{\mathrm{d}}{\mathrm{d}t} } X_i }_{0}
		= \frac{\mathrm{d}\xi ^i}{\mathrm{d}t} X_i,
	\]
	hence
	\[
		\nabla _{\frac{\mathrm{d}}{\mathrm{d}t} }\nabla _{\frac{\mathrm{d}}{\mathrm{d}t} }X = \frac{\mathrm{d}^2 \xi ^i}{\mathrm{d}t^2} X_i .
	\]
	To write the \hyperref[eq:Jacobi]{Jacobi equation} in these coordinates, we first write the \hyperref[def:Riemannian-curvature]{curvature} as
	\[
		R(X, \dot{c} ) \dot{c} = \xi ^i \rho _i^k X_k.
	\]
	\begin{notation}[Rotation]
		Let \(\rho _i^k \coloneqq \left\langle R(X_i, \dot{c} )\dot{c}, X_k \right\rangle \), i.e., \(R(X_i, \dot{c} )\dot{c} = \rho _i^k X_k \).
	\end{notation}
	Then, the \hyperref[eq:Jacobi]{Jacobi equation} becomes
	\[
		\left( \frac{\mathrm{d}^2 \xi ^k}{\mathrm{d}t^2} + \xi ^i \rho _i^k \right) X_k = 0
		\implies \frac{\mathrm{d}^2 \xi ^k(t)}{\mathrm{d}t^2}  + \xi ^i(t) \rho _i^k(t) = 0,\quad k = 1, \dots , d
	\]
	since \(\left\{ X_i \right\} \) is a orthonormal basis. Then, by the linear algebra and ODE theory, we have existence and uniqueness.
\end{proof}

\section{Application of General Relativity}
Consider the universe as a \((\mathcal{M} ^4, g)\) a \hyperref[def:Lorentzian-metric]{Lorentzian manifold},
\begin{center}
	\incfig{GR}
\end{center}

Here, we have \([\partial / \partial s, \partial / \partial t] = [U, V] = 0 \). Hence, the \hyperref[eq:Jacobi]{Jacobi equation} is now
\[
	\nabla ^2_U V + R(V, U, U) = 0.
\]

For given \(U\), the right-hand side defines of each \(p\in \mathcal{M} \) a linear map
\[
	N \mapsto R(N, U)U
\]
for \(N\) unit normal of subspace of \(T_p \mathcal{M} \) perpendicular to \(U\). This is often called the \emph{field force operator}.

Hence, locally,
\begin{itemize}
	\item the gravitational field \(g\), the ``fields strengths'' \(\Gamma \) can be transformed away;
	\item variation of gravitational fields strengths can be described by \hyperref[def:Riemannian-curvature]{Riemannian curvature tensor}, hence cannot be transformed away.
\end{itemize}

All these imply that the \hyperref[eq:Jacobi]{Jacobi equation} with \hyperref[def:Riemannian-curvature]{Riemannian curvature tensor} can describe the relative accelerations (or field forces) of nearby \hyperref[def:geodesic]{geodesics}.
\begin{figure}[H]
	\centering
	\incfig{LIGO}
	\caption{LIGO 2015, \(\frac{\Delta \lambda }{\lambda } \approx 10^{-21}\).}
	\label{fig:LIGO}
\end{figure}

\begin{eg}[\(\mathbb{R} ^n\)]
	The \hyperref[def:Jacobi-field]{Jacobi field} in \(\mathbb{R} ^n\). Since the \hyperref[def:geodesic]{geodesics} are straight lines, the \hyperref[def:Jacobi-field]{Jacobi field} \(X\) along straight line \(c\) with \(X(a) = v, \dot{X}(a) = w\). Let \(V(t), W(t)\) be \hyperref[def:parallel]{parallel} \hyperref[def:vector-field-along-curve]{vector fields along \(c\)} with \(V(a) = v, W(a) = w\), by linearizing, we have
	\[
		X(t) = V(t) + (t-a) W(t).
	\]
\end{eg}

\begin{eg}[\(S^n \subseteq \mathbb{R} ^{n+1}\)]
	Let \(c\colon [0, T] \to S^n\) be a \hyperref[def:geodesic]{geodesic} with \(\lVert \dot{c} \rVert = 1\), and \(v, w\in T_{c(0)}S^n\), \(V, W\) \hyperref[def:parallel]{parallel} \hyperref[def:vector-field-along-curve]{vector fields along \(c\)} with \(V(0) = v, W(0) = w\). Also, assume that \(\left\langle v, \dot{c} (0) \right\rangle = 0 = \left\langle w, \dot{c} (0) \right\rangle \), then the \hyperref[def:Jacobi-field]{Jacobi field} \(X\) is
	\[
		X(t) = V(t) \cos t + W(t) \sin t.
	\]
\end{eg}
\begin{explanation}
	We see that
	\[
		\dot{X}(t) = -V(t) \sin t + W(t) \cos t,
	\]
	and
	\[
		\ddot{X}(t) = -V(t)\cos t - W(t) \sin t.
	\]
	By using the \hyperref[def:Riemannian-curvature]{Riemannian curvature} on \(S^n\), we have
	\[
		R(X, \dot{c} )\dot{c}
		= \underbrace{\left\langle \dot{c}, \dot{c} \right\rangle}_{1} X - \underbrace{\left\langle X, \dot{c} \right\rangle }_{0} \dot{c}
		= X.
	\]
	Then \(\ddot{X} + R(X, \dot{c} )\dot{c} = 0\).
\end{explanation}

\begin{remark}
	We can also consider \(S^n_{\rho } \subseteq \mathbb{R} ^{n+1}\) with \(\lVert \dot{c} \rVert = 1\) and play the above game, i.e., by letting
	\[
		X(t) = V(t) \cos \frac{t}{\rho } + W(t) \sin \frac{t}{\rho }.
	\]
\end{remark}

\section{Jacobi Fields and Geodesics}

Consider a \hyperref[def:Jacobi-field]{Jacobi field} transversal along \(c\), then we can split the \hyperref[def:Jacobi-field]{Jacobi field} into
\begin{itemize}
	\item tangential component: do not depend on geometry of \(\mathcal{M} \), hence no information about \(\mathcal{M} \);
	\item normal component: very useful!
\end{itemize}
Specifically, consider \(X = X^{\top} + X^{\perp} \), we have the following.

\begin{lemma}
	Let \(c\colon [a, b] \to \mathcal{M} \) be a \hyperref[def:geodesic]{geodesic}, and \(\lambda , \mu \in \mathbb{R} \). Then, the \hyperref[def:Jacobi-field]{Jacobi field} \(X\) along \(c\) with \(X(a) = \lambda \dot{c} (a)\), \(\dot{X} (a) = \mu \dot{c} (a)\) is given by
	\[
		X(t) = (\lambda + (t-a)\mu )\dot{c}(t).
	\]
\end{lemma}