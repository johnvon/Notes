\lecture{22}{28 Mar. 14:30}{}
\subsection{Bonnet-Mayers Theorem}
\begin{definition}[Diameter]\label{def:diameter}
	The \emph{diameter} of a \hyperref[def:smooth-manifold]{manifold} \(\mathcal{M} \) is defined as
	\[
		\mathop{\mathrm{diam}}(\mathcal{M} ) \coloneqq \sup _{p, q\in \mathcal{M} } d(p, q).
	\]
\end{definition}

\begin{theorem}[Bonnet-Mayers theorem]\label{thm:Bonnet-Mayers}
	Let \((\mathcal{M}^n , g)\) be a \hyperref[def:geodesically-complete]{complete}\footnote{I.e., closed and any two points can be joined by a minimizing \hyperref[def:geodesic]{geodesic}.} \hyperref[def:Riemannian-manifold]{Riemannian manifold} and \hyperref[def:Ricci-curvature]{Ricci curvature} \(\geq \lambda > 0\), i.e.,
	\[
		\mathop{\mathrm{Ric}}(X, X) \geq \lambda \left\langle X, X \right\rangle
	\]
	for all \(X\in T\mathcal{M} \). Then the diameter of \(\mathcal{M} \) is less than \(\pi \sqrt{(n - 1) / \lambda } \). In particular, \(\mathcal{M} \) is compact and has finite fundamental group \(\pi _1(\mathcal{M} )\).
\end{theorem}
\begin{proof}
	For all \(\rho < \mathop{\mathrm{diam}}(\mathcal{M} ) \), there exists \(p q\in \mathcal{M} \) with \(d(p, q) = \rho \). As \(\mathcal{M} \) \hyperref[def:geodesically-complete]{complete}, there exists a shortest \hyperref[def:geodesic]{geodesic} arc \(c \colon [0, \rho ] \to \mathcal{M} \) with \(c(0) = p\) and \(c(\rho ) = q\). Now, let \(\left\{ e_i \right\} _{i=1}^n\) be an orthonormal basis of \(T_p \mathcal{M} \), such that \(e_1 = \dot{c} (0)\). Now, consider a parallel orthonormal basis \(\left\{ \dot{c} (t), X_1(t), \ldots , X_n(t) \right\} \)  along \(c\). Furthermore, consider \(Y_i(t) \coloneqq (\sin \pi t / \rho )X_i (t)\) with \(i = 2, \ldots , n\). Then,
	\[
		I(Y_i, Y_i)
		= \int_{0}^{\rho } - \langle \ddot{Y_i} , Y_i \rangle - \langle R(Y_i, \dot{c})\dot{c} , Y_i \rangle  \,\mathrm{d}t
		= \int_{0}^{\rho } \sin ^2 \frac{\pi t}{\rho } \left( \frac{\pi ^2}{\rho ^2} - \langle R(X_i, \dot{c} )\dot{c}, X_i  \rangle \right)  \,\mathrm{d}t.
	\]
	Since \(c\) is the shortest \hyperref[def:curve]{curve} connecting \(p, q\), it follows that there are no \hyperref[def:conjugate-point]{conjugate points} between \(p, q\). Hence, \(I(Y_i, Y_i) \geq 0\) for all \(i\), so
	\[
		0
		\leq \sum_{i=2}^{n} I(Y_i, Y_i)
		= \int_{0}^{\rho} \sin ^2 \frac{\pi t}{\rho } \left( \frac{\pi ^2}{\rho ^2} (n-1) - R(\dot{c} , \dot{c} ) \right) \,\mathrm{d}t
		\leq \left( \frac{\pi ^2}{\rho ^2}(n-1) - \lambda \right) \int_{0}^{\rho } \sin ^2 \frac{\pi t}{\rho } \,\mathrm{d}x,
	\]
	implying
	\[
		0 \leq \frac{1}{2} \rho \left( \frac{\pi ^2(n-1)}{\rho ^2} - \lambda  \right)
		\implies \rho ^2 \leq \frac{\pi ^2 (n-1)}{\lambda }
		\implies \rho \leq \pi \sqrt{\frac{n-1}{\lambda }}.
	\]
	Since this is true for all \(\rho < \mathop{\mathrm{diam}}(\mathcal{M} ) \), hence we see that \(\mathop{\mathrm{diam}}(\mathcal{M} ) \leq \pi \sqrt{(n-1) / \lambda } \).

	Furthermore, the universal cover of \(\mathcal{M} \) satisfies the same assumptions as \hyperref[def:Ricci-curvature]{Ricci curvature}, by computation, we have finite \(\pi _1(\mathcal{M} )\).
\end{proof}

\begin{remark}
	We choose \(Y_i(t) = \sin (\pi t / \rho ) X_i(t)\) is just because it satisfies the needed condition, and makes the computation works out nicely.
\end{remark}

\begin{intuition}
	\hyperref[thm:Bonnet-Mayers]{Bonnet-Mayers theorem} says that if \(\mathcal{M} \) has \hyperref[def:Ricci-curvature]{Ricci curvature} not less than the one of \(S^n_r\), then \(\mathop{\mathrm{diam}}(\mathcal{M} ) \) is at most the one of \(S^n_r\).
\end{intuition}

Consider the hyperbolic space \(\mathbb{H} ^n\) in \(\mathbb{R} ^{n+1}\) where we define
\[
	\langle x, x \rangle \coloneqq -(x^0)^2 + (x^1)^2 + \dots + (x^n)^2
\]
for \(x=(x^0, \dots , x^1)\). Then
\[
	\mathbb{H} ^n \coloneqq \left\{ x\in \mathbb{R} ^{n+1} \mid \langle x, x \rangle = -1, x^0 > 0 \right\}.
\]
Also, consider the half-space of \(\mathbb{R} ^n\) such that
\[
	H^n = \left\{ (x_1, \dots , x_n)\in \mathbb{R} ^n \mid x_n > 0 \right\}
\]
with metric on \(\mathbb{H} ^n\), we have
\[
	g_{ij}(x_1, \dots , x_n) = \frac{\delta _{ij}}{x_n^2}.
\]
Then, we see that we have a constant sectional curvature of \(-1\).

\section{Morse Functions}
Let \((\mathcal{M} , g)\) be a \hyperref[def:geodesically-complete]{complete} \hyperref[def:Riemannian-manifold]{Riemannian manifold}. Let \(f\colon \mathcal{M} \to \mathbb{R} \) be a smooth\footnote{It's typically enough to ask for \(f\in C^3(\mathcal{M} , \mathbb{R} )\).} function. Then
\[
	\mathrm{d} f(x) = 0
\]
means that \(x\) is a critical point of \(f\).

\begin{definition}[Non-degenerate]\label{def:non-degenerate}
	A critical point \(a\) of \(f\) is \emph{non-degenerate} if the Hessian of \(f\) is non-singular at \(a\).
\end{definition}

\begin{definition}[Index of non-degenerate]\label{def:index-of-non-degenerate}
	The \emph{index} of \hyperref[def:non-degenerate]{non-degenerate} critical point of \(f\) is the dimension of the largest subspace of \(T_a \mathcal{M} \) on which the Hessian is negative definite.
\end{definition}

\begin{intuition}
	That is, the number of directions in which \(f\) decreases.
\end{intuition}

\begin{note}
	The \hyperref[def:non-degenerate]{degeneracy} and \hyperref[def:index-of-non-degenerate]{index} are independent of coordinate choice.
\end{note}

Now, we define the critical set of \(f\) as
\[
	C(f) \coloneqq \left\{ x\in \mathcal{M} \mid \mathrm{d} f(x) = 0 \right\}.
\]

\begin{definition}[Morse function]\label{def:Morse-function}
	A \emph{Morse function} \(f\) is a function as introduced such that all critical points are \hyperref[def:non-degenerate]{non-degenerate}.
\end{definition}

\begin{eg}
	Consider \(f\) is the height function, which is a \hyperref[def:Morse-function]{Morse function} such that the \hyperref[def:index-of-non-degenerate]{index} of \(s\) is \(2\), \(r\) is \(1\), \(q\) is \(1\), and \(p\) is \(0\) by looking at the decreasing directions.
	\begin{center}
		\incfig{Morse-function-torus}
	\end{center}
	Now, define \(M^a = f^{-1} (-\infty , a]\), then we see that
	\begin{enumerate}[(a)]
		\item Pass \(p\): \(M^a\) for \(0 < a < f(q)\) is a disk, which is homotopy equivalent to a point, i.e., \(0\)-cell.
		\item Pass \(q\): \(M^a\) for \(f(q) < a < f(r)\) is a cylinder, where we attach a \(1\)-cell.
		\item Pass \(r\): \(M^a\) for \(f(r) < a < f(s)\) is a torus with disk removed.
		\item Pass \(s\): \(M^a\) for \(a > f(s)\) is a torus.
	\end{enumerate}
\end{eg}