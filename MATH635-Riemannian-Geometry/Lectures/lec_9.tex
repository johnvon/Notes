\lecture{9}{31 Jan. 14:30}{Tensors and Connections}
\subsection{Contravariant and Covariant Tensors}

\begin{definition}[Tensor field]\label{def:tensor-field}
	Let \(V\) be a vector space of dimension \(m < \infty \), and the dual space \(V^{\ast}\).\footnote{I.e., \(V^{\ast} \coloneqq \left\{ \lambda \colon V \to \mathbb{R} \mid \lambda \text{ linear} \right\} \).} Then the \emph{\(r\)-times contravariant and \(s\)-times covariant tensors over \(V\) tensor field}, denoted as \(T_s^r(V) \), is the vector field defined as
	\[
		T_s^r(V) = \{ A\colon \underbrace{V^{\ast} \times \ldots \times V^{\ast} }_{r} \times \underbrace{V\times \ldots \times V}_{s} \to \mathbb{R} \}
		= \underbrace{V \otimes \ldots \otimes V}_{r} \otimes \underbrace{V^{\ast} \otimes \ldots \otimes V^{\ast} }_{s}.
	\]
\end{definition}

\begin{definition*}
	Let \(\Lambda ^s(V^{\ast} ) \coloneqq \left\{ A\in T_s^0(V) \mid A \text{ skew-symmetric} \right\} \), where \(s\in \mathbb{N} \). Let \(\mathcal{M} ^n \) be a \hyperref[def:smooth-manifold]{manifold}, and \(\pi \colon E \to \mathcal{M} \) the \hyperref[def:vector-bundle]{\(C^{\infty} \) vector bundle} \((E, \pi , \mathcal{M} )\).

	\begin{definition}
		\(\Gamma (E) \coloneqq \left\{ s\in C^{\infty} (\mathcal{M} , E) \mid \pi \circ s = \identity_{\mathcal{M} } \right\}\).
	\end{definition}

	\begin{definition}[Contravariant tensor field]\label{def:contravariant-tensor-field}
		The \emph{contravariant tensor field} \(\Gamma (T\mathcal{M} ) \coloneqq \left\{ \text{vector fields on } \mathcal{M} \right\} \).
	\end{definition}

	\begin{definition}[Covariant tensor field]\label{def:covariant-tensor-field}
		The \emph{covariant tensor field} \(\Gamma (\Lambda _s \mathcal{M} ) \coloneqq \left\{ \text{\(s\)-forms on } \mathcal{M} \right\} \) with \(\Lambda _s \mathcal{M} = \Lambda ^s\left( \bigcup_{p\in \mathcal{M} } T^{\ast} _p \mathcal{M}  \right) \).
	\end{definition}

	\begin{definition}[Covariant tensor field]\label{def:covariant-tensor-field}
		The \emph{covariant tensor field} \( \Gamma (T_s^r \mathcal{M} ) \coloneqq \left\{ \text{\((r,s)\)-tensor fields on } \mathcal{M} \right\} \) with \(T_s^r \mathcal{M} \) is the \hyperref[def:section]{section} of \(T \mathcal{M} \otimes \ldots \otimes T \mathcal{M} \otimes T^{\ast} \mathcal{M} \otimes \ldots \otimes T^{\ast} \mathcal{M} \).
	\end{definition}
\end{definition*}

\begin{eg}
	A \hyperref[def:Riemannian-metric]{Riemannian metric} \(g\) on \(\mathcal{M} \) is a \hyperref[def:tensor-field]{\((0, 2)\)-tensor field}, i.e., \(g\in \Gamma (T_2^0 (\mathcal{M} ))\) for all \(p \in \mathcal{M} \).
\end{eg}
\begin{explanation}
	Since \(g_p \colon T_p \mathcal{M} \times T_p \mathcal{M} \to \mathbb{R} \).
\end{explanation}

\section{Metrics, Connections and Curvatures}
\subsection{Metrics}
We now discuss some other metrics on a \hyperref[def:smooth-manifold]{manifold}.

\begin{definition}[Pseudo-Riemannian metric]\label{def:pseudo-Riemannian-metric}
	A \emph{pseudo-Riemannian metric} on a \hyperref[def:smooth-manifold]{differentiable manifold} \(\mathcal{M} \) is a \hyperref[def:tensor-field]{tensor field} \(g\in T_2^0 (\mathcal{M}) \) with
	\begin{enumerate}[(a)]
		\item \(g(X, Y) = g(Y, X)\) for all \(X, Y\in T \mathcal{M} \).
		\item For all \(p\in \mathcal{M} \), \(g_p\) is non-degenerate bilinear form on \(T_p \mathcal{M} \), i.e., \(g_p(X, Y) = 0\) for all \(X, Y\in T_p \mathcal{M} \) if and only if \(Y = 0\).
	\end{enumerate}
\end{definition}

\begin{definition}[Lorentzian metric]\label{def:Lorentzian-metric}
	A \emph{Lorentzian metric} \(g\) is a continuous assignment of a non-degenerate\footnote{\(g_p(X, Y) = 0\) for all \(Y\in T_p \mathcal{M} \) implies \(X = 0\).} quadratic form \(g_p\) of index \(1\)\footnote{It means that the maximal dimension of a subspace of \(T_p \mathcal{M} \) on which \(g_p\) is negative definite is \(1\).} in \(T_p \mathcal{M} \) for all \(p\in \mathcal{M} \).
\end{definition}

An equivalent definition is the following.

\begin{definition}[Lorentzian]\label{def:Lorentzian}
	A quadratic form \(g_p\) in \(T_{p} \mathcal{M} \) is \emph{Lorentzian} if there exists a vector \(V\in T_p \mathcal{M} \) such that \(g_p(V, V) < 0\) while setting \(\Sigma _V = \left\{ X \mid g_p(X, V) = 0 \right\}\) such that \(\at{g_p}{\Sigma _V}{} \)\footnote{The \(g_p\)-orthogonal complement of \(V\).} is positive definite.
\end{definition}

\subsection{Connections}

\begin{definition}[Linear connection]\label{def:linear-connection}
	A \emph{linear connection} (\emph{covariant derivative}) \(\nabla \) (or \(D\)) on \(T\mathcal{M} \) is a bilinear map
	\[
		\nabla \colon \Gamma (T \mathcal{M} ) \times \Gamma (T \mathcal{M} ) \to \Gamma (T \mathcal{M} ),
	\]
	and we write \(\nabla (X, Y) = \nabla _X Y\) with
	\begin{enumerate}[(a)]
		\item \(\nabla _{fX}Y = f \nabla _X Y\);
		\item \(\nabla _X fY = X(f)Y + f;\nu _X Y\) for all \hyperref[def:vector-field]{vector fields} \(X, Y\in \Gamma (T \mathcal{M} )\), \(f\in C^{\infty} (\mathcal{M} )\).
	\end{enumerate}
\end{definition}


\begin{definition}[Torsion tensor]\label{def:torsion-tensor}
	Given \(\nabla \), the map \(T\colon \Gamma (T \mathcal{M} )\times \Gamma (T \mathcal{M} )\to \Gamma (T \mathcal{M} )\) such that \(T(X, Y) \coloneqq \nabla _X Y - \nabla _Y X - [X, Y]\) is the \emph{torsion tensor} of \(\nabla \).
\end{definition}

\begin{definition}[Torsion-free]\label{def:torsion-free}
	Given \(\nabla \), if the \hyperref[def:torsion-tensor]{torsion tensor} \(T= 0\), then we say \(\nabla \) is \emph{torsion-free}.
\end{definition}

\begin{definition}[Metric connection]\label{def:metric-connection}
	Given \(\nabla \), if \(g\) is a \hyperref[def:Riemannian-metric]{Riemannian metric} \(\mathcal{M} \), then \(\nabla \) is called \emph{metric} (or \emph{Riemannian}) if
	\[
		Z_g((X, Y)) = (\nabla _Z X, Y) + g(X, \nabla _Z Y)
	\]
	for all \(X, Y, Z\in \Gamma (T \mathcal{M} )\).
\end{definition}

\begin{proposition}[Koszul formula]\label{prop:Koszul-formula}
	On each Riemannian manifold \((\mathcal{M} , g)\), there exists a unique \hyperref[def:metric-connection]{metric}, \hyperref[def:torsion-free]{torsion-free} \hyperref[def:linear-connection]{connection} \(\nabla \) on \(T\mathcal{M} \) determined by the \emph{Koszul formula}
	\begin{equation}\label{eq:Koszul-formula}
		\left\langle \nabla _X Y, Z \right\rangle = \frac{1}{2} \left( X \left\langle Y, Z \right\rangle - Z\left\langle X, Y \right\rangle + Y\left\langle Z, X \right\rangle - \left\langle X, [Y, Z] \right\rangle + \left\langle Z, [X, Y] \right\rangle + \left\langle Y, [Z, X] \right\rangle \right).
	\end{equation}
\end{proposition}
\begin{proof}
	Firstly, we prove that for each \hyperref[def:metric-connection]{metric} and torsion-free connection satisfies \autoref{eq:Koszul-formula}. Then it will imply uniqueness. As for existence, we verify that the unique \(\mathbb{R} \)-bilinear map
	\[
		\nabla \colon \Gamma (T \mathcal{M} ) \times \Gamma (T \mathcal{M} )\to \Gamma (T\mathcal{M} )
	\]
	given by \autoref{eq:Koszul-formula} has the desired properties, i.e., \(2\) product rules from connection, torsion-free, and being metric.
\end{proof}

\begin{remark}
	This is called the Levi-Civita connection.
\end{remark}

\begin{definition}[Riemannian curvature tensor]\label{def:Riemannian-curvature-tensor}
	Let \(\nabla \) be the Levi-Civita connection on \(T \mathcal{M} \). Then the \emph{Riemannian curvature tensor} \(R\colon \Gamma (T \mathcal{M} ) \times \Gamma (T \mathcal{M} )\times \Gamma (T \mathcal{M} ) \to \Gamma (\mathcal{M} )\) is defined by
	\[
		R(X, Y)Z \coloneqq \nabla _X \nabla _Y Z - \nabla _Y \nabla _X Z - \nabla _{[X, Y]}Z.
	\]
\end{definition}