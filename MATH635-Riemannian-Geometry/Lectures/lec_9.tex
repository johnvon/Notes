\lecture{9}{2 Feb. 14:30}{Tensors and Connections}
\subsection{Vector Fields}
We can now introduce the notion of \hyperref[def:vector-field]{vector fields} in terms of \hyperref[def:section]{section}.

\begin{definition}[Vector field]\label{def:vector-field*}
	A (smooth) \emph{vector field} \(X\) is a smooth \hyperref[def:section]{section} of a \hyperref[def:bundle]{bundle}.
\end{definition}

\begin{note}
	We see that a smooth \hyperref[def:vector-field]{tangent vector field} is indeed a smooth \hyperref[def:vector-field*]{vector field} with the \hyperref[def:bundle]{bundle} being the \hyperref[def:tangent-bundle]{tangent bundle}.
\end{note}

\begin{notation}
	Since we will nearly always be talking about \hyperref[def:vector-field]{tangent vector fields}, we will abuse the notation a bit and just simply call it \hyperref[def:vector-field]{vector fields}. But always keep in mind that more broadly, a \hyperref[def:vector-field*]{vector field} should be a \hyperref[def:section]{section} of a \hyperref[def:bundle]{bundle}, not always \(T \mathcal{M} \).
\end{notation}

\subsection{Tensor Fields}
We can introduce the notion of ``\hyperref[def:tensor-field]{tensor fields}'' in a brute-fore way. To do this, given a vector space \(V\), we first introduce \hyperref[def:tensor]{tensors}.

\begin{definition}[Tensor]\label{def:tensor}
	Let \(V\) be a vector space of dimension \(m < \infty \), and the dual space \(V^{\ast}\).\footnote{I.e., \(V^{\ast} \coloneqq \left\{ \lambda \colon V \to \mathbb{R} \mid \lambda \text{ linear} \right\} \).} Then the vector space of the \emph{\(r\)-times contravariant and \(s\)-times covariant tensors over \(V\)}, denoted as \(T_s^r(V) \), is the \hyperref[def:vector-field]{vector field} defined as
	\[
		T_s^r(V) = \{ T\colon \underbrace{V^{\ast} \times \ldots \times V^{\ast} }_{r} \times \underbrace{V\times \ldots \times V}_{s} \to \mathbb{R} \}
		= \underbrace{V \otimes \ldots \otimes V}_{r} \otimes \underbrace{V^{\ast} \otimes \ldots \otimes V^{\ast} }_{s}.
	\]
\end{definition}

\begin{notation}
	Let \(\mathcal{M} ^n \) be a \hyperref[def:smooth-manifold]{smooth manifold}, and \(\pi \colon E \to \mathcal{M} \) a \hyperref[def:vector-bundle]{smooth vector bundle}, then
	\[
		\Gamma (E) \coloneqq \left\{ s\in C^{\infty} (\mathcal{M} , E) \mid \pi \circ s = \identity_{\mathcal{M} } \right\}.
	\]
\end{notation}

\begin{eg}
	Consider the \hyperref[def:vector-bundle]{vector bundle} \((T\mathcal{M} , \pi , \mathcal{M} )\), then \(\Gamma (T\mathcal{M} ) \coloneqq \left\{ \text{\hyperref[def:vector-field]{vector fields} on } \mathcal{M} \right\} \).
\end{eg}

\begin{eg}
	\(\Gamma (\Lambda _s \mathcal{M} ) \coloneqq \left\{ \text{\(s\)-forms on } \mathcal{M} \right\} \) with \(\Lambda _s \mathcal{M} = \Lambda ^s\left( \bigcup_{p\in \mathcal{M} } T^{\ast} _p \mathcal{M}  \right) \).\footnote{Here, \(\Lambda ^s(V^{\ast} ) \coloneqq \left\{ A\in T_s^0(V) \mid A \text{ skew-symmetric} \right\} \), where \(s\in \mathbb{N} \).}
\end{eg}

Then, we have the following.

\begin{definition}[Tensor field]\label{def:tensor-field}
	The \emph{\((r, s)\)-tensor fields} on \(\mathcal{M} \) is defined as elements in \(\Gamma (T_s^r \mathcal{M} )\) with \(T_s^r \mathcal{M} = \bigcup_{p\in \mathcal{M} }T^r_s(T_p \mathcal{M} )\).
\end{definition}

\begin{eg}
	A \hyperref[def:Riemannian-metric]{Riemannian metric} \(g\) on \(\mathcal{M} \) is a \hyperref[def:tensor-field]{\((0, 2)\)-tensor field}, i.e., \(g\in \Gamma (T_2^0 (\mathcal{M} ))\) for all \(p \in \mathcal{M} \).
\end{eg}
\begin{explanation}
	Since \(g_p \colon T_p \mathcal{M} \times T_p \mathcal{M} \to \mathbb{R} \).
\end{explanation}

\begin{remark}
	One might ask why \autoref{def:tensor} is not done in the same way as \hyperref[def:vector-field*]{vector fields}? Indeed, we can! See \autoref{subsection:C-infty-module-viewpoint-of-tensor-fields}.
\end{remark}

\section{Metrics, Connections and Curvatures}
\subsection{Metrics}
We now discuss some other metrics on a \hyperref[def:smooth-manifold]{manifold}.

\begin{definition}[Pseudo-Riemannian metric]\label{def:pseudo-Riemannian-metric}
	A \emph{pseudo-Riemannian metric} on a \hyperref[def:smooth-manifold]{differentiable manifold} \(\mathcal{M} \) is a \hyperref[def:tensor-field]{tensor field} \(g\in T_2^0 (\mathcal{M}) \) with
	\begin{enumerate}[(a)]
		\item \(g(X, Y) = g(Y, X)\) for all \(X, Y\in T \mathcal{M} \).
		\item For all \(p\in \mathcal{M} \), \(g_p\) is non-degenerate bilinear form on \(T_p \mathcal{M} \), i.e., \(g_p(X, Y) = 0\) for all \(X, Y\in T_p \mathcal{M} \) if and only if \(Y = 0\).
	\end{enumerate}
\end{definition}

\begin{definition}[Lorentzian metric]\label{def:Lorentzian-metric}
	A \emph{Lorentzian metric} \(g\) is a continuous assignment of a non-degenerate\footnote{\(g_p(X, Y) = 0\) for all \(Y\in T_p \mathcal{M} \) implies \(X = 0\).} quadratic form \(g_p\) of index \(1\)\footnote{It means that the maximal dimension of a subspace of \(T_p \mathcal{M} \) on which \(g_p\) is negative definite is \(1\).} in \(T_p \mathcal{M} \) for all \(p\in \mathcal{M} \).
\end{definition}

An equivalent definition is the following.

\begin{definition}[Lorentzian]\label{def:Lorentzian}
	A quadratic form \(g_p\) in \(T_{p} \mathcal{M} \) is \emph{Lorentzian} if there exists a vector \(V\in T_p \mathcal{M} \) such that \(g_p(V, V) < 0\) while setting \(\Sigma _V = \left\{ X \mid g_p(X, V) = 0 \right\}\) such that \(\at{g_p}{\Sigma _V}{} \)\footnote{The \(g_p\)-orthogonal complement of \(V\).} is positive definite.
\end{definition}

\begin{eg}[Minkowski space]
	The Minkowski space on \(\mathbb{R} ^4\) is the prototypical example from physics (flat spacetime). Namely, the metric is given by the quadratic form
	\[
		\begin{bmatrix}
			-1 & 0 & 0 & 0 \\
			0  & 1 & 0 & 0 \\
			0  & 0 & 1 & 0 \\
			0  & 0 & 0 & 1 \\
		\end{bmatrix}
	\]
	with the coordinates being \((t, x, y, z)\).
\end{eg}

\subsection{Connections}
We now can talk about the notion of \emph{connections}.

\begin{definition}[Linear connection]\label{def:linear-connection}
	A \emph{linear connection} (\emph{covariant derivative}) \(\nabla \) (or \(D\)) on \(T\mathcal{M} \) is a bilinear map
	\[
		\nabla \colon \Gamma (T \mathcal{M} ) \times \Gamma (T \mathcal{M} ) \to \Gamma (T \mathcal{M} ),
	\]
	and for all \hyperref[def:vector-field]{vector fields} \(X, Y\in \Gamma (T \mathcal{M} )\) and \(f\in C^{\infty} (\mathcal{M} )\), we write \(\nabla (X, Y) = \nabla _X Y\) with
	\begin{enumerate}[(a)]
		\item \(\nabla _{fX}Y = f \nabla _X Y\);
		\item \(\nabla _X fY = X(f)Y + f \nabla _X Y\).
	\end{enumerate}
\end{definition}

\begin{definition}[Torsion tensor]\label{def:torsion-tensor}
	Let \(\nabla \) be a \hyperref[def:linear-connection]{linear connection}. The map \(T\colon \Gamma (T \mathcal{M} )\times \Gamma (T \mathcal{M} )\to \Gamma (T \mathcal{M} )\) defined as
	\[
		T(X, Y) \coloneqq \nabla _X Y - \nabla _Y X - [X, Y]
	\]
	is the \emph{torsion tensor} of \(\nabla \).
\end{definition}

\begin{definition*}
	Let \(\nabla \) be a \hyperref[def:linear-connection]{linear connection}.

	\begin{definition}[Torsion-free]\label{def:torsion-free}
		\(\nabla \) is \emph{torsion-free} if \(T = 0\).
	\end{definition}

	\begin{definition}[Riemannian]\label{def:Riemannian}
		If \(g\) is a \hyperref[def:Riemannian-metric]{Riemannian metric} on \(\mathcal{M} \), then \(\nabla \) is \emph{Riemannian} (metric) if
		\[
			Z(g(X, Y)) = g(\nabla _{Z} X, Y) + g(X, \nabla _{Z} Y)
		\]
		for all \(X, Y, Z\in \Gamma (T\mathcal{M} )\).
	\end{definition}
\end{definition*}

\begin{proposition}[Koszul formula]\label{prop:Koszul-formula}
	On each Riemannian manifold \((\mathcal{M} , g)\), there exists a unique \hyperref[def:Riemannian]{Riemannian}, \hyperref[def:torsion-free]{torsion-free} \hyperref[def:linear-connection]{connection} \(\nabla \) on \(T\mathcal{M} \) determined by the \emph{Koszul formula}
	\begin{equation}\label{eq:Koszul-formula}
		\left\langle \nabla _X Y, Z \right\rangle = \frac{1}{2} \left( X \left\langle Y, Z \right\rangle - Z\left\langle X, Y \right\rangle + Y\left\langle Z, X \right\rangle - \left\langle X, [Y, Z] \right\rangle + \left\langle Z, [X, Y] \right\rangle + \left\langle Y, [Z, X] \right\rangle \right).
	\end{equation}
\end{proposition}
\begin{proof}
	Firstly, we prove that for each \hyperref[def:Riemannian]{Riemannian} and \hyperref[def:torsion-free]{torsion-free} \hyperref[def:linear-connection]{connection} satisfies \autoref{eq:Koszul-formula}, then it will imply uniqueness. As for existence, we verify that the unique \(\mathbb{R} \)-bilinear map
	\[
		\nabla \colon \Gamma (T \mathcal{M} ) \times \Gamma (T \mathcal{M} )\to \Gamma (T\mathcal{M} )
	\]
	given by \autoref{eq:Koszul-formula} has the desired properties, i.e., \(2\) product rules from \hyperref[def:linear-connection]{connection}, \hyperref[def:torsion-free]{torsion-free}, and being \hyperref[def:Riemannian-metric]{metric}.
\end{proof}

\begin{definition}[Levi-Civita connection]\label{def:Levi-Civita-connection}
	The \emph{Levi-Civita connection} is the unique \hyperref[def:linear-connection]{linear connection} \(\nabla \) defined by the \hyperref[eq:Koszul-formula]{Koszul formula}.
\end{definition}

\begin{notation}
	We often write \(\mathrm{D} \) and \(\nabla \) interchangeably.
\end{notation}

\begin{definition}[Riemannian curvature tensor]\label{def:Riemannian-curvature-tensor}
	Let \(\nabla \) be the \hyperref[def:Levi-Civita-connection]{Levi-Civita connection} on \(T \mathcal{M} \). Then the \emph{Riemannian curvature tensor} \(R\colon \Gamma (T \mathcal{M} ) \times \Gamma (T \mathcal{M} )\times \Gamma (T \mathcal{M} ) \to \Gamma (T \mathcal{M} )\) of \(\nabla \) is defined by
	\[
		R(X, Y)Z \coloneqq \nabla _X \nabla _Y Z - \nabla _Y \nabla _X Z - \nabla _{[X, Y]}Z.
	\]
\end{definition}