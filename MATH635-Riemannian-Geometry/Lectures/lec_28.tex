\lecture{28}{18 Apr. 13:00}{Epilogue}
\begin{proof}[Proof of \autoref{thm:uniformization}]
	We now start our proof following the \(5\) steps.

	\begin{enumerate}
		\item This step is trivial.
		\item Use the \hyperref[def:exponential-map]{exponential map}. Let \(r_p\) be the \hyperref[def:injectivity-radius]{injectivity radius} (relative to \((\mathcal{M} , g)\)) of \(\exp _p\). Choose \(\epsilon >0\) such that \(2 \epsilon < r_p\). Let \(\rho \) be a \(C^{\infty} \) non-increasing function on \([0, \infty )\) such that \(\rho = 1\) on \([0, 1]\) and \(\rho = 0\) on \([2, \infty )\).
		      \begin{center}
			      \incfig{non-increasing-function}
		      \end{center}
		      Define the cur-off function for \(q\in \mathcal{M} \) such that
		      \[
			      \eta (q) = \begin{dcases}
				      \rho (d_p / \epsilon ), & \text{ if } q \in B_{2\epsilon }(p) ; \\
				      0,                      & \text{ otherwise} .
			      \end{dcases}
		      \]
		      Namely, \(\eta = 1\) on \(B_{\epsilon } (p)\) and \(\eta = 0\) on \(\mathcal{M} \setminus B_{2\epsilon }(p)\). Define on \(\mathcal{M} \) the function
		      \[
			      w_o = \begin{dcases}
				      -2 \eta \log d_p, & \text{ in } B_{2\epsilon }(p)  ;                       \\
				      0,                & \text{ on } \mathcal{M} \setminus B_{2\epsilon }(p)  .
			      \end{dcases}
		      \]
		      In \(B_{\epsilon } (p)\), \(w_0 = - 2 \log d_p\). As \(2\epsilon < r_p\), \(\exp _p\) is a \hyperref[def:diffeomorphism]{diffeomorphism} of the ball of radius \(2\epsilon \) with center \(0\) in \(T_p \mathcal{M} \) onto \(B_{2\epsilon } (p)\) in \(\mathcal{M} \). Consider choosing a polar \hyperref[def:normal-coordinate]{normal coordinate} \((r, \theta )\) in \(R_{2 \epsilon }(p)\) such that \(\mathrm{d} \rho = r\) and \(g = \mathrm{d} r^2 + R^2(r, \theta )\mathrm{d} \theta ^2\) such that
		      \[
			      \int_{0}^{\pi } R(r, \theta ) \,\mathrm{d}\theta = L(r)
		      \]
		      be the perimeter of the \hyperref[def:geodesic]{geodesic} circles. Then \(L(r) / r \to 2\pi \) as \(r \to 0\) by local euclidicity at \(p\).

		      \begin{notation}[Geodesic curavture]
			      The \emph{geodesic curvature} of circles \(\kappa \) is defined as \(\kappa = \frac{1}{R} \frac{\partial R}{\partial r} \).
		      \end{notation}
		      We have
		      \[
			      \frac{\partial \kappa }{\partial r} = - \kappa ^2 - K.
		      \]
		      Now, express the Laplace operator \(\Delta \) in polar coordinates.

		      \begin{remark}
			      In arbitrary coordinates, we have
			      \[
				      \Delta _g = \frac{1}{\sqrt{\det g} } \frac{\partial }{\partial x^a} \cdot \sqrt{\det g} (g^{-1} )^{ab} \frac{\partial }{\partial x^b} .
			      \]
		      \end{remark}

		      Then in polar coordinates,
		      \[
			      \Delta _g = \frac{1}{R}\frac{\partial }{\partial r} \cdot R \frac{\partial }{\partial r} + \frac{1}{r}\frac{\partial }{\partial \theta } \cdot \frac{1}{R}\frac{\partial }{\partial \theta } ,
		      \]
		      and in \(B_{\epsilon (p)} \), we have \(\omega _0=-2\log r\) and
		      \[
			      \Delta _g \omega _0 = - \frac{2}{R} \frac{\partial R}{\partial r} \left( \frac{R}{r} \right) = - \frac{2\lambda }{r}
		      \]
		      with \(\lambda = \frac{1}{R} \frac{\partial R}{\partial r} - \frac{1}{r} = \kappa - \frac{1}{r}\). Hence,
		      \[
			      \underbrace{\frac{\partial \kappa }{\partial r} + \kappa ^2}_{-K } = \frac{\partial \lambda }{\partial r} + \frac{2\lambda }{r}+ \lambda ^2 .
		      \]
		      Set \(\mu = r^2 \lambda \), it becomes
		      \[
			      \frac{\partial \mu }{\partial r} + \frac{\mu ^2}{r^2} = -r^2 K.
		      \]
		      Since as \(r \to 0\), \(\mu \to 0\), hence along each ray, the integral equation \(\mu (r, \theta )\) is
		      \[
			      \mu (r, \theta )
			      = - \int_{0}^{r} \frac{\mu (r^{\prime} , \theta )^2}{{r^{\prime} }^2} + {r^{\prime} }^2 K (r^{\prime} , \theta )  \,\mathrm{d}r^{\prime},
		      \]
		      and we have \(\lambda (r, \theta ) = O(r)\). Moreover, \(\lambda / r \to - K_p / 3\) as \(r \to 0\). It follows that \(\Delta _g \omega _0\) is bounded and \(\Delta _g \omega _0 \to 2K_p / 3\) as approaching \(p\).

		      Now, set \(\omega = \omega _0 + \omega _1\), then \(\omega _1\) has to satisfy
		      \[
			      \Delta _g \omega _1
			      = \Delta _g \omega - \Delta _g \omega _0
			      = K - \Delta _g \omega _0
		      \]
		      on \(\mathcal{M} \setminus p\). Let \(f = K - \Delta _g \omega _0\) be a function on \(\mathcal{M} \). We can show that \(f\) extends to a continuous function on \(\mathcal{M} \).

		      \begin{claim}
			      There is a solution \(w_1\) of \(\Delta _g w_1 = f\) unique up to an additive constant, provided that
			      \[
				      \int _\mathcal{M} f \,\mathrm{d} \mu _g = 0.
			      \]
		      \end{claim}
		      \begin{explanation}
			      To prove this, we integrate \(f\) on \(\mathcal{M} \setminus B_\delta (P)\) with \(0 < \delta \leq \epsilon \), i.e.,
			      \[
				      -\int _{\mathcal{M} \setminus B_{\epsilon (p)} }\Delta _g w_0 \,\mathrm{d} \mu _g = \int _{\partial B_\delta (p)} \nabla _N w_0 \,\mathrm{d} s,
			      \]
			      where \(\mathrm{d} s\) is the element of arc length of \(\partial B_\delta (p)\). In \(\overline{B} _\delta (p)\) we have, in polar coordinates, \(w_0 = -2 \log r\) and \(\nabla _N= \partial / \partial r\), so \(\nabla _N w_0 = -2 / r\). Moreover, it is \(\mathrm{d} s = R\, \mathrm{d} \theta \). So, we have
			      \[
				      \int _{\partial B_\delta (p)}\nabla _N w_0 \,\mathrm{d} s
				      = -\frac{2}{\delta } \int_{0}^{2\pi } R(\delta , \theta ) \,\mathrm{d}\theta \to -4\pi \text{ as } \delta \to 0.
			      \]
			      On the other hand,
			      \[
				      \lim_{\delta \to 0} \int _{\mathcal{M} \setminus B_\delta (p)}K \,\mathrm{d} \mu _g = \int _\mathcal{M} K \,\mathrm{d} \mu _g = 4\pi
			      \]
			      by \hyperref[thm:Gauss-Bonnet]{Gauss-Bonnet}. We conclude that indeed \(\int _\mathcal{M} f \,\mathrm{d} \mu _g = 0\).
		      \end{explanation}
		      So, the equation is solvable for \(w_1\). In fact, we can show that \(w_1\) is bounded on \(\mathcal{M} \) and is in fact continuous.\footnote{For, \(f=\Delta _g w_1\) being bounded, in particular \(f\in L^2(\mathcal{M} )\) implies \(w_1\in H_2(\mathcal{M} )\), hence \(w_1\) is bounded.}
		\item Step \(3\) is also trivial.
		\item Now, it is \(u = w+v\) where \(w = w_0 + w_1\), and \(w_1\) is bounded on \(\mathcal{M} \), while \(w_0 = -2 \eta \log d_p\) and \(d_p\) is the \(g\)-distance from \(p\). On the other hand,
		      \[
			      e^v = \frac{1}{1 + \widetilde{d} ^2_o / 4},
		      \]
		      where \(\widetilde{d} _o\) is the \(\widetilde{g} \)-distance from \(o\). Hence,
		      \[
			      v = -2 \log \widetilde{d} _o + O(1).
		      \]
		\item It follows that \(u\) is bounded on \(\mathcal{M} \) if and only if in \(B_{\epsilon (p)} \) (relative to \(g\) )\(d_p\cdot \widetilde{d} _o\) is bounded above and below by positive constants.\footnote{For detail, see the note.}
	\end{enumerate}
\end{proof}

\subsection{Yamabe Problem}
From the proof of \hyperref[thm:uniformization]{uniformization theorem}, the following problem arises.

\begin{problem}[Yamabe poroblem]\label{prb:Yamabe}
Given a compact \hyperref[def:Riemannian-manifold]{Riemannian manifold} \(\mathcal{M} , g\) of dimension \(n \geq 3\). Find a \hyperref[def:Riemannian-metric]{metric} \(\widetilde{g} \) \hyperref[def:conformal]{conformal} to \(g\) such that the \hyperref[def:Ricci-scalar-curvature]{scalar curvature} of \(\widetilde{g} \) is constant.
\end{problem}

If \(\mathcal{M} \) has no boundary, then Aubin~\cite{Aubin1976ProblmesIE, Aubin1976EquationsDN}, Schoen~\cite{Schoen1984ConformalDO}, Trudinger~\cite{ASNSP_1968_3_22_2_265_0} solves it. With boundary, Escober~\cite{escobar1992yamabe}. If we write \(g = u^{\frac{4}{n-2}} g_0\), the \hyperref[def:Ricci-scalar-curvature]{scalar curvature} \(R_g\) is
\begin{equation}\label{eq:lec28}
	R_g
	= u^{-\frac{u+2}{n-2}} \left( - \frac{4(n-1)}{n-2} \Delta _{g_0} u + R_g u \right) .
\end{equation}
\(g\) has constant \hyperref[def:Ricci-scalar-curvature]{scalar curvature} \(c\) if and only if \(u\) is a solution of the \emph{Yamabe equation}
\begin{equation}\label{eq:Yamabe}
	\frac{4(n-1)}{n-2} \Delta _{g_0} u - R_{g_0} u + cu^{\frac{n+2}{n-2}} = 0.
\end{equation}

To solve this, consider the variational approach, where we have the following.

\begin{definition}[Einstein-Hilbert action]\label{def:Einstein-Hilbert-action}
	The \emph{Einstein-Hilbert action} \(\mathcal{E} (g)\) is defined as
	\[
		\mathcal{E} (g) = \cfrac{\int _\mathcal{M} R_g \,\mathrm{d} \mathop{\mathrm{vol}} _g}{\mathop{\mathrm{vol}}(\mathcal{M} , g)^{\frac{n-2}{n}} }.
	\]
\end{definition}

\begin{definition}[Einstein metric]\label{def:Einstein-metric}
	A \hyperref[def:Riemannian-metric]{metric} \(g\) is called an \emph{Einstein metric} if \(\mathop{\mathrm{Ric}}(g) = c\cdot g \) for some constant \(c\).
\end{definition}
\begin{remark}
	A \hyperref[def:Riemannian-metric]{metric} \(g\) is a critical point of \(\mathcal{E} \) if and only if \(g\) is an \hyperref[def:Einstein-metric]{Einstein metric}.
\end{remark}

Given any positive function \(u\), consider the \emph{Yamabe functional}
\[
	\mathcal{E} _{g_0}(u) = \mathcal{E} (u^{\frac{4}{n-2}} g_0).
\]
\autoref{eq:lec28} implies
\[
	\mathcal{E} _{g_0}(u) = \frac{\int _\mathcal{M} \left( \frac{4(n-1)}{n-2}\vert \mathrm{d} u_{g_0} \vert ^2 + R_{g_0}u^2 \right) \,\mathrm{d} \mathop{\mathrm{vol}}_{g_0} }{\left( \int _\mathcal{M} u^{\frac{2n}{n-2}} \,\mathrm{d} \mathop{\mathrm{vol}}_{g_0} \right) ^{\frac{n-2}{n}}}.
\]
\(u\) is a critical point if and only if \(u\) satisfies the \hyperref[eq:Yamabe]{Yamabe equation}.

\section{Lorentzian Manifolds and General Relativity}
Consider
\[
	R_{\mu \nu } - \frac{1}{2}g_{\mu \nu } R = \frac{8 \pi G}{c^4} T_{\mu \nu }.
\]

\begin{remark}
	We have \(g(X, X) = -(X^0)^2 + \sum_{i=1}^3 (X_i)^2\).
\end{remark}

\begin{definition}[Arc length]
	The \emph{arc length} of causal curve \(\gamma \) between 2 points corresponding to parameter values \(\lambda = a\) and \(\lambda = b\) is
	\[
		L[\gamma ](a, b) = \int_{a}^{b} \sqrt{-g(\dot{\gamma }(\lambda ), \dot{\gamma }(\lambda )  )} \,\mathrm{d}\lambda .
	\]
\end{definition}

If \(q\in \mathcal{J} ^+(p)\), define temporal distance \(q\) from \(p\) as \(\tau (q, p) = \sup L[\gamma ]\) over  all future-directed casual curves \((p, q)\).

\begin{prev}
	In the \hyperref[def:Riemannian-manifold]{Riemannian manifold} case, we have \hyperref[thm:Hopf-Rinow]{Hopf-Rinow theorem}.
\end{prev}

The analogous for \hyperref[def:Lorentzian-metric]{Lorentzian manifolds} is for maximization, i.e., it holds if spacetime admits Cauchy hypersurfaces when the supremum is achieved, and metric \(C^{\prime} \), maximizing curve is a causal geodesic.

\section{Ricci Flow}
The basic idea of Ricci flow is to consider the \hyperref[def:Riemannian-metric]{metric} \(g\) is changing over time, i.e., the shape of the \hyperref[def:Riemannian-manifold]{manifold} changes w.r.t.\ \(g(t)\), described by an O.D.E.\ related to the \hyperref[def:Ricci-curvature]{Ricci curvature} as
\[
	\frac{\partial g(t)}{\partial t} = -2 \mathop{\mathrm{Ric}}(g) .
\]