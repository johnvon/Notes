\chapter{Jacobi Fields}
\lecture{16}{7 Mar. 13:00}{Jacobi Field}

In this chapter, we derive a first relation between the two basic concepts introduced, i.e., \hyperref[def:geodesic]{geodesics} and \hyperref[def:Riemannian-curvature]{curvatures}. This is done by introducing \hyperref[def:Jacobi-field]{Jacobi field}: \hyperref[def:vector-field-along-curve]{vector fields along} \hyperref[def:geodesic]{geodesics}, defined by means of differential equations naturally from \hyperref[def:exponential-map]{exponential map}. Moreover, \hyperref[def:Jacobi-field]{Jacobi fields} allow us to obtain a simple characterization of the singularities of the \hyperref[def:exponential-map]{exponential map}.

\begin{intuition}
	The upshot is, as we will see, the \hyperref[def:sectional-curvature]{curvature} \(K(p, \sigma )\), \(\sigma \subseteq T_p \mathcal{M} \), determines how fast the \hyperref[def:geodesic]{geodesics}, that start from \(p\) and are tangent to \(\sigma \), spread apart. This is described by \hyperref[def:Jacobi-field]{Jacobi field}.
\end{intuition}

\section{Jacobi Fields}
As mentioned, we want to consider neighboring \hyperref[def:geodesic]{geodesics} under a \hyperref[def:vector-field-along-curve]{vector field along which}, and study how do they move. Their behaviors are essentially governed by \hyperref[def:Riemannian-curvature]{curvature}.

\begin{definition}[Jacobi field]\label{def:Jacobi-field}
	Let \(\mathcal{M} \) be a \(d\)-dimensional \hyperref[def:Riemannian-manifold]{Riemannian manifold}. Let \(c\colon I \to \mathcal{M} \) be a \hyperref[def:geodesic]{geodesic}. A \hyperref[def:vector-field-along-curve]{vector field \(X\) along \(c\)} is called a \emph{Jacobi field} if it satisfies the \emph{Jacobi equation}
	\begin{equation}\label{eq:Jacobi}
		\nabla _{\frac{\mathrm{d}}{\mathrm{d}t} }\nabla _{\frac{\mathrm{d}}{\mathrm{d}t} }X + R(X, \dot{c} )\dot{c} = 0.
	\end{equation}
\end{definition}

\begin{notation}
	We write \(\dot{X} \coloneqq \nabla _{\frac{\mathrm{d}}{\mathrm{d}t} }X\) and \(\ddot{X} \coloneqq \nabla _{\frac{\mathrm{d}}{\mathrm{d}t} } \nabla _{\frac{\mathrm{d}}{\mathrm{d}t} } X\).
\end{notation}

Using new notations, the \hyperref[eq:Jacobi]{Jacobi equation} is rewritten as
\[
	\ddot{X} + R (X, \dot{c} ) \dot{c} = 0.
\]

To understand \hyperref[def:Jacobi-field]{Jacobi field}, we first recall the \hyperref[prev:variation]{variation}.

\begin{prev}[Variation]\label{prev:variation}
	For some \(\epsilon > 0\), the \emph{variation} of a \hyperref[def:curve]{smooth curve} \(c\colon [a, b] \to \mathcal{M} \) is a differentiable map \(F\colon [a, b] \times (-\epsilon , \epsilon ) \to \mathcal{M}\) such that \(F(t, 0) = c(t)\) for \(t\in [a, b]\) with \(s\in (-\epsilon , \epsilon )\).
\end{prev}

Essentially, a \hyperref[def:Jacobi-field]{Jacobi field} studies the \hyperref[def:geodesic-variation]{variation of geodesics}: we can label \hyperref[def:geodesic]{geodesics} \(c\) as
\begin{center}
	\incfig{2nd-variation}
\end{center}

\begin{notation}[Proper variation]\label{not:proper-variation}
	A \emph{proper variation} is a \hyperref[prev:variation]{variation} where the endpoints are fixed, i.e., \(F(a, s) = c(a)\) and \(F(b, s)=c(b)\) for all \(s\in (-\epsilon , \epsilon )\).
\end{notation}

\begin{note}
	We might either fix the endpoints or left them open, i.e., we can consider both \hyperref[not:proper-variation]{proper} and non-\hyperref[not:proper-variation]{proper} cases.
\end{note}

\begin{intuition}\label{int:Jacobi-equation}
	The \hyperref[eq:Jacobi]{Jacobi equation} can be viewed as the linearization of the \hyperref[eq:geodesic]{geodesic equation}.
\end{intuition}

Formally, we define the following.

\begin{definition}[Geodesic variation]\label{def:geodesic-variation}
	Let \(\mathcal{M} \) be a (\hyperref[def:pseudo-Riemannian-metric]{semi}-)\hyperref[def:Riemannian-manifold]{Riemannian manifold}. A \hyperref[prev:variation]{variation} of \hyperref[def:curve]{curves} \(c\colon I \times (-\epsilon , \epsilon ) \to \mathcal{M} \) is called a \emph{geodesic variation} if for all \(s\in (-\epsilon , \epsilon )\), the \hyperref[def:curve]{curve} \(t\mapsto c_s(t) \coloneqq c(t, s)\) is a \hyperref[def:geodesic]{geodesic}.
\end{definition}

\begin{notation}
	We set \(c_s(t) = c(t, s)=F(t, s)\), and
	\begin{itemize}
		\item \(\dot{c}(t, s) \coloneqq \frac{\partial }{\partial t} c(t, s) \), i.e., \(\mathrm{d} F(\partial / \partial t) c(t, s)\);
		\item \(c^\prime (t, s) \coloneqq \frac{\partial}{\partial s} c(t, s)\), i.e., \(\mathrm{d} F (\partial / \partial s)c(t, s)\).
	\end{itemize}
\end{notation}

\section{Variations of Length and Energy}
Recall the following.

\begin{prev}
	Given a \hyperref[prev:variation]{variation} of a \hyperref[def:geodesic]{geodesic} \(c_s(t)\), The \hyperref[def:energy]{energy} for \(c_s\) is defined as
	\[
		E(s) \coloneqq \frac{1}{2} \int_{a}^{b} \left\langle \frac{\partial c(t, s)}{\partial t} , \frac{\partial c(t, s)}{\partial t}  \right\rangle  \,\mathrm{d}t,
	\]
	and the \hyperref[def:length]{length} for \(c_s\) is defined as
	\[
		L(s) \coloneqq \int_{a}^{b} \left\langle \frac{\partial c(t, s)}{\partial t} , \frac{\partial c(t, s)}{\partial t}  \right\rangle ^{1 / 2} \,\mathrm{d}t.
	\]
\end{prev}

And we want to compute
\begin{itemize}
	\item the first \hyperref[prev:variation]{variations} \(E^\prime (0) \) and \(L^\prime (0)\), i.e., the first derivatives;
	\item for \(c = c_0\) \hyperref[def:geodesic]{geodesic}, compute the second \hyperref[prev:variation]{variations} \(E^{\prime\prime}(0) \) and \(L^{\prime\prime}(0)\), i.e., the second derivatives.
\end{itemize}
\subsection{First Variations}
Let's consider the first \hyperref[prev:variation]{variations}, i.e., \(E^{\prime} (0)\) and \(L^{\prime} (0)\).

\begin{lemma}\label{lma:1st-variation-geodesic}
	If \(L(s)\), \(E(s)\) are differentiable w.r.t.\ \(s\), then
	\[
		L^\prime (0) = \int_{a}^{b} \left( \frac{\frac{\partial }{\partial t} \left\langle c^\prime , \dot{c}  \right\rangle }{\left\langle \dot{c} , \dot{c} \right\rangle ^{1 / 2}} - \frac{\left\langle c^\prime , \nabla _{\frac{\partial }{\partial t} } \dot{c} \right\rangle }{\left\langle \dot{c} , \dot{c} \right\rangle ^{1 / 2}}\right)  \,\mathrm{d}t,
	\]
	and
	\[
		E^\prime (0) = \left\langle c^\prime (b, 0), \dot{c} (b, 0) \right\rangle - \left\langle c^\prime (a, 0), \dot{c} (a, 0) \right\rangle - \int_{a}^{b} \left\langle \frac{\partial c}{\partial s} , \nabla _{\frac{\partial }{\partial t} }\frac{\partial c}{\partial t} (t, s) \right\rangle  \,\mathrm{d}t.
	\]
\end{lemma}
\begin{proof}
	We have already proved this in different notations.
\end{proof}

\begin{note}
	If \(c = c_0\) is parametrized proportionally to the arc-length, i.e., \(\lVert \dot{c} (t, 0) \rVert \) is a constant. Then \(L^\prime (0)\) becomes
	\[
		L^\prime (0) = \frac{1}{\left\langle \dot{c} , \dot{c}  \right\rangle ^{1 / 2}} \left( \at{\left\langle c^\prime , \dot{c} \right\rangle }{t=a, s=0}{t=b, s=0} - \int_{a}^{b} \left\langle c^\prime , \nabla _{\frac{\partial }{\partial t} }\dot{c}  \right\rangle  \,\mathrm{d}t \right).
	\]
\end{note}

If we consider the fixed endpoints case (i.e., \hyperref[not:proper-variation]{proper variation}), we observe that \(E\) and \(L\) are stationary if and only if
\[
	\nabla _{\frac{\partial }{\partial t}} \dot{c} (t, 0) = 0,
\]
i.e., when \(c\) is a \hyperref[def:geodesic]{geodesic}.

\subsection{Second Variations}
Now, let \(c = c_0\) be a \hyperref[def:geodesic]{geodesic}. Then we compute the second derivatives w.r.t.\ \(s\) of \(E\) and \(L\) at \(s = 0\).

\begin{theorem}\label{thm:2nd-variation-geodesic}
	Let \(c\colon [a, b] \to \mathcal{M} \) be a \hyperref[def:geodesic]{geodesic}. Then
	\[
		E^{\prime\prime}(0)
		= \int_{a}^{b} \left\langle \nabla _{\frac{\partial }{\partial t} } c^{\prime} (t, 0), \nabla _{\frac{\partial }{\partial t} } c^{\prime} (t, 0)\right\rangle  \,\mathrm{d}t
		- \at{\int_{a}^{b} \left\langle R(\dot{c} , c^{\prime} ) c^{\prime} , \dot{c} \right\rangle \,\mathrm{d}t}{s= 0}{}
		+ \at{\left\langle \nabla _{\frac{\partial }{\partial s} } c^{\prime} , \dot{c}  \right\rangle }{t=a, s=0}{t=b, s=0} .
	\]
	By letting \(c^{\prime \perp } \coloneqq c^{\prime} - \left\langle \frac{\dot{c} }{\lVert \dot{c} \rVert } , c^{\prime} \right\rangle \frac{\dot{c} }{\lVert \dot{c} \rVert} \),\footnote{I.e., the component of \(c^{\prime} \) orthogonal to \(\dot{c} \).} we have
	\[
		L^{\prime\prime} (0)
		= \frac{1}{\lVert \dot{c} \rVert } \at{\left(
		\int_{a}^{b} \left\langle \nabla _{\frac{\partial }{\partial t} } c^{\prime\perp }, \nabla _{\frac{\partial }{\partial t} } c^{\prime\perp }\right\rangle \,\mathrm{d}t
		- \int_{a}^{b} \left\langle R(\dot{c}, c^{\prime\perp } )c^{\prime\perp }, \dot{c} \right\rangle \,\mathrm{d}t
		+ \at{\left\langle \nabla _{\frac{\partial }{\partial s} }c^{\prime} , \dot{c} \right\rangle }{t=a}{t=b} \right) }{s=0}{}.
	\]
\end{theorem}

\begin{remark}
	By keeping the endpoints fixed, if the \hyperref[def:sectional-curvature]{sectional curvature} of \(\mathcal{M} \)  is non-positive, then the \hyperref[def:Riemannian-curvature]{Riemannian curvature} in \(E^{\prime\prime} (0)\) and \(L^{\prime\prime} (0)\) are non-negative. This implies \(E^{\prime\prime} (0) > 0\), then \(E(c_s) > E(c_0)\) for small \(\vert s \vert \).
\end{remark}

\begin{corollary}
	On a manifold with non-positive \hyperref[def:sectional-curvature]{sectional curvature}, the \hyperref[def:geodesic]{geodesics} with fixed endpoints are always locally minimizing.
\end{corollary}

\section{Index Form}
\subsection{Pullback Connections}
Let \(\mathcal{M} \) be a \hyperref[def:Riemannian-manifold]{Riemannian manifold} of dimension \(d\), and \(\mathcal{H} \) be a \hyperref[def:smooth-manifold]{differentiable manifold}.\footnote{Often times, \(\mathcal{H} \) is an interval \(I\subseteq \mathbb{R} \) or a square \(I\times I \subseteq \mathbb{R} ^2\).} Let \(f\colon \mathcal{H}  \to \mathcal{M} \), smooth, and \(f\) may not be injective. We ask the following question.

\begin{problem}
What is the \hyperref[def:tangent-space]{tangent space} of \(f(\mathcal{H} )\) of point \(p\in f(\mathcal{H} )\)?
\end{problem}

We see that even if \(f\) is an \hyperref[def:immersion]{immersion}, since it can be non-injective, there maybe issues.

\begin{eg}
	Let \(p = f(x) = f(y)\) for \(x \neq y\). For \(f\) being an \hyperref[def:immersion]{immersion}, we may restrict \(f\) to a sufficiently small neighborhood \(U, V\) at \(x, y\), respectively, such that \(f(U), f(V)\) have well-defined \hyperref[def:tangent-space]{tangent spaces} at \(p\). Then, in a double point (e.g., \(p\)) of \(f(\mathcal{H} )\), the \hyperref[def:tangent-space]{tangent space} can be specified by specifying the preimage (\(x\) or \(y\)).
\end{eg}

Formally, consider \(f^{\ast} (T \mathcal{M} )\), the \hyperref[def:tangent-bundle]{tangent bundle} \(T \mathcal{M} \) \hyperref[def:pullback-bundle]{pullback} by \(f\).

\begin{note}
	The \hyperref[def:fiber]{fiber} over \(x\in \mathcal{H} \) is \(T_{f(x)} \mathcal{M} \).
\end{note}

Then, we can introduce a \hyperref[def:linear-connection]{connection} \(f^{\ast} (\nabla )\) on \(f^{\ast} (T \mathcal{M} )\): let \(X\in T_x \mathcal{H} \), \(Y\) a \hyperref[def:section]{section} of \(f^{\ast} (T \mathcal{M} )\). Set
\[
	(f^{\ast} \nabla )_X Y\coloneqq \nabla _{\mathrm{d} f(X)} Y,
\]
where \(f^{\ast} (T \mathcal{M} )_x\) is identified with \(T_{f(x)}\mathcal{M} \) with \(\nabla \) for \(f^{\ast} \nabla \).

\begin{note}
	For \(\nabla _{\mathrm{d} f(X)}Y\) to be well-defined, we need to extend \(Y\) to a neighborhood of \(f(\mathcal{H} )\).\footnote{Hence, it does not depend on the choice of extension.}
\end{note}

\begin{notation}
	Write \(\nabla \) for \(f^{\ast} \nabla \) in what follows.
\end{notation}

\subsection{Index Form}
Now, let \(f = c\colon I \to \mathcal{M} \) (often, \(c\) is a \hyperref[def:geodesic]{geodesic}), i.e., we consider \hyperref[def:vector-field-along-curve]{vector field along} \(c\).\footnote{In deed, a \hyperref[def:section]{section} of \(f^{\ast} (T \mathcal{M} )\) is a \emph{vector field along \(f\)} in general even for \(f\colon I^2 \to \mathcal{M} \).} Specifically, let \(X\) be a \hyperref[def:vector-field-along-curve]{vector field along \(c\)} where \(c\) is a \hyperref[def:geodesic]{geodesic}. Then, there exists a \hyperref[def:geodesic-variation]{geodesic variation}
\[
	c\colon [a, b] \times (-\epsilon , \epsilon )\to \mathcal{M}
\]
of \(c(t)\) with \(\at{\frac{\partial c}{\partial s} }{s=0}{} = X\). Consider the second \hyperref[prev:variation]{variation} of \hyperref[def:energy]{energy}: inspired from \autoref{thm:2nd-variation-geodesic}, we write
\[
	I(X, X)
	\coloneqq \int_{a}^{b} \left( \langle \nabla _{\frac{\partial }{\partial t} }X, \nabla _{\frac{\partial }{\partial t} } X \rangle - \left\langle R(\dot{c}, X )X, \dot{c}  \right\rangle  \right)  \,\mathrm{d}t,
\]
i.e., \(I(X, X) = \frac{\mathrm{d}^2}{\mathrm{d}s^2} E(0)\) if \(X(a) = X(b) = 0\). Moreover, instead of considering a \(1\)-parameter \hyperref[prev:variation]{variation}, we can also consider a \(2\)-parameter \hyperref[prev:variation]{variation} on \(X\) and \(Y \coloneqq \frac{\partial c}{\partial t}\). In this case, we propose the following.

\begin{definition}[Index form]\label{def:index-form}
	The \emph{index form} of a \hyperref[def:geodesic]{geodesic} \(c\) on \(X = \at{\frac{\partial c}{\partial s} }{s=0}{} \) and \(Y = \frac{\partial c}{\partial t}\) is
	\[
		I(X, Y)
		\coloneqq \int_{a}^{b} \left( \langle \nabla _{\frac{\partial }{\partial t} } X, \nabla _{\frac{\partial }{\partial t} } Y \rangle - \left\langle R(\dot{c} , X) Y, \dot{c} \right\rangle \right)  \,\mathrm{d}t.
	\]
\end{definition}

\begin{note}
	We see that \(I(X, Y)\) is a bilinear and symmetric in \(X, Y\).
\end{note}

\begin{prev}
	Recall the \hyperref[eq:Jacobi]{Jacobi equation}, i.e., \(\nabla _{\frac{\mathrm{d}}{\mathrm{d}t} }\nabla _{\frac{\mathrm{d}}{\mathrm{d}t} }X + R(X, \dot{c} )\dot{c} = 0\).
\end{prev}

\begin{proposition}[Jacobi field]\label{prop:Jacobi-field}
	A \hyperref[def:vector-field-along-curve]{vector field \(X\) along} a \hyperref[def:geodesic]{geodesic} \(c\colon [a, b] \to \mathcal{M} \) is a \hyperref[def:Jacobi-field]{Jacobi-field} if and only if the \hyperref[def:index-form]{index form} of \(c\) satisfies \(I(X, Y) = 0\) for all \hyperref[def:vector-field-along-curve]{vector fields \(Y\) along \(c\)} with \(Y(a) = Y(b) = 0\).
\end{proposition}
\begin{proof}
	Observe that
	\begin{equation}\label{eq:prop:Jacobi-field}
		\begin{split}
			I(X, Y)
			&= \int_{a}^{b} \left( \langle \nabla _{\frac{\partial }{\partial t} } X, \nabla _{\frac{\partial }{\partial t} } Y \rangle - \langle R(\dot{c} , X) Y, \dot{c} \rangle \right)  \,\mathrm{d}t\\
			&= \int_{a}^{b} \left( \langle \nabla _{\frac{\partial }{\partial t} } X, \nabla _{\frac{\partial }{\partial t} } Y \rangle - \langle R(X, \dot{c}) \dot{c}, Y \rangle \right)  \,\mathrm{d}t
			= \int_{a}^{b} \left( \langle -\nabla _{\frac{\partial }{\partial t} } \nabla _{\frac{\partial }{\partial t} } X, Y \rangle - \langle R(X, \dot{c}) \dot{c}, Y \rangle \right)  \,\mathrm{d}t,
		\end{split}
	\end{equation}
	where the second inequality follows from the fact that \(\nabla \) is \hyperref[def:Riemannian]{Riemannian}, hence
	\[
		\nabla _{\frac{\mathrm{d} }{\mathrm{d} t} } \langle \nabla _{\frac{\mathrm{d} }{\mathrm{d} t} }X, Y \rangle
		= \langle \nabla _{\frac{\mathrm{d}}{\mathrm{d}t}} \nabla _{\frac{\mathrm{d}}{\mathrm{d}t} } X, Y \rangle + \langle \nabla _{\frac{\mathrm{d}}{\mathrm{d}t} } X, \nabla _{\frac{\mathrm{d}}{\mathrm{d}t} } Y \rangle,
	\]
	with \(Y(a) = 0 = Y(b)\),
	\[
		\int_{a}^{b} \nabla _{\frac{\mathrm{d} }{\mathrm{d} t} } \langle \nabla _{\frac{\mathrm{d} }{\mathrm{d} t} }X, Y \rangle  \,\mathrm{d}t
		= \at{\langle \nabla _{\frac{\mathrm{d}}{\mathrm{d}t} } X, Y\rangle }{a}{b} = 0,
	\]
	so
	\[
		\int_{a}^{b} \langle \nabla _{\frac{\mathrm{d}}{\mathrm{d}t} } \nabla _{\frac{\mathrm{d}}{\mathrm{d}t} } X, Y\rangle  \,\mathrm{d}t
		= - \int_{a}^{b} \langle \nabla _{\frac{\mathrm{d}}{\mathrm{d}t} } x, \nabla _{\frac{\mathrm{d}}{\mathrm{d}t} } Y \rangle  \,\mathrm{d}t.
	\]
	We see that the right-hand side of \autoref{eq:prop:Jacobi-field} vanishes for every \(Y\) if and only if
	\[
		\nabla _{\frac{\mathrm{d}}{\mathrm{d}t} }\nabla _{\frac{\mathrm{d}}{\mathrm{d}t} }X + R(X, \dot{c} )\dot{c} = 0,
	\]
	which is just the \hyperref[eq:Jacobi]{Jacobi equation}, so the result follows.
\end{proof}

\begin{intuition}
	\autoref{prop:Jacobi-field} is really where the \hyperref[eq:Jacobi]{Jacobi equation} comes from.
\end{intuition}

\begin{remark}
	do Carmo~\cite{flaherty2013riemannian} introduce \hyperref[eq:Jacobi]{Jacobi equation} slightly differently, but it's basically the same.
\end{remark}