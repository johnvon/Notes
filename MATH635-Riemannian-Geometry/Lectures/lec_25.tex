\lecture{25}{6 Apr. 13:00}{Rauch Comparison Theorems and Sphere Theorem}
\subsection{Rauch Comparison Theorem}
We're now ready to provide the general statement of Rauch.

\begin{theorem}[Rauch comparison theorem]\label{thm:Rauch-comparison}
	Let \((\mathcal{M}^m , g)\), \((\overline{\mathcal{M}}^m , \overline{g})\) be \hyperref[def:Riemannian-manifold]{Riemannian manifolds} and \(\gamma \colon [0, a] \to \mathcal{M} \), \(\overline{\gamma} \colon [0, a] \to \overline{\mathcal{M}} \) be normalized \hyperref[def:geodesic]{geodesics} with \(\gamma (0) = p\), \(\overline{\gamma} (0) = \overline{p} \). Let \(X, \overline{X} \) be \hyperref[def:Jacobi-field]{Jacobi fields} along \(\gamma , \overline{\gamma} \), respectively such that \(X(0) = \overline{X} = 0\), \(\vert \nabla_{\dot{\gamma } (0)} X \vert = \vert \overline{\nabla} _{\dot{\overline{\gamma} } (0)}\overline{X}  \vert\), and \(\langle \dot{\gamma }(0) , \nabla _{\dot{\gamma } (0)} X \rangle = \langle \dot{\overline{\gamma} } (0), \overline{\nabla} _{\dot{\overline{\gamma} } (0)} \overline{X} \rangle \). Furthermore, assume that
	\begin{enumerate}[(a)]
		\item \(\gamma \) has no \hyperref[def:conjugate-point]{conjugate points} on \([0, a]\);
		\item \hyperref[def:sectional-curvature]{sectional curvatures} \(K , \overline{K} \) of \(\mathcal{M} , \overline{\mathcal{M}} \) satisfy \(\overline{K} \leq K \) for all \(2\)-planes containing \(\dot{\gamma }, \dot{\overline{\gamma } } \).
	\end{enumerate}

	Then, \(\overline{\gamma} \) has no \hyperref[def:conjugate-point]{conjugate points} on \([0, a]\), and for all \(t\in[0, a]\),
	\[
		\vert X(t) \vert \leq \vert \overline{X} (t) \vert.
	\]
\end{theorem}
\begin{proof}[Proof idea]
	To prove this, we first see a lemma.

	\begin{lemma}\label{lma:lec25}
		Let \((\mathcal{M}^m , g)\), \((\overline{\mathcal{M}}^m , \overline{g})\) be \hyperref[def:Riemannian-manifold]{Riemannian manifolds} and \(\gamma \colon [0, a] \to \mathcal{M} \), \(\overline{\gamma} \colon [0, a] \to \overline{\mathcal{M}} \) be normalized \hyperref[def:geodesic]{geodesics} with \(\gamma (0) = p\), \(\overline{\gamma} (0) = \overline{p} \). Let \(X, \overline{X} \) be \hyperref[def:Jacobi-field]{Jacobi fields} along \(\gamma , \overline{\gamma} \), respectively such that \(X(0) = \overline{X} = 0\). Furthermore, assume that
		\begin{enumerate}[(a)]
			\item \(\gamma \) has no \hyperref[def:conjugate-point]{conjugate points} on \([0, a]\);
			\item \hyperref[def:sectional-curvature]{sectional curvatures} \(K , \overline{K} \) of \(\mathcal{M} , \overline{\mathcal{M}} \) satisfy \(\overline{K} \leq K \) for all \(2\)-planes containing \(\dot{\gamma }, \dot{\overline{\gamma } } \).
		\end{enumerate}
		Finally, assume that \(\vert X(a) \vert = \vert \overline{X} (a) \vert \). Then, \(I(X, X) \leq I(\overline{X} , \overline{X} )\).
	\end{lemma}
	\begin{proof}[Proof idea]
		We first choose an orthonormal frame in \((\mathcal{M} , g)\) and \((\overline{\mathcal{M}} , \overline{g} )\) with \(e_1 = \dot{\gamma } \) and \(\overline{e} _1 = \dot{\overline{\gamma } } \), and \(e_2 = X(a) / \vert X(a) \vert \neq 0\), etc. Consider \(X(t) = X^i(t) e_i(t)\) and the same for \(\overline{X} \). Then, the second variation of the \hyperref[def:energy]{energy} shows \(I(X, X) \leq I(\overline{X} , \overline{X} )\).
	\end{proof}

	Now, consider normal components of \(X, \overline{X} \) only, and we can show that
	\[
		\lim_{t \to 0} \frac{\vert X(t) \vert ^2}{\vert X(t) \vert ^2} \eqqcolon \lim_{t \to 0} \frac{\overline{u} (t)}{u(t)}= 1,
	\]
	thus to prove \(\vert X \vert \leq \vert \overline{X} \vert \), it's enough to show that
	\[
		\frac{\mathrm{d}}{\mathrm{d}t} \frac{\vert X(t) \vert ^2}{\vert X(t) \vert ^2} \geq 0,
	\]
	equivalently, \(\dot{\overline{u} } - \overline{u} \dot{u} \geq 0 \). Then, since \(\gamma \) has no \hyperref[def:conjugate-point]{conjugate points}, we have \(u(t) > 0\). Let \(c\in [0, a]\) be the greatest number such that \(\overline{u} (t) > 0\) on \((0, c)\). Then, for all \(b\in (0, c)\), define
	\[
		X_b (t) = \frac{X(t)}{\vert X(b) \vert },\quad \overline{X} _b(t) = \frac{\overline{X} (t)}{\vert \overline{X} (b) \vert }.
	\]
	From \autoref{lma:lec25} to \(I(X_b, X_b) , I(\overline{X} _b , \overline{X} _b)\), then we're done.
\end{proof}

\begin{corollary}
	Let \((\mathcal{M} , g )\) be a \hyperref[def:geodesically-complete]{complete} and simply-connected \hyperref[def:Riemannian-manifold]{Riemannian manifold} with non-positive \hyperref[def:sectional-curvature]{sectional curvature}, and \(\triangle ABC\) is a \hyperref[def:geodesic]{geodesic} triangle in \(\mathcal{M} \), then
	\begin{enumerate}[(a)]
		\item \(\vert AB \vert ^2 + \vert AB \vert ^2 - 2 \vert AB \vert \vert AC \vert \cos \angle A \leq \vert BC \vert ^2\);
		\item \(\angle A + \angle B + \angle C \leq \pi \).
	\end{enumerate}
\end{corollary}

\begin{corollary}
	Suppose that \hyperref[def:sectional-curvature]{sectional curvature} of \((\mathcal{M} , g)\) satisfies
	\[
		0 < C_1 \leq K \leq C_2
	\]
	for some constants \(C_1, C_2\). Let \(\gamma \) be any \hyperref[def:geodesic]{geodesic} in \(\mathcal{M} \). Then, the distance \(d\) between any two \hyperref[def:conjugate-point]{conjugate points} of \(\gamma \) satisfies
	\[
		\frac{\pi}{\sqrt{C_2} } \leq d \leq \frac{\pi }{\sqrt{C_1} }.
	\]
\end{corollary}

\begin{corollary}
	Let \((\mathcal{M} , g)\) be compact \hyperref[def:Riemannian-manifold]{Riemannian manifold} where the \hyperref[def:sectional-curvature]{sectional curvature} \(K\) satisfies \(K \leq C\) for some constant \(C\). Then, either the \hyperref[def:injectivity-radius]{injectivity radius}
	\[
		i(\mathcal{M} , g) \geq \pi / \sqrt{C},
	\]
	or there exists a closed \hyperref[def:geodesic]{geodesic} \(\gamma \) in \(\mathcal{M} \) whose \hyperref[def:length]{length} is  minimal among all closed \hyperref[def:geodesic]{geodesics} such that
	\[
		i(\mathcal{M} , g) \geq \frac{1}{2} L(\gamma ).
	\]
\end{corollary}

\section{The Sphere Theorem}
In this section, we want to prove the following.

\begin{theorem}[Sphere theorem]\label{thm:sphere}
	Let \(\mathcal{M} ^n\) be a compact and simply-connected \hyperref[def:Riemannian-manifold]{Riemannian manifold} with \hyperref[def:sectional-curvature]{sectional curvature} \(K\) such that
	\[
		0 < h K_{\max } < K \leq K_{\max } .
	\]
	Then if \(h = 1 / 4\), then \(\mathcal{M} \) is homeomorphic to a sphere \(S^n\).
\end{theorem}

\begin{notation}[Pinching number]
	\(h\) in the \hyperref[thm:sphere]{sphere theorem} is called the \emph{pinching number} of \(\mathcal{M} \).
\end{notation}

\begin{remark}\label{rmk:sphere-theorem-scaling}
	Another version of the \hyperref[thm:sphere]{sphere theorem} is to assume \(0 < h < K \leq 1\) by scaling.
\end{remark}

To prove this, Borger~\cite{berger1960varietes}, Klingenberg~\cite{Klingenberg1961} used \hyperref[thm:Rauch-comparison]{Rauch comparison theorem} with \hyperref[thm:Morse-index]{Morse index theorem} in the 1960s.

\subsection{Gauss-Bonnet Theorem and Theorem by Hamilton}
To understand the  \hyperref[thm:sphere]{sphere theorem}, we should consider \(n = 2, 3\). In this case, it suffices to assume \(h \geq 0\), i.e., for a compact and simply-connected \hyperref[def:Riemannian-manifold]{Riemannian manifold} \(\mathcal{M} ^n \) with \(n = 2, 3\) such that it has positive \hyperref[def:sectional-curvature]{sectional curvature}, then \(\mathcal{M} ^n\) is homeomorphic to \(S^n\).

\begin{note}
	For \(n = 2\), it follows from the \hyperref[thm:Gauss-Bonnet]{Gauss-Bonnet theorem}, and for \(n = 3\), it follows from a \hyperref[thm:Hamilton]{theorem by R.\ Hamilton}.
\end{note}

\begin{theorem}[Gauss-Bonnet theorem]\label{thm:Gauss-Bonnet}
	Let \(\mathcal{M} \) be a compact connected \(2\)-dimensional \hyperref[def:Riemannian-manifold]{Riemannian manifold} \(\mathcal{M} \) with \hyperref[rmk:Gauss-curvature]{Gauss curvature} \(K\). Then, its characteristic is given by
	\[
		\chi (\mathcal{M} ) = \frac{1}{2\pi } \int _\mathcal{M} K \,\mathrm{d} \mu _{\mathcal{M} }.
	\]
\end{theorem}

The \hyperref[thm:Gauss-Bonnet]{Gauss-Bonnet theorem} generalizes the so-called \hyperref[note:Gauss-Bonnet-formula]{Gauss-Bonnet formula}.

\begin{note}[Gauss-Bonnet formula]\label{note:Gauss-Bonnet-formula}
	Let \(\gamma \) be a curved polygon on an oriented \hyperref[def:Riemannian-manifold]{Riemannian \(2\)-manifold} \((\mathcal{M} , g)\) such that \(\gamma \) is positive oriented as the boundary of an open set \(\Omega \) with compact closure. Then,
	\[
		\int _\Omega K \,\mathrm{d} A + \int _{\gamma } k_N\,\mathrm{d} s + \sum_{i} \epsilon _i = 2\pi ,
	\]
	where \(k_N(t) = \langle \mathrm{D} _t \dot{\gamma }(t), N(t)  \rangle \),\footnote{\(N(t)\) is the normal vector field.} and \(\epsilon _i\) are the exterior angles.
\end{note}

To understand all these, we need the following concept.

\begin{definition}[Smooth triangulation]\label{def:smooth-triangulation}
	For \(\mathcal{M} \) \hyperref[def:smooth-manifold]{smooth}, compact \(2\)-\hyperref[def:smooth-manifold]{manifold}, a \emph{smooth triangulation} of \(\mathcal{M} \) is a finite collection of curved triangles such that
	\begin{itemize}
		\item the union of the closed regions \(\overline{\Omega} _i\) bounded by the triangles is actually \(\mathcal{M} \);
		\item the intersection of any pair (if not empty) is either a single vertex of each or a single edge of each.
	\end{itemize}
\end{definition}

\begin{theorem}[Radó~\cite{rado1925ober}]
	Every compact \hyperref[def:topological-manifold]{topological \(2\)-manifold} has a \hyperref[def:smooth-triangulation]{triangulation}.
\end{theorem}

\begin{note}
	Let \(\mathcal{M} \) be a \hyperref[def:smooth-triangulation]{triangulated} \(2\)-\hyperref[def:smooth-manifold]{manifold}. Then, the \hyperref[def:Euler-characteristic]{Euler characteristic} is
	\[
		\chi (\mathcal{M} ) = N_v - N_e + N_f.
	\]
	This implies that
	\[
		\int _\mathcal{M} K \,\mathrm{d} A = 2 \pi \chi (\mathcal{M} ).
	\]
\end{note}

Then, we can start proving \hyperref[thm:Gauss-Bonnet]{Gauss-Bonnet theorem}.

\begin{proof}[Proo of \autoref{thm:Gauss-Bonnet}]
	Let \(\{ \Omega _i \}_{i=1}^N \) denote the faces of \hyperref[def:smooth-triangulation]{triangulation}, and for all \(i\), let \(\{ \gamma _{ij} \mid j = 1, 2, 3 \} \) be the edges of \(\Omega _i\) and \(\{ \theta _{ij} \mid j = 1, 2, 3 \} \) be its interior angles.

	As each exterior angle is \(\pi \) minus the interior angle, by applying the \hyperref[note:Gauss-Bonnet-formula]{Gauss-Bonnet formula} to each triangle and sum over \(i\), we have
	\[
		\begin{split}
			&\sum_{i=1}^{N_f} \int _{\Omega _i} K \,\mathrm{d} A
			+ \sum_{i=1}^{N_f} \sum_{j=1}^{3} \int _{\gamma _{ij}} k_N \,\mathrm{d} s
			+ \sum_{i=1}^{N_f} \sum_{j=1}^{3} (\pi - \theta _{ij})
			= \sum_{i=1}^{N_f} 2\pi\\
			\iff & \int _{\mathcal{M}} K \,\mathrm{d} \mu _{\mathcal{M} }
			+ 0
			+ 3\pi N_f
			- \sum_{i=1}^{N_f} \sum_{j=1}^{3} \theta _{ij}
			= 2\pi N_f
		\end{split}
	\]
	where the second term vanishes since each edge appears twice but with opposite sign. Since degrees at each vertex adds up to \(2\pi \), we have
	\[
		\int _{\mathcal{M} } K \,\mathrm{d} A = 2\pi N_v - \pi N_f.
	\]
	As each edge is in exactly \(2\) triangles and each triangle has \(3\) edges, we see that \(2 N_e = 3N_f\),
	\[
		\int _\mathcal{M} K \,\mathrm{d} A = 2 \pi N_v - 2\pi N_e + 2\pi N_f = 2\pi \chi (\mathcal{M} ).
	\]
\end{proof}

\begin{theorem}[Hamilton]\label{thm:Hamilton}
	Let \(\mathcal{M} \) be a compact and simply-connected \(3\)-dimensional \hyperref[def:Riemannian-manifold]{Riemannian manifold} \(\mathcal{M} \) with strictly positive \hyperref[def:Ricci-curvature]{Ricci curvature}. Then, \(\mathcal{M} \) is \hyperref[def:diffeomorphic]{diffeomorphic} to \(S^3\).
\end{theorem}