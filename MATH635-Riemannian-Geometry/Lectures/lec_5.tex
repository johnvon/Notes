\chapter{Riemannian Manifolds}
\lecture{5}{19 Jan. 14:30}{Riemannian Manifolds}
In this chapter, we start our discussion on \hyperref[def:Riemannian-manifold]{Riemannian manifolds}.

\section{Riemannian Metric}
We start by defining the \hyperref[def:Riemannian-metric]{Riemannian metric}.

\begin{definition}[Riemannian metric]\label{def:Riemannian-metric}
	A \emph{Riemannian metric} \(g\) on a \hyperref[def:smooth-manifold]{differentiable manifold} \(\mathcal{M} \) is given by a scalar product \(I\) on each \(T_p \mathcal{M} \) which depends smoothly on the base point \(p\).
\end{definition}

\begin{definition}[Riemannian manifold]\label{def:Riemannian-manifold}
	A \emph{Riemannian manifold} \((\mathcal{M} , g)\) is a \hyperref[def:smooth-manifold]{smooth manifold} \(\mathcal{M} \) equipped with a \hyperref[def:Riemannian-metric]{Riemannian metric} \(g\).
\end{definition}

Let \(x=(x^1, \ldots , x^d)\) be the \hyperref[def:coordinate-chart]{local coordinates}. In these, a \hyperref[def:Riemannian-metric]{metric} is represented by a positive definite symmetric matrix
\[
	\left( g_{ij} (x) \right) _{i, j=1, \ldots , d},
\]
i.e., \(g_{ij} = g_{ji} \), and \(g_{ij} \xi ^i \xi ^j > 0\) for all \(\xi =(\xi ^1, \ldots , \xi ^d) \neq 0\) with coefficients smoothly depending on \(x\).

\subsection{Transformation Behavior}
We now see that the smoothness does not depend on \hyperref[def:coordinate-chart]{coordinates}, i.e., the smooth dependence on the base point (as required in \autoref{def:Riemannian-metric}) can be represented in the \hyperref[def:coordinate-chart]{local coordinates}.

The product of \(2\) \hyperref[def:tangent-vector]{tangent vectors} \(v, w\in T_p \mathcal{M} \) with \hyperref[def:coordinate-chart]{coordinate representations}
\[
	(v^1, \ldots , v^d), (w^1, \ldots , w^d),
\]
i.e., \(v = v^i \frac{\partial }{\partial x^i} \) and \(w = w^i \frac{\partial }{\partial x^i} \). Then
\[
	\left\langle v, w \right\rangle \coloneqq g_{ij} (x(p)) v^i w^j.
\]
In particular,
\[
	\left\langle \frac{\partial }{\partial x^i} , \frac{\partial }{\partial x^j}  \right\rangle = g_{ij} .
\]

\begin{remark}
	The length of \(v\) is given as \(\lVert v \rVert \coloneqq \left\langle v, v \right\rangle ^{1 / 2}\).
\end{remark}

Let \(y = f(x)\) define different \hyperref[def:coordinate-chart]{local coordinates}. In these, \(v, w\) are given as
\[
	(\widetilde{v} ^1, \ldots , \widetilde{v} ^d), (\widetilde{w} ^1, \ldots , \widetilde{w} ^d)
\]
with \(\widetilde{v} ^j = v^i \frac{\partial f^j}{\partial x^i} \) and \(\widetilde{w} ^j = w^i \frac{\partial f^j}{\partial x^i} \). Denote the metric in new \hyperref[def:coordinate-chart]{coordinates} \(y\) by \(h_{k \ell }(y)\), then we have
\[
	h_{k \ell }(f(x))\widetilde{v} ^k \widetilde{w} ^\ell
	= \left\langle v, w \right\rangle
	= g_{ij}(x)v^i w^j .
\]
Plug everything in, we have
\[
	h_{k \ell }(f(x)) \frac{\partial f^k}{\partial x^i} \frac{\partial f^{\ell } }{\partial x^j} v^i w^j
	= g_{ij}(x) v^i w^j.
\]
We see that this holds for any \hyperref[def:tangent-vector]{tangent vectors} \(v, w\), therefore,
\[
	h_{k \ell }(f(x)) \frac{\partial f^k}{\partial x^i} \frac{\partial f^{\ell } }{\partial x^j} = g_{ij} (x),
\]
which is the transformation behavior under \hyperref[def:coordinate-transition]{coordinates changes}.

\begin{remark}
	This shows that the smoothness does not depend on the choice of coordinates!
\end{remark}

\begin{eg}
	Consider the Euclidean space \(\Omega \), then given \(v, w\in T_p \Omega \), we have
	\[
		\left\langle v, w \right\rangle
		= \delta _{ij} v^i w^j
		= v^i w_i.
	\]
\end{eg}

\begin{theorem}
	Every \hyperref[def:smooth-manifold]{differentiable manifold} can be equipped with a \hyperref[def:Riemannian-metric]{Riemannian metric}.\todo{Shown in HW}
\end{theorem}

\subsection{Length and Energy}
Now, we can talk about the

\begin{definition*}
	Let \(\gamma \colon [a, b] \to  \mathcal{M} \) be a smooth curve on a \hyperref[def:Riemannian-manifold]{Riemannian manifold} \((\mathcal{M} , g)\).

	\begin{definition}[Length]\label{def:length}
		The \emph{length} of \(\gamma \) is defined as
		\[
			L(\gamma ) \coloneqq \int_{a}^{b} \left\lVert \frac{\mathrm{d}\gamma }{\mathrm{d}t} (t) \right\rVert  \,\mathrm{d}t .
		\]
	\end{definition}

	\begin{definition}[Energy]\label{def:energy}
		The \emph{energy} of \(\gamma \) is defined as
		\[
			E(\gamma )\coloneqq \frac{1}{2} \int_{a}^{b} \left\lVert \frac{\mathrm{d}\gamma }{\mathrm{d}t} (t) \right\rVert ^2 \,\mathrm{d}t .
		\]
	\end{definition}
\end{definition*}

We now want to compute \(L(\gamma )\), \(E(\gamma )\) in \hyperref[def:coordinate-chart]{local coordinates}. Let the \hyperref[def:coordinate-chart]{local coordinates} be
\[
	(x^1(\gamma (t)), \ldots , x^d(\gamma (t))),
\]
we write
\[
	\dot{x}^i (t) = \frac{\mathrm{d}}{\mathrm{d}t} (x^i (\gamma (t))).
\]
Then, we have
\[
	L(\gamma ) = \int_{a}^{b} \sqrt{g_{ij} (x(\gamma (t))) \dot{x}^i(t)\dot{x}^j(t)}  \,\mathrm{d}t, \quad
	E(\gamma ) = \frac{1}{2} \int_{a}^{b} g_{ij} (x(\gamma (t))) \dot{x}^i(t)\dot{x}^j(t) \,\mathrm{d}t.
\]

\begin{definition}[Distance]\label{def:distance}
	Given a \hyperref[def:Riemannian-manifold]{Riemannian manifold} \((\mathcal{M} , g)\), the \emph{distance} between \(2\) points \(p, q\in \mathcal{M} \) is defined as
	\[
		d(p, q)\coloneqq \inf \left\{ L(\gamma ) \mid \gamma \colon [a, b]\to  \mathcal{M} \text{ piecewise smooth curve with }\gamma (a)=p, \gamma (b)=q \right\} .
	\]
\end{definition}

\begin{note}
	Any \(2\) points \(p, q\in \mathcal{M} \) can be connected by a piecewise smooth curve, hence \(d(p, q)\) always exists.
\end{note}

\begin{corollary}
	The topology of \(\mathcal{M} \) induced by the \hyperref[def:distance]{distance function} \(d\) coincides with the original manifold topology of \(\mathcal{M} \).
\end{corollary}

\begin{lemma}
	If \(\gamma\colon [a, b] \to  \mathcal{M} \) is a smooth curve, and \(\psi \colon [\alpha , \beta ] \to  [a, b]\) is a change of parameter, then \(L(\gamma \circ \psi ) = L(\gamma )\).
\end{lemma}
\begin{proof}
	This can be proved by computation, and the take-away is that the \hyperref[def:length]{length functional} is invariant under parameter changes.
\end{proof}

\begin{notation}
	\(\left( g^{ij} \right) _{i, j=1, \ldots , d} = \left( g_{ij} \right)_{i, j g 1, \ldots , d} ^{-1} \), i.e., \(g^{i \ell} g_{\ell j}=\delta ^i_j \).
\end{notation}

\begin{notation}
	\(g_{j \ell , k} \coloneqq \frac{\partial }{\partial x^k} g_{j \ell }\)
\end{notation}

\begin{definition}[Christoffel symbol]\label{def:Christoffel-symbol}
	The \emph{Christoffel symbol} is defined as
	\[
		\Gamma ^i_{jk} \coloneqq \frac{1}{2}g^{i \ell }\left( g_{j \ell , k} + g_{k \ell , j} - g_{jk, \ell }\right) .
	\]
	for all \(i\).x
\end{definition}

\begin{proposition}
	The \href{https://en.wikipedia.org/wiki/Euler%E2%80%93Lagrange_equation}{Euler-Lagrange equations} for the \hyperref[def:energy]{energy} \(E\) are 
	\[
		\ddot{x}^i(t) + \Gamma ^{i}_{jk}(x(t)) \dot{x}^j(t)\dot{x}^k(t) = 0
	\]
	for \(i = 1, \ldots , d\).
\end{proposition}
\begin{proof}
	The \href{https://en.wikipedia.org/wiki/Euler%E2%80%93Lagrange_equation}{Euler-Lagrange equations} of a functional 
	\[
		I(x) = \int_{a}^{b} f(t, x(t), \dot{x}(t)) \,\mathrm{d}t
	\]
	are
	\[
		\frac{\mathrm{d}}{\mathrm{d}t} \frac{\partial f}{\partial \dot{x}^i} - \frac{\partial f}{\partial x^i} = 0
	\]
	for \(i = 1, \ldots , d\). Just by plugging in, we obtain for \(E\), we have
	\[
		\frac{\mathrm{d}}{\mathrm{d}t} \left( g_{ik} (x(t)) \dot{x}^k (t) + g_{ji}(x(t))\dot{x}^j(t) \right) - g_{jk, i}(x(t))\dot{x}^j(t)\dot{x}^k(t) = 0
	\]
	for \(i = 1, \ldots , d\). Hence,
	\[
		g_{ik} \ddot{x}^k + g_{ji}\ddot{x}^j + g_{ik, \ell }\dot{x}^{\ell }\dot{x}^k + g_{ji, \ell }\dot{x}^{\ell}\dot{x}^j - g_{jk, i} \dot{x}^{\ell } \dot{x}^j = 0
	\]
	Rename some indices and use \(g_{ij} = g_{ji} \), we have that
	\[
		2g_{\ell m}\ddot{x}^m + \left( g_{k \ell , j}+ g_{j \ell , k} - g_{jk, \ell} \right) \dot{x}^j \dot{x}^k = 0
	\]
	for \(\ell = 1, \ldots , d\). Hence, we have
	\[
		g^{i \ell }g_{\ell m}\ddot{x}^m + \frac{1}{2} g^{i \ell } \left( g_{\ell k, j} + g_{j \ell , k} - g_{jk, \ell } \right) \dot{x}^j \dot{x}^k = 0
	\]
	for \(i = 1, \ldots , d\). Finally, observe that
	\[
		g^{i \ell } g_{\ell m} = \delta _{im}
		\implies g^{i \ell } g_{\ell m} \ddot{x}^m = \ddot{x}^i,
	\]
	hence the claim follows.
\end{proof}

\begin{definition}[Geodesic]\label{def:geodesic}
	A smooth curve \(\gamma \colon [a, b] \to  \mathcal{M} \) that obeys
	\[
		\ddot{x}^i(t) + \Gamma ^{i}_{jk}(x(t)) \dot{x}^j(t)\dot{x}^k(t) = 0
	\]
	for \(i = 1, \ldots , d\) is called a \emph{geodesic}.
\end{definition}

\subsection{The Action Functional}

\begin{definition}[Action]
	Let \(\mathcal{L} \) be the Lagrangian, then let
	\[
		I[w(\cdot)] \coloneqq \int_{0}^{t} \mathcal{L} (\dot{w}(s), w(s)) \,\mathrm{d}s
	\]
	defined for functions \(w(\cdot) = (w^1(\cdot), \ldots w^n(\cdot))\) of the admissible class
	\[
		\mathcal{A} =\left\{ w(\cdot)\in C^2([0, t]; \mathbb{R} ^n) \mid w(0)= y, w(t)= x\right\}.
	\]
\end{definition}

From the calculus of variation, we can find a curve \(w(\cdot)\in \mathcal{A} \) such that
\[
	I[w(\cdot)] = \min _{x(\cdot)\in \mathcal{A} }I[x(\cdot)].
\]

\begin{theorem}[Euler-Lagrangian equations]
	\(w(\cdot)\) solves the system of Euler-Lagrangian equations
	\[
		\frac{\mathrm{d}}{\mathrm{d}s} \left( D_{\dot{w}} \mathcal{L} (\dot{w}(s), w(s)) + D_w \mathcal{L} (\dot{w}(s), w(s)) \right) = 0
	\]
	for \(0 \leq s \leq t\).
\end{theorem}