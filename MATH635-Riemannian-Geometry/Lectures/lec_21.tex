
\lecture{21}{23 Mar.\ 13:00}{Morse Index Theorem}
\begin{prev}
	Fix \(p\in \mathcal{M} \), let \(q\in C(p)\) a \hyperref[def:cut-point]{cut point}. Then there exists a \hyperref[def:geodesic]{geodesic} \(c\) such that
	\begin{enumerate}[(a)]
		\item \(c\) minimizing up to and including \(q\), and
		\item \(c\) uniquely minimizing up to but not including \(q\).
	\end{enumerate}
	Thus, \(c\) is not minimizing after that point.
\end{prev}

Let's first see the following under the above setting.

\begin{proposition}\label{prop:lec21}
	At each \hyperref[def:cut-point]{cut point} \(q\in C(p)\), \(q\) is either a \hyperref[def:conjugate-point]{conjugate point} or there exists two minimizing \hyperref[def:geodesic]{geodesics} connecting \(p, q\).
\end{proposition}
\begin{proof}
	See~\cite{flaherty2013riemannian}.
\end{proof}

\chapter{Morse Index, Rauch Comparison, Sphere Theorems, and More}
Now, we have everything to prove three important theorems: the \hyperref[thm:Morse-index]{Morse index theorem}, \hyperref[thm:Rauch-comparison]{Rauch comparison theorem}, and the \hyperref[thm:sphere]{sphere theorem}.

\begin{intuition}
	In short,
	\begin{itemize}
		\item \hyperref[thm:Morse-index]{Morse index theorem} relates the number (with multiplicities) of \hyperref[def:conjugate-point]{conjugate points} on a \hyperref[def:geodesic]{geodesic} segment to the \hyperref[def:index]{index}.
		\item \hyperref[thm:Rauch-comparison]{Rauch comparison theorem} is one of the basic facts in Riemannian geometry. Intuitively, it expresses the plausible fact that as the \hyperref[def:sectional-curvature]{curvature} grows, lengths shorten.
		\item \hyperref[thm:sphere]{Sphere theorem} is one of the most beautiful theorems of global differential geometry, which says that under some mild \hyperref[def:sectional-curvature]{curvature bounds}, the space is homeomorphic to a sphere.
	\end{itemize}
\end{intuition}

In what follows, we prove each theorem one by one.

\begin{note}
	After proving the \hyperref[thm:Morse-index]{Morse index theorem}, we detour to study the \hyperref[def:Morse-function]{Morse function} and \hyperref[def:Morse-homology-group]{Morse homology} before going to the \hyperref[thm:Rauch-comparison]{Rauch comparison theorem}.
\end{note}

\begin{note}
	After prove the \hyperref[thm:sphere]{sphere theorem}, we prove another important \hyperref[thm:uniformization]{uniformization theorem}, which is worth noting here.
\end{note}

Let's start by proving the \hyperref[thm:Morse-index]{Morse index theorem}.

\section{Morse Index Theorem}
In this section, we study the \hyperref[thm:Morse-index]{Morse index theorem}, which gives information about \hyperref[def:conjugate-point]{conjugate points} via \hyperref[def:index-form]{index form}.

\begin{prev}
	The \hyperref[def:index-form]{index form} \(I(X, Y)\), \hyperref[def:index]{index} \(\Ind(c)\), and also \(\Ind_0 \), and \autoref{lma:finite-index-nullity}.
\end{prev}

Let \(c\colon [0, T] \to \mathcal{M} \) be a \hyperref[def:geodesic]{geodesic}. Then the \hyperref[def:index]{index} \(\Ind(c)\) on the space \(\mathring{\nu }_c\) is finite and equals the number of points \(c(t)\) \hyperref[def:conjugate-point]{conjugate} to \(c(0)\) for \(t\in (0, T)\), counted with multiplicities.

\subsection{The Conjugate Locus}
Before proving the \hyperref[thm:Morse-index]{Morse index theorem}, let's see one last definition.

\begin{definition}[Conjugate locus]\label{def:conjugate-locus}
	Let \((\mathcal{M} , g)\) be a \hyperref[def:Riemannian-manifold]{Riemannian manifold}. The set of (first) \hyperref[def:conjugate-point]{conjugate points} of point \(p\in \mathcal{M} \) for all \hyperref[def:geodesic]{geodesics} starting at \(p\) is called the \emph{conjugate locus} of \(p\).
\end{definition}

\begin{proposition}
	Let \((\mathcal{M} , g)\) be a \hyperref[def:geodesically-complete]{complete} \hyperref[def:Riemannian-manifold]{Riemannian manifold}. Let \(c\colon [0, \infty ) \to \mathcal{M} \) be a normalized \hyperref[def:geodesic]{geodesic} with \(c(0) = p\). Assume that \(c(t_0)\) is the \hyperref[def:cut-point]{cut point} at \(p = c(0)\) along \(c\). Then, either \(c(t_0)\) is the first \hyperref[def:conjugate-point]{conjugate point} of \(c(0)\) along \(c\) or there exists another \hyperref[def:geodesic]{geodesic} \(\sigma \neq c\) from \(p\) to \(c(t_0)\) such that \(\ell (\sigma ) = \ell (c)\). Conversely, if either the above are true, then there exists \(t_0 \in (0, t_0]\) such that \(c(t_1)\) is the \hyperref[def:cut-point]{cut point} of \(p\) along \(c\).
\end{proposition}
\begin{proof}
	See~\cite{flaherty2013riemannian}.
\end{proof}

\subsection{Morse Index Theorem}
Consider the following.

\begin{theorem}[Morse index theorem]\label{thm:Morse-index}
	Let \(c\colon [a, b] \to \mathcal{M} \) be a \hyperref[def:geodesic]{geodesic}. Then, there are at most finitely many points \hyperref[def:conjugate-point]{conjugate} to \(c(a)\) along \(c\), and
	\[
		\Ind(c) = \sum_{t\in (a, b)} \dim \mathcal{J} _c^t,\quad
		\Ind_0 (c) = \sum_{t\in (a, b]} \dim \mathcal{J} _c^t.
	\]
\end{theorem}
\begin{proof}
	For all \(t_i\in (a, b]\), for which \(c(t_i)\) \hyperref[def:conjugate-point]{conjugate} to \(c(a)\), there exists a \hyperref[def:Jacobi-field]{Jacobi field} \(X_i\) along with \(X_i(a) = 0 = X_i(t_i)\). Set
	\[
		Y_i(t) \coloneqq \begin{dcases}
			X_i(t), & \text{ if } a \leq t \leq t_i ; \\
			0,      & \text{ otherwise} ,
		\end{dcases}
	\]
	we have that \(Y_i(t)\) are linearly independent such that \(I(Y_i, Y_i) = 0\) for all \(i\). This implies that the number of \hyperref[def:conjugate-point]{conjugate points} is at most \(\Ind_0(c) \), which is finite from \autoref{lma:finite-index-nullity}.

	For \(\tau \in (a, b]\), set
	\[
		\varphi (\tau ) \coloneqq \Ind \left( \at{c}{[a, \tau ]}{} \right),\quad
		\varphi _0(\tau ) \coloneqq \Ind_0 \left( \at{c}{[a, \tau ]}{} \right).
	\]
	\begin{claim}
		\(\varphi (\tau )\) is left-continuous.
	\end{claim}
	\begin{explanation}
		For \(\tau \in (a, b]\), let \(I_{\tau } \) be the \hyperref[def:index-form]{index form} of \(\at{c}{[a, \tau ]}{} \), and let \(X\) be a \hyperref[def:vector-field-along-curve]{vector field along \(\at{c}{[a, \tau ]}{} \)} satisfy \(I_{\tau } (X, X) < 0\) and \(\lVert X \rVert = 1\).

		Let \(\widetilde{X} \) be \hyperref[def:vector-field]{vector field} defined by \(\widetilde{X} (t) \coloneqq X (\tau t / \sigma)\) on \([a, \sigma ]\). Then,
		\[
			\int_{0}^{\sigma } \langle \dot{\widetilde{X} }(t), \dot{\widetilde{X} } (t) \rangle \,\mathrm{d}t
			= \int_{0}^{\sigma } \left( \frac{\tau }{\sigma } \right)^2 \langle \dot{X} (\tau t / \sigma), \dot{X} (\tau t / \sigma) , \rangle \,\mathrm{d}t
			= \frac{\tau}{\sigma } \int_{0}^{\tau } \langle \dot{X} (s), \dot{X} (s) \rangle  \,\mathrm{d}s,
		\]
		implying
		\[
			\int_{0}^{\sigma } \langle \dot{\widetilde{X} } (t), \dot{\widetilde{X} } (t) \rangle  \,\mathrm{d}t
			\to \int_{0}^{\tau } \langle \dot{X} (t), \dot{X} (t) \rangle \,\mathrm{d}t
		\]
		for \(\sigma \to \tau \). Also, we have \(\lVert X \rVert = 1\), and \(X\) is continuous,\footnote{This is from something called Sobolev theorem.} we see that \(\widetilde{X} \) converges point-wise to \(X\) as \(\sigma \to \tau \), hence
		\[
			\int_{0}^{\sigma } \langle R(\dot{c} , \widetilde{X} )\widetilde{X} , \dot{c}  \rangle  \,\mathrm{d}t
			\to \int_{0}^{\tau } \langle R(\dot{c} , X)X, \dot{c} \rangle \,\mathrm{d}t
		\]
		as \(\sigma \to \tau \), hence \(I_\sigma (\widetilde{X} , \widetilde{X} ) \to I_{\tau } (X, X)\) as \(\sigma \to \tau \). Notice that the above also implies \(I_\sigma (\widetilde{X} , \widetilde{X} ) < 0\) if \(\sigma \) is sufficiently close to \(\tau \).

		Finally, for all orthonormal basis of a space on which \(I_{\tau } \) is negative definite, we may also find a basis of some space on which \(I_\sigma \) is negative definite if \(\sigma \) is sufficiently close to \(\tau \). As \(\varphi \) is monotonically increasing, we have left-continuity.
	\end{explanation}

	\begin{claim}
		\(\varphi _0(\tau) \) is right-continuous.
	\end{claim}
	\begin{explanation}
		Let \((\tau _n)_{n\in \mathbb{N} } \subseteq (a, b]\) converge to \(\tau \in (a, b]\) for all \(n\in \mathbb{N} \), let \(X_n\) be a \hyperref[def:vector-field-along-curve]{vector field along \(\at{c}{[0, \tau _n]}{}\)} with \(\lVert X \rVert = 1\) and \(I_{\tau _n} (X_n, X_n) \leq 0\). After selecting a subsequence, \(X_n\) converges weakly in Sobolev space \(H^{1, 2}\) topology to some \hyperref[def:vector-field-along-curve]{vector field \(X\) along \(\at{c}{[a, \tau ]}{} \)}. Then, we just check every ingredient of \hyperref[def:index-form]{index form} (see~\cite{flaherty2013riemannian}).
	\end{explanation}

	Finally, let \(a < t_1 < t_2 < \dots < t_k \leq b\) be the points \(c(t_i)\) \hyperref[def:conjugate-point]{conjugate} to \(c(a)\). Then, \(\varphi _0(t) - \varphi (t) = 0\) for \(t_i\in (a, b]\). Then,
	\[
		\sum_{t\in (a, b]} \dim \mathcal{J} _c^t
		= \sum_{t\in (a, b]} (\varphi _0(t) - \varphi (t))
		= \sum_{i=1}^k (\varphi _0 (t_i) - \varphi (t_i)).
	\]
	Since \(\varphi \) is left-continuous, and \(\varphi _0\) is right-continuous, hence we have
	\[
		\varphi _0(t_i) = \varphi (t_{i+1})
	\]
	for \(i = 1, \dots , k-1\), we finally have
	\[
		\sum_{i=1}^k \left( \varphi _0(t_i) - \varphi (t_i) \right) = \varphi _0(t_k) - \varphi (t_1).
	\]
	From \(\varphi \) being left-continuous again, \(\varphi (t_1) = 0\). Finally, again, from the continuity properties of \(\varphi , \varphi _0\), they can ``jump'' only at the points \(\tau \) where \(\varphi _0(\tau ) \neq \varphi (\tau )\), i.e., at the \hyperref[def:conjugate-point]{conjugate points}. In particular, \(\varphi _0\) is constant on \([t_k, b]\) hence, \(\varphi _0(t_k) = \varphi _0(b)\), i.e.,
	\[
		\varphi _0(b) = \sum_{t\in (a, b]} \dim \mathcal{J} _c^t.
	\]
\end{proof}

\begin{intuition}
	The ``jump'' only happens at \hyperref[def:conjugate-point]{conjugate points}.
\end{intuition}