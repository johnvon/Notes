\lecture{8}{31 Jan. 13:00}{Injectivity Radius and Vector Bundles}
In the proof we did last time, the last step can be shown via \cite[Corollary 3.9]{flaherty2013riemannian}.

\begin{proof}[Proof of \hyperlink{thm:Hopf-Rinow}{Hopf-Rinow theorem} (Continued)]
	We see that \autoref{thm:Hopf-Rinow-4} implies \autoref{thm:Hopf-Rinow-5}, hence we only need to show that \autoref{thm:Hopf-Rinow-1}, \autoref{thm:Hopf-Rinow-2}, \autoref{thm:Hopf-Rinow-3}, and \autoref{thm:Hopf-Rinow-4} are equivalent.
	\begin{itemize}
		\item \autoref{thm:Hopf-Rinow-4} \(\implies \) \autoref{thm:Hopf-Rinow-3} is trivial.
		\item \autoref{thm:Hopf-Rinow-3} \(\implies \) \autoref{thm:Hopf-Rinow-2}: Let \(K \subseteq \mathcal{M} \) be closed and bounded. As \(K\) bounded, \(K \subseteq B(p, r)\) for some \(r > 0\). Then any point in \(B(p, r)\) can be joined with \(p\) by \hyperref[def:geodesic]{geodesic} of length \(\leq r\), and \(B(p, r)\) is the image of the compact ball in \(T_p \mathcal{M} \) of radius \(r\) under continuous map \(\exp _p\), hence \(B(p, r)\) is compact. As \(K\) closed and \(K \subseteq B(p, r)\), \(K\) is compact.
		\item \autoref{thm:Hopf-Rinow-2} \(\implies \) \autoref{thm:Hopf-Rinow-1}: Let \((p_n)_{n \in \mathbb{N} } \subseteq \mathcal{M} \) be a Cauchy sequence, so it's bounded, and by \autoref{thm:Hopf-Rinow-2}, its closure is compact. It contains a convergent subsequence, so it converges, i.e., \(\mathcal{M} \) is \hyperref[def:geodesically-complete]{complete}.
		\item \autoref{thm:Hopf-Rinow-1} \(\implies \) \autoref{thm:Hopf-Rinow-4}: Let \(c\) be a \hyperref[def:geodesic]{geodesic} in \(\mathcal{M} \), parametrized by arc length defined on \(a\) maximal interval \(I\). Since \(I\) s non-empty, and we can show that \(I\) is both open and closed.\todo{Exercise}
	\end{itemize}
\end{proof}

It's worth mentioning that we do have uniqueness after choosing \(p_0\), in other words, after choosing \(p_0\), everything is fixed, so the non-uniqueness really comes from the initial choose of \(p_0\).

\begin{figure}[H]
	\centering
	\incfig{Hopf-Rinow-uniqueness}
	\caption{Consider \(S^2\), after fixing \(p_0\), \(c(t_0)\) is extended uniquely.}
\end{figure}

\section{Injectivity Radius}
Consider the following.

\begin{definition}[Injectivity radius]\label{def:injectivity-radius}
	Let \(\mathcal{M} \) be a \hyperref[def:Riemannian-manifold]{Riemannian manifold}, and \(p\in \mathcal{M} \). The \emph{injectivity radius} \(i(p)\) of \(p\) is
	\[
		i(p) \coloneqq \sup \left\{ \rho > 0 \mid \exp _p \text{ defined on \(B(0, \rho ) \subseteq T_p \mathcal{M} \) and injective} \right\}.
	\]
	Similarly, the \emph{injectivity radius} \(i(\mathcal{M} )\) of \(\mathcal{M} \) is defined as \(i(\mathcal{M} )\coloneqq \inf _{p\in \mathcal{M} }i(p)\).
\end{definition}

\begin{eg}[Sphere]
	\(i(S^n) = \pi \).
\end{eg}

\begin{eg}[Torus]
	\(i(T^n) = 1 / 2\).
\end{eg}

Any manifold carries a \hyperref[def:geodesically-complete]{complete} \hyperref[def:Riemannian-metric]{Riemannian metric}.

If \((\mathcal{M} , g_1)\) is not \hyperref[def:geodesically-complete]{complete}, we can find \(g_2\) such that \((\mathcal{M} , g_2)\) is \hyperref[def:geodesically-complete]{complete}.

\begin{eg}[Hyperbolic half-plane]
	The half-plane \(P=\left\{ (x, y)\in \mathbb{R} ^2 \mid y > 0 \right\}\) equipped with metric induced by the Euclidean metric on \(\mathbb{R} ^2\), which is not \hyperref[def:geodesically-complete]{complete}.

	However, it becomes \hyperref[def:geodesically-complete]{complete} when equipped with the following metric
	\[
		\frac{1}{y^2} (\mathrm{d} x^2 + \mathrm{d} y^2).
	\]
	In fact, \(P\) with the above metric is called the \emph{hyperbolic half-plane} \(H^2\), and we can extend it to \(H^n \).
\end{eg}

Another question we may ask is the following.

\begin{problem*}
	Is the converse of \hyperref[thm:Hopf-Rinow]{Hopf-Rinow theorem} true? I.e., can we show that \autoref{thm:Hopf-Rinow-5} implies \autoref{thm:Hopf-Rinow-4}?
\end{problem*}
\begin{answer}
	No! Any \(2\) points in the open half-sphere can be joint by a unique minimal \hyperref[def:geodesic]{geodesic}, but this manifold is not \hyperref[def:geodesically-complete]{geodesically complete}.
\end{answer}

\begin{eg}
	The \hyperref[def:injectivity-radius]{injectivity radius} of \(H^n\) is \(\infty \).
\end{eg}

\begin{remark}
	Given a compact \(\mathcal{M} \), the \hyperref[def:injectivity-radius]{injectivity radius} is always \(> 0\) by continuity argument.
\end{remark}

Now, given a \hyperref[def:geodesically-complete]{complete} but not compact \(\mathcal{M} \), the \hyperref[def:injectivity-radius]{injectivity radius} can be \(0\).

\begin{eg}
	Take the quotient of the Poincaré half-plane by the translations
	\[
		(x, y) \mapsto (x+n, y),\quad n\in \mathbb{Z} .
	\]
	We then obtain a \hyperref[def:geodesically-complete]{complete} \hyperref[def:Riemannian-manifold]{Riemannian manifold} \(\mathcal{M} \) with \(i(\mathcal{M} ) = 0\).
\end{eg}

\begin{note}
	Finding lower bounds for \(i(\mathcal{M})\) introduces curvature estimates.
\end{note}

\section{Bundles and Fields}
Let's first introduce the theory of \hyperref[def:bundle]{bundles}, which allows us to introduce the notion of \hyperref[def:vector-field]{vector fields}, which is a more general notion of \hyperref[def:tensor-field]{tensor fields}. And noticeably, nearly every structure we can put on a \hyperref[def:Riemannian-manifold]{Riemannian manifold} will be in the form of \hyperref[def:tensor-field]{tensor fields}.

\begin{eg}
	Given a \hyperref[def:vector-field]{tangent vector field} \(X\) of a \hyperref[def:smooth-manifold]{smooth manifold} \(\mathcal{M} \) is where we simply associate \(X(p)\) to a \hyperref[def:tangent-vector]{tangent vector}:
	\begin{figure}[H]
		\centering
		\incfig{motivation-bundle-field}
		\caption{Given \(\mathcal{M} = S^2\), a \hyperref[def:vector-field*]{vector field} assigns every point a ``point'' in the associated ``space.'' In this case, a \hyperref[def:vector-field]{tangent vector field} associates every \(p\) a vector in the corresponding \hyperref[def:tangent-space]{tangent space}.}
		\label{fig:motivation-bundle-field}
	\end{figure}
\end{eg}

Recall the \hyperref[def:tangent-bundle]{tangent bundle} \((T\mathcal{M} , \pi , \mathcal{M} )\), where we only take the name ``\hyperref[def:bundle]{bundle}'' for granted and don't know why it is: however, we should see that it helps us construct the \hyperref[def:vector-field]{vector field}, since it captures the idea of ``for every point \(p\), we have an associated space \(T_p \mathcal{M} \),'' which is exactly what we need here. This idea generalizes quite easily.

\subsection{Bundles}
We start by introducing the notion of \hyperref[def:bundle]{bundles}.

\begin{definition}[Bundle]\label{def:bundle}
	A \emph{bundle} is a tuple \((E, \pi , \mathcal{M} )\) consists of the \hyperref[def:vector-bundle-total-space]{total space} \(E\), the \hyperref[def:base-space]{base space} \(\mathcal{M} \), and the \hyperref[def:bundle-projection]{bundle projection} \(\pi \colon E \to \mathcal{M} \).
	\begin{definition}[Total space]\label{def:vector-bundle-total-space}
		The \hyperref[def:smooth-manifold]{differentiable manifold} \(E\) is called the \emph{total space}.
	\end{definition}

	\begin{definition}[Base space]\label{def:base-space}
		The \hyperref[def:smooth-manifold]{differentiable manifold} \(\mathcal{M} \) is called the \emph{base space}.
	\end{definition}

	\begin{definition}[Bundle projection]\label{def:bundle-projection}
		The (differentiable) continuous surjection \(\pi \colon E\to \mathcal{M} \) is called the \emph{bundle projection}.
	\end{definition}
\end{definition}

\begin{note}
	We see that a \hyperref[def:tangent-bundle]{tangent bundle} \((T\mathcal{M} , \pi , \mathcal{M} )\) is actually a \hyperref[def:bundle]{bundle}.
\end{note}

\begin{eg}
	Let \(E\) be a cylinder, \(\mathcal{M} \) be a circle.

	\begin{center}
		\incfig{cylinder-circle-bundle}
	\end{center}
	As we can see, the number of possible \(\pi \) is enormous, as long as it's surjective and smooth.
\end{eg}

\begin{notation}
	Sometimes, we will just denote a \hyperref[def:bundle]{bundle} as \(E \overset{\pi }{\to } \mathcal{M} \), or even more compactly, just \(\pi \) since it captures all the data.
\end{notation}

\begin{definition}[Fiber]\label{def:fiber}
	Given a \hyperref[def:bundle]{bundle} \((E, \pi , \mathcal{M} )\), the \emph{fiber} over \(p\in \mathcal{M} \) under \(\pi\) is the preimage of a \(\left\{ p \right\} \), i.e., \(\pi ^{-1} (\{p\})\).
\end{definition}

\begin{definition}[Section]\label{def:section}
	A \emph{section} of a \hyperref[def:bundle]{bundle} \((E, \pi , \mathcal{M} )\) is a differentiable map \(s\colon \mathcal{M} \to E\) such that \(\pi \circ s =\identity_{\mathcal{M} } \).
\end{definition}

\begin{remark}
	We see that a \hyperref[def:section]{section} \(s\) encodes lots of information of a \hyperref[def:bundle]{bundle}, since \(s\) includes \(E, \mathcal{M}\), and the condition deal with \(\pi \).
\end{remark}


\begin{eg}
	Again let \(E\) be a cylinder, \(\mathcal{M} \) be a circle. This time, we choose \(\pi \) to be the trivial one.

	\begin{center}
		\incfig{cylinder-circle-section}
	\end{center}

	We see that in this way, this \hyperref[def:bundle]{bundle} really captures all the \hyperref[def:tangent-space]{tangent spaces} structure of a circle!
\end{eg}

\subsection{Vector Bundles}
Then, we're interested in the so-called \hyperref[def:vector-bundle]{vector bundle}.

\begin{definition}[Vector bundle]\label{def:vector-bundle}
	A (differentiable) \emph{vector bundle} of rank \(n\) is a \hyperref[def:bundle]{bundle} \((E, \pi , \mathcal{M} )\) such that each \hyperref[def:fiber]{fiber} \(E_x\coloneqq \pi ^{-1} (x)\) of \(x\in \mathcal{M} \) carries a structure of an \(n\)-dimensional (real) vector space, and \hyperref[def:local-trivialization]{local triviality} condition holds.

	\begin{definition}[Local trivialization]\label{def:local-trivialization}
		For all \(x\in \mathcal{M} \), the \emph{local trivialization} \((U, \varphi )\) consists a neighborhood \(U\) and \hyperref[def:diffeomorphism]{diffeomorphism} \(\varphi \colon \pi ^{-1} (U) \to U\times \mathbb{R} ^n\) such that for all \(y\in U\),
		\[
			\varphi _y \coloneqq \at{\varphi }{E_y}{} \colon E_y \to \left\{ y \right\} \times \mathbb{R} ^n
		\]
		is a vector space isomorphism.
	\end{definition}
\end{definition}

\begin{figure}[H]
	\centering
	\incfig{vector-bundle}
	\caption{An illustration of \hyperref[def:vector-bundle]{vector bundle} \((E, \pi , \mathcal{M} )\).}
	\label{fig:vector-bundle}
\end{figure}

\begin{definition}[Trivial]\label{def:trivial}
	A \hyperref[def:vector-bundle]{vector bundle} is \emph{trivial} if it's isomorphic to \(\mathcal{M} \times \mathbb{R} ^n\).\footnote{\(n\) is the rank of the \hyperref[def:vector-bundle]{vector bundle}.}
\end{definition}

\begin{intuition}
	\hyperref[def:local-trivialization]{Local trivialization} shows that \emph{locally} \(\pi \) looks like the \hyperref[def:bundle-projection]{projection} of \(U\times \mathbb{R} ^n\) on \(U\).
\end{intuition}

\begin{definition}[Bundle chart]\label{def:bundle-chart}
	The pair \((\varphi , U)\) is the \emph{bundle chart} in \hyperref[def:local-trivialization]{local trivialization}.
\end{definition}

\begin{remark}
	From \autoref{def:vector-bundle}, \hyperref[def:vector-bundle]{vector bundle} is locally, but not necessarily globally a product of \hyperref[def:base-space]{base space} and the \hyperref[def:fiber]{fiber}.
\end{remark}

\begin{intuition}
	We may look at a \hyperref[def:vector-bundle]{vector bundle} as a family of vector spaces, all isomorphic to a fixed \(\mathbb{R} ^n\), ``parametrized'' (\hyperref[def:local-trivialization]{locally trivially}) by a \hyperref[def:smooth-manifold]{manifold}.
\end{intuition}

\subsection{Vector Fields}
We can now introduce the notion of \hyperref[def:vector-field]{vector fields} in terms of \hyperref[def:section]{section}.

\begin{definition}[Vector field]\label{def:vector-field*}
	A (smooth) \emph{vector field} \(X\) is a smooth \hyperref[def:section]{section} of a \hyperref[def:bundle]{bundle}.
\end{definition}

\begin{note}
	We see that a smooth \hyperref[def:vector-field]{tangent vector field} is indeed a smooth \hyperref[def:vector-field*]{vector field} with the \hyperref[def:bundle]{bundle} being the \hyperref[def:tangent-bundle]{tangent bundle}.
\end{note}

\begin{notation}
	Since we will nearly always be talking about \hyperref[def:vector-field]{tangent vector fields}, we will abuse the notation a bit and just simply call it \hyperref[def:vector-field]{vector fields}. But always keep in mind that more broadly, a \hyperref[def:vector-field*]{vector field} should be a \hyperref[def:section]{section} of a \hyperref[def:bundle]{bundle}, not always \(T \mathcal{M} \).
\end{notation}