\lecture{11}{9 Feb. 14:30}{Geodesic \& Cogeodesic Flows and Parallel Transport}
\subsection{Geodesic and Cogeodesic Flows}
A particularly interesting \hyperref[def:local-flow]{flow} is the \hyperref[def:cogeodesic-flow]{cogeodesic-flow}, which can be constructed as follows. Let's first transform \autoref{eq:geodesic} (which is a \(2^{nd}\)-ODE) into a \(1^{st} \) order system on the \hyperref[def:cotangent-space]{cotangent} \hyperref[def:vector-bundle]{bundle} \(T^{\ast} \mathcal{M} \), and \hyperref[def:local-trivialization]{locally trivialize} \(T^{\ast} \mathcal{M} \) by \hyperref[def:coordinate-chart]{chart} \(\at{T^{\ast} \mathcal{M} }{U}{} \cong U \times \mathbb{R} ^d\) with coordinates \((x^1, \ldots , x^d, p_1, \ldots , p_d)\). Now, set
\begin{equation}\label{eq:Hamilton}
	H(x, p) = \frac{1}{2} g^{ij} (x) p_i p_j,
\end{equation}

\begin{theorem}
	\autoref{eq:geodesic} is equivalent to the system on \(T^{\ast} \mathcal{M} \):
	\begin{equation}\label{eq:cogeodesic}
		\begin{dcases}
			\dot{x}^i = \frac{\partial H}{\partial p_i} g^{ij}(x) p_j; \\
			\dot{p}_i = -\frac{\partial H}{\partial x^i} = -\frac{1}{2} g^{jk}_{, i}(x)p_j p_k.
		\end{dcases}
	\end{equation}
\end{theorem}
\begin{proof}
	This is just computation (recall that\(g^{ik} g_{kj} = \delta ^i_j \)).
\end{proof}

\begin{definition}[Cogeodesic flow]\label{def:cogeodesic-flow}
	The \emph{cogeodesic flow} is the \hyperref[def:local-flow]{local flow} determined by \autoref{eq:cogeodesic}.
\end{definition}

\begin{definition}[Geodesic flow]\label{def:geodesic-flow}
	The \hyperref[def:geodesic-flow]{geodesic flow} on \(T\mathcal{M} \) is obtained from the \hyperref[def:cogeodesic-flow]{cogeodesic flow} by the first equation in \autoref{eq:cogeodesic}.
\end{definition}

Thus, the \hyperref[def:geodesic]{geodesic} is the projection of the \hyperref[def:integral-curve]{integral curve} of the \hyperref[def:geodesic-flow]{geodesic flow} onto \(\mathcal{M} \).

\begin{remark}[Hamiltonian flow]
	The \hyperref[def:cogeodesic-flow]{cogeodesic flow} is a \emph{Hamiltonian flow} for the Hamiltonian \(H\).
\end{remark}
\begin{explanation}
	By \autoref{eq:cogeodesic}, along the \hyperref[def:integral-curve]{integral curves},
	\[
		\frac{\mathrm{d}H}{\mathrm{d}t}
		= H_{x^i} \dot{x}^i + H_{p_i}\dot{p}^i
		= -\dot{p}_i x\dot{x}^i + \dot{x}^i \dot{p}_i = 0.
	\]
	Observe that the \hyperref[def:cogeodesic-flow]{cogeodesic flow} maps the set \(E_\lambda \coloneqq \left\{ (x, p)\in T^{\ast} \mathcal{M} \mid H(x, p) = \lambda  \right\}\) onto itself for all \(\lambda \geq 0\).
\end{explanation}

\begin{remark}
	If \(\mathcal{M} \) is compact, then all \(E_\lambda \) are compact, then all \hyperref[def:geodesic-flow]{geodesic flows} are defined on all \(E_\lambda \) for all \(\lambda \).
\end{remark}

\begin{remark}
	\(\mathcal{M} = \bigcup_{\lambda \geq 0} P E_\lambda \) for \(P\) being the projection.
\end{remark}

\section{Parallelism}
An important concept related to curvatures is ``parallelism,'' which needs a formal introduction of \hyperref[def:covariant-derivative]{covariant derivatives}.\footnote{Although we say we're going to treat them the same as \(\nabla \).} As a motivating example, the following is an equivalent definition of \hyperref[def:geodesic]{geodesic}.

\begin{eg}[Autoparallel]
	The \hyperref[def:geodesic]{geodesic} \(c\) satisfies \(\nabla _{\dot{c} } \dot{c} = 0\). This is called \hyperref[def:autoparallel]{autoparallel}.
\end{eg}
\begin{explanation}
	In the \hyperref[def:coordinate-chart]{local coordinates}, we have \(\dot{c} = \dot{c}^i \partial / \partial x^i\), and note that
	\begin{equation}\label{eq:autoparallel-geodesic}
		\nabla _{\dot{c}}\dot{c}
		= \dot{c}^i \nabla _{\frac{\partial }{\partial x^i} } \dot{c}^j \frac{\partial }{\partial x^j}
		= \dot{c}^i \dot{c}^j \Gamma _{ij}^k \frac{\partial }{\partial x^k} + \ddot{c}^k \frac{\partial }{\partial x^k}
		= \left( \ddot{c}^k + \Gamma _{ij}^k \dot{c}^i \dot{c}^j \right) \frac{\partial }{\partial x^k}
		= 0
	\end{equation}
	since a \hyperref[def:geodesic]{geodesic} is the solution of \autoref{eq:geodesic}.
\end{explanation}

To understand what \(\nabla _{\dot{c} } \dot{c} \) is doing beyond just calculation, we need to understand \hyperref[def:parallel-transport]{parallel transports}.

\subsection{Covariant Derivatives}


\begin{prev}
	The set of smooth \hyperref[def:vector-field-along-curve]{vector fields along \(c\)} is denoted as \(\mathcal{X} _c(\mathcal{M} )\).
\end{prev}

We can now finally define \hyperref[def:covariant-derivative]{covariant derivative} formally.

\begin{definition}[Covariant derivative]\label{def:covariant-derivative}
	The \emph{covariant derivative} of \(V\) along \(c\) is the \hyperref[def:vector-field-along-curve]{vector field} \(\mathrm{D} V / \mathrm{d} t\) in \autoref{prop:covariant-derivative}.
\end{definition}

\begin{prev}
	Let \(X = X^i \frac{\partial }{\partial x_i}\), \(V = V^k \frac{\partial}{\partial x_k}\), and let \(\mathrm{D} \) be the \hyperref[def:Levi-Civita-connection]{Levi-Civita connection}. Then
	\[
		\mathrm{D}_V X
		= \mathrm{D} _V(X^i \frac{\partial}{\partial x_i})
		= V(X^i) \frac{\partial}{\partial x_i} + X^i \underbrace{\mathrm{D} _V \frac{\partial}{\partial x_i}}_{\mathclap{V^k \mathrm{D}_{\frac{\partial}{\partial x_k}} \frac{\partial}{\partial x_i}}}
		= V(X^i)\frac{\partial}{\partial x_i} + V^k X^i \Gamma ^j_{ki} \frac{\partial }{\partial x_j} .
	\]
\end{prev}

\begin{proposition}[Covariant derivative]\label{prop:covariant-derivative}
	Let \((\mathcal{M} , g)\) be a \hyperref[def:Riemannian-manifold]{Riemannian manifold}, \(\mathrm{D}\) the canonical \hyperref[def:Levi-Civita-connection]{(Levi-Civita) connection}, and \(c\) a \hyperref[def:curve]{smooth curve} in \(\mathcal{M} \). Then there exists a unique operator \(\mathrm{D} / \mathrm{d} t\) defined as the vector space of \hyperref[def:vector-field-along-curve]{vector fields along \(c\)} satisfying
	\begin{enumerate}[(i)]
		\item \label{prop:covariant-derivative-i}\begin{enumerate}[(a)]
			      \item \label{prop:covariant-derivative-i-a} \(\frac{\mathrm{D}}{\mathrm{d}t} (fY)(t) = f^\prime (t) Y(t) + f(t) \frac{\mathrm{D} }{\mathrm{d} t} Y(t)\) for all \(f\in C^{\infty} (I)\) and \(Y \in \mathcal{X} _c(\mathcal{M} )\);
			      \item \label{prop:covariant-derivative-i-b} \(\frac{\mathrm{D} }{\mathrm{d} t} (V+W) = \frac{\mathrm{D} V}{\mathrm{d} t} + \frac{\mathrm{D} W}{\mathrm{d} t}\) for all \(V, W\in \mathcal{X} _c(\mathcal{M} )\);
		      \end{enumerate}
		\item \label{prop:covariant-derivative-ii} if there exists a neighborhood of in \(I\) such that \(Y\) is the restriction to \(c\) of a \hyperref[def:vector-field]{vector field} \(X\) defined on a neighborhood of \(c(t_0)\) in \(\mathcal{M} \), then \(\frac{\mathrm{D} }{\mathrm{d} t} Y(t_0) = (\mathrm{D}_{c(t_0)} X) _{c(t_0)}\).
	\end{enumerate}
\end{proposition}
\begin{proof}
	Consider defining such an operator \(\mathrm{D} / \mathrm{d} t\) as
	\[
		\frac{\mathrm{D}}{\mathrm{d}t} \left( Y^i (t)\frac{\partial}{\partial x_i} \right)
		= \frac{\mathrm{d}V^i}{\mathrm{d}t} \frac{\partial}{\partial x_i} + \dot{c}Y^i \Gamma _{ji}^k(c(t)) \frac{\partial}{\partial x_k},
	\]
	where \(\dot{c} = \dot{c}^k \frac{\partial}{\partial x_k}\). This shows \autoref{prop:covariant-derivative-i} \autoref{prop:covariant-derivative-i-a} and \autoref{prop:covariant-derivative-i-b} hold. Next, to show \autoref{prop:covariant-derivative-ii}, let \(x\) be a smooth \hyperref[def:vector-field]{vector field} in \(\mathcal{M} \). Then the induced \hyperref[def:vector-field-along-curve]{vector field along \(c\)} is given by \(Y(t)=X_{c(t)}\), i.e., in terms of the coordinate basis, we have
	\[
		Y(t) = Y^i(t)\frac{\partial}{\partial x_i},\quad X_x= X^i(x)\frac{\partial}{\partial x_i},\quad Y^i(t) = X^i (c(t)).
	\]
	Then,
	\[
		\begin{split}
			\mathrm{D} _i X
			&= \mathrm{D} _i \left( X^i \frac{\partial}{\partial x_i} \right)
			= \dot{c}(X^i)\frac{\partial}{\partial x_i} + X^i \mathrm{D} _i\frac{\partial}{\partial x_i}
			= X^i \dot{c}^k \underbrace{\mathrm{D} _{\frac{\partial}{\partial x_k}} \frac{\partial}{\partial x_i}}_{\Gamma _{ki}^{\ell } \frac{\partial}{\partial x_{\ell}}}\\
			&= \partial _t(X^i \circ c)\frac{\partial}{\partial x_i} + \dot{c}^k X^i \Gamma ^\ell _{ki} \frac{\partial}{\partial x_{\ell}}
			= \partial _t(X^i \circ c)\frac{\partial}{\partial x_i} + \dot{c}^k Y^i \Gamma ^\ell _{ki} \frac{\partial}{\partial x_{\ell }}
			= \frac{\mathrm{D}}{\mathrm{d} t} Y.
		\end{split}
	\]
\end{proof}

\begin{problem}
Why not just define \(\mathrm{D} Y / \mathrm{d} t\) by \autoref{prop:covariant-derivative-ii}?
\end{problem}
\begin{answer}
	A \hyperref[def:vector-field-along-curve]{vector field \(Y\) along a curve} may not always be extended to a neighborhood of \(c\) in \(\mathcal{M} \). But, in \hyperref[def:coordinate-chart]{local coordinates},
	\[
		Y(t) = \sum_{i=1}^{n} Y^i (t) \left( \frac{\partial }{\partial x^i}  \right) _{c(t)},
	\]
	which shows that a \hyperref[def:vector-field-along-curve]{vector field along \(c\)} is always a linear combination of \hyperref[def:vector-field-along-curve]{vector fields along \(c\)} that can be extended.
\end{answer}

\begin{remark}
	\autoref{prop:covariant-derivative} shows that the choice of an \hyperref[def:linear-connection]{linear connection} on \(\mathcal{M} \) leads to a bona fide (satisfying \autoref{prop:covariant-derivative-i-a} and \autoref{prop:covariant-derivative-i-b}) derivative of \hyperref[def:vector-field-along-curve]{vector fields along curves}. The notion of ``connection'' furnishes, therefore, a manner of differentiating vectors along \hyperref[def:curve]{curves}.
\end{remark}

\subsection{Parallel Transports}
Finally, we introduce the notion of \hyperref[def:parallel]{parallel}.

\begin{definition}[Parallel]\label{def:parallel}
	A \hyperref[def:vector-field-along-curve]{vector field \(X\) on \(\mathcal{M} \) along a curve \(c\)} is \emph{parallel} (or \emph{parallelly transported}) along \(c\) if \(\mathrm{D} X/ \mathrm{d} t = 0\) for all \(t\in I\).
\end{definition}

\begin{intuition}
	In the ordinary Euclidean space, i.e., when the space is flat, we know what is ``parallel,'' and hence we can define the directional derivative.
	\begin{center}
		\incfig{directional-derivative}
	\end{center}
	But now, the logic is reversed: we first define what is \hyperref[def:parallel]{parallel} in a curved space, and then we can make sense of directional derivative in a curved space!
\end{intuition}

Given the definition of a \hyperref[def:parallel]{parallel} \hyperref[def:vector-field-along-curve]{vector fields along curves}, we can talk about \hyperref[def:parallel-transport]{parallel transport}.

\begin{definition}[Parallel transport]\label{def:parallel-transport}
	The \emph{parallel transport} from \(c(0)\) to \(c(t)\) along the \hyperref[def:curve]{curve} \(c\) in a \hyperref[def:Riemannian-manifold]{Riemannian manifold} \((\mathcal{M} , g)\) is the linear map \(P_i \colon T_{c(0)} \mathcal{M} \to T_{c(t) }\mathcal{M} \) associating \(v\in T_{c(0)} \mathcal{M} \) with \(X_v(i)\in T_{c(i)}\mathcal{M} \) with \(X_v\) being the \hyperref[def:parallel]{parallel} \hyperref[def:vector-field-along-curve]{vector field along \(c\)} such that \(X_v (0) = v\).
\end{definition}

\begin{intuition}
	When the space is flat, keeping the ``arrow'' (which defines a \hyperref[def:vector-field]{vector field}) in one direction and moving around won't produce any changes, while when the space is curved, it will.
	\begin{center}
		\incfig{parallel-transport}
	\end{center}
\end{intuition}

We make a surprising remark on the relation between \hyperref[def:Riemannian-curvature]{Riemannian curvature} and \hyperref[def:parallel-transport]{parallel transport}.

\begin{remark}[Geometric significant of Riemannian curvature]
	The idea is that for a \hyperref[def:smooth-manifold]{manifold} with \hyperref[def:torsion-free]{torsion free} \(\nabla \), if we \hyperref[def:parallel-transport]{parallel transporting} along two paths on an infinitesimal patch (which induces \(X, Y\)) such that \([X, Y]=0\), we can detect \hyperref[def:Riemannian-curvature]{curvature} in terms of \(\delta z\), where\footnote{This is a deep theorem! In the \(\ldots \), we use \(T \equiv 0\).}
	\[
		(\delta z)^i = \ldots = R^i_{jk \ell } X^k Y^{\ell} Z^j\cdot \delta s \delta t + O(\delta s^2 \delta t, \delta s \delta t^2).
	\]
	\begin{center}
		\incfig{Riemannian-curvature-geometric-significant}
	\end{center}
	We will come back to this later.
\end{remark}

\begin{proposition}
	The \hyperref[def:parallel-transport]{parallel transport} exists, uniquely.
\end{proposition}
\begin{proof}
	do Carmo~\cite[Proposition 2.6]{flaherty2013riemannian}
\end{proof}

\begin{proposition}
	Let \((\mathcal{M} , g)\) be a \hyperref[def:Riemannian-manifold]{Riemannian manifold}. The \hyperref[def:parallel-transport]{parallel transport} defines for all \(t\) an \hyperref[def:isometry]{isometry} from \(T_{c(0)} \mathcal{M} \) onto \(T_{c(t)} \mathcal{M} \); more generally, if \(X, Y\) are \hyperref[def:vector-field-along-curve]{vector fields along \(c\)}, then
	\[
		\frac{\mathrm{d}}{\mathrm{d}t} g(x(t), y(t))
		= g\left( \frac{\mathrm{D} X(t)}{\mathrm{d} t}, Y(t)\right) + g\left(X(t), \frac{\mathrm{D} Y(t)}{\mathrm{d} t} \right) .
	\]
\end{proposition}
\begin{proof}
	See do Carmo~\cite[Proposition 3.2]{flaherty2013riemannian}
\end{proof}

\subsection{Autoparallel Curves}
Now we can formally introduce the notion of \hyperref[def:autoparallel]{autoparallel}.

\begin{definition}[Autoparallel]\label{def:autoparallel}
	Let \(\nabla \) be a \hyperref[def:linear-connection]{connection} on \(T \mathcal{M} \) of a \hyperref[def:smooth-manifold]{differentiable manifold} \(\mathcal{M} \). A \hyperref[def:curve]{curve} \(c\colon I \to \mathcal{M} \) is called \emph{autoparallel} (or \emph{geodesic}) w.r.t.\ \(\nabla \) if
	\[
		\nabla _{\dot{c}}\dot{c} = 0.
	\]
\end{definition}

\begin{intuition}
	An \hyperref[def:autoparallel]{autoparallel} \hyperref[def:curve]{curve} is the \emph{straightest line} (hence \hyperref[def:geodesic]{geodesic}) in the space w.r.t.\ \(\nabla \)!
\end{intuition}

\begin{remark}[Physical interpretation]
	One can start from introducing \(\nabla \), considering \(\nabla _{\dot{c} }\dot{c} \coloneqq 0 \) (which is just \autoref{eq:geodesic}), and realize that we don't need to consider gravity as a force, rather a ``curvature of spacetime,'' in order to make sense of Newton's first law, i.e., mass without forces will undergo a \hyperref[def:autoparallel]{autoparallel} \hyperref[def:curve]{curve}.
\end{remark}

\begin{eg}[Euclidean plane]
	Let \(U = \mathbb{R} ^2\), \(x = \identity_{\mathbb{R} ^2} \), \(\Gamma ^i_{jk} = 0\), then \(\ddot{c} ^k = 0 \) in \autoref{eq:autoparallel-geodesic}. Hence,
	\[
		c^k (t) = a^k t + b^k \text{ for } a^k, b^k\in \mathbb{R} ^d.
	\]
\end{eg}

\begin{eg}[Round sphere]
	The \hyperref[def:geodesic]{geodesics} on a ``round sphere'' are the great circles.
\end{eg}
\begin{explanation}
	Consider a ``unit round sphere'' \(\mathcal{M} = S^2\) with spherical coordinates \(x(p) = (r, \theta , \varphi )\) such that \(r=1\), \(\theta \in(0, \pi )\), and \(\varphi \in 0, 2\pi \). The ``roundness'' is given by \(\nabla _{\text{round}}\) where we specify (at one point)
	\[
		\Gamma^1_{22} \coloneqq - \sin \theta \cos \theta ,\quad
		\Gamma^2_{21} = \Gamma^2_{12} \coloneqq \cot \theta,
	\]
	where we let \(x^1(p) = \theta (p), x^2(p) = \varphi (p)\). The \hyperref[def:autoparallel]{autoparallel equation} tells us
	\[
		\begin{dcases}
			\ddot{\theta } + \Gamma ^1_{22} \dot{\varphi } \dot{\varphi } = 0; \\
			\ddot{\varphi } + 2\Gamma ^2_{12} \dot{\theta } \dot{\varphi } = 0;
		\end{dcases}
		\iff \begin{dcases}
			\ddot{\theta } - \sin (\theta) \cos (\theta) \dot{\varphi } \dot{\varphi } = 0; \\
			\ddot{\varphi } + 2 \cot (\theta) \dot{\theta } \dot{\varphi } = 0.
		\end{dcases}
	\]
	Then, we see that \(\theta (t) = \pi / 2\), \(\varphi (t) = \omega t + \varphi _0\) is a solution.\footnote{Note that \(\theta (t) = \pi / 2\), \(\varphi (t) = \omega t^2 + \varphi _0\) is not a solution.} Hence, we conclude that if we run at a constant speed around the great circle of \(S^2\), it'll be \hyperref[def:autoparallel]{autoparallel}, hence a \hyperref[def:geodesic]{geodesic}.
\end{explanation}

Similarly, given any \(\nabla \) on a space, we can find the straightest \hyperref[def:curve]{curve} on which.