\lecture{3}{12 Jan. 14:30}{Complex Manifolds, Tangent Spaces and Bundles}
Let's look at two more examples about \hyperref[def:orientation]{orientation}.

\begin{eg}
	Let \(A\colon S^n \to  S^n\) be the antipodal map given by \(A(p) = -p\) for \(p\in \mathbb{R} ^{n+1}\). It's easy to see that \(A\) is differentiable with \(A^2 = \mathbbm{1} \). Furthermore, \(A\) is \hyperref[def:diffeomorphism]{diffeomorphism} of \(S^n \subseteq \mathbb{R} ^{n+1}\). We see that
	\begin{itemize}
		\item if \(n\) is even, \(A\) reverses the \hyperref[def:orientation]{orientation};
		\item if \(n\) is odd, \(A\) preserves the \hyperref[def:orientation]{orientation}.
	\end{itemize}
\end{eg}

\begin{eg}
	\(G(k, n)\) is \hyperref[def:orientation]{orientable} if and only if \(n\) is even or \(n=1\).
\end{eg}

\subsection{Complex Manifolds}
Here we introduce the notion of \hyperref[def:complex-manifold]{complex manifold}.

\begin{definition}[Complex manifold]\label{def:complex-manifold}
	A \emph{complex manifold} \(\mathcal{M} \) of complex dimension \(d\) (\(\dim _{\mathbb{C} }\mathcal{M} = d\)) is a \hyperref[def:smooth-manifold]{differentiable manifold} of (real) dimension \(2d\) (\(\dim _\mathbb{R} \mathcal{M} =2d\)) whose \hyperref[def:coordinate-chart]{charts} take values in open subsets of \(\mathbb{C} ^d\) with holomorphic \hyperref[def:coordinate-transition]{chart transitions}.
\end{definition}

\begin{prev}
	The \hyperref[def:coordinate-transition]{chart transitions} \(z_\beta \circ z_\alpha ^{-1} \colon z_\alpha (U_\alpha \cap U_\beta ) \to  z_\beta (U_\alpha \cap U_\beta )\) is holomorphic if \(\partial z_\beta ^j / \partial \overline{z_\alpha ^k} = 0\) for all \(j, k\) where
	\[
		\frac{\partial }{\partial \overline{z^k}} = \frac{1}{2} \left( \frac{\partial }{\partial \overline{x^k}} + i \frac{\partial }{\partial \overline{y^k}} \right).
	\]
\end{prev}

\begin{remark}
	\hyperref[def:complex-manifold]{Complex} \hyperref[def:Grassmannian-manifold]{Grassmannians} \(G_{\mathbb{C} }(k, n)\) are all \hyperref[def:orientable]{orientable}. More generally, \hyperref[def:complex-manifold]{complex manifolds} are always \hyperref[def:orientable]{orientable} because holomorphic maps always have positive functional determinant.
\end{remark}

\subsection{Partition of Unity}
We state, without proof, of an important lemma about the \hyperref[def:partition-of-unity]{partition of unity}.

\begin{definition}[Partition of unity]\label{def:partition-of-unity}
	Let \(\mathcal{M} \) be a \hyperref[def:smooth-manifold]{differentiable manifold}, and let \((U_\alpha )_{\alpha \in \mathcal{A} }\) be an open covering of \(\mathcal{M} \). Then a \emph{partition of unity} is a \hyperref[def:locally-finite]{locally finite} refinement \((V_\beta )_{\beta \in \mathcal{B} }\) of \((U_\alpha )\) and \(C^{\infty} \)-functions \(\varphi _\beta \colon \mathcal{M} \to  \mathbb{R} \) with
	\begin{enumerate}[(a)]
		\item \(\mathop{\mathrm{supp}}(\varphi _\beta ) \subseteq V_\beta  \) for all \(\beta \in \mathcal{B} \);
		\item \(0 \leq \varphi _\beta (x) \leq 1\) for all \(x\in \mathcal{M} \), \(\beta \in \mathcal{B} \);
		\item \(\sum_{\beta \in \mathcal{B} } \varphi _\beta = 1 \) for all \(x\in \mathcal{M} \).\footnote{There are only finitely many non-vanishing summands of each point, since only finitely many \(\varphi _\beta \) are non-zero of any given point as the covering \((V_\beta )\) is \hyperref[def:locally-finite]{locally finite}.}
	\end{enumerate}
\end{definition}

\begin{lemma}[Partition of unity]\label{lma:partition-of-unity}
	Let \(\mathcal{M} \) be a \hyperref[def:smooth-manifold]{differentiable manifold}, and let \((U_\alpha )_{\alpha \in \mathcal{A} }\) be an open covering of \(\mathcal{M} \). Then there exists a \hyperref[def:partition-of-unity]{partition of unity} subordinate to \((U_\alpha )\),
\end{lemma}

\section{Tangent Vectors}
To discuss the concept of calculus between \hyperref[def:smooth-manifold]{manifolds} formally, we start with our discussion in Euclidean spaces, where we naturally have the coordinates for every point.

\begin{definition}[Tangent space of Euclidean space]\label{def:tangent-space-of-Euclidean-space}
	Given a \(d\) dimensional \hyperref[def:topological-manifold]{manifold} \(\mathcal{M} \), let \(x = (x^1, \ldots , x^d)\) be Euclidean coordinates of \(\mathbb{R} ^d\), and \(x_0\in \Omega \subseteq \mathbb{R} ^d\) where \(\Omega \) is open. The \emph{tangent space} \(T_{x_0}\Omega \) of \(\Omega \) at the point \(x_0\) is the vector space \(\left\{ x_0 \right\} \times E\)\footnote{\(E\) is a \(d\)-dimensional Euclidean space.} spanned by the basis \((\partial / \partial x^1, \ldots , \partial / \partial x^d)\).
\end{definition}

\begin{definition}[Derivative of Euclidean space]\label{def:derivative-of-Euclidean-space}
	If \(\Omega \subseteq \mathbb{R} ^d\), \(\Omega ^\prime \subseteq \mathbb{R} ^d\) open, and \(f\colon \Omega \to \Omega ^\prime \) differentiable, then we define the \emph{derivative} \(\mathrm{d} f(x_0)\) for \(x_0 \in \Omega \) to be the induced linear map between \hyperref[def:tangent-space-of-Euclidean-space]{tangent spaces}
	\[
		\mathrm{d} f(x_0) \colon T_{x_0}\Omega \to T_{f(x_0)}\Omega ^\prime,\quad
		v = v^i \frac{\partial }{\partial x^i} \mapsto v^i \frac{\partial f^j}{\partial x^i} (x_0) \frac{\partial }{\partial f^j}.
	\]
\end{definition}

\begin{notation}[\href{https://en.wikipedia.org/wiki/Einstein_notation}{Einstein notation}]
	The \emph{Einstein notation} abbreviates the summation \(\sum_{i} v^i x_i\) as \(v^i x_i\), where we implicitly sum over the upper and lower index.
\end{notation}

\begin{definition}[Tangent bundle of Euclidean space]\label{def:tangent-bundle-of-Euclidean-space}
	The \emph{tangent bundle} is defined as \(T \Omega\coloneqq \bigsqcup_{x\in \Omega }T_x \Omega \cong \Omega \times E \cong \Omega \times \mathbb{R} ^d\), which is an open subset of \(\mathbb{R} ^d \times \mathbb{R} ^d\).
\end{definition}

\begin{note}[Total space]
	\(T \Omega \) is also called the \emph{total space}.
\end{note}

\begin{remark}
	Given a \hyperref[def:tangent-bundle-of-Euclidean-space]{tangent bundle} \(T \Omega \), we define \(\pi \) to be the projection \(\pi \colon T \Omega \to \Omega\) given by \(\pi (x, v) = x\). This makes \(T \Omega \) naturally a \hyperref[def:smooth-manifold]{differentiable manifold}.
\end{remark}

We also have \(\mathrm{d} f\colon T \Omega \to T \Omega ^\prime \) defined by
\[
	\left( x, v^i \frac{\partial }{\partial x^i} \right) \mapsto \left( f(x), v^i \frac{\partial f^j}{\partial x^i} (x_0) \frac{\partial }{\partial f^j}  \right).
\]

\begin{notation}
	We often write \(\mathrm{d} f(x)(v)\) instead of \(\mathrm{d} f(x, v)\).
\end{notation}

In particular, \(f\colon \Omega \to  \mathbb{R} \), the differentiable function for \(v = v^i \partial / \partial x^i\), we have
\[
	\mathrm{d} f(x)(v) = v^i \frac{\partial f}{\partial x^i} (x)\in T_{f(x)}\mathbb{R} \cong \mathbb{R},
\]
and we write \(v(f)(x)\) for \(\mathrm{d} f(x)(v)\).

Let \(\mathcal{M} \) be a differentiable manifold of dimension \(d\), \(p\in \mathcal{M} \). The tangent space of \(\mathcal{M} \) at point \(p\). Let \(x\colon U \to  \mathbb{R} ^d\) be a chart with \(p\in U \subseteq \mathcal{M} \), open. The tangent space \(T_p \mathcal{M} \) is represented in the chart \(x\) by \(T_{x(p)}x(U)\). Let \(x^\prime \colon U^\prime \to \mathbb{R} ^d\) to be another chart with \(p\in U^\prime \subseteq \mathcal{M} \), open. Denote \(\Omega \coloneqq x(U)\), and \(\Omega ^\prime \coloneqq x^\prime (U^\prime )\), then the transition map
\[
	x^\prime \circ x^{-1} \colon x(U \cap U^\prime )\to x^\prime (U \cap U^\prime )
\]
induces a vector space isomorphism
\[
	L\coloneqq \mathrm{d} (x^\prime \circ x ^{-1} )(x(p)) \colon T_{x(p)}\Omega \to  T_{x^\prime (p)}\Omega ^\prime,
\]
such that \(v\in T_{x(p)}\Omega \) and \(L(v)\in T_{x^\prime (p)}\Omega ^\prime \) represent the same tangent vector in \(T_p \mathcal{M} \).

\begin{remark}
	A tangent vector in \(T_p \mathcal{M} \) is given by the family of the coordinate representations.
\end{remark}

\begin{intuition}
	Let \(f\colon \mathcal{M} \to  \mathbb{R} \) be a differentiable function. Assume that the tangent vector \(w\in T_p \mathcal{M} \) is represented by \(v\in T_{x(p)}x(U)\). We want to define \(\mathrm{d} f(p)\) as a linear map from \(T_p \mathcal{M} \to \mathbb{R} \). In chart \(x\), let \(w\in T_p \mathcal{M} \) given as \(v = v^i \partial /\partial x^i\in T_{x(p)}x(U)\). Say that \(\mathrm{d} f(p)(w)\) in this chart represented by
	\[
		\mathrm{d} (f \circ x ^{-1} )(x(p)) (v).
	\]
	\begin{center}
		\incfig{tangent-space}
	\end{center}
\end{intuition}

\begin{remark}
	\(T_p \mathcal{M} \) is a vectors peace of dimension \(d\) isomorphic to \(\mathbb{R} ^d\), where the isomorphism depends on choice of chart.
\end{remark}

\begin{remark}
	Functions on \(\mathcal{M} \): pull it back by a chart to an open subset of \(\mathbb{R} ^d\), differentiate there.
\end{remark}

\begin{remark}
	In order to obtain object not depending on chart, we need to have transformation behavior under chart changes.
\end{remark}

Let \(F\colon \mathcal{M} \to  \mathcal{N} \) be a differentiable map where \(\mathcal{M} , \mathcal{N} \) are smooth manifolds. Then we want to represent \(\mathrm{d} F\) in local charts \(x\colon U \subseteq \mathcal{M} \to  \mathbb{R} ^d, y\colon V \subseteq \mathcal{N} \to  \mathbb{R} ^c\) by \(\mathrm{d} (y \circ F \circ x ^{-1} )\). The local coordinates on \(U\) is given by \((x^1, \ldots , x^d)\), and on \(V\) is \((F^1, \ldots , F^c)\) such that
\[
	F(x) = \big(F^1(x^1, \ldots , x^d), \ldots , F^c(x^1, \ldots , x^d)\big).
\]
Then, \(\mathrm{d} F\) induces linear map \(\mathrm{d} F\colon T_p \mathcal{M} \to  T_{F(x)}\mathcal{N} \) which in our coordinate representation is given by matrix
\[
	\left( \frac{\partial F^\alpha }{\partial x^i}  \right) _{\substack{\alpha =1, \ldots , c \mathbb{I}  g 1, \ldots , d}}
\]

\begin{center}
	\incfig{differentiation-between-manifolds}
\end{center}

a change of charts is then just the base change at tangent spaces. The transformation behavior: if
\[
	\begin{split}
		(x^1, \ldots , x^d) &\mapsto (\xi ^1, \ldots , \xi ^d)\\
		(F^1, \ldots , F^c) &\mapsto (\phi ^1, \ldots , \phi ^c)
	\end{split}
\]
are coordinate changes, then \(\mathrm{d} F\) represented in the new coordinates is given by
\[
	\left( \frac{\partial \phi ^\beta }{\partial \xi ^j}  \right)
	= \left( \frac{\partial \phi ^\beta }{\partial F^\alpha } \frac{\partial F^\alpha }{\partial x^i} \frac{\partial x^i}{\partial \xi ^j} \right).
\]