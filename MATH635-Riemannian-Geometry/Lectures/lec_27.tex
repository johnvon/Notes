\lecture{27}{13 Apr. 13:00}{Uniformization Theorem}
Before continue the proof of \autoref{thm:sphere*}, we need one more tool.

\begin{theorem}[Brown's theorem]\label{thm:Brown}
	Let \(\mathcal{M} \) be a smooth and compact manifold. If \(\mathcal{M} = U_1 \cap U_2\) with \(U_1, U_2\) open subsets in \(\mathcal{M} \) homeomorphic to \(\mathbb{R} ^n\), then \(\mathcal{M} \) is homeomorphic to \(S^n\).
\end{theorem}

We can then finish the proof.

\begin{proof}[Proof of \autoref{thm:sphere*} (Continue)]
	We have shown that
	\[
		\overline{B_{\frac{\pi}{2\sqrt{k} }}(p)} \cup \overline{B_{\frac{\pi}{2 \sqrt{k} }}(q)} = \mathcal{M}.
	\]
	Denote \(r\) to be
	\[
		r \coloneqq \frac{1}{2} \left( \mathop{\mathrm{inj}}(\mathcal{M} , g) + \frac{\pi}{2 \sqrt{k} } \right) > \frac{\pi}{2 \sqrt{k}} ,
	\]
	then \(\mathcal{M} = B_r(p) \cup B_r(q)\). Moreover, since \(r < \mathop{\mathrm{inj}}(\mathcal{M} , g) \), both \(B_r(p)\) and \(B_r(q)\) are homeomorphic to \(\mathbb{R} ^n\). By \hyperref[thm:Brown]{Brown's theorem}, the result follows.
\end{proof}

\subsection{The Family of the Sphere Theorem}
The \hyperref[thm:sphere*]{sphere theorem} doesn't hold for \(1 / 4 \leq K \leq 1\).\footnote{Where we originally have \(1 / 4 < K \leq 1\).}

\begin{eg}
	The complex projective spaces \(\mathbb{C} \mathbb{P} ^n\) are also compact, simply connected \hyperref[def:Riemannian-manifold]{Riemannian manifold} such that \(1 / 4 \leq K \leq 1\). But they are not homeomorphic to \(S^{2m} \).
\end{eg}

However, this is ``almost'' the only counterexample in the following sense.

\begin{theorem}
	Let \(\mathcal{M} ^n\) be a compact and simply-connected \hyperref[def:Riemannian-manifold]{Riemannian manifold}.
	\begin{enumerate}[(a)]
		\item If \(m\) is even, then there exists \(\epsilon (m) > 0\) such that \(1 / 4 - \epsilon (m) \leq K \leq 1\), then \(\mathcal{M} \) is either homeomorphic to \(S^n\) or \hyperref[def:diffeomorphic]{diffeomorphic} to either \(\mathbb{C} \mathbb{P} ^{m / 2}\), \(\mathbb{H} \mathbb{P} ^{m / 4}\), or \(\mathbb{C} _a \mathbb{P} ^2\)~\cite{AIF_1983__33_2_135_0}.
		\item If \(m\) is odd, then there exists \(\epsilon > 0\) such that if \(1 / 4 - \epsilon \leq K \leq 1\), then \(\mathcal{M} \) homeomorphic to \(S^n\)~\cite{Abresch1996AST}.
	\end{enumerate}
\end{theorem}

Looking back,
\begin{itemize}
	\item Rauch in 1951 proved the \hyperref[thm:sphere*]{sphere theorem} for \(3 / 4 < K \leq 1\)~\cite{10.2307/1969309};
	\item Klingenberg in 1959 proved the \hyperref[thm:sphere*]{sphere theorem} for \(0.55 < K \leq 1\)~\cite{10.2307/1970029};
	\item Berger in 1960 proved the \hyperref[thm:sphere*]{sphere theorem} for \(1 / 4 < K \leq 1\) when \(m\) is even~\cite{berger1960varietes};
	\item Klingenberg in 1961 proved the \hyperref[thm:sphere*]{sphere theorem}~\cite{Klingenberg1961}.
\end{itemize}

Now, what if we want \hyperref[def:diffeomorphism]{diffeomorphism}? ``Exotic spheres'' exists: \hyperref[def:smooth-manifold]{manifolds} that are homeomorphic to sphere but not \hyperref[def:diffeomorphic]{diffeomorphic}.

\begin{eg}[J. Milnor]
	If \(n = 7\), we can construct as \(S^3\)-\hyperref[def:bundle]{bundles} over \(S^4\).
\end{eg}

Consider \(m = 2\), from the \hyperref[thm:Gauss-Bonnet]{Gauss-Bonnet theorem}, \(\mathcal{M} \) is \hyperref[def:diffeomorphic]{diffeomorphic} to \(S^2\) since
\[
	0 < \int _\mathcal{M} K \,\mathrm{d} A = 2 \pi \chi (\mathcal{M} ),
\]
with the fact that \(S^2\) is the only such object. As for \(m = 3\), by the \hyperref[thm:Hamilton]{Hamilton's proof}, if \((\mathcal{M} , g)\) is a \(3\)-dimension compact \hyperref[def:Riemannian-manifold]{Riemannian manifold} with \hyperref[def:Ricci-curvature]{Ricci curvature} \(> 0\) then \((\mathcal{M} , g)\) is \hyperref[def:diffeomorphic]{diffeomorphic} to \(S^3\) using Ricci flow, e.g.,~\cite{brendle2008manifolds}.

\chapter{Epilogue}
\section{Uniformization Theorem}
Recall the \hyperref[thm:Gauss-Bonnet]{Gauss-Bonnet theorem}, where we let \((\mathcal{M} , g)\) be a compact, \(2\)-dimension, oriented \hyperref[def:Riemannian-manifold]{Riemannian manifold} without boundary. And let \(K\) be the \hyperref[rmk:Gauss-curvature]{Gauss curvature}, \(\gamma \) be the genus of \(\mathcal{M} \). Then,
\[
	\int _\mathcal{M} K \,\mathrm{d} \mu _g = 2 \pi \chi .
\]
In particular, there are three cases:
\begin{enumerate}[(a)]
	\item \(\gamma = 1\), then \(\chi = 0\);
	\item \(\gamma \geq 2\), then \(\chi\) is negative integer;
	\item \(\gamma = 0\), then \(\chi = 2\).
\end{enumerate}

\begin{definition}[Conformal]\label{def:conformal}
	We say a \hyperref[def:Riemannian-metric]{metric} \(g\) is \emph{conformal} to another \hyperref[def:Riemannian-metric]{metric} \(\widetilde{g} \) if there exists a smooth positive function \(\Omega \) on \(\mathcal{M} \) such that
	\[
		\widetilde{g} = \Omega ^2 \circ g
	\]
	has constant \hyperref[rmk:Gauss-curvature]{Gauss curvature}.
\end{definition}

\begin{note}[Constant rescaling]
	If we take \(\Omega = c\) for some constant, then \(\widetilde{K} = c^{-2} K\).
\end{note}

\begin{theorem}[Uniformization theorem]\label{thm:uniformization}
	If \(g\) is \hyperref[def:conformal]{conformal} to a \hyperref[def:Riemannian-metric]{metric} of constant \hyperref[rmk:Gauss-curvature]{Gauss curvature}, then
	\[
		\widetilde{K} = \begin{dcases}
			0,  & \text{ if } \gamma = 1 ;    \\
			-1, & \text{ if } \gamma \geq 2 ; \\
			1 , & \text{ if } \gamma = 0.
		\end{dcases}
	\]
\end{theorem}

First, we check how \hyperref[rmk:Gauss-curvature]{Gauss curvature} transforms under \hyperref[def:conformal]{conformal} transformations. Consider an \(n\)-dimensional \hyperref[def:Riemannian-manifold]{Riemannian manifold} \((\mathcal{M} , g)\) and \(\widetilde{g} _{ij} = \Omega ^2 g_{ij}\), then
\[
	\widetilde{\Gamma} ^k_{ij} = \Gamma ^k_{ij} + \Omega ^{-1} (\delta _i^k \partial _j \Omega + \delta _j^k \partial _i \Omega - g^{k \ell } g_{ij} \partial _\ell \Omega ).
\]
\begin{remark}\todo{Fix}
	Moreover, we can also compute how do the \hyperref[def:Riemannian-curvature]{Riemannian curvature}, \hyperref[def:Ricci-curvature]{Ricci curvature}, and also \hyperref[def:sectional-curvature]{sectional curvature} transform.
\end{remark}

In particular, for \(n=2\), set \(\Omega = e^u\) for some smooth function \(u\), then the \hyperref[rmk:Gauss-curvature]{Gauss curvature} \(K\) is transformed as
\[
	\widetilde{K} = e^{-2u} (K - \Delta _g u).
\]

Now, we want to find some smooth solutions \(u\) of
\[
	\Delta _g u + \widetilde{K} e^{2u} = K
\]
when \(\gamma = 0\), \(\chi = 2\), we have
\[
	\Delta _g u + e^{2u} = K.
\]

\begin{note}
	Maximum principle\todo{Fix} does not work.
\end{note}

To prove \autoref{thm:uniformization}, we follow \(5\) steps:
\begin{enumerate}
	\item Find points \(p, o\in \mathcal{M} \) such that \(\mathop{\mathrm{dist}}(p, o) = \mathop{\mathrm{diam}}(\mathcal{M} ) \).
	\item Find a function \(w\) on \(\mathcal{M} \setminus \{ p \} \) such that \(\widetilde{g} = e^{2w} g\) on \(\mathcal{M} \setminus \{ p \} \) makes \((\mathcal{M} \setminus \{ p \} , \widetilde{g} )\) \hyperref[def:isometry]{isometric} to a plane, i.e., \(\widetilde{K} = 0\). From here, we get
	      \[
		      \Delta _g w - K = 0
	      \]
	      on \(\mathcal{M} \setminus \{ p \} \). So naturally, we define \(w\) to be a solution of
	      \[
		      \Delta _g w - K = -4 \pi \delta _p
	      \]
	      on \(\mathcal{M} \), where \(\delta _p\) is the delta Dirac distribution at \(p\).
	\item Let \(\widetilde{d} _o\) be the distance function on \((\mathcal{M} \setminus \{ p \} , \widetilde{g} )\) from \(o\). Set
	      \[
		      e^v = \frac{1}{1 + \widetilde{d} _o^2/4},
	      \]
	      then \hyperref[def:Riemannian-metric]{metric} \(\widetilde{\widetilde{g}} = 2^{2v} \widetilde{g} \) has \hyperref[rmk:Gauss-curvature]{Gauss curvature} \(\widetilde{\widetilde{K}} = 1\). Namely, \((\mathcal{M} \setminus \{ p \} , \widetilde{\widetilde{g}} )\) will be \hyperref[def:isometry]{isometric} to a standard sphere minus the North Pole \(N\).
	\item Set \(u = w + v\), then we show that this function extends continuously to the point \(p\), i.e.,
	      \[
		      \widetilde{\widetilde{g}} = e^{2u} g
	      \]
	      is a \hyperref[def:Riemannian-metric]{metric} on \(\mathcal{M} \) such that \((\mathcal{M} , \widetilde{\widetilde{g}} )\) is \hyperref[def:isometry]{isometric} to a standard sphere.
	\item Show that \(u\) is smooth on \(\mathcal{M} \).
	      \begin{note}
		      \(w \to \infty \) while \(v \to -\infty \) as we approach \(p\). To show \(u = w + v\), we need to analyze the blow-behavior for \(w\) and \(v\).
	      \end{note}
\end{enumerate}

We now start our proof.
\begin{enumerate}[2.]
	\item Use the \hyperref[def:exponential-map]{exponential map}. Let \(r_p\) be the \hyperref[def:injectivity-radius]{injectivity radius} (relative to \((\mathcal{M} , g)\)) of \(\exp _p\). Choose \(\epsilon >0\) such that \(2 \epsilon < r_p\). Let \(\rho \) be a \(C^{\infty} \) non-increasing function on \([0, \infty )\) such that \(\rho = 1\) on \([0, 1]\) and \(\rho = 0\) on \([2, \infty )\). Define the cur-off function for \(q\in \mathcal{M} \) such that
	      \[
		      \eta (q) = \begin{dcases}
			      \rho (d_p / \epsilon ), & \text{ if } q \in B_{2\epsilon }(p) ; \\
			      0,                      & \text{ otherwise} .
		      \end{dcases}
	      \]
	      Namely, \(\eta = 1\) on \(B_{\epsilon } (p)\) and \(\eta = 0\) on \(\mathcal{M} \setminus B_{2\epsilon }(p)\). Define on \(\mathcal{M} \) the function
	      \[
		      w_o = \begin{dcases}
			      -2 \eta \log d_p, & \text{ in } B_{2\epsilon }(p)  ;                       \\
			      0,                & \text{ on } \mathcal{M} \setminus B_{2\epsilon }(p)  .
		      \end{dcases}
	      \]
	      In \(B_{\epsilon } (p)\), \(w_0 = - 2 \log d_p\). As \(2\epsilon < r_p\), \(\exp _p\) is a \hyperref[def:diffeomorphism]{diffeomorphism} of the ball of radius \(2\epsilon \) with center \(0\) in \(T_p \mathcal{M} \) onto \(B_{2\epsilon } (p)\) in \(\mathcal{M} \).
\end{enumerate}