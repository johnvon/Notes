\lecture{6}{24 Jan. 14:30}{Geodesic and the Exponential Map}
\begin{proposition}
	For all smooth curve \(\gamma \colon [a, b] \to \mathcal{M} \),
	\[
		\mathcal{L} (\gamma )^2 \leq 2(b-a) E(\gamma )
	\]
	with equality if and only if \(\lVert \mathrm{d} \gamma / \mathrm{d} t \rVert \) is a constant.
\end{proposition}
\begin{proof}
	From \href{https://en.wikipedia.org/wiki/H%C3%B6lder%27s_inequality}{Hölder's inequality},
	\[
		\int_{a}^{b} \left\lVert \frac{\mathrm{d}\gamma }{\mathrm{d}t} \right\rVert \,\mathrm{d}t
		\leq (b-a)^{1 / 2} \left( \int_{a}^{b} \left\lVert \frac{\mathrm{d}\gamma }{\mathrm{d}t} \right\rVert ^2 \,\mathrm{d}t \right) ^{1 / 2}
	\]
	with equality if and only if \(\lVert \mathrm{d} \gamma / \mathrm{d} t \rVert\) is a constant.
\end{proof}

\begin{eg}
	Let
	\[
		\mathcal{L} (q, x) = \frac{1}{2} m \vert q \vert ^2 - V(x)
	\]
	with \(m > 0, q = \dot{x}\), the Euler-Lagrangian equations is given by
	\[
		m\ddot{x} (s) = F(x(s))
	\]
	for \(F\coloneqq -DV\).
\end{eg}

\begin{prev}
	Regular curves can be parametrized by arc \hyperref[def:length]{length} with unit speed \(\lVert \mathrm{d} \gamma / \mathrm{d} t \rVert = \lVert \dot{\gamma } \rVert \equiv 1\).
\end{prev}

\begin{lemma}
	Each \hyperref[def:geodesic]{geodesic} is parametrized proportionally to the arc \hyperref[def:length]{length}.\footnote{This means that we have constant speed, i.e., \(\lVert \dot{\gamma } \rVert \) is a constant.}
\end{lemma}
\begin{proof}
	For a solution of \(\ddot{x}^i(t) + \Gamma ^{i}_{jk}(x(t)) \dot{x}^j(t)\dot{x}^k(t) = 0\),\todo{Do the computation!}
	\[
		\frac{\mathrm{d}}{\mathrm{d}t} \left\langle \dot{x}, \dot{x} \right\rangle
		= \frac{\mathrm{d}}{\mathrm{d}t} \left( g_{ij} (x(t)) \dot{x}^i(t)\dot{x}^j(t)\right)
		=0.
	\]
\end{proof}

Our goal now is to minimize the \hyperref[def:length]{length} within class of regular smooth curves.

\begin{prev}
	The \hyperref[def:length]{length} and the \hyperref[def:energy]{energy} functionals are invariants under parameter changes.
\end{prev}

This means that it's enough to look at curves parametrized by arc \hyperref[def:length]{length}.

\begin{theorem}\label{thm:lec6}
	Let \(\mathcal{M} \) be a \hyperref[def:Riemannian-manifold]{Riemannian manifold}, \(p\in \mathcal{M} \) and \(v\in T_p \mathcal{M} \). Then there exists an \(\epsilon > 0\) and a unique \hyperref[def:geodesic]{geodesic} such that \(c\colon [0, \epsilon ] \to  \mathcal{M} \) with \(c(0) = p\) and \(\dot{c}(0) = v\). In addition, \(c\) smoothly depend on \(p, v\).
\end{theorem}
\begin{proof}
	Since \autoref{eq:geodesic} is a system of second order ODE, by \href{https://en.wikipedia.org/wiki/Picard%E2%80%93Lindel%C3%B6f_theorem}{Picard-Lindelöf theorem}, we have local existence and uniqueness of the solution with prescribed initial values and derivative such that the solution depends smoothly on \(p, v\).
\end{proof}

\begin{remark}
	If \(x(t)\) is the solution of \autoref{eq:geodesic}, then \(x(\lambda t)\) is also a solution for any constant \(\lambda \in \mathbb{R} \). Denote \hyperref[def:geodesic]{geodesic} from \autoref{thm:lec6} by \(c_v\), then
	\[
		c_v(t) = c_{\lambda v}(t / \lambda )
	\]
	for \(\lambda > 0\), \(t\in [0, \epsilon ]\), and hence \(c_{\lambda v}\) defined on \([0, \epsilon / \lambda ]\). Since \(c_v\) depends smoothly on \(v\), the set \(\left\{ v\in T_p \mathcal{M} \mid \lVert v \rVert = 1 \right\} \) is compact, hence there exists \(\epsilon _0 > 0\) such that for \(\lVert v \rVert = 1\), \(c_v\) defined at least on \([0, \epsilon _0]\), implying that for all \(w\in T_p \mathcal{M} \) with \(\lVert w \rVert \leq \epsilon _0\), \(c_w\) is defined at least on \([0, 1]\).
\end{remark}

\subsection{Exponential Maps}

\begin{definition}[Exponential map]\label{def:exponential-map}
	Let \((\mathcal{M} , g)\) be a \hyperref[def:Riemannian-manifold]{Riemannian manifold}, \(p\in \mathcal{M} \), and \(V_p \coloneqq \left\{ v\in T_p \mathcal{M} \mid c_v \text{ defined on } [0, 1] \right\}\). Then \emph{exponential map of \(\mathcal{M} \) at \(p\)}, \(\exp_p \colon V_p \to \mathcal{M} \), is defined as \(v \mapsto c_v(1)\).
\end{definition}

\begin{theorem}
	The \hyperref[def:exponential-map]{exponential map} \(\exp _p\) maps a neighborhood of \(0\in T_p \mathcal{M} \) \hyperref[def:diffeomorphic]{diffeomorphically} onto a neighborhood of \(p\in \mathcal{M} \).
\end{theorem}

Consider \(\exp _p\colon B(0, \epsilon ) \subseteq T_p \mathcal{M} \to  \mathcal{M} \) \hyperref[def:diffeomorphic]{diffeomorphically} onto its image, we now introduce the coordinates around \(m\). Let \((e_1, \ldots , e_n)\) be the orthonormal basis of \(T_m \mathcal{M} \), and \((x_1, \ldots , x_n)\) be the associated local coordinates. Given \(p\in \mathcal{M} ^n\), \(0\in \mathbb{R} ^n\), we have
\[
	g_{ij}(p) = \delta_{ij},\quad \Gamma _{ij}^k(p) = 0,\quad g_{ij, k} = 0
\]
for all \(i, j, k\).

\begin{definition}[Normal coordinate]\label{def:normal-coordinate}

\end{definition}

\begin{note}
	The first derivative vanishes, so locally, the manifold looks Euclidean.
\end{note}

\begin{theorem}\label{thm:shortest-geodesic}
	For all \(p\in \mathcal{M} \), there exists \(\rho > 0\) such that the Riemannian polar coordinates may be introduced on \(B(p, \rho ) = \left\{ q\in \mathcal{M} \mid d(p, q) \leq \rho  \right\} \). For any such \(\rho \) and \(q\in \partial B(p, \rho )\), there exists a unique \hyperref[def:geodesic]{geodesic} of shortest length (\(=\rho \)) from \(p\) to \(q\) And in the polar coordinates, this \hyperref[def:geodesic]{geodesic} is given by the straight line \(x(t) = (t, \varphi _0)\), \(0 \leq t \leq \rho \), with \(q\) represented by coordinates \((\rho , \varphi _0)\), \(\varphi _0\in S^{d-1}\).
\end{theorem}
\begin{proof}
	Take an arbitrary curve from \(p\) to \(q\), namely \(c(t) = \big(r(t), \varphi (t)\big)\), \(0 \leq t \leq T\), which does not have to be entirely be contained in \(B(p, \rho )\). Let \(t_0\) be defined as
	\[
		t_0 \coloneqq \inf \left\{ t \leq T \mid d(x(t), p) \geq \rho  \right\}.
	\]
	Then \(t_0 \leq T\) such that \(\at{c}{[0, t_0]}{} \) lies entirely in \(B(p, \rho )\). We want to show that
	\begin{enumerate}[(a)]
		\item \(L\left( \at{c}{[0, t_0]}{} \right) \geq \rho \), and
		\item \(L\left( \at{c}{[0, t_0]}{} \right) = \rho \) only for a straight line in the polar coordinates,
	\end{enumerate}
	where
	\[
		L\left( \at{c}{[0 ,t_0]}{} \right) \coloneqq \int_{0}^{t_0} \sqrt{g_{ij}(c(t)) \dot{c}^i \dot{c}^j} \,\mathrm{d}t.
	\]

	Observe that \(g_{r \varphi } = 0\), with \(g_{\varphi \varphi }\) being positive definite, hence
	\[
		L\left( \at{c}{[0, t_0]}{} \right)
		\geq \int_{0}^{t_0} \sqrt{g_{r r}(c(t)) \dot{r} \dot{r}} \,\mathrm{d}t
		= \int_{0}^{t_0} \vert \dot{r} \vert \,\mathrm{d}t
		\geq \int_{0}^{t_0} \dot{r} \,\mathrm{d}t
		= r(t_0)
		= \rho,
	\]
	where we know that \(g_{r r} \equiv 1\).
\end{proof}

\begin{remark}[Compact manifold]
	For compact manifold, from \autoref{thm:shortest-geodesic}, we can prove that Riemannian polar coordinates can be introduced. Also, there exists \(\rho _0 > 0\) such that for any \(2\) points \(p, q\in \mathcal{M} \) with \(d(p, q) \leq \rho _0\) can be connected by minimizing \hyperref[def:geodesic]{geodesic}.
\end{remark}