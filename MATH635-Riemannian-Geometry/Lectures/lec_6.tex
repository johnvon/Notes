\lecture{6}{24 Jan. 13:00}{Geodesics and the Exponential Map}
Now, we draw some relations between \hyperref[def:length]{length} and \hyperref[def:energy]{energy} and see why starting from \hyperref[def:energy]{energy} makes sense.

\begin{proposition}
	For all \hyperref[def:curve]{curves} \(\gamma \colon [a, b] \to \mathcal{M} \),
	\[
		\mathcal{L} (\gamma )^2 \leq 2(b-a) E(\gamma )
	\]
	with equality if and only if \(\lVert \mathrm{d} \gamma / \mathrm{d} t \rVert \) is a constant.
\end{proposition}
\begin{proof}
	From \href{https://en.wikipedia.org/wiki/H%C3%B6lder%27s_inequality}{Hölder's inequality},
	\[
		\int_{a}^{b} \left\lVert \frac{\mathrm{d}\gamma }{\mathrm{d}t} \right\rVert \,\mathrm{d}t
		\leq (b-a)^{1 / 2} \left( \int_{a}^{b} \left\lVert \frac{\mathrm{d}\gamma }{\mathrm{d}t} \right\rVert ^2 \,\mathrm{d}t \right) ^{1 / 2}
	\]
	with equality if and only if \(\lVert \mathrm{d} \gamma / \mathrm{d} t \rVert\) is a constant.
\end{proof}

\begin{eg}
	Let
	\[
		\mathcal{L} (q, x) = \frac{1}{2} m \vert q \vert ^2 - V(x)
	\]
	with \(m > 0, q = \dot{x}\), the Euler-Lagrangian equations is given by \(m\ddot{x} (s) = F(x(s))\) for \(F\coloneqq -\mathrm{D} V\).
\end{eg}

Since regular curves can be parametrized by \hyperref[def:length]{arc length} with unit speed \(\lVert \mathrm{d} \gamma / \mathrm{d} t \rVert = \lVert \dot{\gamma } \rVert \equiv 1\), the following is natural.

\begin{lemma}
	Each \hyperref[def:geodesic]{geodesic} is parametrized proportionally to the \hyperref[def:length]{arc length}, i.e., \(\lVert \dot{\gamma } \rVert \) is a constant.
\end{lemma}
\begin{proof}
	For a solution \(x(t)\) of \(\ddot{x}^i(t) + \Gamma ^{i}_{jk}(x(t)) \dot{x}^j(t)\dot{x}^k(t) = 0\) (i.e., the \hyperref[def:geodesic]{geodesic}), we have
	\[
		\frac{\mathrm{d}}{\mathrm{d}t} \left\langle \dot{x}, \dot{x} \right\rangle
		= \frac{\mathrm{d}}{\mathrm{d}t} \left( g_{ij} (x(t)) \dot{x}^i(t)\dot{x}^j(t)\right)
		=0.
	\]
\end{proof}

\begin{remark}
	This is one of the advantages of working with the \hyperref[def:energy]{energy}rather than the \hyperref[def:length]{length}.
\end{remark}

Since the \hyperref[def:length]{length} and the \hyperref[def:energy]{energy} functionals are invariants under parameter changes, it's enough to look at \hyperref[def:curve]{curves} parametrized by arc \hyperref[def:length]{length}.

\begin{theorem}\label{thm:geodesic-existence-uniqueness}
	Let \(\mathcal{M} \) be a \hyperref[def:Riemannian-manifold]{Riemannian manifold}, \(p\in \mathcal{M} \) and \(v\in T_p \mathcal{M} \). Then there exists an \(\epsilon > 0\) and a unique \hyperref[def:geodesic]{geodesic} such that \(c\colon [0, \epsilon ] \to \mathcal{M} \) with \(c(0) = p\) and \(\dot{c}(0) = v\). In addition, \(c\) smoothly depend on \(p, v\).
\end{theorem}
\begin{proof}
	Since \autoref{eq:geodesic} is a system of second order ODE, by \href{https://en.wikipedia.org/wiki/Picard%E2%80%93Lindel%C3%B6f_theorem}{Picard-Lindelöf theorem}, we have local existence and uniqueness of the solution with prescribed initial values and derivative such that the solution depends smoothly on \(p, v\).
\end{proof}

If \(x(t)\) is the solution of \autoref{eq:geodesic}, then \(x(\lambda t)\) is also a solution for any constant \(\lambda \in \mathbb{R} \). Denote \hyperref[def:geodesic]{geodesic} from \autoref{thm:geodesic-existence-uniqueness} by \(c_v\), then
\[
	c_v(t) = c_{\lambda v}(t / \lambda )
\]
for \(\lambda > 0\), \(t\in [0, \epsilon ]\), and hence \(c_{\lambda v}\) defined on \([0, \epsilon / \lambda ]\).

\begin{remark}
	Since \(c_v\) depends smoothly on \(v\), the set \(\left\{ v\in T_p \mathcal{M} \mid \lVert v \rVert = 1 \right\} \) is compact, hence there exists \(\epsilon _0 > 0\) such that for \(\lVert v \rVert = 1\), \(c_v\) defined at least on \([0, \epsilon _0]\), implying that for all \(w\in T_p \mathcal{M} \) with \(\lVert w \rVert \leq \epsilon _0\), \(c_w\) is defined at least on \([0, 1]\).
\end{remark}

\subsection{Exponential Maps and Normal Coordinates}
The above discussion permits us to introduce the concept of the \hyperref[def:exponential-map]{exponential map} in the following manner.

\begin{definition}[Exponential map]\label{def:exponential-map}
	Let \((\mathcal{M} , g)\) be a \hyperref[def:Riemannian-manifold]{Riemannian manifold}, \(p\in \mathcal{M} \), and \(V_p \coloneqq \left\{ v\in T_p \mathcal{M} \mid c_v \text{ defined on } [0, 1] \right\}\). The \emph{exponential map of \(\mathcal{M} \) at \(p\)}, \(\exp_p \colon V_p \to \mathcal{M} \), is defined as \(v \mapsto c_v(1)\).
\end{definition}

Clearly, \(\exp \) is differentiable, and we shall utilize the restriction of \(\exp \) to an open subset of the \hyperref[def:tangent-space]{tangent space} \(T_q \mathcal{M} \), i.e., we define
\[
	\exp _p\colon B(0, \epsilon ) \subseteq T_p \mathcal{M} \to \mathcal{M} ,
\]
where \(B(0, \epsilon ) \) is an open ball with center at the origin \(0\) of \(T_p \mathcal{M} \) of radius \(\epsilon \). It's easy to see that \(\exp _p\) is differentiable and that \(\exp _p(0) = p\).

\begin{intuition}
	Geometrically, \(\exp _p(v)\) is a point of \(\mathcal{M} \) obtained by going out the \hyperref[def:length]{length} equal to \(\vert v \vert \), starting from \(p\), along a \hyperref[def:geodesic]{geodesic} which passes through \(p\) with velocity equal to \(v / \vert v \vert \).
\end{intuition}

\begin{proposition}
	The \hyperref[def:exponential-map]{exponential map} \(\exp _p\) maps a neighborhood of \(0\in T_p \mathcal{M} \) \hyperref[def:diffeomorphic]{diffeomorphically} onto a neighborhood of \(p\in \mathcal{M} \).
\end{proposition}
\begin{proof}
	We see that
	\[
		\mathrm{d} (\exp _p)_0(v)
		= \at{\frac{\mathrm{d}}{\mathrm{d}t} \exp _p(tv)}{t=0}{}
		= \at{\frac{\mathrm{d}}{\mathrm{d}t} c_{tv}(1)}{t=0}{}
		= \at{\frac{\mathrm{d}}{\mathrm{d}t} c_{v}(t)}{t=0}{}
		= v,
	\]
	i.e., \(\mathrm{d} (\exp _p)_0\) is the identity of \(T_q \mathcal{M} \). By the inverse function theorem, \(\exp _p\) is a local \hyperref[def:diffeomorphism]{diffeomorphism} on a neighborhood of \(0\).
\end{proof}

\begin{eg}
	Let \(\mathcal{M} = \mathbb{R} ^n\), then the \hyperref[def:exponential-map]{exponential map} is the identity.\footnote{With the usual identification of \(T_p \mathbb{R} ^n\) at \(p\) with \(\mathbb{R} ^n\).}
\end{eg}

\begin{eg}
	Let \(\mathcal{M} = S^2\).
	\begin{center}
		\incfig{circle-exponential-map}
	\end{center}
\end{eg}

Now we know that \(\exp _p\colon B(0, \epsilon ) \subseteq T_p \mathcal{M} \to \mathcal{M} \) maps \hyperref[def:diffeomorphic]{diffeomorphically} onto its image, we then define the following.

\begin{definition}[Normal coordinate]\label{def:normal-coordinate}
	Given an \hyperref[def:exponential-map]{exponential map} \(\exp _p\colon B(0, \epsilon ) \to \mathcal{M} \), let \((e_1, \dots , e_n)\) be the orthonormal basis of \(T_p \mathcal{M} \). Then the associated \hyperref[def:coordinate-chart]{local coordinates} are the \emph{normal coordinates}.
\end{definition}

In this case, given \(p\in \mathcal{M} ^n\), \(0\in \mathbb{R} ^n\), for all \(i, j, k\),\footnote{Note that this only holds at \(p\). We will come back to this when we formally introduce the \hyperref[def:linear-connection]{linear connection}.}
\[
	g_{ij}(p) = \delta_{ij},\quad \Gamma _{ij}^k(p) = 0,\quad g_{ij, k} = 0.
\]

\begin{intuition}
	The first derivative vanishes, so locally, the \hyperref[def:Riemannian-manifold]{manifold} looks Euclidean.
\end{intuition}

\begin{note}
	\cite{flaherty2013riemannian} introduces everything above using \(T\mathcal{M} \) instead of \(T_p \mathcal{M} \).
\end{note}

\section{Hopf-Rinow Theorem}
With all the tools we have developed, we now want to characterize the minimizing property of \hyperref[def:geodesic]{geodesics}.

\subsection{Riemannian Polar Coordinates}
A particular useful tool is the Riemannian polar coordinates, which is introduced as follows.

\begin{theorem}\label{thm:shortest-geodesic}
	For all \(p\in \mathcal{M} \), there exists \(\rho > 0\) such that the Riemannian polar coordinates may be introduced on \(B(p, \rho ) = \left\{ q\in \mathcal{M} \mid d(p, q) \leq \rho \right\} \). For any such \(\rho \) and \(q\in \partial B(p, \rho )\), there exists a unique \hyperref[def:geodesic]{geodesic} of shortest length (\(=\rho \)) from \(p\) to \(q\). In the polar coordinates, this \hyperref[def:geodesic]{geodesic} is given by the straight line \(x(t) = (t, \varphi _0)\), \(0 \leq t \leq \rho \), with \(q\) represented by coordinates \((\rho , \varphi _0)\), \(\varphi _0\in S^{d-1}\).
\end{theorem}
\begin{proof}
	Take an arbitrary \hyperref[def:curve]{curve} from \(p\) to \(q\), namely \(c(t) = \big(r(t), \varphi (t)\big)\), \(0 \leq t \leq T\), which does not have to be entirely be contained in \(B(p, \rho )\). Let \(t_0\) be defined as
	\[
		t_0 \coloneqq \inf \left\{ t \leq T \mid d(x(t), p) \geq \rho \right\}.
	\]
	Then \(t_0 \leq T\) such that \(\at{c}{[0, t_0]}{} \) lies entirely in \(B(p, \rho )\). We want to show that
	\begin{enumerate}[(a)]
		\item \(L\left( \at{c}{[0, t_0]}{} \right) \geq \rho \), and
		\item \(L\left( \at{c}{[0, t_0]}{} \right) = \rho \) only for a straight line in the polar coordinates,
	\end{enumerate}
	where
	\[
		L\left( \at{c}{[0 ,t_0]}{} \right) \coloneqq \int_{0}^{t_0} \sqrt{g_{ij}(c(t)) \dot{c}^i \dot{c}^j} \,\mathrm{d}t.
	\]
	Observe that \(g_{r \varphi } = 0\), with \(g_{\varphi \varphi }\) being positive definite and \(g_{r r} \equiv 1\), we have
	\[
		L\left( \at{c}{[0, t_0]}{} \right)
		\geq \int_{0}^{t_0} \sqrt{g_{r r}(c(t)) \dot{r} \dot{r}} \,\mathrm{d}t
		= \int_{0}^{t_0} \vert \dot{r} \vert \,\mathrm{d}t
		\geq \int_{0}^{t_0} \dot{r} \,\mathrm{d}t
		= r(t_0)
		= \rho.
	\]
\end{proof}

\begin{corollary}\label{col:shortest-geodesic}
	Let \(\mathcal{M} \) be a compact \hyperref[def:Riemannian-manifold]{Riemannian manifold}. Then there exists \(\rho _0 > 0\) such that
	\begin{enumerate}[(a)]
		\item for any \(p\in \mathcal{M} \), Riemannian polar coordinates may be introduced on \(B(p, \rho _0)\);
		\item for any \(p, q\in \mathcal{M} \) with \(d(p, q) \leq \rho _0\), they can be connected by precisely one \hyperref[def:geodesic]{geodesic} or shortest length which depends continuously on \(p\) and \(q\).
	\end{enumerate}
\end{corollary}