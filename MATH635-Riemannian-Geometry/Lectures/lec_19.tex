\lecture{19}{14 Mar. 13:00}{Length-Minimizing Geodesics and Conjugacy}
Now, let's prove \autoref{thm:lec18-2}.
\begin{proof}[Proof of \autoref{thm:lec18-2}]
	We prove them one by one.
	\begin{enumerate}[(a)]
		\item First steps: last lecture.\todo{Add this} Now, cover \(\left\{ tV \mid 0 \leq t \leq 1 \right\} \) by finitely many such neighborhoods \(\Omega _i\), \(i = 1, \dots , k\), and let \(U_i = \exp _p \Omega _i\). Assume that \(tV\in \Omega _i\), for \(t_{i-1} \leq t \leq t_i\) (with \(t_0 = 0, t_k = 1\)). Let \(\epsilon > 0\) sufficiently small. Then, for all \hyperref[def:curve]{curve} \(g\colon [0, 1] \to \mathcal{M} \) satisfying the assumption, \(g([t_{i-1}, t_i]) \subseteq U_i\).

		      \begin{claim}
			      For all \(g\) satisfying \(g([t_{i-1}, t_i]) \subseteq U_i\), there exists a \hyperref[def:curve]{curve} \(\gamma \subseteq T_p \mathcal{M} \) such that
			      \[
				      \exp _p \gamma = g
			      \]
			      with \(\gamma (0) = 0 , \gamma (1) g V\).
		      \end{claim}
		      \begin{explanation}
			      Put \(\gamma (t) = \left( \at{\exp _p}{\Omega _i}{} \right)^{-1} (g(t)) \) for \(t_{i-1} \leq t \leq t_i\), so \(\gamma \) satisfies \autoref{col:lec18}.
		      \end{explanation}
		\item  Without loss of generality, let \(a = 0, b= 1\). Let \(X\) be a non-trivial \hyperref[def:Jacobi-field]{Jacobi field} along \(c\) with \(X(0) = 0 = X(\tau )\). We have \(\dot{X} (\tau ) \neq 0\), as otherwise \(X \equiv 0\) by the uniqueness. Let \(Z(t)\) be an arbitrary \hyperref[def:vector-field-along-curve]{vector field \(X\) along \(c\)} with \(Z(0) = 0 = Z(1)\), \(Z(\tau ) = - \dot{X} (\tau )\). Let \(\eta > 0\), set
		      \[
			      Y_\eta (t) = \begin{dcases}
				      Y_\eta ^1(t) = X(t) + \eta Z(t), & \text{ if } 0 \leq t \leq \tau ; \\
				      Y_\eta ^2(t) = \eta Z(t),        & \text{ if } \tau \leq t \leq 1 ,
			      \end{dcases}
		      \]
		      and we let \(Z^1 \coloneqq \at{Z}{[0, \tau ]}{} , Z^2 \coloneqq \at{Z}{[\tau , 1]}{} \). Now, since
		      \[
			      I(Y_\eta ^1, Y_\eta ^1)
			      = \left\langle \dot{X} (\tau ), 2 \eta Z(\tau ) \right\rangle + \eta ^2 I(Z^1, Z^1) - 2 \eta \lVert \dot{X}(\tau)\rVert^2 + \eta ^2 I(Z^1, Z^1),
		      \]
		      with
		      \[
			      I(Y_\eta ^2, Y_\eta ^2) = \eta ^2 I(Z^2, Z^2),
		      \]
		      with
		      \[
			      I(Y_\eta , Y_\eta ) = I(Y_\eta ^1, Y_\eta ^1) + I(Y_\eta ^2, Y_\eta ^2) = -2 \eta \lVert \dot{X} (\tau ) \rVert ^2 + \eta ^2 I(Z, Z)
		      \]
		      for sufficiently small \(\eta > 0\). Now, consider the variation
		      \[
			      c(t, s) \coloneqq \exp _{c(t)} s Y_\eta (t),
		      \]
		      we have \(L^{\prime} (0) = 0\)\footnote{Note that \(L(s) \coloneqq L(c_s)\), \(L(0) = L(c)\).} and
		      \[
			      L^{\prime\prime} (0) = I(Y_\eta , Y_\eta ) < 0.
		      \]
		      By the Taylor theorem, this is a minimum, i.e., \(L(c_s) < L(c)\).
	\end{enumerate}
\end{proof}

\begin{definition}[Order]\label{def:order-of-conjugacy}
	The \emph{order (or multiplicity) of conjugacy} is the dimension of space of \hyperref[def:Jacobi-field]{Jacobi fields} vanishing at two \hyperref[def:conjugate-point]{conjugate points}.
\end{definition}

Given \(\dim \mathcal{M} = n\), by the existence and uniqueness theorem for \hyperref[def:Jacobi-field]{Jacobi fields}, we see the following examples.
\begin{eg}
	There is an \(n\)-dimensional space of \hyperref[def:Jacobi-field]{Jacobi fields} vanishing at \(p\in \mathcal{M} \).
\end{eg}

\begin{eg}
	There is an at most \((n-1)\)-dimensional space of \hyperref[def:Jacobi-field]{Jacobi fields} vanishing at \(p, q\in \mathcal{M} \), as tangential \hyperref[def:Jacobi-field]{Jacobi fields} vanishes at most at one point.
\end{eg}

We now characterize the \hyperref[def:conjugate-point]{conjugate points}: they are precisely the images of singularities of the \hyperref[def:exponential-map]{exponential map}.

\begin{proposition}
	Let \(p\in \mathcal{M} \), \(V\in T_p \mathcal{M} \), \(q = \exp V\). Then \(\exp _p\) is a local \hyperref[def:diffeomorphism]{diffeomorphism} in a neighborhood of \(V\) if and only if \(q\) not \hyperref[def:conjugate-point]{conjugates} to \(p\) along \hyperref[def:geodesic]{geodesic} \(\gamma (t) = \exp _p tV\), \(t\in [0, 1]\).
\end{proposition}

Let \(c\colon [a, b] \to \mathcal{M} \) be a \hyperref[def:curve]{curve}, and let \(\nu _c\) be the space of \hyperref[def:vector-field-along-curve]{vector field \(X\) along \(c\)}, i.e., \(\nu _c = \Gamma (c^{\ast} T \mathcal{M} )\). Let \(\mathring{\nu }_c\) be the space of \hyperref[def:vector-field-along-curve]{vector field \(X\) along \(c\)} with \(V(a) = V(b) = 0\).

\begin{lemma}\label{lma:lec19}
	Let \(c\colon [a, b] \to \mathcal{M} \) be a \hyperref[def:geodesic]{geodesic}. Then there is no pair of \hyperref[def:conjugate-point]{conjugate points} along \(c\) if and only if the \hyperref[def:index-form]{index form} \(I\) of \(c\) is strictly positive definite on \(\mathring{\nu}_c\)
\end{lemma}
\begin{proof}\let\qed\relax
	Assume that \(c\) has no \hyperref[def:conjugate-point]{conjugate points}, then \autoref{thm:lec18-2} implies that
	\(I(X, X) \geq 0\) for all \(X \in \mathring{\nu }_c\) because otherwise \(c(t, s)\coloneqq \exp _{c(t)} s X(t)\) would be locally length-decreasing.

	If \(I(Y, Y) = 0\) for some \(Y\in \mathring{\nu }_c\), then by \(I(X, X ) \geq 0\), we have
	\[
		0 \leq I(Y - \lambda Z, Y - \lambda Z) = 0 - 2\lambda I(Y, Z) + \lambda ^2 I(Z, Z)
	\]
	for all \(Z\in \mathring{\nu }_c\) and \(\lambda \in \mathbb{R} \). We see that this inequality hold only if \(I(Y, Z) = 0\) for all \(Z\in \mathring{\nu }_c\)., which implies \(Y\) is a \hyperref[def:Jacobi-field]{Jacobi field} form \autoref{prop:Jacobi-field}. As no \hyperref[def:conjugate-point]{conjugate points} along \(c\), \(Y = 0\), i.e., \(I\) is strictly positive definite.
\end{proof}