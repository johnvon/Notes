\lecture{13}{16 Feb. 14:30}{}
Let \(X = X^i \partial /\partial x^i\) be a \hyperref[def:vector-field]{vector field}. Then
\[
	(\psi _t)_{\ast} \frac{\partial }{\partial x^i} (\psi _t(x))
	= \frac{\partial \psi _t^k}{\partial x^i} \frac{\partial }{\partial x^k}.
\]
For \(\mathcal{M} = \mathcal{N} \), X and \(\varphi (x)\) in the same coordinate neighborhood, we have
\[
	\frac{\partial }{\partial \varphi ^k} = \frac{\partial }{\partial x^k}.
\]

On the other hand, let \(\omega g \omega _i \mathrm{d} x^i\) be a \(1\)-form, then we have
\[
	(\psi ^{\ast} _t)(\omega )(x) = \omega (\psi _t(x))\frac{\partial \psi _t^i}{\partial x^k} \mathrm{d} x^k,
\]
which is the curve in \(T_x^{\ast} \mathcal{M} \). For \(\varphi \colon \mathcal{M} \to \mathcal{N} \), with the \(1\)-form \(\omega = \omega_i \mathrm{d} x^i\) on \(\mathcal{N} \),
\[
	\varphi ^{\ast} \omega = \omega _i (\varphi (x))\frac{\partial z^i}{\partial x^k} \mathrm{d} x^k.
\]
Let \(\varphi \colon \mathcal{M} \to \mathcal{N} \) be a \hyperref[def:diffeomorphism]{diffeomorphism}, \(Y\) be a \hyperref[def:vector-field]{vector field} on \(\mathcal{N} \). Set
\[
	\varphi ^{\ast} Y\coloneqq (\varphi ^{-1} )_\ast Y.
\]
\begin{remark}
	\(\varphi ^{\ast} \) can be defined in an analogous way in terms of contravariant tensors.
\end{remark}

In particular, let \(X\) be a \hyperref[def:vector-field]{vector field}, and a local \(1\)-parameter group \((\psi _t)_{t\in I}\),
\[
	(\psi _t ^{\ast} X)= (\psi _t)_\ast X.
\]

Let \(x\in \mathcal{M} \), \(X\in T_x \mathcal{M} \), and \(\mathrm{d} f_x \colon T_x \mathcal{M} \to T_{f(x)} \mathcal{N} \). \(\mathrm{d} f_x X\) or \((f_\ast)_x X\) is called the pushforward of \(x\) by \(f\).

Let \(\omega \in T^{\ast} \mathcal{N} \), a \(1\)-form on \(\mathcal{N} \). Then
\[
	(f^{\ast} \omega )_x X = \omega _{f(x)} (\mathrm{d} f_x X)
\]
is called pullback of \(\omega \) by \(f\).

% https://q.uiver.app/?q=WzAsNCxbMCwwLCJcXG1hdGhjYWx7TX0iXSxbMSwwLCJcXG1hdGhjYWx7Tn0iXSxbMCwxLCJUXFxtYXRoY2Fse019Il0sWzEsMSwiVFxcbWF0aGNhbHtOfSJdLFsyLDMsIlxcbWF0aHJte2R9ZiJdLFswLDEsImYiXSxbMiwwLCJcXHBpX3tcXG1hdGhjYWx7TX19Il0sWzMsMSwiXFxwaV9cXG1hdGhjYWx7Tn0iLDJdXQ==
\[
	\begin{tikzcd}
		{\mathcal{M}} & {\mathcal{N}} \\
		{T\mathcal{M}} & {T\mathcal{N}}
		\arrow["{\mathrm{d}f}", from=2-1, to=2-2]
		\arrow["f", from=1-1, to=1-2]
		\arrow["{\pi_{\mathcal{M}}}", from=2-1, to=1-1]
		\arrow["{\pi_\mathcal{N}}"', from=2-2, to=1-2]
	\end{tikzcd}
\]

\begin{definition}[Sectional curvature]\label{def:sectional-curvature}
	The \emph{sectional curvature} of the plane spanned by the (linearly independent) \hyperref[def:tangent-vector]{tangent vectors} \(X = X^i \partial / \partial x^i, Y= Y^i \partial / \partial x^i\in T_x \mathcal{M} \) of the \hyperref[def:Riemannian-manifold]{Riemannian manifold} \(\mathcal{M} \) is
	\[
		K(X \wedge Y) = \frac{g(R(X, Y) Y, X)}{\vert X\wedge Y \vert^2 }
	\]
	where \(\vert X \wedge Y \vert ^2 = g(X, X) g(Y, Y) - g(X, Y)^2\).
\end{definition}

\begin{remark}
	\hyperref[def:sectional-curvature]{Sectional curvature} determines the whole \hyperref[def:Riemannian-curvature-tensor]{curvature tensor}.
\end{remark}
\begin{explanation}
	Given \(g(R(X, Y)Z, W)\), we can express this entirely by \(K\).
\end{explanation}

\begin{remark}[Gauss curvature]
	For \(\dim \mathcal{M} = 2\),
	\[
		R_{ijk\ell} = K(g_{ik} g_{j \ell } - g_{ij} g_{k \ell })
	\]
	where here, \(K\) is called the \emph{Gauss curvature}.	Since \(T_x \mathcal{M} \) contains only one plane, i.e., \(T_x \mathcal{M} \) itself.
\end{remark}

\begin{definition}[Space form]\label{def:space-form}
	Let \((\mathcal{M} , g)\) be a \hyperref[def:Riemannian-manifold]{Riemannian manifold}. \(\mathcal{M} \) is called a \emph{space of constant sectional curvature} or a \emph{space form} if \(K(X\wedge Y) \) is a constant for all linearly independent \hyperref[def:tangent-vector]{tangent vectors} \(X, Y\in T_x \mathcal{M} \) and for all \(x\in \mathcal{M} \).

	\begin{definition}[Spherical]\label{def:space-form-spherical}
		A \hyperref[def:space-form]{space form} is called \emph{spherical} if \(K > 0\).
	\end{definition}

	\begin{definition}[Flat]\label{def:space-form-flat}
		A \hyperref[def:space-form]{space form} is called \emph{flat} if \(K = 0\).
	\end{definition}

	\begin{definition}[Hyperbolic]\label{def:space-form-hyperbolic}
		A \hyperref[def:space-form]{space form} is called \emph{hyperbolic} if \(K > 0\).
	\end{definition}
\end{definition}

\begin{definition}[Einstein manifold]\label{def:Einstein-manifold}
	\(\mathcal{M} \) is called an \emph{Einstein manifold} if \(R_{ik} = c g_{ik} \) for \(c\) being a constant which does not depend on the choice of \hyperref[def:coordinate-chart]{local coordinates}.
\end{definition}

\begin{center}
	\incfig{space-form}
\end{center}

\begin{remark}
	Every manifold with constant sectional curvature is an Einstein manifold.
\end{remark}

\begin{definition}[Flat]\label{def:connection-flat}
	A \hyperref[def:linear-connection]{connection} \(\nabla \) on \(T \mathcal{M} \) is called \emph{flat} if each point in \(\mathcal{M} \) has a neighborhood \(U\) with \hyperref[def:coordinate-chart]{local coordinates} for which all the coordinate \hyperref[def:vector-field]{vector fields} \(\partial / \partial x^i\) are \hyperref[def:parallel]{parallel}, i.e., \(\nabla \partial / \partial x^i = 0\).
\end{definition}

\begin{theorem}
	A \hyperref[def:linear-connection]{connection} \(\nabla \) on \(T \mathcal{M} \) is \hyperref[def:connection-flat]{flat} if and only if its \hyperref[def:Riemannian-curvature-tensor]{curvature} and \hyperref[def:torsion-tensor]{torsion} vanish identically.
\end{theorem}
\begin{proof}
	\hyperref[def:connection-flat]{Flat} \hyperref[def:linear-connection]{connection} implies \(\nabla _{\frac{\partial }{\partial x^i} } \frac{\partial }{\partial x^j} = 0\), hence all \(\Gamma ^k _{ij} = 0\), so \(T, R\) vanish. Conversely, find the \hyperref[def:coordinate-chart]{local coordinates} such that \(\nabla _{\frac{\partial }{\partial x^i} } \frac{\partial }{\partial x^j} = 0\) for all \(i, j\). Then, by computation with Frobenius theorem, we're done.
\end{proof}


\begin{theorem}[Schur theorem]\label{thm:Schur}
	Let \((\mathcal{M} , g)\) be a \hyperref[def:Riemannian-manifold]{Riemannian manifold} with \(\dim \mathcal{M} \geq 3\).
	\begin{enumerate}[(a)]
		\item If the \hyperref[def:sectional-curvature]{sectional curvature} of \(\mathcal{M} \) is constant at each point, i.e., \(K(X\wedge Y) = f(x)\) for \(X, Y\in T_x \mathcal{M} \). Then \(f(x)\) is a constant on \(\mathcal{M} \), so \(\mathcal{M} \) is a \hyperref[def:space-form]{space form}.
		\item If the \hyperref[def:Ricci-curvature-tensor]{Ricci curvature} is a constant at each point, i.e., \(R_{ik} = c(x) g_{ik}\), then \(c(x)\) is a constant and \(\mathcal{M} \) is an \hyperref[def:Einstein-manifold]{Einstein manifold}.
	\end{enumerate}
\end{theorem}

\begin{remark}
	\hyperref[thm:Schur]{Schur theorem} says that the isotropy of a \hyperref[def:Riemannian-manifold]{Riemannian manifold}, i.e., the property that at each point, all directions are geometrically indistinguishable, implies the homogeneity, i.e., all points are geometrically indistinguishable.
\end{remark}

\begin{note}
	A point-wise property implies a global one!
\end{note}

\subsection{Covering Maps}
\begin{definition}[Covering map]\label{def:covering-map}
	Let \(\mathcal{M} , \widetilde{\mathcal{M}} \) be \(2\) manifolds.a map \(p\colon \widetilde{\mathcal{M}} \to \mathcal{M} \) is a \emph{covering map} if
	\begin{enumerate}[(a)]
		\item \(p\) is smooth and surjective, and
		\item for all \(m\in \mathcal{M} \), there exists a neighborhood \(U\) at \(m\) in \(\mathcal{M} \) with \(p ^{-1} (U) = \coprod_{i\in I} U_i\) with \(p\colon U_i \to U\) being a \hyperref[def:diffeomorphism]{diffeomorphism} and \(U_i\) are disjoint open subsets of \(\widetilde{\mathcal{M}} \).
	\end{enumerate}
\end{definition}

\begin{notation}[Covering space]\label{not:covering-space}
	\(\widetilde{\mathcal{M}} \) in \autoref{def:covering-map} is called the \emph{covering space}.
\end{notation}

\begin{notation}[Universal covering space]\label{not:universal-covering-space}
	A \hyperref[not:covering-space]{covering space} is \emph{universal}	if it's simply connected.
\end{notation}

\begin{definition}[Riemannian covering map]\label{def:Riemannian-covering-map}
	Let \((\mathcal{M} , g), (\mathcal{N}, h )\) be \(2\) \hyperref[def:Riemannian-manifold]{Riemannian manifolds}. A map \(p\colon \mathcal{N} \to \mathcal{M} \) is a \emph{Riemannian covering map} if \(p\) is a smooth \hyperref[def:covering-map]{covering map} and is a local isometry.
\end{definition}

\begin{proposition}
	Let \(p\colon \mathcal{N} \to \mathcal{M} \) be a smooth \hyperref[def:covering-map]{covering map}. For every \hyperref[def:Riemannian-metric]{Riemannian metric} \(g\) on \(\mathcal{M} \), there exists a unique \hyperref[def:Riemannian-metric]{Riemannian metric} \(h\) on \(\mathcal{N} \) such that \(p\) is a \hyperref[def:Riemannian-covering-map]{Riemannian covering map}.
\end{proposition}

\begin{note}
	The converse of the above is generally not true.
\end{note}

\begin{eg}
	Every space \hyperref[def:covering-map]{covers} itself trivially.
\end{eg}

\begin{eg}
	\(\mathbb{R} \) is the \hyperref[not:universal-covering-space]{universal covering space} of \(S^1\).
\end{eg}

\begin{eg}
	\(U(n)\) has \hyperref[not:universal-covering-space]{universal covers} \(U(n) \times \mathbb{R} \).
\end{eg}

\begin{eg}
	\(S^n\) is a double \hyperref[not:covering-space]{cover} for \(P_n(\mathbb{R} )\) and is \hyperref[not:universal-covering-space]{universal} for \(n > 1\).
\end{eg}

\begin{proposition}
	Let \((\mathcal{N}, h )\) be a \hyperref[def:Riemannian-manifold]{Riemannian manifold}. Let \(G\) be a free and proper group of isometries of \((\mathcal{N} , h)\). Then there exists a unique \hyperref[def:Riemannian-manifold]{Riemannian manifold} \((\mathcal{M} ,g)\) on the quotient manifold \(\mathcal{M} = \quotient{\mathcal{N} }{G} \) such that the connected projection \(p\colon \mathcal{N}  \to  \mathcal{M} \) is a \hyperref[def:Riemannian-covering-map]{Riemannian covering map}.
\end{proposition}