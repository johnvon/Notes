\lecture{13}{16 Feb. 13:00}{Sectional Curvatures and Space Forms}
Let \(X = X^i \partial /\partial x^i\) be a \hyperref[def:vector-field]{vector field}. Then consider \((\psi _t)_{\ast}X(\psi _t (X))\) to get a \hyperref[def:curve]{curve} \(X_t\) in \(T_x \mathcal{M} \) for \(t\in I\). By differentiate that curve, i.e.,
\[
	(\psi _t)_{\ast} \frac{\partial }{\partial x^i} (\psi _t(x))
	= \frac{\partial \psi _t^k}{\partial x^i} \frac{\partial }{\partial x^k}.
\]

\begin{note}
	For \(\varphi \colon \mathcal{M} \to \mathcal{N} \coloneqq \mathcal{M} \) and \(X\) and \(\varphi (x)\) are in the same \hyperref[def:local-coordinate]{coordinate neighborhood},
	\[
		\varphi _{\ast} \frac{\partial }{\partial x^i} = \frac{\partial \varphi ^k}{\partial x^i} \frac{\partial }{\partial \varphi ^k}
	\]
	since \(\frac{\partial }{\partial \varphi ^k} = \frac{\partial }{\partial x^k}\).
\end{note}

On the other hand, let \(\omega = \omega _i \mathrm{d} x^i\) be a \(1\)-form, then we have
\[
	(\psi ^{\ast} _t)(\omega )(x) = \omega _i (\psi _t(x))\frac{\partial \psi _t^i}{\partial x^k} \mathrm{d} x^k,
\]
which is a \hyperref[def:curve]{curve} in \(T_x^{\ast} \mathcal{M} \).

\begin{note}
	For \(\varphi \colon \mathcal{M} \to \mathcal{N} \) (\(\varphi \) need not be a \hyperref[def:diffeomorphism]{diffeomorphism}) with for the \(1\)-form \(\omega = \omega_i \mathrm{d} x^i\) on \(\mathcal{N} \),
	\[
		\varphi ^{\ast} \omega = \omega _i (\varphi (x))\frac{\partial z^i}{\partial x^k} \mathrm{d} x^k.
	\]
\end{note}

Let \(\varphi \colon \mathcal{M} \to \mathcal{N} \) be a \hyperref[def:diffeomorphism]{diffeomorphism}, \(Y\) be a \hyperref[def:vector-field]{vector field} on \(\mathcal{N} \). Then, set
\[
	\varphi ^{\ast} Y\coloneqq (\varphi ^{-1} )_\ast Y,
\]
and for other \hyperref[def:tensor]{contravariant tensors}, \(\varphi ^{\ast} \) can be defined in an analogous way.

\begin{eg}
	For a \hyperref[def:vector-field]{vector field} \(X\) and a \hyperref[def:local-1-parameter-group]{local \(1\)-parameter group} \((\psi _t)_{t\in I}\), it is \((\psi _t ^{\ast} X)= (\psi _t)_\ast X\).
\end{eg}

\section{Sectional Curvatures}
Beyond \hyperref[def:Riemannian-curvature]{Riemannian curvature} and other ``averaging'' variations of which, the following one is in particular interesting and is the one considered by Riemann.

\begin{definition}[Sectional curvature]\label{def:sectional-curvature}
	The \emph{sectional curvature} of the plane \(\Sigma \) spanned by the (linearly independent) \hyperref[def:tangent-vector]{tangent vectors} \(X = X^i \frac{\partial }{\partial x^i} , Y= Y^i \frac{\partial }{\partial x^i} \in T_x \mathcal{M} \) of a \hyperref[def:Riemannian-manifold]{Riemannian manifold} \((\mathcal{M}, g)\) is
	\[
		K(\sigma ) \coloneqq K(X \wedge Y) = \frac{g(R(X, Y) Y, X)}{\vert X\wedge Y \vert^2 }
	\]
	where \(\vert X \wedge Y \vert ^2 = g(X, X) g(Y, Y) - g(X, Y)^2\).\footnote{Given a vector space \(V\) and \(x, y\in V\), \(\vert x \wedge y \vert \coloneqq \sqrt{\vert x \vert ^2 \vert y \vert ^{2} - \langle x, y \rangle ^2} \) represents the area of the two-dimensional parallelogram spanned by \(x, y\).}
\end{definition}

\begin{note}
	\autoref{def:sectional-curvature} is well-defined since \(K(\sigma )\) is invariant under different bases of \(\sigma \).
\end{note}

\begin{remark}
	\hyperref[def:sectional-curvature]{Sectional curvature} determines the whole \hyperref[def:Riemannian-curvature]{Riemannian curvature}.
\end{remark}
\begin{explanation}
	Given \(g(R(X, Y)Z, W)\), we can express this entirely by \(K\)~\cite[Lemma 3.3]{flaherty2013riemannian}.
\end{explanation}

\begin{remark}[Gauss curvature]\label{rmk:Gauss-curvature}
	For \(\dim \mathcal{M} = 2\), \(R_{ijk\ell} = K(g_{ik} g_{j \ell } - g_{ij} g_{k \ell })\) since \(T_x \mathcal{M} \) contains only one plane, i.e., \(T_x \mathcal{M} \) itself. In this case, \(K\) is called the \emph{Gauss curvature}.
\end{remark}

In particular, the \hyperref[def:space-form]{space form} considers the space with constant \hyperref[def:sectional-curvature]{sectional curvature}.

\begin{definition}[Space form]\label{def:space-form}
	A \hyperref[def:Riemannian-manifold]{Riemannian manifold} \((\mathcal{M} , g)\) is a \emph{space form} if \(K(X\wedge Y) \) is a constant for all linearly independent \hyperref[def:tangent-vector]{tangent vectors} \(X, Y\in T_p \mathcal{M} \) for all \(p\in \mathcal{M} \).

	\begin{definition}[Spherical]\label{def:space-form-spherical}
		A \hyperref[def:space-form]{space form} is called \emph{spherical} if \(K > 0\).
	\end{definition}

	\begin{definition}[Flat]\label{def:space-form-flat}
		A \hyperref[def:space-form]{space form} is called \emph{flat} if \(K = 0\).
	\end{definition}

	\begin{definition}[Hyperbolic]\label{def:space-form-hyperbolic}
		A \hyperref[def:space-form]{space form} is called \emph{hyperbolic} if \(K > 0\).
	\end{definition}
\end{definition}

Generalize \autoref{def:space-form} a bit, we have the so-called \hyperref[def:Einstein-manifold]{Einstein manifolds}.

\begin{definition}[Einstein manifold]\label{def:Einstein-manifold}
	A \hyperref[def:Riemannian-manifold]{Riemannian manifold} \((\mathcal{M} , g)\) is called an \emph{Einstein manifold} if \(R_{ik} = c g_{ik} \) for a constant \(c\).\footnote{Which does not depend on the choice of \hyperref[def:coordinate-chart]{local coordinates}.}
\end{definition}

\begin{remark}
	Every \hyperref[def:space-form]{space form} is an \hyperref[def:Einstein-manifold]{Einstein manifold}.
\end{remark}

\begin{eg}
	\(\mathbb{R} ^n\) is \hyperref[def:space-form-flat]{flat}, \(S^n\) is \hyperref[def:space-form-spherical]{spherical}, and \(\mathbb{H} ^n\) is \hyperref[def:space-form-hyperbolic]{hyperbolic}. And all are \hyperref[def:Einstein-manifold]{Einstein manifolds}.
	\begin{center}
		\incfig{space-form}
	\end{center}
\end{eg}

\begin{definition}[Flat]\label{def:connection-flat}
	A \hyperref[def:linear-connection]{connection} \(\nabla \) on \(T \mathcal{M} \) is \emph{flat} if each point in \(\mathcal{M} \) has a neighborhood \(U\) with \hyperref[def:coordinate-chart]{local coordinates} for which all the coordinate \hyperref[def:vector-field]{vector fields} \(\partial / \partial x^i\) are \hyperref[def:parallel]{parallel}, i.e., \(\nabla \partial / \partial x^i = 0\).
\end{definition}

\begin{theorem}
	A \hyperref[def:linear-connection]{connection} \(\nabla \) on \(T \mathcal{M} \) is \hyperref[def:connection-flat]{flat} if and only if its \hyperref[def:Riemannian-curvature]{curvature} and \hyperref[def:torsion]{torsion} vanish identically.
\end{theorem}
\begin{proof}
	\hyperref[def:connection-flat]{Flat} \hyperref[def:linear-connection]{connection} implies \(\nabla _{\frac{\partial }{\partial x^i} } \frac{\partial }{\partial x^j} = 0\), hence all \(\Gamma ^k _{ij} = 0\), so \(T, R\) vanish. Conversely, find the \hyperref[def:coordinate-chart]{local coordinates} such that \(\nabla _{\frac{\partial }{\partial x^i} } \frac{\partial }{\partial x^j} = 0\) for all \(i, j\) and use \href{https://en.wikipedia.org/wiki/Frobenius_theorem_(differential_topology)}{Frobenius theorem}.
\end{proof}

\begin{eg}
	The following are \hyperref[def:connection-flat]{flat} \hyperref[def:smooth-manifold]{manifolds} with their usual shape, i.e., \hyperref[def:linear-connection]{connections}.
	\begin{multicols}{2}
		\begin{itemize}
			\item \(\mathbb{R} ^n\).
			\item Torus \(T^2\).
			\item Every \(1\)-dimensional \hyperref[def:Riemannian-manifold]{Riemannian manifold}.
			\item Products of \hyperref[def:connection-flat]{flat} \hyperref[def:smooth-manifold]{manifolds}.
			\item Tori.
		\end{itemize}
	\end{multicols}
\end{eg}

\begin{theorem}[Schur theorem]\label{thm:Schur}
	Let \((\mathcal{M} , g)\) be a \hyperref[def:Riemannian-manifold]{Riemannian manifold} with \(\dim \mathcal{M} \geq 3\).
	\begin{enumerate}[(a)]
		\item If the \hyperref[def:sectional-curvature]{sectional curvature} of \(\mathcal{M} \) is constant at each point, i.e., \(K(X\wedge Y) = f(x)\) for \(X, Y\in T_x \mathcal{M} \), then \(f(x)\) is a constant on \(\mathcal{M} \), hence \(\mathcal{M} \) is a \hyperref[def:space-form]{space form}.
		\item If the \hyperref[def:Ricci-curvature]{Ricci curvature} is a constant at each point, i.e., \(R_{ik} = c(x) g_{ik}\), then \(c(x)\) is a constant, hence \(\mathcal{M} \) is an \hyperref[def:Einstein-manifold]{Einstein manifold}.
	\end{enumerate}
\end{theorem}

\begin{remark}
	\hyperref[thm:Schur]{Schur theorem} says that the isotropy\footnote{I.e., the property that at each point, all directions are geometrically indistinguishable.} of a \hyperref[def:Riemannian-manifold]{Riemannian manifold} implies the homogeneity.\footnote{I.e., all points are geometrically indistinguishable.} Hence, a point-wise property implies a global one!
\end{remark}

\section{More on Covariant Derivatives}
To end this chapter, we revisit \hyperref[def:covariant-derivative]{covariant derivative}. But this time, we generalize it from \hyperref[def:vector-field]{vector field} to \hyperref[def:tensor-field]{tensor field}, i.e., we will show that it's also possible to covariantly differentiate \hyperref[def:tensor]{tensors}. The motivation is that given a \(1\)-form \(\omega \), and \hyperref[def:vector-field]{vector fields} \(X, Y\), we have
\[
	X(\omega (Y)) = (\nabla _X \omega )(Y) + \omega (\nabla _X Y),
\]
and for arbitrary \hyperref[def:tensor]{tensors} \(S, T\), we similarly have
\[
	\nabla _X (S \otimes T) = \nabla _X S \otimes T + S \otimes \nabla _X T.
\]
Consider the following.\footnote{\autoref{def:tensor-covariant-derivative} is natural by considering a certain \emph{frame}~\cite[\defaultS 4.5]{flaherty2013riemannian}.}

\begin{definition}[Covariant diffential]\label{def:tensor-covariant-differential}
	Let \(T\) be a \hyperref[def:tensor]{\((0,s)\)-tensor}. The \emph{covariant differential} \(\nabla T\) of \(T\) is a \hyperref[def:tensor]{\((0, s+1)\)-tensor} given by
	\[
		\nabla T(Y_1, \dots , Y_s, Z)
		= Z(T(Y_1, \dots , Y_s)) - T(\nabla _Z Y_1, \dots , Y_s) - \dots - T(Y_1, \dots , Y_{s-1}, \nabla _Z Y_s).
	\]
\end{definition}

\begin{definition}[Covariant derivative]\label{def:tensor-covariant-derivative}
	For each \(Z\in \Gamma (T \mathcal{M} )\), the \emph{covariant derivative} \(\nabla _Z T\) of \(T\) relative to \(Z\) is a \hyperref[def:tensor]{\((0, s)\)-tensor} given by
	\[
		\nabla _Z T(Y_1, \dots , Y_s) = \nabla T (Y_1, \dots , Y_s , Z).
	\]
\end{definition}

We primarily focus on covariant \hyperref[def:tensor]{tensor}, however, we also have the following.

\begin{remark}
	For \(T\) a \hyperref[def:tensor]{\((p, q)\)-tensor},
	\[
		\begin{split}
			(\nabla _Y T) ( \alpha _1, \dots , \alpha _q, X_1, \dots , X_p)
			&= Y(T(\alpha _1, \dots , \alpha _q, X_1, \dots , X_p))\\
			&\qquad - \sum_{i=1}^{q} T(\alpha _1, \dots , \nabla _Y \alpha _i, \dots , \alpha _q , X_1, \dots , X_p)\\
			&\qquad\qquad - \sum_{i=1}^{p} T(\alpha _1, \dots , \alpha _q, X_1, \dots , \nabla _Y X_i, \dots , X_p).
		\end{split}
	\]
\end{remark}

\begin{eg}
	Consider the \hyperref[def:Riemannian-metric]{metric tensor} \(g = g_{ij} \mathrm{d} x^i \otimes \mathrm{d} x^j\), then \(\nabla _X g = 0\) for all \hyperref[def:vector-field]{vector fields} \(X\).
\end{eg}
\begin{explanation}
	For all \(X, Y, Z\in \Gamma (T \mathcal{M} )\),
	\[
		\nabla g(X, Y, Z) = Z \langle X, Y \rangle - \langle \nabla _Z X , Y \rangle  - \langle X, \nabla _Z Y \rangle = 0
	\]
	since \(\nabla \) is \hyperref[def:Riemannian]{Riemannian}.
\end{explanation}

It's convenient to use the following identification.

\begin{notation}
	Let \(X\in \Gamma (T \mathcal{M} )\) and identify \(X\) with the \hyperref[def:tensor]{tensor} that associates to \(Y\in \Gamma (T \mathcal{M} )\) the function \(\langle X, Y \rangle \).
\end{notation}

\begin{intuition}
	Consider the \hyperref[def:tensor-covariant-derivative]{covariant derivative} of the \hyperref[def:tensor]{tensor} \(X\) relative to \(Z\in \Gamma (T \mathcal{M} )\), which is such that for all \(Y\in \Gamma (T \mathcal{M} )\),
	\[
		\nabla _Z X(Y) = \nabla X(Y, Z) = Z(X(Y)) - X(\nabla _Z Y) = Z\langle X, Y \rangle - \langle X, \nabla _Z Y \rangle  = \langle \nabla _Z X, Y \rangle.
	\]
	This shows that the \hyperref[def:tensor]{tensor} \(\nabla _Z X\) can be identified with the \hyperref[def:vector-field]{vector field} \(\nabla _Z X\) as well by our new notation!
\end{intuition}

\begin{remark}
	This justifies the notation adopted, and shows that the \autoref{def:tensor-covariant-derivative} is a generalization of \autoref{def:covariant-derivative}.
\end{remark}

\chapter{Isometric Immersions}
Consider \(f\colon \mathcal{M} \to \widetilde{\mathcal{M}} \) be a differentiable \hyperref[def:immersion]{immersion} of a \hyperref[def:smooth-manifold]{manifold} \(\mathcal{M} ^n\)into a \hyperref[def:Riemannian-manifold]{Riemannian manifold} \(\widetilde{\mathcal{M}} ^{k}\) for \(k = n + m\). The \hyperref[def:Riemannian-metric]{Riemannian metric} of \(\widetilde{\mathcal{M}} \) induces, naturally, a \hyperref[def:Riemannian-metric]{Riemannian metric} on \(\mathcal{M} \): if \(v_1, v_2\in T_p \mathcal{M} \), we let
\[
	\langle v_1, v_2 \rangle \coloneqq \langle \mathrm{d} f_p(v_1), \mathrm{d} f_p(v_2) \rangle .
\]
This makes \(f\) an \hyperref[def:isometry]{isometric} \hyperref[def:immersion]{immersion} of \(\mathcal{M} \) into \(\widetilde{\mathcal{M}} \), and we want to study the relationship between the geometry of \(\mathcal{M} \) and that of \(\widetilde{\mathcal{M}} \).

While do Carmo~\cite{flaherty2013riemannian} directly discusses the \hyperref[def:2nd-fundamental-form]{second fundamental form}, we start by introducing the \hyperref[def:Riemannian-covering-map]{Riemannian covering map}, which has a strong connection to the \hyperref[def:2nd-fundamental-form]{second fundamental form} and furnishes a broader view of the theory of \hyperref[def:isometry]{isometric} \hyperref[def:immersion]{immersions}.

\section{Riemannian Covering Maps}
Let's first review the basic notion in algebraic topology.

\begin{definition}[Covering map]\label{def:covering-map}
	Let \(\mathcal{M} , \widetilde{\mathcal{M}} \) be two \hyperref[def:topological-manifold]{manifolds}. A map \(p\colon \widetilde{\mathcal{M}} \to \mathcal{M} \) is a \emph{covering map} if
	\begin{enumerate}[(a)]
		\item \(p\) is smooth and surjective;
		\item for all \(m\in \mathcal{M} \), there exists a neighborhood \(U\) at \(m\) in \(\mathcal{M} \) with \(p ^{-1} (U) = \coprod_{i\in I} U_i\) with \(p\colon U_i \to U\) being a \hyperref[def:diffeomorphism]{diffeomorphism} and \(U_i\) are disjoint open subsets of \(\widetilde{\mathcal{M}} \).
	\end{enumerate}
\end{definition}

\begin{notation}[Covering space]\label{not:covering-space}
	\(\widetilde{\mathcal{M}} \) in \autoref{def:covering-map} is called the \emph{covering space}.
\end{notation}

\begin{notation}[Universal covering space]\label{not:universal-covering-space}
	A \hyperref[not:covering-space]{covering space} is \emph{universal}	if it's simply connected.
\end{notation}

By introducing \hyperref[def:local-isometry]{local isometry}, we have the so-called \hyperref[def:Riemannian-covering-map]{Riemannian covering map}.

\begin{definition}[Riemannian covering map]\label{def:Riemannian-covering-map}
	Let \((\mathcal{M} , g), (\mathcal{N}, h )\) be \hyperref[def:Riemannian-manifold]{Riemannian manifolds}. A map \(p\colon \mathcal{N} \to \mathcal{M} \) is a \emph{Riemannian covering map} if \(p\) is a smooth \hyperref[def:covering-map]{covering map} and is a \hyperref[def:local-isometry]{local isometry}.
\end{definition}

\subsection{Induced Riemannian Covering Maps}
Given a \hyperref[def:covering-map]{covering map}, from a \hyperref[def:Riemannian-metric]{Riemannian metric} \(g\) on the \hyperref[not:covering-space]{covering space}, we obtain a induced \hyperref[def:Riemannian-metric]{Riemannian metric} on the base space and a \hyperref[def:Riemannian-covering-map]{Riemannian covering map}.

\begin{proposition}\label{prop:Riemannian-covering-map}
	Let \(p\colon \mathcal{N} \to \mathcal{M} \) be a smooth \hyperref[def:covering-map]{covering map}. For every \hyperref[def:Riemannian-metric]{Riemannian metric} \(g\) on \(\mathcal{M} \), there exists a unique \hyperref[def:Riemannian-metric]{Riemannian metric} \(h\) on \(\mathcal{N} \) such that \(p\) is a \hyperref[def:Riemannian-covering-map]{Riemannian covering map}.
\end{proposition}

\begin{note}
	The converse of \autoref{prop:Riemannian-covering-map} is generally not true.
\end{note}

Let's first see some examples.

\begin{eg}
	Every space \hyperref[def:covering-map]{covers} itself trivially.
\end{eg}

\begin{eg}
	\(\mathbb{R} \) is the \hyperref[not:universal-covering-space]{universal covering space} of \(S^1\).
\end{eg}

\begin{eg}
	\(U(n)\) has \hyperref[not:universal-covering-space]{universal covers} \(U(n) \times \mathbb{R} \).
\end{eg}

\begin{eg}
	\(S^n\) is a double \hyperref[not:covering-space]{cover} for \(\mathbb{R} P^n\) and is \hyperref[not:universal-covering-space]{universal} for \(n > 1\).
\end{eg}