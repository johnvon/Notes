\lecture{7}{26 Jan. 13:00}{Hopf-Rinow Theorem}
\section{Hopf-Rinow Theorem}
We have shown the following in the homework.

\begin{theorem}
	Let \((\mathcal{M} , g)\) be a compact \hyperref[def:Riemannian-manifold]{Riemannian manifold}.
	\begin{enumerate}[(a)]
		\item Any \(2\) points \(p, q\in \mathcal{M} \) can be connected by a minimizing \hyperref[def:geodesic]{geodesic}.
		\item For all \(p\in \mathcal{M} \), the \hyperref[def:exponential-map]{exponential map} \(\exp _p\) is defined on all of \(T_p \mathcal{M} \) and any \hyperref[def:geodesic]{geodesic} may be extended indefinitely in each direction.
	\end{enumerate}
\end{theorem}

We now want to generalize it. However, this is not true in the most general setting, and we need one more requirement.

\begin{definition}[Geodesically complete]\label{def:geodesically-complete}
	A \hyperref[def:Riemannian-manifold]{Riemannian manifold} \((\mathcal{M} , g)\) is \emph{geodesically complete} if for all \(p\in \mathcal{M} \), \(\exp _p\) is defined on all of \(T_p \mathcal{M} \).
\end{definition}

In other words, a \hyperref[def:Riemannian-manifold]{Riemannian manifold} \(\mathcal{M} \) is \hyperref[def:geodesically-complete]{geodesically complete} if any \hyperref[def:geodesic]{geodesic} \(c(t)\) with \(c(0) = p\) can be extended for all \(t\in \mathbb{R} \). Then, we have the following.

\begin{theorem}[Hopf-Rinow theorem]\label{thm:Hopf-Rinow}
	Let \((\mathcal{M} , g)\) be a \hyperref[def:Riemannian-manifold]{Riemannian manifold}, then the following statements are equivalent.
	\begin{enumerate}[(a)]
		\item\label{thm:Hopf-Rinow-1} \(\mathcal{M} \) is complete as a metric space.\footnote{Hence, equivalently, complete as a topological space w.r.t.\ the underlying topology.}
		\item\label{thm:Hopf-Rinow-2} The closed and bounded subsets of \(\mathcal{M} \) are compact.
		\item\label{thm:Hopf-Rinow-3} There exists \(p\in \mathcal{M} \) such that \(\exp _p\) is defined on all \(T_p \mathcal{M} \).
		\item\label{thm:Hopf-Rinow-4} \(\mathcal{M} \) is \hyperref[def:geodesically-complete]{geodesically complete}.
	\end{enumerate}
	Furthermore, \autoref{thm:Hopf-Rinow-4} (and hence \autoref{thm:Hopf-Rinow-1}, \autoref{thm:Hopf-Rinow-2}, and \autoref{thm:Hopf-Rinow-3}) implies
	\begin{enumerate}[(e)]
		\item\label{thm:Hopf-Rinow-5} for two points \(p, q\in \mathcal{M} \) can be joined by a minimizing \hyperref[def:geodesic]{geodesic}, i.e., \hyperref[def:geodesic]{geodesic} of the shortest \hyperref[def:distance]{distance} \(d(p, q)\).
	\end{enumerate}
\end{theorem}
\begin{proof}\let\qed\relax
	We start by proving \autoref{thm:Hopf-Rinow-4} implies \autoref{thm:Hopf-Rinow-5}. Let \(\mathcal{M} \) be \hyperref[def:geodesically-complete]{geodesically complete}, and let \(r \coloneqq d(p, q)\), and let \(\rho \) be as in the corollary from handout for HW1. Let \(p_0 \in \partial B(p, \rho )\) be a point where the continuous functional \(d(q, \cdot)\) attains its minimum on the compact set \(\partial B(p, \rho )\). Then, for some \(V \in T_p \mathcal{M} \),
	\[
		p_0 = \exp _p \rho V.
	\]
	Consider the \hyperref[def:geodesic]{geodesic} \(c(t) = \exp _p tV\), by showing
	\[
		c(r) = q,
	\]
	\(\at{c}{[0, r]}{} \) will be the shortest \hyperref[def:geodesic]{geodesic} from \(p\) to \(q\). We start by defining
	\[
		I\coloneqq \left\{ t\in[0, r] \mid d(c(t), q) = r-t \right\},
	\]
	and referring to the following diagram to guide us.
	\begin{center}
		\incfig{Hopf-Rinow-theorem}
	\end{center}

	Now, we want to show that \(I = [0, r]\), which will follow from showing that \(I\) is open.
	\begin{note}
		\(I\) is not empty since by definition it contains \(0\) and \(r\). Further, \(I\) is closed by continuity.
	\end{note}
	Let \(t_0 \in I\), and let \(\rho _1 > 0\) be the radius as in the corollary, without loss of generality, \(\rho _1 < r-t_0\). Let \(p_1\in \partial B(c(t_0), \rho _1)\) be the point where the continuous functional \(d(q, \cdot)\) attains its minimum on the compact set \(\partial B(c(t_0), \rho _1)\). By the triangle inequality,
	\[
		d(p, q) \leq d(p, p_1) + d(p_1, q).
	\]
	For every curve \(\gamma \) from \(c(t_0)\) to \(q\), there exists \(\gamma (t)\in \partial B(c(t_0), \rho _1)\), hence
	\[
		L(\gamma )
		\geq \underbrace{d(c(t_0), \gamma (t))}_{\rho _1} + d(\gamma (t), q)
		= \rho _1 + d(p_1, q),
	\]
	implying \(d(q, c(t_0)) \geq \rho _1 + d(p_1, q)\). But from the triangle inequality, we actually have
	\[
		d(q, c(t_0)) = \rho _1 + d(p_1, q)
		\iff d(p_1, q) = \underbrace{d(q, c(t_0))}_{r-t_0} - \rho _1,
	\]
	hence \(d(p_1, p) \geq r-(r-t_0 - \rho _1) = t_0 + \rho _1\), i.e., this is a minimizing curve!

	On the other hand, there exists a curve from \(p\) to \(p_1\) of length \(t_1 + \rho _1\) since it's composed by the portion from \(p\) to \(c(t_0)\) along \(c(t)\) and the portion being the \hyperref[def:geodesic]{geodesic} from \(c(t_0)\) to \(p_1\) of length \(\rho _1\). Then, by the theorem we have proved in the HW1\#5, this curve is a \hyperref[def:geodesic]{geodesic} curve. Finally, from the uniqueness of \hyperref[def:geodesic]{geodesic} with the given extra data, this \hyperref[def:geodesic]{geodesic} coincides with \(c\). Hence,
	\[
		p_1 = c ( t_0 + \rho _1),
	\]
	with \(d(p_1, q) = r-t_0 - \rho _1\),
	\[
		d(c(t_0 + \rho _1), q)
		= d(p_1, q)
		= r - t_0 - \rho
		= r-(t_0 + \rho _1),
	\]
	thus \(t_0 + \rho _1\in I\), hence \(I\) is open, i.e., \(I = [0, r]\), so \(c(r) = q\) follows.
\end{proof}