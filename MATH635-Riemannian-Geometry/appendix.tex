\chapter{Lie Groups and Lie Algebra}\label{ch:Lie-group-and-Lie-algebra}
\section{Lie Groups}
\hyperref[def:Lie-group]{Lie groups} are an important topic to study for Riemannian geometry, hence we now introduce it now.

\begin{definition}[Lie group]\label{def:Lie-group}
	A \emph{Lie group} is a group \(G\) with a \hyperref[def:smooth-structure]{differentiable structure} such that the mapping \(G \times G \to G\) given by \((x, y) \to xy^{-1} \), \(x, y\in G\), is differentiable.
\end{definition}

\begin{definition*}[Transformation]
	Let \(G\) be a \hyperref[def:Lie-group]{Lie group}.
	\begin{definition}[Left transformation]\label{def:left-transformation}
		The \emph{translations from the left} \(L_x \colon G \to G\) is defined as \(L_x(y) = xy\).
	\end{definition}
	\begin{definition}[Right transformation]\label{def:right-transformation}
		The \emph{translations from the right} \(R_x \colon G \to G\) is defined as \(R_x(y) = yx\).
	\end{definition}
\end{definition*}

\begin{remark}
	Both \(L_x\) and \(R_x\) are \hyperref[def:diffeomorphism]{diffeomorphisms}.
\end{remark}

In the following discussion, let \(G\) be a \hyperref[def:Lie-group]{Lie group}. Turns out that \(G\) admits some nice properties on \hyperref[def:vector-field-left-invariant]{left invariant} \hyperref[def:vector-field]{vector fields}.

\begin{definition*}[Invariant of Riemannian metric]
	Let \(g\) be a \hyperref[def:Riemannian-metric]{Riemannian metric} on \(G\).

	\begin{definition}[Left invariant]\label{def:Riemannian-metric-left-invariant}
		\(g\) is \emph{left invariant} if
		\[
			\left\langle u, v \right\rangle _y = \left\langle \mathrm{d} (L_x) _y u, \mathrm{d} (L_x)_y v \right\rangle _{L_x(y)}
		\]
		for all \(x, y\in G\), \(u, v\in T_y G\), i.e., \(L_x\) is an \hyperref[def:isometry]{isometry}.
	\end{definition}

	\begin{definition}[Right invariant]\label{def:Riemannian-metric-right-invariant}
		\(g\) is \emph{right invariant} if
		\[
			\left\langle u, v \right\rangle _y = \left\langle \mathrm{d} (R_x)_y u, \mathrm{d} (R_x)_y v \right\rangle _{R_x(y)}
		\]
		for all \(x, y\in G\), \(u, v\in T_y G\), i.e., \(R_x\) is an \hyperref[def:isometry]{isometry}.
	\end{definition}

	\begin{definition}[Bi-invariant]\label{def:Riemannian-metric-bi-invariant}
		\(g\) is \emph{bi-invariant} if it's both \hyperref[def:Riemannian-metric-right-invariant]{right} and \hyperref[def:Riemannian-metric-left-invariant]{left invariant}.
	\end{definition}
\end{definition*}

\begin{definition*}[Invariant of vector field]
	Let \(X\) be a \hyperref[def:vector-field]{vector field} on \(G\).

	\begin{definition}[Left invariant]\label{def:vector-field-left-invariant}
		\(X\) is \emph{left invariant} if \(\mathrm{d} L_x X = X\) for all \(x\in G\).
	\end{definition}

	\begin{definition}[Right invariant]\label{def:vector-field-right-invariant}
		\(X\) is \emph{right invariant} if \(\mathrm{d} R_x X = X\) for all \(x\in G\).
	\end{definition}

	\begin{definition}[Bi-invariant]\label{def:vector-field-bi-invariant}
		\(X\) is \emph{bi-invariant} if it's both \hyperref[def:vector-field-right-invariant]{right} and \hyperref[def:vector-field-left-invariant]{left invariant}.
	\end{definition}
\end{definition*}

As we mentioned, the \hyperref[def:vector-field-left-invariant]{left invariant} \hyperref[def:vector-field]{vector fields} are completely determined by their values at a single point of \(G\), which allows us to introduce an additional structure on the \hyperref[def:tangent-space]{tangent space} to the neutral element \(e\in G\) in the following manner.

To each \hyperref[def:tangent-vector]{vector} \(X_e\in T_e G\), we associate the \hyperref[def:vector-field-left-invariant]{left invariant} \(X\) defined by
\[
	X_a \coloneqq \mathrm{d} L_a X_e,\quad a\in G.
\]

\section{Lie Algebras}
Let \(X, Y\) be \hyperref[def:vector-field-left-invariant]{left invariant} \hyperref[def:vector-field]{vector fields} on \(G\). Since for each \(x\in G\) and for any differentiable function \(f\) on \(G\),
\[
	\mathrm{d} L_x [X, Y] f = [X, Y](f \circ L_x) = X(\mathrm{d} L_x Y) f - Y(\mathrm{d} L_x X) f = (XY - YX) f = [X, Y]f,
\]
i.e., \([X, Y]\) is again a \hyperref[def:vector-field-left-invariant]{left invariant} \hyperref[def:vector-field]{vector field} if \(X, Y\) are. Now, if \(X_e, Y_e\in T_e G\), we put \([X_e, Y_e] = [X, Y]_e\).

\begin{definition}[Lie algebra]\label{def:Lie-algebra}
	The \emph{Lie algebra} of \(G\), denoted by \(\mathfrak{g}\), is the vector space \(T_e G\) with the \hyperref[def:bracket]{bracket} \([\cdot, \cdot]\).
\end{definition}

\begin{note}
	The elements in the \hyperref[def:Lie-algebra]{Lie algebra} \(\mathfrak{g}\) will be thought of either as \hyperref[def:tangent-vector]{vectors} in \(T_e G\) or as \hyperref[def:vector-field-left-invariant]{left invariant} \hyperref[def:vector-field]{vector fields} on \(G\).
\end{note}

To introduce a \hyperref[def:Riemannian-metric-left-invariant]{left invariant} \hyperref[def:Riemannian-metric]{metric} on \(g\), take any arbitrary inner product \(\left\langle \cdot, \cdot \right\rangle _e\) on \(\mathfrak{g} \) and define
\begin{equation}\label{eq:inner-product-on-Lie-algebra}
	\left\langle u, v \right\rangle _x \coloneqq \left\langle (\mathrm{d} L_{x ^{-1} })_x(u), (\mathrm{d} L_{x ^{-1} })_x(v) \right\rangle _e
\end{equation}
for \(x\in G\), \(u, v\in T_x G\). Since \(L_x\) depends differentiably on \(x\), this is actually a \hyperref[def:Riemannian-metric]{Riemannian metric}, which is clearly \hyperref[def:Riemannian-metric-left-invariant]{left invariant}.

\begin{remark}
	We can also construct a \hyperref[def:Riemannian-metric-right-invariant]{right invariant} \hyperref[def:Riemannian-metric]{metric} on \(G\), and if \(G\) is compact, \(G\) possesses a \hyperref[def:Riemannian-metric-left-invariant]{bi-invariant} \hyperref[def:Riemannian-metric]{metric}.
\end{remark}

One important characterization for \(G\) having a \hyperref[def:Riemannian-metric-left-invariant]{bi-invariant} \hyperref[def:Riemannian-metric]{metric} is that the inner product that the \hyperref[def:Riemannian-metric]{metric} determines on \(\mathfrak{g} \) satisfies the following relation.

\begin{proposition}
	If \(G\) has a \hyperref[def:Riemannian-metric-left-invariant]{bi-invariant} \hyperref[def:Riemannian-metric]{metric}, then for any \(U, V, X\in \mathfrak{g} \), the inner product that the \hyperref[def:Riemannian-metric]{metric} determines on \(\mathfrak{g} \) satisfies
	\[
		\left\langle [U, X], V \right\rangle = -\left\langle U, [V, X] \right\rangle.
	\]
\end{proposition}
\begin{proof}
	See do Carmo~\cite[Page 40, 41]{flaherty2013riemannian}.
\end{proof}

The important point about this relation is that it characterizes the \hyperref[def:Riemannian-metric-left-invariant]{bi-invariant} \hyperref[def:Riemannian-metric]{metrics} of \(G\) in the following sense.

\begin{remark}
	If a positive bilinear form \(\left\langle \cdot, \cdot \right\rangle _e \) defined on \(\mathfrak{g} \) satisfies this relation, then the \hyperref[def:Riemannian-metric]{Riemannian metrics} defined on \(G\) by \autoref{eq:inner-product-on-Lie-algebra} is \hyperref[def:Riemannian-metric-left-invariant]{bi-invariant}.
\end{remark}

\chapter{Algebra}\label{ch:Algebra}
This chapter will collect some notion about algebras which you might not be familiar with.

\section{Modules}
\begin{definition}[Left module]\label{def:left-module}
	Suppose \(R\) is a ring with \(1\). A \emph{left \(R\)-module} \(M\) consists of an Abelian group \((M, +)\) and n operation \(\cdot \colon R \times M \to M\) such that for all \(r, s\in R\) and \(x, y\in M\),
	\begin{enumerate}[(a)]
		\item \(r\cdot (x+y) = r\cdot x + r\cdot y\);
		\item \((r+s)\cdot x = r\cdot x + s\cdot x\);
		\item \((rs)\cdot x = r\cdot (s\cdot x)\);
		\item \(1\cdot x = x\).
	\end{enumerate}
\end{definition}

\begin{note}
	A \emph{right \(R\)-module} \(M\) can also be defined similarly by consider \(\cdot \colon M \times R \to M\).
\end{note}

\begin{definition}[Module]\label{def:module}
	If \(R\) is commutative, then the \hyperref[def:left-module]{left and right \(R\)-module} \(M\) are the same, and we call \(M\) a \emph{module}.
\end{definition}

\begin{intuition}
	We're basically relaxing the notion of \(\mathbb{F} \)-vector field, but this time, the field \(\mathbb{F} \) is replaced by a ring \(R\).
\end{intuition}

\begin{remark}
	The most noticeable difference between a \hyperref[def:module]{module} and a vector field is that a \hyperref[def:module]{module} usually don't have a basis.
\end{remark}

\subsection{The \(C^{\infty} (\mathcal{M} )\)-Module Viewpoint of Tensor Fields}\label{subsection:C-infty-module-viewpoint-of-tensor-fields}
The reason why we introduce the notion of \hyperref[def:module]{module} is because of the following: we can understand \hyperref[def:tensor-field*]{tensor-field} better in the following way. Firstly, let's introduce the so-called \hyperref[def:tensor-bundle]{tensor bundles}.

\begin{definition}[Tensor bundle]\label{def:tensor-bundle}
	A \emph{tensor bundle} is a \hyperref[def:bundle]{fiber bundle} where the \hyperref[def:fiber]{fiber} is the product of any number of \hyperref[def:tangent-space]{tangent spaces} and/or \hyperref[def:cotangent-space]{cotangent spaces}.
\end{definition}

So in a \hyperref[def:tensor-bundle]{tensor bundle}, the \hyperref[def:fiber]{fiber} is a vector space and the \hyperref[def:tensor-bundle]{tensor bundle} is a special kind of \hyperref[def:vector-bundle]{vector bundle}.\footnote{There are \hyperref[def:vector-bundle]{vector bundles} which are not \hyperref[def:tensor-bundle]{tensor bundles}.} Then, we have the following notion similar to the way we define \hyperref[def:vector-field]{vector fields}.

\begin{definition}[Tensor field]\label{def:tensor-field*}
	A \emph{\((r, s)\)-tensor field} \(T\) is a \hyperref[def:section]{section} of a \hyperref[def:tensor-bundle]{tensor bundle}.
\end{definition}

To elaborate this idea, observe that \(\Gamma (T\mathcal{M} ) = \left\{ X\colon \text{\hyperref[def:vector-field]{vector fields} on \(\mathcal{M} \)}\right\} \) is actually a \hyperref[def:module]{\(C^{\infty} (\mathcal{M} )\)-module}:

\begin{claim}
	\(\Gamma (T \mathcal{M} )\) carries a natural \hyperref[def:module]{\(C^{\infty} (\mathcal{M} )\)-module} structure.
\end{claim}
\begin{explanation}
	Firstly, observe that \(C^{\infty} (\mathcal{M} ) = \big( (C^{\infty} (\mathcal{M} ), +, \cdot)\big)\) is not a field but a ring.\footnote{Since given \(f\in C^{\infty} (\mathcal{M} )\), we might not have \(f^{-1} \).} Then, naturally, the \hyperref[def:module]{\(C^{\infty} (\mathcal{M} )\)-module} \((\Gamma (T\mathcal{M} ), \oplus , \odot)\) where
	\begin{itemize}
		\item \(\oplus \): \((X \oplus \widetilde{X} )(f) \coloneqq (Xf) + \widetilde{X} (f)\);
		\item \(\odot\): \((g\odot X)(f) \coloneqq g\cdot X(f)\),
	\end{itemize}
	for \(X, \widetilde{X} \in \Gamma (T \mathcal{M} )\), \(g, f\in C^{\infty} (\mathcal{M} )\).
\end{explanation}

\begin{notation}
	Notice that given a \hyperref[def:vector-field]{vector field} \(X\colon \mathcal{M} \to  T \mathcal{M} \) with \(p \mapsto X(p)\), we let
	\[
		Xf\colon \mathcal{M} \to \mathbb{R} ,\quad p \mapsto X(p) f.
	\]
\end{notation}

\begin{remark}
	This makes sense since we can't always do things globally, e.g., \href{https://en.wikipedia.org/wiki/Hairy_ball_theorem}{Hairy ball theorem}. More precisely, we can't choose a basis \(X_1, \ldots , X_d\in \Gamma (T \mathcal{M} )\) for our \hyperref[def:vector-field]{vector field} globally as we already know.
\end{remark}

Then, similarly, we can define \(\Gamma (T^{\ast} \mathcal{M} )\), i.e., the set of ``convector field,''\footnote{We won't define it formally, but it's defined similarly.} which is again a \hyperref[def:module]{\(C^{\infty} (\mathcal{M} )\)-module}.

\begin{eg}
	Given \(f\in C^{\infty} (\mathcal{M} )\), let \(\mathrm{d} f \colon \Gamma (T \mathcal{M} ) \to  C^{\infty} (\mathcal{M} )\) with \(X \mapsto \mathrm{d} f(X) \coloneqq Xf\). We see that \(\mathrm{d} f\) is a \hyperref[def:tensor-field*]{\((0, 1)\)-tensor field} since \(\mathrm{d} f\) is linear.
\end{eg}

Then, in this view point,\footnote{This is a more abstract way (but often useful) to characterize a \hyperref[def:tensor-field*]{tensor field}.} \hyperref[def:tensor-field*]{tensor field} \(T\) is a \(C^{\infty} (\mathcal{M} )\) multilinear map
\[
	T \colon \underbrace{\Gamma (T ^{\ast} \mathcal{M} ) \times \ldots \times \Gamma (T ^{\ast} \mathcal{M} )}_{r} \times \underbrace{\Gamma (T \mathcal{M} ) \times \ldots \Gamma (T \mathcal{M} )}_{s} \to C^{\infty} (\mathcal{M} ).
\]

\begin{remark}
	Comparing to \autoref{def:tensor-field}, they're essentially the same.
\end{remark}