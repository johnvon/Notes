\lecture{12}{11 Oct.\ 8:00}{Farkas Lemma}
\begin{proof}[Proof of \autoref{lma:Farkas}]
	As what we have outlined, we divide the proof into two cases.
	\begin{claim}
		\((\mathrm{I})\) and \((\mathrm{II})\) can't both have solutions.
	\end{claim}
	\begin{explanation}
		Suppose \(\hat{x}\) solves I and \(\hat{y}\) solves \((\mathrm{II})\). Then we have
		\[
			\hat{y}^{\top}(\hat{Ax} = b) \implies \underbrace{(\hat{y}^{\top} A)}_{\geq  \vec{0}}\underbrace{\vphantom{\hat{y}^{\top}}\hat{x}}_{\geq  \vec{0}} = \hat{y} b > 0\conta
		\]
	\end{explanation}
	\begin{claim}
		At least one of \((\mathrm{I})\) or \((\mathrm{II})\) has a solution \(\cong\) If \((\mathrm{I})\) has no solution, then \((\mathrm{II})\) has a solution.
	\end{claim}
	\begin{explanation}
		Assume that \((\mathrm{I})\) has no solution, which means that \((\Primal)\) is infeasible with \((\Primal)\) being
		\[
			\begin{aligned}
				\min~          & \vec{0}^{\top} x \\
				               & Ax = b           \\
				(\Primal)\quad & x\geq 0.
			\end{aligned}
		\]

		The dual of this \((\Primal)\) is
		\[
			\begin{aligned}
				\max~        & y^{\top}b                       \\
				(\Dual)\quad & y^{\top} A \leq \vec{0}^{\top}.
			\end{aligned}
		\]

		But this means that \((\Dual)\) is infeasible or unbounded. But we see that \((\Dual)\) can't be infeasible, because \(y = \vec{0}\) is a \hyperref[def:feasible-solution]{feasible solution}, then we know
		\[
			\begin{split}
				 & \implies D \text{ is unbounded}                                                                            \\
				 & \implies \text{ there exist a feasible solution \(\widetilde{y}\) to \((\Dual)\) with positive objective}.
			\end{split}
		\]
	\end{explanation}
\end{proof}

\begin{remark}
	Now, consider \(\lambda \widetilde{y}\) (feasible for \((\Dual)\)). Drive to \(+\infty \) by increasing \(\lambda\). We now see what \hyperref[lma:Farkas]{Farkas Lemma} really tells us.
	\[
		\begin{aligned}
			\min~          & c^{\top} x                                               \\
			               & Ax = b                                                   \\
			(\Primal)\quad & x\geq 0                  &  & \text{feasibility}         \\
			               &                          &  & \Updownarrow               \\
			\max~          & y^{\top}b                &  & \text{unbounded direction} \\
			(\Dual)\quad   & y^{\top} A \leq c^{\top}                                 \\
		\end{aligned}
	\]
	Suppose \(\widetilde{y}\) is feasible to \((\Dual)\) and suppose \(\hat{y}\) satisfies \((\mathrm{II})\), then
	\[
		(\widetilde{y}+\lambda \hat{y})^{\top} A = \underbrace{\widetilde{y}^{\top}A}_{\leq c^{\top}} + \underbrace{\vphantom{y}\lambda}_{>0} \underbrace{\hat{y}^{\top} A}_{\leq \vec{0}} \leq c^{\top}.
	\]
	Furthermore, we have
	\[
		(\widetilde{y}+\lambda \hat{y})^{\top} b = \widetilde{y}^{\top} b+\lambda \hat{y}^{\top} b \implies \infty \text{ as } \lambda \uparrow  .
	\]
\end{remark}

\begin{eg}
	Given
	\[
		\begin{aligned}
			(\mathrm{I})\quad  & Ax\leq b \\
			(\mathrm{II})\quad & ?,
		\end{aligned}
	\]
	find out what \((\mathrm{II})\) is.
\end{eg}
\begin{explanation}
	We simply set up the \((\Primal)\) and then find its dual.
	\[
		\begin{alignedat}{5}
			\min~          & \vec{0}^{\top} x\qquad\qquad &  & \max ~       &  & y^{\top}b           \\
			               & Ax \leq b                    &  &              &  & y^{\top} A =\vec{0} \\
			(\Primal)\quad &                              &  & (\Dual)\quad &  & y\leq \vec{0}
		\end{alignedat}.
	\]

	Then we have
	\[
		\begin{aligned}
			(\mathrm{I})\quad  & Ax\leq b            \\
			(\mathrm{II})\quad & y^{\top}A = \vec{0} \\
			                   & y\leq \vec{0}       \\
			                   & y^{\top} b>0
		\end{aligned}
	\]
	Check\(\colon\)
	\[
		0 = \underbrace{\hat{y}^{\top} A}_{=\vec{0}} \hat{x} \underset{\hat{y} \leq \vec{0}}{\geq}  \hat{y}^{\top} b>0 \conta
	\]
	or,
	\[
		\begin{split}
			 & Ax \overset{y\leq \vec{0}}{\leq} b\qquad (y^{\top}b>0)                          \\
			 & 0 \overset{?}{\geq} \underbrace{y^{\top}A}_{ = \vec{0}} x\geq y^{\top}b>0\conta
		\end{split}
	\]
\end{explanation}

Let's look at another example.

\begin{eg}
	\[
		\begin{alignedat}{4}
			(\min~            & \vec{0}^{\top} &  & x &  & + \vec{0}^{\top} &  & w)                    \\
			                  & A              &  & x &  & + B              &  & w = b                 \\
			                  &                &  &   &  & -F               &  & w \geq f              \\
			(\mathrm{I})\quad &                &  & x &  & \geq  0,         &  & w\text{ unrestricted} \\
		\end{alignedat}
	\]
	with the \hyperref[def:dual]{dual} variables \(y, w\), we have
	\[
		\begin{aligned}
			                   & (\text{Suppose \((\mathrm{I})\) has no solution.}) \\
			\cancel{\max}~     & y^{\top} b + v^{\top} b (>0)                       \\
			                   & y^{\top} A \leq \vec{0}                            \\
			(\mathrm{II})\quad & y^{\top} B - v^{\top} F = \vec{0}
		\end{aligned}
	\]
	with \(y\) unrestricted, \(v\geq  \vec{0}\).
\end{eg}

Now, we should have a general picture about what \hyperref[lma:Farkas]{Farkas Lemma} really means. For conditions \((\mathrm{I})\) and \((\mathrm{II})\), we have
\[
	\begin{alignedat}{3}
		 & (\mathrm{I}) \qquad  &  & Ax = b                                                                                   \\
		 &                      &  & x\geq 0                &  & \iff b \text{ is in the cone }K                              \\
		 & (\mathrm{II}) \qquad &  & y^{\top}b> 0           &  & \iff y \text{ makes an acute angle with }b.                  \\
		 &                      &  & y^{\top}A\leq 0^{\top} &  & \quad y\text{ makes a non-acute angle with all columns of }A
	\end{alignedat}
\]

Suppose \(\hat{z}\) in \(K\), then
\[
	\hat{z} = A \hat{x} \text{ for some }\hat{x} \geq  \vec{0}.
\]
Then we have
\[
	y^{\top} \hat{z} = \underbrace{y^{\top} A}_{\leq\vec{0}^{\top}} \underbrace{\vphantom{y}\vec{x}}_{\geq  \vec{0}}  \leq 0.
\]

We see that \(y\) makes a non-acute angle with everything in \(K\). Now, suppose \(\hat{y}\) solves \((\mathrm{II})\). Consider
\[
	\underbrace{\hat{y}^{\top}}_{\text{numbers}} \underbrace{\vphantom{y}z}_{\text{variables}} = 0.
\]

Now, we have the hyperplane\(\colon\)\(\{z\colon \hat{y} ^{\top} z = 0\}\) separates \(b\) and \(K\).

\begin{figure}[H]
	\centering
	\incfig{Farkas-lemma-extended}
	\caption{Case \((\mathrm{II})\) of the \hyperref[lma:Farkas]{Farkas Lemma} with \(m = 2\)}
	\label{fig:Farkas-lemma-extended}
\end{figure}

\subsection{The Big Picture of Cones}
Consider the \hyperref[def:general-linear-programming-problem]{linear programming problem}
\[
	\begin{aligned}
		\max~ & y^{\top} b               \\
		      & y^{\top} A \leq c^{\top} \\
	\end{aligned}
\]
with the \hyperref[def:partition]{partition} \(\beta, \eta\), we see that
\[
	y^{\top} A \leq c^{\top} \implies \begin{dcases}
		y^{\top} A_{\beta} & \leq c^{\top}_{\beta} \\
		y^{\top} A_{\eta}  & \leq c^{\top}_{\eta}
	\end{dcases}.
\]
By solving only for \(\beta\), then we have \(\overline{y}^{\top} = c_{\beta}^{\top} A^{-1}_{\beta}\). And then, by considering the cones, we have

\begin{figure}[H]
	\centering
	\incfig{opt-cones}
	\caption[Caption for LOF]{Optimality of Cones.\protect\footnotemark}
	\label{fig:opt-cones}
\end{figure}
\footnotetext{This corresponds to the case that we run into the overlapping issue in \autoref{fig:cones-join}.}
with
\begin{figure}[H]
	\centering
	\incfig{cones-join}
	\caption{Cones join together.}
	\label{fig:cones-join}
\end{figure}

\begin{note}
	Consider \(b = \vec{0}\) (\(\hat{y}\)). It's in every cone \(\implies\) every point is \hyperref[def:optimal-solution]{optimal}.
\end{note}

\begin{remark}
	Each corner (\hyperref[def:extreme-point]{extreme point}) corresponds to a \hyperref[def:solution]{solution} for \(\beta\), while the blue vector \(\vec{b}\) corresponds to the \hyperref[def:dual]{dual} constraints \(y^{\top} A_{\eta}<c_{\eta}^{\top}\). Only when the blue vector are in the region of orange sectors span by two \emph{normal vectors} of \(y^{\top}A_{\cdot \beta_i}\leq c_{\beta_i}\), the constraints are satisfied.
\end{remark}

\section{Strict Complementary Slackness}
Consider
\[
	\begin{aligned}
		\min~          & c^{\top}x \\
		               & Ax = b    \\
		(\Primal)\quad & x\geq  0,
	\end{aligned}\quad \begin{aligned}
		\max ~       & y^{\top}b               \\
		             & y^{\top}A\leq c^{\top}. \\
		(\Dual)\quad &
	\end{aligned}
\]

\begin{prev}
	\hyperref[def:complementary]{Complementarity} of \(\hat{x}\) and \(\hat{y}\)\(\colon\)
	\[
		\begin{split}
			(c_{j} - \hat{y}^{\top} A_{\cdot j}) \hat{x}_j = 0 & , \text{ for }j = 1\dots n  \\
			y^{\top}_i (A_{i\cdot}\hat{x} - b_{i}) = 0         & , \text{ for } i = 1\dots m
		\end{split}
	\]
\end{prev}

Now, let's introduce the so-called \hyperref[def:strictly-complementary]{\cancel{over} strictly complementary}.

\begin{definition}[Strictly complementary]\label{def:strictly-complementary}
	For \hyperref[def:feasible-solution]{feasible solutions} \(\hat{x}\) and \(\hat{y}\) are \emph{strictly \hyperref[def:complementary]{complementary}} if
	they are \hyperref[def:complementary]{complementary} and exactly one of
	\[
		c_{j} - \hat{y}^{\top}A_{\cdot j}\text{ and }\hat{x}_j \text{ is } 0.
	\]
\end{definition}

Then, we have the following~\cite[Exercise 5.5]{Linear-Opt}.

\begin{theorem}[Strictly complementarity]\label{thm:strictly-complementarity}
	If \((\Primal)\) and \((\Dual)\) are both \hyperref[def:feasible-solution]{feasible}, then for \((\Primal)\) and \((\Dual)\) \underline{there exist strictly}
	\hyperref[def:complementary]{complementary} (\hyperref[def:feasible-solution]{feasible}) \hyperref[def:optimal-solution]{optimal} solutions.
\end{theorem}

\begin{intuition}
	Let \(v\) be the \hyperref[def:optimal-solution]{optimal} value of \((\Primal)\)\(\colon\)
	\[
		\begin{aligned}
			v = \min~      & c^{\top}x \\
			               & Ax = b    \\
			(\Primal)\quad & x\geq 0
		\end{aligned}
	\]
	Now, we try to find an \hyperref[def:optimal-solution]{optimal solution} with
	\[
		x_{j}>0, \quad \text{fix }j
	\]
	by formulating the following linear programming
	\[
		\begin{aligned}
			\max~      & x_{j}           \\
			           & c^{\top}x\leq v \\
			           & Ax = b          \\
			(P_j)\quad & x\geq  0
		\end{aligned}
	\]
	where \(P_{j}\) seeks an \hyperref[def:optimal-solution]{optimal solution} of \((\Primal)\) that has \(x_{j}\) being positive.
	If failed, then construct an \hyperref[def:optimal-solution]{optimal solution} \(\hat{y}\) to \((\Dual)\) with
	\[
		c_{j} - \hat{y}^{\top} A_{\cdot j}>0.
	\]

	We then see for any \textbf{fixed} \(j\), the desired property holds. The only thing we need to do is combine these \(n\) pairs of \(\hat{x}\) and \(\hat{y}\)
	appropriately to construct \hyperref[def:optimal-solution]{optimal} \(\hat{x}\) and \(\hat{y}\) that are overly \hyperref[def:complementary]{complementary}.
\end{intuition}