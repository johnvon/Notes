\lecture{8}{27 Sep.\ 8:00}{Simplex Algorithm}
\section{Remaining Problems}
Now, a big question is, how do we start with a \hyperref[def:basic-partition]{basic feasible partition}? The answer is to consider the so-called \hyperref[def:phase-one-problem]{phase one problem}.

\subsection{Phase one problem}
Consider the following problem.

\begin{definition}[Phase one problem]\label{def:phase-one-problem}
	Given the \hyperref[def:primal]{primal} \((\mathrm{P})\), the \emph{phase one problem} \((\Phi)\) is defined as follows.
	\[
		\begin{aligned}
			\min~             & c^{\top}x   \\
			                  & Ax = b      \\
			(\mathrm{P})\quad & x\geq  0 	.
		\end{aligned}\quad \begin{aligned}
			\min~       & x_{n+1}                   \\
			            & Ax + A_{n+1}x_{n+1} = b   \\
			(\Phi)\quad & x\geq 0, x_{n+1} \geq  0.
		\end{aligned}
	\]
\end{definition}

\begin{remark}
	We see that
	\begin{itemize}
		\item If min value of \(x_{n+1}\) in \((\Phi)\) is 0, then we get a \hyperref[def:feasible-solution]{feasible solution} of \((\mathrm{P})\).
		\item If min value of \(x_{n+1}\) in \((\Phi)\) is \(>0\), then there is no \hyperref[def:feasible-solution]{feasible solution} of \((\mathrm{P})\).
	\end{itemize}
\end{remark}

By solving \((\Phi)\), we will get a \hyperref[def:feasible-solution]{feasible solution} for \((\mathrm{P})\) or determine whether \((\mathrm{P})\) is solvable in the first place. But to solve the linear program \((\Phi)\), we're facing the same problem as \((\mathrm{P})\).

\begin{problem}\label{prob:phase-one-problem}
How do we get an initial \hyperref[def:basic-solution]{basic} \hyperref[def:feasible-solution]{feasible solution} for \((\Phi)\)?
\end{problem}
\begin{answer}
	In this case, we know how to get a \hyperref[def:basic-solution]{basic} \hyperref[def:feasible-solution]{feasible solution} for \((\Phi)\).
	\begin{enumerate}[(a)]
		\item Start with a \hyperref[def:basic-solution]{basic solution} of \((\mathrm{P})\), \(\tilde{\beta}, \tilde{\eta}\) is the \hyperref[def:basic-partition]{basic partition}.
		\item If \(\overline{x}_{\tilde{\beta}}\) is \hyperref[def:feasible-solution]{feasible} then we just use \(\tilde{\beta}\) and \(\tilde{\eta}\) for \(\beta\) and \(\eta\).
		\item Otherwise, set \(A_{n+1} = -A^{-1}_{\beta}\vec{1}\). If \(\eta_j = n+1\)
		      \[
			      \overline{z} : \overline{z}_{\tilde{\eta}} = \begin{pmatrix}
				      0      \\
				      0      \\
				      \vdots \\
				      1      \\
			      \end{pmatrix},\qquad \overline{z}_{\beta} \coloneqq -A^{-1}_{\tilde{\beta}}(A_{n+1}) = \vec{1}
		      \]
		      and\[
			      \vec{x} \to \vec{x} + \lambda \vec{z} \geq \vec{0}.
		      \] \begin{eg}
			      \[
				      \vec{x}_{\tilde{\beta}} + \lambda \vec{z}_{\tilde{\beta}} = \begin{pmatrix}
					      7  \\
					      0  \\
					      3  \\
					      -5 \\
					      6  \\
					      -8 \\
				      \end{pmatrix} + \lambda \begin{pmatrix}
					      1 \\
					      1 \\
					      1 \\
					      1 \\
					      1 \\
					      1 \\
				      \end{pmatrix},
			      \]
			      then
			      \[
				      i^{\ast} = \underset{i:\vec{x}_{\tilde{\beta}} < 0}{\arg\min}\{-\vec{x}_{\tilde{\beta}}\}.
			      \]
		      \end{eg}
	\end{enumerate}
\end{answer}

\begin{remark}
	If \(x_{n+1} = 0\), we can just stop right before \(x_{n+1} = 0\), let other variable do that.
\end{remark}

\subsection{Perturbed Problem}
Though we now know how to get \hyperref[def:basic-solution]{basic} \hyperref[def:feasible-solution]{feasible solution} for \((\mathrm{P})\) to start our \hyperref[algo:worry-free-simplex-algorithm]{worry-free simplex algorithm}, we have one more problem.

\begin{problem}[Degenerate problem]\label{prob:degenerate-problem}
What if \(\lambda = 0\), i.e. \(x_{\beta _{i} } = 0\) for some \(i\)?
\end{problem}

In this case, we'll need the so-called \hyperref[conj:non-degeneracy-hypothesis]{non-degeneracy hypothesis}.

\begin{conjecture}[Non-degeneracy hypothesis]\label{conj:non-degeneracy-hypothesis}
	At every iteration of \hyperref[algo:worry-free-simplex-algorithm]{worry-free simplex algorithm}, \(x_{\beta _{i} } > 0\) for all \(i\).
\end{conjecture}

\begin{remark}[Termination analysis]
	With \hyperref[conj:non-degeneracy-hypothesis]{non-degeneracy hypothesis}, we see that \hyperref[algo:worry-free-simplex-algorithm]{worry-free simplex algorithm} will certainly terminate.
\end{remark}
\begin{explanation}
	We have
	\[
		\begin{split}
			\vec{x}_{\beta_i} > 0 \text{ for all \(i\) at every iteration}
			\implies &\overline{\lambda} \neq 0\\
			\implies &\text{ objective value decrease at each iteration.}\\
			\implies &\text{ \autoref{algo:worry-free-simplex-algorithm} must terminate}
		\end{split}
	\]
	because there are only finitely many \hyperref[def:basis]{bases}.
\end{explanation}

Though the \hyperref[conj:non-degeneracy-hypothesis]{non-degeneracy hypothesis} helps avoid the mess, but since we want to be able to solve any \hyperref[def:general-linear-programming-problem]{general linear program}, hence we now try to avoid using this hypothesis. We first consider the following problem called \hyperref[def:perturbed-problem]{perturbed problem}.

\begin{definition}[Perturbed problem]\label{def:perturbed-problem}
	Given a \hyperref[def:standard-form]{standard form} problem, the following induced problem is called \emph{perturbed problem}.
	\[
		\begin{aligned}
			\min~ & c^{\top} x               \\
			      & Ax = b + B\begin{pmatrix}
				                  \epsilon   \\
				                  \epsilon^2 \\
				                  \epsilon^3 \\
				                  \vdots     \\
				                  \epsilon^m \\
			                  \end{pmatrix} \\
			      & x\geq 0
		\end{aligned}
	\]
	where \(\epsilon \) is an arbitrarily small \emph{indeterminate}.
\end{definition}

\begin{remark}
	Note that \(\epsilon \neq 0\).
\end{remark}

\begin{note}
	We see that
	\[
		\vec{x}_{\beta} = A^{-1}_{\beta}\left(b + B\begin{pmatrix}
				\epsilon   \\
				\epsilon^2 \\
				\vdots     \\
				\epsilon^m \\
			\end{pmatrix}\right) = A^{-1}_{\beta}b + A^{-1}_{\beta}B\begin{pmatrix}
			\epsilon   \\
			\epsilon^2 \\
			\vdots     \\
			\epsilon^m \\
		\end{pmatrix},
	\]
	which is just a \hyperref[def:polynomial-in-epsilon]{polynomial in \(\epsilon\)}.
\end{note}

\begin{definition}[Polynomial in \(\epsilon \)]\label{def:polynomial-in-epsilon}
	We denote polynomials in \(\epsilon\) as
	\[
		p(\epsilon) = p_0 + p_1\epsilon + p_2 \epsilon^2 + \cdots + p_{m}\epsilon^m,
	\]
	where \(p_i\in\mathbb{R}\).
\end{definition}

Which suggest the following definitions.

\begin{definition}[Sign of polynomial in \(\epsilon \)]\label{def:sign-of-polynomial-in-epsilon}
	Let \(K\) be the minimal index with \(p_K \neq 0\).
	\begin{itemize}
		\item If \(p_K < 0\), then \(p(\epsilon) < 0\)
		\item If \(p_K > 0\), then \(p(\epsilon) > 0\)
		\item If \(p_K = 0\), namely \(p_0 = p_1 = \cdots = p_m = 0\), then \(p(\epsilon) = 0\)
	\end{itemize}
\end{definition}

\begin{note}
	Given
	\[
		\begin{split}
			p(\epsilon) &= p_0 + p_1\epsilon + p_2 \epsilon^2 + \cdots + p_{m}\epsilon^m	\\
			q(\epsilon) &= q_0 + q_1\epsilon + q_2 \epsilon^2 + \cdots + q_{m}\epsilon^m	\\
		\end{split}
	\]
	with \(K_p\) and \(K_q\). Then \(K_{p+q}\) depends on \(K_p\) and \(K_q\). We then see that if \(p(\epsilon) - q(\epsilon) \geq 0\), then
	\(p(\epsilon) \geq  q(\epsilon)\).
\end{note}

\begin{problem*}
	Where does this \(\epsilon \) thing links with the \hyperref[algo:worry-free-simplex-algorithm]{worry-free simplex algorithm}, and how can it solve the \hyperref[prob:degenerate-problem]{degenerate problem}?
\end{problem*}
\begin{answer}
	Suppose
	\[
		\overbrace{p(\epsilon)}^{\substack{\text{value of}\\ \text{some basic}\\ \text{variable}}} = p_0 + p_1 \epsilon + p_2 \epsilon^2 + \cdots + p_m \epsilon^m.
	\]

	Feasibility for the \hyperref[def:perturbed-problem]{perturbed problem} means \(p(\epsilon) \geq \vec{0}\) \(\implies p(0) = p_0 \geq 0\).
	\[
		\begin{aligned}
			\min~ & c^{\top}x                         \\
			      & Ax = b + \cancel{B\vec{\epsilon}} \\
			      & x\geq 0.
		\end{aligned}
	\]
	Find an initial feasible basis \(\beta, \eta \) for unperturbed problem, \(B\coloneqq A_{\beta}\),
	\[
		\vec{x}_{\beta}
		= A_{\beta}^{-1}(b + A_{\beta} \vec{\epsilon})
		= \underbrace{A^{-1}_{\beta}b}_{\geq \vec{0}} + \vec{\epsilon}
		= \vec{x}_{\beta} + \begin{pmatrix}
			\epsilon   \\
			\epsilon^2 \\
			\epsilon^3 \\
			\vdots     \\
			\epsilon^m \\
		\end{pmatrix} = \begin{pmatrix}
			\vec{x}_{\beta_1} + \epsilon   \\
			\vec{x}_{\beta_2} + \epsilon^2 \\
			\vdots                         \\
			\vec{x}_{\beta_m} + \epsilon^m \\
		\end{pmatrix} \geq \vec{0}.
	\]

	\begin{claim}
		\hyperref[def:perturbed-problem]{Perturbed problem} is \hyperref[conj:non-degeneracy-hypothesis]{non-degenerate}.
	\end{claim}
	\begin{explanation}
		This is equivalent to show that there are no \(i\) in the later basis \(\tilde{\beta}\) such that \(\vec{x}_{\tilde{\beta_i}} = 0\). Suppose there is an \(i\) such that
		\(\vec{x}_{\tilde{\beta_i}} = 0\). But since
		\[
			\vec{x}_{\tilde{\beta}} \coloneqq A^{-1}_{\tilde{\beta}}(b + A_{\beta}\vec{\epsilon}) = A^{-1}_{\tilde{\beta}}b + A^{-1}_{\tilde{\beta}}A_{\beta}\vec{\epsilon},
		\]
		if \(\vec{x} _{\widetilde{\beta}_i} = 0\), we must have
		\[
			\begin{split}
				i^{\text{th}} \text{ element of }A^{-1}_{\tilde{\beta}}A_{\beta}\begin{pmatrix}
					\epsilon   \\
					\epsilon^2 \\
					\epsilon^3 \\
					\vdots     \\
					\epsilon^m \\
				\end{pmatrix} = 0
				&\implies \left< i^{\text{th}} \text{ row of } A^{-1}_{\tilde{\beta}}A_{\beta},\ \vec{\epsilon} \right> = 0\\
				&\implies i^{\text{th}} \text{ row of } A^{-1}_{\tilde{\beta}}A_{\beta} = \vec{0}\ \conta
			\end{split}
		\]
		because \(A^{-1}_{\tilde{\beta}}A_{\beta}\) is invertible where \(A^{-1}_{\beta}A_{\tilde{\beta}}\) is its inverse.
	\end{explanation}
\end{answer}