\lecture{2}{1 Sep. 08:00}{Duality}
\section{First Glance of Duality}
By looking at the \hyperref[def:standard-form]{standard form}, another natural \hyperref[def:general-linear-programming-problem]{linear programming problem} called the \hyperref[def:dual]{dual} of the original problem arises.
\begin{definition}
	Given a \hyperref[def:standard-form]{standard form} \hyperref[def:general-linear-programming-problem]{linear programming problem} called \hyperref[def:primal]{primal} \((\mathrm{P})\), the \hyperref[def:dual]{dual} \((\mathrm{D})\) of \((\mathrm{P})\) is the following induced problem.
	\[
		\begin{aligned}
			\min~             & c^{\top}x \\
			                  & Ax = b    \\
			(\mathrm{P})\quad & x\geq  0,
		\end{aligned}\quad \begin{aligned}
			\max ~            & y^{\top}b               \\
			                  & y^{\top}A\leq c^{\top}. \\
			(\mathrm{D})\quad &
		\end{aligned}
	\]
	\begin{definition}[Primal]\label{def:primal}
		The problem \((\mathrm{P})\) is the \emph{primal}.
	\end{definition}

	\begin{definition}[Dual]\label{def:dual}
		The problem \(D\) is the \emph{dual} of the \hyperref[def:primal]{primal} \((\mathrm{P})\).
	\end{definition}
\end{definition}

\begin{note}
	The \hyperref[def:dual]{dual} is equivalent to
	\[
		\begin{aligned}
			\max~ & b^{\top}y         \\
			      & A^{\top}y \leq c.
		\end{aligned}
	\]
\end{note}

Then we have a direct, but important theorem.
\begin{theorem}[Weak duality theorem]\label{thm:weak-duality}
	If \(\hat{x}\) is \hyperref[def:feasible-solution]{feasible} for \((\mathrm{P})\) and \(\hat{y}\) is \hyperref[def:feasible-solution]{feasible} for \((\mathrm{D})\), then
	\[
		c^{\top}\hat{x} \geq  \hat{y}^{\top} b.
	\]
\end{theorem}
\begin{proof}
	Since we have
	\[
		\hat{y}^{\top}A\leq c^{\top} \underset{\hat{x}\geq 0}{\implies} \hat{y}^{\top}A \hat{x} \leq \hat{c}^{\top} \hat{x} \underset{A \hat{x} = b}{\implies} \hat{y}^{\top}b \leq c^{\top} \hat{x},
	\]
	the result follows.
\end{proof}

\begin{problem*}
	Consider
	\[
		\begin{aligned}
			\min~ & c^{\top}x  \\
			      & Ax \geq b,
		\end{aligned}
	\]
	turn this into the \hyperref[def:standard-form]{standard form} problem and find the \hyperref[def:dual]{dual}.
\end{problem*}
\begin{answer}
	We see that \(x\) is unrestricted. We first minus a \hyperref[def:surplus-variable]{surplus variable} \(s\), we have
	\[
		\begin{aligned}
			\min~ & c^{\top}x \\
			      & Ax - S= b \\
			      & s \geq 0.
		\end{aligned}
	\]
	Now, we turn \(x\) into \(x^+ - x^-\), namely
	\[
		x = \begin{pmatrix}
			x_1    \\
			\vdots \\
			x_n    \\
		\end{pmatrix},\qquad
		x^+ \coloneqq \begin{pmatrix}
			x^+_1  \\
			\vdots \\
			x^+_n  \\
		\end{pmatrix},\qquad
		x^- \coloneqq \begin{pmatrix}
			x^-_1  \\
			\vdots \\
			x^-_n  \\
		\end{pmatrix},
	\]
	with \(x^\pm \geq \vec{0}\). Then the original problem becomes
	\[
		\begin{aligned}
			\min~ & c^{\top}(x^+ - x^-)  \\
			      & A(x^+ - x^-) - s = b \\
			      & x^+, x^-, s \geq 0,
		\end{aligned}
	\]
	equivalently,
	\[
		\begin{aligned}
			\min~ & \begin{pmatrix}
				        c^{\top} & -c^{\top} & 0 \\
			        \end{pmatrix}\begin{pmatrix}
				                     x^+ \\
				                     x^- \\
				                     s   \\
			                     \end{pmatrix}    \\
			      & \begin{pmatrix}
				        A & -A & -I \\
			        \end{pmatrix}\begin{pmatrix}
				                     x^+ \\
				                     x^- \\
				                     s   \\
			                     \end{pmatrix} = b \\
			      & \begin{pmatrix}
				        x^+ \\
				        x^- \\
				        s   \\
			        \end{pmatrix}\geq 0.
		\end{aligned}
	\]
	Set the \hyperref[def:dual]{dual} variable being \(y\), we further have
	\[
		\begin{aligned}
			\max~ & y^{\top}b                                                   \\
			      & y^{\top} \begin{pmatrix}
				                 A & -A & -I \\
			                 \end{pmatrix} \leq \begin{pmatrix}
				                                    c^{\top} & -c^{\top} & 0^{\top}
			                                    \end{pmatrix}.
		\end{aligned}
	\]
\end{answer}

\begin{exercise}
	Show that the \hyperref[def:dual]{dual} of the \hyperref[def:dual]{dual} is the \hyperref[def:primal]{primal}.
\end{exercise}