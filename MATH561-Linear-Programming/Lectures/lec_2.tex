\lecture{2}{1 Sep. 08:00}{Duality}
\section{First Glance of Duality}
We can associate the \hyperref[def:standard-form]{standard form} problem with another linear programming problem, called the \hyperref[def:dual]{dual} of the original problem.
\begin{definition}[Primal and Dual]\label{def:primal-and-dual}
	Given a \hyperref[def:standard-form]{standard form} linear programming problem denoted as \((P)\), we have the induced problem denoted as \((D)\) as follows.
	\[
		\begin{alignedat}{5}
			\min~&c^{\top}x\qquad\qquad &&\max ~ &&y^{\top}b\\
			&Ax = b && &&y^{\top}A\leq c^{\top}\\
			(P)\quad&x\geq  0 &&(D)\quad&&
		\end{alignedat}.
	\]
	\begin{definition}[Primal]\label{def:primal}
		We sometimes called \((P)\) the \emph{primal}.
	\end{definition}

	\begin{definition}[Dual]\label{def:dual}
		The \emph{dual} of the \hyperref[def:primal]{primal} is the problem \((D)\).
	\end{definition}
\end{definition}


\begin{note}
	We see that the \hyperref[def:dual]{dual} is equivalent to
	\begin{align*}
		\max~ & b^{\top}y         \\
		      & A^{\top}y \leq c.
	\end{align*}
\end{note}

Then we have a direct, but important theorem.
\begin{theorem}[Weak duality theorem]\label{thm:weak-duality-theorem}
	If \(\hat{x}\) is \hyperref[def:feasible-solution]{feasible} for \((P)\), and \(\hat{y}\) is \hyperref[def:feasible-solution]{feasible} for \((D)\), then we have
	\[
		c^{\top}\hat{x} \geq  \hat{y}^{\top} b.
	\]
\end{theorem}
\begin{proof}
	Since we have
	\[
		\hat{y}^{\top}A\leq c^{\top} \underset{\hat{x}\geq 0}{\implies} \hat{y}^{\top}A \hat{x} \leq \hat{c}^{\top} \hat{x} \underset{A \hat{x} = b}{\implies} \hat{y}^{\top}b \leq c^{\top} \hat{x},
	\]
	the result follows.
\end{proof}

\begin{eg}
	Consider
	\begin{align*}
		\min~ & c^{\top}x  \\
		      & Ax \geq b,
	\end{align*}
	turn this into the \hyperref[def:standard-form]{standard form} problem and find the \hyperref[def:dual]{dual}.
\end{eg}
\begin{explanation}
	We see that \(x\) is unrestricted. We first minus a \hyperref[def:surplus-variable]{surplus variable} \(S\), we have
	\begin{align*}
		\min~ & c^{\top}x \\
		      & Ax - S= b \\
		      & S \geq 0.
	\end{align*}
	Now, we turn \(x\) into \(x^+ - x^-\), namely
	\[
		x = \begin{pmatrix}
			x_1    \\
			\vdots \\
			x_n    \\
		\end{pmatrix},\qquad
		x^+ \coloneqq \begin{pmatrix}
			x^+_1  \\
			\vdots \\
			x^+_n  \\
		\end{pmatrix},\qquad
		x^- \coloneqq \begin{pmatrix}
			x^-_1  \\
			\vdots \\
			x^-_n  \\
		\end{pmatrix}, x^\pm \geq \vec{0}.
	\]
	Then we see the original problem becomes
	\begin{align*}
		\min~ & c^{\top}(x^+ - x^-)  \\
		      & A(x^+ - x^-) - S = b \\
		      & x^+, x^-, S \geq 0
	\end{align*}

	We can further have
	\begin{align*}
		\min~ & \begin{pmatrix}
			        c^{\top} & -c^{\top} & 0 \\
		        \end{pmatrix}\begin{pmatrix}
			                     x^+ \\
			                     x^- \\
			                     S   \\
		                     \end{pmatrix}    \\
		      & \begin{pmatrix}
			        A & -A & -I \\
		        \end{pmatrix}\begin{pmatrix}
			                     x^+ \\
			                     x^- \\
			                     S   \\
		                     \end{pmatrix} = b \\
		      & \begin{pmatrix}
			        x^+ \\
			        x^- \\
			        S   \\
		        \end{pmatrix}\geq 0.
	\end{align*}
	Set the \hyperref[def:dual]{dual} variable being \(y\), we further have
	\begin{align*}
		\max~ & y^{\top}b                                                    \\
		      & y^{\top} \begin{pmatrix}
			                 A & -A & -I \\
		                 \end{pmatrix} \leq \begin{pmatrix}
			                                    c^{\top} & -c^{\top} & 0^{\top}.
		                                    \end{pmatrix}
	\end{align*}
\end{explanation}

\begin{note}
	The \hyperref[def:dual]{dual} of the \hyperref[def:dual]{dual} is the \hyperref[def:primal]{primal}.
\end{note}
\begin{exercise}
	Show the above assertion.
\end{exercise}