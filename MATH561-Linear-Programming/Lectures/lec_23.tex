\lecture{23}{01 Dec. 08:00}{Branch and Bound Algorithm}
\subsection{Branch and Bound Algorithm}\label{subsec:branch-and-bound}
We first dive into \hyperref[eg:branch-and-bound]{branch and bound algorithm} as we mentioned.
\begin{prev}
	The worst case in terms of time complexity for \hyperref[algo:simplex-algorithm]{simplex algorithm} is
	\[
		\Theta(2^n - 1)
	\]
	for \(n\) variables, but it's efficient in practice. And this is similar to the \hyperref[eg:branch-and-bound]{branch and bound algorithm} for the integral programming problem.
\end{prev}

We now focus on the following integer programming,
\[
	\begin{aligned}
		\max~                            & y^{\top}b(\eqqcolon z)                                              \\
		                                 & y^{\top}A\leq c^{\top}                                              \\
		(\mathrm{D}_{\mathcal{I} })\quad & y\in\mathbb{R}^m(y_{i}\in\mathbb{Z} \text{ for } i\in \mathcal{I}),
	\end{aligned}
\]
where \(\mathcal{I} \subseteq \{1, 2, \dots , m\}\). By taking the \hyperref[def:dual]{dual}, we have
\[
	\begin{aligned}
		\min~             & c^{\top}x \\
		                  & Ax = b    \\
		(\mathrm{P})\quad & x\geq 0.
	\end{aligned}
\]

We'll see that the branch and bound algorithm maintains the following:
\begin{itemize}
	\item \(\mathcal{L} \): A list \(\mathcal{L} \) of \emph{subproblems} that have the form of \((\mathrm{D}_{\mathcal{I}^\prime})\) where \(\mathcal{I} ^\prime \subset \mathcal{I}\)
	\item \(\mathrm{LB} \): The current best lower bound on \(z\) such that \(\mathrm{LB}\leq z \).
	\item \(\overline{y}_{\mathrm{LB}}\): The \(\overline{y}\) corresponds to \(\mathrm{LB}\).
\end{itemize}

\begin{note}
	\(\mathrm{LB} \) is the objective value of the best \hyperref[def:feasible-solution]{feasible solution} to the original problem seen so far. And we'll set
	\[
		\mathrm{LB} \coloneqq  -\infty
	\]
	if there is no known \hyperref[def:feasible-solution]{feasible solution}.
\end{note}
\begin{remark}
	The key property of \(\mathcal{L} \) is that if there is a \hyperref[def:feasible-solution]{feasible solution} to the original problem that is better than \(\mathrm{LB} \), it should be \hyperref[def:feasible-solution]{feasible} for some sub-problems on \(\mathcal{L} \). Initially, we have
	\[
		\mathcal{L} \coloneqq \left\{\mathrm{D} _{\mathcal{I}^\prime }\right\}.
	\]
	And we stop if
	\[
		\mathcal{L} = \varnothing,
	\]
	since this implies \(z = \mathrm{LB} \).
\end{remark}

The general procedure is to take some problem \((\widetilde{\mathrm{D}}_{\mathcal{I}^\prime })\) from \(\mathcal{L}\) and remove it, and then solve its continuous relaxation \((\widetilde{\mathrm{D}})\) and proceed. Rigorously, we have the following pseudocode.

\begin{algorithm}[H]\label{algo:branch-and-bound-algorithm}
	\DontPrintSemicolon
	\caption{Branch and Bound Algorithm}
	\SetKwData{result}{result}
	\KwData{(Mixed) Integer programming \((\mathrm{D}_{\mathcal{I} } )\) with \emph{feasible} \((\mathrm{P})\)\footnote{This makes sure that the \hyperref[def:feasible-region]{feasible region} of the continuous relaxation \((\widetilde{\mathrm{D}})\) of any \((\widetilde{\mathrm{D}} _{\mathcal{I} ^\prime })\) is a bounded set, so we can guarantee finite termination.}}
	\KwResult{\hyperref[def:optimal-solution]{optimal solution} \(\overline{y}\) and \hyperref[def:optimal-solution]{optimal} value, or report \((\mathrm{D}_{\mathcal{I} } )\) is infeasible}

	\SetKwFunction{help}{SolveContRelax}

	\BlankLine
	\(\mathcal{L} \gets \{\mathrm{D} _{\mathcal{I}^\prime } \colon \mathcal{I}^\prime  \subseteq \mathcal{I}  \}\)\;
	\(\mathrm{LB} \gets -\infty \)\;
	\(\overline{y} _{\mathrm{LB}}\gets\) random vector\;
	\;
	\While{\(\mathcal{L} \neq \varnothing \)}{
	\(\mathrm{D}_{\mathcal{I}^\prime} \gets \) an element in \(\mathcal{L} \)\Comment*[r]{See \autoref{subsec:node-selection}}
	\(\mathcal{L} \gets \mathcal{L} \setminus \{\mathrm{D}_{\mathcal{I} ^\prime }\}\)\;
	\(\result\gets\)\help{\(\mathrm{D} _{\mathcal{I}^\prime }\)}\;\label{algo:branch-and-bound-algorithm:line8}
	\If{\(\result\) is not infeasible}{
	\(\overline{y} \gets \result\)\Comment*[r]{Retrieve \hyperref[def:basic-solution]{basic} \hyperref[def:optimal-solution]{optimal solution}}
	\If{\(\overline{y} ^{\top} b > \mathrm{LB}\)}{
	\uIf(\Comment*[f]{\(\overline{y}\) solves \(\mathrm{D} _\mathcal{I}\)}){\(\overline{y} _{i} ^{\top} \in \mathbb{Z} \) for \(i\in \mathcal{I}\)}{
		\(\mathrm{LB}\gets \overline{y} ^{\top} b \)\Comment*[r]{Record \hyperref[def:optimal-solution]{optimal} value}
		\(\overline{y} _\mathrm{LB} \gets \overline{y} \)\Comment*[r]{Record \hyperref[def:optimal-solution]{optimal solution}}
	}\Else(\Comment*[f]{\(\overline{y}\) doesn't solve \(\mathrm{D}_\mathcal{I}\), \(\overline{y} _{i} ^{\top} \notin \mathbb{Z} \) for some \(i\in \mathcal{I}\)}){
	\(i^{\ast} \gets\) an \(i\in \mathcal{I} \) such that \(\overline{y} _{i} \notin \mathbb{Z} \)\label{algo:branch-and-bound-algorithm:line16}\Comment*[r]{See \autoref{subsec:branching-variable-selection}}
	\(\mathrm{D} ^\text{u}_{\mathcal{I} ^\prime }\gets \mathrm{D}_{\mathcal{I}^\prime } \) with \(y_{i^{\ast} }\geq \left\lceil \overline{y}_{i^{\ast} } \right\rceil\)\Comment*[r]{\hyperref[rmk:up-branch]{Up branch}}
	\(\mathrm{D}^\text{d}_{\mathcal{I} ^\prime }\gets \mathrm{D}_{\mathcal{I}^\prime } \) with \(y_{i^{\ast} }\leq \left\lfloor \overline{y}_{i^{\ast} } \right\rfloor\)\footnote{To match the form, we use \(-y_{i}\leq -\left\lceil \overline{y}_{i}\right\rceil\).}\Comment*[r]{\hyperref[rmk:down-branch]{Down branch}}
	\(\mathcal{L} \gets \mathcal{L} \cup \{\mathrm{D}^\text{u}_{\mathcal{I} ^\prime }, \mathrm{D}^\text{d}_{\mathcal{I} ^\prime }\}\)\;
	}
	}
	}
	}
	\;
	\uIf(){\(\mathrm{LB} = -\infty\) }{
		\Return{infeasible}\;
	}\Else{
		\Return{\(\overline{y} _\mathrm{LB} \), \(\mathrm{LB}\)}\;
	}
\end{algorithm}

\begin{remark}[Finite termination]
	We see that there are only finitely many (though exponential) \(\mathrm{D}_{\mathcal{I} ^\prime}\) can be added into \(\mathcal{L} \), hence \hyperref[algo:branch-and-bound-algorithm]{branch and bound algorithm} is finitely terminating.
\end{remark}

Because the above algorithm maintains the key invariant for \hyperref[algo:branch-and-bound-algorithm]{branch and bound}, i.e., every \hyperref[def:feasible-solution]{feasible solution} of \((\mathrm{D}_{\mathcal{I} } )\) with greater objective value than \(\mathrm{LB}\) is \hyperref[def:feasible-solution]{feasible} for a problem on the list \(\mathcal{L} \) we have the following result.
\begin{theorem}\label{thm:lec23-1}
	Suppose that \((\mathrm{P})\) is feasible. Then at termination of \hyperref[algo:branch-and-bound-algorithm]{branch and bound}, we have \(\mathrm{LB}=-\infty\) if \((\widetilde{\mathrm{D}} _\mathcal{I})\) is infeasible, or with \(\overline{y} _{\mathrm{LB}}\) being an \hyperref[def:optimal-solution]{optimal solution} of \((\mathrm{D}_{\mathcal{I} } )\).
\end{theorem}

\begin{remark}[Detail of the algorithm]
	We make some remarks on \hyperref[algo:branch-and-bound-algorithm]{branch and bound algorithm}.
	\begin{itemize}
		\item When calling \autoref{algo:branch-and-bound-algorithm:line8}, we first obtain the continuous relaxation \((\widetilde{\mathrm{D}})\) of \((\mathrm{D}_{\mathcal{I} ^\prime })\) and its \hyperref[def:primal]{primal} \((\widetilde{\mathrm{P}} )\) as follows.
		      \[
			      \begin{alignedat}{5}
				      \max ~&y^{\top}b\qquad\qquad	&&\min~		&&c^{\top}x\\
				      &y^{\top}A\leq c^{\top} 		&&			&&Ax = b\\
				      (\widetilde{\mathrm{D}})\quad	& 				&&(\widetilde{\mathrm{P}})\quad	&&x\geq  0
			      \end{alignedat}.
		      \]
		      Then what we're really doing is solving \((\widetilde{\mathrm{P}})\) instead of \((\widetilde{\mathrm{D}} )\) by \hyperref[algo:simplex-algorithm]{simplex algorithm} and get an \hyperref[def:optimal-solution]{optimal} basis \(\beta\), and this give us an \hyperref[def:optimal-solution]{optimal} \hyperref[def:dual-basic-solution]{dual solution} \(\overline{y}^{\top}\coloneqq c^{\top}_{\beta}A^{-1} _{\beta}\).
		\item When calling \autoref{algo:branch-and-bound-algorithm:line16}, after choosing \(i^{\ast} \), adding a constraint to \((\mathrm{D}_{\mathcal{I} ^\prime })\) effectively adds a variable to the corresponding continuous relaxation \((\widetilde{\mathrm{D}})\), hence adds a variable to the \hyperref[def:standard-form]{standard form} problem \((\widetilde{\mathrm{P}} )\). So, a \hyperref[def:basis]{basis} for \((\widetilde{\mathrm{P}})\) remains \hyperref[def:feasible-solution]{feasible} after we introduce such a variable.
		      \begin{itemize}
			      \item\label{rmk:down-branch} Down branch: The constraint \(y_{i^{\ast}} \leq \left\lfloor \overline{y}_{i^{\ast}} \right\rfloor\) dualize to a new variable \(x_{\text{down}}\) in \((\widetilde{\mathrm{P}})\), which has a new column \(A_{\text{down}}\coloneqq e_{i^{\ast}}\) and a cost coefficient \(c_{\text{down}}\coloneqq \left\lfloor \overline{y} _{i^{\ast}}\right\rfloor \).
			      \[
				      \begin{alignedat}{5}
					      \max ~	&y^{\top}b\qquad\qquad				&&\min~	&&c^{\top}x+\left\lfloor \overline{y}_{i^{\ast}} \right\rfloor x_{\text{down}}\\
					      &y^{\top}A\leq c^{\top} 				&&		&&Ax + e_{i^{\ast}}x_{\text{down}} = b\\
					      (\widetilde{\mathrm{D}})\quad& y_{i^{\ast}}\leq \left\lfloor \overline{y}_{i^{\ast}} \right\rfloor	&&(\widetilde{\mathrm{P}})\quad&&x\geq 0, x_{\text{down}}\geq 0.
				      \end{alignedat}
			      \]
			      The \hyperref[def:reduced-cost]{reduced cost} of \(x_{\text{down}}\) is
			      \[
				      \overline{c}_{\text{down}}=c_{\text{down}}-\overline{y}^{\top}A_{\text{down}} = \left\lfloor \overline{y}_{i^{\ast}} \right\rfloor - \overline{y}^{\top}e_{i^{\ast}} = \left\lfloor \overline{y}_{i^{\ast}} \right\rfloor - \overline{y}_{i^{\ast}} < 0
			      \]
			      since \(\overline{y} _{i^{\ast} }\) is not an integer. Hence, \(x_{\text{down}}\) is eligible to enter the \hyperref[def:basis]{basis}.
			      \item\label{rmk:up-branch} Up branch: Similarly, we have a new variable \(x_{\text{up}}\) in \((\widetilde{\mathrm{P}})\) for the new constraint \(y_{i^{\ast}} \geq \left\lceil \overline{y} _{i^{\ast}} \right\rceil \).
			      \[
				      \begin{alignedat}{5}
					      \max ~	&y^{\top}b\qquad\qquad				&&\min~	&& c^{\top}x - \left\lceil \overline{y}_{i^{\ast}} \right\rceil x_{\text{up}}\\
					      &y^{\top}A\leq c^{\top} 				&&		&&Ax - e_{i^{\ast}} x_{\text{up}}= b                                      \\
					      (\widetilde{\mathrm{D}})\quad& y_{i^{\ast}} \geq \left\lceil \overline{y} _{i^{\ast}} \right\rceil	&&(\widetilde{\mathrm{P}})\quad&&x\geq 0, x_{\text{up}}\geq 0.
				      \end{alignedat}
			      \]

			      The \hyperref[def:reduced-cost]{reduced cost} of \(x_{\text{up}}\) is
			      \[
				      -\left\lceil \overline{y}_{i^{\ast}} \right\rceil - \overline{y}^{\top}(-e_{i^{\ast}}) = \overline{y}_{i^{\ast}} - \left\lceil \overline{y}_{i^{\ast}} \right\rceil < 0,
			      \]
			      since \(\overline{y} _{i^{\ast} }\) is not an integer. Hence, \(x_{\text{up}}\) is eligible to enter the \hyperref[def:basis]{basis}.
		      \end{itemize}
	\end{itemize}
\end{remark}

\begin{remark}[Partially solving \(\widetilde{\mathrm{P}}\)]
	In practice, when solving \((\widetilde{\mathrm{P}})\) (induced from any \((\mathrm{D}_{\mathcal{I} ^\prime})\)) via \hyperref[algo:simplex-algorithm]{simplex algorithm}, we are generating a sequence of \emph{decreasing objective values} of \((\widetilde{\mathrm{P}})\), each one of which is an upper-bound on the \hyperref[def:optimal-solution]{optimal} value of its parent, and it's also a \emph{potential new \(\mathrm{LB}\)}. We see that when the \hyperref[def:optimal-solution]{optimal} value of \((\widetilde{\mathrm{P}})\) \(\leq \mathrm{LB}\), we terminate immediately since solving this \((\widetilde{\mathrm{P}})\) will not improve \(\mathrm{LB}\).
\end{remark}

\subsection{Global Upper Bound}
Since in practice, there are many errors in the data, so we may just want to solve it approximately, which means we only want to get a global upper bound. Conceptually,
\[
	\mathrm{UB} \coloneqq \max\left\{\mathrm{LB}, \max\left\{\text{\(\mathrm{LP}\) relaxation values for all problems on \(\mathcal{L}\) }\right\}\right\}
\]
To calculate the set in the \(\max\), whenever children are created, solve their LP relaxation upon insertion into list. And we stop if
\[
	\mathrm{UB} - \mathrm{LB} < \text{absolute tolerance}.
\]

\begin{remark}
	Apparently, we see that we can do this by reordering the algorithm. But for the \hyperref[algo:branch-and-bound-algorithm]{original algorithm}, we don't care about \(\mathrm{UB}\).
\end{remark}

\subsection{Node Selection}\label{subsec:node-selection}
Node Selection means which problem to select from \(\mathcal{L} \) to process. There are several ways to do this.

\begin{enumerate}
	\item FIFO (First In First Out) \(\cong\) \href{https://en.wikipedia.org/wiki/Breadth-first_search}{Breadth First Search} (BFS). New problems go at the end of the list, select from the front. We see that this strategy will \textbf{maximize memory usage}.
	\item LIFO (Last In First Out) \(\cong\) \href{https://en.wikipedia.org/wiki/Depth-first_search}{Depth First Search} (DFS). New problems go to the first of the list, select from the front. We see that this strategy will \textbf{increase \(\mathrm{LB}\) quickly}.
	\item Best Bound. Need the LP upper bound for all problems on the list. We see that this strategy will \textbf{decrease \(\mathrm{UB}\) quickly}.
\end{enumerate}

\begin{remark}
	For any reasonable solver, it will first do the second strategy for several times, and they exclusively do the third strategy.
\end{remark}

\subsection{Branching Variable selection}\label{subsec:branching-variable-selection}
\begin{enumerate}
	\item Random: Choose randomly among \(y_{i}\) such that \(\overline{y}_{i}\notin\mathbb{Z}\).
	\item Biggest Cost: Choose based on the biggest \(c_{i}\).
	\item Most Fractional: Choose \(i\) with \(\overline{y}_{i}\) \emph{most fractional}.
	\item \textbf{Pseudo Cost Branching}.
\end{enumerate}

\begin{note}
	Someone argues that the \emph{most fractional} rules is as bad as choosing randomly.
\end{note}
