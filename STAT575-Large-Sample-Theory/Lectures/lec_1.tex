\chapter{Introduction}
\lecture{1}{16 Jan.\ 9:30}{Introduction to Large Sample Theory}
Say we first collect \(n\) data points \(x_1, \dots , x_n \in \mathbb{R} ^d\), where we may treat \(x_i\) as a realization of a random vector \(X_i\) on a probability space \((\Omega , \mathscr{F} , \mathbb{P} )\). In this course, we will primarily consider the case that \(X_i\)'s are i.i.d., i.e., independent and identically distributed from a distribution function, or the \emph{cumulative density function} (cdf) \(F\) such that
\[
	X = (X^1, \dots , X^d) \sim F(x_1, \dots , x_d) \equiv \mathbb{P} (X^1 \leq x_1, \dots , X^d \leq x_d )
\]
for all \(x_i \in \mathbb{R} \). If we have access to \(F\), we can compute the corresponding \emph{probability density function} (pdf) \(f\), and then have access to \(\mathbb{P} (X \in A)\) for all (measurable) \(A \subseteq \mathbb{R} ^d\) of interest.

\begin{notation}
	In the measure-theoretic sense, the measure \(\mathbb{P} \) in \((\Omega , \mathscr{F} , \mathbb{P} )\) is the \href{https://en.wikipedia.org/wiki/Lebesgue-Stieltjes_integration}{Lebesgue-Stieltjes measure} \(\mu _F\) induced by the distribution function \(F\). When doing integration, we will often denote
	\[
		\mathrm{d} \mu _F (x) = \mathrm{d} \mathbb{P}(x) \eqqcolon F(\mathrm{d} x) \eqqcolon \mathrm{d} F(x) \eqqcolon f(x) \mathrm{d} x
	\]
\end{notation}

\begin{remark}
	If we know any of the above, we know every thing about the population.
\end{remark}

Hence, the goal is to compute this by collecting data \(x_i\)'s, which is a statistical inference problem. Notably, \emph{large sample theory} concerns with the limiting theory as \(n \to \infty \).

\section{Parametrized Approach}
There are various ways of doing this task, one way is the so-called parametrized approach. By postulating a family of cdfs \(\{ F_\theta , \theta \in \Theta \} \) where \(\Theta \) is often a subset of \(\mathbb{R} ^m\) for some \(m\) (generally \(\neq n\)), the goal is to select a member of this family that is the ``closet'', or the ``best fit'' to the truth, i.e., \(F\), based on the data.

\begin{note}
	To emphasize that this depends on the data, we sometimes write the function we found as \(\hat{\theta} _n(x_1, \dots  , x_n)\) so that \(F_{\hat{\theta} _n(x_1, \dots , x_n)} \) is our proxy for \(F\).
\end{note}

Now, assume that the family is initially given, the problem is then how to select \(\hat{\theta} _n\).

\begin{eg}
	Fisher suggested that we should look at the maximum likelihood estimator (MLE).
\end{eg}

The justification for MLE is not about finite \(n\), but about its asymptotic behavior when \(n \to \infty \). Specifically, we have the following theorem due to Fisher (informally stated).

\begin{theorem}[Fisher]
	If \(F \in \{ F_\theta \colon \theta \in \Theta \} \), i.e., if \(F = F_{\theta ^{\ast} }\) for some \(\theta ^{\ast} \in \Theta \), then under certain conditions, \(\hat{\theta} _n\) will be ``close'' to \(\theta ^{\ast} \) as \(n \to \infty \). Under some other conditions, \(\sqrt{n}  (\hat{\theta} _n - \theta )\) is approximately Gaussian with variance being the ``best possible'' in some sense.
\end{theorem}

On the other hand, in the misspecified case, i.e., \(F \notin \{ F_{\theta } , \theta \in \Theta \} \), we can still compute the MLE, which leads to another justification for MLE since even in this case, \(\hat{\theta} _n\) will still be ``close'' to \(\theta ^{\ast} \) such that \(F_{\theta ^{\ast} }\) is, in some sense, the ``closest'' to \(F\) among all possible \(F_\theta \) (minimizing divergence, to be precise).

\section{Hypothesis Testing}
We will also develop theory for hypothesis testing for some hypothesis we're interested in, e.g., whether the data we collect is really i.i.d., or whether our proposed family is reasonable enough. Say now \(X_i\)'s are scalar random variable with \(\mathbb{E}_{}\left[X \right] = \mu \), and we want to test the null hypothesis \(H_0 \colon \mu = 0\).

\begin{eg}
	Consider a controlled group \(Z\) and a treatment group \(Y\), and we observe \(Z_1, \dots , Z_n\), and \(Y_1, \dots , Y_n\), respectively, and compute \(X_i = Z_i - Y_i\) for all \(i\). Testing \(H_0\) on the distribution of \(X\) will show the effect of the treatment.
\end{eg}

To do this, a well-known, elementary, and fundamental method is the so-called \(t\)-test.

\begin{definition}[\(t\)-statistic]\label{def:t-statistic}
	Given a sample \(X_1, \dots , X_n\), the \emph{\(t\)-statistic} is defined as
	\[
		T_n = \frac{\overline{X} _n}{s_n / \sqrt{n} },
	\]
	where \(s_n\) is the sample standard derivation.
\end{definition}

\begin{note}
	As long as \(X\) is Gaussian, \(T_n \sim t_{n-1}\), i.e., the \(t\)-distribution with \(n-1\) degrees of freedom.
\end{note}

Hence, one can reject \(H_0\) when \(T_n\) is too large (or small) when \(X \sim \mathcal{N} \) as we know the exact distribution \(T_n\) will follow. What if \(X\) is not an Gaussian? We will show that even if \(X\) is not Gaussian, this result is ``approximately valid'' when \(n\) is ``large enough'' as long as \(\Var_{}\left[X \right] < \infty \).

\begin{remark}[Sample Size]
	When we say \(n\) is ``large enough'', what we mean really depends on how fast the underlying distribution will approach Gaussian as \(n\) grows. Hence, if we can say more about the underlying population, we can say more about when does \(n\) is ``large enough''; otherwise such a limiting theory might be completely useless in practice.
\end{remark}

What if \(\Var_{}\left[X \right] \) doesn't exit for some heavy tailed distribution like the Cauchy?

\begin{eg}[Cauchy distribution]
	Cauchy distribution doesn't have finite moment of order greater than \(1\).
\end{eg}

In this case, other tests are needed. A simple test would be looking at the sign of \(X_i\).

\begin{eg}[Sign test]
	We might reject \(H_0\) if \(\sum_{i=1}^{n} \mathbbm{1}_{X_i > 0} \) is large. Note that under \(H_0\), \(\sum_{i=1}^{n} \mathbbm{1}_{X_i > 0} \sim \operatorname{Bin}(n, 1/2) \), and this test is valid even if expectation doesn't exist.
\end{eg}

We see that without saying anything about \(F\), the sign test is valid even for \(n = 3\) or \(5\) as the sum is exactly binomial distribution under \(H_0\). Although simple and have good property, only looking at the sign of \(X_i\) might be too weak. A natural idea is to look at the absolute value of \(X_i\).

\begin{eg}[Wilcoxon's rank-sum test]
	Let \(R_{i, n} \) to be the rank of \(\vert X_i \vert \), then consider the so-called \emph{Wilcoxon's rank-sum test}
	\[
		\sum_{i=1}^{n} \mathbbm{1}_{X_i > 0} R_{i, n}.
	\]
	As one can imagine, the closed form of the above sum will be complicated; however, asymptotically, the above statics will follow Gaussian again, such that the rate of convergence doesn't depend on the underlying population.
\end{eg}

Finally, we also ask how can we compare these different tests? This will also be addressed in this course.