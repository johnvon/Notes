\lecture{18}{26 Mar.\ 9:30}{Wilcoxon Signed-Rank Test}
We will now focus on explaining the previous answer to the non i.i.d.\ problem. However, before we dive in, let's point out other potential problems. Firstly, since even for i.i.d.\ \(X_i\)'s, the results for \(T_n\) is asymptotic, and we might wonder what's the rate of convergence.

\begin{problem*}
	How fast does \(T_n \overset{D}{\to} \mathcal{N} (0, 1)\) under \(H_0\)?
\end{problem*}

Clearly, the answer to this problem depends heavily on \(F\), so it's unlikely to get a universal answer. On the other hand, what if the underlying distribution has heavy tails?

\begin{problem*}
	What if \(\Var_{}[X_i] \), or even \(\mathbb{E}_{}[X_i] \), doesn't exist?
\end{problem*}

It turns out that this is something we will solve along the way when we deal with the i.i.d.\ problem.

\subsection{Wilcoxon Signed-Rank Test}
We now see how to solve the non i.i.d.\ problem. Consider the following \(H_0\) such that for every \(1 \leq i \leq n\),
\[
	\mathbb{P} (X_i = \pm \vert X_i \vert \mid \vert X_j \vert , 1 \leq j \leq n) = \frac{1}{2}.
\]
In other words, \(\mathbb{P} (\sgn (X_i) = \pm 1 \mid \vert X_j \vert , 1 \leq j \leq n) = 1 / 2\).

\begin{intuition}
	In practice, when doing inference, we usually implicitly condition on \(\vert X_j \vert \)'s too: for example, by doing inference on the selected ``design points'', we're essentially condition on those.
\end{intuition}

By writing \(\sum_{i=1}^{n} X_i = \sum_{i=1}^{n} \vert X_i \vert \sgn (X_i)\) and treating \(\vert X_i \vert \)'s as constants, under \(H_0\),
\[
	\frac{\sum_{i=1}^{n} \vert X_i \vert \sgn (X_i)}{\sqrt{\sum_{i=1}^{n} X_i^2} }
	\overset{D}{\to} \mathcal{N} (0, 1)
\]
by the \hyperref[thm:Lindeberg-CLT]{Lindeberg central limit theorem}. We can then use the left-hand side for out test as usual.

\begin{problem*}
	Is it possible to avoid using the information of \(\vert X_j \vert \)'s?
\end{problem*}
\begin{answer}
	Inspired by the above, we may reject \(H_0\) if \(\sum_{i=1}^{n} \sgn (X_i) \) is large. In particular, under \(H_0\),
	\[
		\sum_{i=1}^{n} \sgn (X_i)
		= 2 \sum_{i=1}^{n} \mathbbm{1}_{X_i > 0} - n,
	\]
	where we know that \(\sum_{i=1}^{n} \mathbbm{1}_{X_i > 0} \overset{H_0}{\sim} \operatorname{Bin}(n, 1 / 2)\). We see that this is independent of \(F\), and we know its exact distribution, hence we can do the test with this. On the other hand, under \(H_0\),
	\[
		\frac{\sum_{i=1}^{n} \sgn (X_i)}{\sqrt{n} } \overset{D}{\to} \mathcal{N} (0, 1),
	\]
	so we can also reject \(H_0\) when \(\sum_{i=1}^{n} \sgn (x_i) / \sqrt{n} \geq Z_\alpha \).
\end{answer}

However, since we're not using the information of \(\vert X_j \vert \) now, it's questionable that how powerful the above tests are. If we want to use \(\vert X_j \vert \)'s, consider the following.

\begin{intuition}
	Replace \(\vert X_i \vert \)'s by their ``rank'' information.
\end{intuition}

Consider the rank \(R_i\) for \(1 \leq i \leq n\)
\[
	R_i
	\coloneqq \sum_{j=1}^{n} \mathbbm{1}_{\vert X_j \vert \leq \vert X_i \vert }.
\]
It's clear that \(R_i\)'s are dependent to each other.

\begin{prev}
	This is different from the previous definition \(R_i = \sum_{j=1}^{i} \mathbbm{1}_{X_j < X_i} \).
\end{prev}

Following the intuition, consider the so-called \emph{Wilcoxon signed-rank test}, which rejects \(H_0\) if
\[
	W_n = \sum_{i=1}^{n} R_i \sgn (X_i)
\]
is ``large.'' To figure out the critical value, since we condition on \(\{ \vert X_j \vert \}_{j = 1}^n \), without loss of generality, \(R_i = i\) for every \(1 \leq i \leq n\).\footnote{This is doable since \(\vert X_j \vert \)'s are treated as constants.} In this case, \(W_n = \sum_{i=1}^{n} i \sgn (X_i)\), hence under \(H_0\),
\[
	\frac{\sum_{i=1}^{n} i \sgn (X_i)}{\sqrt{\sum_{i=1}^{n} i^2} }
	\overset{D}{\to} \mathcal{N} (0, 1)
\]
by the \hyperref[thm:Lindeberg-CLT]{Lindeberg central limit theorem}. We can then use this to test \(H_0\).

\begin{note}
	This test statistic is again independent of \(F\) (under \(H_0\)).
\end{note}

Without conditioning on \(\vert X_j \vert \)'s, the above doesn't hold. However, a simple sign test can still be done.

\begin{eg}[Sign test]
	If the data is i.i.d., consider \(H_0 \colon \sgn (X_i) = \epsilon _i\) where \(\epsilon _i\) is a Rademacher random variable, i.e., \(\mathbb{P} (X > 0) = \mathbb{P} (X < 0) = 1 / 2\).
\end{eg}

Let's study this simple sign test. Observe that the above doesn't need \(\mathbb{E}_{}[X_i] \) to be finite. But if we further assume \(\Var_{}[X_i] < \infty \), then the Wilcoxon signed-rank test still makes sense. However, since we're not conditioning on \(\vert X_j \vert \)'s, we only have
\[
	W_n = \sum_{i=1}^{n} R_i \sgn (X_i).
\]

\begin{proposition}\label{prop:Wilcoxon-signed-rank-test}
	Let \(h(x_1, x_2) \coloneqq \mathbbm{1}_{x_1 + x_2 \geq 0}\). Then we have
	\[
		W_n - \sum_{i=1}^{n} \sgn (X_i)
		= \sum_{i \neq j} \left( h(X_i , X_j) - \frac{1}{2} \right).
	\]
\end{proposition}
\begin{proof}
	Consider \(W_n \eqqcolon W_n^+ - W_n^-\) where
	\[
		W_n^+ = \sum_{i=1}^{n} R_i \mathbbm{1}_{X_i > 0} ,\text{ and }
		W_n^- = \sum_{i=1}^{n} R_i \mathbbm{1}_{X_i < 0}.
	\]
	Some calculation gives the following.

	\begin{claim}
		\(W_n^+ - \sum_{i=1}^{n} \mathbbm{1}_{X_i > 0} = \frac{1}{2} \sum_{i \neq j} \mathbbm{1}_{X_i + X_j \geq 0}\) and \(W_n^- - \sum_{i=1}^{n} \mathbbm{1}_{X_i < 0} = \frac{1}{2} \sum_{i \neq j} \mathbbm{1}_{X_i + X_j < 0}\).
	\end{claim}
	\begin{explanation}
		Let's first focus on \(W_n^+\). We see that
		\[
			W_n^+
			= \sum_{i=1}^{n} \left( \sum_{j=1}^{n} \mathbbm{1}_{\vert X_j \vert \leq \vert X_i \vert } \right) \mathbbm{1}_{X_i > 0}
			= \sum_{i=1}^{n} \mathbbm{1}_{X_i > 0} + \sum_{\substack{1 \leq i, j \leq n \\ i \neq j}} \mathbbm{1}_{\vert X_j \vert \leq \vert X_i \vert , X_i > 0} ,
		\]
		Let's abbreviate the argument \(1 \leq i, j \leq n \colon i \neq j\) in the double summation by \(i \neq j\), we have
		\[
			\begin{split}
				W_n^+ - \sum_{i=1}^{n} \mathbbm{1}_{X_i > 0}
				 & = \sum_{i \neq j} \mathbbm{1}_{\vert X_j \vert \leq X_i}                                       \\
				 & = \sum_{i \neq j} \mathbbm{1}_{-X_i \leq X_j \leq X_i}                                         \\
				 & = \underbrace{\sum_{i \neq j} \mathbbm{1}_{X_i + X_j \geq 0} \mathbbm{1}_{X_j \leq X_i}}_{S_1}
				= \sum_{i \neq j} \mathbbm{1}_{X_i + X_j \geq 0} - \underbrace{\sum_{i \neq j} \mathbbm{1}_{X_i + X_j \geq 0} \mathbbm{1}_{X_j > X_i}}_{S_2},
			\end{split}
		\]
		where the last equality can be justified by the fact that \(1 = \mathbbm{1}_{X_j \leq X_i} + \mathbbm{1}_{X_j > X_i} \). Observe that
		\[
			S_1
			= \sum_{i \neq j} \mathbbm{1}_{X_i + X_j \geq 0} \mathbbm{1}_{X_j \leq X_i}
			= \sum_{i \neq j} \mathbbm{1}_{X_j + X_i \geq 0} \mathbbm{1}_{X_i \leq X_j}
			= \sum_{i \neq j} \mathbbm{1}_{X_i + X_j \geq 0} \mathbbm{1}_{X_j > X_i}
			= S_2,
		\]
		as we assume that \(F\) is continuous. Hence, by letting \(S \coloneqq S_1 = S_2\), the above calculation gives
		\[
			W_n^+ - \sum_{i=1}^{n} \mathbbm{1}_{X_i > 0}
			= S
			= \sum_{i \neq j} \mathbbm{1}_{X_i + X_j \geq 0} - S
			\implies S
			= W_n^+ - \sum_{i=1}^{n} \mathbbm{1}_{X_i > 0}
			= \frac{1}{2} \sum_{i \neq j} \mathbbm{1}_{X_i + X_j \geq 0}.
		\]
		The results for \(W_n^-\) follows similarly.
	\end{explanation}

	By subtracting the above, since \(\mathbbm{1}_{X_i > 0} - \mathbbm{1}_{X_i < 0} = \sgn (X_i) \), we have
	\[
		W_n - \sum_{i=1}^{n} \sgn (X_i)
		= \frac{1}{2} \sum_{i \neq j} \left( \mathbbm{1}_{X_i + X_j \geq 0} - \mathbbm{1}_{X_i + X_j < 0} \right)
		= \frac{1}{2} \sum_{i \neq j} \left( 2 \mathbbm{1}_{X_i + X_j \geq 0} - 1 \right).
	\]
	Plugging the definition of \(h\) yields the result.
\end{proof}

Hence, from \autoref{prop:Wilcoxon-signed-rank-test}, we should first study \(\sum_{i \neq j} h(X_i, X_j)\). First, consider testing
\[
	H_0 \colon X, X_1, \dots , X_n \overset{\text{i.i.d.} }{\sim } F \text{ where \(F\) is continuous such that } X \overset{D}{=} -X,
\]
i.e., \(\mathbb{P} (X \geq x) = \mathbb{P} (X \leq -x)\) for all \(x \in \mathbb{R}\). Note that we're not assuming \(\mathbb{E}_{}[X_i] \) to exists.

\begin{note}
	Since \(X_i\)'s are i.i.d.\ under \(H_0\), \(h(X_i, X_j)\)'s are identically distributed under \(H_0\) for \(i \neq j\).
\end{note}

However, it's clear that \(h(X_i, X_j)\)'s are not independent. Anyway, recall what we're trying to study.

\begin{prev}
	From \autoref{prop:Wilcoxon-signed-rank-test}, under \(H_0\) (in particular, \(F\) being continuous),
	\[
		W_n - \sum_{i=1}^{n} \sgn (X_i)
		= \sum_{i \neq j} \left( h(X_i , X_j) - \frac{1}{2} \right)
	\]
\end{prev}
This is a sum of identically distributed random variables subtracting something.

\begin{intuition}
	It \(1 / 2\) is the expectation of \(h(X_i, X_j)\) for \(i \neq j\), we're getting close to the familiar form.
\end{intuition}

\begin{claim}
	Indeed, under \(H_0\), \(\mathbb{E}_{}[h(X_i, X_j)] = 1 / 2\) for \(i \neq j\).
\end{claim}
\begin{explanation}
	We see that
	\[
		\begin{split}
			\mathbb{E}_{H_0}[h(X_i, X_j)]
			= \mathbb{E}_{H_0}[h(X_1, X_2)]
			 & = \mathbb{P} (X_1 + X_2 \geq 0)                             \\
			 & = \mathbb{P} (X_1 \geq -X_2)                                \\
			 & = \int _\mathbb{R} \mathbb{P} (X_1 \geq -x) F(\mathrm{d} x)
			= \int _\mathbb{R} \mathbb{P} (X \leq x) F(\mathrm{d} x)
			= \mathbb{E}_{}[F(X)]
			= \frac{1}{2}
		\end{split}
	\]
	since \(F\) is continuous, and \(F(X) \sim \mathcal{U} (0, 1)\) as we have shown in the homework.
\end{explanation}

Hence, under \(H_0\),
\[
	\sum_{i \neq j} \left( h(X_i, X_j) - \frac{1}{2} \right)
	= 2 \sum_{i < j} \left( h(X_i, X_j) - \frac{1}{2} \right)
	= 2 \sum_{i < j} \left( h(X_i, X_j) - \mathbb{E}_{}[h(X_i, X_j)] \right).
\]
By rescaling by the number of terms in the summation, we're looking at
\[
	\frac{1}{\binom{n}{2}} \left( W_n - \sum_{i=1}^{n} \sgn (X_i) \right)
	= \frac{2}{\binom{n}{2}} \sum_{i < j} \left( h(X_i, X_j) - \mathbb{E}_{}[h(X_i, X_j)] \right)
	\eqqcolon 2 \left( U_n - \frac{1}{2} \right)
\]
where we define \(U_n\) for some permutation symmetric function \(h\)\footnote{Indeed, since \(\mathbb{E}_{}[h(X_1, X_2)] = \mathbb{P} (X_1 + X_2 \geq 0)\), so \(h(X_1, X_2) = h(X_2, X_1)\).} as
\[
	U_n \coloneqq \frac{1}{\binom{n}{2}} \sum_{i < j} h(X_i, X_j).
\]
This is the so-called \emph{\(U\)-statistic}, where \(U\) stands for unbiased. We will show that by multiplying \(\sqrt{n} \) on the both sides, this \hyperref[def:converge-in-distribution]{converges} to a standard normal, i.e.,
\[
	\frac{\sqrt{n}}{\binom{n}{2}}  \left( W_n - \sum_{i=1}^{n} \sgn (X_i) \right)
	= \sqrt{n} 2 \left( U_n - \frac{1}{2} \right)
	\overset{D}{\to} \mathcal{N} (0, 1).
\]

\begin{note}
	This asymptotic result will imply an asymptotic distribution for \(W_n\) exactly alone since
	\[
		2 \frac{\sqrt{n} }{n(n-1)} \sum_{i=1}^{n} \sgn (X_i)
		= \frac{2}{n-1} \left( \frac{1}{\sqrt{n} } \sum_{i=1}^{n} \sgn (X_i) \right)
		\overset{p}{\to} 0.
	\]
\end{note}

\section{\(U\)-Statistics}
In this section, we develop the asymptotic theory for \(U\)-statistics. Let's first recall the following proposition we have shown in the homework.

\begin{proposition}\label{prop:correlation-converging-together}
	If \(\Var_{}[S_n] / \Var_{}[\widetilde{S} _n] \to 1\) and \(\Corr_{}(S_n, \widetilde{S} _n) \to 1\), then
	\[
		\left\vert \frac{S_n - \mathbb{E}_{}[S_n] }{\sqrt{\Var_{}[S_n] } } - \frac{\widetilde{S} _n - \mathbb{E}_{}[\widetilde{S} _n] }{\sqrt{\Var_{}[S_n] } } \right\vert
		\overset{p}{\to} 0.
	\]
\end{proposition}

Hence, to show \((S_n - \mathbb{E}_{}[S_n] ) / \sqrt{\Var_{}[S_n] } \overset{D}{\to} Y\), we may find \((\widetilde{S} _n)\) such that \((\widetilde{S} _n - \mathbb{E}_{}[\widetilde{S} _n]) / \sqrt{\Var_{}[S_n] } \overset{D}{\to} Y\) and show
\[
	\left\vert \frac{S_n - \mathbb{E}_{}[S_n] }{\sqrt{\Var_{}[S_n] } } - \frac{\widetilde{S} _n - \mathbb{E}_{}[\widetilde{S} _n] }{\sqrt{\Var_{}[S_n] } }\right\vert
	\overset{p}{\to} 0
\]
by \autoref{prop:correlation-converging-together}.