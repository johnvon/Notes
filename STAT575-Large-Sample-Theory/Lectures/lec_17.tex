\lecture{17}{21 Mar.\ 9:30}{Rank Test and Two-Sample Problem}
\begin{proof}
	Firstly, we see that \(\Var_{}[S_n] = \sum_{j=1}^{K_n} c_{nj}^2 \sigma ^2\), hence with our usual notation, we define
	\[
		Y_{nj}
		\coloneqq \frac{c_{nj} (Z_j - \mu )}{\sqrt{\sum_{i=1}^{K_n} c_{ni}^2} \sigma }
		\eqqcolon \frac{c_{nj}}{\sqrt{\sum_{i=1}^{K_n} c_{ni}^2} } W_j,
	\]
	where \(W_j \coloneqq (Z_j - \mu ) / \sigma \). Then, for any \(\epsilon > 0\), we can check that \hyperref[def:Lindeberg-condition]{Lindeberg condition} as
	\begin{align*}
		\sum_{j=1}^{K_n} \mathbb{E}_{}[Y_{nj}^2 \mathbbm{1}_{\vert Y_{nj} \vert > \epsilon } ]
		 & = \sum_{j=1}^{K_n} \frac{c_{nj}^2}{\sum_{i=1}^{K_n} c_{ni}^2} \mathbb{E}_{}\left[ W_j^2 \mathbbm{1}_{\frac{\vert c_{nj} \vert }{\sqrt{\sum_{i=1}^{K_n} c_{ni}^2} } \vert W_j \vert > \epsilon } \right]                               \\
		\shortintertext{since the only dependence of \(j\) in the expectation is \(\vert c_{nj} \vert \) in the indicator,}
		 & \leq \sum_{j=1}^{K_n} \frac{c_{nj}^2}{\sum_{i=1}^{K_n} c_{ni}^2} \mathbb{E}_{}\left[ W_j^2 \mathbbm{1}_{ \max _{1 \leq k \leq K_n} \frac{\vert c_{nk} \vert }{\sqrt{\sum_{i=1}^{K_n} c_{ni}^2} } \vert W_k \vert > \epsilon } \right] \\
		\shortintertext{which makes the term in the expectation i.i.d., so we can replace both \(W_j\) and \(W_k\) by \(W \coloneqq W_1\),}
		 & = \sum_{j=1}^{K_n} \frac{c_{nj}^2}{\sum_{i=1}^{K_n} c_{ni}^2} \mathbb{E}_{}\left[ W^2 \mathbbm{1}_{ \max _{1 \leq k \leq K_n} \frac{\vert c_{nk} \vert }{\sqrt{\sum_{i=1}^{K_n} c_{ni}^2} } \vert W \vert > \epsilon } \right]        \\
		 & = \mathbb{E}_{}\left[ W^2 \mathbbm{1}_{ \max _{1 \leq k \leq K_n} \frac{\vert c_{nk} \vert }{\sqrt{\sum_{i=1}^{K_n} c_{ni}^2} } \vert W \vert > \epsilon } \right] .
	\end{align*}
	Hence, it reduces to show \(\mathbb{E}_{}[ W^2 \mathbbm{1}_{\vert W \vert > x} ] \to 0\) as \(n \to \infty \) for \(x \coloneqq \epsilon \sqrt{\sum_{i=1}^{K_n} c_{ni}^2} / \max _{1 \leq k \leq K_n} \vert c_{nk} \vert \). From our assumption, for any \(\epsilon > 0\), \(x \to \infty \) as \(n \to \infty \), hence the expectation indeed goes to \(0\) as long as \(W\) has finite second moment, which is indeed the case by our assumption.
\end{proof}

We see that \hyperref[thm:Hajek-Sidak-CLT]{Hajek-Sidak central limit theorem} is very common in practice.

\begin{intuition}
	Often time for every \(n\), \(c_{ni} = c_n\), the same for every \(1 \leq i \leq K_n\). In this case, we may write \(X_{n i} = c_n \cdot X_{ni} / c_n \eqqcolon c_n Z_i\) such that \(Z_i \coloneqq X_{ni} / c_n\) is i.i.d.\ distributed, ready for checking the \hyperref[thm:Hajek-Sidak-CLT]{Hajek-Sidak condition}.
\end{intuition}

Let's see three examples.

\begin{eg}[Uniform distribution]
	For every \(n \geq 1\), let \((X_{n K_n}) \overset{\text{i.i.d.} }{\sim } \mathcal{U} (-c_n, c_n)\) for some \(c_n > 0\). Then we see that \(Z_i \coloneqq X_{n i} / c_n \sim \mathcal{U} (-1, 1)\) is now i.i.d.\ distributed.
\end{eg}

\begin{eg}[Rademacher distribution]
	For every \(n \geq 1\), let \((X_{n K_n})\) be i.i.d.\ such that \(\mathbb{P} (X_{ni} / c_n = \pm 1) = 1 / 2\). Then \(Z_i \coloneqq X_{ni} / c_n\) is now i.i.d.\ distributed.
\end{eg}

\begin{eg}[Exponential distribution]
	For every \(n \geq 1\), let \((X_{n K_n}) \overset{\text{i.i.d.} }{\sim } \operatorname{Exp}(1 / c_n) \) for some \(c_n > 0\), hence \(Z_i \coloneqq X_{ni} / c_n \sim \operatorname{Exp}(1) \) is now i.i.d.\ distributed.
\end{eg}

On the other hand, if we insist the same setup as in the \hyperref[thm:Lindeberg-CLT]{Lindeberg central limit theorem}, it suffices to check a slightly higher moment rather than the truncated one used in the \hyperref[def:Lindeberg-condition]{Lindeberg condition}.

\begin{corollary}[Lyopunov's central limit theorem]\label{thm:Lyopunov-CLT}
	Consider the setup as in the \hyperref[thm:Lindeberg-CLT]{Lindeberg central limit theorem}. If \(\sum_{j=1}^{K_n} \mathbb{E}_{}[\vert Y_{nj} \vert ^{2 + \delta }] \to 0\) for some \(\delta > 0\), then the \hyperref[def:Lindeberg-condition]{Lindeberg condition} holds.
\end{corollary}
\begin{proof}
	Fix some \(\delta > 0\) such that the assumption holds. Then for any \(\epsilon > 0\), we have
	\[
		\sum_{j=1}^{K_n} \mathbb{E}_{}[\vert Y_{nj} \vert ^2 \mathbbm{1}_{\vert Y_{nj} \vert > \epsilon } ]
		\leq \sum_{j=1}^{K_n} \mathbb{E}_{}\left[ \left( \frac{\vert Y_{nj} \vert }{\epsilon } \right) ^\delta \vert Y_{nj} \vert ^2 \mathbbm{1}_{\vert Y_{nj} \vert > \epsilon } \right]
		\leq \sum_{j=1}^{K_n} \mathbb{E}_{}\left[ \left( \frac{\vert Y_{nj} \vert }{\epsilon } \right) ^\delta \vert Y_{nj} \vert ^2 \right]
		\to 0
	\]
	by taking \(\epsilon ^{-\delta }\) out and then the result follows from the assumption.
\end{proof}

For bounded random variable, we have a simpler form.

\begin{corollary}\label{col:Lyopunov-CLT}
	Let \(\vert X_{ni} \vert \leq C_n\) for all \(1 \leq i \leq K_n\). Then if \(C_n / \sqrt{\Var_{}[S_n] } \to 0\), the \hyperref[def:Lindeberg-condition]{Lindeberg condition} holds. In particular, when \(C_n \eqqcolon C\), it suffices to check \(\Var_{}[S_n] \to \infty \).
\end{corollary}
\begin{proof}
	We see that for every \(1 \leq j \leq K_n\),
	\[
		\vert Y_{nj} \vert
		= \frac{\vert X_{nj} - \mathbb{E}_{}[X_{nj}] \vert }{\sqrt{\Var_{}[S_n] } }
		\leq \frac{2 C_n}{\sqrt{\Var_{}[S_n] } }.
	\]
	In this case, for any \(\delta > 0\), recall that \(\sum_{j=1}^{K_n} \mathbb{E}_{}[Y_{nj}^2] = 1\), hence
	\[
		\sum_{j=1}^{K_n} \mathbb{E}_{}[\vert Y_{nj} \vert ^{2 + \delta }]
		\leq \left( \frac{2C_n}{\sqrt{\Var_{}[S_n] } } \right) ^\delta \sum_{j=1}^{K_n} \mathbb{E}_{}[Y_{nj}^2]
		= \left( \frac{2C_n}{\sqrt{\Var_{}[S_n] } } \right) ^\delta
		\to 0,
	\]
	hence the \hyperref[thm:Lyopunov-CLT]{Lyopunov's condition} holds.
\end{proof}

\begin{eg}[Bernoulli distribution]\label{eg:Lyopunov-CLT}
	For every \(n \geq 1\), let \(X_{ni} \sim \operatorname{Ber}(p_{ni}) \) for all \(1 \leq i \leq K_n\). Since \(X_{ni} \leq 1\), from \autoref{col:Lyopunov-CLT}, it suffices to check
	\[
		\Var_{}[S_n]
		= \sum_{i=1}^{K_n} p_{ni} (1 - p_{ni}) \to \infty .
	\]
	\begin{itemize}
		\item If \(p_{ni} = 1 / i\): then \(\Var_{}[S_n] = \sum_{i=1}^{K_n} 1 / i - \sum_{i=1}^{K_n} 1 / i^2 \to \infty \).
		\item If \(p_{ni} = p_n\): then \(\Var_{}[S_n] = K_n p_n (1 - p_n)\). In particular, for \(p_n = 1 / n\) and \(K_n = n\),
		      \[
			      \Var_{}[S_n]
			      = n \cdot \frac{1}{n} \cdot \frac{n-1}{n}
			      \to 1
			      \neq \infty .
		      \]
		      In general, if \(n p_n \to \lambda > 0\), then the sum \(S_n\) \hyperref[def:converge-in-distribution]{converges} to \(\operatorname{Pois}(\lambda ) \).
	\end{itemize}
\end{eg}

\subsection{}\todo{Fill}
Consider collect a sequence of data \(X_1, \dots , X_n\) and assume a distribution function \(F\) being continuous. We want to test whether \(X_i\)'s are i.i.d.\ from \(F\). Let the rank be
\[
	R_i = \sum_{j=1}^{i} \mathbbm{1}_{X_j < X_i}
\]
for all \(i \geq 1\). We see that
\begin{itemize}
	\item if \(R_i = i\), then \(X_i\) is the largest of \(X_1, \dots , X_i\);
	\item if \(R_i = 1\), then \(X_i\) is the smallest of \(X_1, \dots , X_i\).
\end{itemize}

\begin{note}
	We don't need to worry about the equality in the indicator since \(F\) is continuous.
\end{note}

\begin{theorem}\label{thm:rank}
	Let \(X_1, X_2 , \dots \overset{\text{i.i.d.} }{\sim } F\) where \(F\) is continuous. Then the following hold.
	\begin{enumerate}[(a)]
		\item\label{thm:rank-a} \(R_1, R_2, \dots \) are independent and \(\mathbb{P} (R_i = r) = 1 / i\) for all \(1 \leq r \leq i\), i.e., \(R_i \sim \mathcal{U} (\{ 1, \dots , i \} )\).
		\item\label{thm:rank-b} \(\sum_{i=1}^{n} (R_i - \mathbb{E}_{}[R_i] ) / \sqrt{\Var_{}[\sum_{i=1}^{n} R_i] } \overset{D}{\to} \mathcal{N} (0, 1)\).
	\end{enumerate}
\end{theorem}
\begin{proof}
	For \autoref{thm:rank-a}, since \(X_i\)'s are i.i.d.\ and \(F\) it continues (hence no ties),
	\[
		\mathbb{P} (X_{\sigma (1)} < X_{\sigma (2)} < \dots < X_{\sigma (i)})
		= \frac{1}{i!}
	\]
	for any permutation \(\sigma \) of \(\{ 1, \dots , i \} \). This implies \(\mathbb{P} (R_1 = r_1, \dots , R_i = r_i) = 1 / i!\), hence
	\[
		\mathbb{P} (R_i = r)
		= \sum_{r_1, \dots , r_{i-1}} \mathbb{P} (R_1 = r_1 , \dots , R_{i-1} = r_{i-1} , R_i = r)
		= (i - 1)! \cdot \frac{1}{i!}
		= \frac{1}{i},
	\]
	which proves \autoref{thm:rank-a}. We now see that \autoref{thm:rank-b} is immediate from \autoref{thm:rank-a} since from \autoref{col:Lyopunov-CLT},
	\[
		\frac{n}{\sqrt{\Var_{}[S_n] }}
		= \frac{n}{\sqrt{\sum_{i=1}^{n} \frac{i^2 - 1}{12}}}
		= \frac{6 \sqrt{2} n}{\sqrt{n (2 n^{2} + 3 n - 5)}}
		\to 0
	\]
	as \(n \to \infty \), proving the result.
\end{proof}

\begin{remark}
	We may instead consider
	\[
		Z_i = \begin{dcases}
			1, & \text{ if } R_i = i ; \\
			0, & \text{ if } R_i < i .
		\end{dcases}
	\]
	Hence, \(\mathbb{P} (Z_i = 1) = 1 / i\), i.e., \(Z_i \sim \operatorname{Ber}(1 / i) \). Then from the \hyperref[eg:Lyopunov-CLT]{previous example}, we can also get
	\[
		\frac{\sum_{i=1}^{n} Z_i - \mathbb{E}_{}[Z_i] }{\sqrt{\Var_{}[\sum_{i=1}^{n} Z_i] } }
		\overset{D}{\to} \mathcal{N} (0, 1).
	\]
\end{remark}

\subsection{Two-Sample Problem with Paired Observation}
Say we want to test the effect of a treatment. Ideally, we have the testament group and the control group, and observe \((Y_1^{\text{T} }, Y_1^{\text{C} }), (Y_2^{\text{T} }, Y_2^{\text{C} }), \dots , (Y_n^{\text{T} }, Y_n^{\text{C} })\). We consider \(Z_i = Y_i^{\text{T} } - Y_i^{\text{C} }\) for all \(1 \leq i \leq n\).

Usually, we assume \(Z_1 , \dots , Z_n\) are i.i.d.\ with mean \(\mu \) and finite variance. The null hypothesis is then \(H_0 \colon \mu = 0\). The \(t\)-test rejects \(H_0\) if
\[
	T_n
	= \sqrt{n} \frac{\overline{Z} _n}{\hat{\sigma} _n}
	= \frac{\sum_{i=1}^{n} Z_i}{\sqrt{n} \sqrt{\frac{1}{n} \sum_{i=1}^{n} Z_i^2 - \overline{Z} _n^2} }
	> Z_{\alpha }.
\]
However, we know that in reality, \(Z_i\)'s are never i.i.d., but people still use the \(t\)-test. Why? This is because we usually give the treatment in a randomized way. In addition, we condition on the \(\vert Z_i \vert \) for all \(i \geq 1\). In this case,
\[
	H_0 \colon \mathbb{P} (Z_i = \pm \vert Z_i \vert ) \text{ for }  1 \leq i \leq n,
\]
i.e., \(\mathbb{P} (Z_i / \vert Z_i \vert = \pm 1) = 1 / 2\), i.e., \(Z_i\) are now i.i.d., and we can apply the \hyperref[thm:CLT]{central limit theorem}, we have \(\sum_{i=1}^{n} Z_i / \sqrt{\sum_{i=1}^{n} Z_i^2} \).