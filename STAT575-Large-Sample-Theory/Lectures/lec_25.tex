\lecture{25}{18 Apr.\ 9:30}{Understanding Assumptions for Consistency}
We first prove \autoref{thm:M-estimator-consistency}.

\begin{proof}
	From the definition, we have
	\begin{itemize}
		\item \(\hat{\theta} _n \overset{p}{\to} \theta ^{\ast} \iff \mathbb{P} (\rho (\hat{\theta} _n, \theta ^{\ast} ) \geq \epsilon ) \to 0\) for all \(\epsilon > 0\);
		\item \(\hat{\theta} _n \overset{\text{a.s.} }{\to} \theta ^{\ast} \iff \mathbb{P} (\limsup_{n \to \infty} \rho (\hat{\theta} _n, \theta ^{\ast} ) \geq \epsilon ) = 0\) for all \(\epsilon > 0\).
	\end{itemize}
	Hence, we need to show that for all \(\epsilon > 0\), \(\mathbb{P} (\rho (\hat{\theta} _n , \theta ^{\ast} ) \geq \epsilon ) \to 0\) or \(\mathbb{P} (\rho (\hat{\theta} _n, \theta ^{\ast} ) > \epsilon \text{ i.o.} ) = 0\). Firstly, suppose \(\rho (\hat{\theta} _n, \theta ^{\ast} ) \geq \epsilon \), which implies \(\hat{\theta} _n \notin B(\theta ^{\ast} , \epsilon )\), and with \autoref{eq:approximate-M-estimator},
	\[
		\sup _{\theta \notin B(\theta ^{\ast} , \epsilon )} M_n(\theta )
		\geq M_n(\hat{\theta} _n)
		\geq \sup _{\theta \in \Theta } M_n(\theta ) - \xi _n
		\geq M_n(\theta ^{\ast} ) - \xi _n.
	\]
	This gives \(\sup _{\theta \notin B(\theta ^{\ast} , \epsilon )}M_n(\theta ) - M_n (\theta ^{\ast} ) + \xi _n \geq 0\), i.e.,
	\[
		\sup _{\theta \notin B(\theta ^{\ast} , \epsilon )} M_n(\theta ) - M_n(\theta ^{\ast} ) + \xi _n - \sup _{\theta \notin B(\theta ^{\ast} , \epsilon )} M(\theta ) + M(\theta ^{\ast} )
		\geq M(\theta ^{\ast} ) - \sup _{\theta \notin B(\theta ^{\ast} , \epsilon )} M(\theta )
		\eqqcolon \delta _\epsilon > 0
	\]
	by \autoref{asp:MLE-1}. Write \(A_n \coloneqq \sup _{\theta \notin B(\theta ^{\ast} , \epsilon )} M_n(\theta ) - \sup _{\theta \notin B(\theta ^{\ast} , \epsilon )} M(\theta )\), \(C_n \coloneqq M(\theta ^{\ast} ) - M_n(\theta ^{\ast} )\), we have
	\[
		A_n + C_n + \xi _n \geq \delta _\epsilon .
	\]
	This gives \(\{ \rho (\hat{\theta} _n, \theta ^{\ast} ) \geq \epsilon \} \subseteq \{ A_n + C_n + \xi _n \geq \delta _\epsilon  \}\), hence
	\[
		\mathbb{P} (\rho (\hat{\theta} _n, \theta ^{\ast} ) \geq \epsilon)
		\leq \mathbb{P} \left( A_n \geq \frac{\delta _\epsilon}{3} \right) + \mathbb{P} \left( C_n \geq \frac{\delta _\epsilon}{3} \right) + \mathbb{P} \left( \xi _n \geq \frac{\delta _\epsilon}{3} \right).
	\]
	Now, with \autoref{asp:MLE-2}, \(\mathbb{P} (A_n \geq \delta _\epsilon / 3)\) vanishes. Moreover, as \(M_n(\theta ) \overset{p}{\to} M(\theta )\) for any fixed \(\theta \), in this case, \(\theta ^{\ast} \) for \(C_n\), \(\mathbb{P} (C_n \geq \delta _\epsilon / 3)\) also vanishes. Finally, \(\mathbb{P} (\xi _n \geq \delta _\epsilon / 3)\) vanishes if we require \(\xi _n \overset{p}{\to} 0\).

	The case for \autoref{asp:MLE-3} is the same, but this time we need \(M_n(\theta ) \overset{\text{a.s.} }{\to} M(\theta ) \) and \(\xi _n \overset{\text{a.s.} }{\to} 0\).
\end{proof}

\subsection{Proving the Assumptions}
With \autoref{thm:M-estimator-consistency}, the only task left is to check \autoref{asp:MLE-1} and \autoref{asp:MLE-2} or \autoref{asp:MLE-3}. Although we have briefly seen how to prove \autoref{asp:MLE-1}, i.e., \autoref{col:asp:MLE-1}, but there are still lots to be discussed. Let's assume that \(X_1, \dots , X_n\) are i.i.d.\ and \(M_n(\theta ) = \frac{1}{n}\sum_{i=1}^{n} m_\theta (X_i)\) where \(\theta \mapsto m_\theta (x)\) is \hyperref[def:upper-semi-continuous]{upper semi-continuous} for all \(x\).

\begin{note}
	\(M_n(\theta )\) has a maximizer since \(\theta \mapsto M_n(\theta )\) is also \hyperref[def:upper-semi-continuous]{upper semi-continuous}.
\end{note}
\begin{explanation}
	Since the sum of \hyperref[def:upper-semi-continuous]{upper semi-continuous} functions is still \hyperref[def:upper-semi-continuous]{upper semi-continuous}.
\end{explanation}

But what about \(M\)? I.e., can we say that the \hyperref[def:upper-semi-continuous]{upper semi-continuity} is preserved in the limit?

\begin{proposition}\label{prop:M-estimator-upper-semi-continuous}
	If for all \(\theta \), there exists \(r_\theta > 0\) such that \(\mathbb{E}_{}[\sup _{u \in B(\theta , r_\theta )} m_u(X)] < \infty \), then \(M\) is \hyperref[def:upper-semi-continuous]{upper semi-continuous}.
\end{proposition}
\begin{proof}
	We want to show that as \(\theta _n \to \theta \), \(M(\theta ) \geq \limsup_{n \to \infty} M(\theta _n)\). From the \hyperref[thm:SLLN]{strong law of large number}, \(M_n(\theta ) \overset{\text{a.s.} }{\to} M(\theta ) = \mathbb{E}_{}[m_\theta (X)] \in [-\infty , \infty ]\). Since \(\theta \mapsto m_\theta (x)\) is \hyperref[def:upper-semi-continuous]{upper semi-continuous},
	\[
		M(\theta )
		= \mathbb{E}_{}[m_\theta (X)]
		\geq \mathbb{E}_{}\left[\limsup_{n \to \infty} m_{\theta _n}(X)\right]
		\geq \limsup_{n \to \infty} \mathbb{E}_{}[m_{\theta _n}(X)]
		= \limsup_{n \to \infty} M(\theta _n),
	\]
	where the last inequality is provided by the \href{https://en.wikipedia.org/wiki/Fatou%27s_lemma#Reverse_Fatou_lemma}{reverse Fatou's lemma}. In particular, it requires \(\mathbb{E}_{}[\sup _{n \geq n_0} m_{\theta _n}(X)] < \infty \) for some large enough \(n_0\). But since for any \(\epsilon > 0\), as \(\theta _n \to \theta \), there exists \(n_0\) such that for all \(n \geq n_0\), \(\theta _n \in B(\theta , \epsilon )\), such a requirement becomes what we assumed.
\end{proof}

With \autoref{prop:M-estimator-upper-semi-continuous}, we now have a tool to check \hyperref[def:upper-semi-continuous]{upper semi-continuous} for \(M\), and with \autoref{col:asp:MLE-1}, we only need to check that \(M\) has a unique maximizer.

\begin{eg}[MLE]
	For MLE, if the model class \(\{ g_\theta \colon \theta \in \Theta \} \) we postulated is identifiable is well-specified, i.e., \(g_{\theta } = g_{\theta ^{\prime} }\) implies \(\theta = \theta ^{\prime} \) and the true pdf is in \(\{ g_\theta  \} \), then \(M\) has a unique maximizer.
\end{eg}
\begin{explanation}
	For MLE, recall that \(m_\theta (x) = \log (g_\theta (x))\), and say \(X, X_1, \dots , X_n \overset{\text{i.i.d.} }{\sim } f = g_{\theta ^{\ast} }\). In this case,
	\[
		M(\theta )
		= \mathbb{E}_{}[\log g_\theta (X)]
		= - \KL(f \Vert g_\theta ) + C
		\leq C
	\]
	for some constant \(C\) as we have seen. As \(\theta = \theta ^{\ast} \), the maximum is achieved. However, suppose there exists \(\widetilde{\theta} \) such that \(M(\widetilde{\theta} ) = M(\theta ^{\ast} )\), we have \(\KL(g_{\theta ^{\ast} } \Vert g_{\widetilde{\theta} }) = 0\) implies \(g_{\theta ^{\ast} } = g_{\widetilde{\theta} }\) almost surely. Hence, if \(g_{\theta ^{\ast} } = g_{\widetilde{\theta} }\) doesn't imply \(\theta ^{\ast} = \widetilde{\theta} \), i.e., not identifiable, the maximizer might not be unique.
\end{explanation}

In view of this, \autoref{asp:MLE-1} can be checked in general. Before we turn our focus to checking either \autoref{asp:MLE-2} or \autoref{asp:MLE-3}, we see one more tool related to \hyperref[def:upper-semi-continuous]{upper semi-continuity}, which will be useful later.

\begin{proposition}\label{prop:lec25}
	Let \(M\) be an \hyperref[def:upper-semi-continuous]{upper semi-continuous} function on a compact \(\Theta \), and let \(\theta \in \Theta \) be fixed. Then, as \(r \to 0\), \(\sup _{u \in B(\theta , r)} M(u) \to M(\theta )\).
\end{proposition}
\begin{proof}
	Suppose not, then we have the following.

	\begin{itemize}
		\item If \(M(\theta ) > -\infty \): there exists \(\epsilon > 0\) and \((r_n) \to 0\) such that \(\sup _{u \in B(\theta , r_n)} M(u) \geq M(\theta ) + \epsilon\).
		\item If \(M(\theta ) = -\infty \): there exists \(C_0 \in \mathbb{R} \) such that for all \(C \leq C_0\), \(\sup _{u \in B(\theta , r_n)}M(u) \geq C\).
	\end{itemize}
	The above holds for all \(n \geq 1\). Combining the above, we see that
	\[
		\sup _{u \in B(\theta , r_n)} M(u)
		\geq \max (M(\theta ) + \epsilon , C)
		\implies \sup _{u \in \overline{B(\theta , r_n)} } M(u)
		\geq \max (M(\theta ) + \epsilon , C).
	\]
	Since \(\overline{B(\theta , r_n)} \) is also compact, there exists a \(\theta _n\) in \(\overline{B(\theta , r_n)}\) such that \(M(\theta _n) \geq \max (M(\theta ) + \epsilon , C)\) for all \(n \geq 1\), \(C \leq C_0\), for some \(\epsilon > 0\). Taking the limit, we see that
	\[
		\liminf_{n \to \infty} M(\theta )
		\geq \max (M(\theta ) + \epsilon , C)
		\geq \max \left( \limsup_{n \to \infty} M(\theta _n) + \epsilon , C \right)
	\]
	since \(M\) is \hyperref[def:upper-semi-continuous]{upper semi-continuous} and \(\theta _n \to \theta \) as \(r_n \to 0\). By letting \(C \to -\infty \), we have
	\[
		\liminf_{n \to \infty} M(\theta _n)
		\geq \limsup_{n \to \infty} M(\theta _n) + \epsilon
		> \limsup_{n \to \infty} M(\theta _n),
	\]
	which is a contradiction.
\end{proof}

This concludes the discussion on \hyperref[def:upper-semi-continuous]{upper semi-continuity}. Now, let's focus on the stronger assumption, i.e., \autoref{asp:MLE-3}, since as we will soon see, the assumption used in \autoref{prop:M-estimator-upper-semi-continuous} suffices to prove \autoref{asp:MLE-3}. Explicitly, for \autoref{asp:MLE-3}, we want to show that for any \(\epsilon > 0\),
\[
	\mathbb{P} \left( \limsup_{n \to \infty} \sup _{\theta \notin B(\theta ^{\ast} , \epsilon )} M_n(\theta ) \leq \sup _{\theta \notin B(\theta ^{\ast} , \epsilon )} M(\theta ) \right)
	= 1.
\]
Replacing \(\theta \notin B(\theta ^{\ast} , \epsilon )\) by \(\theta \in (B(\theta ^{\ast} , \epsilon ))^{c} \), where \((B(\theta ^{\ast} , \epsilon ))^{c}\) is compact as shown before. Now, consider a trivial open cover
\[
	(B(\theta ^{\ast} , \epsilon ))^{c}
	\subseteq \bigcup_{\theta \in (B(\theta ^{\ast} , \epsilon ))^{c} } B(\theta , r_\theta ) \text{ for some } r_\theta > 0 \text{ for all } \theta \in (B(\theta ^{\ast} , \epsilon ))^{c} ,
\]
from the compactness of \((B(\theta ^{\ast} , \epsilon ))^{c} \), there exists a finite subset \(K \subseteq (B(\theta ^{\ast} , \epsilon ))^{c} \) such that
\[
	(B(\theta ^{\ast} , \epsilon ))^{c}
	\subseteq \bigcup_{\theta \in K} B(\theta , r_\theta ).
\]
This implies that
\[
	\sup _{\theta \notin B(\theta ^{\ast} , \epsilon )} M_n(\theta )
	\leq \max _{\theta \in K} \sup _{u \in B(\theta , r_\theta )} M_n(u)
	\leq \max _{\theta \in K} \frac{1}{n}\sum_{i=1}^{n} \sup _{u \in B(\theta , r_\theta )} m_u(X_i)
	\overset{\text{a.s.} }{\to} \max _{\theta \in K} \mathbb{E}_{}\left[\sup _{u \in B(\theta , r_\theta )} m_u(X) \right]
\]
by the \hyperref[thm:SLLN]{strong law of large number}. However, we need to be careful here by combining \(\overset{\text{a.s.} }{\to} \) and \(\max _{\theta \in K}\).

\begin{claim}
	We can indeed interchange the limit and the maximum.
\end{claim}
\begin{explanation}
	Since the \hyperref[def:converge-almost-surely]{convergence} happens for every \(\theta \in K\) such that
	\[
		\frac{1}{n} \sum_{i=1}^{n} \sup _{u \in B(\theta , r_\theta )} m_u(X_i(\omega ))
		\overset{\text{a.s.} }{\to} \mathbb{E}_{}\left[ \sup _{u \in B(\theta , r_\theta )} m_u(X(\omega )) \right].
	\]
	In other words, for all \(\theta \in K\), there exists a null set \(N_\theta \) such that the convergence holds for all \(\omega \notin N_\theta \). With \(\lvert K \rvert < \infty \), \(\bigcup_{\theta \in K} N_\theta \) is a \emph{countable} union of null sets, which is still a null set. Hence, \(K\) might be countable if this is the only issue. However, this is not the only story.

	Consider \(\lvert K \rvert = 2\), or in particular, \(K = \{ \theta _1, \theta _2 \} \). In this case, we can say that the maximum of the limits is the limits of the maximum since for these two sequences of convergence sequence, we're saying that for \(\theta _i\), the limiting statement involves ``there exists \(n_i\) such that for all \(n \geq n_{\theta _i}\), such that \dots.'' Hence, if \(K\) is finite, one can just take \(n_0 \coloneqq \max _{\theta \in K} n_{\theta }\), which is guaranteed to be finite. If \(K\) is countable, \(n_0\) can be unbounded, causing problems.
\end{explanation}

Hence, with the above justification, taking limits on both sides we have
\[
	\limsup_{n \to \infty} \sup _{\theta \notin B(\theta ^{\ast} , \epsilon )} M_n(\theta )
	\leq \max _{\theta \in K} \mathbb{E}_{}\left[\sup _{u \in B(\theta , r_\theta )} m_u(X)\right] .
\]
With this, the upshot is that by using the same assumption used in \autoref{prop:M-estimator-upper-semi-continuous}, the above can be controlled carefully. Specifically, since \(\theta \mapsto m_\theta (x)\) is \hyperref[def:upper-semi-continuous]{upper semi-continuous} for all \(x\), we know that as \(r \to 0\), \(\sup _{u\in B(\theta , r)} m_u(X) \to m_\theta (X)\) for every \(\theta \in \Theta \), as shown in \autoref{prop:lec25}. Hence,
\[
	M(\theta )
	= \mathbb{E}_{}[m_\theta (X)]
	= \lim_{r \to 0} \mathbb{E}_{}\left[\sup _{u \in B(\theta , r)} m_u(X) \right]
\]
by the \href{https://en.wikipedia.org/wiki/Monotone_convergence_theorem}{monotone convergence theorem} as the sequence is the supremum over smaller and smaller sets. This implies that for all \(\theta \in \Theta \) and \(\delta > 0\), there exists \(r_\theta > 0\) such that \(\mathbb{E}_{}[\sup _{u \in B(\theta , r_\theta )} m_u(X)] < M(\theta ) + \delta \). Hence, combining this with the above, by choosing \(\theta _{\theta }\) for every \(\theta \in \Theta \) in this way, we have
\[
	\limsup_{n \to \infty} \sup _{\theta \notin B(\theta ^{\ast} , \epsilon )} M_n(\theta )
	\leq \max _{\theta \in K} \mathbb{E}_{}\left[\sup _{u \in B(\theta , r_\theta )} m_u(X)\right]
	\leq \max _{\theta \in K} M(\theta ) + \delta
	\leq \sup _{\theta \notin B(\theta ^{\ast} , \epsilon )}M(\theta ) + \delta
\]
almost surely, which is exactly \autoref{asp:MLE-3}.\footnote{We didn't handle the case that \(M(\theta ) = -\infty \). However, this can be easily addressed as in \autoref{prop:lec25}.}

\begin{intuition}
	We can use a single assumption to prove both \autoref{asp:MLE-1} and \autoref{asp:MLE-3}!
\end{intuition}