\lecture{3}{25 Jan.\ 9:30}{Weak Convergence Portmanteau theorem}
\section{Weak Convergence}
All convergences we have discussed are in some senses ``point-wise'' but not ``distribution-wise'', and the latter is more powerful. Consider working with a probability space \((\Omega , \mathscr{F} , \mathbb{P} )\) and the following.

\begin{definition}[Total variation]\label{def:total-variation}
	The \emph{total variation} distance between \(X\) and \(Y\) in \(\Omega \) is defined as
	\[
		\operatorname{TV}(X, Y) = \sup_{B \in \mathscr{F} } \left\vert \mathbb{P} (X \in B) - \mathbb{P} (Y \in B) \right\vert
	\]
\end{definition}

Returning to our situation, consider a sequence or random variables \((X_n)\) and a random variable \(X\).

\begin{remark}
	If \(X_n\) has density \(f_n\) and \(X\) has density \(f\), then \(\operatorname{TV}(X_n, X) = \frac{1}{2} \int \vert f_n - f \vert \).
\end{remark}

\begin{definition}[Convergence in total variation]\label{def:convergence-in-total-variation}
	\((X_n)\) \emph{converges in total variation} to \(X\), denoted as \(X_n \overset{\operatorname{TV}}{\to } X\), if \(\operatorname{TV}(X_n, X) \to 0\) as \(n \to \infty \).
\end{definition}

\begin{remark}
	If \(X_n\) and \(X\) have densities \(f_n\) and \(f\), \(f_n \to f\) implies \(X_n \overset{\operatorname{TV} }{\to } X\) from \hyperref[thm:Scheffe]{Scheffé's theorem}.
\end{remark}

\begin{note}
	The above could make sense even if \(X_n\) is defined on different \((\Omega _n , \mathscr{F} _n, \mathbb{P} _n)\) for every \(n\).
\end{note}

Let's see some examples.

\begin{eg}
	Consider \(X_n \sim \operatorname{Bin}(n, p_n) \) such that \(n p_n \to \lambda \in \mathbb{R} \). As this happens,
	\[
		X_n \sim \operatorname{Bin}(n, p_n)
		\overset{\operatorname{TV} }{\to } X \sim \operatorname{Pois}(\lambda ).
	\]
\end{eg}

\begin{eg}
	Let \(X_n \sim f_{\theta _n}\) where \(f_{\theta }(x) = f(x) e^{\theta x - \psi (\theta )}\) for some \(\theta \in \Theta \). If \(\theta _n \to \theta \), then \(X_n \overset{\operatorname{TV} }{\to } X \sim f_\theta \). For example, if \(X_n \sim \operatorname{Pois}(\theta _n) \) and \(\theta _n \to \theta \), then \(X_n \overset{\operatorname{TV} }{\to } X \sim \operatorname{Pois}(\theta ) \).
\end{eg}

However, \hyperref[def:convergence-in-total-variation]{convergence in total variation} might be too strong to work with.

\begin{eg}
	Let \(X_n \sim \mathcal{U} \{ 0, 1 / n, \dots , (n-1) / n \} \), which should be converging to \(X \sim \mathcal{U} (0, 1)\). However, this doesn't happen in total variation distance as we can take \(B\) to be \(\mathbb{Q} \).
\end{eg}

This suggests that we should look at something weaker.

\begin{definition}[Weak convergence]\label{def:weak-convergence}
	\((X_n)\) \emph{converges weakly} to \(X\), denoted as \(X_n \overset{\text{w} }{\to } X\), if for all bounded continuous \(g \colon \mathbb{R} ^d \to \mathbb{R} \), \(\mathbb{E}_{}\left[g(X_n) \right] \to \mathbb{E}_{}\left[g(X) \right]\).
\end{definition}

To see how is \hyperref[def:weak-convergence]{weak convergence} compared to \hyperref[def:convergence-in-total-variation]{convergence in total variation}, we revisit the above.

\begin{eg}
	Let \(X_n \sim \mathcal{U} \{ 0, 1 / n, \dots , (n-1) / n \} \), which should be converging to \(X \sim \mathcal{U} (0, 1)\). We have
	\[
		\mathbb{E}_{}\left[g(X_n) \right]
		= \sum_{k=0}^{n-1} g(k / n) \left( \frac{k+1}{n} - \frac{k}{n} \right)
		\to \int_{0}^{1} g(x) \,\mathrm{d}x
		= \mathbb{E}_{}\left[g(X) \right]
	\]
	as \(g\) is bounded and continuous on \([0, 1]\), hence Riemann integrable.
\end{eg}

\subsection{Portmanteau Theorem}
The following is our main tool of proving \hyperref[def:weak-convergence]{weak convergence}.

\begin{theorem}[Portmanteau theorem]\label{thm:Portmanteau}
	The following are equivalent.
	\begin{enumerate}[(a)]
		\item\label{thm:Portmanteau-a} \(X_n \overset{\text{w} }{\to } X\).
		\item\label{thm:Portmanteau-b} \(\mathbb{E}_{}\left[g(X_n) \right] \to \mathbb{E}_{}\left[g(X) \right] \) for all bounded Lipschitz \(g \colon \mathbb{R} ^d \to \mathbb{R} \).
		\item\label{thm:Portmanteau-c} \(\mathbb{P} (X \in A) \leq \liminf_{n \to \infty} \mathbb{P} (X _n \in A)\) for all \(A \subseteq \mathbb{R} ^d\) open.
		\item\label{thm:Portmanteau-d} \(\mathbb{P} (X \in A) \geq \limsup_{n \to \infty} \mathbb{P} (X _n \in A)\) for all \(A \subseteq \mathbb{R} ^d\) closed.
		\item\label{thm:Portmanteau-e} \(\mathbb{P} (X_n \in A) \to \mathbb{P} (X \in A)\) for all \(A\) such that \(\mathbb{P} (X \in \partial A) = 0\).
	\end{enumerate}
\end{theorem}

Before we prove \hyperref[thm:Portmanteau]{Portmanteau theorem}, we should note that all our discussion can be extended to metric spaces from Euclidean spaces. Let's first recall some basic results for metric spaces.

\begin{claim}
	Given a metric space \((S, \rho )\), \(\rho (\cdot, A)\) is Lipschitz for any \(A \subseteq S\), i.e., for any \(x, y\in S\),
	\[
		\vert \rho (x, A) - \rho (y, A)\vert \leq \rho (x, y).
	\]
\end{claim}
\begin{explanation}
	Since for any \(z\in S\), \(\rho (x, z) \leq \rho (x, y) + \rho (y, z)\), hence \(\rho (x, A) - \rho (y, A) \leq \rho (x, y)\) by taking the infimum over \(z \in A\). Interchanging \(x\) and \(y\) gives another inequality.
\end{explanation}

\begin{claim}
	Given a metric space \((S, \rho )\), for any \(A \subseteq S\), \(x\in \overline{A} \iff \rho (x, A) = 0\).
\end{claim}
\begin{explanation}
	If \(x\in \overline{A} \), there exists \((x_n)\) in \(A\) such that \(\rho (x_n, x) \to 0\). Then for any \(z\in A\), \(\rho (x, z) \leq \rho (x, x_n) + \rho (x_n, z)\), implying
	\[
		\rho (x, A) \leq \rho (x, x_n) + \rho (x_n, A) \to 0,
	\]
	hence \(\rho (x, A) = 0\). On the other hand, suppose \(\rho (x, A) = 0\). As \(\rho (x, A) = \inf _{y\in A} \rho (x, y)\), there exists \((y_n)\) in \(A\) such that \(\rho (x, y_n) \to \rho (x, A) = 0\), i.e., \(x\in \overline{A} \).
\end{explanation}

The crucial lemma we're going to use to prove \hyperref[thm:Portmanteau]{Portmanteau theorem} is the following.

\begin{lemma}\label{lma:lec3}
	Given a metric space \((S, \rho )\) and let \(A \subseteq S\) be a closed subset. Then there exists bounded Lipschitz \(g_k \colon S \to \mathbb{R} \), decreasing in \(k\) such that \(g_k (x) \searrow \mathbbm{1}_{A}(x) \).
\end{lemma}
\begin{proof}
	Since \(A\) is closed, \(A = \overline{A} \) and
	\[
		\mathbbm{1}_{A}(x) = \begin{dcases}
			1, & \text{ if } x \in A \iff \rho (x, A) = 0 ;   \\
			0, & \text{ if } x \notin A \iff \rho (x, A) > 0.
		\end{dcases}
	\]
	Now, we let
	\[
		g_k (x) = \begin{dcases}
			0,                  & \text{ if } \rho (x, A) > \frac{1}{k} ; \\
			1 - k \rho (x, A) , & \text{ otherwise} ;
		\end{dcases}
		= 1 - \left( k \rho (x, A) \land 1 \right) .
	\]
	We see that
	\begin{itemize}
		\item if \(x\in A\): \(\mathbbm{1}_{A} (x) = 1\), and \(g_k(x) = 1\) since \(\rho (x, A) = 0\);
		\item if \(x \notin A\): \(\mathbbm{1}_{A} (x) = 0\), and \(\rho (x, A) > 0\) since \(A\) closed, and \(g_k(x) = 0\) for all large enough \(k\).
	\end{itemize}
	Finally, it's clear that \(g_k(x)\) takes values in \([0, 1]\), and we now show it's Lipschitz. We have
	\[
		\vert g_k(x) - g_k(y) \vert
		= \vert ( k \rho (x, A) \land 1 ) - ( k \rho (y, A) \land 1 ) \vert
		\leq k \rho (x, y)
	\]
	for all \(x, y \in S\).
\end{proof}

Then we can prove the \hyperref[thm:Portmanteau]{Portmanteau theorem}.

\begin{proof}[Proof of \autoref{thm:Portmanteau}]\let\qed\relax
	\autoref{thm:Portmanteau-a} \(\implies \) \autoref{thm:Portmanteau-b} is clear. And we start by proving \autoref{thm:Portmanteau-c} \(\iff \) \autoref{thm:Portmanteau-d}.
	\begin{claim}
		\autoref{thm:Portmanteau-c} \(\iff \) \autoref{thm:Portmanteau-d}.
	\end{claim}
	\begin{explanation}
		We first prove that \autoref{thm:Portmanteau-d} \(\implies \) \autoref{thm:Portmanteau-c}. Since when \(A\) is open,
		\begin{align*}
			\mathbb{P} (X \in A)
			= 1 - \mathbb{P} (X \in A^{c} )
			 & \leq 1 - \limsup_{n \to \infty} \mathbb{P} (X_n \in A^{c} )             \tag*{\autoref{thm:Portmanteau-d}} \\
			 & = 1 - \limsup_{n \to \infty} \left( 1 - \mathbb{P} (X_n \in A ) \right)
			= \liminf_{n \to \infty} \mathbb{P} (X_n \in A).
		\end{align*}
		\autoref{thm:Portmanteau-c} \(\implies \) \autoref{thm:Portmanteau-d} is exactly the same, hence \autoref{thm:Portmanteau-c} \(\iff \) \autoref{thm:Portmanteau-d}.
	\end{explanation}

	Next, we prove \autoref{thm:Portmanteau-b} \(\implies \) \autoref{thm:Portmanteau-d}, which gives us \autoref{thm:Portmanteau-a} \(\implies \) \autoref{thm:Portmanteau-b} \(\implies \) \autoref{thm:Portmanteau-d} \(\iff \) \autoref{thm:Portmanteau-c}.

	\begin{claim}
		\autoref{thm:Portmanteau-b} \(\implies \) \autoref{thm:Portmanteau-d}.
	\end{claim}
	\begin{explanation}
		From \autoref{lma:lec3}, there exists bounded Lipschitz \(g_k \searrow \mathbbm{1}_{A} \) such that for all closed \(A\),
		\[
			\mathbb{P} (X_n \in A)
			= \mathbb{E}_{}\left[\mathbbm{1}_{A} (X_n) \right]
			\leq \mathbb{E}_{}\left[g_k(X_n) \right].
		\]
		This is true for every \(k\) and \(n\) since \(g_k \geq \mathbbm{1}_{A} \), and by taking the limit as \(n \to \infty \),
		\[
			\limsup_{n \to \infty} \mathbb{P} (X_n \in A)
			\leq \limsup_{n \to \infty} \mathbb{E}_{}\left[g_k(X_n) \right]
			= \mathbb{E}_{}\left[g_k(X) \right]
		\]
		from our assumption \autoref{thm:Portmanteau-b}. Finally, as \(k \to \infty \), it goes to \(\mathbb{E}_{}\left[\mathbbm{1}_{A} (X) \right]
		= \mathbb{P} (X \in A)\) as desired.
	\end{explanation}
	\emph{The proof will be \hyperref[pf:thm:Portmanteau-cont]{continued}\dots}
\end{proof}