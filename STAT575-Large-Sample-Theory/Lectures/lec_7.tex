\lecture{7}{6 Feb.\ 9:30}{Skorohod's Representation Theorem}
\section{Skorohod's Representation Theorem}
So far, we have seen the following.
% https://q.uiver.app/#q=WzAsNixbMCwwLCJcXHRleHR7XFxoeXBlcnJlZltkZWY6Y29udmVyZ2UtY29tcGxldGVseV17Y29tcGxldGVseX19Il0sWzEsMCwiXFx0ZXh0e1xcaHlwZXJyZWZbZGVmOmFsbW9zdC1zdXJlbHktY29udmVyZ2Vde2FsbW9zdC1zdXJlbHl9fSJdLFsyLDAsIlxcdGV4dHtcXGh5cGVycmVmW2RlZjpjb252ZXJnZS1pbi1wcm9iYWJpbGl0eV17aW4gcHJvYmFiaWxpdHl9fSJdLFszLDAsIlxcdGV4dHtcXGh5cGVycmVmW2RlZjpjb252ZXJnZS13ZWFrbHlde3dlYWtseX19Il0sWzIsMSwiXFx0ZXh0e1xcaHlwZXJyZWZbZGVmOmNvbnZlcmdlLWluLUxwXXtpbiBcXChMXnBcXCl9fSJdLFszLDEsIlxcdGV4dHtcXGh5cGVycmVmW2RlZjpjb252ZXJnZS1pbi1kaXN0cmlidXRpb25de2luIGRpc3RyaWJ1dGlvbn19Il0sWzAsMSwiIiwwLHsibGV2ZWwiOjJ9XSxbMSwyLCIiLDAseyJsZXZlbCI6Mn1dLFs0LDIsIiIsMCx7ImxldmVsIjoyfV0sWzMsNSwiIiwwLHsibGV2ZWwiOjIsInN0eWxlIjp7InRhaWwiOnsibmFtZSI6ImFycm93aGVhZCJ9fX1dLFsyLDMsIiIsMCx7ImxldmVsIjoyfV1d
\[
	\begin{tikzcd}[ampersand replacement=\&]
		{\text{\hyperref[def:converge-completely]{completely}}} \& {\text{\hyperref[def:almost-surely-converge]{almost-surely}}} \& {\text{\hyperref[def:converge-in-probability]{in probability}}} \& {\text{\hyperref[def:converge-weakly]{weakly}}} \\
		\&\& {\text{\hyperref[def:converge-in-Lp]{in \(L^p\)}}} \& {\text{\hyperref[def:converge-in-distribution]{in distribution}}}
		\arrow[Rightarrow, from=1-1, to=1-2]
		\arrow[Rightarrow, from=1-2, to=1-3]
		\arrow[Rightarrow, from=2-3, to=1-3]
		\arrow[Rightarrow, 2tail reversed, from=1-4, to=2-4]
		\arrow[Rightarrow, from=1-3, to=1-4]
	\end{tikzcd}
\]
Now, we want to show some interesting result that we might not expect.

\begin{theorem}[Skorohod's representation theorem]\label{thm:Skorohod-representation}
	If \(X_n \overset{D}{\to } X\), there exists \((\widetilde{\Omega} , \widetilde{\mathscr{F}} , \widetilde{\mathbb{P}} )\) on which we can define random vectors \((Y_n)\) and \(Y\) such that \(Y_n \overset{D}{=} X_n\) for all \(n\) and \(Y \overset{D}{=} X\), and \(\widetilde{\mathbb{P}} (Y_n \to Y) = 1\).
\end{theorem}

\begin{intuition}
	We have ``\hyperref[def:converge-in-distribution]{convergence in distribution} implies \hyperref[def:almost-surely-converge]{convergence almost surely}.''
\end{intuition}

We want to prove \hyperref[thm:Skorohod-representation]{Skorohod's representation theorem} for \(d = 1\). To start, say \(X \sim F\) on \((\Omega , \mathscr{F} , \mathbb{P} )\). We will consider \(F^{-1} (p)\), which exists if there exists a unique \(t \in \mathbb{R} \) such that \(F(t) = p\), then \(F^{-1} (p) = t\). However, this is not really practical since in the discrete case, the preimage might not exist; and even if in the continuous \(F\), when \(F\) flats out (at \(p = 1\)), the preimage is not unique.

\begin{definition}[Quantile]\label{def:quantile}
	The \emph{\(p^{\text{th} }\) quantile} of \(X\) is defined as any \(t \in \mathbb{R} \) such that
	\[
		\mathbb{P} (X \geq t) \geq p \geq \mathbb{P} (X < t).
	\]
\end{definition}

Now, we can define \(F^{-1} (p)\) as the smallest \hyperref[def:quantile]{quantile}.

\begin{definition}[Quantile function]\label{def:quantile-function}
	The \emph{quantile function} of \(X\) is defined as
	\[
		F^{-1} (p) = \inf \{ t \in \mathbb{R} \colon F(t) \geq p \}.
	\]
\end{definition}

We sometimes also call \(F^{-1} \) as the \emph{generalized inverse} of \(F\).

\begin{remark}\label{rmk:quantile-function}
	\(t \geq F^{-1} (p)\) if and only if \(F(t) \geq p\); in other words, \(t < F^{-1} (p) \) if and only if \(F(t) < p\).
\end{remark}

Consider given any CDF \(F\), we can actually construct a corresponding random variable.

\begin{remark}
	Let \(U \sim \mathcal{U} (0, 1)\) be a uniform random variable on \((\widetilde{\Omega} , \widetilde{\mathscr{F}} , \widetilde{\mathbb{P}} )\). Then, \(F^{-1} (U) \eqqcolon Y\) is a random variable with CDF \(F\).
\end{remark}
\begin{explanation}
	Since for any \(t \in \mathbb{R} \),
	\[
		\widetilde{\mathbb{P}} (Y \leq t)
		= \widetilde{\mathbb{P}} (F^{-1} (U) \leq t)
		= \mathbb{P} (U \leq F(t))
		= F(t).
	\]
\end{explanation}

Now we can prove \hyperref[thm:Skorohod-representation]{Skorohod's representation theorem}.
\begin{proof}[Proof of \autoref{thm:Skorohod-representation}]
	Consider \(\widetilde{\Omega} = (0, 1)\), and \(\widetilde{\mathbb{P}} ( (a, b) ) = b - a\) for all \(a < b\). Then, we can define \(U(p) = p\) for all \(p \in \widetilde{\Omega} \), i.e., \(U \sim \mathcal{U} (0, 1)\). Define \(Y_n = F_{X_n} ^{-1} (U)\) and \(Y = F_X^{-1} (U)\) from the \hyperref[def:quantile-function]{quantile functions}. Denote \(\Phi \) be the CDF of \(\mathcal{N} (0, 1)\), and let \(Z = \Phi ^{-1} (U)\).

	It's clear that \(Y_n \overset{D}{=} X_n\) and \(Y \overset{D}{=} X\), so we just need to show \(\widetilde{\mathbb{P}} (Y_n \to Y) = 1\).

	\begin{claim}
		It's equivalent to \(\widetilde{\mathbb{P}} (F_{X_n}(Z) < p) \to \widetilde{\mathbb{P}} (F_X(Z) < p)\) for almost all \(p\)'s.
	\end{claim}
	\begin{explanation}
		Observe further that \(Y_n(p) = F_{X_n}^{-1} (p)\), \(Y(p) = F_X ^{-1} (p)\), and \(Z(p) = \Phi ^{-1} (p)\) for all \(p\in (0, 1)\). Since for almost all \(p\)'s, \(Y_n(p) \to Y(p)\) if and only if \(\Phi (Y_n(p)) \to \Phi (Y(p))\) as \(\Phi \) is strictly increasing and continuous, or equivalently,
		\[
			\Phi (Y_n(p))
			= \widetilde{\mathbb{P}} (Z \leq Y_n(p))
			\to \widetilde{\mathbb{P}} (Z \leq Y(p))
			= \Phi (Y(p)).
		\]
		As \(Z\) is continuous, this is equivalent to \(\widetilde{\mathbb{P}} (Z < Y_n(p)) \to \widetilde{\mathbb{P}} (Z < Y(p))\), i.e.,
		\[
			\widetilde{\mathbb{P}} (Z < F_{X_n}^{-1} (p))
			\to \widetilde{\mathbb{P}} (Z < F_X ^{-1} (p)),
		\]
		which holds if and only if \(\widetilde{\mathbb{P}} (F_{X_n}(Z) < p) \to \widetilde{\mathbb{P}} (F_X(Z) < p)\).\footnote{Follows from \hyperref[rmk:quantile-function]{the reamrk}. Explicitly, firstly, it's equivalent to \(\widetilde{\mathbb{P}} (Z \geq F_{X_n}^{-1} (p)) \to \widetilde{\mathbb{P}} (Z \geq F_X ^{-1} (p))\), and with \(\widetilde{\mathbb{P}} (Z \geq F_{X_n}^{-1} (p)) = \widetilde{\mathbb{P}} (F_{X_n}(Z) \geq p)\) and \(\widetilde{\mathbb{P}} (Z \geq F_{X}^{-1} (p)) = \widetilde{\mathbb{P}} (F_{X}(Z) \geq p)\), the result follows.}
	\end{explanation}

	Now we show \(\widetilde{\mathbb{P}} (F_{X_n}(Z) < p) \to \widetilde{\mathbb{P}} (F_X(Z) < p)\) for almost all \(p\)'s. Since \(X_n \overset{D}{\to } X\) means \(F_{X_n}(t) \to F_X(t)\). From \autoref{lma:lec5}, it further implies \(F_{X_n}(t^-) \to F_X(t^-)\) for all \(t \in C_{F_X}\). Note that \(\widetilde{\mathbb{P}} (Z \in C_{F_X})=1\) since there can be only countably many discontinuities of \(F_X\). Hence,
	\[
		\widetilde{\mathbb{P}} (F_{X_n}(Z) \to F_X(Z)) = 1,
	\]
	i.e., \hyperref[def:almost-surely-converge]{converges almost surely}, which implies \(F_{X_n}(Z) \overset{D}{\to } F_X(Z)\), i.e., for all \(p \in C_{F_{X}(Z)}\)
	\[
		\widetilde{\mathbb{P}} (F_{X_n}(Z) \leq p)
		\to \widetilde{\mathbb{P}} (F_{X}(Z) \leq p),
	\]
	and also \(\widetilde{\mathbb{P}} (F_{X_n}(Z) < p) \to \widetilde{\mathbb{P}} (F_{X}(Z) < p)\) from \autoref{lma:lec5}. Again, as \(F_{X}\) can have only countably many discontinuities, this holds for almost all \(p\)'s, which is what we want to show.
\end{proof}

\subsection{Applications}
We see one immediate application to Fatou's lemma.

\begin{proposition}[Fatou's lemma]
	Let \(X_n \overset{D}{\to } X\)\footnote{Can be on different probability spaces.} and \(g\colon \mathbb{R} ^d \to [0, \infty )\) continuous. Then
	\[
		\mathbb{E}_{}[g(X)]
		\leq \liminf_{n \to \infty} \mathbb{E}_{}[g(X_n)] .
	\]
\end{proposition}
\begin{proof}
	Let \((\widetilde{\Omega} , \widetilde{\mathscr{F}} , \widetilde{\mathbb{P}} )\), from \hyperref[thm:Skorohod-representation]{Skorohod's representation theorem}, we can construct \(Y_n \overset{D}{=} X_n\), \(Y \overset{D}{=} X\), and \(Y_n \overset{\text{a.s.} }{\to } Y\), which implies \(g(Y_n) \overset{\text{a.s.} }{\to } g(Y)\). From Fatou's lemma in \(d = 1\), \(\widetilde{\mathbb{E}} _{}[g(Y)] \leq \liminf_{n \to \infty} \widetilde{\mathbb{E}} _{}[g(Y_n)] \). The result follows directly from
	\[
		\mathbb{E}_{}[g(X)]
		= \widetilde{\mathbb{E}} _{}[g(Y)]
		\leq \liminf_{n \to \infty} \widetilde{\mathbb{E}} _{}[g(Y_n)]
		= \liminf_{n \to \infty} \mathbb{E}_{}[g(X_n)] .
	\]
\end{proof}

The following is well-known from real analysis (dominant convergence theorem).

\begin{theorem}\label{thm:DCT-lec7}
	\(X_n \overset{\text{a.s.} }{\to } X\) and \(g \colon \mathbb{R} ^d \to \mathbb{R} \) is continuous and \((g(X_n))\) is uniformly integrable\footnote{I.e., \(\lim_{t \to \infty} \sup _{n \geq 1} \mathbb{E}_{}[\vert g(X_n) \vert \mathbbm{1}_{g(X_n) \geq t} ] = 0\).} if and only if \(\mathbb{E}_{}[g(X_n)] \to \mathbb{E}_{}[g(X)] \).
\end{theorem}

If \(X_n \overset{\text{w} }{\to } X\), then from the definition, we will have \(\mathbb{E}_{}[g(X_n)] \to \mathbb{E}_{}[g(X)] \) if \(g\) is continuous and bounded. We can indeed relax both continuity and boundedness as follows.

\begin{proposition}\label{prop:lec7}
	If \(X_n \overset{\text{w} }{\to } X\) and \(\mathbb{P} (X \in C_g) = 1\) where \(g \colon \mathbb{R} ^d \to \mathbb{R} \) such that \((g(X_n))\) is uniformly integrable, then \(\mathbb{E}_{}[g(X_n)] \to \mathbb{E}_{}[g(X)] \).\todo{why need u.i.}
\end{proposition}
\begin{proof}
	From \(\mathbb{P} (X \in C_g) = 1\) and \(X_n \overset{\text{w} }{\to } X\), from \hyperref[thm:continuous-mapping]{continuous mapping theorem}, \(g(X_n) \overset{\text{w} }{\to } g(X)\), hence \(\mathbb{E}_{}[g(X_n)] \to \mathbb{E}_{}[g(X)] \).
\end{proof}

\begin{remark}
	\autoref{prop:lec7} can be proved with \hyperref[thm:Skorohod-representation]{Skorohod's representation theorem} also.
\end{remark}

\subsection{Proof Strategy}
We often want to prove \(X_n \overset{D}{\to } X\), which is not efficient if we start from definition. To get some intuition for potential proof strategy, consider a deterministic sequence \((x_n)\) in a metric space \((S, \rho )\).

\begin{theorem}
	\((x_n)\) converges to \(x\) if and only if every subsequence of \((x_n)\) has a subsequence that converges to the same limit \(x\).
\end{theorem}
\begin{proof}
	The forward direction is clear. For the backward direction, if not, there exists \((x_{n_k})\) and \(\epsilon > 0\) such that \(\rho (x_{n_k}, x) \geq \epsilon\) for every \(k \geq 1\). But if there exists a subsubsequence \((x_{n_{k_{\ell } }})\) that converges to \(x\), this is clearly a contradiction.
\end{proof}

In the same vein, we have the following.

\begin{theorem}
	\(X_n \overset{\text{w} }{\to } X\) if there exists a subsequence has a subsequence that \hyperref[def:converge-weakly]{converges weakly} and all \hyperref[def:converge-weakly]{weakly convergent} subsequences have the same limit \(X\).
\end{theorem}